\documentclass[12pt]{article}
\usepackage[utf8]{inputenc}
\usepackage{amsmath}
\usepackage{enumitem}
\usepackage{amssymb}
\usepackage{tikz}
\usepackage{gensymb}
\usepackage{graphicx}

\title{Balancing Act}
\author{Michael Brodskiy\\\\\begin{tabular}{l l} Lab Partners: & Ryan Jacoby\\& Graham Horrigan\\& McKenna Dixon\end{tabular}}
\date{\today}
\pagenumbering{gobble}

\usepackage{everysel}
\usepackage{ragged2e}
\renewcommand*\familydefault{\ttdefault}
\EverySelectfont{%
\fontdimen2\font=0.4em% interword space
\fontdimen3\font=0.2em% interword stretch
\fontdimen4\font=0.1em% interword shrink
\fontdimen7\font=0.1em% extra space
\hyphenchar\font=`\-% to allow hyphenation
}

\begin{document}

\maketitle

\begin{center}
\section{Data Table} 
\end{center}
\begin{center}
    

\vspace{16pt}
\begin{tabular}{|l|c|c|}
    \hline
    Trial Number & Distance to Known Mass [$cm$] & Known Mass [$g$] \\
    \hline
    Trial 1 & 27 & 147.71 \\
    \hline
    Trial 2 & 28.6 & 145.21 \\
    \hline
    Trial 3 & 38.1 & 105.07 \\
    \hline
    Trial 4 & 32.1 & 125.13 \\
    \hline
    Trial 5 & 30.8 & 130.15 \\
    \hline
    Trial 6 & 19.6 & 205.18 \\
    \hline
    Trial 7 & 49.8 & 80.23 \\
    \hline
\end{tabular}
\vspace{16pt}
\end{center}

The rock was kept at a constant distance 30.1[$cm$] from the center of the balance.


\begin{center}

\section{Torque Equations}

\end{center}

\begin{justify}

Because the balance is at equilibrium, we can assume that

\end{justify}

\newline

\begin{center}

$\Sigma\tau = 0$

\end{center}

\newline

\begin{justify}

Therefore, because the forces are acting on opposite sides of the meter stick, they counteract, and can be set equal to each other to find:
\hspace{100pt} $\tau_{rock} = \tau_{mass}$

\vspace{14pt}

\flushleft This can be expanded to:

\end{justify}

\begin{center}
    
$r_{rock}*F_{rock}*sin(\theta) = r_{mass}*F_{mass}*sin(\theta)$
    
\end{center}

\begin{justify}

Because both forces are exerted at a 90$\degree$ angle, the sine functions can be cancelled out to yield:

\end{justify}

\begin{center}
    
$r_{rock}*F_{rock} = r_{mass}*F_{mass}$
    
\end{center}

\begin{justify}

Furthermore, because the forces of rock and known mass are both forces of gravity, gravitational acceleration can be cancelled in order to achieve:

\end{justify}

\begin{center}
    
$r_{rock}*m_{rock} = r_{mass}*m_{mass}$
    
\end{center}

\begin{justify}

Finally, to achieve the needed form, the function can be rearranged to get:

\end{justify}

\begin{center}
    
\Large $r_{mass} = \frac{.301 * m_{rock}}{m_{mass}}$
    
\end{center}

\begin{justify}

$\therefore$ The slope of the graph is equal to $.301 * m_{rock}$

\end{justify}

\begin{center}
    
\section{Graphing}

\includegraphics[height=200mm]{Picture1.png}

\end{center}

\begin{center}

\section{Final Calculation}

\end{center}

\begin{justify}

By setting the slope of the line of best fit equal to the slope function obtained earlier we get:

\end{justify}

$$.301 * m_{rock} = .03952$$
    
\begin{justify}

After rearranging, we obtain the experimental value for the rock, which is:

\end{justify}

$$m_{rock} = .1313[kg]\text{, or }131.3[g]$$

\begin{center}
    

\section{Analysis Questions}

\end{center}

\begin{enumerate}
   
\item \centering The actual mass of the rock is found to be: 133.06. This can then be used to find the percent error:
$$\frac{133.06 - 131.3}{133.06} * 100 = 1.323\%$$

\item \centering The brother should sit at half the distance of the sister, so that they exert the same torque upon the seesaw, so that the seesaw balances.

\item \centering The rope at an angle of 60$\degree$ will have a greater tension. This is because torque ($\tau$) depends on the angle, and $\sin(60\degree) > \sin(45\degree)$, so therefore, the rope at the 60$\degree$ angle will have a greater tension.

\item \flushleft A truck moves across a bridge...

\begin{enumerate}[label=(\alph*)]
    
\item \\\includegraphics[]{Picture2.png}

\item The forces on Pier A are: rotational torque, the truck, the bridge itself, gravity, and the normal force.

\item The forces on Pier B are: rotational torque, the truck, the bridge itself, gravity, and the normal force.

\end{enumerate}
    
    
\end{enumerate}









\end{document}
