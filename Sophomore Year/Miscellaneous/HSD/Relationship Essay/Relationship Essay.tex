
%----------------------------------------------------------------------------------------
%	PACKAGES AND OTHER DOCUMENT CONFIGURATIONS
%----------------------------------------------------------------------------------------

\documentclass[11pt]{Essay} % Font size (can be 10pt, 11pt or 12pt)

%----------------------------------------------------------------------------------------
%	TITLE SECTION
%----------------------------------------------------------------------------------------

\title{\textbf{\texttt{Relationship Essay, Unit Assessment}} \\ {\Large\itshape \texttt{Human Social Development}}} % Title and subtitle

\author{\textbf{\texttt{Michael Brodskiy}} \\ \textit{\texttt{LLHS, Ginsberg}}} % Author and institution

\date{\texttt{\today}} % Date, use \date{} for no date

%----------------------------------------------------------------------------------------

\begin{document}

\maketitle % Print the title section

\pagenumbering{gobble}

%----------------------------------------------------------------------------------------
%	ABSTRACT AND KEYWORDS
%----------------------------------------------------------------------------------------

%\renewcommand{\abstractname}{Summary} % Uncomment to change the name of the abstract to something else



\vspace{24pt} % Vertical whitespace between the abstract and first section

%----------------------------------------------------------------------------------------
%	ESSAY BODY
%----------------------------------------------------------------------------------------

\section*{\texttt{Me, Myself, and I}}

\doublespacing

\paragraph{} \texttt{This essay will focus on my relationship with myself. I chose to analyze my relationship with myself out of respect for the privacy of others, that is to say I do not want to subject the confidence my friends and family have in me by analyzing our respective relationships. It is fitting for this assignment to analyze myself, because it is healthy to reflect on one's beliefs and abilities. I have known myself since June 8th, 2019. I spend a large portion of my time thinking of various subjects, one of which is thinking, or \textit{metacognition}.}

%------------------------------------------------



\paragraph{} \texttt{In my relationship with myself, I have been doing well in focusing on my studies. As a self-respecting student, I take tremendous pride in my erudition and the corresponding ambition that follows. I guide myself through both advanced and extracurricular coursework at \textit{Las Lomas} and \textit{DVC}. I do not require motivation nor have I developed dependence on a given instructor, certain material, or means of studying. Through further examination of my relationship it is observable that what works well is my own perseverance and indomitable spirit. I do not need anyone else, I have my conscience and that is all I need to succeed $-$ me, myself, and I.}

%------------------------------------------------



\paragraph{} \texttt{Through the course of my studies, I have been quite \underline{disprivileged}. First of all, I am placed as a last priority student when signing up for classes in both \textit{Las Lomas} and \textit{DVC}. I have also tried to apply for several jobs for which I was more than qualified for, however did not get accepted due to my age. This ageism, although it has slowed me down, has not been able to stop my perseverance. These kinds of obstacles create the need for \underline{self} \underline{love} and \underline{compassion}. Without this kind of self-supporting way of thinking, I would most likely not be able to handle so many obstacles at once. Whenever such an obstacle occurs, I think to myself, ``It can always be worse, and I can always do better.'' The most important part of my relationship, though, is the aforementioned way I `\underline{metacognate}'. Before I call it a night, I think to myself of not only the events that occurred during the day, but also what I thought of during these events. For example, when I learn about some abstract topic, I think to myself how I can understand it better, or possibly what confuses me about it.}

%------------------------------------------------


\paragraph{} \texttt{There are a few ways I think I can improve my relationship with myself. First of all, I should get more sleep, as it would improve my overall health. This can also cause a heightened chance of \underline{amygdala triggering} and easier agitation. Second, I should also try to react less to this kind of ageist discrimination, as it will only annoy me more in the long run. Finally, I think I should try to read more so that I can learn more, and thus, have a more advanced level of \underline{metacognition}. Overall, though, I believe I have a pretty good relationship with me, myself, and I.}

\vspace{15pt}

\begin{center}
\underline{\texttt{word count}}\texttt{: 509} 
\end{center}

%-------------------------------------------------

\end{document}
