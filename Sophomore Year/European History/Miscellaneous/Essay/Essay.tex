\documentclass[12pt,letterpaper]{article}

\usepackage[utf8]{inputenc}
\usepackage[english]{babel}
\usepackage{ragged2e}
\usepackage{schemata}
\usepackage{ifpdf}
\usepackage{soul}
\usepackage{mla}

\definecolor{pink}{rgb}{1.0, 0.33, 0.64}
\newcommand{\hlpink}[1]{{\sethlcolor{pink}\hl{#1}}}

\begin{document}

\begin{mla}{Michael}{Brodskiy}{AP Euro}{Period 1}{\today}{Industrial Revolution}

\flushleft{\textcircled{\raisebox{-0.9pt}{1}}} ~~~~~ 
\justifying{The Industrial Revolution, estimated to have begun around 1780, was a time of rapid technological expansion; the British cities that were most effected by include London, Liverpool, and most importantly, Manchester. The industrialization in Manchester spawned many issues, which had very different reactions from politicians and common folk. \hlpink{The issues include pollution and overpopulation, which elicited responses from political idols, which ranged from rapid expansion of cities to lawful reform; citizens reacted by having mass protests.}}

\schema[close]{Problems, caused by the burning of coal and overall industrialization are recorded from many sources. For example, Edwin Chadwick, an important health reformer from Britain, wrote about the pollution in his ``Report on Sanitary Conditions.'' Chadwick states, ``Diseases caused or aggravated by atmospheric influences . . . prevail among the laboring classes.'' This excerpt, written by a reputable health worker, exemplifies the extent to which nature was harmed during industrialization. As this is a report, it can be reasonably assumed that this was a document intended for the public.}{Doc 6}

\flushleft\schema[close]{Furthermore, Flora Tristan, a French worker's rights advocate, states, ``. . . with every breath of foul air they [the workers] absorb fibers of cotton, wool of flax, or particles of copper, lead or iron.'' Such a quote was published in Tristan's journal. Tristan later states that the workers are kept in disgusting rooms for 12--14 hours a day, and still do not earn enough for even potatoes, the cheapest crop at the time.}{Doc 7}

\flushleft{\textcircled{\raisebox{-0.9pt}{2}}} ~~~~~\\
\schema[close]{Most importantly, ``The Lancet,'' a British medical journal edited by Thomas Wakley, a medical reformer shows the statistics of life. The Lancet shows that life expectancy was significantly greater in all classes, in rural areas, as opposed to areas like Manchester. For example, the life expectancy for a laborer in Rutland was 38. This is significantly greater than that of a laborer in Leeds and Manchester, with the life expectancy of the two cities being 19 and 17, respectively. It is, therefore, false to argue that quality of life was better in industrial cities.}{Doc 8}

\justifying{\schema[close]{Because of such worsening conditions, the people living in Manchester became outraged at such governmental apathy. Many took to the streets in order to protest. One such example of protest is recorded in Frances Anne Kemble's first journey through the Liverpool-Manchester railroad (the first British railway), in which she states, "Shouting 'No Corn Laws,' the vast Manchester was the lowest order of artisans and mechanics, among whom a dangerous spirit of discontent with the Government prevailed." Through this primary source, it is evident that the citizens were not simply conforming to the Industrial Revolution. Although it is clear she has a negative bias towards the lower classes, the events precipitating in her documentation are inarguably accurate.}{Doc 4}}
\flushleft
\schema[close]{Another reaction of citizens to the rapid modernization occurs in Alexis de Tocqueville's account of their journey in England. In said account, Tocqueville states, "Nowhere do you see happy ease taking his leisurely walk in the streets of the city..." According to this excerpt, the mean streets of Manchester are cleared. It is fitting to infer that the streets are empty because of the environmental pollution, and as a protest to the industrial revolution. Because Tocqueville is a visitor in Manchester, at a time when France was not yet fully industrialized, it is safe to trust Tocqueville's document.}{Doc 5}

\justifying{\schema[close]{Although most politicians responded to issues by simply not responding, there are some important changes that took place. Such changes, exemplified in two choroplethic, birds-eye maps, one picturing Manchester in 1750, and the other showing 1850's Manchester. These constructions were clearly added in order to cope with the oncoming flow of urbanizing workers. Although the rent was high, and the buildings crammed, this was still beneficial to the citizens, as it gave them a roof to live under.}{Doc 1}}
\flushleft
\schema[close]{Finally, towards the end of the early industrial revolution, acts that would actually benefit the workers were passed. These acts included, but are not limited to, the Reform Bill of 1832, the Factory Act of 1833, and the Mines Act of 1842. In an article, William Alexander Abram states, "[t]he condition of the factory laborers has been vastly improved within the last quarter of a century." As this is a historian speaking, this is not a source biased towards making the government looking good, and therefore, can be trusted. It is clear that, around the 1830s, the government truly began to respond to the needs of the people. These acts would limit work hours for children, mandate education, promote social welfare, and much, much, more.}{Doc 10}
\vspace{16pt}
\justifying{Although the industrial revolution began with many issues, including, but not limited to: pollution, poor working conditions, and dirty living conditions, the government began to pass reforms following the protesting of the people. This rapid industrialization, although initially detrimental to people and the environment, began to benefit all levels of society. This was much like the rapid industrialization which took place in the Union of Soviet Socialist Republics.}

\end{mla}

\end{document}
