\documentclass[12pt]{article} 

\usepackage{titling}
\usepackage{amsmath}
\usepackage{enumitem}
\usepackage{amssymb}

\newcommand{\subtitle}[1]{%
  \posttitle{%
    \par\end{center}
    \begin{center}\large#1\end{center}
    \vskip0.5em}%
}

%\pagenumbering{Roman}

\begin{document}

%------------------------------------------------------------------------------------------
% Title
%------------------------------------------------------------------------------------------


\author{Michael Brodskiy}
\title{Defining Chapter 22\\European History AP}
\subtitle{Mrs Fisher}
\maketitle


%------------------------------------------------------------------------------------------
% Questions
%------------------------------------------------------------------------------------------

\begin{enumerate}

\item The Industrial Revolution was a period that brought with it radical changes that raised the quality of life of the common person. It was more important than the French Revolution because, rather than being a political revolution, it took place in economy and society, and its effects are still seen to this day.

\item Many factories did not use the newer technology. People were also worried that the total population could increase so steeply, that it would cause a huge economic collapse.

\item The Industrial Revolution occurred in England for three main reasons. Firstly, at the time of the revolution, the English economy was booming, providing them with a lot of money. Second, English farmers were very efficient without machinery in agriculture, second only to the Dutch, meaning that food was cheaper, which meant the average person could spend more money on other goods. Finally, England had a very good banking and crediting system.\\


\begin{itemize}


\item The revolution occurred in England first because of the aforementioned, and the below reasons.\\


\begin{itemize}[label=$\circ$]

\item \textbf{Most Important} - England had political stability and relative peace, permitting the government and citizens to experiment with new technology.

\item \textbf{Natural Resources} - England had a surplus of grain, and lots of precious metals from their colonies.

\item \textbf{Excess Capital} - Because of England's economic boom, they were able to spend more money on innovations.

\item \textbf{Stable Banks} - England had a good crediting system for the time, so people were able to take out money to start a factory.

\item \textbf{Population} - England's population provided for a good workforce in factories.

\end{itemize}

\end{itemize}

\item The Industrial Revolution began in Britain around the year 1800. The continental colonies would not adopt new tchnologies until around 1815.\\

\begin{itemize}

\item The Industrial Revolution really begins around 1780.

\item The 1780s saw the rise in many new inventions which permitted more production.

\item Continental Countries - mid-1800s

\end{itemize}

\item Several inventions in cotton would ameliorate the issue.\\

\begin{itemize}[label=$\circ$]

\item \textbf{James Hargreaves} - Hargreaves created the Spinning Jenny in 1765. It used from 6 - 24 spindles to spin cotton into thread. It was easy to use because power could be supplied by a single person.

\item \textbf{Richard Arkwright} - Arkwright created the Waterframe. It was a large machine, with around several hundred spindles. It is called the waterframe because it required the power of water to work. Factories would employ 1000s of men to take care of these machines.

\end{itemize}

\item This was significant because now all classes were able to purchase clothes. Also, cottage industry workers could purchase cotton from a single source rather than searching from many sources.

\item Edmund Cartwright invented the power loom in 1785, however it wasn't perfected until 1800. It was essentially an early self-working machine. This caused factories to pop up, as this machine would reduce labor costs.

\item Orphans would be taken and used for labor. These kids would be as young as 5 years old. They would have to be apprenticed for 14 years at least, and, as such, spent a great portion of their lives in the factories.

\item Early factories were built near water, as water was used to power the machines.

\item The problem was caused by deforestion to open up clearings for agriculture. This meant less wood left, however it was still in high demand. It was important for heat, and it was used in factories. The issue was resolved by importing wood from Russia.

\item Early coal mines were inefficient. They would also flood with water, which was fixed through the use of animal power. The animal power, however, was inefficient as well. This was fixed with the use of Savery and Newcomen's steam engine. Their steam engine was in widespread use by the early 1770s.

\item Watt was a physics student who noted the inefficiency of Newcomen's engine. He fixed the issue by adding a separate condenser. Partnerships with wealthy people allowed for extra money to be used as risk capital. Skilled workers could install and regulate the steam engines, allowing for use in factories.

\item Blank Space on Question Sheet

\item The invention of trains and railroad systems permitted for better shipping for factories. This allowed for even greater production, which gave more jobs for workers, and made goods cheaper for the average person.It showed the speed and power of this new age.

\item The age of railroads led to the rise of artists such as Turner and Monet. The awe of these machines were captured in paintings. Trains also led to the carving out of tunnels and building of bridges.

\item By 1860, Britain had $\frac{2}{3}$s of the worlds coal. They also had $\frac{1}{2}$ of the iron and cloth. Britain also produced roughly 20\% of the world's goods.

\item This was one of the early World Fairs. It called many inventors and companies to show off their products. Judges award the win to a crystal palace built with cheap iron ang glass.

\item \textbf{Thomas Malthus} - Malthus was a British philosopher. He postulated that, especially due to industrialization, population would outgrow food supply.

\item \textbf{David Ricardo} - Ricardo was an economic theorist. He theorized that, because of overpopulation, wages always drop. Because of this, he wanted to implement the Iron Law of Wages. This would require wages to stay at a certain level.

\item The period following the French Revolution destroyed continental Europe. The Napoleonic Wars caused inflation and disrupted trade. Thus, continental Europe was not able to industrialize as quickly.

\item Continental Europe did have some advantages, though. Firstly, they did not need to create their own machinery, as they could just copy the English. Also, France and Russia had strong governments, which did not fall under foreign control. This meant the governments could serve their own interests. Finally, continental firms could adapt well to the market conditions.

\item Harkort shows that not everyone succeeded in the Industrial Revolution. Cockerill shows that with work, they can succeed. List shows that the government had a great role in industrialization in England.

\item The Zollverein was a customs union in Germany. It controlled trade so that there were no tariffs within German trade, and only one tariff in international trade. The Credit Mobilier (or Isaac and Emile Pereire) used money of many small investors and used resources from big investors to succeed.

\item Many people into the Industrial revolution included the merchant families, Scots, Quakers, Protestants, and the Jewish people.

\item It became increasingly difficult to participate in the Industrial Revolution as a worker for two reasons. First of all, women were less likely to participate in the more business oriented world. Second, the need for skilled, as opposed to unskilled, labor grew, and education was expensive.

\item Blake viewed factories as terribly evil things that destroyed workers' lives. Wordsworth despised that factories destroyed nature to produce goods. Luddites were a group that would violently attack factories, or sometimes, when hired, destroy machines on purpose. They all hated the Industrial revolution.

\item Friedrich Engels was a young, middle class German man. He studied the rise of factories in England. He wrote \textit{The Condition of the Working Class in England} in 1844. He wrote that the middle class was evil for putting workers through such labor. He believed capitalism was the reason behind such atrocities. This led to the creation of the idea of Marxism.

\item Andrew Ure believed that conditions were not harsh, and were even quite good. Chadwick viewed factories as good because it let the masses buy items for less.

\item The middle class did not exploit the working class, as such hard labor was common before laws. It was standard to expect to be treated poorly at a workplace, much unlike modern day.

\item Conditions did not improve by much. Workers still did not have much purchasing power. Also, The work week increased to 6 days instead of the normal 5. Only diets improved slightly, as they became more varied.

\item Robert Owen raised the age of employment in his factories. This promoted the education of the young. Workers would argue for such laws to be enacted in all factories.

\item The Factory Act of 1833 was a big improvement from previous conditions. It limited the workday for children from 14 to 18 to 12 hours, and children 9 to 13 to 8 hours. Children younger than those requirements would be enrolled in education. This led to a drastic decrease in child employment.

\item Subcontractors were like modern day recruiters. Work was found through them. Their relations with the hired people, however, would often be close and personal, instead of hiring the best man. Therefore, this only benefitted those close to a subcontractor.

\item The `Sexual Division of Labor' split factories into two. Men would be the primary workers, working full time. Women were allowed to work, but were given a limited amount of jobs. Theories argue that this strengthened patriarchal tradition.

\item Working class people's solidarity showed through unions. Craft unions would represent the worker, and would not be afraid to strike to get what they wanted.

\item The Combination Acts of 1799 outlawed unions and strikes. Workers' pretty much ignored these acts, and continued going about their business. In 1824, the Combination Acts were repealed.

\item Chartists wanted full democracy. They wanted all men to be able to vote. They actively campaigned for their beliefs. Chartists shaped the industrial system.

\end{enumerate}







\end{document}