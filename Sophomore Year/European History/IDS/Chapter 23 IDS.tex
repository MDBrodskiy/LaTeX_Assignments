\documentclass[12pt]{article} 

\usepackage{titling}
\usepackage{amsmath}
\usepackage{enumitem}
\usepackage{amssymb}

\newcommand{\subtitle}[1]{%
  \posttitle{%
    \par\end{center}
    \begin{center}\large#1\end{center}
    \vskip0.5em}%
}

%\pagenumbering{Roman}

\begin{document}

%$-$$-$$-$$-$$-$$-$$-$$-$$-$$-$$-$$-$$-$$-$$-$$-$$-$$-$$-$$-$$-$$-$$-$$-$$-$$-$$-$$-$$-$$-$$-$$-$$-$$-$$-$$-$$-$$-$$-$$-$$-$$-$$-$$-$$-$$-$$-$$-$$-$$-$$-$$-$$-$$-$$-$$-$$-$$-$$-$$-$$-$$-$$-$$-$$-$$-$$-$$-$$-$$-$$-$$-$$-$$-$$-$$-$$-$$-$$-$$-$$-$$-$$-$$-$$-$$-$$-$$-$$-$$-$
% Title
%$-$$-$$-$$-$$-$$-$$-$$-$$-$$-$$-$$-$$-$$-$$-$$-$$-$$-$$-$$-$$-$$-$$-$$-$$-$$-$$-$$-$$-$$-$$-$$-$$-$$-$$-$$-$$-$$-$$-$$-$$-$$-$$-$$-$$-$$-$$-$$-$$-$$-$$-$$-$$-$$-$$-$$-$$-$$-$$-$$-$$-$$-$$-$$-$$-$$-$$-$$-$$-$$-$$-$$-$$-$$-$$-$$-$$-$$-$$-$$-$$-$$-$$-$$-$$-$$-$$-$$-$$-$$-$


\author{Michael Brodskiy}
\title{Defining Chapter 23\\European History AP}
\subtitle{Mrs Fisher}
\maketitle


%---------------------------------------------------------------------------------------
% Questions
%---------------------------------------------------------------------------------------
\begin{enumerate}

\item \textbf{Dual Revolution $-$} This was the name given to the period in which many radical ideologies appeared. It is called the `dual' revolution because, due to industrialization, the breakthroughs during this period were political and economic. Such an example is Karl Marx's \textit{Das Kapital}, in which he synthesizes politics and economy.

\item \textbf{Congress of Vienna $-$} The Congress of Vienna was the series of meetings, held in Vienna, to discuss post-Napoleonic Europe. The main goal of these meetings was to establish balance of power, and ultimately, peace. This was achieved for roughly 100 years following the meetings.

\begin{center}

Representatives:

\end{center}

\begin{itemize}

\item Austria: Klemens von Metternich

\item Russia: Alexander I

\item Prussia: King Frederick Wilhelm III

\item Britain: Viscount Castlereagh

\item France: Maurice de Tallyrand

\end{itemize}

\item \textbf{Battle of Waterloo $-$} The battle of Waterloo took place in Waterloo, Belgium. This was Napoleon's last battle. It took place during the hundred days,

\item \textbf{Quadruple Alliance $-$} Prussia, Russia, Austria, and Great Britain joined together to form this alliance. The core goal was to maintain a balance of power, which had been disrupted by Napoleon.

\item \textbf{Holy Alliance $-$} The Holy Alliance was formed by the leaders of Austria, Prussia, and Russia. This followed the Congress of Vienna. This was an early form of the League of Nations or the United Nations.

\item \textbf{German Confederation $-$} The German Confederation, which was initially over 300 states, was compacted into 38. This decision was made during the Congress of Vienna.

\item \textbf{Carlsbad Decrees $-$} These were released in the German Confederation. They stated that subversive ideas were to be rooted out. This led to the formation of early spies and secret police.

\item \textbf{Metternich $-$} Klemens von Metternich was the Austrian leader at the Congress of Vienna. He strongly believed in conservatism to maintain balance of power. He feared the spread of nationalism.

\item \textbf{Conservatism $-$} The belief that traditions are to be conserved. This was mostly supported by the nobility and monarchy to maintain power. NOTE: this differs significantly from modern conservatism.

\item \textbf{Liberalism $-$} The belief that more freedom for the people was the correct way for society. This was mostly supported by the lower, working classes. NOTE: this differs significantly from modern liberalism.

\item \textbf{Laissez Faire $-$} Translates from French to `let do.' This was a belief in free market capitalism. Adam Smith was one of this ideas greatest supporters.

\item \textbf{Adam Smith $-$} Adam Smith is the father of capitalism. He wrote \textit{The Wealth of Nations}. He was a strong supporter of free market economies.

\item \textbf{Nationalism $-$} This was used as a method to achieve unity within a region. It was a call for people to think of themselves as belonging to the country, rather than a region. People began to look at their common past, culture, and language.

\item \textbf{Socialists:}

\begin{itemize}

\item \textbf{Saint-Simon $-$} Saint-Simon was an early socialist. He believed that the key to reclaim society for the workers was to eliminate the \textbf{parasites} for the \textbf{doers}.

\begin{itemize}[label=$\circ$]

\item \textbf{Parasites $-$} Saint-Simon believed that parasites were the wealthy middle class, such as lawyers, factory owners, and aristocracy.

\item \textbf{Doers $-$} Those who carried production and did the actual work, or the working class.

\end{itemize}

\item \textbf{Charles Fourier $-$} Fourier was an early mathematician. He envisioned a socialist society in which people lived in communes. He stated, with mathematical precision, how each commune would be structured. He also came up with the mathematical concept of Fourier series.

\item \textbf{Louis Blanc $-$} Blanc is also an example of an early socialist. He believed that workers should be persistent in their efforts to gain rights. He worked to reach max employment in factories.

\item \textbf{Proudhon $-$} Proudhon is also an early socialist. He feared government, rather than despise it. Government scared him because it stole property from the people.

\item \textbf{Robert Owen $-$} Robert Owen was heavily engaged in social reform. He wanted to establish experimental utopias. These utopias would be designated for the working class.

\end{itemize}

\item \textbf{Karl Marx and Friedrich Engels $-$} These were the major philosophers that created the idea known as \textit{Marxism}. Together, they wrote \textit{Das Kapital}. The first volume was released in 1848. Engels published two more volumes following the death of Marx.

\begin{itemize}

\item \textbf{Bourgeoisie $-$} The name given to the wealthy middle class.

\item \textbf{Proletariat $-$} The name given to the working class.

\item \textbf{Hegel $-$} German Georg Hegel was the predecessor to Marx, and the man whose work gave inspiration to Marx.

\end{itemize}

\item \textbf{Romanticism (recognize major names) $-$} This was an artistic (and literary) movement. This was a movement that promoted passions. Romantic music was violent and encouraged emotion. 

\begin{center}

Romantic Works:

\end{center}

\begin{itemize}

\item Wordsworth: Daffodils (Poem)

\item Coleridge: Wrote \textit{Lyrical Ballads}

\item Scott: Translated Goethe's \textit{Gotz von Berlichingen}

\item Byron: English Poet Affiliated with Greek Revolution

\item Shelley: English Poet

\item Keats: English Poet

\item Victor Hugo: \textit{Hunchback of Notre Dame}

\item Dumas: \textit{The Count of Monte Cristo} and \textit{The Three Musketeers}

\item George Sand: \textit{Lelia}

\item Delacroix: \textit{Liberty Leading the People}

\item Beethoven: Great Musician

\item Liszt: Great Pianist

\end{itemize}

\item \textbf{Sturm und Drang (Goethe) $-$} Early German romantics called themselves Sturm und Drang. This translates to Storm and Stress. Storm refers to the volatile tendency of romantic art.

\item \textbf{Greek Revolution of 1820 $-$} Suppressed by the Turks, the Greeks decided to revolt. They were led by Alexander Ypsilanti. Greece succeeded, but was made a protectorate of Russia, with a German on the throne.

\item \textbf{Tory Party $-$} A party controlled by the wealthy aristocracy. Because of this, it feared radical reforms.

\item \textbf{Whig Party $-$} A party also controlled by the aristocrats. Different from the Tories in that they held interests in trade,

\item \textbf{Battle of Peterloo $-$} Nicknamed for the Battle of Waterloo. Civilians protested the infamous Six Acts. Cavalry fired into the crowd.

\item \textbf{Reform Bill 1832 $-$} This allowed for peaceful reform. It added about half as many votes as there were previously. This permitted about 12\% of the male population to vote.

\item \textbf{Anti-Corn Law League $-$} This league was formed in order to combat the corn laws. People argued that cheaper food prices would result in a higher living standard. They believed that repealing of the Corn Laws would lower food prices.

\item \textbf{Ten Hours Act $-$} This limited the workday (for women and children) to 10 hours a day.

\item \textbf{Great Famine $-$} Better known as the Irish Potato Famine. Bad weather followed by a drought caused the potato crops to perish. Benefited the English because it allowed them to buy land for cheap.

\item \textbf{Louis XVIII's Constitutional Charter $-$} This was a charter that was intended to be democratic. It actually was anything but democratic. Only 100,000 of the richest males were allowed to vote.

\item \textbf{Charles X $-$} Charles X succeeded Louis XVIII. He was crowned in a lavish, medieval setting. During his reign, he rallied support through nationalism to get more troops. This was done to combat the Muslim Algerians.

\item \textbf{French Revolution of 1830 $-$} Charles X repudiation of Louis XVIII's charter began this revolution. People revolted because they were no longer in pre-1789 mindsets.

\item What was the impact of the Congress of Vienna on:

\begin{enumerate}[label=(\alph*)]

\item Politics in Europe? $-$ The Congress of Vienna would ultimately maintain balance in Europe. Peace would be kept until The Great War.

\item Borders of Countries? $-$ Many country's borders were changed to promote peace. France was set to 1792 borders. Prussia received some of France's eastern edge. Many borders were changed.

\item Alliances? $-$ The Congress of Vienna would cause underground alliances to take place. These secret alliances would ultimately lead to The Great War.

\item French Colonies? $-$ French colonies were seized and redistributed among alternative European powers.

\end{enumerate}

\item How did nationalism develop in the different countries of Europe?

\begin{center}
In most European countries, nationalism rose in one of two forms. In some countries, it arose in the form of revolutions, inspired by America and France. In others, it was intentionally promoted to raise popular support.
\end{center}

\item Where did they occur, and what were the results of the revolutions of 1848?

\begin{center}
In 1848, there were revolutions in France, German States, the Austrian Empire, Hungary, Italian states, Denmark, and Poland. Although most failed, some succeeded in getting a liberal constitution, which was usually repealed by force from the Holy Alliance.
\end{center}



 

\end{enumerate}







\end{document}