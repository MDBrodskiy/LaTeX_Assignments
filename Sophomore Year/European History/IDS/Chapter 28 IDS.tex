\documentclass[12pt]{article} 

\usepackage{titling}
\usepackage{amsmath}
\usepackage{enumitem}
\usepackage{amssymb}
\usepackage{everysel}
\usepackage{ragged2e}
\usepackage{geometry}
\geometry{top=1.0in,bottom=1.0in,left=1.0in,right=1.0in}
\newcommand{\subtitle}[1]{%
  \posttitle{%
    \par\end{center}
    \begin{center}\large#1\end{center}
    \vskip0.5em}%
}
\renewcommand*\familydefault{\ttdefault}
\EverySelectfont{%
\fontdimen2\font=0.4em% interword space
\fontdimen3\font=0.2em% interword stretch
\fontdimen4\font=0.1em% interword shrink
\fontdimen7\font=0.1em% extra space
\hyphenchar\font=`\-% to allow hyphenation
}

\begin{document}

%------------------------------------------------------------------------------------------
% Title
%------------------------------------------------------------------------------------------

\author{Michael \textsc{Brodskiy}}
\title{Defining Chapter 28 \\European History AP}
\subtitle{Mrs Fisher}
\date{March 16, 2020 $\longrightarrow$ April 6, 2020}
\maketitle

%------------------------------------------------------------------------------------------
% Questions
%------------------------------------------------------------------------------------------

\begin{flushleft}
\begin{enumerate}

\item  \underline{James Connolly (\emph{Sheamus \'O'Connolly})} $-$ A Scottish born Irish Marxist responsible for conglomerating Irish paramilitary and socialist worker's groups to form the Irish Republican Brotherhood (IRB) during the early days of the Irish republican nationalist movements. He was sentenced to death by firing squad, among fifteen other members of the IRB, for his participation in the 1916 Easter Sunday uprising. His memory, kept alive in poems and provisional army songs, became that of a martyred, loyal son of Ireland, and contributed to the growing tensions between the oppressing British constabulary and rebelling Irish citizenry. His death allowed for the promotion of junior Irish Republicans such as Michael Collins, Arthur Griffith, and \`Eamon de Valera to assume senior leadership positions in the Irish government. \\\vspace{8pt}\emph{Also, James Connolly had a gnarly mustache}

\item \underline{Michael Collins} $-$ Lead the revolution as Chairman of the Provisional Irish Government following the death of James Connolly. He organized the independence of 26 Irish counties after a period of negotiation with Britain following a mutual ceasefire. The longevity of the Provisional Government was contingent on the reception of the Anglo-Irish treaty by the Irish citizenry who disrupted into civil war following the rejection of allegiance to the crown. Fighting occurred predominantly around the north six counties where the Royal Ulster Constabulary maintained a military presence. The ensuing chaos is referred to as the troubles. Michael Collins died in an ambushed staged by his former comrades whilst on his way to golden B\'eal na Bl\'ath in 1922.

\item \underline{\`Eamon de Valera} $-$ The only American born Irishman to take up arms during the Easter Sunday uprising, he was spared from execution at  Kilmainham wall on account of his birthplace and enjoyed a fruitful and long political career spanning over half a century. He was the Irish head of state during the emergency (Second World War) and the troubles (Northern Ireland political/religious violence).

\item \underline{``Black and Tans''} $-$ Were British loyalist constables on patrol recruited into the Irish Constabulary as reinforcements to royal allegiants during the Irish War of Independence. A corresponding response to the Irish Republican Army (IRA) tendencies of engaging in guerrilla tactics against stationed British constables in Ireland. The term ``\emph{Black and Tans}'' originates from the coloration of their uniforms.

\item \underline{Statute of Westminster} $-$ The legislative officiation of the decolonization of the British Empire \emph{circa} 1931.

\item \underline{British Commonwealth} $-$ The vast political/socio-economic unity among the subject territories established by the United Kingdom following their respective decolonization. 

\item \underline{Ruhr Crisis} $-$ The increase of French and Belgian military personnel and constable advisors stationed in the German Ruhr regions escalated the anti-Germanic racial tensions by prohibiting a German to walk on the same sidewalk as a Frenchman, mandating Germans to tip their hats to Frenchmen, and even patiently receive a slap to the face for an infraction of the aforementioned or other segregations. This racially centered territorial occupation would spark the radical Germanic pro-Aryan movements associated with the rise of fascism.

\item \underline{Dawes Plan} $-$ In 1924 a United States banker, Charles Dawes, proposed a resolution to the World War I reparations mandated upon Germany. However benevolent the logic behind this resolution proposal seemed, the alleviation of diplomatic tensions following World War I and the Treaty of Versailles behind European nations masked the increase in control by western (\emph{predominantly American}) banks of German institutions. The supervised loans provided to Germany was essentially a proactive method to containing the influence of the communist international. 

\item \underline{Young Plan} $-$ The portion of the Paris Peace Conference attributed to designated Germany's First World War reparations.

\item \underline{Washington Naval Conference} $-$ The first globally recognized conference of disarmament attended by Britain, France, Italy, China, Japan, Belgium, Netherlands, and Portugal; however the Union of Soviet Socialist Republics was not recognized and therefore not invited. Although this conference transpired outside the scope of both the League of Nations and the contemporary non-interventionist inclinations of United States foreign policy doctrines, it was, essentially, a precursor to the United Nations. 

\item \underline{Locarno Treaties} $-$ The portion of the Paris Peace Conference attributed to negotiating agreements (\emph{at Locarno}, \emph{Switzerland}) concerning the European Allied powers relations with the defeated German Reich (\emph{the newly established Weimar Republic}) and the newly established states of Central Europe formed after the dissolution of Austria-Hungary following the final ceasefire of the First World War.

\item \underline{Kellogg-Briand Pact} $-$ The 1928 attestation of \textbf{Renunciation of War as an Instrument of National Policy} signed, initially, by Australia, Belgium, Britain, Canada, Czechoslovakia, France, Germany, India, Ireland, Italy, Japan, New Zealand, Poland, South Africa, and the United States.

\item \underline{Maginot Line} $-$ A heavily fortified infantry defense installment constructed during the 1930s along the French-German border. The logic behind the construction of this militarized obstacle was to force the enemy against committing to an entrenched stalemate as had been done in the First World War.

\item \underline{Sigmund Freud} $-$ A self-proclaimed cocaine addict, Sigmund Freud contributed greatly to the field of psychoanalysis through actively sustained conversation with his patients. He would diagnosis and prescribe treatments and would often relate the issues of his patients to sexual desires/repressions.

\item \underline{Psychoanalysis} $-$ The Freudian concept of analyzing the dreams of his patients through interrogatory dialogue.

\item \underline{\emph{The Interpretation of Dreams}} $-$ The English translation of \emph{Die Traumdeutung}, the name of Sigmund Freud's original publication. 

\item \underline{Carl Jung} $-$ An influential Swiss born  psychiatrist and psychoanalyst who contributed to the newly forming fields of analytical psychology whose was taken under the tutelage of Sigmund Freud.

\item \underline{Vassily Kandinsky} $-$ A Russian born abstract artist famous for his sketches, drawings, and paintings. He was born in, and attended university in, Moscow but spent much of his adolescence in the internationally cultured port town of Odessa (\emph{Ukraine}). Kandinsky fled what was left of the Russian Empire following the proletarian socialist revolution and settled in Germany where he would eventually matriculate at the Bauhaus School of Art.

\item \underline{Cubism} $-$ A revolutionized abstract art movement centered around cubic units comprising an entire painting or sculpture.

\item \underline{Pablo Picasso} $-$ A Spanish born abstract artist famous for his surrealist-cubist style of sculptures and paintings. His work would depict many periods of Spanish and moreover, European, life such as the aerial bombings during the Spanish Civil War, ravaging effects of the global Great Depression, and the Second World War.

\item \underline{Georges Braque} $-$ A French born cubist abstractionist painter and sculptor whose fame may be attributed to his friendship with Pablo Picasso.

\item \underline{Surrealism} $-$ A revolutionized abstract form of expression through art, largely popularized through the globally recognized works of Pablo Picasso and Salvador Dal\'i. 

\item \underline{Salvador Dal\'i} $-$ Another Spanish born surrealist abstractionist whose work includes graphic depictions of the Spanish Civil War and the Second World War.

\item \underline{Marcel Proust} $-$ A homosexual French essayist responsible for drafting the famous novel \emph{The Search of Lost Time}. 

\item \underline{\emph{Remembrance of Things Past}} $-$ Another name for \emph{The Search of Lost Time}, a French existentialist literary production spanning seven volumes written by Marcel Proust.

\item \underline{Franz Kafka} $-$ A German born essayist whose fame may be attributed to his controversial exploration of existentialism, alienation, and depression. His writing style essentially demonstrates the sentiment of Europe's \emph{Lost Generation} of the 1920s.

\item \underline{\emph{The Trial}} $-$ The English translation of \emph{Der Proze\ss}, the name of Franz Kafka's most famous publication. 

\item \underline{Kafkaesque} $-$ A post World War/1920s\emph{Lost Generation}'s melancholic style.

\item \underline{James Joyce} $-$ An Irish born essayist famous for his publication of \emph{Ulysses}, a modernized literary production of Homer's original \emph{Odyssey}.

\item \underline{\emph{Ulysses}} $-$ A modernized literary production of Homer's original \emph{Odyssey} published by James Joyce.

\item \underline{Virginia Woolf} $-$ A British born essayist whose fame is most probably attributed to her publication of \emph{A Room of One's Own}. 

\item \underline{\emph{A Room of One's Own}} $-$ Virginia Woolf's publication exploring multiple feminist themes such as lesbianism.

\item \underline{Thomas Mann} $-$ A German born essayist and philanthropist made famous by his 1929 award of a Nobel Prize.

\item \underline{D.H. Lawrence} $-$ An English born essayist whose controversial exploits of sexual health as a topic of literary exploration earned him both fame and censorship.

\item \underline{Whilhelm R\"ontgen} $-$ A German born mechanical engineer turned physicist credited with the discovery of electromagnetic radiation in the form of high-energy waves (\emph{x}-\emph{rays}). R\"ontgen was awarded the Nobel Prize in Physics and had an element (\emph{roentgenium} [Rg-111]) and a unit of measurement named after him.  

\item \underline{J.J. Thomson} $-$ An English born physicist responsible for detecting the property of electric conductivity through gaseous media. He was awarded the Nobel Prize in Physics five years after R\"ontgen was.

\item \underline{Pierre \& Marie Curie} $-$ The scientists credited with detecting and documenting the properties and affects of radiation in tandem. Marie Curie was also the first woman to be awarded the Nobel Prize twice.

\item \underline{Ernest Rutherford} $-$ A New Zealand born physicist whose experimentation concerned radioactivity and atomic structure. He categorized radioactive particles into beta and alpha rays and received the Nobel Prize in Chemistry for his discovery of atomic nucleus.

\item \underline{Max Planck} $-$ A German born theoretical physicist widely considered the last of the classical physicists due to his role in the interpretation of quantum theory. He received the Nobel Prize in physics for his achievements concerning the echeloned ``\emph{quantized}'' energy levels of electrons. 

\item \underline{Quantum Physics} $-$ The Schr\"odinger Equation (\emph{Schr\"odinger Equation implies either the general equation, or the specific nonrelativistic equation}):
$$ \text{Time-Dependent Equation:} $$
$$ i\hbar\frac{\partial}{\partial t}\bigl| \Psi(t)\bigr>=\hat{H}\bigr|\Psi(t)\bigr>$$
$$ \text{Time-Independent Equation:} $$
$$ \hat{H}\bigr|\Psi\bigr>=E\bigr|\Psi\bigr>$$
$$ \text{Time-Dependent Relativistic Equation:} $$
$$ i\hbar\frac{\partial}{\partial t}\Psi(x,y,z,t) = \biggl[ \frac{-\hbar^2}{2m}\nabla^2 +V(x,y,z,t)\biggr]\Psi(x,y,z,t)$$
$$ \text{Time-Independent Relativistic Equation:} $$
$$ \biggl[ \frac{-\hbar^2}{2m}\nabla^2 +V(x,y,z)\biggr]\Psi(x,y,z) = E\Psi(x,y,z) $$

\item \underline{Werner Heisenberg} $-$ A German born theoretical physicist and an integral advocate of quantum theory. Like many of his constituents, he too received the Nobel Prize in Physics.

\item \underline{\emph{the} ``Uncertainty Principle''} $-$ Another name for \emph{Heisenberg}'\emph{s uncertainty principle}, a fundamental limit to the precision of computable values of certain physical properties (ex. \emph{momentum} or \emph{velocity}) of a quantized particle.

\item \underline{Sir Alexander Fleming} $-$ A Scottish born biologist responsible for accidentally discovering the penicillin antibiotic from mold left overnight in a laboratory petri dish. He received a Nobel Prize for his findings in Medicine, however shared the award with Howard Florey. 

\item \underline{Sir Howard Florey} $-$ An Australian born pathological biologist famous for sharing the 1945 Nobel Prize in Medicine with Alexander Fleming. His work mainly concerning fields of virology and although Alexander Fleming received most of the credit for the discovery of penicillin, Howard Florey performed the first treatment of penicillin on a patient.

\item \underline{Max Weber} $-$ A German born sociological philosopher, he was heavily influenced by people like Karl Marx, Charles Darwin, Sigmund Freud, and Friedrich Nietzsche. His work was primarily involved with the effects of secularism, mass disenchantment, and existentialism.

\item \underline{Walter Gropius} $-$ A German born architect and the founder of the esteemed Bauhaus School of Art.

\item \underline{Bauhaus} $-$ The Weimar art school in Germany founded by Walter Gropius. A young and impressionable Adolf Hitler would famously write in an autobiography about his rejection from the Vienna School of Fine Arts and would engage with the architects from Bauhaus during his reign as F\"urher to to satisfy his artist tendencies.

\item \underline{Fascism} $-$ An ultra-leftist from of totalitarian government centered around the belief of a society's necessity to triumph above the necessity for the existence of an individual. Ultimately, as history would demonstrate, the only difference between fascism and communism is foreign policy. Under a fascist regime industry is nationalized and the state is centralized under a single party will.

\item \underline{Benito Mussolini} $-$ The Italian born leader commonly referred to as Il Duce (\emph{The Leader}) of the Italian National Fascist Party. His reign began legitimately, under the Italian constitution, until he assumed totalitarian power and established a despotic regime by purging his political rivals, predominantly socialists like Giacomo Matteotti. He was captured and arrested after evading allied forces and eventually rescued by them by a specialized Fallschirmj\"ager (\emph{paratroopers}) unit and Waffen-SS commandos under direct orders from the F\"urher. Although successfully rescued he once again attempted to evade allied forces following their victories along the Italian peninsula and was caught at the border where he was returned to Rome to be executed.

\item \underline{Black Shirts} $-$ Officially, the Milizia Volontaria per la Sicurezza Nazionale (\emph{Voluntary Militia for National Security}) an Italian fascist paramilitary organization formed in 1923, similar to the SS in Germany, whose members whore black shirts and remained extremely loyal to the Italian National Fascist Party and Il Duce Mussolini. 

\item \underline{$\text{E}=\text{mc}^\text{2}$ (\emph{Mass-Energy Equivalence})} $-$ Simplified derivation:
$$E_K = \frac{mv^2}{2}; \quad W=\vec{F}\cdot \vec{x}; \quad \vec{p} = m\vec{v} $$ 
$$ W = \int_{x_o}^{x_f}\vec{F}\cdot d\vec{x} \Rightarrow \int_{x_o}^{x_f}\frac{\partial\vec{p}}{\partial t} d\vec{x} \Rightarrow \int_{x_o}^{x_f}\frac{\partial m\vec{v}}{\partial t} d\vec{x} $$ 
$$ \text{from Einstein's Zur Elektrodynamik bewegter K\"orper }$$
$$\text{(\emph{On The Electrodynamics of Moving Bodies}) essay:}$$
$$\vec{p} = \lambda m\vec{v}; \quad \lambda = \biggl(\sqrt{1-\frac{v}{c}}\biggr)^{-1} $$
$$ \frac{\partial p}{\partial v} \Rightarrow \frac{\partial}{\partial v}\biggl(mv\biggl(\sqrt{1-v^2/c^2}^{-1}\biggr)\biggr) $$
$$ W = \int_o^f\frac{\partial p}{\partial v} v\cdot dv \Rightarrow \int_o^f\frac{mv}{\biggl(1-v^2/c^2\biggr)^{3/2}} dv \Rightarrow \frac{mc^2}{\sqrt{1-v^2/c^2}}-mc^2 = E_K $$
$$ \text{where $\text{E}_\text{tot}$ is the total energy: } E_{tot} = \frac{mc}{\sqrt{1-v^2/c^2}} $$
$$ \text{and mc}^\text{2} \text{ is a constant independent of the velocity and kinetic energy} $$ 
$$ \therefore \Delta E = \Delta mc^2 $$

\item An example of heightened authoritarian rule lies in the French Revolution of 1789. During the Reign of Terror, Maximillien Robespierre granted himself emergency powers in order to ``root out all enemies of the revolution.'' His fellow revolutionaries accepted this (until the Thermidorian Reaction). Using this extreme power, Robespierre executed all possible enemies of the revolution, without fair trial (which was the purpose of the revolution).

\item Conservative Authoritarianism was an anti-democratic form of government. Leaders tried to \textit{conserve} the government, and prevent any major changes from occuring. As such, it was important for citizens to remain sedated and obedient. Communist ideals (pre-World War II) were abolished in favor of this system in Poland and Hungary. This very same idea allowed communism to advance in Yugoslavia. Finally, Portugal became a strong dictatorship following the system of conservative authoritarianism.

\item The difference between Totalitarianism and Authoritarianism can be seen in the etymology of the words themselves; that is, Totalitarianism implies \textit{total} control over every facet of the society, whereas Authoritarianism only implies control over major changes within the society.

\item Stalin's (\textsc{Originally \textit{Trotsky's}}) Five Year Plan was intended to create a planned economy. One of its main points was to push heavy industry to levels beyond the other already industrialized countries. As the phrase goes, "5 years' worth of work, finished in 4, with 3 shifts, and 2 people, under one salary."   

\item Under Lenin, the New Economic Policy was intended to kickstart the economy, which was supposed to later be converted back into a communist system. It proved to be extrememly unsuccesful; it was a fitting metaphor for the remainder of the existence of the Soviet Union. 

\item During the first few years of the Soviet Union, Stalin began appointing people he 'befriended' to positions of power. Examples include Kamenev, Zenoviev, and Voroshilov. Also, when Vladimir Ilyich Ulyanov fell ill, Stalin took care of all correspondence between Lenin and the outside world. As such, he could censor Lenin's input and control his decisions. This would make it easy to oust Trotsky outside of the party in 1926, and out of the country in 1929.

\item Stalin's (\textsc{Originally \textit{Trotsky's}}) Five Year Plans were the manifestation of a command economy. It was beneficial, because, thanks to this system, the Soviet Union was able to defeat the Axis powers through industrial might. A downfall of this was that it was a planned economy. This meant that there was no economic growth, and thus, no growth overall.

\item The Soviet Union relied on heavy propaganda to sway the masses into working harder. This propaganda appealed to nationalistic and patriotic sentiments, and as such, provided the workers with a sense of accomplishment.

\item Stalin gained respect through fear. He proved that it is greater to be feared than loved. The great purge was the Soviet Union's equivalent (which was actually significantly greater) to the French Reign of Terror. What was baffling was that all citizens knew what was going on, but were too afraid to stop it. One example is the well-known story of Pavlik Morozov, who turned in his own father as anti-Soviet because he believed he would get recognition for it. 

\item As compared to other countries, women gained status. They were able to become party members, and get the same quality education as their male counterparts. Also, they were included in the Soviet Armed Forces.    

\item The Italian born leader commonly referred to as Il Duce (\textit{The Leader}) of the Italian National Fascist Party. His reign began legitimately, under the Italian constitution, until he assumed totalitarian power and established a despotic regime by purging his political rivals, predominantly socialists like Giacomo Matteotti.

\item Mussolini was different in that he didn't actually hold full power. He had to make compromises with the people, which, obviously, Stalin and Hitler did not do. Also, Mussolini never tried to control the people or reform land. He also made and agreement with pope (pope $\longrightarrow$ religion).

\item Hitler wanted to be an artist, however, after he was not accepted to art school, he began to read anti-semetic books. He was also took part in the First World War on behalf of Germany. Also, he had a failed coup in 1923, known as the Bier Hall Putsch. He was in prison, and, following the economic collapse of 1929, he knew something needed to be done.

\item At the end of World War I, Hitler was in a hospital following a gas attack. He had suffered damage to his eyes. In the years leading up to the Great Depression, he was trying to get more people to join his cause. He led a march on a German Bier Hall in 1923, known as the Bier Hall Putsch, which he failed, and, consequently, was imprisoned for. During his time in prison, he wrote the well-known book titled, \textit{Mein Kampf}.

\item During the Great Depression, Germany Marks were pretty much useless. Their value was so low, that blocks of the currency would be used as toys for children, or burned for warmth. These poor conditions, coupled with the poor response of the German Weimar State was fuel for Hitler, as he wanted Germany to be perfect.

\item First of all, Hitler promised jobs and food for all. Of course, this was an appealing promise, as mothers were unable to feed their children. Hitler promised a better Germany for all Germans, through socialism. His National Socialist (\textsc{Nazist}) tactics promoted great economic growth, and as such, provided the working class with jobs centered around providing for Germany (and later the war effort).

\item Much alike Stalin, Hitler used fear tactics to gain respect. He held purges on political opponents, especially communists (as fascism and communism can not co-exist), and the Jewish. He used Kristallnacht (1938) and book burnings to burn fear into people, so he would be able to gain control.

\item Being a totalitarian, Hitler relied heavily on heavy propaganda. Propaganda was used to appeal to the working class, and to gain their trust. With public support, Hitler was able to 
do whatever he wished.

\item The Munich Conference ended on September 30, 1938. British Prime Minister Chamberlain exclaimed triumphantly, "Now we have peace for our time!" Although this was not to be so, the steps leading up to the war were just as crucial as those prior to this 'momentous' peace. Before 1938, Hitler accomplished two crucial things: the invasion of Czechoslovakia, and the Anschuluss.

\item The Munich Conference was held to prevent war. Chamberlain returned triumphantly, believing he had achieved "peace for our time." This policy of 'appeasement' would actually be detrimental to the allies, as Hitler walked all over them to achieve his goals. The Munich Conference would actually prove useless.

\item The Molotov-Ribbentrop pact shocked the west because the western nations were relying on Germany to invade and defeat the Soviet Union, but be weak enough afterwards to be defeated by the other western nations. Stalin signed the pact with Hitler in order to set a buffer (Poland) between himself and Germany.

\item Up to April 1941, the Axis powers (mostly Germany) would sweep through all of Europe, parts of the Balkans, and Northern Africa using the Blitzkrieg tactic. It was extremely effective in flat terrain with warmer climates, and short distances between cities. 

\item Hitler should not have invaded the Soviet Union. This would drain his resources and ultimately prove a fatal attempt. 

\item Hitler wanted for the 'Aryans' to be the master race. He also believed that Women were supposed to make more children for Germany.

\item The Grand Alliance included the United States, Britain, and the Soviet Union. A definite strength was the military power of two of its members: the United States, and the Soviet Union.

\item Stalingrad began on August 23rd, 1942, and continued until February 2, 1943. This was the turning point of the war because it drained German resources. The 6th army, the largest German army consisting of 1 million men, was crushed. Oil reserves were depleted, and equipment such as tanks were destroyed.

\item The United States and Britain would invade Northern Africa, and move up from there, invading Italy next. The western powers joined the fight in Europe in 1944, starting with the Invasion of Normandy.

\item Following mid-1942, Germany was unable to maintain the war as well. Their advance was being pushed back by the Soviets, and their infrastructure was weakening at an exponential rate.

\end{enumerate}
\end{flushleft}
\end{document}
