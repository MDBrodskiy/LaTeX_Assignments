\documentclass[12pt]{article}
\usepackage[letterpaper]{geometry}
\usepackage{amsmath, amsthm, amssymb, amsfonts}
\usepackage{graphicx}
\usepackage{titling}
\usepackage{hyperref}
\newcommand{\subtitle}[1]{%
  \posttitle{%
    \par\end{center}
    \begin{center}\large#1\end{center}
    \vskip0.5em}%
}
\hypersetup{
    colorlinks=true,
    linkcolor=blue,
    filecolor=magenta,      
    urlcolor=blue,
} 
\urlstyle{same}
\title{The Living Earth\\\underline{\textbf{\textit{i}}-Phone Mystery: Science to the Rescue!}}
\subtitle{Laboratory Assignment \#1}
\author{Michael Brodskiy}
\begin{document}
\maketitle
\begin{center} \textbf{\textsc{Post$-$Lab Conclusion}} \end{center} 
\newpage
\begin{enumerate}
    \item \textit{In 2$-$3 sentences describe the purpose of this lab, and how it relates to what we're currently learning about in class.}
    \paragraph{} The primary objective of this laboratory assignment pertains to the analysis of macromolecules, specifically; lipids, proteins, and carbohydrates, to deduce the culprit responsible for stealing Jason's \textbf{\textit{i}$-$}phone.
        In this manner utilizing data gathered from organic compounds found in potential suspects' lunches as evidence directly related to the crime scene. Therefore, complementing biochemistry with an applicable forensic analysis. 
    \item \textit{Briefly, (in no more than 5$-$6 sentences) describe the procedure you followed to gather your data.}
    \paragraph{} There were five food substances tested for constituency in Jason's evidence: \begin{center} \textsc{Pretzels}, \textsc{Butter}, \textsc{Jelly}, \textsc{Fat$-$Free Yogurt}, \& \textsc{Beans} \end{center} 
    \paragraph{} The initial comparison between food substance and macromolecule concluded that lipids were indeed a constituent of Jason's evidence. Next, the five food substances were tested for starches, which Jason's evidence did not contain. The next test was for sugars, which Jason's evidence did contained. Lastly the test for proteins concluded that Jason's evidence contained that macromolecule as well.
    \item \textit{Discuss the purpose of an indicator, which ones were used, and how they were used in this experiment.}
    \paragraph{} There were three indicators used:
    \begin{center} \textsc{Benedict's solution}, \textsc{Iodine}, \& \textsc{Biuret Reagent} \end{center}
    \paragraph{} The indicators were used to test which food substances contained which macromolecules. Benedict's solution turns orange or red in the presence of a monosaccharide, Iodine turns blue/black in the presence of a starch, and Biuret Reagent turns pink or violet in the presence of a protein. Using this knowledge, the food substances were tested for different macromolecules to arrive at the aforementioned conclusions.
    \item \textit{Summarize the data: which foods tested positive for protein? Starch? Lipids? Sugars?}
        \begin{itemize} 
            \item Pretzels tested positive for starches. 
            \item Butter tested positive  for lipids and sugars. 
            \item Jelly tested positive for sugars. 
            \item Fat$-$Free yogurt tested positive for sugars and proteins. 
            \item Beans tested positive for starches, sugars, and proteins. 
        \end{itemize}
    \item \textit{Analyze the data you received from testing Jason's evidence, and use it to propose a ``\textit{theory}" as to which coworker most likely took Jason's \textbf{\textit{i}}-phone. Be sure to cite \textsc{specific} evidence from the lab to support your conclusion.} 
    \paragraph{} Jason's evidence contained lipids, sugars, and proteins. Jessie had pretzels, and they only tested positive for starches, so it can't be Jessie. Jeff had a bean burrito with cheese and beans tested positive for proteins, and sugars, as well as starches, so it can't be Jeff. Emma had fat free yogurt, and that tested positive for sugars and proteins, but not lipids, so it can't be Emma. Therefore, it is only safe to assume that it was Cooper, because he had toast with butter and jelly, and butter tested positive for lipids and sugars, and jelly tested positive for sugars, and even though it didn't test positive for the proteins Jason's evidence had, it didn't contradict it by testing positive for starches. 
        \begin{center} Therefore, it is \underline{Cooper}. \end{center}  
    \item  \textit{Identify at least 3 sources of error from this experiment, and how you might be able to reduce these errors in future experiments.}
    \begin{itemize} 
        \item A potential source of error is the lipid test. Due to its dependency on the thickness of the brown paper. 
        \item A potential source of error is the contamination of a food substance. When washing out the well plate, some water may have stayed in the spaces for the substances, and diluted or contaminated the substance. 
        \item A potential source of error is the contamination of a food substance. The substance could have been left out for a prolonged period of time, therefore exposed to airborne contamination.
    \end{itemize}
    \paragraph{} Some possible solutions pertinent to the reduction of these sources of error for future experiments are: 
    \begin{itemize}
        \item Verifying the uniformity of paper thickness.
        \item Verifying the complete dryness of the well plate.
        \item Maintaining the food substances sealed, in a controlled environment, until they are ready for testing.
    \end{itemize}
\end{enumerate}
\end{document}
