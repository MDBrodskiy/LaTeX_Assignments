\documentclass{letter}
\usepackage{amssymb,amsmath}
\usepackage{graphicx}
\usepackage{fouriernc}
\usepackage{hyperref}
\usepackage{lipsum}
\usepackage{parskip}
\usepackage{everysel}
\usepackage{ragged2e}
\hypersetup{
    colorlinks=true,
    linkcolor=blue,
    filecolor=magenta,      
    urlcolor=blue,
} 
\urlstyle{same}


\renewcommand*\familydefault{\ttdefault}
\EverySelectfont{%
\fontdimen2\font=0.4em% interword space
\fontdimen3\font=0.2em% interword stretch
\fontdimen4\font=0.1em% interword shrink
\fontdimen7\font=0.1em% extra space
\hyphenchar\font=`\-% to allow hyphenation
}

\pagenumbering{gobble}

\begin{document}
\begin{center}
On The Effects Of Organized Religion
\end{center}
\justifying Religion is the manifestation of the innate human tendency to desire gratification. By our very nature, we desire to \textit{be desired} $-$ meaning that every individual, primitively speaking, wants to feel special. An exemplification of this desire exists in games of chance when a gambler believes the next roll, throw, or hand will yield a favorable outcome to them personally. In essence it is this desire to receive gratification which causes the gambler to blindly believe in the next or the next, \textit{and so on}, outcome. Even if it becomes detrimental to the gambler personally, they are chasing the gratification which helps them feel special and therefore desired. Another maleficent side effect of religion involves radical fanaticism. An example of this is clear in present day Central Asia where interpretation of religious texts leads individuals to control, manipulate, and/or coerce others into adhering to radical laws, activities, and practices. In other words it creates a superiority complex against \textit{non-believers} which yields feelings of gratification for spreading \textit{their word}, i.e. the \textit{word of god}, which helps them feel special and therefore desired among their followers and deities alike. Therefore, organized religion fills the human void of desire to feel special. Thus, it is a dangerous addiction and creates more detriment than benevolence.
\begin{center}
word count: \underline{210}
\end{center}
\end{document}
