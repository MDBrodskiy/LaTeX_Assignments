%%%%%%%%%%%%%%%%%%%%%%%%%%%%%%%%%%%%%%%%%%%%%%%%%%%%%%%%%%%%%%%%%%%%%%%%%%%%%%%%%%%%%%%%%%%%%%%%%%%%%%%%%%%%%%%%%%%%%%%%%%%%%%%%%%%%%%%%%%%%%%%%%%%%%%%%%%%%%%%%%%%%%%%%%%%%%%%%%%%%%%%%%%%%
% Written By Michael Brodskiy
% Class: AP English Literature & Composition
% Instructor: Mr. Cautero
%%%%%%%%%%%%%%%%%%%%%%%%%%%%%%%%%%%%%%%%%%%%%%%%%%%%%%%%%%%%%%%%%%%%%%%%%%%%%%%%%%%%%%%%%%%%%%%%%%%%%%%%%%%%%%%%%%%%%%%%%%%%%%%%%%%%%%%%%%%%%%%%%%%%%%%%%%%%%%%%%%%%%%%%%%%%%%%%%%%%%%%%%%%%

\documentclass[12pt]{article} 
\usepackage{alphalph}
\usepackage[utf8]{inputenc}
\usepackage[russian,english]{babel}
\usepackage{titling}
\usepackage{amsmath}
\usepackage{graphicx}
\usepackage{enumitem}
\usepackage{amssymb}
\usepackage{physics}
\usepackage{tikz}
\usepackage{mathdots}
\usepackage{yhmath}
\usepackage{cancel}
\usepackage{color}
\usepackage{siunitx}
\usepackage{array}
\usepackage{multirow}
\usepackage{gensymb}
\usepackage{tabularx}
\usepackage{booktabs}
\usepackage{soul}
\usetikzlibrary{fadings}
\usetikzlibrary{patterns}
\usetikzlibrary{shadows.blur}
\usetikzlibrary{shapes}
\usepackage[super]{nth}
\usepackage{expl3}
\usepackage[version=4]{mhchem}
\usepackage{hpstatement}
\usepackage{rsphrase}
\usepackage{everysel}
\usepackage{ragged2e}
\usepackage{geometry}
\usepackage{fancyhdr}
\usepackage{cancel}
\geometry{top=1.0in,bottom=1.0in,left=1.0in,right=1.0in}
\newcommand{\subtitle}[1]{%
  \posttitle{%
    \par\end{center}
    \begin{center}\large#1\end{center}
    \vskip0.5em}%

}
\usepackage{hyperref}
\hypersetup{
colorlinks=true,
linkcolor=blue,
filecolor=magenta,      
urlcolor=blue,
citecolor=blue,
}

\urlstyle{same}


\title{Crime and Punishment Quiz 2}
\date{\today $-$ Period 5}
\author{Michael Brodskiy\\ \small Instructor: Mr. Cautero}

% Mathematical Operations:

% Sum: $$\sum_{n=a}^{b} f(x) $$
% Integral: $$\int_{lower}^{upper} f(x) dx$$
% Limit: $$\lim_{x\to\infty} f(x)$$

\begin{document}

\maketitle

\begin{enumerate}

\item (Bottom of 299 to End of Chapter) Discuss the author's use of at least TWO literary techniques and how they enhance meaning at this point in the novel

  \begin{justify}

    In the hallway encounter between Razumikhin and Raskolnikov, Dostoevsky utilizes a significant syntactical structure and allusion to highlight the importance of this interaction. Firstly, as Razumikhin and Raskolnikov stand, looking at each other in silence, Dostoevsky writes, “All his life Razumikhin would remember this moment.” The author then goes on to describe Raskolnikov's intense gaze. In this manner, it is evident that an implicit realization has passed to Razumikhin — a suspicion that may have occurred as early as when Razumikhin left Raskolnikov's place for the first time, Zosimov by his side. During that scene, Razumikhin seems to come to the realization that Raskolnikov only stirred in his 'fever' when the murder of Aliona Ivanovna was mentioned. Since that scene, Razumikhin has assisted Raskolnikov in many ways, especially during encounters with police or authorities (such as Zamiotov or Porfiry Petrovich), by saying that Raskolnikov was ill, and, therefore, could not have committed the crime. Though Razumikhin supports Raskolnikov in such a way, there was never a clear indication to Razumikhin that Raskolnikov had committed the crime — that is, until Raskolnikov's “burning and intent gaze” made Razumikhin shudder, confirming his suspicion. Furthermore, the author uses the explicit diction that “Razumikhin turned pale as a corpse.” The use of corpse is meant to signify that it was this exact secret — the secret of Raskolnikov's murders — that had passed between them. Therefore, the placement of the third-person omniscient sentence signifying Razumikhin's realization, considered in tandem with Razumikhin turning as pale as a corpse, convey the importance and the weight of the secret passed on implicitly to Razumikhin. With the realization of this secret, Razumikhin understands he must take the place of Raskolnikov, signified by the condcluding sentence of chapter three, which states “Razumikhin took his place with them [Raskolnikov's family] as a son and a brother.”

  \end{justify}

  \end{enumerate}

\end{document}

