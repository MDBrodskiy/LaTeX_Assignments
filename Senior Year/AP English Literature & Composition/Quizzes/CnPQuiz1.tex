%%%%%%%%%%%%%%%%%%%%%%%%%%%%%%%%%%%%%%%%%%%%%%%%%%%%%%%%%%%%%%%%%%%%%%%%%%%%%%%%%%%%%%%%%%%%%%%%%%%%%%%%%%%%%%%%%%%%%%%%%%%%%%%%%%%%%%%%%%%%%%%%%%%%%%%%%%%%%%%%%%%%%%%%%%%%%%%%%%%%%%%%%%%%
% Written By Michael Brodskiy
% Class: AP English Literature & Composition
% Instructor: Mr. Cautero
%%%%%%%%%%%%%%%%%%%%%%%%%%%%%%%%%%%%%%%%%%%%%%%%%%%%%%%%%%%%%%%%%%%%%%%%%%%%%%%%%%%%%%%%%%%%%%%%%%%%%%%%%%%%%%%%%%%%%%%%%%%%%%%%%%%%%%%%%%%%%%%%%%%%%%%%%%%%%%%%%%%%%%%%%%%%%%%%%%%%%%%%%%%%

\documentclass[12pt]{article} 
\usepackage{alphalph}
\usepackage[utf8]{inputenc}
\usepackage[russian,english]{babel}
\usepackage{titling}
\usepackage{amsmath}
\usepackage{graphicx}
\usepackage{enumitem}
\usepackage{amssymb}
\usepackage{physics}
\usepackage{tikz}
\usepackage{mathdots}
\usepackage{yhmath}
\usepackage{cancel}
\usepackage{color}
\usepackage{siunitx}
\usepackage{array}
\usepackage{multirow}
\usepackage{gensymb}
\usepackage{tabularx}
\usepackage{booktabs}
\usepackage{soul}
\usetikzlibrary{fadings}
\usetikzlibrary{patterns}
\usetikzlibrary{shadows.blur}
\usetikzlibrary{shapes}
\usepackage[super]{nth}
\usepackage{expl3}
\usepackage[version=4]{mhchem}
\usepackage{hpstatement}
\usepackage{rsphrase}
\usepackage{everysel}
\usepackage{ragged2e}
\usepackage{geometry}
\usepackage{fancyhdr}
\usepackage{cancel}
\geometry{top=1.0in,bottom=1.0in,left=1.0in,right=1.0in}
\newcommand{\subtitle}[1]{%
  \posttitle{%
    \par\end{center}
    \begin{center}\large#1\end{center}
    \vskip0.5em}%

}
\usepackage{hyperref}
\hypersetup{
colorlinks=true,
linkcolor=blue,
filecolor=magenta,      
urlcolor=blue,
citecolor=blue,
}

\urlstyle{same}


\title{Crime and Punishment Quiz 1}
\date{\today $-$ Period 5}
\author{Michael Brodskiy\\ \small Instructor: Mr. Cautero}

% Mathematical Operations:

% Sum: $$\sum_{n=a}^{b} f(x) $$
% Integral: $$\int_{lower}^{upper} f(x) dx$$
% Limit: $$\lim_{x\to\infty} f(x)$$

\begin{document}

\maketitle

\begin{enumerate}

  \item Describes Raskolnikov's reaction to the woman jumping off the bridge, and what does he then decide to do? 

    \begin{justify}

      Initially, upon witnessing the women jump off of the bridge, Raskolnikov watches somewhat blankly, with no emotion whatsoever. After a few moments of consideration, though, Raskolnikov is led to the conclusion that he should turn himself in; however, he begins taking various, longer routes to the police station, rather than making a beeline for it, showing his internal and implicit uncertainty, which contradicts his explicit thoughts of turning himself in. Raskolnikov most likely arrived at this decision by concluding that the women was driven to (attempted) suicide through mental torment, akin to his own torment following the murder. Raskolnikov's intentional waste of time was simply a way for him to actually avoid carrying out his own verdict, though he knows it is the right thing to do. Marmeladov's death ultimately “saves” him from visiting the police station, as Raskolnikov rushes to do a good deed, as if it would save him from the mental stress.
      
    \end{justify}

  \item In Raskolnikov's encounter with his mother and sister (page 186), what are two literary terms that are used?

    \begin{justify}

      Most importantly, in this encounter, the author utilizes various similes and repetition of the word “both”, to refer to Raskolnikov and his family, respectively. First of all, the author states, “he did not know why they had been the last people in the world he had expected, \dots, although twice that day he had received news they were on their way,” yet Raskolnikov “stood there as one dead”, with a “sudden intolerable awareness”. This contradictory reaction complements the aforementioned mental torment manifesting itself inside of Raskolnikov, as his guilt is transforming him into a paranoiac, who does not even trust his own family. In addition to this, any actions performed by Raskolnikov's kin were done together, as the author says “[b]oth had been weeping”, “[b]oth \dots had been through agonies of suspense”, and “[b]oth flung themselves in his arms”. As such, the repetitive use of the word “both” strengthen the idea that Raskolnikov's mental distress is making him paranoid of all, as “both” implies that he perceives the actions of his kin as a single entity or a unified being. He does not see them as actual people close to him, but rather as an additional problem to his ever-increasing mental stress. In this manner, it is evident that, to Raskolnikov, the presence of his family members are only a threat to his own freedom.

    \end{justify}

\end{enumerate}

\end{document}

