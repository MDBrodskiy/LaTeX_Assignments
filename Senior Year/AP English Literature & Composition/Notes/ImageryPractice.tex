%%%%%%%%%%%%%%%%%%%%%%%%%%%%%%%%%%%%%%%%%%%%%%%%%%%%%%%%%%%%%%%%%%%%%%%%%%%%%%%%%%%%%%%%%%%%%%%%%%%%%%%%%%%%%%%%%%%%%%%%%%%%%%%%%%%%%%%%%%%%%%%%%%%%%%%%%%%%%%%%%%%%%%%%%%%%%%%%%%%%%%%%%%%%
% Written By Michael Brodskiy
% Class: AP English Literature & Composition
% Instructor: Mr. Cautero
%%%%%%%%%%%%%%%%%%%%%%%%%%%%%%%%%%%%%%%%%%%%%%%%%%%%%%%%%%%%%%%%%%%%%%%%%%%%%%%%%%%%%%%%%%%%%%%%%%%%%%%%%%%%%%%%%%%%%%%%%%%%%%%%%%%%%%%%%%%%%%%%%%%%%%%%%%%%%%%%%%%%%%%%%%%%%%%%%%%%%%%%%%%%

\documentclass[12pt]{article} 
\usepackage{alphalph}
\usepackage[utf8]{inputenc}
\usepackage[russian,english]{babel}
\usepackage{titling}
\usepackage{amsmath}
\usepackage{graphicx}
\usepackage{enumitem}
\usepackage{amssymb}
\usepackage{physics}
\usepackage{tikz}
\usepackage{mathdots}
\usepackage{yhmath}
\usepackage{cancel}
\usepackage{color}
\usepackage{siunitx}
\usepackage{array}
\usepackage{multirow}
\usepackage{gensymb}
\usepackage{tabularx}
\usepackage{booktabs}
\usepackage{soul}
\usetikzlibrary{fadings}
\usetikzlibrary{patterns}
\usetikzlibrary{shadows.blur}
\usetikzlibrary{shapes}
\usepackage[super]{nth}
\usepackage{expl3}
\usepackage[version=4]{mhchem}
\usepackage{hpstatement}
\usepackage{rsphrase}
\usepackage{everysel}
\usepackage{ragged2e}
\usepackage{geometry}
\usepackage{fancyhdr}
\usepackage{cancel}
\geometry{top=1.0in,bottom=1.0in,left=1.0in,right=1.0in}
\newcommand{\subtitle}[1]{%
  \posttitle{%
    \par\end{center}
    \begin{center}\large#1\end{center}
    \vskip0.5em}%

}
\usepackage{hyperref}
\hypersetup{
colorlinks=true,
linkcolor=blue,
filecolor=magenta,      
urlcolor=blue,
citecolor=blue,
}

\urlstyle{same}


\title{Imagery Practice}
\date{\today $-$ Period 5}
\author{Michael Brodskiy\\ \small Instructor: Mr. Cautero}

% Mathematical Operations:

% Sum: $$\sum_{n=a}^{b} f(x) $$
% Integral: $$\int_{lower}^{upper} f(x) dx$$
% Limit: $$\lim_{x\to\infty} f(x)$$

\begin{document}

\maketitle

\begin{enumerate}

  \item \begin{tabular}{|c|c|c|c|} \hline \textsc{Visual} & \textsc{Olfactory} & \textsc{Auditory} & \textsc{Gustatory}\\ \hline Dirty Plates & Rank & Snored & Milk\\ \hline Small Table & Dirty Plates & Tones & Food\\ \hline Disheveled & & Rousing with Anger & \\ \hline She Lay & & Cry & \\\hline  \end{tabular}

  \item The piece transitions from a disgusted description of the woman and her conditions, to a vivid image of the woman sleeping. For example, the first stanza describes the conditions as “dirty”, “small”, “rank”, and “disheveled”. As such, this seems repulsive and nasty, as no one would want to spend their time in a room that is tiny and crammed with filth, stench, and disorder. In the second stanza, though, the woman is described as “wrinkled”, “nearly blind”, and her actions “rouse with anger \dots to cry for food”. This creates a pitiful image, as the reader may come to the conclusion that the disastrous living conditions are a result of her inability to take care of herself, whether it be due to old age or visual problems.

\end{enumerate}

\end{document}

