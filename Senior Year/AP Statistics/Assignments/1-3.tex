%%%%%%%%%%%%%%%%%%%%%%%%%%%%%%%%%%%%%%%%%%%%%%%%%%%%%%%%%%%%%%%%%%%%%%%%%%%%%%%%%%%%%%%%%%%%%%%%%%%%%%%%%%%%%%%%%%%%%%%%%%%%%%%%%%%%%%%%%%%%%%%%%%%%%%%%%%%%%%%%%%%%%%%%%%%%%%%%%%%%%%%%%%%%
% Written By Michael Brodskiy
% Class: AP Statistics
% Professor: M. Thompson
%%%%%%%%%%%%%%%%%%%%%%%%%%%%%%%%%%%%%%%%%%%%%%%%%%%%%%%%%%%%%%%%%%%%%%%%%%%%%%%%%%%%%%%%%%%%%%%%%%%%%%%%%%%%%%%%%%%%%%%%%%%%%%%%%%%%%%%%%%%%%%%%%%%%%%%%%%%%%%%%%%%%%%%%%%%%%%%%%%%%%%%%%%%%

\documentclass[12pt]{article} 
\usepackage{alphalph}
\usepackage[utf8]{inputenc}
\usepackage[russian,english]{babel}
\usepackage{titling}
\usepackage{amsmath}
\usepackage{graphicx}
\usepackage{enumitem}
\usepackage{amssymb}
\usepackage[super]{nth}
\usepackage{expl3}
\usepackage[version=4]{mhchem}
\usepackage{hpstatement}
\usepackage{rsphrase}
\usepackage{everysel}
\usepackage{ragged2e}
\usepackage{geometry}
\usepackage{fancyhdr}
\usepackage{cancel}
\usepackage{siunitx}
\usepackage{multicol}
\usepackage{pgfplots}
\usepgfplotslibrary{fillbetween}
\geometry{top=1.0in,bottom=1.0in,left=1.0in,right=1.0in}
\newcommand{\subtitle}[1]{%
  \posttitle{%
    \par\end{center}
    \begin{center}\large#1\end{center}
    \vskip0.5em}%

}
\DeclareSIUnit\Molar{\textsc{m}}
\usepackage{hyperref}
\hypersetup{
colorlinks=true,
linkcolor=blue,
filecolor=magenta,      
urlcolor=blue,
citecolor=blue,
}

\urlstyle{same}


\title{1.3 Homework\\45$-$49 odd, 55, 59, 63, 69, 79}
\date{August 20, 2021}
\author{Michael Brodskiy\\ \small Instructor: Mr. Thompson}

% Mathematical Operations:

% Sum: $$\sum_{n=a}^{b} f(x) $$
% Integral: $$\int_{lower}^{upper} f(x) dx$$
% Limit: $$\lim_{x\to\infty} f(x)$$

\begin{document}

\maketitle

\begin{enumerate}

    \setcounter{enumi}{44}

  \item

    \begin{enumerate}

      \item Dotplot:

        \begin{center}

            \begin{tikzpicture}
              \begin{axis}[ title={Hours of Sleep per Student},xlabel={Hours of Sleep}, ylabel={Number of Students}, yticklabels={,,2,4,6}, ymin=0, ymax=7, xline, xmin=0, xmax=12]
        \addplot[scatter,only marks, scatter src=y] coordinates { (9,0) (6,0) (8,0) (7,0) (8,1) (8,2) (6,1) (6.5,0) (7,1) (7,2) (9,1) (4,0) (3,0) (4,1) (5,0) (6,2) (11,0) (6,3) (3,1) (7,3) (6,4) (10,0) (7,4) (8,3) (4.5,0) (9,2) (7,5) (7,6) };
                \end{axis}
            \end{tikzpicture}
    
        \end{center}

      \item 5 out of the 28 students

    \end{enumerate}

    \setcounter{enumi}{46}

  \item

    \begin{enumerate}

      \item The dot above 3 signifies one game in which 3 goals were scored by the team

      \item The graph shows that the team scored a median of 2 goals and a mean of 3.75. The two outliers, 9 and 10, effect the mean greatly, but the median signifies that 2 goals is the center amount scored. Two goals is quite well, which shows that the team performed well.

    \end{enumerate}

    \setcounter{enumi}{48}

  \item The distribution looks approximately symmetric, as there are no significant tails to the left or right, and no significant outliers.

    \setcounter{enumi}{54}

  \item The New Jersey family income distribution is much more spread out as compared to the Indiana distribution. Both seem skewed right, as they have tails trailing to the right; however, New Jersey has more outliers. 

    \setcounter{enumi}{58}

  \item

    \begin{enumerate}

      \item Stemplot:
        
        \begin{center}
          \begin{tabular}{r | l l l l l l l l l} 15 & 9 & & & &  & & & &\\ 16 & 0 & 5 & 5 & 6 & 7 & 8 & & & \\ 17 & 1 & 1 & 1 & 3 & 4 & 4 & 7 & 7 & 8\\ 18 & & & & & & & & &\\ 19 & 2 & & & & & & & & \end{tabular}
        \end{center}
        
        \begin{flushright}
          \hfill Key:\newline
          \begin{tabular}{r | l} 17 & 1 \end{tabular} = 17.1
        \end{flushright}

      \item There is an outlier at 19.2

      \item 7 of these bars are less than advertised

    \end{enumerate}

    \setcounter{enumi}{62}

  \item

    \begin{enumerate}

      \item As a general rule of thumb, there should be at least 5 stems. This is done to better spread out and view the data to come to a more plausible conclusion.

      \item Key: \begin{tabular}{r | l} 13 & 1  \end{tabular} = 13.1

      \item The distribution is roughly symmetric, as there are no significant tails. There is, however, a significant outlier at 16.0

    \end{enumerate}

    \setcounter{enumi}{68}

  \item This histogram is roughly symmetric. There are no outliers, and the range is 40. The median is between 30-40.

    \begin{center}
        \begin{tikzpicture}
          \begin{axis}[xlabel={Third-grade DRP Scores}, ylabel={Frequency}, title={Frequency of Third-grade DRP Scores}, ymin=0, ymax=15, minor y tick num = 4, area style]
              \addplot+[ybar interval,mark=no] plot coordinates {(0, 0) (10, 6) (20, 9) (30, 13) (40, 12) (50, 4) (60, 0)};
            \end{axis}
        \end{tikzpicture}
    \end{center}

    \setcounter{enumi}{78}

\newpage

  \item 

    \begin{enumerate}

      \item Plots:

        \begin{multicols}{2}

    \begin{center}
      \begin{tikzpicture}
        \begin{axis}[xlabel={AP Calculus AB Score}, ylabel={Frequency}, title={Score Distribution}, symbolic x coords={1, 2, 3, 4, 5}, xtick=data, ymin=0, width=.425\textwidth]
            \addplot[ybar,fill=red!50] coordinates {(1, 94712) (2,  30017) (3, 53533) (4, 53467) (5, 76486)};
            \end{axis}
        \end{tikzpicture}
        \begin{tikzpicture}
          \begin{axis}[xlabel={AP Statistics Score}, ylabel={}, title={Score Distribution}, symbolic x coords={1, 2, 3, 4, 5}, xtick=data, ymin=0, ymax=100000, width=.425\textwidth]
            \addplot[ybar,fill=blue!50] coordinates {(1, 48565) (2,  32120) (3, 51367) (4, 44884) (5, 29627)};
            \end{axis}
        \end{tikzpicture}
    \end{center}

  \end{multicols}

      \item The variability of the frequency of calculus AB scores is much greater, with the range being roughly 3 times that of the statistics scores. Both, however, appear to be bimodal, with calculus AB having peaks at scores of 1 and 5, and statistics having peaks at scores of 1 and 3. As compared to statistics, calculus AB has a much higher 5 and 1 rate.

    \end{enumerate}

\end{enumerate}

\end{document}

