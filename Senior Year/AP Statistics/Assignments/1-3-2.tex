%%%%%%%%%%%%%%%%%%%%%%%%%%%%%%%%%%%%%%%%%%%%%%%%%%%%%%%%%%%%%%%%%%%%%%%%%%%%%%%%%%%%%%%%%%%%%%%%%%%%%%%%%%%%%%%%%%%%%%%%%%%%%%%%%%%%%%%%%%%%%%%%%%%%%%%%%%%%%%%%%%%%%%%%%%%%%%%%%%%%%%%%%%%%
% Written By Michael Brodskiy
% Class: AP Statistics
% Professor: M. Thompson
%%%%%%%%%%%%%%%%%%%%%%%%%%%%%%%%%%%%%%%%%%%%%%%%%%%%%%%%%%%%%%%%%%%%%%%%%%%%%%%%%%%%%%%%%%%%%%%%%%%%%%%%%%%%%%%%%%%%%%%%%%%%%%%%%%%%%%%%%%%%%%%%%%%%%%%%%%%%%%%%%%%%%%%%%%%%%%%%%%%%%%%%%%%%

\documentclass[12pt]{article} 
\usepackage{alphalph}
\usepackage[utf8]{inputenc}
\usepackage[russian,english]{babel}
\usepackage{titling}
\usepackage{amsmath}
\usepackage{graphicx}
\usepackage{enumitem}
\usepackage{amssymb}
\usepackage[super]{nth}
\usepackage{expl3}
\usepackage[version=4]{mhchem}
\usepackage{hpstatement}
\usepackage{rsphrase}
\usepackage{everysel}
\usepackage{ragged2e}
\usepackage{geometry}
\usepackage{fancyhdr}
\usepackage{cancel}
\usepackage{siunitx}
\usepackage{multicol}
\usepackage{pgfplots}
\usepgfplotslibrary{fillbetween}
\usepgfplotslibrary{statistics}
\geometry{top=1.0in,bottom=1.0in,left=1.0in,right=1.0in}
\newcommand{\subtitle}[1]{%
  \posttitle{%
    \par\end{center}
    \begin{center}\large#1\end{center}
    \vskip0.5em}%

}
\DeclareSIUnit\Molar{\textsc{m}}
\usepackage{hyperref}
\hypersetup{
colorlinks=true,
linkcolor=blue,
filecolor=magenta,      
urlcolor=blue,
citecolor=blue,
}

\urlstyle{same}


\title{1.3 Homework part 2\\ 87$-$97 odd, 109, 111}
\date{August 23, 2021}
\author{Michael Brodskiy\\ \small Instructor: Mr. Thompson}

% Mathematical Operations:

% Sum: $$\sum_{n=a}^{b} f(x) $$
% Integral: $$\int_{lower}^{upper} f(x) dx$$
% Limit: $$\lim_{x\to\infty} f(x)$$

\begin{document}

\maketitle

\begin{enumerate}

    \setcounter{enumi}{86}

  \item

    \begin{enumerate}

      \item $\frac{86 + 84 + 91 + 75 + 78 + 80 + 74 + 87 + 76 + 96 + 82 + 90 + 98 + 93}{14}=85$ points

      \item $\frac{86 + 84 + 91 + 75 + 78 + 80 + 74 + 87 + 76 + 96 + 82 + 90 + 98 + 93 + 0}{15}=79.\bar{3}$ points. This demonstrates the non-resistant characteristics of the mean (or that it can be influenced easily by outliers).

    \end{enumerate}

    \setcounter{enumi}{88}

  \item

    \begin{enumerate}

      \item $\{ 74, 75, 76, 78, 80, 82, 84, 86, 87, 90, 91, 93, 96, 98\}\rightarrow \frac{84 + 86}{2}=85$ points

      \item $\{ 0, 74, 75, 76, 78, 80, 82, 84, 86, 87, 90, 91, 93, 96, 98\}\rightarrow 84$ points. The median is resistant to outliers, which means that the absence has very little influence over the grade.

    \end{enumerate}

    \setcounter{enumi}{90}

  \item

    \begin{enumerate}

      \item The median is 8

      \item Because of the outlier, the mean is most likely much greater than the median

      \item Because it describes the 50 states (and District of Columbia), it must be a parameter

    \end{enumerate}

    \setcounter{enumi}{92}

  \item The mean is most likely \$276,200, while the median is probably \$234,200. House prices are most likely to have expensive/high outliers, which, due to the non-resistant nature of the mean, makes the mean significantly greater than the median.

    \setcounter{enumi}{94}

  \item

    \begin{enumerate}

      \item Because this distribution has an even number of terms, the average of the \nth{37} and \nth{38} terms is the median. According to the histogram, this is equal to 2. 

      \item $\bar{x}=\frac{11+30+33+32+25+18+21+24}{74}=2.62$

    \end{enumerate}

    \setcounter{enumi}{96}

  \item

    \begin{enumerate}

      \item Before: $98 - 74=24$ points; After: $98 - 0=98$ points

      \item Because range is non-resistant, it is a very poor measure of variability. Part (a) confirms this.

    \end{enumerate}

    \setcounter{enumi}{108}

  \item

    \begin{enumerate}

      \item Because of the difference between $Q_3$ and the max, this distribution is most likely skewed far to the right

      \item 21.70 means that, on average, each value in the distribution is 21.70 away from the mean

      \item $Q_3-Q_1=26.13\rightarrow Q_1-1.5(26.13)=-19.93\text{ and } Q_3+1.5(26.13)=84.6\rightarrow (-19.93,84.6)$. Because the max is greater than the upper fence, there must be at least one outlier present in the data set

    \end{enumerate}

    \setcounter{enumi}{110}

  \item $\{ 0, 0, 0, 1, 1, 3, 3, 5, 5, 7, 8, 8, 9, 14, 25, 25, 26, 29, 42, 44, 52, 72, 92, 98, 118  \}\rightarrow\left\{\begin{array}{c c} \text{Min:} & 0\\ Q_1: & 3\\ \text{Med:} & 9\\ Q_3: & 43\\ \text{Max:} & 118  \end{array}$

    \begin{enumerate}

      \item Box Plot:

        \begin{center}
            \begin{tikzpicture}
              \begin{axis} [axis y line=none,axis x line=bottom,yticklabel={\ },xlabel={Texts Sent in the Last 24 Hours},height={.175\textheight},width={.75\textwidth}]
                \addplot+ [boxplot prepared={lower whisker=0, lower quartile=3, median=9, upper quartile=43, upper whisker=118, box extend=.1, whisker extend=.15, every box/.style={very thick,fill=blue!40,draw=blue},every median/.style={very thick, draw=blue},every whisker/.style={blue,very thick},},]
                    table[] {};
                \end{axis}
            \end{tikzpicture}
          \end{center}

        \item The article is incorrect in its statements, as the third quartile is already less than the amount claimed in the article. This means that at least 75\% of teenagers text less than stated in the article. Additionally, the maximum number is an outlier, meaning that, most likely, more than 75\% of teenagers text less. In this manner, the article is incorrect

    \end{enumerate}

\end{enumerate}

\end{document}

