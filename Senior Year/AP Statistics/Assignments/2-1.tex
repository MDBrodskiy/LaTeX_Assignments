%%%%%%%%%%%%%%%%%%%%%%%%%%%%%%%%%%%%%%%%%%%%%%%%%%%%%%%%%%%%%%%%%%%%%%%%%%%%%%%%%%%%%%%%%%%%%%%%%%%%%%%%%%%%%%%%%%%%%%%%%%%%%%%%%%%%%%%%%%%%%%%%%%%%%%%%%%%%%%%%%%%%%%%%%%%%%%%%%%%%%%%%%%%%
% Written By Michael Brodskiy
% Class: AP Statistics
% Professor: M. Thompson
%%%%%%%%%%%%%%%%%%%%%%%%%%%%%%%%%%%%%%%%%%%%%%%%%%%%%%%%%%%%%%%%%%%%%%%%%%%%%%%%%%%%%%%%%%%%%%%%%%%%%%%%%%%%%%%%%%%%%%%%%%%%%%%%%%%%%%%%%%%%%%%%%%%%%%%%%%%%%%%%%%%%%%%%%%%%%%%%%%%%%%%%%%%%

\documentclass[12pt]{article} 
\usepackage{alphalph}
\usepackage[utf8]{inputenc}
\usepackage[russian,english]{babel}
\usepackage{titling}
\usepackage{amsmath}
\usepackage{graphicx}
\usepackage{enumitem}
\usepackage{amssymb}
\usepackage[super]{nth}
\usepackage{expl3}
\usepackage[version=4]{mhchem}
\usepackage{hpstatement}
\usepackage{rsphrase}
\usepackage{everysel}
\usepackage{ragged2e}
\usepackage{geometry}
\usepackage{fancyhdr}
\usepackage{cancel}
\usepackage{siunitx}
\usepackage{multicol}
\usepackage{pgfplots}
\usepgfplotslibrary{fillbetween}
\usepgfplotslibrary{statistics}
\geometry{top=1.0in,bottom=1.0in,left=1.0in,right=1.0in}
\newcommand{\subtitle}[1]{%
  \posttitle{%
    \par\end{center}
    \begin{center}\large#1\end{center}
    \vskip0.5em}%

}
\DeclareSIUnit\Molar{\textsc{m}}
\usepackage{hyperref}
\hypersetup{
colorlinks=true,
linkcolor=blue,
filecolor=magenta,      
urlcolor=blue,
citecolor=blue,
}

\urlstyle{same}


\title{2.1 Homework Worksheet}
\date{September 3, 2021}
\author{Michael Brodskiy\\ \small Instructor: Mr. Thompson}

% Mathematical Operations:

% Sum: $$\sum_{n=a}^{b} f(x) $$
% Integral: $$\int_{lower}^{upper} f(x) dx$$
% Limit: $$\lim_{x\to\infty} f(x)$$

\begin{document}

\maketitle

\begin{enumerate}

  \item

    \begin{enumerate}

      \item $\frac{6}{20}=.3\rightarrow \nth{30}$ percentile

      \item $\frac{18}{20}=.9\rightarrow\nth{90}$ percentile

      \item The boy with 22 pairs of shoes is more unusual. Unlike the girl, whose collection of shoes is greater than or equal to 30\% of other girls, the boy has more than (or equal to) the amount of 90\% of other boys

    \end{enumerate}

  \item This means that 85\% of vehicle speeds on those roads are less than or equal to the speed limit. The other 15\% of speeds are greater than the speed limit.

  \item

    \begin{enumerate}

      \item $\frac{320-450}{70}=-1.86$ standard deviations

      \item $\frac{475-450}{70}=.36$ standard deviations

      \item $\frac{610-450}{70}=2.29$ standard deviations

    \end{enumerate}

  \item $\frac{680-500}{100}=1.8$ and $\frac{27-18}{6}=1.5$ standard deviations. This means that, interpreting the z-scores, Eleanor performed better, as her value is more above than mean than Gerald.

  \item $\frac{.42-.266}{.0371}=4.15$, $\frac{.406-.267}{.0326}=4.26$, and $\frac{.39-.261}{.0317}=4.07$ standard deviations. According to the z-scores, Ted Williams performed the best, with Ty Cobb in second, and George Brett in third.

  \item C

  \item C

  \item \textsc{skip}

  \item

    \begin{enumerate}

      \item This would mean the student is 2.2 standard deviations above the average test score

      \item This would mean the student is .4 standard deviations below the average test score

      \item This would mean the student is 1.8 standard deviations below the average test score

      \item This student is exactly one standard deviation above the average test score

      \item This student received the average score on the test

    \end{enumerate}

  \item

  \item

  \item

\end{enumerate}

\end{document}

