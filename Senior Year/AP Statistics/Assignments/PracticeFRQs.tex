%%%%%%%%%%%%%%%%%%%%%%%%%%%%%%%%%%%%%%%%%%%%%%%%%%%%%%%%%%%%%%%%%%%%%%%%%%%%%%%%%%%%%%%%%%%%%%%%%%%%%%%%%%%%%%%%%%%%%%%%%%%%%%%%%%%%%%%%%%%%%%%%%%%%%%%%%%%%%%%%%%%%%%%%%%%%%%%%%%%%%%%%%%%%
% Written By Michael Brodskiy
% Class: AP Statistics
% Professor: M. Thompson
%%%%%%%%%%%%%%%%%%%%%%%%%%%%%%%%%%%%%%%%%%%%%%%%%%%%%%%%%%%%%%%%%%%%%%%%%%%%%%%%%%%%%%%%%%%%%%%%%%%%%%%%%%%%%%%%%%%%%%%%%%%%%%%%%%%%%%%%%%%%%%%%%%%%%%%%%%%%%%%%%%%%%%%%%%%%%%%%%%%%%%%%%%%%

\documentclass[12pt]{article} 
\usepackage{alphalph}
\usepackage[utf8]{inputenc}
\usepackage[russian,english]{babel}
\usepackage{titling}
\usepackage{amsmath}
\usepackage{graphicx}
\usepackage{enumitem}
\usepackage{amssymb}
\usepackage[super]{nth}
\usepackage{expl3}
\usepackage[version=4]{mhchem}
\usepackage{hpstatement}
\usepackage{rsphrase}
\usepackage{everysel}
\usepackage{ragged2e}
\usepackage{geometry}
\usepackage{fancyhdr}
\usepackage{cancel}
\usepackage{siunitx}
\usepackage{multicol}
\usepackage{pgfplots}
\usepgfplotslibrary{fillbetween}
\usepgfplotslibrary{statistics}
\geometry{top=1.0in,bottom=1.0in,left=1.0in,right=1.0in}
\newcommand{\subtitle}[1]{%
  \posttitle{%
    \par\end{center}
    \begin{center}\large#1\end{center}
    \vskip0.5em}%

}
\DeclareSIUnit\Molar{\textsc{m}}
\usepackage{hyperref}
\hypersetup{
colorlinks=true,
linkcolor=blue,
filecolor=magenta,      
urlcolor=blue,
citecolor=blue,
}

\urlstyle{same}


\title{Practice FRQs}
\date{April 26, 2022}
\author{Michael Brodskiy\\ \small Instructor: Mr. Thompson}

% Mathematical Operations:

% Sum: $$\sum_{n=a}^{b} f(x) $$
% Integral: $$\int_{lower}^{upper} f(x) dx$$
% Limit: $$\lim_{x\to\infty} f(x)$$

\begin{document}

\maketitle

\begin{enumerate}

  \item

    \begin{enumerate}

      \item On Campus = $\frac{17 + 7}{33} = .7273$; Off Campus = $\frac{25 + 12}{67} = .5522$

      \item The graph shows that a higher proportion of off campus students do not participate in an activity (a little over 40\%) than on campus students (a little over 20\%). The groups are roughly likely to participate in two or more activities (a little over 20\% versus a little under 20\%). A much larger portion of on campus students participate in one activity (about 70\%) than do off campus students (a little under 40\%). Thus, there is some evidence to indicate that off campus students are less likely to be in an activity, though on campus students generally do one activity.

      \item Assuming a significance level of $\alpha = .05$, we do not have convincing evidence to suggest association between residential status and activity participation. Thus, because $.23 > .05$, we fail to reject $H_0$, and the administrator should conclude that there is no association.

    \end{enumerate}

  \item

    \begin{enumerate}

      \item $\frac{3}{9} \cdot \frac{2}{8} \cdot \frac{1}{7} = .0119$

      \item Because the probability is quite small (.0119), it would make sense to doubt management's claim, as an event like this is quite unlikely to happen.

      \item This appropriately models the given scenario. Because there are 6 men and 3 women, the ratio of man to woman must be kept $2:1$. Because the situation with the dice is $4:2$, the ratio is kept the same, and, thus, models the situation appropriately.

    \end{enumerate}

  \item

    \begin{enumerate}

      \item To use cluster sampling, the landlord should select each floor as a cluster. Because each floor has four apartments and eight are needed, two floors should be selected at random. This can be done by assigning numbers to each floor, say 0-8. Then, using a random number generator, two different values from 0 to 8 should be selected. The corresponding floors are selected.

      \item As the strata are children and no children, and there are 8 apartments with children and 24 without, the landlord should use simple random samples by numbering apartments with children 0-7 and apartments without children 0-23. After this, the landlord should generate two different, random integers from 0 to 7 and six different, random integers from 0 to 23. Each integer corresponds to a given apartment, with or without children.

    \end{enumerate}

  \item

    \begin{tabular}{|c|}
    \hline
    \textbf{State:}\\
    \hline
    \end{tabular}

    $H_0:$ There is no association between age group and eating five or more servings of fruits and vegetables a day; $H_a:$ There is an association between age group and eating five or more servings of fruits or vegetables a day\\
    $\alpha = .05$

    \begin{tabular}{|c|}
    \hline
    \textbf{Plan}\\
    \hline
    \end{tabular}

    Procedure: Chi-square test for independence\\
    Conditions: \begin{tabular}{l}Random: Stated in problem\\ 10\%: 8,866 adults $\leq\frac{1}{10}$ (all adults)\\ Large Counts: 240.2 \geq 5 \end{tabular}

    \begin{tabular}{|c|}
    \hline
    \textbf{Do:}\\
    \hline
    \end{tabular}

    $\chi^2 = \sum \frac{(O-E)^2}{E}$\\
    $\chi^2 = \frac{(231 - 240.2)^2}{240.2} + \frac{(741-731.8)^2}{731.8} + \dots + \frac{(3692-3751.6)^2}{3751.6}=8.98$\\
    $p(\chi^2 > 8.98,\text{ df}=2)=.0112$\\

    \begin{tabular}{|c|}
    \hline
    \textbf{Conclude:}\\
    \hline
    \end{tabular}

    The p-value is equal to .0112. This is the probability of obtaining a result equal to more extreme than obtained, assuming that there is no association between age group and eating five or more servings of fruit a day. Thus, because $.0112 < \alpha = .05$, we reject $H_0$, and conclude that there is an association between age group and eating five servings of fruit or vegetables a day.

  \item 

    \begin{tabular}{|c|}
    \hline
    \textbf{State:}\\
    \hline
    \end{tabular}

    $H_0: \mu_{\text{diff}} = 0, H_a: \mu_{\text{diff}} > 0$, where $\mu_{\text{diff}}$ is the difference in means between the price paid by women and the price paid by men for the same car model.\\
    $\alpha=.05, \bar{x}_{\text{diff}} = \$585, S_x = \$530.71, \text{df}=7$

    \begin{tabular}{|c|}
    \hline
    \textbf{Plan}\\
    \hline
    \end{tabular}

    One-sample $t^*$ interval for $\mu_{\text{diff}}$
    Conditions: \begin{tabular}{l}Random: Stated in problem\\ 10\%: 8 women and 8 men $<\frac{1}{10}$ (all women and men)\\ Normal: No significant skews or outliers \end{tabular}

    \begin{tabular}{|c|}
    \hline
    \textbf{Do:}\\
    \hline
    \end{tabular}

    $t^* = \frac{\bar{x}_{\text{diff}} - \mu_{\text{diff}}}{\left( \frac{S_x}{\sqrt{n}}} \right)$\\
    $t^* = 3.1178$\\
    $p(t^* > 3.1178,\text{ df}=7)=.0084$\\

    \begin{tabular}{|c|}
    \hline
    \textbf{Conclude:}\\
    \hline
    \end{tabular}

    The p-value is equal to .0084. This is the probability of obtaining a result equal to more extreme than obtained, assuming that there is no difference in the price paid by men and women. Thus, because $.0084 < \alpha = .05$, we reject $H_0$, and conclude that women do pay more than men for identical car models.

  \item 

    \begin{enumerate}

      \item $R = A - P\Rightarrow 5.88 - (-1.595789 + .0372614 \cdot 175) = .955$. This means that the observed FCR of a 175-inch car is .955 greater than the one predicted by the LSRL.

      \item

        \begin{enumerate}

          \item $A$ is located at the point at (93, .955)

          \item This indicates that $B$ is nearly equal to the FCR predicted by the LSRL, as it means that the residual is nearly zero.

        \end{enumerate}

      \item Both graphs seem to represent a positive, linear association. Graph II, however, seems to portray a stronger association than the one shown in Graph III. This means that, most likely, engine size has more of an effect on FCR than wheel base; however, this can not be confirmed without a significance test. Graph III seems to have more variability than Graph II.

      \item Jamal should utilize engine size. This is evident by the fact that Graph II, which corresponds to engine size, contains a stronger association than that of Graph III. Thus, it is more likely to affect FCR.

    \end{enumerate}

\end{enumerate}

\end{document}

