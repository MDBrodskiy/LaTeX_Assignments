%%%%%%%%%%%%%%%%%%%%%%%%%%%%%%%%%%%%%%%%%%%%%%%%%%%%%%%%%%%%%%%%%%%%%%%%%%%%%%%%%%%%%%%%%%%%%%%%%%%%%%%%%%%%%%%%%%%%%%%%%%%%%%%%%%%%%%%%%%%%%%%%%%%%%%%%%%%%%%%%%%%%%%%%%%%%%%%%%%%%%%%%%%%%
% Written By Michael Brodskiy
% Class: AP Macroeconomics
% Instructor: Mr. Bremer
%%%%%%%%%%%%%%%%%%%%%%%%%%%%%%%%%%%%%%%%%%%%%%%%%%%%%%%%%%%%%%%%%%%%%%%%%%%%%%%%%%%%%%%%%%%%%%%%%%%%%%%%%%%%%%%%%%%%%%%%%%%%%%%%%%%%%%%%%%%%%%%%%%%%%%%%%%%%%%%%%%%%%%%%%%%%%%%%%%%%%%%%%%%%

\documentclass[12pt]{article} 
\usepackage{alphalph}
\usepackage[utf8]{inputenc}
\usepackage[russian,english]{babel}
\usepackage{titling}
\usepackage{amsmath}
\usepackage{graphicx}
\usepackage{enumitem}
\usepackage{amssymb}
\usepackage{physics}
\usepackage{tikz}
\usepackage{mathdots}
\usepackage{yhmath}
\usepackage{cancel}
\usepackage{color}
\definecolor{yellow}{rgb}{.95, .95, 0.0}
\usepackage{siunitx}
\usepackage{array}
\usepackage{multirow}
\usepackage{gensymb}
\usepackage{tabularx}
\usepackage{booktabs}
\usepackage{soul}
\usepackage{multicol}
\usepackage{pgfplots}
\pgfplotsset{compat=newest}
\usetikzlibrary{fadings}
\usetikzlibrary{patterns}
\usetikzlibrary{shadows.blur}
\usetikzlibrary{shapes}
\usepackage[super]{nth}
\usepackage{expl3}
\usepackage[version=4]{mhchem}
\usepackage{hpstatement}
\usepackage{rsphrase}
\usepackage{everysel}
\usepackage{ragged2e}
\usepackage{geometry}
\usepackage{fancyhdr}
\usepackage{cancel}
\geometry{top=1.0in,bottom=1.0in,left=1.0in,right=1.0in}
\newcommand{\subtitle}[1]{%
  \posttitle{%
    \par\end{center}
    \begin{center}\large#1\end{center}
    \vskip0.5em}%

}
\usepackage{hyperref}
\hypersetup{
colorlinks=true,
linkcolor=blue,
filecolor=magenta,      
urlcolor=blue,
citecolor=blue,
}

\urlstyle{same}


\title{Notes — Week 8}
\date{Period 3}
\author{Michael Brodskiy\\ \small Instructor: Mr. Bremer}

\begin{document}

\maketitle

\begin{enumerate}

  \item Bonds

    \begin{itemize}

      \item Bonds represent debt

      \item Important characteristics:

        \begin{itemize}

          \item Term

          \item Interest rate/risk

          \item Tax treatment

        \end{itemize}

    \end{itemize}

  \item Stocks

    \begin{itemize}

      \item Stocks represent ownership in a corporation (\textit{i.e.\ equity})

        \begin{itemize}

          \item How is price set?

            \begin{itemize}

              \item Supply and Demand

            \end{itemize}

          \item Where are they traded?

            \begin{itemize}

              \item New York Stock Exchange

              \item NASDAQ

              \item Many More

            \end{itemize}

          \item How do we measure \underline{market} performance?

            \begin{itemize}

              \item Stock indices

                \begin{itemize}

                  \item S\&P 500

                  \item NASDAQ

                  \item Dow Jones Industrial Average

                \end{itemize}

            \end{itemize}

        \end{itemize}

      \item Making Money With Stocks

        \begin{itemize}

          \item Capital Gains

          \item Dividends

        \end{itemize}

    \end{itemize}

  \item Money must function as:

    \begin{enumerate}

      \item Medium of exchange

        \begin{itemize}

          \item “Greases the wheels” of transactions

        \end{itemize}

      \item Measure of value or “unit of account”

        \begin{itemize}

          \item Expresses worth in terms people understand

          \item Especially useful to compare value of dissimilar items

        \end{itemize}

      \item Store of Value

        \begin{itemize}

          \item Can have time between earning and spending

        \end{itemize}

    \end{enumerate}

  \item Commodity Money

    \begin{itemize}

      \item Money that has intrinsic value (\textit{i.e.\ something that can be used for something other than money, like gold or a tomato})

    \end{itemize}

  \item Fiat Money

    \begin{itemize}

      \item Money that has no intrinsic value

    \end{itemize}

  \item Characteristics of Money:

    \begin{itemize}

      \item Portability

      \item Durability

      \item Divisibility

      \item Divisibility

      \item Stability (in value)

    \end{itemize}

  \item Modern Money

    \begin{itemize}

      \item Coins and currency

      \item Demand and other checkable deposits

      \item Savings and time deposits

      \item $M_1 = $ coins and currency in circulation plus checking account balances (Note: vault cash in banks is NOT part of $M_1$)

      \item $M_2 = M_1$ plus money market funds, savings accounts, and certificates of deposit under \$100k

      \item $M_1$ is much more liquid\footnote{A measure of how quickly an asset may be converted to cash} than $M_2$

      \item Things in $M_2$ are much more interesting-bearing than $M_1$

    \end{itemize}

  \item Responsibilities of the Fed:

    \begin{itemize}

      \item Check clearing

      \item Bank regulation and supervision

      \item Consumer legislation

      \item Maintaining the currency

      \item Regulating the money supply

        \begin{itemize}

          \item Easy money policy — supply grows and stimulates the economy

          \item Tight money policy — restricts growth of the money supply

        \end{itemize}

    \end{itemize}

  \item How Banks Create Money

    \begin{itemize}

      \item Banks operate under a \textit{fractional reserve system}

        \begin{itemize}

          \item Required reserves

          \item Excess reserves

        \end{itemize}

      \item How much can be created?

        \begin{itemize}

          \item Money multiplier $= \frac{1}{\text{reserve ratio}}$

          \item Maximum amount created $= \frac{1}{\text{reserve ratio}}\cdot\text{deposit}$

        \end{itemize}

    \end{itemize}

  \item Be Careful:

    \begin{itemize}

      \item What is the source of deposit?

        \begin{itemize}

          \item Existing currency

          \item Fed purchase of securities
            
        \end{itemize}

      \item What is being asked?

        \begin{itemize}

          \item How much will $M_1$ change?

          \item How much will bank reserves change?

          \item How much will demand deposit or checking account balances change?

          \item $M_1 = \text{reserves} = \text{DD}$

          \item Loans$ = (1-RR)\cdot(M_1)$

        \end{itemize}

      \item Timeframe

        \begin{itemize}

          \item “Immediate” or “maximum” amount

        \end{itemize}
        
    \end{itemize}

  \item What could prevent multiple expansion?

    \begin{itemize}

      \item Banks don't make loans

      \item People don't borrow

      \item Money leaves the system

        \begin{itemize}

          \item Spent in Europe

          \item Buried in a backyard

        \end{itemize}

    \end{itemize}

  \item Interest Rates Matter

    \begin{center}
      \begin{tabular}{|c|c|c|}
        \hline
        Rate (\%) & Monthly ($1,000$s of \$) & Total ($1,000,000$s of \$)\\
        \hline
        1 & 40 & 11\\
        \hline
        3 & 55 & 13\\
        \hline
        5 & 66 & 15.8\\
        \hline
        7 & 78 & 18.6\\
        \hline
        9 & 90 & 21.6\\
        \hline
        11 & 103 & 25\\
        \hline
      \end{tabular}\\
      \vspace{10pt}
      \footnotesize \textit{Monthly and Total Cost Computed with \$10,000,000 loan for 20 years}
    \end{center}

    \begin{itemize}

      \item When interest rates are high, less investment spending will occur, while low interest rates mean high investment

      \item The Investment Demand Model:

        \begin{center}


        \begin{tikzpicture}
          \begin{axis}[ymin=0, xmin=750, xmax=1250, title={Investment-Demand Model}, xlabel={Investment per year (billions of base-year dollars)}, ylabel={Interest rate (percent)}, ymajorgrids=true, xmajorgrids=true, grid style=dashed]
            \addplot [domain=800:1125,black] {46-.04 * x};
            \node[label={0:{ID Curve}},circle,fill,inner sep=2pt, color=black] at (axis cs:1070,3.2);
                \end{axis}
            \end{tikzpicture}
          
        \end{center}

    \end{itemize}

  \item Monetary Policy Tools

    \begin{itemize}

      \item Reserve requirement

        \begin{itemize}

          \item Higher — Contracts money supply

          \item Lower — Expands money supply

        \end{itemize}

      \item Open market operations

        \begin{itemize}

          \item Sell — Contracts money supply

          \item Buy — Expands money supply

          \item Federal Funds Rate

            \begin{itemize}

              \item Very commonly used

              \item Fed sets “target”

              \item Rate banks charge each other to borrow

            \end{itemize}

          \item Discount Rate

            \begin{itemize}

              \item Banks borrow from the Fed to:

                \begin{itemize}

                  \item Make up reserves

                  \item Meet local or seasonal demands

                \end{itemize}

              \item Higher Rate — Contracts money supply

              \item Lower Rate — Expands money supply

            \end{itemize}

        \end{itemize}

    \end{itemize}

\end{enumerate}

\end{document}

