%%%%%%%%%%%%%%%%%%%%%%%%%%%%%%%%%%%%%%%%%%%%%%%%%%%%%%%%%%%%%%%%%%%%%%%%%%%%%%%%%%%%%%%%%%%%%%%%%%%%%%%%%%%%%%%%%%%%%%%%%%%%%%%%%%%%%%%%%%%%%%%%%%%%%%%%%%%%%%%%%%%%%%%%%%%%%%%%%%%%%%%%%%%%
% Written By Michael Brodskiy
% Class: AP Macroeconomics
% Instructor: Mr. Bremer
%%%%%%%%%%%%%%%%%%%%%%%%%%%%%%%%%%%%%%%%%%%%%%%%%%%%%%%%%%%%%%%%%%%%%%%%%%%%%%%%%%%%%%%%%%%%%%%%%%%%%%%%%%%%%%%%%%%%%%%%%%%%%%%%%%%%%%%%%%%%%%%%%%%%%%%%%%%%%%%%%%%%%%%%%%%%%%%%%%%%%%%%%%%%

\documentclass[12pt]{article} 
\usepackage{alphalph}
\usepackage[utf8]{inputenc}
\usepackage[russian,english]{babel}
\usepackage{titling}
\usepackage{amsmath}
\usepackage{graphicx}
\usepackage{enumitem}
\usepackage{amssymb}
\usepackage{physics}
\usepackage{tikz}
\usepackage{mathdots}
\usepackage{yhmath}
\usepackage{cancel}
\usepackage{color}
\usepackage{siunitx}
\usepackage{array}
\usepackage{multirow}
\usepackage{gensymb}
\usepackage{tabularx}
\usepackage{booktabs}
\usepackage{soul}
\usepackage{multicol}
\usepackage{pgfplots}
\pgfplotsset{compat=newest}
\usetikzlibrary{fadings}
\usetikzlibrary{patterns}
\usetikzlibrary{shadows.blur}
\usetikzlibrary{shapes}
\usepackage[super]{nth}
\usepackage{expl3}
\usepackage[version=4]{mhchem}
\usepackage{hpstatement}
\usepackage{rsphrase}
\usepackage{everysel}
\usepackage{ragged2e}
\usepackage{geometry}
\usepackage{fancyhdr}
\usepackage{cancel}
\geometry{top=1.0in,bottom=1.0in,left=1.0in,right=1.0in}
\newcommand{\subtitle}[1]{%
  \posttitle{%
    \par\end{center}
    \begin{center}\large#1\end{center}
    \vskip0.5em}%

}
\usepackage{hyperref}
\hypersetup{
colorlinks=true,
linkcolor=blue,
filecolor=magenta,      
urlcolor=blue,
citecolor=blue,
}

\urlstyle{same}


\title{Notes — Week 4}
\date{Period 3}
\author{Michael Brodskiy\\ \small Instructor: Mr. Bremer}

\begin{document}

\maketitle

\begin{enumerate}

  \item GDP and Prices

    \begin{enumerate}

      \item Focus needs to be on growth \undeline{not} prices

      \item To compensate, we use a price index

        \begin{itemize}

          \item Choose a base year

          \item Choose a market basket of goods and services

        \end{itemize}

      \item Consumer Price Index (CPI)

        \begin{itemize}

          \item Thousands of consumer items

          \item Base years 1982$-$1984 (average of those)

          \item Compiled monthly

          \item $CPI = \frac{y_{curr}}{y_{base}}\cdot100$

        \end{itemize}

      \item Real vs. Current GDP

        \begin{itemize}

          \item Nominal or current — \underline{Not} adjusted for inflation

          \item Real — \underline{Adjusted} for inflation

        \end{itemize}

      \item GDP per capita is useful for comparison of countries

      \item CPI vs. GDP Deflator

        \begin{center}
          \begin{tabular}{|p{.45\textwidth}|p{.45\textwidth}|}
            \hline
            GDP Deflator & CPI\\
            \hline
            Ratio of nominal GDP to real GDP & Measure of a cost of a market basket of consumer goods\\
            \hline
            Counts all goods and services produced domestically & Counts all goods and services bought by consumers\\
            \hline
          \end{tabular}
        \end{center}

      \item GDP Deflator

        \begin{itemize}

          \item $r_{GDP} = \frac{GDP_{nom}}{GDP_{def}}\cdot 100$

          \item Any “real” calculation uses the simple formula $r_{x} = \frac{nominal}{index}\cdot 100$

        \end{itemize}

      \item Inflation:

        \begin{itemize}

          \item Winners: Debtors

          \item Losers: Creditors

        \end{itemize}

      \item Interest can be conceptualized as “rent” on money

      \item Interest rate formula:

        \begin{itemize}

          \item $nominal = real + inflation$

        \end{itemize}

    \end{enumerate}

\end{enumerate}

\end{document}

