%%%%%%%%%%%%%%%%%%%%%%%%%%%%%%%%%%%%%%%%%%%%%%%%%%%%%%%%%%%%%%%%%%%%%%%%%%%%%%%%%%%%%%%%%%%%%%%%%%%%%%%%%%%%%%%%%%%%%%%%%%%%%%%%%%%%%%%%%%%%%%%%%%%%%%%%%%%%%%%%%%%%%%%%%%%%%%%%%%%%%%%%%%%%
% Written By Michael Brodskiy
% Class: AP Macroeconomics
% Instructor: Mr. Bremer
%%%%%%%%%%%%%%%%%%%%%%%%%%%%%%%%%%%%%%%%%%%%%%%%%%%%%%%%%%%%%%%%%%%%%%%%%%%%%%%%%%%%%%%%%%%%%%%%%%%%%%%%%%%%%%%%%%%%%%%%%%%%%%%%%%%%%%%%%%%%%%%%%%%%%%%%%%%%%%%%%%%%%%%%%%%%%%%%%%%%%%%%%%%%

\documentclass[12pt]{article} 
\usepackage{alphalph}
\usepackage[utf8]{inputenc}
\usepackage[russian,english]{babel}
\usepackage{titling}
\usepackage{amsmath}
\usepackage{graphicx}
\usepackage{enumitem}
\usepackage{amssymb}
\usepackage{physics}
\usepackage{tikz}
\usepackage{mathdots}
\usepackage{yhmath}
\usepackage{cancel}
\usepackage{color}
\usepackage{xcolor}
\definecolor{yellow}{rgb}{.95, .95, 0.0}
\usepackage{siunitx}
\usepackage{array}
\usepackage{multirow}
\usepackage{gensymb}
\usepackage{tabularx}
\usepackage{booktabs}
\usepackage{soul}
\usepackage{multicol}
\usepackage{pgfplots}
\pgfplotsset{compat=newest}
\usetikzlibrary{fadings}
\usetikzlibrary{patterns}
\usetikzlibrary{shadows.blur}
\usetikzlibrary{shapes}
\usepackage[super]{nth}
\usepackage{expl3}
\usepackage[version=4]{mhchem}
\usepackage{hpstatement}
\usepackage{rsphrase}
\usepackage{everysel}
\usepackage{ragged2e}
\usepackage{geometry}
\usepackage{fancyhdr}
\usepackage{cancel}
\geometry{top=1.0in,bottom=1.0in,left=1.0in,right=1.0in}
\newcommand{\subtitle}[1]{%
  \posttitle{%
    \par\end{center}
    \begin{center}\large#1\end{center}
    \vskip0.5em}%

}
\usepackage{hyperref}
\hypersetup{
colorlinks=true,
linkcolor=blue,
filecolor=magenta,      
urlcolor=blue,
citecolor=blue,
}

\urlstyle{same}


\title{Notes — Week 14}
\date{Period 3}
\author{Michael Brodskiy\\ \small Instructor: Mr. Bremer}

\begin{document}

\maketitle

\begin{enumerate}

  \item Open Economy

    \begin{itemize}

      \item $AD = C + I + G + $\begin{tabular}{|c|}\hline \hl{$NX$}\\ \hline \end{tabular}$\longleftarrow$ focus of this chapter 

    \end{itemize}

  \item Four Factors in an Open Economy:

    \begin{enumerate}

      \item Balance of Payments

      \item Exchange Rates

      \item Role of Central Bank

      \item Barriers to Trade

    \end{enumerate}

  \item Some Basics:

    \begin{itemize}

      \item Countries don't trade with each other — people do

      \item Financial capital (money) flows all around the world seeking the highest possible return

      \item In the \nth{21} century, making a financial investment in another country is a simple click away

      \item People want to be paid in their domestic currency

    \end{itemize}

  \item Balance of Payments Accounts — A summary of the country's transactions with other countries. There are two main types of transactions:

    \begin{enumerate}

      \item People trading currently produced goods or services for money

      \item People trading pre-existing assets for money

    \end{enumerate}

  \item Capital Account — Flow of currently produced goods and services

    \begin{itemize}

      \item Net exports

      \item Net financial investment income (income and dividend income)

      \item Net transfers (remittances and foreign aid)

    \end{itemize}

  \item Financial Account — Flow of Financial Capital (Pre-existing Assets)

    \begin{itemize}

      \item Real assets

      \item Financial assets

    \end{itemize}

  \item A deficit in one will be matched by a surplus in the other: $CA + FA = 0$

  \item Debits and Credits

    \begin{itemize}

      \item Debit (-) — Any transaction that requires foreign currency (\textit{i.e.} imports, money outflows)

      \item Credit (+) — Any transaction that earns foreign currency (\textit{i.e.} exports, money inflows)

    \end{itemize}

  \item Foreign Exchange

    \begin{itemize}

      \item An increase in the demand for one currency results in an increase in supply for another

    \end{itemize}

  \item Determinants of Demand for Currency

    \begin{itemize}

      \item Tastes and preferences

      \item Price Level — Inflation tends to depreciate currencies (don't conflate domestic monetary policy and foreign exchange markets)

      \item National income and relative income changes

      \item Interest rates

      \item Speculation/expectations

    \end{itemize}

    \item Role of Central Bank

      \begin{itemize}

          \item Three ways central bank impacts foreign exchange

            \begin{enumerate}

                \item Direct intervention in foreign exchange markets, buying or selling reserve currencies to influence exchange rates

                \item Manipulation of interest rates

                  \begin{itemize}

                      \item Domestic monetary policy and foreign exchange are separate worlds

                      \item Increasing money supply impacts foreign exchange, but NOT simply because supply of money has increased (in Unit 5 we learned causes inflation)

                      \item Link is monetary policy impact on interest rates, and how this attracts or discourages foreign purchases of financial assets and hence demand for the currency, and, hence, exchange rate

                    \end{itemize}

                    \item Inflation tends to cause currency to depreciate

                      \begin{itemize}

                        \item Exports become less competitive (Canadians want to buy less American maple syrup), causing demand for currency to decrease

                        \item Imports become relatively cheaper (Americans want to buy more Canadian maple syrup), causing supply of currency to increase

                        \end{itemize}

              \end{enumerate}

        \end{itemize}

      \item Barriers to Trade

        \begin{itemize}

          \item National Security

          \item Infant industry

          \item Saves jobs

        \end{itemize}

      \item How is Trade Restricted?

        \begin{itemize}

          \item Tariffs

          \item Quotas (VER's)

          \item Export subsidies

          \item Informal barriers (\textit{e.g.} “standards”)

        \end{itemize}

\end{enumerate}

\end{document}

