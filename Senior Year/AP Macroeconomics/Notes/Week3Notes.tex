%%%%%%%%%%%%%%%%%%%%%%%%%%%%%%%%%%%%%%%%%%%%%%%%%%%%%%%%%%%%%%%%%%%%%%%%%%%%%%%%%%%%%%%%%%%%%%%%%%%%%%%%%%%%%%%%%%%%%%%%%%%%%%%%%%%%%%%%%%%%%%%%%%%%%%%%%%%%%%%%%%%%%%%%%%%%%%%%%%%%%%%%%%%%
% Written By Michael Brodskiy
% Class: AP Macroeconomics
% Instructor: Mr. Bremer
%%%%%%%%%%%%%%%%%%%%%%%%%%%%%%%%%%%%%%%%%%%%%%%%%%%%%%%%%%%%%%%%%%%%%%%%%%%%%%%%%%%%%%%%%%%%%%%%%%%%%%%%%%%%%%%%%%%%%%%%%%%%%%%%%%%%%%%%%%%%%%%%%%%%%%%%%%%%%%%%%%%%%%%%%%%%%%%%%%%%%%%%%%%%

\documentclass[12pt]{article} 
\usepackage{alphalph}
\usepackage[utf8]{inputenc}
\usepackage[russian,english]{babel}
\usepackage{titling}
\usepackage{amsmath}
\usepackage{graphicx}
\usepackage{enumitem}
\usepackage{amssymb}
\usepackage{physics}
\usepackage{tikz}
\usepackage{mathdots}
\usepackage{yhmath}
\usepackage{cancel}
\usepackage{color}
\usepackage{siunitx}
\usepackage{array}
\usepackage{multirow}
\usepackage{gensymb}
\usepackage{tabularx}
\usepackage{booktabs}
\usepackage{soul}
\usepackage{multicol}
\usepackage{pgfplots}
\pgfplotsset{compat=newest}
\usetikzlibrary{fadings}
\usetikzlibrary{patterns}
\usetikzlibrary{shadows.blur}
\usetikzlibrary{shapes}
\usepackage[super]{nth}
\usepackage{expl3}
\usepackage[version=4]{mhchem}
\usepackage{hpstatement}
\usepackage{rsphrase}
\usepackage{everysel}
\usepackage{ragged2e}
\usepackage{geometry}
\usepackage{fancyhdr}
\usepackage{cancel}
\geometry{top=1.0in,bottom=1.0in,left=1.0in,right=1.0in}
\newcommand{\subtitle}[1]{%
  \posttitle{%
    \par\end{center}
    \begin{center}\large#1\end{center}
    \vskip0.5em}%

}
\usepackage{hyperref}
\hypersetup{
colorlinks=true,
linkcolor=blue,
filecolor=magenta,      
urlcolor=blue,
citecolor=blue,
}

\urlstyle{same}


\title{Notes — Week 3}
\date{Period 3}
\author{Michael Brodskiy\\ \small Instructor: Mr. Bremer}

\begin{document}

\maketitle

\begin{enumerate}

  \item Three Broad Macroeconomic Goals:

    \begin{enumerate}

      \item Full Employment

      \item Price Stability

      \item Economic Growth

    \end{enumerate}

  \item Gross Domestic Product (GDP) — The dollar value of all final goods and services produced in a country in a year

    \begin{itemize}

      \item The most comprehensive and important single statistic regarding an economy

      \item Computed by the Department of Commerce

      \item Two ways to measure:

        \begin{itemize}

          \item Expenditure Model

          \item Income Model

        \end{itemize}

      \item What is excluded:

        \begin{itemize}

          \item Intermediate products

        \end{itemize}

      \item A very important measure indeed, but it does have some caveats:

        \begin{itemize}

          \item Composition of output

            \begin{itemize}

              \item Only reflects total production (e.g.\ what if GDP is due exclusively to weapons productions)

            \end{itemize}

          \item Exclusion of other activities

            \begin{itemize}

              \item What about when I mow my own lawn?

              \item What about the “underground” economy?

            \end{itemize}

          \item Difficult to adjust for changes in the quality of output

          \item Doesn't tell us how the benefits are distributed

        \end{itemize}

      \item Therefore, GDP is a necessary, but not sufficient, measure of the healthy of any economy

      \item $GDP = C + I + G + NX$

        \begin{itemize}

          \item $C$ — Goods and services purchased by consumers

          \item $I$ — Investment by businesses

          \item $G$ — Goods and services purchased by government

          \item $NX$ — The net amount of exports

        \end{itemize}

      \item Like movie gross receipts, GDP is subject to distortions over time because of inflation. Hence, comparisons over time are meaningless

      \item In its simplest form: $GDP = P\cdot Q$

    \end{itemize}

\end{enumerate}

\end{document}

