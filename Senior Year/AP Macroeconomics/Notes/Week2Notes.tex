%%%%%%%%%%%%%%%%%%%%%%%%%%%%%%%%%%%%%%%%%%%%%%%%%%%%%%%%%%%%%%%%%%%%%%%%%%%%%%%%%%%%%%%%%%%%%%%%%%%%%%%%%%%%%%%%%%%%%%%%%%%%%%%%%%%%%%%%%%%%%%%%%%%%%%%%%%%%%%%%%%%%%%%%%%%%%%%%%%%%%%%%%%%%
% Written By Michael Brodskiy
% Class: AP Macroeconomics
% Instructor: Mr. Bremer
%%%%%%%%%%%%%%%%%%%%%%%%%%%%%%%%%%%%%%%%%%%%%%%%%%%%%%%%%%%%%%%%%%%%%%%%%%%%%%%%%%%%%%%%%%%%%%%%%%%%%%%%%%%%%%%%%%%%%%%%%%%%%%%%%%%%%%%%%%%%%%%%%%%%%%%%%%%%%%%%%%%%%%%%%%%%%%%%%%%%%%%%%%%%

\documentclass[12pt]{article} 
\usepackage{alphalph}
\usepackage[utf8]{inputenc}
\usepackage[russian,english]{babel}
\usepackage{titling}
\usepackage{amsmath}
\usepackage{graphicx}
\usepackage{enumitem}
\usepackage{amssymb}
\usepackage{physics}
\usepackage{tikz}
\usepackage{mathdots}
\usepackage{yhmath}
\usepackage{cancel}
\usepackage{color}
\usepackage{siunitx}
\usepackage{array}
\usepackage{multirow}
\usepackage{gensymb}
\usepackage{tabularx}
\usepackage{booktabs}
\usepackage{soul}
\usepackage{multicol}
\usepackage{pgfplots}
\pgfplotsset{compat=newest}
\usetikzlibrary{fadings}
\usetikzlibrary{patterns}
\usetikzlibrary{shadows.blur}
\usetikzlibrary{shapes}
\usepackage[super]{nth}
\usepackage{expl3}
\usepackage[version=4]{mhchem}
\usepackage{hpstatement}
\usepackage{rsphrase}
\usepackage{everysel}
\usepackage{ragged2e}
\usepackage{geometry}
\usepackage{fancyhdr}
\usepackage{cancel}
\geometry{top=1.0in,bottom=1.0in,left=1.0in,right=1.0in}
\newcommand{\subtitle}[1]{%
  \posttitle{%
    \par\end{center}
    \begin{center}\large#1\end{center}
    \vskip0.5em}%

}
\usepackage{hyperref}
\hypersetup{
colorlinks=true,
linkcolor=blue,
filecolor=magenta,      
urlcolor=blue,
citecolor=blue,
}

\urlstyle{same}


\title{Notes — Week 2}
\date{Period 3}
\author{Michael Brodskiy\\ \small Instructor: Mr. Bremer}

\begin{document}

\maketitle

\begin{enumerate}


  \item Elasticity — If demand/supply is elastic, a small change in price has a big effect on quantity demanded (\textit{i.e.\ a slope close to zero}), and vice versa

  \item Determinants of Elasticity:

    \begin{itemize}

      \item Urgency of Need

      \item Availability of Substitutes

      \item Proportion of Income

    \end{itemize}

  \item Supply — The schedule of quantities that would be offered for sale at all of the possible prices that might prevail in the market

    \begin{multicols}{2}

      \begin{tabular}{|c|c|}
        \hline
        Price, ($P$, $\text{\$}$) & Quantity Supplied ($Q_s$)\\
        \hline
        25 & 0\\
        \hline
        50 & 8\\
        \hline
        75 & 11\\
        \hline
        100 & 18\\
        \hline
        200 & 21\\
        \hline
      \end{tabular}

    \begin{center}
      \begin{tikzpicture}
        \begin{axis}[xmin=0, ymin=0,title={Supply}, xlabel={Units}, ylabel={Price ($P$, $\text{\$}$)}, ymajorgrids=true, xmajorgrids=true, grid style=dashed]
          \addplot[blue] coordinates {(0,25) (8,50) (11, 75) (18, 100) (21, 200)};
          \node[label={180:{$Q_s$}},circle,fill,inner sep=2pt, color=blue] at (axis cs:10,66.3333) {};
        \end{axis}
      \end{tikzpicture}
    \end{center}

    \end{multicols}

  \item Law of Supply — The tendency of suppliers to offer more for sale at high prices than at low prices

    \begin{itemize}

      \item Quantity supplied can only be changed by price

    \end{itemize}

  \item Determinants of Supply:

    \begin{enumerate}

      \item Change in the cost of factors of production

      \item Change in technology — New technology often reduces producers' costs, leading to an increase in supply

      \item Change in profit opportunities producing other products — If producers expect to maker more selling something else the supply of what they currently produce decreases. If profit opportunities producing other things decrease, more sellers will begin producing this product, increasing supply

      \item Change in producers' price expectations

      \item Change in the amount of producers

    \end{enumerate}

  \item Supply vs. Demand

    \begin{itemize}

      \item Demand is downward-sloping

      \item Supply is upward-sloping

      \item On one graph, the two lines intersect. The intersection point is equilibrium (sometimes called market-clearing point)

      \item If the price is too high, or demand is too low, there is a surplus

      \item If the price is too low, or demand is too high, there is a shortage

    \end{itemize}

    \begin{center}
      \begin{tikzpicture}
        \begin{axis}[xmin=0, title={Supply vs. Demand}, xlabel={Units}, ylabel={Price ($P$, $\text{\$}$)}, ymin=0, xmax=150, ymax=150, ymajorgrids=true, xmajorgrids=true, grid style=dashed]
          \addplot [domain=10:140,red] {x};
          \addplot [domain=10:140,blue,fill=blue] {-x+150};
          \addplot [domain=0:75,dashed,green!50!black] {75};
          \addplot [dashed,green!50!black] coordinates {(75,0) (75,75)};
          \node[label={270:{$Q_s$}},circle,fill,inner sep=2pt, color=blue] at (axis cs:20,130) {};
          \node[label={90:{$Q_d$}},circle,fill,inner sep=2pt, color=red] at (axis cs:20,20) {};
          \node[label={0:{Equilibrium}},circle,fill,inner sep=2pt, color=green!50!black] at (axis cs:75,75) {};
          \legend{Demand, Supply}
        \end{axis}
      \end{tikzpicture}
    \end{center}

  \item Using the Supply and Demand model to predict market outcomes

    \begin{itemize}

      \item Does an event effect supply/demand?

      \item Is supply/demand increasing or decreasing

      \item Draw a new curve

      \item Use the new curve to determine impact on price and quantity

    \end{itemize}

\end{enumerate}

\end{document}

