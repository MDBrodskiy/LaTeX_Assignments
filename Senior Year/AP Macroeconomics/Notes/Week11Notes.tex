%%%%%%%%%%%%%%%%%%%%%%%%%%%%%%%%%%%%%%%%%%%%%%%%%%%%%%%%%%%%%%%%%%%%%%%%%%%%%%%%%%%%%%%%%%%%%%%%%%%%%%%%%%%%%%%%%%%%%%%%%%%%%%%%%%%%%%%%%%%%%%%%%%%%%%%%%%%%%%%%%%%%%%%%%%%%%%%%%%%%%%%%%%%%
% Written By Michael Brodskiy
% Class: AP Macroeconomics
% Instructor: Mr. Bremer
%%%%%%%%%%%%%%%%%%%%%%%%%%%%%%%%%%%%%%%%%%%%%%%%%%%%%%%%%%%%%%%%%%%%%%%%%%%%%%%%%%%%%%%%%%%%%%%%%%%%%%%%%%%%%%%%%%%%%%%%%%%%%%%%%%%%%%%%%%%%%%%%%%%%%%%%%%%%%%%%%%%%%%%%%%%%%%%%%%%%%%%%%%%%

\documentclass[12pt]{article} 
\usepackage{alphalph}
\usepackage[utf8]{inputenc}
\usepackage[russian,english]{babel}
\usepackage{titling}
\usepackage{amsmath}
\usepackage{graphicx}
\usepackage{enumitem}
\usepackage{amssymb}
\usepackage{physics}
\usepackage{tikz}
\usepackage{mathdots}
\usepackage{yhmath}
\usepackage{cancel}
\usepackage{color}
\definecolor{yellow}{rgb}{.95, .95, 0.0}
\usepackage{siunitx}
\usepackage{array}
\usepackage{multirow}
\usepackage{gensymb}
\usepackage{tabularx}
\usepackage{booktabs}
\usepackage{soul}
\usepackage{multicol}
\usepackage{pgfplots}
\pgfplotsset{compat=newest}
\usetikzlibrary{fadings}
\usetikzlibrary{patterns}
\usetikzlibrary{shadows.blur}
\usetikzlibrary{shapes}
\usepackage[super]{nth}
\usepackage{expl3}
\usepackage[version=4]{mhchem}
\usepackage{hpstatement}
\usepackage{rsphrase}
\usepackage{everysel}
\usepackage{ragged2e}
\usepackage{geometry}
\usepackage{fancyhdr}
\usepackage{cancel}
\geometry{top=1.0in,bottom=1.0in,left=1.0in,right=1.0in}
\newcommand{\subtitle}[1]{%
  \posttitle{%
    \par\end{center}
    \begin{center}\large#1\end{center}
    \vskip0.5em}%

}
\usepackage{hyperref}
\hypersetup{
colorlinks=true,
linkcolor=blue,
filecolor=magenta,      
urlcolor=blue,
citecolor=blue,
}

\urlstyle{same}


\title{Notes — Week 11}
\date{Period 3}
\author{Michael Brodskiy\\ \small Instructor: Mr. Bremer}

\begin{document}

\maketitle

\begin{enumerate}

  \item National Debt

    \begin{itemize}

      \item Debt — An accumulation of deficits

      \item Deficit — The amount gained or owed in a certain amount of time (usually an annual number) that represents the difference in income and spending

        \begin{itemize}
            
          \item Trade Deficit — When imports are greater than exports

          \item Budget Deficit — When spending exceeds income

        \end{itemize}

      \item National Debt — Too much spending, not enough income

      \item Entitlement — Something received from the government when one qualifies for it (\textit{i.e.} social security, food stamps, etc)

      \item “Crowding Out” — Government spending increases, which increases AD.\ This, in turn, increases the demand for loanable funds, which increases interest rate.\ The increase in interest rates decreases AD, creating a cycle of “crowding out” (investment keeps decreasing, government spending keeps increasing)

      \item Interest expenses crowd out other spending

      \item Composition of spending: investment v.\ consumption

      \item When investors become less confident, they are less willing to invest

    \end{itemize}

  \item Types/Causes of Inflation

    \begin{itemize}

      \item Demand-Pull Inflation — Demand, in tandem with limited supply, pulls prices higher

        \begin{itemize}

          \item Excessive monetary growth

        \end{itemize}

      \item Cost-Push Inflation — Results from shifts in the aggregate supply curve, usually supply shocks

    \end{itemize}

  \item The Quantity Theory of Money

    \begin{itemize}

      \item A theory that emphasizes the positive relationship between the price level and the money supply

      \item The equation of exchange: $M\cdot V = P\cdot Q$, where $M$ is the quantity of money, $V$ is the velocity, $P$ is the price level, and $Q$ is the real GDP

        \begin{itemize}

          \item Velocity is the measure of the rate of exchange of a dollar, and can be measured by: nominal GDP / money supply

        \end{itemize}

    \end{itemize}

\end{enumerate}

\end{document}

