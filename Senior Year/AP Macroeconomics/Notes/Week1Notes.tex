%%%%%%%%%%%%%%%%%%%%%%%%%%%%%%%%%%%%%%%%%%%%%%%%%%%%%%%%%%%%%%%%%%%%%%%%%%%%%%%%%%%%%%%%%%%%%%%%%%%%%%%%%%%%%%%%%%%%%%%%%%%%%%%%%%%%%%%%%%%%%%%%%%%%%%%%%%%%%%%%%%%%%%%%%%%%%%%%%%%%%%%%%%%%
% Written By Michael Brodskiy
% Class: AP Macroeconomics
% Instructor: Mr. Bremer
%%%%%%%%%%%%%%%%%%%%%%%%%%%%%%%%%%%%%%%%%%%%%%%%%%%%%%%%%%%%%%%%%%%%%%%%%%%%%%%%%%%%%%%%%%%%%%%%%%%%%%%%%%%%%%%%%%%%%%%%%%%%%%%%%%%%%%%%%%%%%%%%%%%%%%%%%%%%%%%%%%%%%%%%%%%%%%%%%%%%%%%%%%%%

\documentclass[12pt]{article} 
\usepackage{alphalph}
\usepackage[utf8]{inputenc}
\usepackage[russian,english]{babel}
\usepackage{titling}
\usepackage{amsmath}
\usepackage{graphicx}
\usepackage{enumitem}
\usepackage{amssymb}
\usepackage{physics}
\usepackage{tikz}
\usepackage{mathdots}
\usepackage{yhmath}
\usepackage{cancel}
\usepackage{color}
\usepackage{siunitx}
\usepackage{array}
\usepackage{multirow}
\usepackage{gensymb}
\usepackage{tabularx}
\usepackage{booktabs}
\usepackage{soul}
\usepackage{pgfplots}
\pgfplotsset{compat=newest}
\usetikzlibrary{fadings}
\usetikzlibrary{patterns}
\usetikzlibrary{shadows.blur}
\usetikzlibrary{shapes}
\usepackage[super]{nth}
\usepackage{expl3}
\usepackage[version=4]{mhchem}
\usepackage{hpstatement}
\usepackage{rsphrase}
\usepackage{everysel}
\usepackage{ragged2e}
\usepackage{geometry}
\usepackage{fancyhdr}
\usepackage{cancel}
\geometry{top=1.0in,bottom=1.0in,left=1.0in,right=1.0in}
\newcommand{\subtitle}[1]{%
  \posttitle{%
    \par\end{center}
    \begin{center}\large#1\end{center}
    \vskip0.5em}%

}
\usepackage{hyperref}
\hypersetup{
colorlinks=true,
linkcolor=blue,
filecolor=magenta,      
urlcolor=blue,
citecolor=blue,
}

\urlstyle{same}


\title{Notes — Week 1}
\date{Period 3}
\author{Michael Brodskiy\\ \small Instructor: Mr. Bremer}

\begin{document}

\maketitle

\begin{enumerate}

  \item There is no such thing as a f*** lunch

    \begin{itemize}

      \item Everything has a cost (\textit{i.e.\ opportunity cost})

    \end{itemize}

  \item People respond to incentives

    \begin{itemize}

      \item Incentives motivate people to do things (can be negative or positive)

    \end{itemize}

  \item The fundamental problem facing all societies is \underline{scarcity}

  \item Given the problem of scarcity, all societies must answer the three basic questions:

    \begin{itemize}

      \item What to produce

      \item How to produce

      \item For whom to produce

    \end{itemize}

  \item Command economies have the government decide what, how, and for whom to produce goods

  \item Factors of Production (Necessary for production to take place):

    \begin{itemize}

      \item Land (gifts of nature — \textit{i.e.\ navigable rivers, ocean access, etc.})

      \item Labor (\textit{i.e.\ human capital})

      \item Capital (tools — \textit{i.e.\ computers, bulldozers, etc.})

      \item \textit{Optional: Entrepreneurship}

    \end{itemize}

  \item Guide to Economic Reasoning

    \begin{itemize}

      \item People Choose — People choose the alternative that seems best to them because it involves the least cost and the greatest benefit (people economize)

      \item People's Choices Involve Costs — Opportunity cost is the second best alternative people give up in making a choice

        \begin{itemize}

          \item One way to view these costs is with a production possibilities curve

            \begin{center}

                \begin{tikzpicture}
                  \begin{axis}[xmin=0, xlabel={Good $x$}, ylabel={Good $y$}, ymin=0, ymajorgrids=true, xmajorgrids=true, grid style=dashed]
                      \addplot[color=blue]{(25 - x^2)^.5};
                      \node[label={180:{Efficient}},circle,fill,inner sep=2pt, color=blue!50] at (axis cs:3,4) {};
                      \node[label={180:{Feasible}},circle,fill,inner sep=2pt, color=green!50] at (axis cs:2,1) {};
                      \node[label={180:{Infeasible}},circle,fill,inner sep=2pt, color=red!50] at (axis cs:5,5) {};
                    \end{axis}
                \end{tikzpicture}
              
            \end{center}

        \end{itemize}

      \item People respond to incentives in predictable ways — Incentives are benefits or rewards that encourage people to act. When incentives change, people's choices change

      \item People create economic systems, and these systems influence incentives and people's choices — How people cooperate is governed by written and unwritten rules. As rules change, incentives change and choices change

      \item People gain when they trade voluntarily — People can produce more in less time by concentrating on what they do best. The surplus goods or services they produce can be traded for other valuable goods or services

    \end{itemize}

  \item Terminology

    \begin{itemize}

      \item When someone is better at something than someone else, they have an \underline{absolute advantage} (\textit{i.e.\ a plumber has an absolute advantage in plumbing})

      \item Comparative advantage is used to compare which choice is better through comparison of opportunity costs. Essentially, the best choice is wherever the opportunity cost is the lowest

      \item Advantage problems:

        \begin{itemize}

          \item Output or input

          \item Who has absolute advantage

          \item Calculate opportunity costs

          \item Who has comparative advantage

          \item Determine terms of trade

        \end{itemize}

    \end{itemize}

  \item Demand — Desire plus the ability and willingness to pay

  \item Demand Schedule — A table that shows the quantity demanded for a particular item at various prices

    \begin{center}

      \begin{tabular}{| c | c |}

        \hline
        Price ($P$) & Quantity ($Q$)\\
        \hline
        \$1.00 & 1,500\\
        \hline
        \$3.50 & 1,000\\
        \hline
        \$5.00 & 700\\
        \hline

      \end{tabular}

    \end{center}

  \item Demand Curve — A plotted demand schedule:

    \begin{center}
      \begin{tikzpicture}
        \begin{axis}[title={Demand}, xlabel={Units}, ylabel={Price ($P$, \$ )}, ymajorgrids=true, xmajorgrids=true, grid style=dashed]
          \addplot[color=red] coordinates {(1500,1) (1000,3.5) (700, 5)};
          \node[label={180:{$Q_d$}},circle,fill,inner sep=2pt, color=red] at (axis cs:1000,3.5) {};
        \end{axis}
      \end{tikzpicture}
    \end{center}

  \item Income effect — Change in quantity demanded because of change in consumer purchasing power when the price of a commodity changes. This has the effect of raising/lowering their relative income

  \item Substitution effect — Change in quantity demanded because of a change in the relative price of a product

  \item Movement along a demand curve is known as change in quantity demanded

  \item Rightward movement of a line is an increase in demand (and vice versa)

  \item A point shows quantity demanded, a curve shows demand

    \begin{itemize}

      \item If only price changes, you will have a change in quantity demanded

      \item If market conditions change (necessitating a new demand schedule and a new curve), you will have a change in demand

    \end{itemize}

    \newpage

  \item Demand changes are based on several determinants of demand:

    \begin{enumerate}

      \item Change in consumer income

      \item Change in consumer tastes

      \item Change in the price of a substitute good

      \item Change in the price of a complementary good

      \item Change in consumers' price expectations

      \item Change in the number of consumers in the market

    \end{enumerate}
    
    \begin{center}
      \begin{tikzpicture}
        \begin{axis}[xmin=0, title={Supply vs. Demand}, xlabel={Units}, ylabel={Price ($P$, \$)}, ymin=0, xmax=150, ymax=150, ymajorgrids=true, xmajorgrids=true, grid style=dashed]
          \addplot [domain=10:140,red] {x};
          \addplot [domain=10:140,blue,fill=blue] {-x+150};
          \node[label={180:{$Q_s$}},circle,fill,inner sep=2pt, color=blue] at (axis cs:20,130) {};
          \node[label={180:{$Q_d$}},circle,fill,inner sep=2pt, color=red] at (axis cs:20,20) {};
          \node[label={180:{Equilibrium}},circle,fill,inner sep=2pt, color=green!50!black] at (axis cs:75,75) {};
          \legend{Demand, Supply}
        \end{axis}
      \end{tikzpicture}
    \end{center}

\end{enumerate}

\end{document}

