%%%%%%%%%%%%%%%%%%%%%%%%%%%%%%%%%%%%%%%%%%%%%%%%%%%%%%%%%%%%%%%%%%%%%%%%%%%%%%%%%%%%%%%%%%%%%%%%%%%%%%%%%%%%%%%%%%%%%%%%%%%%%%%%%%%%%%%%%%%%%%%%%%%%%%%%%%%%%%%%%%%%%%%%%%%%%%%%%%%%%%%%%%%%
% Written By Michael Brodskiy
% Class: AP Environmental Science
% Instructor: Mrs. Stansbury
%%%%%%%%%%%%%%%%%%%%%%%%%%%%%%%%%%%%%%%%%%%%%%%%%%%%%%%%%%%%%%%%%%%%%%%%%%%%%%%%%%%%%%%%%%%%%%%%%%%%%%%%%%%%%%%%%%%%%%%%%%%%%%%%%%%%%%%%%%%%%%%%%%%%%%%%%%%%%%%%%%%%%%%%%%%%%%%%%%%%%%%%%%%%

\documentclass[12pt]{article} 
\usepackage{alphalph}
\usepackage[utf8]{inputenc}
\usepackage[russian,english]{babel}
\usepackage{titling}
\usepackage{amsmath}
\usepackage{graphicx}
\usepackage{enumitem}
\usepackage{amssymb}
\usepackage{physics}
\usepackage{tikz}
\usepackage{mathdots}
\usepackage{yhmath}
\usepackage{cancel}
\usepackage{color}
\usepackage{siunitx}
\usepackage{array}
\usepackage{multirow}
\usepackage{gensymb}
\usepackage{tabularx}
\usepackage{booktabs}
\usepackage{soul}
\usetikzlibrary{fadings}
\usetikzlibrary{patterns}
\usetikzlibrary{shadows.blur}
\usetikzlibrary{shapes}
\usepackage[super]{nth}
\usepackage{expl3}
\usepackage[version=4]{mhchem}
\usepackage{hpstatement}
\usepackage{rsphrase}
\usepackage{everysel}
\usepackage{ragged2e}
\usepackage{geometry}
\usepackage{fancyhdr}
\usepackage{cancel}
\geometry{top=1.0in,bottom=1.0in,left=1.0in,right=1.0in}
\newcommand{\subtitle}[1]{%
  \posttitle{%
    \par\end{center}
    \begin{center}\large#1\end{center}
    \vskip0.5em}%

}
\usepackage{hyperref}
\hypersetup{
colorlinks=true,
linkcolor=blue,
filecolor=magenta,      
urlcolor=blue,
citecolor=blue,
}

\urlstyle{same}


\title{Why Did the Bald Eagle Almost Become Extinct?}
\date{\today $-$ Period 1}
\author{Michael Brodskiy\\ \small Instructor: Mrs. Stansbury}

\begin{document}

\maketitle

\begin{enumerate}

  \item Evidence One — Bald Eagles rely on fish in bodies of water as a food source

  \item Evidence Two — This graph indicates that Bald Eagle nesting pairs in the lower 48 states have increased from 1963 $-$ 2014

  \item Evidence Three — In \textit{Silent Spring}, Rachel Carson states that, in the 1950s and late 1940s, there were fewer Bald Eagles spotted than in the 1930s and early 1940s. She describes that, since 1947, although there had been evidence of egg laying, no young eagles were produced

  \item Evidence Four — This timeline corresponds the usage and banning of DDT to the amount of wild births of Bald Eagles from the period 1925 to 1996

  \item Evidence Five — This chart demonstrates the hydrologic cycle, where a body of water is evaporate to be then precipitated into run off. Additionally, infiltration from plant roots and subsurface flow occur

  \item Evidence Six — This diagram relates DDT to its concentrations in various animals across a food web. It begins with concentration in water, then travels through producers into zooplankton, then small to large fish, and, finally, into Bald Eagles, where concentrations are 10,000,000 times that of the concentration in water

  \item Evidence Seven — This is an image of a Bald Eagle egg, which, instead of being hard, as usual, is soft and squishy, almost like a toy

\end{enumerate}

\newpage

\begin{center}
  \underline{Conclusion}
\end{center}

\begin{justify}
It seems that the Bald Eagle population has declined due to a hindered ability to reproduce, though not in the ability to lay eggs, caused by DDT run off into bodies of water, where fish have absorbed it through zooplankton. Firstly, the most important clue is Evidence 6, a diagram which shows that concentrations of DDT in Bald Eagles are 10,000,000 times that of the concentration in water, as the chemical travels through the producers and zooplankton, into the fish population, and, finally into the Bald Eagle. Evidence 1 complements this, as it shows an image of an eagle fishing, signifying that the population was reliant on fish. It can most safely be assumed that DDT was run off into water from Evidence 5, which demonstrates the ability for infiltration into groundwater to occur through plants, which, subsequently, leaks into bodies of water through subsurface flow. Evidence 3 and 4, taken in tandem, signify that, through the publication of books, such as Silent Spring by Rachel Carson, it was recognized that Bald Eagle populations were declining, and, consequently, DDT usage was banned in 1972. Evidence 2 signifies that, even before DDT was banned, starting from 1963, the amount of nesting pairs of eagles has increased. Though this may be indicative of an increasing population, taking Evidence 7, or a soft, almost toy-like egg, into consideration indicates that, though Bald Eagles did lay eggs, these eggs were ineffective; therefore, it is most plausible to conclude that DDT was the culprit that, through fish, hindered the reproductive capabilities of Bald Eagles.
\end{justify}

\end{document}

