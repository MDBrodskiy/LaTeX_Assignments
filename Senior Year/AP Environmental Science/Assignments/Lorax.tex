%%%%%%%%%%%%%%%%%%%%%%%%%%%%%%%%%%%%%%%%%%%%%%%%%%%%%%%%%%%%%%%%%%%%%%%%%%%%%%%%%%%%%%%%%%%%%%%%%%%%%%%%%%%%%%%%%%%%%%%%%%%%%%%%%%%%%%%%%%%%%%%%%%%%%%%%%%%%%%%%%%%%%%%%%%%%%%%%%%%%%%%%%%%%
% Written By Michael Brodskiy
% Class: AP Environmental Science
% Instructor: Mrs. Stansbury
%%%%%%%%%%%%%%%%%%%%%%%%%%%%%%%%%%%%%%%%%%%%%%%%%%%%%%%%%%%%%%%%%%%%%%%%%%%%%%%%%%%%%%%%%%%%%%%%%%%%%%%%%%%%%%%%%%%%%%%%%%%%%%%%%%%%%%%%%%%%%%%%%%%%%%%%%%%%%%%%%%%%%%%%%%%%%%%%%%%%%%%%%%%%

\documentclass[12pt]{article} 
\usepackage{alphalph}
\usepackage[utf8]{inputenc}
\usepackage[russian,english]{babel}
\usepackage{titling}
\usepackage{amsmath}
\usepackage{graphicx}
\usepackage{enumitem}
\usepackage{amssymb}
\usepackage{physics}
\usepackage{tikz}
\usepackage{mathdots}
\usepackage{yhmath}
\usepackage{cancel}
\usepackage{color}
\usepackage{siunitx}
\usepackage{array}
\usepackage{multirow}
\usepackage{gensymb}
\usepackage{tabularx}
\usepackage{booktabs}
\usepackage{soul}
\usetikzlibrary{fadings}
\usetikzlibrary{patterns}
\usetikzlibrary{shadows.blur}
\usetikzlibrary{shapes}
\usepackage[super]{nth}
\usepackage{expl3}
\usepackage[version=4]{mhchem}
\usepackage{hpstatement}
\usepackage{rsphrase}
\usepackage{everysel}
\usepackage{ragged2e}
\usepackage{geometry}
\usepackage{fancyhdr}
\usepackage{cancel}
\geometry{top=1.0in,bottom=1.0in,left=1.0in,right=1.0in}
\newcommand{\subtitle}[1]{%
  \posttitle{%
    \par\end{center}
    \begin{center}\large#1\end{center}
    \vskip0.5em}%

}
\usepackage{hyperref}
\hypersetup{
colorlinks=true,
linkcolor=blue,
filecolor=magenta,      
urlcolor=blue,
citecolor=blue,
}

\urlstyle{same}


\title{The Lorax}
\date{\today $-$ Period 1}
\author{Michael Brodskiy\\ \small Instructor: Mrs. Stansbury}

% Mathematical Operations:

% Sum: $$\sum_{n=a}^{b} f(x) $$
% Integral: $$\int_{lower}^{upper} f(x) dx$$
% Limit: $$\lim_{x\to\infty} f(x)$$

\begin{document}

\maketitle

\begin{enumerate}

  \item What is the Lorax? What is his role in the book?

    \begin{justify}

      In the story, the Lorax is a spawn, or rather, an extension of the trees. Because “the trees have no tongues” the Lorax is tasked with protecting and conserving them. In response to a disturbance (starting with the chopping down of one tree), the Lorax spawns from a tree stump. As time goes by, the Lorax also speaks of conserving the wildlife in the area, which was suffering immensely. In this manner, the Lorax plays the role of an environmental conservationist.

    \end{justify}

  \item Though the Once-ler polluted the area where he lived, environmentalists have now concluded that the new concern for our planet should be one of global environmentalism, because we are all interconnected and events that occur on the other side of the globe have profound and immediate effects on our lives. List three things that could have global effects in the production of the thneed.

    \begin{justify}

      Evidently, the most profound impact on thneed production would occur as a result of a shortage of thneed supplies (as portrayed in the movie-film). Lack of inputs means that there are not enough materials, which would halt the production of thneeds. Another possibility is the active protest of thneed employees. A parallel to the Luddites during the Industrial Revolution, it is possible (though unlikely, as the workers were, at least in the beginning, the Once-ler's relatives) that workers would sabotage machinery as a form of environmental protest. In this manner, thneed production would be slowed. A third and final possibility is that the Once-ler could actually come to his senses and recognize the effects of his economic empire. In such a case, thneed production would most likely be slowed greatly to account for the time required for a new generation of seeds to germinate and grow. As such, three possibilities are: shortage of supplies (the case in the film), protest of workers, or recognition of the perspiring events.
      
    \end{justify}

  \item Look at the list below and discuss the implications of how any three of the items can cause a worsened environmental effect.  Highlight the 3 you choose and write about them below.

    \begin{center}

      Deforestation; Energy consumption; Food shortages; Global warming; \hl{Human population explosion}; Loss of biodiversity; Political unrest; Soil erosion; \hl{Waste disposal}; \hl{Water pollution} 

    \end{center}

    \begin{enumerate}

      \item Human Population Explosion — The root cause of environmental degradation may be boiled down to a simple reason: human resolve. Because humans always strive for comfortable living conditions, overconsumption leads to a strain on the Earth's natural resources, which, in turn, only leads to worse ecological effects. The more humans require resources, the quicker resources will be used, and, as a result, the less resources will be available.

      \item Waste Disposal — As portrayed in the film, waste disposal is a big problem, especially in industrial areas. Incorrect disposal of materials has the possibility of poisoning the air, water, and soil, which, in turn, leads to worse implications, such as soil erosion, diminishing biodiversity, and, ultimately global warming.

      \item Water Pollution — Also portrayed in the film, water pollution effects species much like waste disposal. Garbage floating in the ocean strains species, as fish and various other creatures have been known to swallow or get stuck in plastic, ultimately resulting in their death.

    \end{enumerate}

  \item In the last part of the book, the Lorax uses the word “unless”. What does that mean and how can you, as an average citizen, make a difference in the environment?

    \begin{justify}

      The “unless” may be interpreted in several ways. Most directly, this means “unless there is a human who cares enough,” as the Once-ler now understands that he did not care enough. His contempocentrist world-view resulted in virtual annihilation of the trees necessary to his production. Any average citizen can make a difference in the environment by beginning with small steps, and, eventually, transitioning to something bigger. For example, I transitioned to a completely paperless workflow, as all of my work is done on my computer.

    \end{justify}

\end{enumerate}


\end{document}

