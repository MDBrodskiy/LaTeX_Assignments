%%%%%%%%%%%%%%%%%%%%%%%%%%%%%%%%%%%%%%%%%%%%%%%%%%%%%%%%%%%%%%%%%%%%%%%%%%%%%%%%%%%%%%%%%%%%%%%%%%%%%%%%%%%%%%%%%%%%%%%%%%%%%%%%%%%%%%%%%%%%%%%%%%%%%%%%%%%%%%%%%%%%%%%%%%%%%%%%%%%%%%%%%%%%
% Written By Michael Brodskiy
% Class: AP Environmental Science
% Instructor: Mrs. Stansbury
%%%%%%%%%%%%%%%%%%%%%%%%%%%%%%%%%%%%%%%%%%%%%%%%%%%%%%%%%%%%%%%%%%%%%%%%%%%%%%%%%%%%%%%%%%%%%%%%%%%%%%%%%%%%%%%%%%%%%%%%%%%%%%%%%%%%%%%%%%%%%%%%%%%%%%%%%%%%%%%%%%%%%%%%%%%%%%%%%%%%%%%%%%%%

\documentclass[12pt]{article} 
\usepackage{alphalph}
\usepackage[utf8]{inputenc}
\usepackage[russian,english]{babel}
\usepackage{titling}
\usepackage{amsmath}
\usepackage{graphicx}
\usepackage{enumitem}
\usepackage{amssymb}
\usepackage{physics}
\usepackage{tikz}
\usepackage{mathdots}
\usepackage{yhmath}
\usepackage{cancel}
\usepackage{color}
\usepackage{siunitx}
\usepackage{array}
\usepackage{multirow}
\usepackage{gensymb}
\usepackage{tabularx}
\usepackage{booktabs}
\usepackage{soul}
\usetikzlibrary{fadings}
\usetikzlibrary{patterns}
\usetikzlibrary{shadows.blur}
\usetikzlibrary{shapes}
\usepackage[super]{nth}
\usepackage{expl3}
\usepackage[version=4]{mhchem}
\usepackage{hpstatement}
\usepackage{rsphrase}
\usepackage{everysel}
\usepackage{ragged2e}
\usepackage{geometry}
\usepackage{fancyhdr}
\usepackage{cancel}
\geometry{top=1.0in,bottom=1.0in,left=1.0in,right=1.0in}
\newcommand{\subtitle}[1]{%
  \posttitle{%
    \par\end{center}
    \begin{center}\large#1\end{center}
    \vskip0.5em}%

}
\usepackage{hyperref}
\hypersetup{
colorlinks=true,
linkcolor=blue,
filecolor=magenta,      
urlcolor=blue,
citecolor=blue,
}

\urlstyle{same}


\title{Systems}
\date{\today $-$ Period 1}
\author{Michael Brodskiy\\ \small Instructor: Mrs. Stansbury}

\begin{document}

\maketitle

\begin{enumerate}

  \item Easter Island

    \begin{itemize}

      \item A closed system

      \item Once a paradise

      \item Tragedy of the commons

      \item Time delay

      \item Exponential devastation

      \item Population crash

      \item Used as a model of the current state of the planet

    \end{itemize}

  \item Positive Feedback Loop

    \begin{itemize}

      \item \textbf{Runaway Cycle} — A change in the system (input) causes the output to increase, which causes more input

    \end{itemize}

  \item Negative Feedback Loop

    \begin{itemize}

      \item A change in input creates an output which causes the input to decrease

      \item Homeostasis (balance)

      \item Ex. As pollution becomes less of a problem, it will bother fewer people, which means regulations on pollution will become less stringent

    \end{itemize}

  \item Time Delays

    \begin{itemize}

      \item Typical for environmental systems

      \item Do not see/feel the consequences until it is too late

      \item Ex. Smoking — Years of smoking may lead to cancer, which makes it too late to quit

    \end{itemize}

  \item Synergistic Interactions

    \begin{itemize}

      \item When two processes create a stronger effect together than the sum of their individual parts

      \item Ex. Smog — Heat and UV radiation from the sun combine with car emissions and create a toxic substance worse than either alone

    \end{itemize}

  \item Organic Compounds

    \begin{itemize}

      \item Contain Carbon

        \begin{itemize}

          \item Exception: \ce{CO2}

        \end{itemize}

      \item Hydrocarbons — Fossil fuels (methane)

      \item Chlorinated Hydrocarbons — DDT

      \item Chlorofluorocarbons (CFCs) — Freon

    \end{itemize}

  \item \ce{pH}

    \begin{itemize}

      \item Percent Hydronium

      \item Logarithmic scale from 0 (strong acid, more \ce{H+} ions) to 14 (Strong base, more \ce{OH-} ions), with 7 being neutral

    \end{itemize}

  \item Quality Matter

    \begin{itemize}

      \item High quality matter is easily used by man in terms of creating a product

      \item Low quality matter is difficult to obtain or to convert into usable objects

    \end{itemize}

  \item Energy

    \begin{itemize}

      \item The ability to do work and transfer heat

      \item Kinetic Energy — Motion

      \item Potential Energy — Stored energy

    \end{itemize}

  \item Law of Conservation of Matter

    \begin{itemize}

      \item Matter can not be created nor destroyed

        \begin{itemize}

          \item Matter is changed either physically or chemically, but still present

        \end{itemize}

      \item We will never run out of matter, only matter in easily used forms

    \end{itemize}

  \item Matter Breaks Down

    \begin{itemize}

      \item Concentration — How much is there?

      \item Persistence — How long will it last?

      \item Degradable — Via physical, chemical, or biological

        \begin{itemize}

          \item Slowly degradable — (plastics, DDT (takes decades))

          \item Nondegradable — Elements like lead and mercury never break down

        \end{itemize}

    \end{itemize}

  \item Nuclear Changes

    \begin{itemize}

      \item Radioactive change (decay) is another possible change to matter

      \item Natural radioactive decay

        \begin{itemize}

          \item Nuclear Fission — Splitting of atoms

            \begin{itemize}

              \item Typically a large mass isotope (\ce{U}-235)

              \item When neutrons are shot at nucleus, the nucleus splits, releasing energy, and more neutrons

              \item Creates chain reaction

              \item Used in power generation (nuclear power plants)

            \end{itemize}

          \item Nuclear Fusion

            \begin{itemize}

              \item Two light-speed nuclei are slammed together at high speed

              \item Fusing produces new nucleus and release energy

              \item Typically isotopes of hydrogen are used

              \item Occurs inside a hydrogen bomb

            \end{itemize}

        \end{itemize}
        
    \end{itemize}

  \item First Law of Thermodynamics

    \begin{itemize}

      \item Also called the “law of conservation of energy”

      \item Energy is neither created nor destroyed but may be converted from one form to another

      \item Energy in = Energy out

    \end{itemize}

  \item Second Law of Thermodynamics

    \begin{itemize}

      \item When energy changes form, some energy is degraded into lower quality energy


      \item In other words, heat is lost to the surrounding environment in all energy conversions or transfers (entropy)

    \end{itemize}

\end{enumerate}

\end{document}

