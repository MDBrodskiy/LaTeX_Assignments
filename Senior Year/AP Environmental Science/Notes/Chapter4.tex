%%%%%%%%%%%%%%%%%%%%%%%%%%%%%%%%%%%%%%%%%%%%%%%%%%%%%%%%%%%%%%%%%%%%%%%%%%%%%%%%%%%%%%%%%%%%%%%%%%%%%%%%%%%%%%%%%%%%%%%%%%%%%%%%%%%%%%%%%%%%%%%%%%%%%%%%%%%%%%%%%%%%%%%%%%%%%%%%%%%%%%%%%%%%
% Written By Michael Brodskiy
% Class: AP Environmental Science
% Instructor: Mrs. Stansbury
%%%%%%%%%%%%%%%%%%%%%%%%%%%%%%%%%%%%%%%%%%%%%%%%%%%%%%%%%%%%%%%%%%%%%%%%%%%%%%%%%%%%%%%%%%%%%%%%%%%%%%%%%%%%%%%%%%%%%%%%%%%%%%%%%%%%%%%%%%%%%%%%%%%%%%%%%%%%%%%%%%%%%%%%%%%%%%%%%%%%%%%%%%%%

\documentclass[12pt]{article} 
\usepackage{alphalph}
\usepackage[utf8]{inputenc}
\usepackage[russian,english]{babel}
\usepackage{titling}
\usepackage{amsmath}
\usepackage{graphicx}
\usepackage{enumitem}
\usepackage{amssymb}
\usepackage{physics}
\usepackage{tikz}
\usepackage{mathdots}
\usepackage{yhmath}
\usepackage{cancel}
\usepackage{color}
\usepackage{siunitx}
\usepackage{array}
\usepackage{multirow}
\usepackage{gensymb}
\usepackage{tabularx}
\usepackage{booktabs}
\usepackage{soul}
\usetikzlibrary{fadings}
\usetikzlibrary{patterns}
\usetikzlibrary{shadows.blur}
\usetikzlibrary{shapes}
\usepackage[super]{nth}
\usepackage{expl3}
\usepackage[version=4]{mhchem}
\usepackage{hpstatement}
\usepackage{rsphrase}
\usepackage{everysel}
\usepackage{ragged2e}
\usepackage{geometry}
\usepackage{fancyhdr}
\usepackage{cancel}
\geometry{top=1.0in,bottom=1.0in,left=1.0in,right=1.0in}
\newcommand{\subtitle}[1]{%
  \posttitle{%
    \par\end{center}
    \begin{center}\large#1\end{center}
    \vskip0.5em}%

}
\usepackage{hyperref}
\hypersetup{
colorlinks=true,
linkcolor=blue,
filecolor=magenta,      
urlcolor=blue,
citecolor=blue,
}

\urlstyle{same}


\title{Biodiversity and Evolution}
\date{\today $-$ Period 1}
\author{Michael Brodskiy\\ \small Instructor: Mrs. Stansbury}

\begin{document}

\maketitle

\begin{enumerate}

  \item Keystone Species — A species whose role has an effect on the types and abundance of other species in an ecosystem. Best examples are top predators (Sharks) and pollinators (Bees)

  \item Species — A set of individuals who can mate and produce fertile offspring

  \item Species Diversity — A variety of species in a given area

  \item Ecosystem Diversity — A variety of ecosystems in a given area

  \item Genetic Diversity — Diversity within the same species

  \item Biome — Regions with distinct climates and species

  \item Biological Evolution through Natural Selection explains how life changes over time

    \begin{enumerate}

      \item Fossils — Physical evidence of ancient organisms that reveal what their external structures looked like

        \begin{itemize}

          \item Fossil Record — Entire body of fossil evidence. Only have fossils of 1\% of all species that lived on Earth

        \end{itemize}

      \item Biological Evolution — How life on Earth changes over time due to changes in the genetic characteristics of populations (Remember Darwin: \textit{Origin of Species})

      \item Natural Selection — Individuals with certain traits are more likely to survive and reproduce under a certain set of environmental conditions (survival of the fittest)
        
    \end{enumerate}

  \item Populations evolve by becoming genetically different

  \item Genetic Variations

    \begin{enumerate}

      \item Occur through mutations in reproductive cells

      \item Mutations — Random changes in DNA

    \end{enumerate}

  \item Natural Selection — Acts on individuals

  \item Myths about Evolution:

    \begin{enumerate}

      \item “Survival of the fittest” is not “survival of the strongest”

      \item Organisms do not develop traits out of need or want

      \item No grand plan of nature for perfect adaptation

    \end{enumerate}

  \item Geologic Processes Affect Natural Selection

    \begin{enumerate}

      \item Tectonic plates affect evolution and the location of life on Earth

        \begin{itemize}

          \item Locations of continents and oceans have shifted

          \item Species physically move, or adapt, or form new species through natural selection

        \end{itemize}

      \item Earthquakes

      \item Volcanic Eruptions

    \end{enumerate}

  \item Earth is just right for life to thrive

    \begin{enumerate}

      \item Temperature range: supports life

      \item Orbit size: moderate temperature

      \item Liquid water: necessary for life

      \item Rotation speed: sun doesn't overheat surface

      \item Size: gravity keeps atmosphere

    \end{enumerate}

  \item Speciation — One species splits into two or more species

  \item Geographic Isolation — Happens first; physical isolation of populations for a long period

  \item Reproductive Isolation — Mutations and natural selection in geographically isolated populations lead to inability to produce viable offspring when members of two different populations mate

  \item Geographic isolation can lead to reproductive isolation

  \item Artificial Selection — Use selective breeding/crossbreeding

  \item Genetic Engineering — Gene splicing (controversial)

    \begin{itemize}

      \item Ethics

      \item Morals

      \item Privacy issues

      \item Harmful effects

    \end{itemize}

  \item Extinction

    \begin{itemize}

      \item Biological Extinction — Gone from the entire planet

      \item Local Extinction — Gone from a certain region

    \end{itemize}

  \item Endemic Species

    \begin{itemize}

      \item Found only in one area

      \item Particularly vulnerable

      \item Ex. Island fox on the Channel Islands

    \end{itemize}

  \item Background Extinction (normal extinction rate) — Natural extinction that occurs over time


  \item Mass Extinction — 5 over 500 million years

  \item Species Diversity — Measure of the diversity in a community

  \item Species Richness — The number of different species in a given area

  \item Species Evenness — Refers to how close in number each species is in an environment (ex. 50 of species A and 50 of species B, or 50 of species A and 1 of species B)

  \item Diversity varies with geographical location

  \item Most species-rich communities:

    \begin{itemize}

      \item Tropical rain forests

      \item Large tropical lakes

      \item Coral reefs

      \item Ocean bottom zone

    \end{itemize}

  \item Species equilibrium model, theory of island biogeography

    \begin{itemize}

      \item Number of species found on an undisturbed island is determined solely by the number of species immigrating to the island and by extinction rates

      \item Species may follow evolutionary routes that are different than species on land masses that are not isolated

      \item Island size and distance from the mainland need to be considered

    \end{itemize}

  \item Ecological Niche

    \begin{itemize}

      \item Pattern of living — Everything that affects survival and reproduction

      \item Water, space, sunlight, food, temperatures

    \end{itemize}

  \item Generalist Species

    \begin{itemize}

      \item Broad Niche — Wide range of tolerance

    \end{itemize}

  \item Specialist Species

    \begin{itemize}

      \item Narrow Niche — Narrow range of tolerance

    \end{itemize}

  \item Native Species — An indigenous of endemic species in a specific region or area

  \item Non-native Species — An introduced species that has been brought to the area by humans

  \item Indicator Species — A species that provides early warnings of damage to community or ecosystem (canary in the coal mine); can monitor environmental quality (Trout, Birds, Butterflies, Frogs)

  \item Keystone Species — Species whose role has a large effect on the types or abundance of other species

  \item Foundation Species — A species that forms a foundation for the community; create or enhance their habitats, which benefit others (Elephants, Beavers, Alligators)

\end{enumerate}

\end{document}

