%%%%%%%%%%%%%%%%%%%%%%%%%%%%%%%%%%%%%%%%%%%%%%%%%%%%%%%%%%%%%%%%%%%%%%%%%%%%%%%%%%%%%%%%%%%%%%%%%%%%%%%%%%%%%%%%%%%%%%%%%%%%%%%%%%%%%%%%%%%%%%%%%%%%%%%%%%%%%%%%%%%%%%%%%%%%%%%%%%%%%%%%%%%%
% Written By Michael Brodskiy
% Class: AP Environmental Science
% Instructor: Mrs. Stansbury
%%%%%%%%%%%%%%%%%%%%%%%%%%%%%%%%%%%%%%%%%%%%%%%%%%%%%%%%%%%%%%%%%%%%%%%%%%%%%%%%%%%%%%%%%%%%%%%%%%%%%%%%%%%%%%%%%%%%%%%%%%%%%%%%%%%%%%%%%%%%%%%%%%%%%%%%%%%%%%%%%%%%%%%%%%%%%%%%%%%%%%%%%%%%

\documentclass[12pt]{article} 
\usepackage{alphalph}
\usepackage[utf8]{inputenc}
\usepackage[russian,english]{babel}
\usepackage{titling}
\usepackage{amsmath}
\usepackage{graphicx}
\usepackage{enumitem}
\usepackage{amssymb}
\usepackage{physics}
\usepackage{tikz}
\usepackage{mathdots}
\usepackage{yhmath}
\usepackage{cancel}
\usepackage{color}
\usepackage{siunitx}
\usepackage{array}
\usepackage{multirow}
\usepackage{gensymb}
\usepackage{tabularx}
\usepackage{booktabs}
\usepackage{soul}
\usetikzlibrary{fadings}
\usetikzlibrary{patterns}
\usetikzlibrary{shadows.blur}
\usetikzlibrary{shapes}
\usepackage[super]{nth}
\usepackage{expl3}
\usepackage[version=4]{mhchem}
\usepackage{hpstatement}
\usepackage{rsphrase}
\usepackage{everysel}
\usepackage{ragged2e}
\usepackage{geometry}
\usepackage{fancyhdr}
\usepackage{cancel}
\geometry{top=1.0in,bottom=1.0in,left=1.0in,right=1.0in}
\newcommand{\subtitle}[1]{%
  \posttitle{%
    \par\end{center}
    \begin{center}\large#1\end{center}
    \vskip0.5em}%

}
\usepackage{hyperref}
\hypersetup{
colorlinks=true,
linkcolor=blue,
filecolor=magenta,      
urlcolor=blue,
citecolor=blue,
}

\urlstyle{same}


\title{Sustaining Aquatic Biodiversity}
\date{\today $-$ Period 1}
\author{Michael Brodskiy\\ \small Instructor: Mrs. Stansbury}

\begin{document}

\maketitle

\begin{enumerate}
    
  \item We have explored about 5\% of the oceans

  \item Biodiversity is higher near the coast than in the open sea and in the bottom region of the ocean more so than the surface region

  \item Marine

    \begin{itemize}

      \item Coral reefs

      \item Mangrove forests

      \item Seagrass beds

      \item Ocean acidification

    \end{itemize}

  \item Freshwater

    \begin{itemize}

      \item Dams

    \end{itemize}

  \item Invasive species

    \begin{itemize}

      \item Threaten native species

      \item Disrupt and degrade whole ecosystems

      \item Blamed for about two-thirds of all fish extinctions since 1900

      \item Ballast water from ships

      \item Accidentally or deliberately introduced

      \item Ex. Lionfish in the Atlantic

    \end{itemize}

  \item 80\% of all humans live along coasts

  \item Nitrates and phosphates, mainly from fertilizers, enter water

    \begin{itemize}

      \item Leads to eutrophication

      \item Increase in dissolved oxygen

      \item Kills off fish

      \item Increase in algae blooms

    \end{itemize}

  \item Toxic pollutants from industrial and urban areas

  \item Plastics

    \begin{itemize}

      \item Ocean garbage

    \end{itemize}

  \item Climate Change

    \begin{itemize}

      \item Sea levels will rise and aquatic biodiversity is threatened

        \begin{itemize}

          \item Coral reefs

          \item Swamp some low-lying islands

          \item Drown many highly productive coastal wetlands

          \item Warmer ocean water stresses phytoplankton

        \end{itemize}

      \item Coral bleaching

    \end{itemize}

  \item Fishery — Concentration of a particular wild aquatic species suitable for commercial harvesting in a specific area

  \item Fishing is the key factor in the depletion of up to 80\% of the population of some wild fish species in only 10-15 years

  \item Fishprint — Area of ocean needed to sustain the fish consumption of an average person, nation, or the world

  \item Overfishing leads to commercial extinction

    \begin{itemize}

      \item Commercially valuable fish become scarce

      \item Bluefin tuna ranching

    \end{itemize}

  \item Some marine mammals are also threatened due to overfishing

  \item Biological extinction

    \begin{itemize}

      \item Overfishing, water pollution, wetlands destruction, excessive removal of water from lakes and rivers

      \item 34\% of marine species are threatened

      \item 71\% of freshwater species are threatened

    \end{itemize}

  \item We can help to sustain marine biodiversity by:

    \begin{itemize}

      \item Using laws and economic incentives to protect species

      \item Setting aside marine reserves to protect ecosystems and ecosystem services
        
      \item Using community-based integrated coastal management

    \end{itemize}

  \item Marine Reserves

    \begin{itemize}

      \item Closed to:

        \begin{itemize}

          \item Commercial fishing

          \item Dredging

          \item Mining and waste disposal

        \end{itemize}

      \item Core zone

        \begin{itemize}

          \item No human activity allowed

        \end{itemize}

      \item Less harmful activities allowed

    \end{itemize}

  \item Fully protected marine reserves work fast

    \begin{itemize}

      \item Fish populations double

      \item Fish size grows

      \item Reproduction triples

      \item Species diversity increase by almost one-fourth

    \end{itemize}

  \item Cover less than 1\% of world's oceans

    \begin{itemize}

      \item Marine scientists want 30-50\%
        
    \end{itemize}

  \item Sustaining marine fisheries will require:

    \begin{itemize}

      \item Improved monitoring of fish and shellfish populations

      \item Cooperatives fisheries management among communities and nations

      \item Reduction of fishing subsidies

      \item Careful consumer choices in buying seafood

    \end{itemize}

  \item Co-management of the fisheries with the government

    \begin{itemize}

      \item Government sets quotas for species and divides the quotas among communities

      \item Limits fishing seasons

      \item Regulate fishing gear

    \end{itemize}

  \item Government spends over 30 billion dollars per year subsidizing fishing (2015)

    \begin{itemize}

      \item Often leads to overfishing

      \item Discourages long-term sustainability of fish populations

    \end{itemize}

  \item 40\% of the world's rivers are dammed

  \item Many freshwater wetlands are destroyed

  \item Invasive species

  \item Overfishing

  \item Human population pressures

  \item Collectively, the world's largest body of freshwater are the Great Lakes

  \item Invaded by at least 162 non-native species

    \begin{itemize}

      \item Sea lamprey

      \item Zebra mussel

      \item Quagga mussel

      \item Asian Carp

    \end{itemize}

  \item Columbia River — US and Canada

    \begin{itemize}

      \item 119 Dams

    \end{itemize}

  \item Dams

    \begin{itemize}

      \item Provide hydroelectric power

      \item Provide irrigation water

      \item Hurt salmon

    \end{itemize}

  \item Ecosystem services of rivers

    \begin{itemize}

      \item Deliver nutrients to sea to help sustain coastal fisheries

      \item Deposit silt that maintains deltas

      \item Purify water

      \item Renew and nourish wetlands

      \item Provide habitats for wildlife

    \end{itemize}

  \item Sustainable management

    \begin{itemize}

      \item Support populations of commercial and sport fish species

      \item Prevent overfishing

      \item Reduce or eliminate invasive species

    \end{itemize}

  \item Be More Sustainable:

    \begin{itemize}

      \item Complete the mapping of the world's aquatic biodiversity

      \item Identify and preserve aquatic diversity hotspots

      \item Create large and fully protected marine reserves

      \item Protect and restore the world's lakes and rivers

      \item Ecological restoration projects worldwide

      \item Make conservation financially rewarding

    \end{itemize}

  \item The world's aquatic systems provide important economic and ecosystem services

    \begin{itemize}

      \item There could be immense ecological and economic benefits

      \item Aquatic ecosystems and fisheries are being severely degraded by human activities

      \item We can sustain aquatic biodiversity

    \end{itemize}
    
\end{enumerate}

\end{document}

