%%%%%%%%%%%%%%%%%%%%%%%%%%%%%%%%%%%%%%%%%%%%%%%%%%%%%%%%%%%%%%%%%%%%%%%%%%%%%%%%%%%%%%%%%%%%%%%%%%%%%%%%%%%%%%%%%%%%%%%%%%%%%%%%%%%%%%%%%%%%%%%%%%%%%%%%%%%%%%%%%%%%%%%%%%%%%%%%%%%%%%%%%%%%
% Written By Michael Brodskiy
% Class: AP Environmental Science
% Instructor: Mrs. Stansbury
%%%%%%%%%%%%%%%%%%%%%%%%%%%%%%%%%%%%%%%%%%%%%%%%%%%%%%%%%%%%%%%%%%%%%%%%%%%%%%%%%%%%%%%%%%%%%%%%%%%%%%%%%%%%%%%%%%%%%%%%%%%%%%%%%%%%%%%%%%%%%%%%%%%%%%%%%%%%%%%%%%%%%%%%%%%%%%%%%%%%%%%%%%%%

\documentclass[12pt]{article} 
\usepackage{alphalph}
\usepackage[utf8]{inputenc}
\usepackage[russian,english]{babel}
\usepackage{titling}
\usepackage{amsmath}
\usepackage{graphicx}
\usepackage{enumitem}
\usepackage{amssymb}
\usepackage{physics}
\usepackage{tikz}
\usepackage{mathdots}
\usepackage{yhmath}
\usepackage{cancel}
\usepackage{color}
\usepackage{siunitx}
\usepackage{array}
\usepackage{multirow}
\usepackage{gensymb}
\usepackage{tabularx}
\usepackage{booktabs}
\usepackage{soul}
\usetikzlibrary{fadings}
\usetikzlibrary{patterns}
\usetikzlibrary{shadows.blur}
\usetikzlibrary{shapes}
\usepackage[super]{nth}
\usepackage{expl3}
\usepackage[version=4]{mhchem}
\usepackage{hpstatement}
\usepackage{rsphrase}
\usepackage{everysel}
\usepackage{ragged2e}
\usepackage{geometry}
\usepackage{fancyhdr}
\usepackage{cancel}
\geometry{top=1.0in,bottom=1.0in,left=1.0in,right=1.0in}
\newcommand{\subtitle}[1]{%
  \posttitle{%
    \par\end{center}
    \begin{center}\large#1\end{center}
    \vskip0.5em}%

}
\usepackage{hyperref}
\hypersetup{
colorlinks=true,
linkcolor=blue,
filecolor=magenta,      
urlcolor=blue,
citecolor=blue,
}

\urlstyle{same}


\title{Biodiversity, Species Interactions, and Population Control}
\date{\today $-$ Period 1}
\author{Michael Brodskiy\\ \small Instructor: Mrs. Stansbury}

\begin{document}

\maketitle

\begin{enumerate}

  \item Populations change in response to environmental conditions

    \begin{itemize}

      \item Size — Number of individuals

      \item Density – Number of individuals in a certain space

      \item Age Distribution Structure — Percentage of individuals in each age group

      \item Other Types of Distributions — Spatial pattern (\textit{i.e.\ clumping}), uniform dispersion, random dispersion

    \end{itemize}

  \item Limits of Population Growth

    \begin{itemize}

      \item Birth

      \item Death

      \item Emigration/Immigration

    \end{itemize}

  \item Population Growth ($n$)

    \begin{itemize}

      \item $n=$ (birth + immigration) - (death + emigration)

      \item Dependent on resource availability

    \end{itemize}

  \item Interspecific Competition — 2 or more species interact to gain access to limited resources

  \item Intraspecific Competition — Competition within a species

  \item Predation — Prey/predator

  \item Symbiosis — Relationships between organisms

    \begin{itemize}

      \item Parasitism — Parasite/host

      \item Mutualism — Benefits both species

      \item Commensalism — Benefits on species, no effect on the other

    \end{itemize}

  \item Cyclic Changes — Sharp increases in number followed by seemingly periodic crashes

  \item Resource Partitioning — Specialized traits allow species to use shared resources at different times

  \item Competitive Exclusion — Intense competition between 2 equal species, where both suffer (one more than the other) by having reduced access to resources

  \item The Intrinsic Rate of Increase – $r$

    \begin{itemize}

      \item Rate at which a population could grow if it had unlimited resources (this will never happen)

      \item Always limits — Light, water, space, nutrients

      \item “High $r$” — Reproduce early in life, short generations (\textit{i.e.\ reproduce many times and many offspring like flies})

    \end{itemize}

  \item Carrying Capacity — The capacity for growth ($K$)

    \begin{itemize}

      \item Number of individuals of a given species that can be sustained indefinitely in a given area

      \item Determined by interaction between biotic potential and environmental resistance (factors that act jointly)

    \end{itemize}

  \item Bioitic Poential

    \begin{itemize}

      \item Reproductive rate

      \item Ability to migrate (animals) or disperse (seeds)

      \item Ability to invade new habitats

      \item Defense mechanisms

      \item Ability to cope with adverse conditions

    \end{itemize}

  \item Environmental Resistance

    \begin{itemize}

      \item Lack of food or nutrients

      \item Lack of water

      \item Lack of suitable habitat

      \item Adverse weather conditions

      \item Predators

      \item Disease

      \item Parasites

      \item Competitors

    \end{itemize}

  \item Population Density (or ecological population density) — Is the amount of individuals in a population per unit habitat area

    \begin{itemize}

      \item High density: Mice

      \item Low density: Mountain lions

    \end{itemize}

  \item Density depends upon:

    \begin{itemize}

      \item Social/population structure

      \item Mating relationships

      \item Time of year

    \end{itemize}

  \item Population Growth:

    \begin{itemize}

      \item J-shaped — Exponential growth curve, starts slowly then speeds up

      \item S-shaped — Logistic growth curve - slow start, rapid exponential growth, then levels off when $K$ is reached

    \end{itemize}

  \item Goal of every species is to produce as many offspring as possible

  \item Each individual has a limited amount of energy to put towards life and reproduction

  \item This leads to a trade-off of long life or high reproductive rate

  \item Natural Selection has led to two strategies for species: $r$-strategists and $K$-strategists

  \item $r$-Startegists

    \begin{itemize}

      \item Many small offspring

      \item Little or no parental care and protection of offspring

      \item Early reproductive age

      \item Usually generalist species

    \end{itemize}

  \item $K$-Strategists

    \begin{itemize}

      \item Fewer, larger offspring

      \item High parental care and protection of offspring

      \item Later reproductive age

      \item Usually specialist species

    \end{itemize}

  \item Survivorship Curves

    \begin{itemize}

      \item Late Loss: $K$-strategists that produce few yound and care for them until they reach reproductive age, thus reducing juvenile mortality

        \begin{itemize}

          \item Type I — Elephant
            
        \end{itemize}

      \item Constant Loss: Typically intermediate reproductive strategies with fairly constant mortality throughout all age classes, $K$-strategists

        \begin{itemize}

          \item Type II — Songbirds

        \end{itemize}

      \item Early Loss: $r$-strategists with many offspring, high infant mortality and high survivorship once a certain size and age

        \begin{itemize}

          \item Type III — Sea stars

        \end{itemize}

    \end{itemize}

  \item Population cycles:

    \begin{itemize}

      \item Relatively Stable — Slight fluctuation above and below carrying capacity, tropical rain forest

      \item Irruptive — High peak, crash (raccoons)

      \item Cyclic — “boom” and “bust”

    \end{itemize}

  \item Density Independent Factors — Floods, drought, hurricane, habitat destruction, pesticide spraying

  \item Density Dependent Factors — Competition for resources, predation, disease (infectious)

  \item Primary Succession — Gradual establishment of biotic communities in lifeless areas where there is no soil in a terrestrial ecosyste or no bottom sediment in an aquatic ecosystem

    \begin{itemize}

      \item Bare rock subject weathering crumbles into particles, releasing nutrients

      \item Pioneer or early successional species (lichens or mosses) attach to rock and start the process of rock formation by secreting mild acids

      \item Mid successional plants — Grasses, herbs, small plants

      \item Late successional spcies — Trees that can tolerate shade

    \end{itemize}

  \item Secondary Succession — Series of communities or ecosystems with different species development in places containing soil or bottom sediment

    \begin{itemize}

      \item Ecosystem has been disturbed, removed or destroyed, some soil or bottom sediment remains

      \item Abandoned farmland, burned or cut forests, heavily polluted streams, flooded lands

    \end{itemize}

\end{enumerate}

\end{document}

