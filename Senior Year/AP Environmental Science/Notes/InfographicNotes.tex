%%%%%%%%%%%%%%%%%%%%%%%%%%%%%%%%%%%%%%%%%%%%%%%%%%%%%%%%%%%%%%%%%%%%%%%%%%%%%%%%%%%%%%%%%%%%%%%%%%%%%%%%%%%%%%%%%%%%%%%%%%%%%%%%%%%%%%%%%%%%%%%%%%%%%%%%%%%%%%%%%%%%%%%%%%%%%%%%%%%%%%%%%%%%
% Written By Michael Brodskiy
% Class: AP Environmental Science
% Instructor: Mrs. Stansbury
%%%%%%%%%%%%%%%%%%%%%%%%%%%%%%%%%%%%%%%%%%%%%%%%%%%%%%%%%%%%%%%%%%%%%%%%%%%%%%%%%%%%%%%%%%%%%%%%%%%%%%%%%%%%%%%%%%%%%%%%%%%%%%%%%%%%%%%%%%%%%%%%%%%%%%%%%%%%%%%%%%%%%%%%%%%%%%%%%%%%%%%%%%%%

\documentclass[12pt]{article} 
\usepackage{alphalph}
\usepackage[utf8]{inputenc}
\usepackage[russian,english]{babel}
\usepackage{titling}
\usepackage{amsmath}
\usepackage{graphicx}
\usepackage{enumitem}
\usepackage{amssymb}
\usepackage{physics}
\usepackage{tikz}
\usepackage{mathdots}
\usepackage{yhmath}
\usepackage{cancel}
\usepackage{color}
\usepackage{siunitx}
\usepackage{array}
\usepackage{multirow}
\usepackage{gensymb}
\usepackage{tabularx}
\usepackage{booktabs}
\usepackage{soul}
\usetikzlibrary{fadings}
\usetikzlibrary{patterns}
\usetikzlibrary{shadows.blur}
\usetikzlibrary{shapes}
\usepackage[super]{nth}
\usepackage{expl3}
\usepackage[version=4]{mhchem}
\usepackage{hpstatement}
\usepackage{rsphrase}
\usepackage{everysel}
\usepackage{ragged2e}
\usepackage{geometry}
\usepackage{fancyhdr}
\usepackage{cancel}
\geometry{top=1.0in,bottom=1.0in,left=1.0in,right=1.0in}
\newcommand{\subtitle}[1]{%
  \posttitle{%
    \par\end{center}
    \begin{center}\large#1\end{center}
    \vskip0.5em}%

}
\usepackage{hyperref}
\hypersetup{
colorlinks=true,
linkcolor=blue,
filecolor=magenta,      
urlcolor=blue,
citecolor=blue,
}

\urlstyle{same}


\title{Infographic Notes}
\date{\today $-$ Period 1}
\author{Michael Brodskiy\\ \small Instructor: Mrs. Stansbury}

\begin{document}

\maketitle

\begin{enumerate}

  \item Greenhouse Effect — A process that occurs when gases in Earth's atmosphere trap the sun's heat

    \begin{itemize}

      \item Greenhouse Gas — A gas that contributes to the greenhouse effect by absorbing infrared radiation

        \begin{itemize}

          \item Carbon dioxide, methane, nitrous oxide, ozone, water vapor, and fluorinated gases

        \end{itemize}

      \item Most greenhouse gases are created from burning fossil fuels

    \end{itemize}

  \item The 6 Pollutants in the Clean Air Act:

    \begin{itemize}

      \item Particulate Matter

      \item Tropospheric (Ground-Level) Ozone

      \item Carbon Monoxide (CO)

      \item Sulfur Dioxide (\ce{SO2})

      \item Lead (Pb)

      \item Nitrogen Dioxide (\ce{NO2})

    \end{itemize}

  \item La Ni\~na and El Ni\~no

    \begin{itemize}

      \item La Ni\~na — Strong trade winds push warm water west across the Pacific, leaving cold water on the west coast. This shifts the Pacific Jet Stream north, giving rain to the pacific northwest and Canada

      \item El Ni\~no — Weak trade winds mean warm water stays on the west coast, shifting the Pacific Jet Stream south, giving rain to California and the southern USA. Occurs once every 2-7 years
    \end{itemize}

  \item Montreal Protocol — An international treaty that limits or prohibits use of certain pollutants

    \begin{itemize}

      \item CFCs — Phased out

      \item HCFCs — Phased out

      \item HFCs — In progress

    \end{itemize}

  \item Ground-Level vs. Stratospheric Ozone

    \begin{itemize}

      \item Ground-level ozone is bad and is created by chemical reactions

      \item Stratospheric ozone protects us from the sun's ultraviolet radiation (good ozone)

    \end{itemize}

  \item Solutions to Ground-Level Ozone

    \begin{itemize}

      \item Drive less

      \item Travel wise

      \item Use air-friendly products

      \item Strict \ce{NO_{\text{x}}} emission limits

    \end{itemize}

  \item Atmospheric Levels

    \begin{enumerate}

      \item Endless space — Beyond 3000 kilometers

      \item Exosphere — 800 to 3000 kilometers

      \item Thermosphere — 80-90 to 800 kilometers

      \item Mesosphere — 40-50 kilometers to 80-90 kilometers

      \item Stratosphere — 11 kilometers to 50 kilometers

      \item Troposphere — Ground to 11 kilometers

    \end{enumerate}

  \item Kyoto Protocol — An international treaty for climate change. Parties are to reduce greenhouse gas emissions from 1990 levels by 5\%

  \item Industrial vs. Photochemical Smog

    \begin{itemize}

      \item Industrial

        \begin{enumerate}

          \item Burning fossil fuels

          \item Sulfur dioxide released

        \end{enumerate}

      \item Photochemical

        \begin{enumerate}

          \item Accumulation of Pollutants

          \item Oxidant formation

        \end{enumerate}

    \end{itemize}

  \item Sea and Ice Level Changes

    \begin{itemize}

      \item Average global temperature increased from 57.3$^{\circ}$F to 58.76$^{\circ}$F

      \item Arctic ice mass has shrunk

    \end{itemize}

  \item Solutions to the Ozone

    \begin{itemize}

      \item Aerosol sprays and CFCs have led to widening of the ozone hole

      \item Damage can be minimized by buying local groceries, minimizing car use, and avoiding purchase of harmful cleaning products

    \end{itemize}

  \item Climate Change — A Positive Feedback Loop

    \begin{enumerate}

      \item \ce{CO2} increases

      \item The planet warms up

      \item The oceans release more carbon

      \item So less \ce{CO2} is taken from the atmosphere

      \item So \ce{CO2} increases again

      \item And the loop repeats

    \end{enumerate}

  \item Proxy Indicators

    \begin{itemize}

      \item Different things in the environment can indicate different levels of environmental damage:

        \begin{itemize}

          \item Ocean sediment

          \item Tree cores

          \item Ice cores

          \item Pollen

          \item Coral

        \end{itemize}

    \end{itemize}

  \item Temperature and Precipitation Changes

    \begin{itemize}

      \item Global temperature has increased

      \item The average surface air temperature in the Arctic Siberia has also increased

      \item Global precipitation rates have gone up

    \end{itemize}

  \item Changes in Animals and Plants

    \begin{itemize}

      \item Air

        \begin{itemize}

          \item Rising temperatures result in increased wildfires

          \item The extra trapped heat disrupts a lot of the environment

        \end{itemize}

      \item Land

        \begin{itemize}

          \item Drought leads to lack of food for animals

          \item Floods destroy land and aquatic habitats, which kills and displaces many animals and plants

        \end{itemize}

      \item Ocean

        \begin{itemize}

          \item Excess \ce{CO2} absorbed leads to higher temperatures

          \item This leads to coral bleaching and the degradation of habitats and deaths of aquatic species

        \end{itemize}

    \end{itemize}

\end{enumerate}

\end{document}

