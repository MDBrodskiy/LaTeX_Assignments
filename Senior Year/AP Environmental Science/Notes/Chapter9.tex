%%%%%%%%%%%%%%%%%%%%%%%%%%%%%%%%%%%%%%%%%%%%%%%%%%%%%%%%%%%%%%%%%%%%%%%%%%%%%%%%%%%%%%%%%%%%%%%%%%%%%%%%%%%%%%%%%%%%%%%%%%%%%%%%%%%%%%%%%%%%%%%%%%%%%%%%%%%%%%%%%%%%%%%%%%%%%%%%%%%%%%%%%%%%
% Written By Michael Brodskiy
% Class: AP Environmental Science
% Instructor: Mrs. Stansbury
%%%%%%%%%%%%%%%%%%%%%%%%%%%%%%%%%%%%%%%%%%%%%%%%%%%%%%%%%%%%%%%%%%%%%%%%%%%%%%%%%%%%%%%%%%%%%%%%%%%%%%%%%%%%%%%%%%%%%%%%%%%%%%%%%%%%%%%%%%%%%%%%%%%%%%%%%%%%%%%%%%%%%%%%%%%%%%%%%%%%%%%%%%%%

\documentclass[12pt]{article} 
\usepackage{alphalph}
\usepackage[utf8]{inputenc}
\usepackage[russian,english]{babel}
\usepackage{titling}
\usepackage{amsmath}
\usepackage{graphicx}
\usepackage{enumitem}
\usepackage{amssymb}
\usepackage{physics}
\usepackage{tikz}
\usepackage{mathdots}
\usepackage{yhmath}
\usepackage{cancel}
\usepackage{color}
\usepackage{siunitx}
\usepackage{array}
\usepackage{multirow}
\usepackage{gensymb}
\usepackage{tabularx}
\usepackage{booktabs}
\usepackage{soul}
\usetikzlibrary{fadings}
\usetikzlibrary{patterns}
\usetikzlibrary{shadows.blur}
\usetikzlibrary{shapes}
\usepackage[super]{nth}
\usepackage{expl3}
\usepackage[version=4]{mhchem}
\usepackage{hpstatement}
\usepackage{rsphrase}
\usepackage{everysel}
\usepackage{ragged2e}
\usepackage{geometry}
\usepackage{fancyhdr}
\usepackage{cancel}
\geometry{top=1.0in,bottom=1.0in,left=1.0in,right=1.0in}
\newcommand{\subtitle}[1]{%
  \posttitle{%
    \par\end{center}
    \begin{center}\large#1\end{center}
    \vskip0.5em}%

}
\usepackage{hyperref}
\hypersetup{
colorlinks=true,
linkcolor=blue,
filecolor=magenta,      
urlcolor=blue,
citecolor=blue,
}

\urlstyle{same}


\title{Sustaining Biodiversity: Saving Speices and Ecosystem Services}
\date{\today $-$ Period 1}
\author{Michael Brodskiy\\ \small Instructor: Mrs. Stansbury}

\begin{document}

\maketitle

\begin{enumerate}

  \item Biological Extinction — No species members are left alive

  \item Trophic Cascade — Addition or elimination of the top predator causes a cascade effect down the trophic levels

  \item Background Extinction Rate — One extinction of species per year per million species

  \item Endangered Species — Species of animal or plant that is seriously at risk of extinction

  \item Threatened Species — Vulnerable to endangerment

  \item Regionally Extinct or Local Extinction — Extinct in specific areas

  \item Functionally Extinct — Some individuals remain, however, the species can no longer play a functional role in the ecosystem

    \begin{itemize}

      \item Ex. Old growth redwood trees; Abalone

    \end{itemize}

  \item Biodiversity Hotspots

    \begin{itemize}

      \item Biogeographic regions that are a reservoir of biodiversity, but is threatened with destruction

      \item Extinction rates projected to be much higher than average

      \item Biologically diverse environments are being eliminated or fragmented

    \end{itemize}

  \item The 6 Top Threats to Species: HIPPCO

    \begin{itemize}

      \item Habitat destruction, degradation, and fragmentation

      \item Invasive (non-native) species

      \item Population and resource use growth

      \item Pollution

        \begin{itemize}

          \item Bioaccumulation can cause extinctions of species not directly affected by pollution

        \end{itemize}

      \item Climate Change

      \item Overexploitation

        \begin{itemize}

          \item Poaching and smuggling of animals and plants

            \begin{itemize}

              \item Animal parts

              \item Pets

              \item Plants for landscaping and enjoyment

            \end{itemize}

        \end{itemize}

    \end{itemize}

  \item Habitat Fragmentation — Large intact habitat divided by roads, crops, and urban development

  \item National parks and nature reserves are habitat islands

  \item Invasive (non-native) species

    \begin{itemize}

      \item There are 7,100 harmful invasive species that have been deliberately or accidentally introduced into the United States

    \end{itemize}

\end{enumerate}

\end{document}

