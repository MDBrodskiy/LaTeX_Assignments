%%%%%%%%%%%%%%%%%%%%%%%%%%%%%%%%%%%%%%%%%%%%%%%%%%%%%%%%%%%%%%%%%%%%%%%%%%%%%%%%%%%%%%%%%%%%%%%%%%%%%%%%%%%%%%%%%%%%%%%%%%%%%%%%%%%%%%%%%%%%%%%%%%%%%%%%%%%%%%%%%%%%%%%%%%%%%%%%%%%%%%%%%%%%
% Written By Michael Brodskiy
% Class: AP Environmental Science
% Instructor: Mrs. Stansbury
%%%%%%%%%%%%%%%%%%%%%%%%%%%%%%%%%%%%%%%%%%%%%%%%%%%%%%%%%%%%%%%%%%%%%%%%%%%%%%%%%%%%%%%%%%%%%%%%%%%%%%%%%%%%%%%%%%%%%%%%%%%%%%%%%%%%%%%%%%%%%%%%%%%%%%%%%%%%%%%%%%%%%%%%%%%%%%%%%%%%%%%%%%%%

\documentclass[12pt]{article} 
\usepackage{alphalph}
\usepackage[utf8]{inputenc}
\usepackage[russian,english]{babel}
\usepackage{titling}
\usepackage{amsmath}
\usepackage{graphicx}
\usepackage{enumitem}
\usepackage{amssymb}
\usepackage{physics}
\usepackage{tikz}
\usepackage{mathdots}
\usepackage{yhmath}
\usepackage{cancel}
\usepackage{color}
\usepackage{siunitx}
\usepackage{array}
\usepackage{multirow}
\usepackage{gensymb}
\usepackage{tabularx}
\usepackage{booktabs}
\usepackage{soul}
\usetikzlibrary{fadings}
\usetikzlibrary{patterns}
\usetikzlibrary{shadows.blur}
\usetikzlibrary{shapes}
\usepackage[super]{nth}
\usepackage{expl3}
\usepackage[version=4]{mhchem}
\usepackage{hpstatement}
\usepackage{rsphrase}
\usepackage{everysel}
\usepackage{ragged2e}
\usepackage{geometry}
\usepackage{fancyhdr}
\usepackage{cancel}
\geometry{top=1.0in,bottom=1.0in,left=1.0in,right=1.0in}
\newcommand{\subtitle}[1]{%
  \posttitle{%
    \par\end{center}
    \begin{center}\large#1\end{center}
    \vskip0.5em}%

}
\usepackage{hyperref}
\hypersetup{
colorlinks=true,
linkcolor=blue,
filecolor=magenta,      
urlcolor=blue,
citecolor=blue,
}

\urlstyle{same}


\title{The Human Population and Its Impact}
\date{\today $-$ Period 1}
\author{Michael Brodskiy\\ \small Instructor: Mrs. Stansbury}

\begin{document}

\maketitle

\begin{enumerate}

  \item As of January 2020, the population is 7.8 billion

  \item As the human population grows, so does the global total human ecological footprint

  \item Cultural Carrying Capacity — Total number of people who could live in reasonable freedom and comfort indefinitely, without decreasing the ability of the Earth to sustain future generations

  \item Fertility Rate — Number of children born to a woman during her lifetime

  \item Replacement-Level Fertility — Average number of children a couple must produce to replace themselves 

    \begin{itemize}

      \item Approximately 2.1 in developed countries

      \item up to 2.5 in developing countries

    \end{itemize}

  \item Total Fertility Rate (TFR) — Average number of children born to women in a population

    \begin{itemize}

      \item Between 1955 and 2012, the global TFR dropped from 5 to 2.4

      \item To eventually halt population growth, the global TFR will have to drop to 2.1

    \end{itemize}

  \item Several Factors Affect Birth Rates and Fertility Rates:

    \begin{itemize}

      \item Children as part of the labor force

      \item Cost of raising and educating children

      \item Availability of private and public pension

      \item Urbanization

      \item Educational and employment opportunities for women

      \item Average age of a woman at marriage

      \item Availability of legal abortions

      \item Availability of reliable birth control methods

      \item Religious beliefs, traditions, and cultural norms

    \end{itemize}

  \item Life Expectancy — Average length of life

  \item Infant Mortality Rate — Number of live births that die in the first year


  \item High infant mortality rate indicates:

    \begin{itemize}

      \item Insufficient food

      \item Poor nutrition

      \item High incidence of infectious disease

    \end{itemize}

  \item Migration — The movement of people into and out of specific geographic areas

    \begin{itemize}

      \item Causes:

        \begin{itemize}

          \item Economic improvement

          \item Religious and political freedom

          \item Wars

        \end{itemize}

    \end{itemize}

  \item Age structure categories

    \begin{itemize}

      \item Prereproductive Ages (0-14)

      \item Reproductive Ages (15-44)

      \item Postreproductive Ages (45 and older)

    \end{itemize}

  \item Seniors are the fastest-growing age group

  \item Demographic Transition — As countries become more industrialized:

    \begin{itemize}

      \item First death rates decline

      \item Then birth rates decline

    \end{itemize}

  \item Four Stages:

    \begin{enumerate}

      \item Preindustrial

      \item Transitional

      \item Industrial

      \item Postindustrial

    \end{enumerate}

  \item Factors that decrease total fertility rates:

    \begin{itemize}

      \item Education

      \item Paying jobs

      \item Ability to control fertility

    \end{itemize}

\end{enumerate}

\end{document}

