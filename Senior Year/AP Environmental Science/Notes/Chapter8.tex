%%%%%%%%%%%%%%%%%%%%%%%%%%%%%%%%%%%%%%%%%%%%%%%%%%%%%%%%%%%%%%%%%%%%%%%%%%%%%%%%%%%%%%%%%%%%%%%%%%%%%%%%%%%%%%%%%%%%%%%%%%%%%%%%%%%%%%%%%%%%%%%%%%%%%%%%%%%%%%%%%%%%%%%%%%%%%%%%%%%%%%%%%%%%
% Written By Michael Brodskiy
% Class: AP Environmental Science
% Instructor: Mrs. Stansbury
%%%%%%%%%%%%%%%%%%%%%%%%%%%%%%%%%%%%%%%%%%%%%%%%%%%%%%%%%%%%%%%%%%%%%%%%%%%%%%%%%%%%%%%%%%%%%%%%%%%%%%%%%%%%%%%%%%%%%%%%%%%%%%%%%%%%%%%%%%%%%%%%%%%%%%%%%%%%%%%%%%%%%%%%%%%%%%%%%%%%%%%%%%%%

\documentclass[12pt]{article} 
\usepackage{alphalph}
\usepackage[utf8]{inputenc}
\usepackage[russian,english]{babel}
\usepackage{titling}
\usepackage{amsmath}
\usepackage{graphicx}
\usepackage{enumitem}
\usepackage{amssymb}
\usepackage{physics}
\usepackage{tikz}
\usepackage{mathdots}
\usepackage{yhmath}
\usepackage{cancel}
\usepackage{color}
\usepackage{siunitx}
\usepackage{array}
\usepackage{multirow}
\usepackage{gensymb}
\usepackage{tabularx}
\usepackage{booktabs}
\usepackage{soul}
\usetikzlibrary{fadings}
\usetikzlibrary{patterns}
\usetikzlibrary{shadows.blur}
\usetikzlibrary{shapes}
\usepackage[super]{nth}
\usepackage{expl3}
\usepackage[version=4]{mhchem}
\usepackage{hpstatement}
\usepackage{rsphrase}
\usepackage{everysel}
\usepackage{ragged2e}
\usepackage{geometry}
\usepackage{fancyhdr}
\usepackage{cancel}
\geometry{top=1.0in,bottom=1.0in,left=1.0in,right=1.0in}
\newcommand{\subtitle}[1]{%
  \posttitle{%
    \par\end{center}
    \begin{center}\large#1\end{center}
    \vskip0.5em}%

}
\usepackage{hyperref}
\hypersetup{
colorlinks=true,
linkcolor=blue,
filecolor=magenta,      
urlcolor=blue,
citecolor=blue,
}

\urlstyle{same}


\title{Aquatic Biodiversity}
\date{\today $-$ Period 1}
\author{Michael Brodskiy\\ \small Instructor: Mrs. Stansbury}

\begin{document}

\maketitle

\begin{enumerate}

  \item Saltwater — 71\% of Earth's surface

    \begin{itemize}

      \item Oceans and estuaries

      \item Coastlands and shorelines

      \item Coral reefs

      \item Mangrove forests
        
    \end{itemize}

  \item Freshwater — 2.2\% of Earth's surfaces

    \begin{itemize}

      \item Lakes, rivers, and streams

      \item Ice, glaciers

    \end{itemize}

  \item Phytoplankton

    \begin{itemize}

      \item Tiny, photosynthetic organisms. Primary producers for most aquatic food webs

    \end{itemize}

  \item Ultraplankton

    \begin{itemize}

      \item Tiny photosynthetic bacteria

    \end{itemize}

  \item Zooplankton

    \begin{itemize}

      \item Secondary consumers

      \item Can be single-celled and up to large invertebrates like jellyfish

    \end{itemize}

  \item Nekton — Strong swimmers (fish, turtles, whales)

  \item Benthos — Bottom dwellers (oysters, sea starts, clams, lobsters, clams)

  \item Decomposers — Mostly bacteria

  \item Distribution of organisms and biodiversity depends on:

    \begin{itemize}

      \item Temperature

      \item Dissolved oxygen content

      \item Availability of food

      \item Availability of light and nutrients needed for photosynthesis

      \item Turbidity — Degree of cloudiness in water; inhibits photosynthesis

    \end{itemize}

  \item Zones: Euphotic $\rightarrow$ Bathyal $\rightarrow$ Abyssal

  \item Water temperature drops rapidly between the euphotic zone and the abyssal zone in an area called the thermocline

  \item Estuaries — Where rivers meet the sea

  \item Coastal Wetlands — Coastal land covered with water all or part of the year

  \item Brackish Water — Seawater mixes with freshwater

  \item These ecosystems are all very productive with high nutrient levels

  \item Highly Productive Areas:

    \begin{itemize}

      \item River mouths

      \item Inlets

      \item Bays

      \item Sounds

      \item Salt marshes

      \item Mangrove forests

    \end{itemize}

  \item Intertidal zone

    \begin{itemize}

      \item Area of shore between high and low tides

      \item Rocky shore

      \item Sandy shore, barrier beach

    \end{itemize}

  \item Organism adaptations necessary to deal with daily salinity and moisture changes

  \item Coral reefs are the marine equivalent of tropical rainforests

  \item Reefs are being destroyed and damaged worldwide

  \item Ocean Acidification

    \begin{itemize}

      \item Ocean absorbs \ce{CO2}

      \item \ce{CO2} reacts with ocean water to form a weak acid that decreases levels of carbonate ions (\ce{CO3^2-}) needed to form coral

    \end{itemize}

  \item Major threats to marine systems include:

    \begin{itemize}

      \item Coastal development

      \item Overfishing; use of fishing trawlers

      \item Runoff of nonpoint source pollution

      \item Habitat destruction

      \item Introduction of invasive species

    \end{itemize}

  \item Standing (lentic) bodies of freshwater

    \begin{itemize}

      \item Lakes

      \item Ponds

      \item Inland wetlands

    \end{itemize}

  \item Flowing (lotic) systems of fresh water

    \begin{itemize}

      \item Streams

      \item Rivers

    \end{itemize}

  \item Lakes have four zones based on depth and distance from shore

    \begin{itemize}

      \item Littoral zone

        \begin{itemize}

          \item Near shore where rooted plants grow; high biodiversity

          \item Turtles, frogs, crayfish, some fish

        \end{itemize}

      \item Limnetic zone

        \begin{itemize}

          \item Open, sunlight area away from shore; main photosynthetic zone

          \item Some larger fish

        \end{itemize}

      \item Profundal zone

        \begin{itemize}

          \item Deep water too dark for photsynthesis

          \item Low oxygen levels

          \item Some fish

        \end{itemize}

      \item Benthic zone

        \begin{itemize}

          \item Decomposers

          \item Detritus feeders

          \item Some fish

          \item Nourished primarily by dead matter

        \end{itemize}

    \end{itemize}

  \item Oligotrophic lakes

    \begin{itemize}

      \item Low levels of nutrients and low Net Primary Productivity

      \item Very clear water

    \end{itemize}

  \item Eutrophic lakes

    \begin{itemize}

      \item High levels of nutrients and high NPP

      \item Murky water with high turbidity

    \end{itemize}

  \item Cultural eutrophication of lakes from human input of nutrients

  \item Inland wetlands

    \begin{itemize}

      \item Lands located away from coats that are covered with freshwater all or part of the time

      \item Includes: Marshes, swamps, prairie potholes, floodplains, and arctic tundra

    \end{itemize}

\end{enumerate}

\end{document}

