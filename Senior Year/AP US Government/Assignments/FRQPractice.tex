%%%%%%%%%%%%%%%%%%%%%%%%%%%%%%%%%%%%%%%%%%%%%%%%%%%%%%%%%%%%%%%%%%%%%%%%%%%%%%%%%%%%%%%%%%%%%%%%%%%%%%%%%%%%%%%%%%%%%%%%%%%%%%%%%%%%%%%%%%%%%%%%%%%%%%%%%%%%%%%%%%%%%%%%%%%%%%%%%%%%%%%%%%%%
% Written By Michael Brodskiy
% Class: AP US Government
% Instructor: Mr. Bradshaw
%%%%%%%%%%%%%%%%%%%%%%%%%%%%%%%%%%%%%%%%%%%%%%%%%%%%%%%%%%%%%%%%%%%%%%%%%%%%%%%%%%%%%%%%%%%%%%%%%%%%%%%%%%%%%%%%%%%%%%%%%%%%%%%%%%%%%%%%%%%%%%%%%%%%%%%%%%%%%%%%%%%%%%%%%%%%%%%%%%%%%%%%%%%%

\documentclass[12pt]{article} 
\usepackage{alphalph}
\usepackage[utf8]{inputenc}
\usepackage[russian,english]{babel}
\usepackage{titling}
\usepackage{diagbox}
\usepackage{pifont}
\usepackage{amsmath}
\usepackage{graphicx}
\usepackage{enumitem}
\usepackage{amssymb}
\usepackage{physics}
\usepackage{tikz}
\usepackage{mathdots}
\usepackage{yhmath}
\usepackage{cancel}
\usepackage{color, colortbl}
\definecolor{BurntOrange}{rgb}{0.85, 0.6, 0.3}
\definecolor{Gray}{gray}{.5}
\usepackage{siunitx}
\usepackage{array}
\usepackage{multirow}
\usepackage{gensymb}
\usepackage{tabularx}
\usepackage{booktabs}
\usepackage{soul}
\usetikzlibrary{fadings}
\usetikzlibrary{patterns}
\usetikzlibrary{shadows.blur}
\usetikzlibrary{shapes}
\usepackage[super]{nth}
\usepackage{expl3}
\usepackage[version=4]{mhchem}
\usepackage{hpstatement}
\usepackage{rsphrase}
\usepackage{everysel}
\usepackage{ragged2e}
\usepackage{geometry}
\usepackage{fancyhdr}
\usepackage{cancel}
\newcommand{\xmark}{\ding{55}}
\geometry{top=1.0in,bottom=1.0in,left=1.0in,right=1.0in}
\newcommand{\subtitle}[1]{%
  \posttitle{%
    \par\end{center}
    \begin{center}\large#1\end{center}
    \vskip0.5em}%

}
\usepackage{hyperref}
\hypersetup{
colorlinks=true,
linkcolor=blue,
filecolor=magenta,      
urlcolor=blue,
citecolor=blue,
}

\urlstyle{same}


\title{Practice FRQs}
\date{Period 3}
\author{Michael Brodskiy\\ \small Instructor: Mr. Bradshaw}

% Mathematical Operations:

% Sum: $$\sum_{n=a}^{b} f(x) $$
% Integral: $$\int_{lower}^{upper} f(x) dx$$
% Limit: $$\lim_{x\to\infty} f(x)$$

\begin{document}

\maketitle

\begin{enumerate}

  \item

    \begin{enumerate}

      \item

        \begin{itemize}

          \item Scandals — If a president is caught in some kind of scandal (\textit{i.e.} corruption, sex scandal, etc.), their popularity may take a big dip, as the public sees them as unqualified to rule

          \item War — If a country is locked in endless war, the president may be seen as a warmonger, and, therefore, popularity of the president may decline as they seem less competent

        \end{itemize}

      \item 

        \begin{itemize}

          \item Taxes — A president may become more popular if they decrease the taxes. As the population will always be reluctant to pay more taxes, a campaign to lower taxes may make a president more popular

          \item Peace — If the world sees relative peace under a president, the public may see this as a successful presidential achievement, thereby increasing support for said president

        \end{itemize}

    \end{enumerate}

  \item

    \begin{enumerate}

      \item 

        \begin{itemize}

          \item Specialization — Specialization is a system in which a special committee is created for a certain task (\textit{i.e.} taxes, healthcare, etc.). By creating specialized committees, more qualified individuals may be assigned to evaluate certain propositions, thereby streamlining the process, as more educated individuals may be able to make a decision quicker, limiting debates

          \item Party Representation — By setting up representation in committees, parties may be able to influence the outcomes of certain committees. For example, a committee with a majority of Democrats could lead to more Democrat-positive decisions

        \end{itemize}

      \item

        \begin{itemize}

          \item Influence — First and foremost, party leaders, due to their positions within the party, may influence the votes within their party, which would cause homogeneity within a party

          \item Logrolling — Second, party leaders may influence the decisions of others by exchanging political favors

        \end{itemize}

    \end{enumerate}

  \item

    \begin{enumerate}

      \item 

        \begin{itemize}

          \item Commander in Chief

          \item Ability to meet foreign heads of state

        \end{itemize}

      \item

        \begin{itemize}

          \item Holds the key to the U.S. Treasury (power of the purse)

          \item May ratify treaties or foreign agreements

        \end{itemize}

      \item

        \begin{itemize}

          \item Head of the executive

          \item Media access

        \end{itemize}

      \item

        \begin{itemize}

          \item As head of the executive branch, the president is regarded as the foremost representative of the United States, which makes him more accepted by foreign heads of state

          \item As a single person, the president has easy access to platform through which he can stir Congress into action

        \end{itemize}

    \end{enumerate}

  \item

    \begin{enumerate}

      \item 

        \begin{itemize}

          \item Veto — The power to veto, as a check on the legislative branch, allows the president to influence decisions over controversial bills, as the president holds the final decision

          \item Appointment of Judges — As the nominator for judges, the president may nominate judges who may help him carry out his policies, and, with respect to Congress, proclaim a law (even once it passed a veto) unconstitutional

        \end{itemize}

      \item

        \begin{itemize}

          \item Party Polarization — If Congress if heavily polarized against the president, as is often the case, they may be able to pass over the president's veto with a super majority, allowing for easier passage of polar laws

          \item Mandatory Spending — Because of mandatory spending, the president may not be able to fit some part of their policy into the budget

        \end{itemize}

    \end{enumerate}

\end{enumerate}

\end{document}

