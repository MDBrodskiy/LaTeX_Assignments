%%%%%%%%%%%%%%%%%%%%%%%%%%%%%%%%%%%%%%%%%%%%%%%%%%%%%%%%%%%%%%%%%%%%%%%%%%%%%%%%%%%%%%%%%%%%%%%%%%%%%%%%%%%%%%%%%%%%%%%%%%%%%%%%%%%%%%%%%%%%%%%%%%%%%%%%%%%%%%%%%%%%%%%%%%%%%%%%%%%%%%%%%%%%
% Written By Michael Brodskiy
% Class: AP US Government
% Instructor: Mr. Bradshaw
%%%%%%%%%%%%%%%%%%%%%%%%%%%%%%%%%%%%%%%%%%%%%%%%%%%%%%%%%%%%%%%%%%%%%%%%%%%%%%%%%%%%%%%%%%%%%%%%%%%%%%%%%%%%%%%%%%%%%%%%%%%%%%%%%%%%%%%%%%%%%%%%%%%%%%%%%%%%%%%%%%%%%%%%%%%%%%%%%%%%%%%%%%%%

\documentclass[12pt]{article} 
\usepackage{alphalph}
\usepackage[utf8]{inputenc}
\usepackage[russian,english]{babel}
\usepackage{titling}
\usepackage{diagbox}
\usepackage{pifont}
\usepackage{amsmath}
\usepackage{graphicx}
\usepackage{enumitem}
\usepackage{amssymb}
\usepackage{physics}
\usepackage{tikz}
\usepackage{mathdots}
\usepackage{yhmath}
\usepackage{cancel}
\usepackage{color, colortbl}
\definecolor{BurntOrange}{rgb}{0.85, 0.6, 0.3}
\definecolor{Gray}{gray}{.5}
\usepackage{siunitx}
\usepackage{array}
\usepackage{multirow}
\usepackage{gensymb}
\usepackage{tabularx}
\usepackage{booktabs}
\usepackage{soul}
\usetikzlibrary{fadings}
\usetikzlibrary{patterns}
\usetikzlibrary{shadows.blur}
\usetikzlibrary{shapes}
\usepackage[super]{nth}
\usepackage{expl3}
\usepackage[version=4]{mhchem}
\usepackage{hpstatement}
\usepackage{rsphrase}
\usepackage{everysel}
\usepackage{ragged2e}
\usepackage{geometry}
\usepackage{fancyhdr}
\usepackage{cancel}
\newcommand{\xmark}{\ding{55}}
\geometry{top=1.0in,bottom=1.0in,left=1.0in,right=1.0in}
\newcommand{\subtitle}[1]{%
  \posttitle{%
    \par\end{center}
    \begin{center}\large#1\end{center}
    \vskip0.5em}%

}
\usepackage{hyperref}
\hypersetup{
colorlinks=true,
linkcolor=blue,
filecolor=magenta,      
urlcolor=blue,
citecolor=blue,
}

\urlstyle{same}


\title{Locke Reading Questions}
\date{Period 3}
\author{Michael Brodskiy\\ \small Instructor: Mr. Bradshaw}

% Mathematical Operations:

% Sum: $$\sum_{n=a}^{b} f(x) $$
% Integral: $$\int_{lower}^{upper} f(x) dx$$
% Limit: $$\lim_{x\to\infty} f(x)$$

\begin{document}

\maketitle

\begin{enumerate}

  \item What is the legitimate purpose of government?

    \begin{justify}
      According to Locke, the foremost purpose of government is to protect the property of its citizens, as Locke states, “\textit{[t]he great and chief end, therefore, of men's uniting into commonwealths, and putting themselves under government, \underline{is the preservation of their property}}”
    \end{justify}

  \item What are man's rights in a state of nature?

    \begin{justify}
      As Locke states, in a state of nature, mankind is “\textit{absolute lord of his own person and possessions, equal to the greatest, and subject to no body}”
    \end{justify}

  \item Why would man give up these rights?

    \begin{justify}
      In Locke's opinion, mankind is not secure in its rights in a state of nature, or, “\textit{the enjoyment of the property he has in this state is very unsafe, very unsecure.}” Therefore, to safeguard these rights, man would establish a government and “\textit{make them take sanctuary under the established laws of government, and therein seek the preservation of their property}”
    \end{justify}

  \item What is meant by the term “the common good”

    \begin{justify}
      “Common good” refers to the good of a group, such as a community, society, or family. This “common good” defines what is beneficial for a group rather than the individual
    \end{justify}

  \item What is meant by the “general name, property”? Is this term used in our founding documents?

    \begin{justify}
      Locke defines the “general name, property” as a person's life, their freedoms, and their belongings, as he states that “\dots\textit{lives, liberties, and estates, which I call by the general name, property}”
    \end{justify}

\end{enumerate}

\end{document}

