%%%%%%%%%%%%%%%%%%%%%%%%%%%%%%%%%%%%%%%%%%%%%%%%%%%%%%%%%%%%%%%%%%%%%%%%%%%%%%%%%%%%%%%%%%%%%%%%%%%%%%%%%%%%%%%%%%%%%%%%%%%%%%%%%%%%%%%%%%%%%%%%%%%%%%%%%%%%%%%%%%%%%%%%%%%%%%%%%%%%%%%%%%%%
% Written By Michael Brodskiy
% Class: AP US Government
% Instructor: Mr. Bradshaw
%%%%%%%%%%%%%%%%%%%%%%%%%%%%%%%%%%%%%%%%%%%%%%%%%%%%%%%%%%%%%%%%%%%%%%%%%%%%%%%%%%%%%%%%%%%%%%%%%%%%%%%%%%%%%%%%%%%%%%%%%%%%%%%%%%%%%%%%%%%%%%%%%%%%%%%%%%%%%%%%%%%%%%%%%%%%%%%%%%%%%%%%%%%%

\documentclass[12pt]{article} 
\usepackage{alphalph}
\usepackage[utf8]{inputenc}
\usepackage[russian,english]{babel}
\usepackage{titling}
\usepackage{diagbox}
\usepackage{pifont}
\usepackage{amsmath}
\usepackage{graphicx}
\usepackage{enumitem}
\usepackage{amssymb}
\usepackage{physics}
\usepackage{tikz}
\usepackage{mathdots}
\usepackage{yhmath}
\usepackage{cancel}
\usepackage{color, colortbl}
\definecolor{BurntOrange}{rgb}{0.85, 0.6, 0.3}
\definecolor{Gray}{gray}{.5}
\usepackage{siunitx}
\usepackage{array}
\usepackage{multirow}
\usepackage{gensymb}
\usepackage{tabularx}
\usepackage{booktabs}
\usepackage{soul}
\usetikzlibrary{fadings}
\usetikzlibrary{patterns}
\usetikzlibrary{shadows.blur}
\usetikzlibrary{shapes}
\usepackage[super]{nth}
\usepackage{expl3}
\usepackage[version=4]{mhchem}
\usepackage{hpstatement}
\usepackage{rsphrase}
\usepackage{everysel}
\usepackage{ragged2e}
\usepackage{geometry}
\usepackage{fancyhdr}
\usepackage{cancel}
\newcommand{\xmark}{\ding{55}}
\geometry{top=1.0in,bottom=1.0in,left=1.0in,right=1.0in}
\newcommand{\subtitle}[1]{%
  \posttitle{%
    \par\end{center}
    \begin{center}\large#1\end{center}
    \vskip0.5em}%

}
\usepackage{hyperref}
\hypersetup{
colorlinks=true,
linkcolor=blue,
filecolor=magenta,      
urlcolor=blue,
citecolor=blue,
}

\urlstyle{same}


\title{Notes — Week 3}
\date{Period 3}
\author{Michael Brodskiy\\ \small Instructor: Mr. Bradshaw}

% Mathematical Operations:

% Sum: $$\sum_{n=a}^{b} f(x) $$
% Integral: $$\int_{lower}^{upper} f(x) dx$$
% Limit: $$\lim_{x\to\infty} f(x)$$

\begin{document}

\maketitle

\begin{itemize}

  \item While controlling the colonies, England employed a mercantilist system

    \begin{itemize}

      \item Mercantilism focuses on more gold flowing in than flowing out, as people thought nations grew richer by keeping more within

      \item Adam Smith corrected this economic view by developing capitalism

      \item War was a byproduct of mercantilist theory

      \item \textit{An Inquiry into the Nature and Causes of the Wealth of Nations}, written by Adam Smith, was published in 1776

    \end{itemize}

  \item Should the English colonies have gone to war?

    \begin{center}
      \begin{tabular}{| c |}
        \hline
        \rowcolor{BurntOrange} \textcolor{white}{Pros for England}\\
        \hline
        Navy\\
        \rowcolor{Gray!50} Economy\\
        Army\\
        \rowcolor{Gray!50} Education\\
        Organized Gov't\\
        \hline
      \end{tabular}
    \end{center}

    \begin{center}
      \begin{tabular}{| p{.3\textwidth} | p{.2\textwidth} | p{.2\textwidth} |}
        \hline
        & \rowcolor{BurntOrange} \textcolor{white}{England} & \textcolor{white}{Colonies}\\
      \end{tabular}\\
      \vspace{-1.1pt}
      \begin{tabular}{| p{.3\textwidth} | p{.087\textwidth} | p{.087\textwidth} | p{.087\textwidth} | p{.087\textwidth}|}
        & \rowcolor{red!20} Yes & \cellcolor{blue!20} No & Yes & \cellcolor{blue!20} No\\
        \hline
        Navy & \textcolor{blue}{\checkmark} & & &\textcolor{red}{\xmark} \\
        \rowcolor{Gray!50} Army &  \textcolor{blue}{\checkmark} & & &\textcolor{red}{\xmark} \\
        Economy & \textcolor{blue}{\checkmark}  & & &\textcolor{red}{\xmark} \\
        \rowcolor{Gray!50} Leadership/Military & \textcolor{blue}{\checkmark}  & & &\textcolor{red}{\xmark} \\
        Arms & \textcolor{blue}{\checkmark}  & & & \textcolor{red}{\xmark}\\
        \hline
      \end{tabular}
    \end{center}

  \item Most colonists fleeing religious persecution; Colonists had to pay out of pocket for the trip to America (not government funded)

  \item Colonists wrote their own laws due to distance from England

  \item Prior to the American Revolution, no colony had ever won its own independence from a mother country

  \item Magna Carta (1215) is the first document limiting government power, limits monarch against the nobles

  \item English Civil War (1641) led by Oliver Cromwell, executed Charles I

  \item Petition of Right signed by Charles I, limits monarch power against parliament

  \item English Bill of Rights (1689) passes rights to common males

  \item The most revolutionary idea at the time was that ordinary people could govern themselves

  \item Prior to the revolution, most colonists revered the king and used British products

  \item Britain defeats France in the Seven Years' War (French and Indian War) in 1763 (started in 1756)

  \item Very little connection between the English colonies (population 3,000,000); Most colonists did not travel farther than 30 miles. Most people (95\%) live in the countryside. Most are farmers who own their own land. The opposite is true in England, where most people are tenant farmers

  \item The aristocracy in Europe was protected by law

  \item George Washington grew up in the poorer middle class. He married a wealthy widow, and was determined to become the richest and most influential person in Virginia

  \item In 1765, the British Stamp Act places a tax on many manufactured goods in the Americas. The Stamp Act became an outrage in the colonies, as it meant that the colonies did not have a voice in their own laws, as they had been taxing themselves for over 150 years. The tax was seen as arbitrary (without reason or cause). The reason for arguing wasn't the tax itself, rather, it questioned whether parliament had the right to tax the colonies

  \item The power to tax is the power to destroy

  \item Sons of liberty organized protests with effigies

  \item People ruled under monarchies are called subjects

  \item There was a severe hierarchy in eighteenth century society

  \item March 5, 1770 — Boston Massacre

  \item Crispus Attucks was an African-American regarded as the first to be killed in the Boston Massacre, and, therefore, the American Revolution

  \item Unlike the American colonies, England did not have a significant middle class

\end{itemize}

\end{document}

