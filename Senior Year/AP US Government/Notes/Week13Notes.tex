%%%%%%%%%%%%%%%%%%%%%%%%%%%%%%%%%%%%%%%%%%%%%%%%%%%%%%%%%%%%%%%%%%%%%%%%%%%%%%%%%%%%%%%%%%%%%%%%%%%%%%%%%%%%%%%%%%%%%%%%%%%%%%%%%%%%%%%%%%%%%%%%%%%%%%%%%%%%%%%%%%%%%%%%%%%%%%%%%%%%%%%%%%%%
% Written By Michael Brodskiy
% Class: AP US Government
% Instructor: Mr. Bradshaw
%%%%%%%%%%%%%%%%%%%%%%%%%%%%%%%%%%%%%%%%%%%%%%%%%%%%%%%%%%%%%%%%%%%%%%%%%%%%%%%%%%%%%%%%%%%%%%%%%%%%%%%%%%%%%%%%%%%%%%%%%%%%%%%%%%%%%%%%%%%%%%%%%%%%%%%%%%%%%%%%%%%%%%%%%%%%%%%%%%%%%%%%%%%%

\documentclass[12pt]{article} 
\usepackage{alphalph}
\usepackage[utf8]{inputenc}
\usepackage[russian,english]{babel}
\usepackage{titling}
\usepackage{diagbox}
\usepackage{pifont}
\usepackage{amsmath}
\usepackage{graphicx}
\usepackage{enumitem}
\usepackage{amssymb}
\usepackage{physics}
\usepackage{tikz}
\usepackage{mathdots}
\usepackage{yhmath}
\usepackage{cancel}
\usepackage{color, colortbl}
\definecolor{BurntOrange}{rgb}{0.85, 0.6, 0.3}
\definecolor{Gray}{gray}{.5}
\usepackage{siunitx}
\usepackage{array}
\usepackage{multirow}
\usepackage{gensymb}
\usepackage{tabularx}
\usepackage{booktabs}
\usepackage{soul}
\usetikzlibrary{fadings}
\usetikzlibrary{patterns}
\usetikzlibrary{shadows.blur}
\usetikzlibrary{shapes}
\usepackage[super]{nth}
\usepackage{expl3}
\usepackage[version=4]{mhchem}
\usepackage{hpstatement}
\usepackage{rsphrase}
\usepackage{everysel}
\usepackage{ragged2e}
\usepackage{geometry}
\usepackage{fancyhdr}
\usepackage{cancel}
\newcommand{\xmark}{\ding{55}}
\geometry{top=1.0in,bottom=1.0in,left=1.0in,right=1.0in}
\newcommand{\subtitle}[1]{%
  \posttitle{%
    \par\end{center}
    \begin{center}\large#1\end{center}
    \vskip0.5em}%

}
\usepackage{hyperref}
\hypersetup{
colorlinks=true,
linkcolor=blue,
filecolor=magenta,      
urlcolor=blue,
citecolor=blue,
}

\urlstyle{same}


\title{Notes — Week 13}
\date{Period 3}
\author{Michael Brodskiy\\ \small Instructor: Mr. Bradshaw}

% Mathematical Operations:

% Sum: $$\sum_{n=a}^{b} f(x) $$
% Integral: $$\int_{lower}^{upper} f(x) dx$$
% Limit: $$\lim_{x\to\infty} f(x)$$

\begin{document}

\maketitle

\begin{itemize}

  \item \textit{Hernandez v. Texas} (1954) — People of color could now be on juries

    \begin{itemize}

      \item Easiest way to keep a race off of a jury was to forbid voting registration

    \end{itemize}

  \item The Constitution set up a bicameral legislature. Two influences on that decision were:

    \begin{itemize}

      \item Historical: British Parliament

      \item Practical: The “Great Compromise”

    \end{itemize}

  \item The two Houses of Congress are:

    \begin{itemize}

      \item House of Representatives (Lower House, based on population)

      \item Senate (Upper House, two from each state creating equality)

    \end{itemize}

  \item Congressional Terms

    \begin{itemize}

      \item Lasts 2 years, starts each odd numbered year

      \item 1st Congress was in 3/4/1789, today's Congress is the \nth{117}

      \item Sessions, which are one year of a term, begin on January \nth{3} (used to begin in March, but changed by the \nth{20} amendment)

      \item Reps/Senators are in session until the Congress decides to adjourn

      \item Special sessions

        \begin{itemize}

          \item The President can call a special session (has not been done since 1948)

        \end{itemize}

    \end{itemize}

  \item House of Representatives

    \begin{enumerate}

      \item Size

        \begin{itemize}

          \item Determined by Congress (435 since 1911)

          \item Elected by districts

          \item Population of state determines the \# of Reps (increase in Sun Belts, decrease in Frost Belt)

        \end{itemize}

      \item Fixed Terms

        \begin{itemize}

          \item Two years/entire body up for re-election

          \item Term limits ruled unconstitutional in \textit{US Term Limits v. Thornton} (added a qualification)

        \end{itemize}

      \item Qualifications

        \begin{itemize}

          \item 25 years old

          \item Citizen for 7 years

          \item Residency in state

        \end{itemize}

    \end{enumerate}

  \item Senate

    \begin{enumerate}

      \item Size

        \begin{itemize}

          \item 100 members (2 per state)

        \end{itemize}

      \item Term

        \begin{itemize}

          \item 6 years — one-third of the senate is up for reelection every two years

        \end{itemize}

      \item Qualifications

        \begin{itemize}

          \item 30 years old

          \item Citizen for 9 years

          \item Resident of state

        \end{itemize}

    \end{enumerate}

  \item Congressional Elections

    \begin{itemize}

      \item Every member of the House of Representatives seat is up for reelection, while a third of the Senate seats are also up every two years (33 or 34)

      \item About 90\% of incumbents in the House and 85\% in the Senate are usually reelected

    \end{itemize}

  \item Congressional Statistics

    \begin{itemize}

      \item The average age is 56 years

      \item The average member is white, male, protestant, and a lawyer

      \item The average term length is 9.3 years

    \end{itemize}

  \item Incumbent Advantages

    \begin{itemize}

      \item Name Recognition

      \item Franking Privilege (Free Mail)

      \item Full staffs 22 Senate/17 House

      \item Committees in each house for each party that raise money for their members

      \item 90\% of all Political Action Committee money is given to incumbents

      \item Constituent Services

      \item Federal campaign finance laws favorite incumbents over challengers

    \end{itemize}

  \item Political Action Committee (PAC) — A popular term for a political committee organized for the purpose of raising and spending money to elect and defeat candidates

    \begin{itemize}

      \item Anyone, including members of Congress, can start a political action committee

      \item PACs started by politicians are often referred to as Leadership PACs

      \item SuperPACs do not contribute to candidates or parties; they may make independent expenditures in federal races, such as running ads, sending mail, or communicating in other ways (independent expenditure groups)

      \item Roughly 2900 SuperPACs and 8000 PACs exist

    \end{itemize}

  \item The Federal Election Campaign Act of 1971

    \begin{itemize}

      \item The original campaign financing act

      \item Created the Federal Election Commission (FEC)

      \item Created dollar limits that individuals could donate to candidates

      \item Any money given directly to a single candidate is known as a hard dollar

    \end{itemize}

  \item Carey Committees

    \begin{itemize}

      \item Known as Hybrid PACs

      \item One account acts as a political committee, another acts as an account for independent expenditures

    \end{itemize}

  \item 527 Organization

    \begin{itemize}

      \item Nonprofit groups

      \item Have to disclose donors

        \begin{itemize}

          \item Shield identities through loopholes

        \end{itemize}

    \end{itemize}

  \item Redistribution

    \begin{itemize}

      \item Congress has to redistribute the seats after each decennial census

      \item States win or lose seats based on their population gains or losses (CA has 52)

      \item California Population 2000 = 33,871,648

      \item California Population 2019 = 39,539,223

      \item One seat represents roughly 730,000 people per the 2020 US Census

      \item California lost a seat after the last census (used to be 53)

    \end{itemize}

  \item Constituent — A person who makes up a congressional district

  \item Gerrymandering — The redrawing of district boundaries to favor the party in power (mostly legal)

\end{itemize}

\end{document}

