%%%%%%%%%%%%%%%%%%%%%%%%%%%%%%%%%%%%%%%%%%%%%%%%%%%%%%%%%%%%%%%%%%%%%%%%%%%%%%%%%%%%%%%%%%%%%%%%%%%%%%%%%%%%%%%%%%%%%%%%%%%%%%%%%%%%%%%%%%%%%%%%%%%%%%%%%%%%%%%%%%%%%%%%%%%%%%%%%%%%%%%%%%%%
% Written By Michael Brodskiy
% Class: AP US Government
% Instructor: Mr. Bradshaw
%%%%%%%%%%%%%%%%%%%%%%%%%%%%%%%%%%%%%%%%%%%%%%%%%%%%%%%%%%%%%%%%%%%%%%%%%%%%%%%%%%%%%%%%%%%%%%%%%%%%%%%%%%%%%%%%%%%%%%%%%%%%%%%%%%%%%%%%%%%%%%%%%%%%%%%%%%%%%%%%%%%%%%%%%%%%%%%%%%%%%%%%%%%%

\documentclass[12pt]{article} 
\usepackage{alphalph}
\usepackage[utf8]{inputenc}
\usepackage[russian,english]{babel}
\usepackage{titling}
\usepackage{diagbox}
\usepackage{pifont}
\usepackage{amsmath}
\usepackage{graphicx}
\usepackage{enumitem}
\usepackage{amssymb}
\usepackage{physics}
\usepackage{tikz}
\usepackage{mathdots}
\usepackage{yhmath}
\usepackage{cancel}
\usepackage{color, colortbl}
\definecolor{BurntOrange}{rgb}{0.85, 0.6, 0.3}
\definecolor{Gray}{gray}{.5}
\usepackage{siunitx}
\usepackage{array}
\usepackage{multirow}
\usepackage{gensymb}
\usepackage{tabularx}
\usepackage{booktabs}
\usepackage{soul}
\usetikzlibrary{fadings}
\usetikzlibrary{patterns}
\usetikzlibrary{shadows.blur}
\usetikzlibrary{shapes}
\usepackage[super]{nth}
\usepackage{expl3}
\usepackage[version=4]{mhchem}
\usepackage{hpstatement}
\usepackage{rsphrase}
\usepackage{everysel}
\usepackage{ragged2e}
\usepackage{geometry}
\usepackage{fancyhdr}
\usepackage{cancel}
\newcommand{\xmark}{\ding{55}}
\geometry{top=1.0in,bottom=1.0in,left=1.0in,right=1.0in}
\newcommand{\subtitle}[1]{%
  \posttitle{%
    \par\end{center}
    \begin{center}\large#1\end{center}
    \vskip0.5em}%

}
\usepackage{hyperref}
\hypersetup{
colorlinks=true,
linkcolor=blue,
filecolor=magenta,      
urlcolor=blue,
citecolor=blue,
}

\urlstyle{same}


\title{Notes — Week 14}
\date{Period 3}
\author{Michael Brodskiy\\ \small Instructor: Mr. Bradshaw}

% Mathematical Operations:

% Sum: $$\sum_{n=a}^{b} f(x) $$
% Integral: $$\int_{lower}^{upper} f(x) dx$$
% Limit: $$\lim_{x\to\infty} f(x)$$

\begin{document}

\maketitle

\begin{itemize}

  \item The President of the Senate is the vice president

    \begin{itemize}

      \item The vice president only votes when there is a tie

    \end{itemize}

  \item The most powerful members of the senate are:

    \begin{itemize}

      \item Majority leader

      \item Minority leader

      \item President Pro Tempore

    \end{itemize}

  \item Succession:

    \begin{enumerate}

      \item President

      \item Vice President

      \item Speaker of the House

      \item President Pro Tempore

      \item Secretary of the State

    \end{enumerate}

  \item Inherent Powers — Powers which the national government naturally has to represent the country in relation with other countries

    \begin{itemize}

      \item Regulate immigration

      \item Determine citizenship

      \item Declare war

      \item Make a treaty

    \end{itemize}

  \item Other Powers of Congress

    \begin{itemize}

      \item Investigatory Power

      \item Oversight of the Executive Branch

      \item Constitutional Amendments

      \item Electoral Duties

      \item Impeachment

      \item Confirm Appointments

    \end{itemize}

  \item Committees

    \begin{itemize}

      \item Congress has been called “a collection of committees that come together periodically to approve one another's actions”

      \item Congress — Does most of its work in committees (around 95\% of all work)

      \item Most bills don't make it to the floor, but rather die in committees through the method known as “pigeonholing”. “Pigeonholing” a bill is done by burying the bill on the committee calendar.

    \end{itemize}

  \item Committee Structure

    \begin{itemize}

      \item Both houses have organized permanent “standing” committees

      \item Standing Committee numbers

        \begin{itemize}

          \item House: 21

          \item Senate: 20 (68 subcommittees)

        \end{itemize}

      \item For a bill to get to the floor, it must first get a majority vote in at least one standing committee. This is called “reporting out”

    \end{itemize}

  \item Bills can only be introduced by members of Congress, but anyone can write a bill

  \item Standing committees will often assign bills to subcommittees to get more information on a topic

  \item Congress passes between 200 and 400 bills a year (most are naming things)

  \item Only standing committees can vote a bill up or down

  \item When the whole House is together, it is called “Committee of the Whole”

  \item Committees controlled by majority party

  \item All chairpersons are selected by seniority first

  \item The speaker, at any time, can replace any chairperson

  \item According to Article I, Section 7, all tax bills must begin in the House of Representatives

  \item Conference committees are temporary

\end{itemize}

\end{document}

