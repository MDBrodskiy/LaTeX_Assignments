%%%%%%%%%%%%%%%%%%%%%%%%%%%%%%%%%%%%%%%%%%%%%%%%%%%%%%%%%%%%%%%%%%%%%%%%%%%%%%%%%%%%%%%%%%%%%%%%%%%%%%%%%%%%%%%%%%%%%%%%%%%%%%%%%%%%%%%%%%%%%%%%%%%%%%%%%%%%%%%%%%%%%%%%%%%%%%%%%%%%%%%%%%%%
% Written By Michael Brodskiy
% Class: AP US Government
% Instructor: Mr. Bradshaw
%%%%%%%%%%%%%%%%%%%%%%%%%%%%%%%%%%%%%%%%%%%%%%%%%%%%%%%%%%%%%%%%%%%%%%%%%%%%%%%%%%%%%%%%%%%%%%%%%%%%%%%%%%%%%%%%%%%%%%%%%%%%%%%%%%%%%%%%%%%%%%%%%%%%%%%%%%%%%%%%%%%%%%%%%%%%%%%%%%%%%%%%%%%%

\documentclass[12pt]{article} 
\usepackage{alphalph}
\usepackage[utf8]{inputenc}
\usepackage[russian,english]{babel}
\usepackage{titling}
\usepackage{diagbox}
\usepackage{pifont}
\usepackage{amsmath}
\usepackage{graphicx}
\usepackage{enumitem}
\usepackage{amssymb}
\usepackage{physics}
\usepackage{tikz}
\usepackage{mathdots}
\usepackage{yhmath}
\usepackage{cancel}
\usepackage{color, colortbl}
\definecolor{BurntOrange}{rgb}{0.85, 0.6, 0.3}
\definecolor{Gray}{gray}{.5}
\usepackage{siunitx}
\usepackage{array}
\usepackage{multirow}
\usepackage{gensymb}
\usepackage{tabularx}
\usepackage{booktabs}
\usepackage{soul}
\usetikzlibrary{fadings}
\usetikzlibrary{patterns}
\usetikzlibrary{shadows.blur}
\usetikzlibrary{shapes}
\usepackage[super]{nth}
\usepackage{expl3}
\usepackage[version=4]{mhchem}
\usepackage{hpstatement}
\usepackage{rsphrase}
\usepackage{everysel}
\usepackage{ragged2e}
\usepackage{geometry}
\usepackage{fancyhdr}
\usepackage{cancel}
\newcommand{\xmark}{\ding{55}}
\geometry{top=1.0in,bottom=1.0in,left=1.0in,right=1.0in}
\newcommand{\subtitle}[1]{%
  \posttitle{%
    \par\end{center}
    \begin{center}\large#1\end{center}
    \vskip0.5em}%

}
\usepackage{hyperref}
\hypersetup{
colorlinks=true,
linkcolor=blue,
filecolor=magenta,      
urlcolor=blue,
citecolor=blue,
}

\urlstyle{same}


\title{Notes — Week 4}
\date{Period 3}
\author{Michael Brodskiy\\ \small Instructor: Mr. Bradshaw}

% Mathematical Operations:

% Sum: $$\sum_{n=a}^{b} f(x) $$
% Integral: $$\int_{lower}^{upper} f(x) dx$$
% Limit: $$\lim_{x\to\infty} f(x)$$

\begin{document}

\maketitle

\begin{itemize}

  \item Virginia Plan represented the larger states (population-based); New Jersey Plan represented the smaller states (2 representatives per state)

  \item Federalism is separation of powers between the national government and state governments

  \item Check page 89 for a list of words necessary to know

  \item Read article 5 of the US constitution

  \item Lexington and Concord — April 19, 1775; called the “shot heard around the world”

    \begin{itemize}

      \item British attacked Concord to get weapons

      \item Paul Revere and others spread the message of this attack

    \end{itemize}

  \item 60\% of the clergy supported revolution and fought; about 40\% did not

  \item African-Americans were minutemen too

    \begin{itemize}

      \item Peter Salem is said to have been the one to fire the shot that killed British Maj. Pitcairn

      \item Oliver Cromwell fought alongside George Washington at the battles of: Trenton, Princeton, Brandywine, Monmouth, and Yorktown

      \item Very few African-Americans from the South participated on the rebel side and gained their freedom

    \end{itemize}

  \item Women had rifles too

  \item 20\% of George Washington's troops in the north were African-Americans

  \item Between five and ten thousand African-Americans fought on the American side

  \item At least fifteen thousand fought on the British side

  \item Northern colonies offered freedom to African-Americans to fight on the rebel side

  \item There was slavery in the thirteen colonies because the king mandated it

  \item Colonists wrote to the king to stop the importation of slaves, and to abolish all slavery

  \item George Mason wrote the Virginia Declaration of Rights

  \item Articles of Confederation

    \begin{itemize}

      \item All states had to agree to amend it

      \item One vote per state

      \item Weaknesses:

        \begin{itemize}

          \item Could not levy taxes

          \item No executive branch, so no law enforcement

          \item No (national) judicial branch

          \item No (national) army

          \item No (national) navy

        \end{itemize}

      \item Founding fathers were afraid of a standing army during peacetime — the “failures” of the Articles of Confederation were a result of their own experiences

      \item “Union” is used 9 times; “perpetual union” is used 7 times

      \item Union preceded the constitution

      \item Created 13 independent countries

      \item Referred to as a “firm league of friendship”

      \item Only free inhabitants got rights; this excluded slaves, women, paupers, vagabonds, and fugitives

      \item The states pay the salary of delegates; therefore, states could fire delegates

    \end{itemize}

\end{itemize}

\end{document}

