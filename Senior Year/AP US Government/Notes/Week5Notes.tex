%%%%%%%%%%%%%%%%%%%%%%%%%%%%%%%%%%%%%%%%%%%%%%%%%%%%%%%%%%%%%%%%%%%%%%%%%%%%%%%%%%%%%%%%%%%%%%%%%%%%%%%%%%%%%%%%%%%%%%%%%%%%%%%%%%%%%%%%%%%%%%%%%%%%%%%%%%%%%%%%%%%%%%%%%%%%%%%%%%%%%%%%%%%%
% Written By Michael Brodskiy
% Class: AP US Government
% Instructor: Mr. Bradshaw
%%%%%%%%%%%%%%%%%%%%%%%%%%%%%%%%%%%%%%%%%%%%%%%%%%%%%%%%%%%%%%%%%%%%%%%%%%%%%%%%%%%%%%%%%%%%%%%%%%%%%%%%%%%%%%%%%%%%%%%%%%%%%%%%%%%%%%%%%%%%%%%%%%%%%%%%%%%%%%%%%%%%%%%%%%%%%%%%%%%%%%%%%%%%

\documentclass[12pt]{article} 
\usepackage{alphalph}
\usepackage[utf8]{inputenc}
\usepackage[russian,english]{babel}
\usepackage{titling}
\usepackage{diagbox}
\usepackage{pifont}
\usepackage{amsmath}
\usepackage{graphicx}
\usepackage{enumitem}
\usepackage{amssymb}
\usepackage{physics}
\usepackage{tikz}
\usepackage{mathdots}
\usepackage{yhmath}
\usepackage{cancel}
\usepackage{color, colortbl}
\definecolor{BurntOrange}{rgb}{0.85, 0.6, 0.3}
\definecolor{Gray}{gray}{.5}
\usepackage{siunitx}
\usepackage{array}
\usepackage{multirow}
\usepackage{gensymb}
\usepackage{tabularx}
\usepackage{booktabs}
\usepackage{soul}
\usetikzlibrary{fadings}
\usetikzlibrary{patterns}
\usetikzlibrary{shadows.blur}
\usetikzlibrary{shapes}
\usepackage[super]{nth}
\usepackage{expl3}
\usepackage[version=4]{mhchem}
\usepackage{hpstatement}
\usepackage{rsphrase}
\usepackage{everysel}
\usepackage{ragged2e}
\usepackage{geometry}
\usepackage{fancyhdr}
\usepackage{cancel}
\newcommand{\xmark}{\ding{55}}
\geometry{top=1.0in,bottom=1.0in,left=1.0in,right=1.0in}
\newcommand{\subtitle}[1]{%
  \posttitle{%
    \par\end{center}
    \begin{center}\large#1\end{center}
    \vskip0.5em}%

}
\usepackage{hyperref}
\hypersetup{
colorlinks=true,
linkcolor=blue,
filecolor=magenta,      
urlcolor=blue,
citecolor=blue,
}

\urlstyle{same}


\title{Notes — Week 5}
\date{Period 3}
\author{Michael Brodskiy\\ \small Instructor: Mr. Bradshaw}

% Mathematical Operations:

% Sum: $$\sum_{n=a}^{b} f(x) $$
% Integral: $$\int_{lower}^{upper} f(x) dx$$
% Limit: $$\lim_{x\to\infty} f(x)$$

\begin{document}

\maketitle

\begin{itemize}

  \item Under the Articles of Confederation, there was no regulated interstate trade, which resulted in many going hungry as states refused to pay each others tariffs

  \item Article I Section 8 is an example of “Positive Law,” which lists the powers granted to Congress (also called the enumerated powers)

  \item Article I Section 9 is an example of “Negative Law,” which lists the powers denied to Congress

  \item The writ of Habeas Corpus comes from the Magna Carta

  \item No Bill of Attainder — A law passed by Congress to take away a persons life, liberty, or property by majority vote only

  \item No Ex Post Facto laws — A bill making something that was legal now illegal after the fact


  \item Amendments — Congress proposes, States ratify

    \begin{center}
      \begin{tabular}{|p{.45\textwidth}|p{.45\textwidth}|}
        \hline
        \begin{center}Propose\end{center} & \begin{center}Approve/Ratify\end{center}\\
        \hline
        Amendments are proposed by two-thirds of both the House of Reps.\ and the Senate & Three-fourths of the state legislatures\newline ($\frac{3}{4}$ of 50 $\rightarrow 38$)\\
        \hline
        Application of two-thirds of the state legislatures to Congress to call a convention for proposing amendments (never used) & Conventions in three-fourths of the States (used once — twenty-first amendment)\\
        \hline
      \end{tabular}
    \end{center}

  \item There are no term limits at the federal level

  \item Article V is a great example of federalism

  \item A super majority for ratification is necessary because only the most important issues should be amended

  \item The Articles of Confederation created nothing more than a league of friendship. Some weaknesses were:

    \begin{itemize}
        
      \item Could not levy taxes

      \item Could not regulate commerce
        
      \item Had an army in name only (no real army) — Also, no navy

      \item No national judicial or executive branch to interpret and enforce laws, respectively

      \item Amendment required all thirteen states to agree

    \end{itemize}

  \item The Articles of Confederation created 13 countries (each state was its own sovereign, independent nation) — Simple majority was 7/13 and 9/13 was a super majority (laws needed 9/13 votes to be passed)

  \item Per the Articles of Confederation, delegates were picked and paid for by the state legislature (which meant delegates acted more as a United Nations, where each state acted in its own interest)

  \item The Continental Congress had only one chamber, which meant it was fast and efficient

\end{itemize}

\end{document}

