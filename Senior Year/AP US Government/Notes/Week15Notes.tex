%%%%%%%%%%%%%%%%%%%%%%%%%%%%%%%%%%%%%%%%%%%%%%%%%%%%%%%%%%%%%%%%%%%%%%%%%%%%%%%%%%%%%%%%%%%%%%%%%%%%%%%%%%%%%%%%%%%%%%%%%%%%%%%%%%%%%%%%%%%%%%%%%%%%%%%%%%%%%%%%%%%%%%%%%%%%%%%%%%%%%%%%%%%%
% Written By Michael Brodskiy
% Class: AP US Government
% Instructor: Mr. Bradshaw
%%%%%%%%%%%%%%%%%%%%%%%%%%%%%%%%%%%%%%%%%%%%%%%%%%%%%%%%%%%%%%%%%%%%%%%%%%%%%%%%%%%%%%%%%%%%%%%%%%%%%%%%%%%%%%%%%%%%%%%%%%%%%%%%%%%%%%%%%%%%%%%%%%%%%%%%%%%%%%%%%%%%%%%%%%%%%%%%%%%%%%%%%%%%

\documentclass[12pt]{article} 
\usepackage{alphalph}
\usepackage[utf8]{inputenc}
\usepackage[russian,english]{babel}
\usepackage{titling}
\usepackage{diagbox}
\usepackage{pifont}
\usepackage{amsmath}
\usepackage{graphicx}
\usepackage{enumitem}
\usepackage{amssymb}
\usepackage{physics}
\usepackage{tikz}
\usepackage{mathdots}
\usepackage{yhmath}
\usepackage{cancel}
\usepackage{color, colortbl}
\definecolor{BurntOrange}{rgb}{0.85, 0.6, 0.3}
\definecolor{Gray}{gray}{.5}
\usepackage{siunitx}
\usepackage{array}
\usepackage{multirow}
\usepackage{gensymb}
\usepackage{tabularx}
\usepackage{booktabs}
\usepackage{soul}
\usetikzlibrary{fadings}
\usetikzlibrary{patterns}
\usetikzlibrary{shadows.blur}
\usetikzlibrary{shapes}
\usepackage[super]{nth}
\usepackage{expl3}
\usepackage[version=4]{mhchem}
\usepackage{hpstatement}
\usepackage{rsphrase}
\usepackage{everysel}
\usepackage{ragged2e}
\usepackage{geometry}
\usepackage{fancyhdr}
\usepackage{cancel}
\newcommand{\xmark}{\ding{55}}
\geometry{top=1.0in,bottom=1.0in,left=1.0in,right=1.0in}
\newcommand{\subtitle}[1]{%
  \posttitle{%
    \par\end{center}
    \begin{center}\large#1\end{center}
    \vskip0.5em}%

}
\usepackage{hyperref}
\hypersetup{
colorlinks=true,
linkcolor=blue,
filecolor=magenta,      
urlcolor=blue,
citecolor=blue,
}

\urlstyle{same}


\title{Notes — Week 15}
\date{Period 3}
\author{Michael Brodskiy\\ \small Instructor: Mr. Bradshaw}

% Mathematical Operations:

% Sum: $$\sum_{n=a}^{b} f(x) $$
% Integral: $$\int_{lower}^{upper} f(x) dx$$
% Limit: $$\lim_{x\to\infty} f(x)$$

\begin{document}

\maketitle

\begin{itemize}

  \item Justice Jackson nominated to Supreme Court by Biden

  \item Nominating Process for Judges

    \begin{enumerate}

      \item President nominates a judge

      \item Name sent to Senate Judiciary Committee

      \item Senate Judiciary Committee holds hearings with nominee\footnote{Political interest groups work here to influence}

      \item Senate Judiciary Committee votes “yes” or “no” on nominee

      \item If nominee passes\footnote{Majority vote needed to get out of committee} committee vote, his/her name is placed on Senate Floor calendar\footnote{51 votes to pass or 50 if filibuster is in action} for final vote

    \end{enumerate}

  \item All courts are named after the Chief Justice

  \item United States Parallel Judiciary System (Federalism)

    \begin{enumerate}

      \item Federal

        \begin{enumerate}

          \item First — Federal District Courts (Trial Courts) — Hear cases involving federal laws

          \item Second — Federal Court of Appeals — Hears cases involving federal laws and constitutional questions

          \item Last — Supreme Court of the United States

        \end{enumerate}

      \item State

        \begin{enumerate}

          \item First — State District Courts (Trial Courts) — Hear cases involving state laws

          \item Second — State Court of Appeals

          \item Last — Supreme Court of a State

        \end{enumerate}

      \item In the United States, we run a Parallel Court system. We have Federal Courts \& State Courts that work in parallel (cases can cross from state to federal, but not the other way)

      \item There are 11 Court of Appeals circuits

      \item Congress makes all of the courts below the Supreme Court

    \end{enumerate}

    \begin{center} \end{center}

    \hspace{-12pt}\begin{tabular}{r}

      \vspace{-10pt}\\
      Article III\Bigg\{\\
        \vspace{-10pt}\\
        Article I\big\{\\

        \end{tabular} \hspace{-10pt}\begin{tabular}{|c|}

      \hline
      US Supreme Court (9 Justices)\\
      \hline
      US Court of Appeals (11 Appellate + 2 Specialty)\\
      \hline
      Federal District Courts, Trial Courts (94 in total)\\
      \hline
      US Tax, Trade, Claims, Bankrupt Courts\\ 
      \hline

    \end{tabular} \hspace{-9pt} \begin{tabular}{l}

      $\longleftarrow$ 80-100 cases/year\\
      $\longleftarrow$ 8k-10k cases/year\\
      \\
      $\longleftarrow$ Specialty Courts\\

    \end{tabular}

    \begin{center} Article I: Tenure is determined by Congress\\ Article III: Tenure is for Life\end{center}\begin{center}\end{center}

  \item Briefs are legal written arguments

  \item Congress determines the amount of justices

  \item Judicial Restraint and Judicial Activism

    \begin{itemize}

      \item Judicial Restraint — A judicial philosophy whereby judges adhere closely to the constitution, statutes, and precedents in reaching their decisions. Limits the courts abilities to craft “solutions” that should be rendered by the political branches of the federal or state governments

      \item Judicial Activism — A judicial philosophy whereby judges interpret existing laws and precedents loosely and interject their own values in court decisions to legislate solutions from the bench and not from the elected branches of government

      \item According to many scholars, judicial activism may be conservative or liberal

    \end{itemize}

  \item Around 44\% of cases end in unanimous decision

  \item 80-120 cases take place per year. Only about 5 are highlighted by media, as these are usually close calls (with 5-4 votes). This makes the court appear much more divided than it really is.

  \item Original Jurisdiction — The legal authority to first hear a case (Trial Court)

  \item Appellate Jurisdiction — The legal authority to override a previous case by appeal (Appellate Court)

  \item The Supreme Court holds both jurisdictions

  \item There are eight to ten thousand appeals to the Supreme Court every year. Only 80 to 120 are heard. These cases are both criminal and civil.

  \item Civil cases deal with the seventh amendment

  \item The “rule of four” means that four of nine justices may choose to take into oral argument some case (four out of nine are required to release a \textit{writ of certiorari})

  \item Floor calendars are controlled by the Majority Leader (currently, Chuck Schumer D-NY $\longrightarrow$ rules committee) in the Senate and the Speaker of the House (currently, Nancy Pelosi D-CA $\longrightarrow$ rules committee) in the House

  \item Judges checked by Congress through impeachment

  \item Advisory opinions are not allowed by the Supreme Court

  \item Congress created the courts and their jurisdiction, meaning it can change them (with the exception of the Supreme Court)

  \item Supreme Court Justices

    \begin{itemize}

      \item 9 Justices

        \begin{itemize}

          \item Chief Justice

          \item 8 Associate Justices

        \end{itemize}

      \item Congress has the power to change the number

        \begin{itemize}

          \item Number of Justices has varied from 5 to 10

          \item Nine justices since 1869

          \item 1937 Roosevelt tried to increase number and Congress turned him down. This is known as the “Court Packing Caper”

        \end{itemize}

    \end{itemize}

  \item The \textit{writ of certiorari} is the quickest ticket to the Supreme Court

  \item \textit{Amicus Curiae} (friends of the court) are outsiders who may write statements supporting a certain side in a case

  \item In \textit{Wolf v. Colorado}, the Fourth Amendment was applied to the states for the first time

    \begin{itemize}

      \item Because the exclusionary rule was not in the constitution, it still did not apply to states

      \item As it evolved, the “Good Faith Exception” was added. This stated that, if a warrant was used in an incorrect way, evidence could still be used if the use of the warrant was determined to be in good faith

    \end{itemize}

  \item The “Due Process Revolution” was led by the Warren Court

  \item The Court must wait for an appeal following a trial to reach the Supreme Court

  \item Only a person in “standing” may be heard by the Supreme Court

  \item The police developed the third degree — this meant a severe interrogation (psychological + physical)

\end{itemize}

\end{document}

