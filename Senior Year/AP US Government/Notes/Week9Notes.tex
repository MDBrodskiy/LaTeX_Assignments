%%%%%%%%%%%%%%%%%%%%%%%%%%%%%%%%%%%%%%%%%%%%%%%%%%%%%%%%%%%%%%%%%%%%%%%%%%%%%%%%%%%%%%%%%%%%%%%%%%%%%%%%%%%%%%%%%%%%%%%%%%%%%%%%%%%%%%%%%%%%%%%%%%%%%%%%%%%%%%%%%%%%%%%%%%%%%%%%%%%%%%%%%%%%
% Written By Michael Brodskiy
% Class: AP US Government
% Instructor: Mr. Bradshaw
%%%%%%%%%%%%%%%%%%%%%%%%%%%%%%%%%%%%%%%%%%%%%%%%%%%%%%%%%%%%%%%%%%%%%%%%%%%%%%%%%%%%%%%%%%%%%%%%%%%%%%%%%%%%%%%%%%%%%%%%%%%%%%%%%%%%%%%%%%%%%%%%%%%%%%%%%%%%%%%%%%%%%%%%%%%%%%%%%%%%%%%%%%%%

\documentclass[12pt]{article} 
\usepackage{alphalph}
\usepackage[utf8]{inputenc}
\usepackage[russian,english]{babel}
\usepackage{titling}
\usepackage{diagbox}
\usepackage{pifont}
\usepackage{amsmath}
\usepackage{graphicx}
\usepackage{enumitem}
\usepackage{amssymb}
\usepackage{physics}
\usepackage{tikz}
\usepackage{mathdots}
\usepackage{yhmath}
\usepackage{cancel}
\usepackage{color, colortbl}
\definecolor{BurntOrange}{rgb}{0.85, 0.6, 0.3}
\definecolor{Gray}{gray}{.5}
\usepackage{siunitx}
\usepackage{array}
\usepackage{multirow}
\usepackage{gensymb}
\usepackage{tabularx}
\usepackage{booktabs}
\usepackage{soul}
\usetikzlibrary{fadings}
\usetikzlibrary{patterns}
\usetikzlibrary{shadows.blur}
\usetikzlibrary{shapes}
\usepackage[super]{nth}
\usepackage{expl3}
\usepackage[version=4]{mhchem}
\usepackage{hpstatement}
\usepackage{rsphrase}
\usepackage{everysel}
\usepackage{ragged2e}
\usepackage{geometry}
\usepackage{fancyhdr}
\usepackage{cancel}
\newcommand{\xmark}{\ding{55}}
\geometry{top=1.0in,bottom=1.0in,left=1.0in,right=1.0in}
\newcommand{\subtitle}[1]{%
  \posttitle{%
    \par\end{center}
    \begin{center}\large#1\end{center}
    \vskip0.5em}%

}
\usepackage{hyperref}
\hypersetup{
colorlinks=true,
linkcolor=blue,
filecolor=magenta,      
urlcolor=blue,
citecolor=blue,
}

\urlstyle{same}


\title{Notes — Week 9}
\date{Period 3}
\author{Michael Brodskiy\\ \small Instructor: Mr. Bradshaw}

% Mathematical Operations:

% Sum: $$\sum_{n=a}^{b} f(x) $$
% Integral: $$\int_{lower}^{upper} f(x) dx$$
% Limit: $$\lim_{x\to\infty} f(x)$$

\begin{document}

\maketitle

\begin{itemize}

  \item There are 435 members of the House of Representatives

  \item Great Compromise — Created a House of Representatives based on population

  \item Created a senate based on equality (2 per state = 100)

  \item Congress contains 535 members (100 from senate + 435 from HoR)

  \item Minimum majority to win in the electoral college is 270

  \item A census occurs every 10 years, at the beginning of a decade (2020, 2030, 2040, etc.)

  \item Ordered Liberty:

    \begin{enumerate}

      \item Separation of Powers [Branches]

      \item Checks + Balances [Branches + Federalism]

      \item Federalism: Division of Power between Federal \& State Government

      \item Popular Sovereignty [Majority Rule w/ Minority Rights]

      \item Judicial Review

      \item Limited Government [Positive Law (Art. 1, Sect. 8) + Negative Law (Art. 1, Sect. 9/10)]

    \end{enumerate}

  \item Congress raises all of its money through taxes

  \item Congress controls fiscal policy; the Federal Reserve controls monetary policy

  \item Congress has the right to borrow in the name of the United States

  \item Congress has used the commerce clause to grow its powers exponentially

    \begin{itemize}

      \item \textit{United States v. Lopez} — Congress had exceeded its commerce clause power by prohibiting guns in a school zone

    \end{itemize}

\end{itemize}

\end{document}

