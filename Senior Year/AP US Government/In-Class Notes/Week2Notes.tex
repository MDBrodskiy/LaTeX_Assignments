%%%%%%%%%%%%%%%%%%%%%%%%%%%%%%%%%%%%%%%%%%%%%%%%%%%%%%%%%%%%%%%%%%%%%%%%%%%%%%%%%%%%%%%%%%%%%%%%%%%%%%%%%%%%%%%%%%%%%%%%%%%%%%%%%%%%%%%%%%%%%%%%%%%%%%%%%%%%%%%%%%%%%%%%%%%%%%%%%%%%%%%%%%%%
% Written By Michael Brodskiy
% Class: AP US Government
% Instructor: Mr. Bradshaw
%%%%%%%%%%%%%%%%%%%%%%%%%%%%%%%%%%%%%%%%%%%%%%%%%%%%%%%%%%%%%%%%%%%%%%%%%%%%%%%%%%%%%%%%%%%%%%%%%%%%%%%%%%%%%%%%%%%%%%%%%%%%%%%%%%%%%%%%%%%%%%%%%%%%%%%%%%%%%%%%%%%%%%%%%%%%%%%%%%%%%%%%%%%%

\documentclass[12pt]{article} 
\usepackage{alphalph}
\usepackage[utf8]{inputenc}
\usepackage[russian,english]{babel}
\usepackage{titling}
\usepackage{amsmath}
\usepackage{graphicx}
\usepackage{enumitem}
\usepackage{amssymb}
\usepackage{physics}
\usepackage{tikz}
\usepackage{mathdots}
\usepackage{yhmath}
\usepackage{cancel}
\usepackage{color}
\usepackage{siunitx}
\usepackage{array}
\usepackage{multirow}
\usepackage{gensymb}
\usepackage{tabularx}
\usepackage{booktabs}
\usepackage{soul}
\usetikzlibrary{fadings}
\usetikzlibrary{patterns}
\usetikzlibrary{shadows.blur}
\usetikzlibrary{shapes}
\usepackage[super]{nth}
\usepackage{expl3}
\usepackage[version=4]{mhchem}
\usepackage{hpstatement}
\usepackage{rsphrase}
\usepackage{everysel}
\usepackage{ragged2e}
\usepackage{geometry}
\usepackage{fancyhdr}
\usepackage{cancel}
\geometry{top=1.0in,bottom=1.0in,left=1.0in,right=1.0in}
\newcommand{\subtitle}[1]{%
  \posttitle{%
    \par\end{center}
    \begin{center}\large#1\end{center}
    \vskip0.5em}%

}
\usepackage{hyperref}
\hypersetup{
colorlinks=true,
linkcolor=blue,
filecolor=magenta,      
urlcolor=blue,
citecolor=blue,
}

\urlstyle{same}


\title{Notes — Week 2}
\date{Period 3}
\author{Michael Brodskiy\\ \small Instructor: Mr. Bradshaw}

% Mathematical Operations:

% Sum: $$\sum_{n=a}^{b} f(x) $$
% Integral: $$\int_{lower}^{upper} f(x) dx$$
% Limit: $$\lim_{x\to\infty} f(x)$$

\begin{document}

\maketitle

\begin{enumerate}

  \item What are the three purposes of government?

    \begin{justify}

      Protect its citizens, maintain order, and, more recently, preserve equality

    \end{justify}

    \begin{justify}

      “Power tends to corrupt and absolute power corrupts absolutely. Great men are almost always bad men, even when they exercise influence and not authority; still more when you superadd the tendency of the certainty of corruption by authority” — Lord Acton, 1887

    \end{justify}

  \item Preamble to the Constitution:

    \begin{justify}

      “We the People of the United States, in Order to form a more perfect Union, establish Justice, ensure domestic Tranquility, provide for the common defense, promote the general welfare, and secure the Blessings of Liberty to ourselves and our Posterity, do ordain and establish this Constitution for the United States of America”

    \end{justify}

    \begin{itemize}

      \item Posterity — \textit{All} future generations (for generations to come)

      \item All of the actions are ongoing (i.e.\ establishing justice, ensuring tranquility, etc.)

      \item There are no bars or limits, and change is always allowed. Men and women could rule themselves

    \end{itemize}

  \item Theme $\longrightarrow$ change:

    \begin{enumerate}

      \item Original — How were things originally?

      \item \underline{How} did they change? (The mechanism for change, i.e.\ court decision, law, election, etc.)

      \item \underline{Why} did they change? (Politics)

      \item B \& C together make up the \textit{explain} verb on FRQs

    \end{enumerate}

    \newpage

  \item What makes government (Nation-States/Countries/States \underline{not} Nations) legitimate?

    \begin{enumerate}

      \item Borders usually determined by treaty

      \item Ordered \& stable population

      \item Code of Laws

      \item Natural sovereignty (national)

      \item Nations don't control their \underline{borders} \& are \underline{not sovereign}

    \end{enumerate}

  \item Ancient forms of democratic rule come from Greece \& Rome

    \begin{itemize}

      \item Most common form of government known to humans is autocracy — Rule by a single person

    \end{itemize}

  \item Police Power: Laws \& regulations that promote health, safety, welfare, and morals. 90\% of police power is held by states (in a federalist system)

    \begin{itemize}

      \item Usually challenged in a court for women \& LGBT communities

      \item Used to maintain “social order”

      \item Court cases (most) are going to arise from state courts

    \end{itemize}

  \item Dolores Huerta Questions:

    \begin{enumerate}

      \item Where was Dolores Huerta born and raised?

        \begin{justify}

          Huerta was born in Dawson, New Mexico, and, following her parents divorce, moved to Stockton, California.

        \end{justify}

      \item What union was Ms. Dolores Huerta part of and what was her role in the union?

        \begin{justify}

          Huerta was co-founder of the United Farm Workers Association.

        \end{justify}

      \item What were some of the activities of Ms. Huerta's union?

        \begin{justify}

          Huerta, through her union, helped organize strikes, negotiated contracts, advocated for improved working conditions, and fought for unemployment benefits for farm workers.

        \end{justify}

      \item How did Ms. Huerta help people in the 1990-2000s? Explain.

        \begin{justify}

          Huerta helped elect more Latinos to political office, as well as aided in women's issues.

        \end{justify}

      \item What awards has Ms. Huerta been awarded in her lifetime according to the article?

        \begin{justify}

          Huerta received the Eleanor Roosevelt Human Rights Award in 1998, and the Presidential Medal of Freedom in 2012.

        \end{justify}

    \end{enumerate}

  \item Cesar Chavez Questions:

    \begin{enumerate}

      \item Where was Cesar Chavez born and raised?

        \begin{justify}

          Cesar Chavez was born on March 31, 1927, near Yuma, Arizona. Due to the economic repercussions of the Great Depression, Chavez's family was unable to pay for their land, and, as such, they moved to California for new opportunities.

        \end{justify}

      \item What union was Mr. Chavez part of and what was his role in the union?

        \begin{justify}

          Chavez created the United Farm Workers of America union. As the founder, he helped organize strikes and fight for the rights of farm workers.

        \end{justify}

      \item What were some of the goals of Mr. Chavez' union?

        \begin{justify}

          Chavez, with his union, fought for the rights of farm workers, and laborers in general. He was able to get farm workers higher wages, medical insurance, unemployment insurance, and many other benefits.

        \end{justify}

      \item What is “La Huelga”, to Mr. Chavez' union?

        \begin{justify}

          La Huelga was a five-year period during which Chavez led a strike and became known to much of the world.

        \end{justify}

      \item What was the Agricultural Labor Relations Act and what did it do?

        \begin{justify}

          This Act established collective bargaining rights to farm workers. Additionally, it granted union rights.

        \end{justify}

      \item Who was Pat Brown?

        \begin{justify}

          Pat Brown was a Democrat who marched with the farm workers, prior to his election as Governor in 1974. He helped pass the Agricultural Labor Relations Act.

        \end{justify}

    \end{enumerate}

\end{enumerate}

\end{document}

