%%%%%%%%%%%%%%%%%%%%%%%%%%%%%%%%%%%%%%%%%%%%%%%%%%%%%%%%%%%%%%%%%%%%%%%%%%%%%%%%%%%%%%%%%%%%%%%%%%%%%%%%%%%%%%%%%%%%%%%%%%%%%%%%%%%%%%%%%%%%%%%%%%%%%%%%%%%%%%%%%%%%%%%%%%%%%%%%%%%%%%%%%%%%
% Written By Michael Brodskiy
% Class: AP US Government
% Instructor: Mr. Bradshaw
%%%%%%%%%%%%%%%%%%%%%%%%%%%%%%%%%%%%%%%%%%%%%%%%%%%%%%%%%%%%%%%%%%%%%%%%%%%%%%%%%%%%%%%%%%%%%%%%%%%%%%%%%%%%%%%%%%%%%%%%%%%%%%%%%%%%%%%%%%%%%%%%%%%%%%%%%%%%%%%%%%%%%%%%%%%%%%%%%%%%%%%%%%%%

\documentclass[12pt]{article} 
\usepackage{alphalph}
\usepackage[utf8]{inputenc}
\usepackage[russian,english]{babel}
\usepackage{titling}
\usepackage{amsmath}
\usepackage{graphicx}
\usepackage{enumitem}
\usepackage{amssymb}
\usepackage{physics}
\usepackage{tikz}
\usepackage{mathdots}
\usepackage{yhmath}
\usepackage{cancel}
\usepackage{color}
\usepackage{siunitx}
\usepackage{array}
\usepackage{multirow}
\usepackage{gensymb}
\usepackage{tabularx}
\usepackage{booktabs}
\usepackage{soul}
\usetikzlibrary{fadings}
\usetikzlibrary{patterns}
\usetikzlibrary{shadows.blur}
\usetikzlibrary{shapes}
\usepackage[super]{nth}
\usepackage{expl3}
\usepackage[version=4]{mhchem}
\usepackage{hpstatement}
\usepackage{rsphrase}
\usepackage{everysel}
\usepackage{ragged2e}
\usepackage{geometry}
\usepackage{fancyhdr}
\usepackage{cancel}
\geometry{top=1.0in,bottom=1.0in,left=1.0in,right=1.0in}
\newcommand{\subtitle}[1]{%
  \posttitle{%
    \par\end{center}
    \begin{center}\large#1\end{center}
    \vskip0.5em}%

}
\usepackage{hyperref}
\hypersetup{
colorlinks=true,
linkcolor=blue,
filecolor=magenta,      
urlcolor=blue,
citecolor=blue,
}

\urlstyle{same}


\title{Notes — Week 2}
\date{Period 3}
\author{Michael Brodskiy\\ \small Instructor: Mr. Bradshaw}

% Mathematical Operations:

% Sum: $$\sum_{n=a}^{b} f(x) $$
% Integral: $$\int_{lower}^{upper} f(x) dx$$
% Limit: $$\lim_{x\to\infty} f(x)$$

\begin{document}

\maketitle

\begin{enumerate}

  \item What are the three purposes of government?

    \begin{justify}

      Protect its citizens, maintain order, and, more recently, preserve equality

    \end{justify}

    \begin{justify}

      “Power tends to corrupt and absolute power corrupts absolutely. Great men are almost always bad men, even when they exercise influence and not authority; still more when you superadd the tendency of the certainty of corruption by authority” — Lord Acton, 1887

    \end{justify}

  \item Preamble to the Constitution:

    \begin{justify}

      “We the People of the United States, in Order to form a more perfect Union, establish Justice, ensure domestic Tranquility, provide for the common defense, promote the general welfare, and secure the Blessings of Liberty to ourselves and our Posterity, do ordain and establish this Constitution for the United States of America”

    \end{justify}

    \begin{itemize}

      \item Posterity — \textit{All} future generations (for generations to come)

      \item All of the actions are ongoing (i.e.\ establishing justice, ensuring tranquility, etc.)

      \item There are no bars or limits, and change is always allowed. Men and women could rule themselves

    \end{itemize}

  \item Theme $\longrightarrow$ change:

    \begin{itemize}

      \item Original — How were things originally?

      \item \underline{How} did they change? (The mechanism for change, i.e.\ court decision, law, election, etc.)

      \item \underline{Why} did they change?

      \item B \& C together make up the \textit{explain} verb on FRQs

    \end{itemize}

  \item What makes government legitimate?

\end{enumerate}

\end{document}

