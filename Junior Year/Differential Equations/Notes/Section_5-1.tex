%%%%%%%%%%%%%%%%%%%%%%%%%%%%%%%%%%%%%%%%%%%%%%%%%%%%%%%%%%%%%%%%%%%%%%%%%%%%%%%%%%%%%%%%%%%%%%%%%%%%%%%%%%%%%%%%%%%%%%%%%%%%%%%%%%%%%%%%%%%%%%%%%%%%%%%%%%%%%%%%%%%%%%%%%%%%%%%%%%%%%%%%%%%%
% Written By Michael Brodskiy
% Class: Differential Equations (MATH-294)
% Professor: M. Shah
%%%%%%%%%%%%%%%%%%%%%%%%%%%%%%%%%%%%%%%%%%%%%%%%%%%%%%%%%%%%%%%%%%%%%%%%%%%%%%%%%%%%%%%%%%%%%%%%%%%%%%%%%%%%%%%%%%%%%%%%%%%%%%%%%%%%%%%%%%%%%%%%%%%%%%%%%%%%%%%%%%%%%%%%%%%%%%%%%%%%%%%%%%%%

\documentclass[12pt]{article} 
\usepackage{alphalph}
\usepackage[utf8]{inputenc}
\usepackage[russian,english]{babel}
\usepackage{titling}
\usepackage{amsmath}
\usepackage{graphicx}
\usepackage{enumitem}
\usepackage{amssymb}
\usepackage[super]{nth}
\usepackage{everysel}
\usepackage{ragged2e}
\usepackage{geometry}
\usepackage{fancyhdr}
\usepackage{cancel}
\usepackage{siunitx}
\geometry{top=1.0in,bottom=1.0in,left=1.0in,right=1.0in}
\newcommand{\subtitle}[1]{%
  \posttitle{%
    \par\end{center}
    \begin{center}\large#1\end{center}
    \vskip0.5em}%

}
\usepackage{hyperref}
\hypersetup{
colorlinks=true,
linkcolor=blue,
filecolor=magenta,      
urlcolor=blue,
citecolor=blue,
}

\urlstyle{same}


\title{Linear Models $-$ Initial-Value Problems}
\date{\today}
\author{Michael Brodskiy\\ \small Professor: Meetal Shah}

% Mathematical Operations:

% Sum: $$\sum_{n=a}^{b} f(x) $$
% Integral: $$\int_{lower}^{upper} f(x) dx$$
% Limit: $$\lim_{x\to\infty} f(x)$$

\begin{document}

\maketitle

\begin{itemize}

  \item This section will focus on several linear dynamical systems, modeled by second-order differential equations

  \item The function $g$ is variously called the driving function, forcing function, or input of the system

  \item A solution $y(t)$ of the differential equation on an interval $I$ containing $t=0$ that satisfies the initial conditions is called the response or output of the system

  \item A Spring/Mass System:

    \begin{enumerate}

      \item Newton's Second Law: When a mass $m$ is attached to the lower end of a spring of negligible mass, it stretches the spring by an amount $s$ and attains an equilibrium position or rest position at which its weight $W$ is balanced by the restoring force $ks$ of the spring. Hooke's Law is shown in \eqref{1}
        \begin{equation}
          F=-kx
          \label{1}
        \end{equation}

    \end{enumerate}

  \item When the spring is in free motion, or when no external forces act on the system, Newton's second law gives \eqref{2}

    \begin{equation}
      m\frac{d^2x}{dt^2}=-k(x+s)+mg=-kx+mg-ks=-kx
      \label{2}
    \end{equation}

  \item For simple harmonic motion, the differential equation looks like \eqref{3}, where $\omega^2=\frac{k}{m}$

    \begin{equation}
      \frac{d^2x}{dt^2}+\omega^2x=0
      \label{3}
    \end{equation}

  \item The general solution for this type of equation is \eqref{4}

    \begin{equation}
      x(t)=c_1\cos \omega t + c_2\sin\omega t
      \label{4}
    \end{equation}

  \item The period of motion can be found using $T=\frac{2\pi}{\omega}$. The frequency of motion is the inverse of the period, $f=\frac{\omega}{2\pi}$

  \item The number $\omega=\sqrt{\frac{k}{m}}$ is the circular frequency, in radians per second

  \item Functions in form \eqref{3} are problematic, though, because it is difficult to determine the amplitude. It can be rewritten in form \eqref{5} using $A=\sqrt{c_1^2+c_2^2}$, and $\phi$ is a phase angle which may be found using $\tan \phi = \frac{c_1}{c_2}

    \begin{equation}
      x(t)=A\sin(\omega t + \phi)
      \label{5}
    \end{equation}

  \item Double Spring Systems:

    \begin{enumerate}

      \item Springs in parallel:

    \begin{enumerate}

      \item The effective spring constant is $k_{eff}=k_1+k_2$

      \item Once $k_{eff}$ is found, the whole process is the same as a single-springed system

    \end{enumerate}

  \item Springs in series:

    \begin{enumerate}

      \item $-k_{eff}(x_1+x_2)=-k_1x_1=-k_2x_2$, because the force exerted on each spring is the same

      \item Simplifying this, we get $k_{eff}=\frac{k_1k_2}{k_1+k_2}$

    \end{enumerate}

\end{enumerate}

\item Systems with Variable Spring Constants

  \begin{enumerate}

    \item Aging Spring Function is shown in \eqref{6}

      \begin{equation}
        K(t)=ke^{-\alpha t}
        \label{6}
      \end{equation}

    \item In an environment where the temperature is rapidly decreasing, $K(t)=kt$, and Airy's differential equation can be used to model this: \eqref{7}

      \begin{equation}
        mx''+ktx=0
        \label{7}
      \end{equation}

  \end{enumerate}

\item In damped motion, the object has a retarding force act on it

\item For damped motion, the object follows the differential equation \eqref{8}, and can be rewritten as \eqref{9}

  \begin{equation}
    m\frac{d^2x}{dt^2}=-kx-\beta\frac{dx}{dt}
    \label{8}
  \end{equation}

  \begin{equation}
    \begin{split}
      \frac{d^2x}{dt^2}+\frac{\beta}{m}\frac{dx}{dt}+\frac{k}{m}x&=0\\
      \frac{d^2x}{dt^2}+2\lambda\frac{dx}{dt}+\omega^2x&=0\\
      2\lambda=\frac{\beta}{m} \text{ and } \omega^2=\frac{k}{m}
    \end{split}
    \label{9}
  \end{equation}

\item Therefore, for the differential equation shown in equation \eqref{9}, the terms of the complementary solution can be found using \eqref{10}

  \begin{equation}
    \begin{split}
      \text{Case One (Overdamped): }& \lambda^2-\omega^2>0\\
      m_{1,2}&=-\lambda \pm\sqrt{\lambda^2-\omega^2}\\
      x(t)&=e^{-\lambda t}\left(c_1e^{\sqrt{\lambda^2-\omega^2}t}+c_2e^{-\sqrt{\lambda^2-\omega^2}} \right)\\
      \text{Case Two (Critically Damped): }& \lambda^2-\omega^2=0\\
      m_{1,2}&=0\\
      x(t)&=e^{-\lambda t}(c_1+c_2t)\\
      \text{Case Three (Underdamped): }& \lambda^2-\omega^2<0\\
      m_{1,2}&=-\lambda \pm\sqrt{\omega^2-\lambda^2}i\\
      x(t)&=e^{-\lambda t}\left(c_1\cos\sqrt{\omega^2-\lambda^2}t+c_2\sin\sqrt{\omega^2-\lambda^2}t\right)\\
    \end{split}
    \label{10}
  \end{equation}

\item For such problems, it is important to remember $g\approx32\left[  \frac{ft}{s}\right]$

\item The term $Ae^{-\lambda t}$ is called the damped amplitude of vibrations. The quasi period and the quasi frequency are defined as $\frac{2\pi}{\sqrt{\omega^2-\lambda^2}}$ and $\frac{\sqrt{\omega^2-\lambda^2}}{2\pi}$

\item In a system with driven or forced motion, $f(t)$ is defined as a function representing a force at a time. The differential equation is then modified to be \eqref{11}

  \begin{equation}
    \frac{d^2x}{dt^2}=-2\lambda\frac{dx}{dt}-\omega^2x+f(t)
    \label{11}
  \end{equation}

\item The complementary function is called the transient solution, while the particular  function is called the steady-state solution

\item Driven motion without damping motion can be written in form \eqref{12}

  \begin{equation}
    \frac{d^2x}{dt^2}+\omega^2x=f(t)
    \label{12}
  \end{equation}

\item When $\gamma=\omega$, the function reaches what is known as pure resonance

\item Series Circuits in Analogue $-$ LRC-Series Circuits may be modelled by a second-order differential equation \eqref{13}

  \begin{equation}
    L\frac{d^2q}{dt^2}+R\frac{dq}{dt}+\frac{1}{C}q=E(t)
    \label{13}
  \end{equation}

\item Inductor ($L$) $-$ Units of henries (h). Voltage drop across: $L\frac{di}{dt}$

\item Resistor ($R$) $-$ Units of ohms ($\omega$). Voltage drop across: $iR$

\item Capacitor ($C$) $-$ Units of farads (f). Voltage drop across: $q\frac{1}{C}$

\item If the $E(t)$ function equals zero, the electrical vibrations of the circuit are saidd to be free. 

\item The circuit is\dots

  \begin{tabular}[h]{l r}
    Overdamped if & $R^2-\frac{4L}{C}>0$\\
    & \\
    Critically damped if & $R^2-\frac{4L}{C}=0$\\
    & \\
    Underdamped if & $R^2-\frac{4L}{C}<0$\\
  \end{tabular}

\item When the equation is underdamped and $q(0)=q_0$, the charge of the capacitor oscillates as it decays.

\item When $E(t)=0$ and $R=0$, the circuit is undamped and the circuit is in a simple harmonic state. 

\item When there is an impressed voltage $E(t)$ on the circuit the electrical vibrations are said to be forced.

\item In the case when $R\neq0$, the complementary function $q_c(t)$ is called a transient solution.

\item If $E(t)$ is periodic or a constant, then the particular solution $q_p(t)$ is a steady-state solution.

\end{itemize}

\end{document}

