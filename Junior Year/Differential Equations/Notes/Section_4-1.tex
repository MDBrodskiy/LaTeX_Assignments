%%%%%%%%%%%%%%%%%%%%%%%%%%%%%%%%%%%%%%%%%%%%%%%%%%%%%%%%%%%%%%%%%%%%%%%%%%%%%%%%%%%%%%%%%%%%%%%%%%%%%%%%%%%%%%%%%%%%%%%%%%%%%%%%%%%%%%%%%%%%%%%%%%%%%%%%%%%%%%%%%%%%%%%%%%%%%%%%%%%%%%%%%%%%
% Written By Michael Brodskiy
% Class: Differential Equations (MATH-294)
% Professor: M. Shah
%%%%%%%%%%%%%%%%%%%%%%%%%%%%%%%%%%%%%%%%%%%%%%%%%%%%%%%%%%%%%%%%%%%%%%%%%%%%%%%%%%%%%%%%%%%%%%%%%%%%%%%%%%%%%%%%%%%%%%%%%%%%%%%%%%%%%%%%%%%%%%%%%%%%%%%%%%%%%%%%%%%%%%%%%%%%%%%%%%%%%%%%%%%%

\documentclass[12pt]{article} 
\usepackage{alphalph}
\usepackage[utf8]{inputenc}
\usepackage[russian,english]{babel}
\usepackage{titling}
\usepackage{amsmath}
\usepackage{graphicx}
\usepackage{enumitem}
\usepackage{amssymb}
\usepackage[super]{nth}
\usepackage{everysel}
\usepackage{ragged2e}
\usepackage{geometry}
\usepackage{fancyhdr}
\usepackage{cancel}
\usepackage{siunitx}
\geometry{top=1.0in,bottom=1.0in,left=1.0in,right=1.0in}
\newcommand{\subtitle}[1]{%
  \posttitle{%
    \par\end{center}
    \begin{center}\large#1\end{center}
    \vskip0.5em}%

}
\usepackage{hyperref}
\hypersetup{
colorlinks=true,
linkcolor=blue,
filecolor=magenta,      
urlcolor=blue,
citecolor=blue,
}

\urlstyle{same}


\title{Preliminary Theory $-$ Linear Equations}
\date{\today}
\author{Michael Brodskiy\\ \small Professor: Meetal Shah}

% Mathematical Operations:

% Sum: $$\sum_{n=a}^{b} f(x) $$
% Integral: $$\int_{lower}^{upper} f(x) dx$$
% Limit: $$\lim_{x\to\infty} f(x)$$

\begin{document}

\maketitle

\begin{itemize}

  \item Boundary-value Problem \eqref{1}:

    \begin{equation}
      \begin{split}
        \text{Solve: }& a_2(x)\frac{d^2y}{dx^2}+a_1(x)\frac{dy}{dx}+a_0(x)y=g(x)\\
        \text{Subject to: }& y(a)=y_0, y(b)=y_1\\
      \end{split}
      \label{1}
    \end{equation}

  \item $y(a)=y_0$ and $y(b)=y_1$ are called boundary values

  \item If the function purely of $x$ (on the right side) in a linear differential equation is equal to zero, that function is called homogeneous. When the term is not zero, it is called nonhomogeneous.

  \item The symbol $D$ is called the differential operator. Ex. \eqref{2}

    \begin{equation}
      \frac{d}{dx}\left( \frac{dy}{dx} \right)=\frac{d^2y}{dx^2}=D(D_y)=D^2y
      \label{2}
    \end{equation}

  \item In addition to this, we can define an \textbf{$\bold{n}$th-order differential operator} or polynomial operator (or linear operator)to be \eqref{3}

    \begin{equation}
      L\{\alpha f(x)+\beta g(x)\}=\alpha L(f(x))+\beta L(g(x))
      \label{3}
    \end{equation}

  \item An example which combines both aforementioned operators: Ex \eqref{4}

    \begin{equation}
      \begin{split}
      y''+5y'+6y=5x-3\\
      D^2y+5D_y+6y=5x-3\\
    (D^2+5D+6)y=5x-3\\
    L(y)=5x-3\\
  \end{split}
      \label{4}
    \end{equation}

  \item Many combinations of solutions may be found if $y_1,y_2\dots,y_k$ are solutions and are combined using a linear combination \eqref{5}

    \begin{equation}
      y=c_1y_1(x)+c_2y_2(x)+\dots+c_ky_k(x)
      \label{5}
    \end{equation}

  \item A constant multiple $y=c_1y_1(x)$ of a solution $y_1(x)$ of a homogeneous differential equation is also a solution

  \item A homogeneous linear differential equation always possesses the trivial solution $y=0$

  \item A set of functions $f_1(x), f_2(x)\dots,f_n(x)$ is said to be linearly dependent on an interval $I$ if there exist constants $c_1,c_2,\dots,c_n$, not all zero such that:

    $$c_1f_1(x)+c_2f_2(x)+\dots+c_nf_n(x)=0$$

  \item For every $x$ in the interval. If the set of functions is not linearly dependent on the interval, it is said to be linearly independent.

  \item The Wronskian: \eqref{6}

    \begin{equation}
      W(f_1,f_2,\dots,f_n)=\begin{vmatrix} f_1 & f_2 & \dots & f_n \\ f_1' & f_2' & \dots & f_n' \\ \vdots & \vdots & \dots & \vdots \\ f_1^{n-1} & f_2^{n-1} & \dots & f_n^{n-1} \\   \end{vmatrix}
      \label{6}
    \end{equation}

  \item If the Wronskian does not equal zero, the solutions are said to be linearly independent

  \item Any set $y_1, y_2, \dots, y_n$ of $n$ linearly independent solutions of the homogeneous linear $n$th-order differential equation on an interval $I$ is said to be a fundamental set of solutions on the interval

  \item The general solution of the equation on the interval is \eqref{7}

    \begin{equation}
      y=c_1y_1(x)+c_2y_2(x)+\dots+c_ny_n(x)
      \label{7}
    \end{equation}

  \item The general solution of the nonhomogeneous equation is \eqref{8}

    \begin{equation}
      \begin{split}
        y & = \text{complementary function} + \text{particular solution}\\
        & = y_c+y_p\\
      \end{split}
      \label{8}
    \end{equation}

\end{itemize}

\end{document}

