%%%%%%%%%%%%%%%%%%%%%%%%%%%%%%%%%%%%%%%%%%%%%%%%%%%%%%%%%%%%%%%%%%%%%%%%%%%%%%%%%%%%%%%%%%%%%%%%%%%%%%%%%%%%%%%%%%%%%%%%%%%%%%%%%%%%%%%%%%%%%%%%%%%%%%%%%%%%%%%%%%%%%%%%%%%%%%%%%%%%%%%%%%%%
% Written By Michael Brodskiy
% Class: Differential Equations (MATH-294)
% Professor: M. Shah
%%%%%%%%%%%%%%%%%%%%%%%%%%%%%%%%%%%%%%%%%%%%%%%%%%%%%%%%%%%%%%%%%%%%%%%%%%%%%%%%%%%%%%%%%%%%%%%%%%%%%%%%%%%%%%%%%%%%%%%%%%%%%%%%%%%%%%%%%%%%%%%%%%%%%%%%%%%%%%%%%%%%%%%%%%%%%%%%%%%%%%%%%%%%

\documentclass[12pt]{article} 
\usepackage{alphalph}
\usepackage[utf8]{inputenc}
\usepackage[russian,english]{babel}
\usepackage{titling}
\usepackage{amsmath}
\usepackage{graphicx}
\usepackage{enumitem}
\usepackage{amssymb}
\usepackage[super]{nth}
\usepackage{everysel}
\usepackage{ragged2e}
\usepackage{geometry}
\usepackage{fancyhdr}
\usepackage{cancel}
\usepackage{siunitx}
\geometry{top=1.0in,bottom=1.0in,left=1.0in,right=1.0in}
\newcommand{\subtitle}[1]{%
  \posttitle{%
    \par\end{center}
    \begin{center}\large#1\end{center}
    \vskip0.5em}%

}
\usepackage{hyperref}
\hypersetup{
colorlinks=true,
linkcolor=blue,
filecolor=magenta,      
urlcolor=blue,
citecolor=blue,
}

\urlstyle{same}


\title{Homogeneous Linear Systems}
\date{\today}
\author{Michael Brodskiy\\ \small Professor: Meetal Shah}

% Mathematical Operations:

% Sum: $$\sum_{n=a}^{b} f(x) $$
% Integral: $$\int_{lower}^{upper} f(x) dx$$
% Limit: $$\lim_{x\to\infty} f(x)$$

\begin{document}

\maketitle

\begin{itemize}

  \item The general solution of the homogeneous system $\bold{X}'=\begin{pmatrix}1 & 3\\ 5 & 3\\ \end{pmatrix}\bold{X}$ is \eqref{1}

    \begin{equation}
      \bold{X}=c_1\bold{X}_1+c_2\bold{X}_2=c_1\begin{pmatrix} 1 \\ -1 \end{pmatrix}e^{-2t}+c_2\begin{pmatrix} 3 \\ 5\end{pmatrix}e^{6t}
      \label{1}
    \end{equation}

  \item Because the solution vectors $\bold{x}_1$ and $\bold{x}_2$ have the form \eqref{2}, where $k_1$, $k_2$, $\lambda_1$, and $\lambda_2$ are constants, we are prompted to ask whether we can always find a solution of the form \eqref{3} for the general homogeneous linear first-order system $\bold{X}'=\bold{AX}$, where $\bold{A}$ is an $nxn$ matrix of constants.

    \begin{equation}
      \bold{x}_i=\begin{pmatrix} k_1 \\ k_2 \end{pmatrix}e^{\lambda_it},\,\,i=1,2
      \label{2}
    \end{equation}

    \begin{equation}
      \bold{X}=\begin{pmatrix}k_1\\k_2\\\vdots\\k_n\end{pmatrix}e^{\lambda t}=\bold{K}e^{\lambda t}
      \label{3}
    \end{equation}

  \item Given the above, $\bold{X}'=\bold{K}\lambda e^{\lambda t}$, so the system becomes $\bold{K}\lambda e^{\lambda t}=\bold{AK}e^{\lambda t}$. After dividing $e^{\lambda t}$ and rearranging, we obtain $\bold{AK}-\lambda\bold{K}=0$, giving us $(\bold{A}-\lambda\bold{I})\bold{K}=0$

  \item To figure out whether solutions exist for $(\bold{A}-\lambda\bold{I})\bold{K}=0$, we take $\det(\bold{A}-\lambda\bold{I})=0$

  \item The polynomial equation $\lambda$ is called the characteristic equation of the matrix $\bold{A}$; its solutions are the eigenvalues of $\bold{A}$ A solution $\bold{K}\neq0$ of the above corresponding to an eigenvalue, $\lambda$ is called an eigenvector of $\bold{A}$. A solution of the homogeneous system is then $\bold{X}=\bold{K}e^{\lambda t}$

  \item If $\lambda_1,\lambda_2,\dots,\lambda_n$ has $n$ distinct real eigenvalues of the coefficient matrix $\bold{A}$ of the homogeneous system and let $\bold{K}_1,\bold{K}_2,\dots,\bold{K}_n$ be the corresponding eigenvectors. Then the general solution of the system on the interval $(-\infty,\infty)$ is given by \eqref{4}

    \begin{equation}
      \bold{X}=c_1\bold{K}_1e^{\lambda_1t}+c_2\bold{K}_2e^{\lambda_2 t}+\dots+c_n\bold{K}_ne^{\lambda_nt}
      \label{4}
    \end{equation}

  \item For the two parametrized functions of $t$, they can be represented together in the $xy$-plane, or the phase plane. For different values of the arbitrary constants, each new curve is called a trajectory.

  \item A collection of trajectories is a phase portrait.

  \item When both lambdas are positive, it is an attractor, whereas, when both lambdas are negative, it is a repeller

  \item If there is only one eigenvalue corresponding to the eigenvalue $\lambda_1$ of multiplicity $m$, then $m$ linearly independent solutions of the form \eqref{5} exist, where $\bold{K}_{ij}$ are column vectors, and can always be found

    \begin{equation}
      \begin{split}
        \bold{X}_1&=\bold{K}_{11}e^{\lambda_1t}\\
        \bold{X}_2&=\bold{K}_{21}te^{\lambda_1t}+K_{22}e^{\lambda_1t}\\
        &\hspace{5pt}\vdots\\
        \bold{X}_m&=\bold{K}_{m1}\frac{t^{m-1}}{(m-1)!}e^{\lambda_1t}+\bold{K}_{m2}\frac{t^{m-2}}{(m-2)!}e^{\lambda_1t}+\dots+\bold{K}_{mm}e^{\lambda_1t}\\
      \end{split}
      \label{5}
    \end{equation}

  \item If $\lambda_1$ is an eigenvalue of multiplicity two and that there is only one eigenvector associated with this value. A second solution can be found of the form \eqref{6}

    \begin{equation}
      \bold{X}_2=\bold{K}te^{\lambda_1t}+\bold{P}e^{\lambda_1t}
      \label{6}
    \end{equation}

  \item To confirm \eqref{6}, we can substitute into the system $\bold{X}'=\bold{AX}$:

    \begin{equation}
      (\bold{AK}-\lambda_1\bold{K})te^{\lambda_1t}+(\bold{AP}+\lambda_1\bold{P}-\bold{K})e^{\lambda_1t}
      \label{7}
    \end{equation}

\end{itemize}

\end{document}

