%%%%%%%%%%%%%%%%%%%%%%%%%%%%%%%%%%%%%%%%%%%%%%%%%%%%%%%%%%%%%%%%%%%%%%%%%%%%%%%%%%%%%%%%%%%%%%%%%%%%%%%%%%%%%%%%%%%%%%%%%%%%%%%%%%%%%%%%%%%%%%%%%%%%%%%%%%%%%%%%%%%%%%%%%%%%%%%%%%%%%%%%%%%%
% Written By Michael Brodskiy
% Class: Differential Equations (MATH-294)
% Professor: M. Shah
%%%%%%%%%%%%%%%%%%%%%%%%%%%%%%%%%%%%%%%%%%%%%%%%%%%%%%%%%%%%%%%%%%%%%%%%%%%%%%%%%%%%%%%%%%%%%%%%%%%%%%%%%%%%%%%%%%%%%%%%%%%%%%%%%%%%%%%%%%%%%%%%%%%%%%%%%%%%%%%%%%%%%%%%%%%%%%%%%%%%%%%%%%%%

\documentclass[12pt]{article} 
\usepackage{alphalph}
\usepackage[utf8]{inputenc}
\usepackage[russian,english]{babel}
\usepackage{titling}
\usepackage{amsmath}
\usepackage{graphicx}
\usepackage{enumitem}
\usepackage{amssymb}
\usepackage[super]{nth}
\usepackage{everysel}
\usepackage{ragged2e}
\usepackage{geometry}
\usepackage{fancyhdr}
\usepackage{cancel}
\usepackage{siunitx}
\geometry{top=1.0in,bottom=1.0in,left=1.0in,right=1.0in}
\newcommand{\subtitle}[1]{%
  \posttitle{%
    \par\end{center}
    \begin{center}\large#1\end{center}
    \vskip0.5em}%

}
\usepackage{hyperref}
\hypersetup{
colorlinks=true,
linkcolor=blue,
filecolor=magenta,      
urlcolor=blue,
citecolor=blue,
}

\urlstyle{same}


\title{Fourier Cosine and Sine Series}
\date{\today}
\author{Michael Brodskiy\\ \small Professor: Meetal Shah}

% Mathematical Operations:

% Sum: $$\sum_{n=a}^{b} f(x) $$
% Integral: $$\int_{lower}^{upper} f(x) dx$$
% Limit: $$\lim_{x\to\infty} f(x)$$

\begin{document}

\maketitle

\begin{itemize}

  \item The definition of an even and odd function is defined in \eqref{1}

    \begin{equation}
      \begin{split}
      \text{Even if } f(-x)=f(x)\\
      \text{Odd if } f(-x)=-f(x)\\
    \end{split}
      \label{1}
    \end{equation}

  \item Some properties are:

    \begin{enumerate}

      \item The product of two even functions is even.

      \item The product of two odd functions is even.

      \item The product of an even function and an odd function is odd.

      \item The sum (difference) of two even functions is even.

      \item The sum (difference) of two odd functions is odd.

      \item If $f$ is even, then $\int_{-a}^a f(x)\,dx=2\int_0^a f(x)\,dx$

      \item If $f$ is odd, then $\int_{-a}^a f(x)\,dx=0$

    \end{enumerate}

  \item The Fourier Series of an even function $f$ defined on the interval $(-p,p)$ is the cosine series \eqref{2}

    \begin{equation}
      \begin{split}
        f(x)=\frac{a_0}{2}+\sum_{n=1}^{\infty} a_n\cos\frac{n\pi}{p}x\\
        a_0=\frac{2}{p}\int_0^pf(x)\,dx\\
      a_n=\frac{2}{p}\int_0^pf(x)\left(\cos\frac{n\pi}{p}x\right)\,dx\\
      \end{split}
      \label{2}
    \end{equation}

  \item The Fourier Series of an odd function $f$ defined on the interval $(-p,p)$ is the sine series \eqref{3}

    \begin{equation}
      \begin{split}
        f(x)=\sum_{n=1}^{\infty} b_n\sin\frac{n\pi}{p}x\\
      b_n=\frac{2}{p}\int_0^pf(x)\left(\sin\frac{n\pi}{p}x\right)\,dx\\
      \end{split}
      \label{3}
    \end{equation}

  \item Gibbs Phenomenon applies to points near discontinuities.

  \item Half Range Expansions $-$ May be used to express $f$ on interval $0< x< L$ by using a ``dummy'' function setup.

    \begin{enumerate}

      \item Even Reflection $-$ Across $y$-axis, and then create a cosine series

      \item Odd Reflection $-$ Across origin, and then create a sine series

      \item Periodic $-$ Transform the function so it is periodic, then create a Fourier Series

    \end{enumerate}

\end{itemize}

\end{document}

