%%%%%%%%%%%%%%%%%%%%%%%%%%%%%%%%%%%%%%%%%%%%%%%%%%%%%%%%%%%%%%%%%%%%%%%%%%%%%%%%%%%%%%%%%%%%%%%%%%%%%%%%%%%%%%%%%%%%%%%%%%%%%%%%%%%%%%%%%%%%%%%%%%%%%%%%%%%%%%%%%%%%%%%%%%%%%%%%%%%%%%%%%%%%
% Written By Michael Brodskiy
% Class: Differential Equations (MATH-294)
% Professor: M. Shah
%%%%%%%%%%%%%%%%%%%%%%%%%%%%%%%%%%%%%%%%%%%%%%%%%%%%%%%%%%%%%%%%%%%%%%%%%%%%%%%%%%%%%%%%%%%%%%%%%%%%%%%%%%%%%%%%%%%%%%%%%%%%%%%%%%%%%%%%%%%%%%%%%%%%%%%%%%%%%%%%%%%%%%%%%%%%%%%%%%%%%%%%%%%%

\documentclass[12pt]{article} 
\usepackage{alphalph}
\usepackage[utf8]{inputenc}
\usepackage[russian,english]{babel}
\usepackage{titling}
\usepackage{amsmath}
\usepackage{graphicx}
\usepackage{enumitem}
\usepackage{amssymb}
\usepackage[super]{nth}
\usepackage{everysel}
\usepackage{ragged2e}
\usepackage{geometry}
\usepackage{fancyhdr}
\usepackage{cancel}
\usepackage{siunitx}
\geometry{top=1.0in,bottom=1.0in,left=1.0in,right=1.0in}
\newcommand{\subtitle}[1]{%
  \posttitle{%
    \par\end{center}
    \begin{center}\large#1\end{center}
    \vskip0.5em}%

}
\usepackage{hyperref}
\hypersetup{
colorlinks=true,
linkcolor=blue,
filecolor=magenta,      
urlcolor=blue,
citecolor=blue,
}

\urlstyle{same}


\title{Exact Equations}
\date{\today}
\author{Michael Brodskiy\\ \small Professor: Meetal Shah}

% Mathematical Operations:

% Sum: $$\sum_{n=a}^{b} f(x) $$
% Integral: $$\int_{lower}^{upper} f(x) dx$$
% Limit: $$\lim_{x\to\infty} f(x)$$

\begin{document}

\maketitle

\begin{itemize}

  \item Recall that the \textbf{differential} is defined as:

    $$dz=\frac{\partial f}{\partial x}dx+\frac{\partial f}{\partial y}dy$$

  \item $f(x,y)=c$ is said to be a solution of the differential equation:

    $$dz=\frac{\partial f}{\partial x}dx+\frac{\partial f}{\partial y}dy$$

  \item A differential expression $M(x,y)\,dx+N(x,y)\,dy$ is an exact differential if it corresponds to the differential of some function $f(x,y)$.

  \item $M(x,y)\,dx+N(x,y)\,dy$ is said to be an exact equation if the expression on the left-hand side is an exact differential

  \item If $M(x,y)$ and $N(x,y)$ are continuous and have continuous first partial derivatives, then, to be an exact differential:

    $$\frac{\partial M}{\partial y}=\frac{\partial N}{\partial x}$$

  \item If the equality above holds true, then a solution exists for a function $f$ for which:

    $$\frac{\partial f}{\partial x}=M(x,y)$$

  \item We can find $f$ by integrating $M(x,y)$ with respect to $x$ while holding $y$ constant:

    $$f(x,y)=\int M(x,y)\,dx+g(y) \text{ where $g(y)$ represents the ``constant'' of integration}$$

  \item This can be confirmed when we differentiate with respect to $x$:

    $$\frac{\partial f(x,y)}{\partial x}=\frac{\partial}{\partial x}\int M(x,y)\,dx+ \frac{\partial g(y)}{\partial x}$$
    $$\frac{\partial f(x,y)}{\partial x}= M(x,y)$$

  \item And when we differentiate with respect to $y$:

    
    $$\frac{\partial f(x,y)}{\partial y}=\frac{\partial}{\partial y}\int M(x,y)\,dx+ \frac{\partial g(y)}{\partial y}$$
    $$\frac{\partial f(x,y)}{\partial y}=\frac{\partial}{\partial y}\int M(x,y)\,dx+ g'(y)$$
    $$g'(y)=N(x,y)-\frac{\partial}{\partial y}\int M(x,y)\,dx$$

  \item The integrating factor technique may be used for exact equations as well:

    $$\mu_xN-\mu_yM=(M_y-N_x)\mu$$

  \item For $\mu_x=d\mu/dx, \mu_y=0$, then:

    $$\frac{d\mu}{dx}=\frac{M_y-N_x}{N}\mu$$
    $$\therefore \mu(x)=e^{\int\frac{M_y-N_x}{N}\,dx}$$

\item If, conversely $\mu_y=d\mu/dy, \mu_x=0$, then:
  
    $$\frac{d\mu}{dy}=\frac{N_x-M_y}{M}\mu$$
    $$\therefore \mu(y)=e^{\int\frac{N_x-M_y}{M}\,dy}$$
    
  \item This method may be used on non-exact DEs to make them exact

\end{itemize}

\end{document}

