%%%%%%%%%%%%%%%%%%%%%%%%%%%%%%%%%%%%%%%%%%%%%%%%%%%%%%%%%%%%%%%%%%%%%%%%%%%%%%%%%%%%%%%%%%%%%%%%%%%%%%%%%%%%%%%%%%%%%%%%%%%%%%%%%%%%%%%%%%%%%%%%%%%%%%%%%%%%%%%%%%%%%%%%%%%%%%%%%%%%%%%%%%%%
% Written By Michael Brodskiy
% Class: Differential Equations (MATH-294)
% Professor: M. Shah
%%%%%%%%%%%%%%%%%%%%%%%%%%%%%%%%%%%%%%%%%%%%%%%%%%%%%%%%%%%%%%%%%%%%%%%%%%%%%%%%%%%%%%%%%%%%%%%%%%%%%%%%%%%%%%%%%%%%%%%%%%%%%%%%%%%%%%%%%%%%%%%%%%%%%%%%%%%%%%%%%%%%%%%%%%%%%%%%%%%%%%%%%%%%

\documentclass[12pt]{article} 
\usepackage{alphalph}
\usepackage[utf8]{inputenc}
\usepackage[russian,english]{babel}
\usepackage{titling}
\usepackage{amsmath}
\usepackage{graphicx}
\usepackage{enumitem}
\usepackage{amssymb}
\usepackage[super]{nth}
\usepackage{everysel}
\usepackage{ragged2e}
\usepackage{geometry}
\usepackage{fancyhdr}
\usepackage{cancel}
\usepackage{siunitx}
\geometry{top=1.0in,bottom=1.0in,left=1.0in,right=1.0in}
\newcommand{\subtitle}[1]{%
  \posttitle{%
    \par\end{center}
    \begin{center}\large#1\end{center}
    \vskip0.5em}%

}
\usepackage{hyperref}
\hypersetup{
colorlinks=true,
linkcolor=blue,
filecolor=magenta,      
urlcolor=blue,
citecolor=blue,
}

\urlstyle{same}


\title{A Numerical Method}
\date{\today}
\author{Michael Brodskiy\\ \small Professor: Meetal Shah}

% Mathematical Operations:

% Sum: $$\sum_{n=a}^{b} f(x) $$
% Integral: $$\int_{lower}^{upper} f(x) dx$$
% Limit: $$\lim_{x\to\infty} f(x)$$

\begin{document}

\maketitle

\begin{itemize}

  \item A \textbf{Euler's Method} \eqref{eq1} may be used to approximate values of a solution curve, as long as those points are relatively near the values used, and the step size, $h$, is not too big:

    \begin{equation}
      \begin{split}
        y_{n+1}=y_n + hf(x_n,y_n)
      \end{split}
      \label{eq1}
    \end{equation}

  \item \textbf{Absolute Error} \eqref{eq2} gives you the deviation between the correct value and the value obtained from an approximation:

    \begin{equation}
      | value - approximation|
      \label{eq2}
    \end{equation}

  \item The relative error \eqref{eq3} and percentage relative error \eqref{eq4} are shown below:

    \begin{equation}
      \frac{absolute}{|value|}
      \label{eq3}
    \end{equation}

    \begin{equation}
      \frac{absolute}{|value|}\cdot 100\%
      \label{eq4}
    \end{equation}

  \item Euler's Method will not be used often, we will use the Runge-Kutta method for often.

  \item A computer program made to graph or process numerical data is known as a numerical solver. Many computer programs can give corresponding approximations, or numerical solution curves.


\end{itemize}

\end{document}

