%%%%%%%%%%%%%%%%%%%%%%%%%%%%%%%%%%%%%%%%%%%%%%%%%%%%%%%%%%%%%%%%%%%%%%%%%%%%%%%%%%%%%%%%%%%%%%%%%%%%%%%%%%%%%%%%%%%%%%%%%%%%%%%%%%%%%%%%%%%%%%%%%%%%%%%%%%%%%%%%%%%%%%%%%%%%%%%%%%%%%%%%%%%%
% Written By Michael Brodskiy
% Class: Differential Equations (MATH-294)
% Professor: M. Shah
%%%%%%%%%%%%%%%%%%%%%%%%%%%%%%%%%%%%%%%%%%%%%%%%%%%%%%%%%%%%%%%%%%%%%%%%%%%%%%%%%%%%%%%%%%%%%%%%%%%%%%%%%%%%%%%%%%%%%%%%%%%%%%%%%%%%%%%%%%%%%%%%%%%%%%%%%%%%%%%%%%%%%%%%%%%%%%%%%%%%%%%%%%%%

\documentclass[12pt]{article} 
\usepackage{alphalph}
\usepackage[utf8]{inputenc}
\usepackage[russian,english]{babel}
\usepackage{titling}
\usepackage{amsmath}
\usepackage{graphicx}
\usepackage{enumitem}
\usepackage{amssymb}
\usepackage[super]{nth}
\usepackage{everysel}
\usepackage{ragged2e}
\usepackage{geometry}
\usepackage{fancyhdr}
\usepackage{cancel}
\usepackage{siunitx}
\geometry{top=1.0in,bottom=1.0in,left=1.0in,right=1.0in}
\newcommand{\subtitle}[1]{%
  \posttitle{%
    \par\end{center}
    \begin{center}\large#1\end{center}
    \vskip0.5em}%

}
\usepackage{hyperref}
\hypersetup{
colorlinks=true,
linkcolor=blue,
filecolor=magenta,      
urlcolor=blue,
citecolor=blue,
}

\urlstyle{same}


\title{The Dirac Delta Function}
\date{\today}
\author{Michael Brodskiy\\ \small Professor: Meetal Shah}

% Mathematical Operations:

% Sum: $$\sum_{n=a}^{b} f(x) $$
% Integral: $$\int_{lower}^{upper} f(x) dx$$
% Limit: $$\lim_{x\to\infty} f(x)$$

\begin{document}

\maketitle

\begin{itemize}

  \item The Dirac Delta ``Function'' may be modeled by \eqref{1}

    \begin{equation}
      \delta(t-t_0)=\left\{\begin{array}{ll} 0,\, & 0\leq t < t_0-a\\ \frac{1}{2a},\, & t_0-a\leq t < t_0+a\\ 0,\, & t\geq t_0 + a\\ \end{array}
      \label{1}
    \end{equation}

  \item This function could serve as a model for a big force exerted over little time (an impulse)

  \item The impulse (force over time), or the area under the graph of the force always equals 1. This means that, for a shorter period of time, the force is greater

  \item The function $\delta_a(t-t_0)$ is called the unit impulse, or the Dirac Delta Function. The limit that approximates this is defined as \eqref{2}

    \begin{equation}
      \delta (t-t_0)=\lim_{a\to0}\delta_a(t-t_0)
      \label{2}
    \end{equation}

  \item The Laplace transform is defined as \eqref{3}

    \begin{equation}
      \mathcal{L}\left\{ \delta(t-t_0) \right\}=e^{-st_0}
      \label{3}
    \end{equation}

  \item The integral of any function multiplied by the Dirac delta function is the function evaluated at the point $t_0$, or \eqref{4}

    \begin{equation}
      \int_0^{\infinity} f(t)\delta(t-t_0)\,dt=f(t_0)
      \label{4}
    \end{equation}

\end{itemize}

\end{document}

