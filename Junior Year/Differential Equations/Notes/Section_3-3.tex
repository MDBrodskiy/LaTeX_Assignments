%%%%%%%%%%%%%%%%%%%%%%%%%%%%%%%%%%%%%%%%%%%%%%%%%%%%%%%%%%%%%%%%%%%%%%%%%%%%%%%%%%%%%%%%%%%%%%%%%%%%%%%%%%%%%%%%%%%%%%%%%%%%%%%%%%%%%%%%%%%%%%%%%%%%%%%%%%%%%%%%%%%%%%%%%%%%%%%%%%%%%%%%%%%%
% Written By Michael Brodskiy
% Class: Differential Equations (MATH-294)
% Professor: M. Shah
%%%%%%%%%%%%%%%%%%%%%%%%%%%%%%%%%%%%%%%%%%%%%%%%%%%%%%%%%%%%%%%%%%%%%%%%%%%%%%%%%%%%%%%%%%%%%%%%%%%%%%%%%%%%%%%%%%%%%%%%%%%%%%%%%%%%%%%%%%%%%%%%%%%%%%%%%%%%%%%%%%%%%%%%%%%%%%%%%%%%%%%%%%%%

\documentclass[12pt]{article} 
\usepackage{alphalph}
\usepackage[utf8]{inputenc}
\usepackage[russian,english]{babel}
\usepackage{titling}
\usepackage{amsmath}
\usepackage{graphicx}
\usepackage{enumitem}
\usepackage{amssymb}
\usepackage[super]{nth}
\usepackage{everysel}
\usepackage{ragged2e}
\usepackage{geometry}
\usepackage{fancyhdr}
\usepackage{cancel}
\usepackage{siunitx}
\geometry{top=1.0in,bottom=1.0in,left=1.0in,right=1.0in}
\newcommand{\subtitle}[1]{%
  \posttitle{%
    \par\end{center}
    \begin{center}\large#1\end{center}
    \vskip0.5em}%

}
\usepackage{hyperref}
\hypersetup{
colorlinks=true,
linkcolor=blue,
filecolor=magenta,      
urlcolor=blue,
citecolor=blue,
}

\urlstyle{same}


\title{Modeling with Systems of First-Order DEs}
\date{\today}
\author{Michael Brodskiy\\ \small Professor: Meetal Shah}

% Mathematical Operations:

% Sum: $$\sum_{n=a}^{b} f(x) $$
% Integral: $$\int_{lower}^{upper} f(x) dx$$
% Limit: $$\lim_{x\to\infty} f(x)$$

\begin{document}

\maketitle

\begin{itemize}

  \item Say we are given two populations which interact, $x(t)$ and $y(t)$. The two differential equations \eqref{1} can be used to model population growth. A \textbf{linear system} would be of form \eqref{2}, where $c_i$ could depend on $t$. Any model of another form is said to be nonlinear.

    \begin{equation}
      \begin{split}
        \frac{dx}{dt} & \frac{dy}{dt}
      \end{split}
      \label{1}
    \end{equation}

    \begin{equation}
      \begin{split}
        g_1(t,x,y) & = c_1x+c_2y+f_1(t) \\
        g_2(t,x,y) & = c_3x+c_4y+f_2(t)
      \end{split}
      \label{2}
    \end{equation}


  \item Given different elements with radioactive decay, where $y$ is gaining atoms from decay of $x$ and itself decaying, it depends on $x$ and $y$ the differential equations \eqref{3}

    \begin{equation}
      \begin{split}
        \frac{dx}{dt} & = -\lambda_1x \\
        \frac{dy}{dt} & = \lambda_1x-\lambda_2y \\
        \frac{dz}{dt} & = \lambda_2y 
      \end{split}
      \label{3}
    \end{equation}

  \item Say $x(t)$ and $y(t)$ are fox and rabbit populations, respectively. The model for the fox population, without rabbits, may be found using equation 4 \eqref{4}. If there are rabbits in the system, there could be a better model, equation 5 \eqref{5}.

    \begin{equation}
      \begin{split}
        \frac{dx}{dt} & = -ax \\
      \end{split}
      \label{4}
    \end{equation}

    \begin{equation}
      \begin{split}
        \frac{dx}{dt} & = -ax+bxy \\
      \end{split}
      \label{5}
    \end{equation}



\end{itemize}

\end{document}

