%%%%%%%%%%%%%%%%%%%%%%%%%%%%%%%%%%%%%%%%%%%%%%%%%%%%%%%%%%%%%%%%%%%%%%%%%%%%%%%%%%%%%%%%%%%%%%%%%%%%%%%%%%%%%%%%%%%%%%%%%%%%%%%%%%%%%%%%%%%%%%%%%%%%%%%%%%%%%%%%%%%%%%%%%%%%%%%%%%%%%%%%%%%%
% Written By Michael Brodskiy
% Class: Differential Equations (MATH-294)
% Professor: M. Shah
%%%%%%%%%%%%%%%%%%%%%%%%%%%%%%%%%%%%%%%%%%%%%%%%%%%%%%%%%%%%%%%%%%%%%%%%%%%%%%%%%%%%%%%%%%%%%%%%%%%%%%%%%%%%%%%%%%%%%%%%%%%%%%%%%%%%%%%%%%%%%%%%%%%%%%%%%%%%%%%%%%%%%%%%%%%%%%%%%%%%%%%%%%%%

\documentclass[12pt]{article} 
\usepackage{alphalph}
\usepackage[utf8]{inputenc}
\usepackage[russian,english]{babel}
\usepackage{titling}
\usepackage{amsmath}
\usepackage{graphicx}
\usepackage{enumitem}
\usepackage{amssymb}
\usepackage[super]{nth}
\usepackage{everysel}
\usepackage{ragged2e}
\usepackage{geometry}
\usepackage{fancyhdr}
\usepackage{cancel}
\usepackage{siunitx}
\geometry{top=1.0in,bottom=1.0in,left=1.0in,right=1.0in}
\newcommand{\subtitle}[1]{%
  \posttitle{%
    \par\end{center}
    \begin{center}\large#1\end{center}
    \vskip0.5em}%

}
\usepackage{hyperref}
\hypersetup{
colorlinks=true,
linkcolor=blue,
filecolor=magenta,      
urlcolor=blue,
citecolor=blue,
}

\urlstyle{same}


\title{Orthogonal Functions}
\date{\today}
\author{Michael Brodskiy\\ \small Professor: Meetal Shah}

% Mathematical Operations:

% Sum: $$\sum_{n=a}^{b} f(x) $$
% Integral: $$\int_{lower}^{upper} f(x) dx$$
% Limit: $$\lim_{x\to\infty} f(x)$$

\begin{document}

\maketitle

\begin{itemize}

  \item Properties of inner (dot) product:

    \begin{enumerate}

      \item $\langle \bold{u}, \bold{v}\rangle=\langle \bold{v}, \bold{u}\rangle$

      \item $\langle k\bold{u}, \bold{v}\rangle=k\langle \bold{v}, \bold{u}\rangle$, where $k$ is a scalar

      \item $\langle \bold{u}, \bold{u}\rangle=0$ if $\bold{u}=0$ and $\langle \bold{u}, \bold{u}\rangle >0$ if $\bold{u}>0$

      \item $\langle \bold{u}+\bold{v}, \bold{w}\rangle=\langle \bold{u},\bold{w}\rangle+\langle \bold{v},\bold{w}\rangle$

    \end{enumerate}

  \item The inner product of two functions $f_1$ and $f_2$ on an interval $[a,b]$ is a number given by \eqref{1}

    \begin{equation}
      (f_1,f_2)=\int_a^b f_1(x)f_2(x)\,dx
      \label{1}
    \end{equation}

  \item Two functions $f_1$ and $f_2$ are orthogonal if \eqref{2} is true

    \begin{equation}
      (f_1,f_2)=\int_a^b f_1(x)f_2(x)\,dx=0
      \label{2}
    \end{equation}

  \item A set of real-valued functions $\{\phi_0(x),\phi_1(x),\phi_2(x),\dots\}$ is said to be orthogonal on an interval $[a,b]$ if \eqref{3} and $m\neq n$

    \begin{equation}
      (\phi_m,\phi_n)=\int_a^b \phi_m(x)\phi_n(x)\,dx=0
      \label{3}
    \end{equation}

  \item The square norm of a function $\phi_n$ is $||\phi_n(x)||^2=(\phi_n,\phi_n)$, meaning that the norm, or its generalized length, is $||\phi_n(x)||=\sqrt{(\phi_n,\phi_n)}$

  \item The above means that \eqref{4}

    \begin{equation}
      \begin{split}
        ||\phi_n(x)||^2&=\int_a^b \phi_n^2(x)\,dx\\
        ||\phi_n(x)||&=\sqrt{\int_a^b \phi_n^2(x)\,dx}\\
    \end{split}
      \label{4}
    \end{equation}

  \item If $\{\phi_n(x)\}$ is an orthogonal set of functions on the interval $[a,b]$ with the additional property that $||\phi_n(x)||=1$ for $n=0,1,2\dots,$ then $\{\phi_n(x)\}$ is said to be an orthonormal set on the interval

  \item The norm of $\phi_0(x)=1$ is $||\phi_0(x)||=\sqrt{2\pi}$

  \item The process of normalizing a function set consists of dividing each function by its norm

  \item Given the components $c_i$ where $i=1,2,3$, $\bold{u}=c_1\bold{v}_1+c_2\bold{v}_2+c_3\bold{v}_3$, each component may be found using \eqref{5}

    \begin{equation}
      \begin{split}
        c_1&=\frac{\langle\bold{u},\bold{v}_1\rangle}{||v_1||^2}\\
        c_2&=\frac{\langle\bold{u},\bold{v}_2\rangle}{||v_2||^2}\\
        c_3&=\frac{\langle\bold{u},\bold{v}_3\rangle}{||v_3||^2}\\
      \end{split}
      \label{5}
    \end{equation}

  \item In inner product notation, each component may be found using \eqref{6}

    \begin{equation}
      \begin{split}
      f(x)=\sum_{n=0}^{\infty}c_n\phi_n(x)\\
      c_n=\frac{\int_a^bf(x)\phi_n(x)\,dx}{||\phi_n(x)||^2}
    \end{split}
      \label{6}
    \end{equation}

  \item A set of real-valued functions $\{\phi_0(x),\phi_1(x),\phi_2(x),\dots\}$ is said to be orthogonal with respect to a weight function $w(x)$ on an interval $[a,b]$ if \eqref{7}, where $w(x)$ is usually greater than zero

    \begin{equation}
      \int_a^bw(x)\phi_m(x)\phi_n(x)\,dx=0,\,\,\,\,\,\,m\neq n 
      \label{7}
    \end{equation}

\end{itemize}

\end{document}

