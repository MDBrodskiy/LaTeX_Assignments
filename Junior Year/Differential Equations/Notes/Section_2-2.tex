%%%%%%%%%%%%%%%%%%%%%%%%%%%%%%%%%%%%%%%%%%%%%%%%%%%%%%%%%%%%%%%%%%%%%%%%%%%%%%%%%%%%%%%%%%%%%%%%%%%%%%%%%%%%%%%%%%%%%%%%%%%%%%%%%%%%%%%%%%%%%%%%%%%%%%%%%%%%%%%%%%%%%%%%%%%%%%%%%%%%%%%%%%%%
% Written By Michael Brodskiy
% Class: Differential Equations (MATH-294)
% Professor: M. Shah
%%%%%%%%%%%%%%%%%%%%%%%%%%%%%%%%%%%%%%%%%%%%%%%%%%%%%%%%%%%%%%%%%%%%%%%%%%%%%%%%%%%%%%%%%%%%%%%%%%%%%%%%%%%%%%%%%%%%%%%%%%%%%%%%%%%%%%%%%%%%%%%%%%%%%%%%%%%%%%%%%%%%%%%%%%%%%%%%%%%%%%%%%%%%

\documentclass[12pt]{article} 
\usepackage{alphalph}
\usepackage[utf8]{inputenc}
\usepackage[russian,english]{babel}
\usepackage{titling}
\usepackage{amsmath}
\usepackage{graphicx}
\usepackage{enumitem}
\usepackage{amssymb}
\usepackage[super]{nth}
\usepackage{everysel}
\usepackage{ragged2e}
\usepackage{geometry}
\usepackage{fancyhdr}
\usepackage{cancel}
\usepackage{siunitx}
\geometry{top=1.0in,bottom=1.0in,left=1.0in,right=1.0in}
\newcommand{\subtitle}[1]{%
  \posttitle{%
    \par\end{center}
    \begin{center}\large#1\end{center}
    \vskip0.5em}%

}
\usepackage{hyperref}
\hypersetup{
colorlinks=true,
linkcolor=blue,
filecolor=magenta,      
urlcolor=blue,
citecolor=blue,
}

\urlstyle{same}


\title{Separable Equations}
\date{\today}
\author{Michael Brodskiy\\ \small Professor: Meetal Shah}

% Mathematical Operations:

% Sum: $$\sum_{n=a}^{b} f(x) $$
% Integral: $$\int_{lower}^{upper} f(x) dx$$
% Limit: $$\lim_{x\to\infty} f(x)$$

\begin{document}

\maketitle

\begin{itemize}

  \item Separable Differential $-$ A differential equation that may be broken apart into a function of $x$ and a function of $y$:

    $$\frac{dy}{dx}=g(x)h(y)$$

  \item If this form is divided by $h(y)$ (where $h(y)=\frac{1}{p(y)}$:

      $$p(y)\frac{dy}{dx}=g(x)$$

  \item If $y=\phi(x)$, then:

    $$p(\phi(x))\phi'(x)=g(x)$$
    $$\int p(\phi(x))\phi'(x)\,dx=\int g(x)\,dx$$
    $$\frac{dy}{dx}=\phi'(x)\Rightarrow dy=\phi'(x)\,dx$$
    $$\int p(\phi(x))\,dy=\int g(x)\,dx\Rightarrow H(y)=G(x)+c$$

  \item Often, it will be necessary to create an integral-defined function, where $(x_o,y_o)$ is the initial condition:

    $$y=y_o+\int_{x_o}^xf(t)\,dt$$

  \item One example of such a case would be where $\frac{dy}{dx}=e^{-x^2}$. It is not possible to differentiate this, and, therefore, one ends up with:

    $$y=y_o+\int_{x_o}^xe^{-t^2}\,dt$$

\end{itemize}

\end{document}

