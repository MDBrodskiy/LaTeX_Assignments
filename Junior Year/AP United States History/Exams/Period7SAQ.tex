
\documentclass[12pt]{article} 
\usepackage{alphalph}
\usepackage[utf8]{inputenc}
\usepackage[russian,english]{babel}
\usepackage{titling}
\usepackage{amsmath}
\usepackage{graphicx}
\usepackage{enumitem}
\usepackage{amssymb}
\usepackage[super]{nth}
\usepackage{everysel}
\usepackage{ragged2e}
\usepackage{geometry}
\usepackage{fancyhdr}
\geometry{top=1.0in,bottom=1.0in,left=1.0in,right=1.0in}
\newcommand{\subtitle}[1]{%
  \posttitle{%
    \par\end{center}
    \begin{center}\large#1\end{center}
    \vskip0.5em}%

}

\usepackage{setspace}
\doublespacing

\usepackage{hyperref}
\hypersetup{
colorlinks=true,
linkcolor=blue,
filecolor=magenta,      
urlcolor=blue,
citecolor=blue,
}

\urlstyle{same}
%\renewcommand*\familydefault{\ttdefault}
%\EverySelectfont{%
%\fontdimen2\font=0.4em% interword space
%\fontdimen3\font=0.2em% interword stretch
%\fontdimen4\font=0.1em% interword shrink
%\fontdimen7\font=0.1em% extra space
%\hyphenchar\font=`\-% to allow hyphenation
%}

\pagestyle{fancy}
\lhead{Michael Brodskiy}
\rhead{Page \thepage}
\chead{APUSH SAQ}
\cfoot{}

\begin{document}

\begin{enumerate}

  \item $-$ Question 1

    \begin{enumerate}

      \item Hounshell's and Flink's interpretations differ in that Hounshell interprets the effects through the lens of the public as a whole, while Flink analyzes the effects of mass production through the lens of an average individual. Thus, Hounshell argues that it was beneficial for the whole public, as goods, like clothes, became more widely distributed and cheaper to purchase, whereas Flink makes the argument that the factories made their workers replaceable, unknown parts in a system, which, in the long run, was worse to those who held these jobs. In this manner, their views differ on whether the long term effects were beneficial or detrimental to the average citizen.

      \item The idea of interchangeable parts was a major breakthrough with respect to Hounshell's view. This notion led to easily fixable devices, such as automobiles, weapons, and various kitchen supplies that could be fixed with individual parts that were mass produced. In this manner, even when a device was broken, the average citizen could easily fix it themselves.

      \item The fire at the Triangle Shirtwaist Factory is probably one of the most famous events that would support Flink's point of view. Instead of trying to support the workers (which then wasn't required, but now is known as worker's compensation), the company tried to make the public forget of this. In this manner, the actions of the Triangle company show Flink's interpretation of the “individual [who] became an anonymous, interchangeable robot” of a factory worker.

    \end{enumerate}

    \newpage

  \item $-$ Question 2

    \begin{enumerate}

      \item The text in the image says that “Mr. Motorist”, as they call the hypothetical man, knows what “Mrs. Motorist” knows and more. By giving the man more knowledge, even in this hypothetical situation, Fisk is subtly claiming male intellectual superiority. As such, it assumes that women need the guidance of men in every day life, including which tires to buy.

      \item Between the period 1900 to 1929, the right of the vote was federally extended to females. The right to vote changed womens' roles both socially and culturally, as female voices now had a greater influence in basically all issues. In this manner, women challenged the “inferior to man” culture established in the image.

      \item One cultural development that influenced such images was the creation of flapper culture. Flappers were women who, unlike before, dressed more revealing and had looser morals. The independence brought about by societal roles such as flappers most likely influences such advertisements in that the advertisements now also mentioned women, though not as much as they valued the male customers.

    \end{enumerate}

    \newpage

  \item $-$ Question \underline{3}

    \begin{enumerate}

      \item The Great Depression, which began for most on Black Tuesday in October of 1929, puts Roosevelt's Fireside Chat into context, as, without understanding that this statement was within the scope of the depression, it wouldn't make sense why Roosevelt is commenting on the recovering industry. As such, it is clear that Roosevelt is referring to the economic disaster of the Great Depression.

      \item One of the most important pieces of legislation to this day is the Glass-Steagall Act of 1933. This act created a differentiation between investment banks and commercial banks. To this day, many banks either say FDIC Insured/Not Insured, which can be used to differentiate between commercial or investment banks, as the former is usually insured while the latter is not.

      \item During the Guilded Age, the government followed through with Adam Smith's \textit{Laissez-Faire} policies, and let the economy play out on its own, whereas, following the Great Depression, the government took on a never-before-seen level of regulation in the economy, which exists to this day. The passage of regulatory acts, which was meant to put a buffer between economic ruin and the citizens of the United States, differed greatly to the anti-interventionist role of the Gilded Age government.

    \end{enumerate}

\end{enumerate}

\end{document}
