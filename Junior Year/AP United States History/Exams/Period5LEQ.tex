
\documentclass[12pt]{article} 
\usepackage{alphalph}
\usepackage[utf8]{inputenc}
\usepackage[russian,english]{babel}
\usepackage{titling}
\usepackage{amsmath}
\usepackage{graphicx}
\usepackage{enumitem}
\usepackage{amssymb}
\usepackage[super]{nth}
\usepackage{everysel}
\usepackage{ragged2e}
\usepackage{geometry}
\usepackage{fancyhdr}
\geometry{top=1.0in,bottom=1.0in,left=1.0in,right=1.0in}
\newcommand{\subtitle}[1]{%
  \posttitle{%
    \par\end{center}
    \begin{center}\large#1\end{center}
    \vskip0.5em}%

}

\usepackage{setspace}
\doublespacing

\usepackage{hyperref}
\hypersetup{
colorlinks=true,
linkcolor=blue,
filecolor=magenta,      
urlcolor=blue,
citecolor=blue,
}

\urlstyle{same}
%\renewcommand*\familydefault{\ttdefault}
%\EverySelectfont{%
%\fontdimen2\font=0.4em% interword space
%\fontdimen3\font=0.2em% interword stretch
%\fontdimen4\font=0.1em% interword shrink
%\fontdimen7\font=0.1em% extra space
%\hyphenchar\font=`\-% to allow hyphenation
%}

\pagestyle{fancy}
\lhead{Michael Brodskiy}
\rhead{Page \thepage}
\chead{APUSH LEQ Final}
\cfoot{}

\begin{document}

\begin{center}
  \underline{Tensions Leading to the Civil War} 
\end{center}

\paragraph{I.} Even since the establishment of the United States of America, it was evident that slavery was going to be a cause of tensions. Although the Constitution did allow acts of prohibition of importation of slaves after 1807, this was not nearly enough to oust slavery from America, especially the South. Compromises such as the Missouri Compromise of 1820 would be quite successful in the short run, but detrimental in the long run, as it only caused more debate and complications. An early form of appeasement would take place, where both sides wanted to maintain the Union, which meant that both would want to compromise to preserve it $-$ that is, until the North wanted to preserve it more than the South. As such, although many of the debates over slavery were intended to heal the wound spreading across America, they had the repercussion of sparking violence and hatred from both sides, which meant that these debates would have a significant effect on sparking the fire that would become the Civil War. Many events, namely the Compromise of 1850, Bleeding Kansas, and the election of 1860 would be the greatest contributors to the raging fire.

\paragraph{II.} Although the issue of slavery was a bit dormant until the Missouri Compromise, and then about two decades more, the Wilmot Proviso, albeit quite unsuccessful in itself, would provoke questions of slavery indirectly. Proposed by David Wilmot, this proviso was supposed to be a play for more territory (much like the Missouri Compromise); however, it held that slavery was to be banned in the territories gained from the Mexican-American war. This made the act extremely unpopular among slaveholders, which led to its rejection. This, however, was not the end. The proviso instigated a lust of more territory, with which came the question of slavery. Because both, the North and the South, wanted more land, each side was willing to compromise to actually obtain this land. As shown by historical trends, compromises made about slavery usually tended to be successful only for a short time. As such, the Compromise of 1850 was born. Proposed by Henry Clay (nicknamed ``The Great Compromiser,'' this was meant to appease both sides, while also acquiring the wanted territories. This act had four main statements: first, California became a free state, while New Mexico was a territory (meaning slavery could not be banned). Second, some of the Texas border was to be given away to New Mexico, but Texas received ten million to repay war debts. Third, the government was to assist in retrieving runaway slaves, and, finally, fourth, slave trade was abolished in the District of Columbia (slavery itself, however, was not). Obviously, this compromise heavily favored the South, which quickly accepted. The third statement, which establish federal support for slavery, would eventually become the Fugitive Slave Act of 1850, which held even more implications. Clearly, the Compromise of 1850 was just that $-$ a compromise; however, this compromise would lead to even more struggles, as the federal government, especially under Franklin Pierce, would interfere with the Underground Railroad. As such, it is clear that this debate, started by a lust for territory, led to increased tensions, and, ultimately, the Civil War. Lust for new land did not stop there, however, as the question of a new, trans-continental railroad meant that slavery was to be a hot topic once again.

  \paragraph{III.} A few years following the Compromise of 1850, two congressmen decided that a railroad would be beneficial for the country. There was a problem, however: more land was needed to construct this railroad. The congressmen turned to the Kansas and Nebraska region. The congressmen did keep slavery in mind, though, as they came up with the idea of popular sovereignty. This meant that the states would be allowed to choose whether they were slave or not, which directly violated the Missouri Compromise. This made the South embrace the Kansas-Nebraska Act of 1854, as it meant more slave states for them. As Kansas came into political existence, Northerners and Southerners rushed to outpopulate each other in order to vote for their beliefs (pro or anti slavery). As expected, politics did not remain peaceful, and both sides committed violence to uphold their beliefs. The South held pseudo-votes, where they declared the state a slave state without word from the Northerners. In addition to this, both sides led attacks on others to ``entice'' them to see slavery their own way. Evidently, this debate, in itself, became a miniature Civil War, as sides battled out to determine the future of the state. In the end, Kansas was admitted as a free state, but not without half a decade of violence, which was nicknamed Bleeding Kansas. Tensions were heating up quickly, but it would not be until the election of 1860 that the South had had enough.

  \paragraph{IV.} The Election of 1860 was quite odd. Both sides wanted their candidate to be victorious, of course, but, at least for the North, their candidate was quite unknown. Abraham Lincoln, formerly a congressman from Illinois, was running for president. As with modern elections, both sides spent large sums of money to push through and propagandize their political candidate. In the end, as is well known, Lincoln won the election. The Southern democrats, unable to admit defeat, decided to secede from the Union, even before Lincoln stepped in as president. On December 20, 1860, South Carolina seceded. It was then followed in January and February by Mississippi, Florida, Alabama, Georgia, Louisiana, and Texas. For the South, no matter how many public statements he made, Lincoln was anti-slave, and, therefore, a threat to their existence. At this point, it was clear that war was imminent, however, like with every other war, no one thought it would last nearly as long as it would, and both sides were ready to fight for the glory of their beliefs. Ergo, it is, without a doubt, obvious that the debates over slavery would be a huge factor in the Civil war.

  \paragraph{V.} All in all, both sides fought fiercely for their beliefs, and no one can blame them for it, as everyone is entitled to their own beliefs. It wouldn't be until the ethical implications of slavery were brought up that people began to question its necessity. As usual, when people had their beliefs questioned, violence, whether physical or philosophical ensued. Therefore, with the Compromise of 1850, the Kansas-Nebraska Act of 1854, and the actual election of Abraham Lincoln, it is evident that the debates over slavery were a big cause of the Civil War.

\end{document}
