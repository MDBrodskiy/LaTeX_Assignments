
\documentclass[12pt]{article} 
\usepackage{alphalph}
\usepackage[utf8]{inputenc}
\usepackage[russian,english]{babel}
\usepackage{titling}
\usepackage{amsmath}
\usepackage{graphicx}
\usepackage{enumitem}
\usepackage{amssymb}
\usepackage[super]{nth}
\usepackage{everysel}
\usepackage{ragged2e}
\usepackage{geometry}
\usepackage{fancyhdr}
\geometry{top=1.0in,bottom=1.0in,left=1.0in,right=1.0in}
\newcommand{\subtitle}[1]{%
  \posttitle{%
    \par\end{center}
    \begin{center}\large#1\end{center}
    \vskip0.5em}%

}

\usepackage{setspace}
\doublespacing

\usepackage{hyperref}
\hypersetup{
colorlinks=true,
linkcolor=blue,
filecolor=magenta,      
urlcolor=blue,
citecolor=blue,
}

\urlstyle{same}
%\renewcommand*\familydefault{\ttdefault}
%\EverySelectfont{%
%\fontdimen2\font=0.4em% interword space
%\fontdimen3\font=0.2em% interword stretch
%\fontdimen4\font=0.1em% interword shrink
%\fontdimen7\font=0.1em% extra space
%\hyphenchar\font=`\-% to allow hyphenation
%}

\pagestyle{fancy}
\lhead{Michael Brodskiy}
\rhead{Page \thepage}
\chead{APUSH SAQ}
\cfoot{}

\begin{document}

\begin{enumerate}

  \item $-$ Question 1

    \begin{enumerate}

      \item The reasons for involvement in the Korean and Vietnam wars for the United States were quite similar. Essentially, the reason for entry into both wars was driven by “Domino Theory” which stated that, if one country fell to the communist ideology, other countries were more likely to join, creating a “Domino” effect. In this manner, the US involved itself in the Korean and Vietnam wars in order to prevent the spread of communism.

      \item The main difference between the two wars was that the Americans held a partial militaristic success in Korea. Unlike the Vietnam war, where Americans had to withdraw completely, leaving some of the communist northerners in the south, the Korean war saw some troops left behind in South Korea. This meant that, unlike the Vietnam war, which was not measured in territory held, the Korean war could actually be measured in land taken.

      \item The Vietnam war held a lasting impact on US society. Most importantly, this war saw major protests, as many young citizens were sent off to fight. The draft was seen in a poor light for many reasons. Aside from just being unfair, the draft often took into account race, wealth, and other factors, making some people more likely to be sent off. As such, many protests later, the draft would be abolished, leaving a lasting impact on society.

    \end{enumerate}

    \newpage

  \item $-$ Question 2

    \begin{enumerate}

      \item One of the most striking similarities between the two women's movements was that they originated from the same groups. Both of the movements were generally started and led by middle class women. Poorer women simply did not have the finances or interest to take part in the movements, and upper class women saw it simply as unnecessary.

      \item One major difference between the two movements was that, from 1890 to 1910, many women argued against alcohol consumption, leading to the passage of prohibition. On the other hand, movements in the 60s and 70s were often about going against the grain, and generally supported substance use, such as marijuana.

      \item One major reason for the difference between the two movements was their place in time. From 1890-1910, these movements took place prior to a major war, whereas, in the 60s and 70s, the movement revolved greatly around the Vietnam war. In fact, most of the societal reactions, including the women's rights movements spawned as a reaction to the Vietnam war. In this manner, the “being different from societal norms” attitude of the 60s and 70s movements revolves around the proximity of these movements to the Vietnam war.

    \end{enumerate}

    \newpage

  \item $-$ Question 3

    \begin{enumerate}

      \item Gaddis describes the state of Cold War as a constant tension in which the move made by one side would yield a response from the other. On the other hand, Westad sees this as a conflict in which Third World countries were decimated, both politically and physically. To summarize, Gaddis sees this as a war between the West and East, while Westad sees this as a war fought in Third World countries.

      \item Gaddis' interpretation is supported by events such as the Cuban Missile Crisis, most active in 1961 and 1962. Prevented by Oleg Penkovsky's clandestine contacts with CIA, the crisis escalated to the hottest point of the Cold War, as each side responded to the other side with diplomatic threats. In this manner, the polarization of sides during this crisis show the two-faction nature of the Cold War.

      \item Westad's interpretation is supported by the fighting of Proxy Wars such as the Korean War. Even though both sides wanted to learn of military secrets about each other, neither wanted to get involved in direct military contact. This war saw American involvement in Korea, which was intentionally done so that the Soviet Union could learn of American military tactics. As such, the fighting of proxy wars supports Westad's point of view.

    \end{enumerate}

\end{enumerate}

\end{document}
