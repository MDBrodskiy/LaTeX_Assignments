
\documentclass[12pt]{article} 
\usepackage{alphalph}
\usepackage[utf8]{inputenc}
\usepackage[russian,english]{babel}
\usepackage{titling}
\usepackage{amsmath}
\usepackage{graphicx}
\usepackage{enumitem}
\usepackage{amssymb}
\usepackage[super]{nth}
\usepackage{everysel}
\usepackage{ragged2e}
\usepackage{geometry}
\usepackage{fancyhdr}
\geometry{top=1.0in,bottom=1.0in,left=1.0in,right=1.0in}
\newcommand{\subtitle}[1]{%
  \posttitle{%
    \par\end{center}
    \begin{center}\large#1\end{center}
    \vskip0.5em}%

}

\usepackage{setspace}
\doublespacing

\usepackage{hyperref}
\hypersetup{
colorlinks=true,
linkcolor=blue,
filecolor=magenta,      
urlcolor=blue,
citecolor=blue,
}

\urlstyle{same}
%\renewcommand*\familydefault{\ttdefault}
%\EverySelectfont{%
%\fontdimen2\font=0.4em% interword space
%\fontdimen3\font=0.2em% interword stretch
%\fontdimen4\font=0.1em% interword shrink
%\fontdimen7\font=0.1em% extra space
%\hyphenchar\font=`\-% to allow hyphenation
%}

\pagestyle{fancy}
\lhead{Michael Brodskiy}
\rhead{Page \thepage}
\chead{APUSH SAQ}
\cfoot{}

\begin{document}

\begin{enumerate}

  \item $-$ Version A

    \begin{enumerate}

      \item The image portrays the movement of Americans across great plains, with mountains and a rising sun in the distance. On top of this, it shows Kentucky in the top left, with California in the bottom right, and the text ``The March of Destiny'' in a ribbon on the top. Therefore, this image perfectly exemplifies the American ideal of \textit{Manifest Destiny}$-$that the great American nation should stretch from coast to coast (Atlantic to Pacific, or vice versa).

      \item One development that aided in the \textit{Manifest Destiny} ideal was the Mexican-American war. This war, beginning as a simple border dispute, saw combat between the US and M\'exico on Mexican soil, from modern-day Texas to the Mexican capital itself. The outcome of this venture was that the US gained significant territories, including Texas (now up to the Rio Grande), parts of modern New Mexico and Arizona, and, later, California. The seizing of these areas facilitated American movement west, which was already occurring at a quick pace.

      \item One major event was the California Gold Rush. While California was barely even American property, gold was discovered at Sutter's Mill in Sacramento. As such, the settlers, which were already flocking to these new areas, became frenzied, and more began to travel to settle there. This gold rush caused an economic boom not only in California, but throughout the US as well, as the immigration numbers grew and grew. The Californian Gold Rush is probably one of the best known results of westward expansion, and definitely one that had the greatest economic effect.

    \end{enumerate}

  \item $-$ Version A

    \begin{enumerate}

      \item One big difference is in manufacturing. While the North began to transition from farming into new and heavier industries, the South continued to farm crops such as indigo, rice, and cotton. As such, a split began to form in the economies of the two regions, as the North became focused on foreign relations and commerce, while the South just wanted more crop output.

      \item Another difference is that the South produced cash crops and unfinished goods, which meant that, the greater the labor pool and land they had, the greater their output, and, therefore, income. As such, the South imported and used large amounts of slaves to support their economy. On the other hand, the North became strictly against slavery, as they began to understand the ethical infractions which it caused. The industries in the North began to rely on some specialization, which caused them to decrease their use of slaves. 

      \item Regardless of their stance on slavery, the North and the South needed each other. The South produced a great deal of cotton, which could be used in the Northern factories to produce clothes for both, the North and the South. As such, the regions became somewhat reliant on each other, and, therefore, domestic commerce, to be able to meet their living requirements.

    \end{enumerate}

\end{enumerate}

\end{document}
