
\documentclass[12pt]{article} 
\usepackage{alphalph}
\usepackage[utf8]{inputenc}
\usepackage[russian,english]{babel}
\usepackage{titling}
\usepackage{amsmath}
\usepackage{graphicx}
\usepackage{enumitem}
\usepackage{amssymb}
\usepackage[super]{nth}
\usepackage{everysel}
\usepackage{ragged2e}
\usepackage{geometry}
\usepackage{fancyhdr}
\geometry{top=1.0in,bottom=1.0in,left=1.0in,right=1.0in}
\newcommand{\subtitle}[1]{%
  \posttitle{%
    \par\end{center}
    \begin{center}\large#1\end{center}
    \vskip0.5em}%

}

\usepackage{setspace}
\doublespacing

\usepackage{hyperref}
\hypersetup{
colorlinks=true,
linkcolor=blue,
filecolor=magenta,      
urlcolor=blue,
citecolor=blue,
}

\urlstyle{same}
%\renewcommand*\familydefault{\ttdefault}
%\EverySelectfont{%
%\fontdimen2\font=0.4em% interword space
%\fontdimen3\font=0.2em% interword stretch
%\fontdimen4\font=0.1em% interword shrink
%\fontdimen7\font=0.1em% extra space
%\hyphenchar\font=`\-% to allow hyphenation
%}

\pagestyle{fancy}
\lhead{Michael Brodskiy}
\rhead{Page \thepage}
\chead{APUSH SAQ}
\cfoot{}

\begin{document}

\begin{enumerate}

  \item $-$ Version C

    \begin{enumerate}

      \item The author (most likely Thomas Nast) clearly portrays William Tweed, the most infamous Boss of the Tammany Hall political machine. It is clear, through the artist's interpretation, that Tweed, by controlling Tammany Hall, was able to influence election outcomes in his favor, which would ultimately gain him more money. In this manner, the artist would most likely want to abolish the political machine system.

      \item The most important development was the rise of political machines. Many machines, which are somewhat equivalent to the modern day SUPERPACs, influenced voting and elections to benefit themselves, although they did give back to the community in the form of jobs. Essentially, political machines funneled their money into contracts and agreements that would get more of their candidates elected, and, therefore, make more money. These kinds of corrupt political machines led to the development shown in the image.

      \item During this period, ``muckraking'', or investigative journalists were on the rise. One example of an influence of a journalist/author was that of Upton Sinclair's novel \textit{The Jungle}. Here, Sinclair portrays the meat industry in gruesome, horrid colors. Having reached as far as Theodore Roosevelt himself, reforms were made to the federal regulation of food, and the Food and Drug Administration was established. 

    \end{enumerate}

  \item $-$ Version C

    \begin{enumerate}

      \item Booker T. Washington, a famous African-American writer and philosopher, explains that he believes that the black men should settle down instead of protesting for change. He believes it is safer to continue working as is, with the \textit{limited} freedom they were given. As such, Washington convinced many freed slaves to stop pushing and protesting for their rights.

      \item Following the Civil War, the formation of the KKK and other violent groups contributed strongly to Washington's perspective. Many African-Americans would be lynched at the slightest error, such as not allowing a white woman to pass first, or not calling a white man sir. As a result, the violent attacks dissuaded many, like Washington, from protesting.

      \item W.E.B Du Bois differed greatly from Washington. Du Bois, unlike Washington, was a strong believer in protesting and fighting for one's rights. Du Bois was Washington's biggest critic, as she questioned whether this country was truly free, as the black population did not truly receive their promised rights. As such, Washington and Du Bois's views were strikingly different.

    \end{enumerate}

\end{enumerate}

\end{document}
