%%%%%%%%%%%%%%%%%%%%%%%%%%%%%%%%%%%%%%%%%%%%%%%%%%%%%%%%%%%%%%%%%%%%%%%%%%%%%%%%%%%%%%%%%%%%%%%%%%%%%%%%%%%%%%%%%%%%%%%%%%%%%%%%%%%%%%%%%%%%%%%%%%%%%%%%%%%%%%%%%%%%%%%%%%%%%%%%%%%%%%%%%%%%
% Written By Michael Brodskiy
% Class: AP US History
% Professor: D. Speir
%%%%%%%%%%%%%%%%%%%%%%%%%%%%%%%%%%%%%%%%%%%%%%%%%%%%%%%%%%%%%%%%%%%%%%%%%%%%%%%%%%%%%%%%%%%%%%%%%%%%%%%%%%%%%%%%%%%%%%%%%%%%%%%%%%%%%%%%%%%%%%%%%%%%%%%%%%%%%%%%%%%%%%%%%%%%%%%%%%%%%%%%%%%%

\documentclass[12pt]{article} 
\usepackage{alphalph}
\usepackage[utf8]{inputenc}
\usepackage[russian,english]{babel}
\usepackage{titling}
\usepackage{amsmath}
\usepackage{graphicx}
\usepackage{enumitem}
\usepackage{amssymb}
\usepackage[super]{nth}
\usepackage{everysel}
\usepackage{ragged2e}
\usepackage{geometry}
\usepackage{fancyhdr}
\usepackage{cancel}
\usepackage{siunitx}
\geometry{top=1.0in,bottom=1.0in,left=1.0in,right=1.0in}
\newcommand{\subtitle}[1]{%
  \posttitle{%
    \par\end{center}
    \begin{center}\large#1\end{center}
    \vskip0.5em}%

}
\usepackage{hyperref}
\hypersetup{
colorlinks=true,
linkcolor=blue,
filecolor=magenta,      
urlcolor=blue,
citecolor=blue,
}

\urlstyle{same}


\title{Jefferson vs. Jackson LEQ}
\date{November 12, 2020}
\author{Michael Brodskiy\\ \small Instructor: Mr. Speir}

% Mathematical Operations:

% Sum: $$\sum_{n=a}^{b} f(x) $$
% Integral: $$\int_{lower}^{upper} f(x) dx$$
% Limit: $$\lim_{x\to\infty} f(x)$$

\begin{document}

    \maketitle

    \paragraph{I} Moving into the eighteenth century, American politics saw the dissolution of the Federalist party, and the ultimate rise of the Democratic-Republican party$-$that is, until the split of the Democratic-Republican party into the Jacksonian Democrats and the National Republicans. The party of the Jacksonian Democrats had views that were similar, although not exactly the same as the Jeffersonian Democratic-Republicans. This was a split which was a direct result of questioning of Jackson's political views, as well as his strife with Calhoun. On the other hand, Jefferson, a staunch believer in Democratic-Republicanism and conservatism, used his power to limit the federal government, as he pared down branches of the already ``complex'' government. Ultimately, although Jackson was a Democratic-Republican in name, and he abolished the national bank, as Jefferson had, Jackson did not embody Jeffersonian ideals, as he used federal power to force Indian removal, passed tariffs on trade, and even opposed states' rights to nullify$-$something he would have supported as a citizen.
 
 
    \paragraph{II} First of all, Andrew Jackson was dedicated in his efforts to remove native Americans from their land. As such, he drew more power to the federal government to have to ability to relocate these natives. As much as Jefferson opposed and despised the constant nagging caused by the Indians, he would never agree to giving more power to the government to rid them of such an enemy. Furthermore, Jefferson even pared down the military at a time when the natives were highly active in western territories$-$something that clearly indicates he did not see the natives as much of a problem as he saw the federal government. Furthermore, during Jackson's presidency it would become clear that Jackson was not nearly as opposed to the federal government as Jefferson was.

    \paragraph{III} Jackson's presidency would show that he was not as concerned about federal power as Jefferson was$-$actually, it would show quite the opposite, as Jackson, as much as his civilian-self may have wanted to limit the government, wanted the states to be more obedient to his actions. Jackson's relations with the south would go down the drain when he passed Adamsonian tariffs which were meant to preserve the industrializing North. These tariffs hurt the South, as it meant less income, which made less happy citizens. The passage of these tariffs was even quite surprising, as Jackson ran on a platform which disagreed with any tariffs, but passed it anyway. Jefferson, on the other hand, opposed any tariffs, and even tried to loosen them, as he was a Southern gentlemen himself, and, therefore, he had an income to protect as well. On top of this, Jackson would add even more power to the federal government.

    \paragraph{IV} The nullification crisis distinctly showed which side Jackson was on, and it became clear that he could no longer be the candidate for the South. This crisis would be a direct result of the passage of the tariffs (which Jackson \textit{should} have been against). The state of South Carolina declared that these tariffs were unconstitutional, and that, therefore, the state was allowed to deviate from them. Jackson, who was somewhat mad with power, threatened the state, as he wanted obedience$-$again, something that Jefferson would strongly oppose, and would even call an overstep of the federal government. In the end, the crisis was somewhat resolved, although tensions remained. Therefore, if Jackson had truly continued the platform on which he ran, instead of flipping sides, he may have been similar to Jefferson, although it did not actually turn out this way.

    \paragraph{V} Overall, it is evident that Jackson was similar to Jefferson's ideology only in name and destruction of the national bank. Jackson's actions clearly demonstrated that, although he may have opposed any additional federal rule over the states as a citizen, he became somewhat thirsty for power when he actually became president$-$something that Jefferson could never agree with or follow. Thus, due to Jackson's actions, relative to Jefferson's, Jackson was not very similar to Jefferson. 

\end{document}

