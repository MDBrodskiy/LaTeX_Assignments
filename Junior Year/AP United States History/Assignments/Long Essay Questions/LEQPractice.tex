%%%%%%%%%%%%%%%%%%%%%%%%%%%%%%%%%%%%%%%%%%%%%%%%%%%%%%%%%%%%%%%%%%%%%%%%%%%%%%%%%%%%%%%%%%%%%%%%%%%%%%%%%%%%%%%%%%%%%%%%%%%%%%%%%%%%%%%%%%%%%%%%%%%%%%%%%%%%%%%%%%%%%%%%%%%%%%%%%%%%%%%%%%%%
% Written By Michael Brodskiy
% Class: AP US History
% Professor: D. Speir
%%%%%%%%%%%%%%%%%%%%%%%%%%%%%%%%%%%%%%%%%%%%%%%%%%%%%%%%%%%%%%%%%%%%%%%%%%%%%%%%%%%%%%%%%%%%%%%%%%%%%%%%%%%%%%%%%%%%%%%%%%%%%%%%%%%%%%%%%%%%%%%%%%%%%%%%%%%%%%%%%%%%%%%%%%%%%%%%%%%%%%%%%%%%

\documentclass[12pt]{article} 
\usepackage{alphalph}
\usepackage[utf8]{inputenc}
\usepackage[russian,english]{babel}
\usepackage{titling}
\usepackage{amsmath}
\usepackage{graphicx}
\usepackage{enumitem}
\usepackage{amssymb}
\usepackage[super]{nth}
\usepackage{everysel}
\usepackage{ragged2e}
\usepackage{geometry}
\usepackage{fancyhdr}
\usepackage{cancel}
\usepackage{siunitx}
\geometry{top=1.0in,bottom=1.0in,left=1.0in,right=1.0in}
\newcommand{\subtitle}[1]{%
  \posttitle{%
    \par\end{center}
    \begin{center}\large#1\end{center}
    \vskip0.5em}%

}
\usepackage{hyperref}
\hypersetup{
colorlinks=true,
linkcolor=blue,
filecolor=magenta,      
urlcolor=blue,
citecolor=blue,
}

\urlstyle{same}


\title{LEQ Practice}
\date{October 29, 2020}
\author{Michael Brodskiy\\ \small Instructor: Mr. Speir}

% Mathematical Operations:

% Sum: $$\sum_{n=a}^{b} f(x) $$
% Integral: $$\int_{lower}^{upper} f(x) dx$$
% Limit: $$\lim_{x\to\infty} f(x)$$

\begin{document}

\begin{enumerate}

    \maketitle

  \item In the early eighteenth century, around 1820, the Second Great Awakening spread across the United States. Although it might not seem relevant, this awakening sparked social reforms and movements for equality and freedom for all. These reform movements, which included early feminism, abolitionism, and alcohol abstinence, were quite successful, as, even if they did not succeed in the short term, they accomplished their goals, as movements for womens' suffrage sparked more of these movements through the US, the spread of abolitionism would lead to tension, and, ultimately, the Civil War, and many senators swore an oath of abstinence from alcohol, as well as the removal of military alcohol rations. 

  \item Moving into the eighteenth century, American politics saw the dissolution of the Federalist party, and the ultimate rise of the Democratic-Republican party$-$that is, until the split of the Democratic-Republican party into the Jacksonian Democrats and the National Republicans. This was a split which was a direct result of questioning of Jackson's political views, as well as his strife with Calhoun. Ultimately, although Jackson was a Democratic-Republican in name, and he abolished the national bank, as Jefferson had, Jackson did not embody Jeffersonian ideals, as he used federal power to force Indian removal, passed tariffs on trade, and even opposed states' rights to nullify$-$something he would have supported as a citizen.

\end{enumerate}

\end{document}

