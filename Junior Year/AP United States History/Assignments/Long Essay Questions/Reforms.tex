%%%%%%%%%%%%%%%%%%%%%%%%%%%%%%%%%%%%%%%%%%%%%%%%%%%%%%%%%%%%%%%%%%%%%%%%%%%%%%%%%%%%%%%%%%%%%%%%%%%%%%%%%%%%%%%%%%%%%%%%%%%%%%%%%%%%%%%%%%%%%%%%%%%%%%%%%%%%%%%%%%%%%%%%%%%%%%%%%%%%%%%%%%%%
% Written By Michael Brodskiy
% Class: AP US History
% Professor: D. Speir
%%%%%%%%%%%%%%%%%%%%%%%%%%%%%%%%%%%%%%%%%%%%%%%%%%%%%%%%%%%%%%%%%%%%%%%%%%%%%%%%%%%%%%%%%%%%%%%%%%%%%%%%%%%%%%%%%%%%%%%%%%%%%%%%%%%%%%%%%%%%%%%%%%%%%%%%%%%%%%%%%%%%%%%%%%%%%%%%%%%%%%%%%%%%

\documentclass[12pt]{article} 
\usepackage{alphalph}
\usepackage[utf8]{inputenc}
\usepackage[russian,english]{babel}
\usepackage{titling}
\usepackage{amsmath}
\usepackage{graphicx}
\usepackage{enumitem}
\usepackage{amssymb}
\usepackage[super]{nth}
\usepackage{everysel}
\usepackage{ragged2e}
\usepackage{geometry}
\usepackage{fancyhdr}
\usepackage{cancel}
\usepackage{siunitx}
\geometry{top=1.0in,bottom=1.0in,left=1.0in,right=1.0in}
\newcommand{\subtitle}[1]{%
  \posttitle{%
    \par\end{center}
    \begin{center}\large#1\end{center}
    \vskip0.5em}%

}
\usepackage{hyperref}
\hypersetup{
colorlinks=true,
linkcolor=blue,
filecolor=magenta,      
urlcolor=blue,
citecolor=blue,
}

\urlstyle{same}


\title{Reform Movements LEQ}
\date{November 12, 2020}
\author{Michael Brodskiy\\ \small Instructor: Mr. Speir}

% Mathematical Operations:

% Sum: $$\sum_{n=a}^{b} f(x) $$
% Integral: $$\int_{lower}^{upper} f(x) dx$$
% Limit: $$\lim_{x\to\infty} f(x)$$

\begin{document}

    \maketitle

    \paragraph{I} In the early nineteenth century, around 1820, the Second Great Awakening spread across the United States. Although it might not seem relevant, this awakening sparked social reforms and movements for equality and freedom for all. These reform movements, which included early feminism, abolitionism, and alcohol abstinence, were quite successful, as, even if they did not succeed in the short term, they accomplished their goals, as movements for womens' suffrage sparked more of these movements through the US, the spread of abolitionism would lead to tension, and, ultimately, the Civil War, and many senators swore an oath of abstinence from alcohol, as well as the removal of military alcohol rations. 
 
    \paragraph{II} One of the most successful results of Second Great Awakening was the creation of feminist movements. This was directly correlated with the religious revival, as, during the Second Great Awakening, a call for the betterment of society and oneself resonated. This caused a reevaluation of ethical standards, as many women began to question whether it was fair that they did not receive the same rights as men. As a result of this, many women, including Lucy Stone, the Grimk\'e Sisters, Elizabeth Cady Stanton, Lucretia Mott, and many more to participate in protests, lobbying, and gatherings. One such example is the meeting at Seneca Falls in 1848, where Stanton and Mott gained much more traction than they had anticipated. From all over the US, people flocked to the event. Although most of it was negative, the event received much publicity, meaning that this was now officially a big issue. As such, it was clear that the Great Awakening had sparked the beginning of such thought, as well as questions about rights of slaves and effects of alcohol.

    \paragraph{III} In addition to the establishment of First-Wave Feminism, the Second Great Awakening would evoke issues regarding the rights of slaves. As with feminism, many began to question the ethical implications of slavery. Although this mostly took place in the North, as the Southern civilians wanted greatly to keep their slaves, many slaves which had escaped through the Underground Railroad would join the new Northern abolitionist forces. One example of an abolitionist is that of William Lloyd Garrison$-$an eccentric man who once burned the Fugitive Slave Act of 1850 in front of an audience. Garrison befriended many slaves, including the famous Frederick Douglass, a former slave-turned-abolitionist. Many others, such as Theodore Dwight Weld, husband to one of the Grimk\'e sisters (who also lobbied against slavery), joined in the fight. As such, it was clear that there was a rising tension between the South and the growing abolitionist force in the North.

    \paragraph{IV} Another, slightly less meaningful issue that was changed during the early to middle nineteenth century was that of alcoholism, as it was rampant across all of the United States. The religious revival led many to believe that alcohol was an evil substance, and that men and women alike should abstain from use. This new belief elicited responses from many Congressmen, as they pledged not to consume alcohol at all. Furthermore, changes in the military ensued. Rations, which used to include some amount of an alcoholic beverage were now removed. Anti-alcohol sentiments would not reach levels as high as it did in the early nineteenth century up until prohibition.

    \paragraph{V} Overall, the early to middle nineteenth century was a time which sparked many reforms and reform movements. Of these movements, first-wave feminism, abolition of slavery, and prohibition were among the most important. Although it wouldn't be until around the late nineteenth to early twentieth century, many of these movements would achieve exactly what they sought to do.

\end{document}

