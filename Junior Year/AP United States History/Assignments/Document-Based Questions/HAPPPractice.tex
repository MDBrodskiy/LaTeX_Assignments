%%%%%%%%%%%%%%%%%%%%%%%%%%%%%%%%%%%%%%%%%%%%%%%%%%%%%%%%%%%%%%%%%%%%%%%%%%%%%%%%%%%%%%%%%%%%%%%%%%%%%%%%%%%%%%%%%%%%%%%%%%%%%%%%%%%%%%%%%%%%%%%%%%%%%%%%%%%%%%%%%%%%%%%%%%%%%%%%%%%%%%%%%%%%
% Written By Michael Brodskiy
% Class: AP US History
% Professor: D. Speir
%%%%%%%%%%%%%%%%%%%%%%%%%%%%%%%%%%%%%%%%%%%%%%%%%%%%%%%%%%%%%%%%%%%%%%%%%%%%%%%%%%%%%%%%%%%%%%%%%%%%%%%%%%%%%%%%%%%%%%%%%%%%%%%%%%%%%%%%%%%%%%%%%%%%%%%%%%%%%%%%%%%%%%%%%%%%%%%%%%%%%%%%%%%%

\documentclass[12pt]{article} 
\usepackage{alphalph}
\usepackage[utf8]{inputenc}
\usepackage[russian,english]{babel}
\usepackage{titling}
\usepackage{amsmath}
\usepackage{graphicx}
\usepackage{enumitem}
\usepackage{amssymb}
\usepackage[super]{nth}
\usepackage{everysel}
\usepackage{ragged2e}
\usepackage{geometry}
\usepackage{fancyhdr}
\usepackage{cancel}
\usepackage{siunitx}
\geometry{top=1.0in,bottom=1.0in,left=1.0in,right=1.0in}
\newcommand{\subtitle}[1]{%
  \posttitle{%
    \par\end{center}
    \begin{center}\large#1\end{center}
    \vskip0.5em}%

}
\usepackage{hyperref}
\hypersetup{
colorlinks=true,
linkcolor=blue,
filecolor=magenta,      
urlcolor=blue,
citecolor=blue,
}

\urlstyle{same}


\title{HAPP Practice}
\date{January 14, 2020}
\author{Michael Brodskiy\\ \small Instructor: Mr. Speir}

% Mathematical Operations:

% Sum: $$\sum_{n=a}^{b} f(x) $$
% Integral: $$\int_{lower}^{upper} f(x) dx$$
% Limit: $$\lim_{x\to\infty} f(x)$$

\begin{document}

    \maketitle

  \item Artist Frederick Remington�s depiction of the 1890 Ghost Dance

    \begin{enumerate}

      \item Historical Situation: In the image, Frederick Remington portrays the process known as the ``Ghost Dance,'' a Native American calling to return back to their roots. Most importantly, this calling spread through the Lakota Sioux, who were relocated to the Black Hills. This movement held that, with return to their ancestral ways, the natives would be freed from influence of the white man, which led many of the forcefully migrated natives to rebel.   

      \item APP: The image depicts many natives dancing, with some looking out at the audience. In addition to this, the title of ``Harper's Weekly Journal'' suggests this was a widespread publication. In this manner, it seems that the intention of this piece is to bring awareness and show how repressed the native population was on their reservations.

    \end{enumerate}

  \item Railroad cars surrounded by many workers

    \begin{enumerate}

      \item Historical Situation: Given the background with railroad cars, it can reasonably be inferred that this image was taken in the middle to late eighteenth century. Furthermore, given the meeting of the two train cars, this is most likely the point at which the two sides of what became the first transcontinental railroad met. This railway line became one of the most important American projects of the nineteenth century, as the interconnectedness and economic opportunities which came with it were unparalleled. 

      \item APP: As is clear with the poses and the amount of people present, this was some kind of celebration. As a result, this image was most likely taken to commemorate and praise those who contributed to this massive project.

    \end{enumerate}

  \item Homesteaders in Nebraska (1887)

    \begin{enumerate}

      \item Historical Situation: Although the migration rate westward was already high, Lincoln, as promised, wanted to stimulate movement, which resulted in the Homestead Act. This act, which guaranteed a 160 acre plot of land that could be sold after five or more years, was successful, with tens, maybe even hundreds of thousands migrating in search for opportunity. Furthermore, this land was often converted into a farm or ranch, which was a huge boost to the American economy.

      \item APP: In reference to the sour expressions the four women hold, this image was most likely taken to show the living conditions of homesteaders. Although many started farms, which is suggested with two horses, doing so was expensive, costing upwards of a thousand dollars. As such, it makes sense that, in the image, the women appear to be living in poor conditions.

    \end{enumerate}

\end{document}

