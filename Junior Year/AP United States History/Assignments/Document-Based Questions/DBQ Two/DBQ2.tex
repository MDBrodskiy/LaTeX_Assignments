%%%%%%%%%%%%%%%%%%%%%%%%%%%%%%%%%%%%%%%%%%%%%%%%%%%%%%%%%%%%%%%%%%%%%%%%%%%%%%%%%%%%%%%%%%%%%%%%%%%%%%%%%%%%%%%%%%%%%%%%%%%%%%%%%%%%%%%%%%%%%%%%%%%%%%%%%%%%%%%%%%%%%%%%%%%%%%%%%%%%%%%%%%%%
% Written By Michael Brodskiy
% Class: AP US History
% Professor: D. Speir
%%%%%%%%%%%%%%%%%%%%%%%%%%%%%%%%%%%%%%%%%%%%%%%%%%%%%%%%%%%%%%%%%%%%%%%%%%%%%%%%%%%%%%%%%%%%%%%%%%%%%%%%%%%%%%%%%%%%%%%%%%%%%%%%%%%%%%%%%%%%%%%%%%%%%%%%%%%%%%%%%%%%%%%%%%%%%%%%%%%%%%%%%%%%

\documentclass[12pt]{article} 
\usepackage{alphalph}
\usepackage[utf8]{inputenc}
\usepackage[russian,english]{babel}
\usepackage{titling}
\usepackage{amsmath}
\usepackage{graphicx}
\usepackage{enumitem}
\usepackage{amssymb}
\usepackage[super]{nth}
\usepackage{everysel}
\usepackage{ragged2e}
\usepackage{geometry}
\usepackage{fancyhdr}
\usepackage{cancel}
\usepackage{siunitx}
\usepackage{chronology}
\usepackage{xcolor}
\geometry{top=1.0in,bottom=1.0in,left=1.0in,right=1.0in}
\newcommand{\subtitle}[1]{%
  \posttitle{%
    \par\end{center}
    \begin{center}\large#1\end{center}
    \vskip0.5em}%

}
\usepackage{hyperref}
\hypersetup{
colorlinks=true,
linkcolor=blue,
filecolor=magenta,      
urlcolor=blue,
citecolor=blue,
}

\urlstyle{same}

\title{Practice DBQ 2}
\date{March 4, 2021}
\author{Michael Brodskiy\\ \small Instructor: Mr. Speir}

\pagestyle{fancy}
\lfoot[\vspace{-15pt} \hline]{\vspace{-15pt} \hline}
\rfoot[\vspace{-15pt} \hline]{\vspace{-15pt} \hline}
\cfoot[\thepage]{\thepage}
\lhead[\copyright 2021 $-$ \textit{All Rights Reserved} ]{\copyright 2021 $-$ \textit{All Rights Reserved}}
\chead[AP United States History]{AP United States History}
\rhead[Michael Brodskiy]{Michael Brodskiy}

\begin{document}

\maketitle

\paragraph{} As the end of the First World War came, the world rejoiced. With the industrial call-to-arms the war came with, this meant an economic boom, which resulted in an age of celebration and worldly withdrawal for the United States. The horrors witnessed by those who fought in the war resulted in their, and eventually an American withdrawal from the world, as Europe's problems seemed far away. Then came the Great Depression, plunging the world into economic withdrawal. Countless reforms later, the US still wasn't even close to its previous economic levels, but there was one thing that could save the deteriorating economy — another war. Although, after the end of the First World War, America and Americans withdrew from foreign problems, America would ultimately end its isolationist policy, as politicians urged the angry public that the country needed to get ready for war — something that public opinion would turn to support when the war hit close to home. 

\paragraph{} As the Great War came to a halt, Americans began to turn away, and even despise the problems of Europe. Woodrow Wilson's proposal of a “League of Nations” saw public outrage, as the public did not even want to hear mentions of Europe. This League, which would be founded on January 10 of 1920, was one of the main reasons for American rejection of the Treaty of Versailles, as isolationism was seen as an American value, and becoming involved in Europe was seen as unpatriotic. Warren Harding's 1920 speech in Iowa demonstrates this sentiment, where he explains that he rejects this idea, and that it is even unconstitutional. Harding's candidate speech is clearly intended for the public, as, by appealing to the public by demonstrating his rejection of European ideals, he hopes that his candidacy would be viewed in a good light, and he would eventually be elected president. In this manner, it is clear that, because Harding was elected president, these anti-globalist words resonated with the American population. Additionally, Charles Hughes' comments on limiting armament, which urge the government to consider limitations, shows the length to which politicians would go to just to prevent involving the nation in foreign affairs. It is clear that the purpose of Hughes' words are not only a rejection of Europe, but a suggestion for future prevention of involvement, as limiting arms would mean that war, at least to the scale it occurred in the Great War, would not be possible. In this manner, it is clear that, during the 20s, America and its citizens were strictly isolationist.

\paragraph{} As dictatorial regimes rose, and Japan began its anti-imperialist imperialism, politicians began and some journalists began to view the world in a different light, although citizens continued to despise war and embrace American isolationism. In 1937, the New York Times published a statement on the events in Nanking. The horrifyingly vivid descriptions denounce Japanese actions and comment on the cruelty of the Japanese troops. Although this is more of an informative piece, and it did little to stir public opinion, politicians began to worry, and some even publicly commented against the actions of the Japanese and in support of China, although little was actually done to support the Chinese troops. At this time, although full involvement did not occur, politicians began to recommend building up the army and defenses, in addition to aiding foreign allies, such as Britain and the Soviet Union. On top of this, a Republican Part Platform from 1940 describes the party's stance on the war: they want to stand with Americanism, which was closely tied with isolationism, and that they did not want involvement in the war — though they recommended securing outposts and bases, as well as fully equipping and mechanizing their army. As such, this statement was intended to define the policy the party was to follow, though it did not make sense how they wanted to build defenses and bases without involvement in the war. In this manner, it is clear that, as America entered into the 40s, politicians began to urge war preparations, whereas they used to side with the public. The public's opinion, however, remained static, as the St. Louis Post-Dispatch advertisement states. This full-page advertisement from September 1940 denounces Roosevelt's “act of war” actions, most likely in reference to the “battleships-for-bases” deal, which occurred early in the same month. The battleships-for-bases deal clearly indicates that politicians, especially Roosevelt himself, was becoming invested in the war, while public criticisms ran high. This is important, as this document displays the hatred the public held for war, as it was meant to be seen by all due to its publication in a newspaper. As such, it is clear that, in the late 30s and the beginning of the 40s, politicians and citizens began to develop very different opinions on foreign policy and warfare.

\paragraph{} Moving into the end of 1940 and into 1941, American citizens still mostly promoted isolationism, although some began to support entry into the war. Many politicians, though, began to promote the war effort through propaganda and other means. One example is the Chicago Daily News cartoon, which shows Uncle Sam lost in a maze of appeasement-like questions and statements that people used to defer America's entry into the war. By making such a statement to the public, it is evident that the Chicago Daily News was trying to make a sarcastic comment on the nation, and promote entry into the war. Still, such efforts were not enough to sway the public. Roosevelt began to comment on events himself, as he mentioned the war and his efforts in his fireside chats and press conferences. Because of a human affinity for analogous abstractions, Roosevelt describes his Lend-Lease Act by giving a hypothetical scenario in which someone's neighbor's house was burning. In this situation, Roosevelt questions whether American citizens would give the hose to their neighbor, and, in this way, save their property. Roosevelt concludes by asking if, given this situation, the public would also give their neighbor a hose and be glad that the neighbor used it. It is clear that such a scenario was brought up by Roosevelt to gain public support for the Lend-Lease act, as, clearly, the lending of the hose represents the lending of military equipment. In this manner, Roosevelt was able to gain some support for his actions. It wouldn't be until the attack of the Japanese thalassocracy on the Hawaiian Pearl Harbor base that public support would truly turn. This “day that would live in infamy” became the turning point for the Americans, as, all of a sudden, the public supported entry, which gave the Roosevelt Administration the go ahead to declare war. In this manner, it is clear that, by the end of 1941, especially due to the events at Pearl Harbor, Americans and America as a whole was ready to go to war.

\paragraph{} Overall, it is evident that, through the 20s and into the 30s, the American government was strictly against armament and warfare. As time moved into the late 30s, and into early 1940, though, it became apparent to the American politicians that building defenses, harnessing resources, and gaining public support would be necessary, as war, which seemed far away in Europe and the Pacific, was actually much closer than America thought. With the bombing of Pearl Harbor, it became clear that America was no longer isolationist, and that it must now switch to an interventionist attitude, and, with Roosevelt's announcement of war, America stepped into World War II.

\end{document}

