%%%%%%%%%%%%%%%%%%%%%%%%%%%%%%%%%%%%%%%%%%%%%%%%%%%%%%%%%%%%%%%%%%%%%%%%%%%%%%%%%%%%%%%%%%%%%%%%%%%%%%%%%%%%%%%%%%%%%%%%%%%%%%%%%%%%%%%%%%%%%%%%%%%%%%%%%%%%%%%%%%%%%%%%%%%%%%%%%%%%%%%%%%%%
% Written By Michael Brodskiy
% Class: AP US History
% Professor: D. Speir
%%%%%%%%%%%%%%%%%%%%%%%%%%%%%%%%%%%%%%%%%%%%%%%%%%%%%%%%%%%%%%%%%%%%%%%%%%%%%%%%%%%%%%%%%%%%%%%%%%%%%%%%%%%%%%%%%%%%%%%%%%%%%%%%%%%%%%%%%%%%%%%%%%%%%%%%%%%%%%%%%%%%%%%%%%%%%%%%%%%%%%%%%%%%

\documentclass[12pt]{article} 
\usepackage{alphalph}
\usepackage[utf8]{inputenc}
\usepackage[russian,english]{babel}
\usepackage{titling}
\usepackage{amsmath}
\usepackage{graphicx}
\usepackage{enumitem}
\usepackage{amssymb}
\usepackage[super]{nth}
\usepackage{everysel}
\usepackage{ragged2e}
\usepackage{geometry}
\usepackage{fancyhdr}
\usepackage{cancel}
\usepackage{siunitx}
\usepackage{chronology}
\usepackage{xcolor}
\usepackage{soul}
\geometry{top=1.0in,bottom=1.0in,left=1.0in,right=1.0in}
\newcommand{\subtitle}[1]{%
  \posttitle{%
    \par\end{center}
    \begin{center}\large#1\end{center}
    \vskip0.5em}%

}
\usepackage{hyperref}
\hypersetup{
colorlinks=true,
linkcolor=blue,
filecolor=magenta,      
urlcolor=blue,
citecolor=blue,
}

\urlstyle{same}

\title{Practice DBQ 1 (Rewrite)}
\date{March 1, 2020}
\author{Michael Brodskiy\\ \small Instructor: Mr. Speir}

\pagestyle{fancy}
\lfoot[\vspace{-15pt} \hline]{\vspace{-15pt} \hline}
\rfoot[\vspace{-15pt} \hline]{\vspace{-15pt} \hline}
\cfoot[\thepage]{\thepage}
\lhead[\copyright 2021 $-$ \textit{All Rights Reserved} ]{\copyright 2021 $-$ \textit{All Rights Reserved}}
\chead[AP United States History]{AP United States History}
\rhead[Michael Brodskiy]{Michael Brodskiy}

\begin{document}

\maketitle

\paragraph{} As America moved into its second Industrial Revolution, mighty tycoons, such as Carnegie, Rockefeller, and JP Morgan practiced predatory capitalism to establish absolute wealth and power. The influence exerted by such business empires was a never before seen level of interaction between private corporations and government entities, which frightened many civilians, as it only strengthened the position of the businessmen. Concepts such as ``buy low, sell high'', ``cutting costs'', and paying workers the minimum price possible made such monopolies extremely profitable, but kept those who worked for them in factories, mines, or even towns established by the companies at just the right level to prevent protest; however, oftentimes, protests still occurred. Eventually, the urban workers began to notice the tactics of the monopolistic empires, and they called for change. Although the corporations were able to maintain control for a long period due to their interaction with government employees, which caused rampant corruption, the people campaigned and worked hard for change, which caused a significant shift from their initial position. Most of the aforementioned change was accomplished through making corruption known, which erupted into protests and organizations, in addition to the election of Progressive politicians.

\paragraph{} First of all, reform work during this period was successful due to something that had not existed during earlier years: interconnectedness. With the second industrial revolution came railroads, telegraphs, and even phones, which allowed for a level of networking that had never been seen before. Joseph Keppler's, ``\textit{The Bosses of the Senate}'' depicts the greedy, corrupt trusts that gained wealth through government ties, standing in the senate, while the doors are closed to the public. The audience of the cartoon is key, as it demonstrates not only the corruption present at the time, but also how such simple formats conveyed Progressive ideals to the masses. Clearly the image was aimed at the public, especially white-collar and urban workers whose lives depended on greediness of the trusts, as, if they wanted to, they could simply raise prices on any product they wished. In this manner, it was important for people to know the state of politics, and such images did the job well. In addition to this, ``muckraking'' journalists were on the rise during this period. Such journalists ``raked the mud'' of corruption, and brought Americans the facts about industry. \hl{Any change would have been virtually impossible without the key ingredient of information. When uninformed, people tend to believe anything given to them at face value, which results in an easily manipulated population; however, with the rise of muckrakers, people were now able to judge and criticize corrupt politicians, which kickstarted the reform movements (one example of which is the publishing of Sinclair's \textit{The Jungle} which shook the meat industry and led to the FDA)}.  Additionally, Florence Kelley demonstrates a similar concept in her pamphlet, as she writes about the decrease in school populations over the previous five years. The rest of her quote holds analysis and argumentation on why and how this has occurred. Again, intended for the public to read, people like Kelley hoped to spark public outrage at such issues, in order to cause reform\hl{, as any legislation presented is useless without public support and backing}. Finally, George Rice also demonstrates his intent to disseminate information when he discusses his experiences with Rockefeller. He writes about how Rockefeller practiced predation of other companies by uniting, or even buying out railroads. He would charge other companies high prices, so they would have to charge more for oil, while Rockefeller himself charged much less. In this manner, he put many of business, or even bought them out for dirt cheap. Most important is to realize that, although each of these sources uses the same technique to spread awareness of different topics, all of these topics are actually interconnected, as they relate back to the rich monopolists in some way, shape, or form. In addition to spreading information, as the first three sources did, many movements began to advertise themselves to keep people who had read about issues interested.

\paragraph{} Both for large and small organizations, marketing was on the up an up during the time. After information on issues was given to the public, it was necessary for the public to unite in order to achieve actual change. This was largely done through poster advertisements that would be hung up around cities. For example, the American Publishing Company print, depicting an Anti-Saloon League cartoon shows a call for people to join. Although, like the previous cartoon, it is intended for the public, the purpose is different in that, instead of informing, it is meant to organize and unite the masses, in order to call them into action. As the temperance movement was big during the time, many women, more specifically wives, begged for the right to vote, claiming that it was necessary to protect themselves from their drunk, abusive husbands who came back from work late. This movement, which garnered hundreds of thousands of followers, as is clear in the image, begged those who could vote to vote against pro-alcohol politicians. \hl{Additionally, Prohibition signifies the extent to which campaigns were able to disseminate information and garner support. The passage of the eighteenth amendment, and the Volstead Act, demonstrate that reform campaigns weren't just hollow promises, but actually effective ways of making change on a federal scale.} In this manner\hl{, with the Anti-Saloon League Poster and Prohibition taken in tandem,} the success of the Temperance Movement is inarguable, as there was actual legislation passed that supported their position, which could not have been done without support from allied politicians.

\paragraph{} In addition to the new level of interconnectedness throughout the United States, reforms would not have been successful without the support of politicians, namely T. Roosevelt, Taft, and Wilson, who were known as trustbusters due to their ``busting'' of major trusts. First in line came Roosevelt, whose anti-trust sentiment is evident in his speech at Providence, where he describes trusts as creatures belonging to the government, and that, therefore, the government must control them, when necessary. In this manner, he would use the 1890 Antitrust Act, \hl{passed as a result of major protests against monopolies,} to take down the newly-formed Northern Securities Company Limited, which was a joint agreement between three major railroad owners, most prominently JP Morgan. As such, given his speech and his actions, Roosevelt was one of the rare presidents who falls under the ``promises made, promises kept'' category. \hl{As such, even though T. Roosevelt's values dead lead him to ``bust'' the trusts, his efforts would have been meaningless without the legislation passed due public appeal, which goes to show that the public was responsible for the power needed to take down the monopolists}.  Additionally, Hiram Johnson, a Progressive governor of California, stood against corruption and control of the government by private entities. As such, his inaugural address stated that government should only consist of those who serve the public and no one else. This speech, aimed at the corrupt politicians in the Assembly of California, was a direct threat to their careers. \hl{Again, as with T. Roosevelt's efforts, it is clear that the public is responsible for the change, as they saw the progressive values held by Hiram Johnson, and they decided to elect officials for the betterment of society}. Additionally, organizations and reform movements commonly reached out to public figures to further their beliefs. The NAACP reached out to Woodrow Wilson to protest his segregation of ``Departments in Washington''. Clearly the intent of this letter is to further the rights of African-Americans. Although this wouldn't officially be recognized until around the 1950s and 60s, small steps such as this letter to Wilson furthered their beliefs, as the NAACP grew from the smaller Niagara Movement, instigated by W.E.B. Du Bois. Overall, it is clear that government alliances were an important part of reform movements, as nothing could be done without government intervention.

\paragraph{} As is clear, the reform movements of the late nineteenth and early twentieth centuries were quite successful, as they followed three important steps: First, they had to inform the public of issues. Then, the public would need to be organized into movements and groups which allowed them to unify their power. Finally, the organizations, now more powerful than they were when the people were separate, petitioned and worked with government officials to actually achieve change. In this way, even though not all reform movements were successful in the short term, those that were not successful in the short term became precursors to future rights movements.

\end{document}

