%%%%%%%%%%%%%%%%%%%%%%%%%%%%%%%%%%%%%%%%%%%%%%%%%%%%%%%%%%%%%%%%%%%%%%%%%%%%%%%%%%%%%%%%%%%%%%%%%%%%%%%%%%%%%%%%%%%%%%%%%%%%%%%%%%%%%%%%%%%%%%%%%%%%%%%%%%%%%%%%%%%%%%%%%%%%%%%%%%%%%%%%%%%%
% Written By Michael Brodskiy
% Class: AP US History
% Professor: D. Speir
%%%%%%%%%%%%%%%%%%%%%%%%%%%%%%%%%%%%%%%%%%%%%%%%%%%%%%%%%%%%%%%%%%%%%%%%%%%%%%%%%%%%%%%%%%%%%%%%%%%%%%%%%%%%%%%%%%%%%%%%%%%%%%%%%%%%%%%%%%%%%%%%%%%%%%%%%%%%%%%%%%%%%%%%%%%%%%%%%%%%%%%%%%%%

\documentclass[12pt]{article} 
\usepackage{alphalph}
\usepackage[utf8]{inputenc}
\usepackage[russian,english]{babel}
\usepackage{titling}
\usepackage{amsmath}
\usepackage{graphicx}
\usepackage{enumitem}
\usepackage{amssymb}
\usepackage[super]{nth}
\usepackage{everysel}
\usepackage{ragged2e}
\usepackage{geometry}
\usepackage{fancyhdr}
\usepackage{cancel}
\usepackage{siunitx}
\usepackage{chronology}
\usepackage{xcolor}
\geometry{top=1.0in,bottom=1.0in,left=1.0in,right=1.0in}
\newcommand{\subtitle}[1]{%
  \posttitle{%
    \par\end{center}
    \begin{center}\large#1\end{center}
    \vskip0.5em}%

}
\usepackage{hyperref}
\hypersetup{
colorlinks=true,
linkcolor=blue,
filecolor=magenta,      
urlcolor=blue,
citecolor=blue,
}

\urlstyle{same}

\title{Period Six Timeline Write-up}
\date{February 9, 2020}
\author{Michael Brodskiy\\ \small Instructor: Mr. Speir}

\definecolor{maroon}{rgb}{0.5, 0.0, 0.0}
\definecolor{gold}{rgb}{0.72, 0.53, 0.04}
\definecolor{darkmidnightblue}{rgb}{0.0, 0.2, 0.4}
\definecolor{darkolivegreen}{rgb}{0.33, 0.42, 0.18}

\pagestyle{fancy}
\lfoot[\vspace{-15pt} \hline]{\vspace{-15pt} \hline}
\rfoot[\vspace{-15pt} \hline]{\vspace{-15pt} \hline}
\cfoot[\thepage]{\thepage}
\lhead[\copyright 2021 $-$ \textit{All Rights Reserved} ]{\copyright 2021 $-$ \textit{All Rights Reserved}}
\chead[AP United States History]{AP United States History}
\rhead[Michael Brodskiy]{Michael Brodskiy}

\begin{document}

\maketitle

\begin{itemize}

  \item \textcolor{maroon}{Homestead Act Passes (1862)} $-$ The Homestead Act, promised prior to election by Lincoln, was a government initiative that directly promoted westward movement. This act promised settlers 160 acres of land, which, after five years, could be sold. Obviously, many wanted to take the government up on this offer, and, as a result, people poured into the western territories. Evidently, this had a great impact over the amount of people moving, and, consequently, increased the frequency with which contact with Native Americans was had. In this manner, not only did this act promote westward expansion, but it also resulted in greater conflict with the natives.

  \item \textcolor{maroon}{Construction of First Transcontinental Railroad (1863$-$1869)} $-$ The transcontinental railroad was, arguably, the greatest creation of the nineteenth century, as it held an impact on economy, politics, and society as a whole. First of all, it brought a new level of interconnectedness, not only domestically, but worldwide. Products could now be shipped in refrigerated train cars all across the US, which made it easier to ship products from different coasts. Additionally, the new shipping abilities resulted in the rise of new industries, such as cattle ranching. The new ``cowboys'' rose up, as they would take cattle on long, cross country drives. As such, the railroad held a huge impact over the American nation.

  \item \textcolor{maroon}{Carlisle Indian School Constructed (1879)} $-$ As if the persecution placed upon the Native Americans was already not enough, the Carlisle Indian School (and later, many more) opened up. These schools opened under the pretense that they wanted to educate the Native American population. In reality this meant that they wanted to ``kill the Indian, leave the man'', or, in other words, assimilate the natives into their culture. This yielded poor results for the natives, as they strayed further from their tribal ways.

  \item \textcolor{maroon}{Dawes Act Passed (1887)} $-$ On top of the Carlisle school, the government passed the Dawes Act. Essentially, this was like the Homestead Act, but targeted towards the Native American population; however, there was one major difference between the Dawes and Homestead Acts: Natives could only sell after 25, not 5 years. This had two major effects on the natives: first, it shattered their tribal way of life, as it separated the natives onto their own properties. In addition to this, natives lost millions of acres of land, as the 160 acres per native was much less than the reservations which had previously been organized.

  \item \textcolor{maroon}{Ghost Dance Initiated (1890)} $-$ With all the pressure on the native population, response came. Started by a Sioux shaman, the Ghost Dance spread among the population. This Ghost Dance was a call for a return to their tribal ways. It said that, if the natives were to get rid of the western ways, and away from the white man, then they would be able to return to their old ways, with plenty of bison and buffalo on the plains. This call frightened many whites, and it resulted in a massacre.

  \item \textcolor{gold}{Moody Starts Religious Revival (1875)} $-$ With movement into a new, more industrial age, came religious reform. Unlike previous religious shifts, this movement was more social and personal. Dwight L. Moody perfectly embodied the religious shift, as he took on a more religious, ``business-man attitude.'' This shift spread across the United States, and tripled the following of the Protestant church over a forty year period.

  \item \textcolor{gold}{Chinese Exclusion Act Passed (1882)} $-$ As more and more people flooded into the United States, politicians called for immigration control. One of the greatest immigrating groups was the Chinese, especially near the west coast around San Francisco. Many people were worried that the Chinese would take all of their jobs, and, as such, they petitioned Congress to do something. This resulted in the passage of the Chinese Exclusion Act, which made immigration to the United States near impossible for the Chinese. This significantly disrupted the migration patterns of the Chinese population.

  \item \textcolor{gold}{Nikola Tesla Discovers Alternating Current (1886$-$1888)} $-$ During this inventive period, one of the most important inventions was Nikola Tesla's alternating current. Although it was described as ``the current that kills'' by businessman Thomas Edison, it was actually significantly more advantageous than Edison's direct current. Tesla's current reversed direction periodically, which was not only more efficient, but also made it so if one lamp went out in a house, they didn't all go out. In this manner, Tesla was able to better the world, even though he never received much financial gain from it.

  \item \textcolor{gold}{Ellis Island Opens (1892)} $-$ The largest and most important immigration center in the United States would be Ellis Island. Near a port of entry in New York, Ellis Island would have workers examine immigrants for diseases and administer basic tests. Upon passing such examinations and filling out paperwork, people would be let into the country. Ellis Island, along with its western counterpart, Angel Island, would be etched into the history of the United States due to their profound impact.

  \item \textcolor{gold}{Wright Brothers' First Flight (1903)} $-$ One of the most well known inventions of the Gilded Age is the Wright Brothers' Flying Machine. Although many cynics and critics did not believe that flight would be achieved for another hundred years, the Wright Brothers' proved them wrong. Although the effects were not instant, as the mechanisms needed to be perfected, the Wright Brothers' would end up connecting the world even more so than the Transcontinental Railroad did, as, within half a century, people would be able to fly to various other countries in hours, rather than days.

  \item \textcolor{darkmidnightblue}{Farmers' Alliance Forms (1870s)} $-$ The Farmers' Alliance was one of the most important precursors to unions. Similar to the soon-to-be-established unions in industrial cities, the purpose of the Farmers' Alliance was to unify the working people in order to receive better benefits for all. Goals included: lowering shipping costs, lowering tool prices, and making access to loans easier. Additionally, the greatest effect of the Farmers' Alliance was that it would later lead to the creation of the Populist Party.

  \item \textcolor{darkmidnightblue}{Events at Haymarket Square Perspire (1886)} $-$ Haymarket would become one of the most important events for the working class, as well as socialists and anarchists. When socialists and anarchists met at Haymarket square to protest, a bomb was thrown at a crowd of police officers. Albert Parsons and August Spies, were teo of the men convicted for the crime, although it was not them. They were later executed, which caused further protest and anger from the public.

  \item \textcolor{darkmidnightblue}{Populist Party forms at Ohio Convention (1891)} $-$ Many workers, blue or white collar, wanted change to take place. This led to a combination of several unions and workers' alliances into the People's Party, better known as the Populist Party. This party advocated for reform of industry for the betterment of the people, in addition to abolishing the gold standard. With William Jennings Bryan as the chosen representative of the party, the Populists hoped to win the 1896 election, as they had gained steam like no party had in decades, since the Republican Party. The failure to elect Bryan in 1896 led to almost complete dissolution of the party.

  \item \textcolor{darkmidnightblue}{Industrial Workers of the World (IWW) Founded (1905)} $-$ The Industrial Workers of the World, or the IWW, was supposed to be an international union for the laborers of the world. The IWW cared greatly about its cause, and went to great lengths to advertise it. A song book of working songs was printed and spread around. Often, this organization was attacked by the government, press, and large corporations, yet, for 15 years, it remained a force to be reckoned with.

  \item \textcolor{darkmidnightblue}{NAACP Emerges from Niagara Movement (1909)} $-$ With leaders like W.E.B Du Bois and Ida Tarbell came movements for civil rights and the betterment of society. Du Bois aided in the creation of the Niagara Movement, which formed into the National Association for the Advancement of Colored People. This group fought for true interpretations of the fourteenth and fifteenth amendments, and exists to this day.

  \item \textcolor{darkolivegreen}{Pendleton Reform Act (1883)} $-$ Shortly after the assassination of President Garfield, the Pendleton Civil Service Reform Act was passed. This act made it so that federal government employees were hired based on merit, rather than patronage. Additionally, the Civil Service Commission was formed, which was meant to help employees keep their jobs. The act was a bit unsuccessful though, albeit it did cover about ten percent of federal workers.

  \item \textcolor{darkolivegreen}{Sherman Antitrust Act (1890)} $-$ Although the Sherman Antitrust Act was not actually used for about a decade after its passage, it would later become an extremely important piece of legislation. Through it, Theodore Roosevelt was able to sue the Northern Securities Company Limited, which was a large collection of railroads, which involved J.P. Morgan, as it was created to raise shipping and traveling costs. Additionally, following the Roosevelt Administration, Taft and Wilson would use the act much more than Roosevelt.

  \item \textcolor{darkolivegreen}{McKinley Assassinated (1901)} $-$ Although McKinley's assassination itself wasn't too important of an event, it did lead to the rise of Theodore Roosevelt as president. Tom Platt, Republican Party Boss, wanted to prevent Roosevelt from becoming president, as Roosevelt actually kept his promises. In this manner, Platt tried to make Roosevelt vice president, and he succeeded. That is, until McKinley was assassinated, and Roosevelt became president anyway. As such, Roosevelt became known for his ecological conservationism and his trust-busting stance.

  \item \textcolor{darkolivegreen}{Wilson Elected President (1912)} $-$ Wilson was elected president in 1912. Wilson was quite economically and financially active as a president, as he signed the Federal Trade Commission into action. In other sectors, he did go back on some promises, such as not signing a bill against child labor or a bill aiding farmers in getting loans. Unlike other presidents, though, Wilson would have to be president during a war $-$ the First World War.

  \item \textcolor{darkolivegreen}{FTC is Created (1914)} $-$ The Federal Trade Commission (FTC) was created by Woodrow Wilson. The main reason for the creation of this commission was to combat unfair monopolies that destroyed smaller businesses, and squander the meager funds that workers had. This commission pleased many worker rights' activists, and, to this day, the FTC exists. 

\end{itemize}

\end{document}

