%%%%%%%%%%%%%%%%%%%%%%%%%%%%%%%%%%%%%%%%%%%%%%%%%%%%%%%%%%%%%%%%%%%%%%%%%%%%%%%%%%%%%%%%%%%%%%%%%%%%%%%%%%%%%%%%%%%%%%%%%%%%%%%%%%%%%%%%%%%%%%%%%%%%%%%%%%%%%%%%%%%%%%%%%%%%%%%%%%%%%%%%%%%%
% Written By Michael Brodskiy
% Class: AP US History
% Professor: D. Speir
%%%%%%%%%%%%%%%%%%%%%%%%%%%%%%%%%%%%%%%%%%%%%%%%%%%%%%%%%%%%%%%%%%%%%%%%%%%%%%%%%%%%%%%%%%%%%%%%%%%%%%%%%%%%%%%%%%%%%%%%%%%%%%%%%%%%%%%%%%%%%%%%%%%%%%%%%%%%%%%%%%%%%%%%%%%%%%%%%%%%%%%%%%%%

\documentclass[12pt]{article} 
\usepackage{alphalph}
\usepackage[utf8]{inputenc}
\usepackage[russian,english]{babel}
\usepackage{titling}
\usepackage{amsmath}
\usepackage{graphicx}
\usepackage{enumitem}
\usepackage{amssymb}
\usepackage[super]{nth}
\usepackage{everysel}
\usepackage{ragged2e}
\usepackage{geometry}
\usepackage{fancyhdr}
\usepackage{cancel}
\usepackage{siunitx}
\geometry{top=1.0in,bottom=1.0in,left=1.0in,right=1.0in}
\newcommand{\subtitle}[1]{%
  \posttitle{%
    \par\end{center}
    \begin{center}\large#1\end{center}
    \vskip0.5em}%

}
\usepackage{hyperref}
\hypersetup{
colorlinks=true,
linkcolor=blue,
filecolor=magenta,      
urlcolor=blue,
citecolor=blue,
}

\urlstyle{same}


\title{Alan Taylor Reading Questions}
\date{September 2, 2020}
\author{Michael Brodskiy\\ \small Instructor: Mr. Speir}

% Mathematical Operations:

% Sum: $$\sum_{n=a}^{b} f(x) $$
% Integral: $$\int_{lower}^{upper} f(x) dx$$
% Limit: $$\lim_{x\to\infty} f(x)$$

\begin{document}

\begin{enumerate}

  \item Why did planters in Virginia shift from owning white indentured servants to black slaves?

    \begin{justify} The main reason for this switch was the decreasing numbers of indentured servants. This may be attributed to many things, such as: increasing wages, tension between common whites and planters, and immigration to other colonies. Of these, the most important reason was that immigrants from Europe preferred other colonies, such as Jamaica, Carolina, and Pennsylvania. As a result, the number of indentured servants declined. This is shown in trend from 1660 to 1680, where the amount of servants dropped from 18,000 to 13,000. As a result, the planters began to import African slaves in larger quantities. \end{justify}

  \item Why did rich white plantation owners need the support of poor, non-slaveholding whites?

    \begin{justify} The influx of African slaves had caused the population of slaves to rise rapidly. Because of this, the rich, white plantation owners feared there would be a rebellion. They decided that, for this reason, they needed to rally a militia. They decided to draw from the poorer white class to formulate a small army to combat a possible rebellion. \end{justify}

  \item What did the colonial government of Virginia do to create a sense of white supremacy?

    \begin{justify} The colonial governments needed a way to control, and, therefore subdue the African slave population. As a result, they came up with legal codes to keep the Africans in place. This code, however, was not only meant for slaves, but free blacks too. They barred Africans from practicing their culture, such as speaking their language or beating drums. More laws were later passed that forbid slaves from free travel. It required that they receive a pass to travel away from their plantations. These laws made it so that the white population did not view the Africans as people, meaning that they were animals. This created a general sense of white supremacy amongst the white population. \end{justify}

  \item Who was Anthony Johnson and why does his family history help prove that racism in Virginia was created intentionally?

    \begin{justify} Anthony Johnson was the most successful black freedman. He acquired a 250 acre tobacco plantation. He himself owned slaves, proving that, initially, slavery was not a matter of race. A neighbor of Johnson's once lured one of Johnson's slaves away from him. In response, Johnson sued his neighbor, and won. As a result of legislation enacted in (3), Johnson's grandchildren moved away from Virginia, for fear of becoming slaves. This goes to show that, initially, slavery was not pointed at a certain race, whereas, in a few decades, a system of racism would be set up. \end{justify}

  \item What effect did the establishment of white supremacy have on the relationship between rich and poor whites?

    \begin{justify} In terms of the whites, white supremacy was a unifying factor for the rich and the poor. One piece of legislation forbid black men from owning livestock. If livestock were found, it would be redistributed to the poorer whites. This caused many whites to testify against their black neighbors, as they received a reward for it. This means that, although the economic gap between the rich and poor widened, the social tensions dissipated, and brought them closer.   \end{justify}

\end{enumerate}

\end{document}

