\documentclass[a4paper]{article} 
\usepackage{tcolorbox}
\usepackage[super]{nth}
\tcbuselibrary{skins}

\title{
\vspace{-3em}
\begin{tcolorbox}[colframe=white,opacityback=0]
\begin{tcolorbox}
\Huge\sffamily AP US History Chapter 2 Notes
\end{tcolorbox}
\end{tcolorbox}
\vspace{-3em}
}

\date{}

\usepackage{background}
\SetBgScale{1}
\SetBgAngle{0}
\SetBgColor{grey}
\SetBgContents{\rule[0em]{4pt}{\textheight}}
\SetBgHshift{-2.3cm}
\SetBgVshift{0cm}

\usepackage{lipsum}% just to generate filler text for the example
\usepackage[margin=2cm]{geometry}

\usepackage{tikz}
\usepackage{tikzpagenodes}

\parindent=0pt

\usepackage{xparse}
\DeclareDocumentCommand\topic{ m m g g g g g}
{
\begin{tcolorbox}[sidebyside,sidebyside align=top,opacityframe=0,opacityback=0,opacitybacktitle=0, opacitytext=1,lefthand width=.3\textwidth]
\begin{tcolorbox}[colback=red!05,colframe=red!25,sidebyside align=top,width=\textwidth,before skip=0pt]
#1\end{tcolorbox}%
\tcblower
\begin{tcolorbox}[colback=blue!05,colframe=blue!10,width=\textwidth,before skip=0pt]
#2
\end{tcolorbox}
\IfNoValueF {#3}{
\begin{tcolorbox}[colback=blue!05,colframe=blue!10,width=\textwidth]
#3
\end{tcolorbox}
}
\IfNoValueF {#4}{
\begin{tcolorbox}[colback=blue!05,colframe=blue!10,width=\textwidth]
#4
\end{tcolorbox}
}
\IfNoValueF {#5}{
\begin{tcolorbox}[colback=blue!05,colframe=blue!10,width=\textwidth]
#5
\end{tcolorbox}
}
\IfNoValueF {#6}{
\begin{tcolorbox}[colback=blue!05,colframe=blue!10,width=\textwidth]
#6
\end{tcolorbox}
}
\IfNoValueF {#7}{
\begin{tcolorbox}[colback=blue!05,colframe=blue!10,width=\textwidth]
#7
\end{tcolorbox}
}
\end{tcolorbox}
}

\def\summary#1{
\begin{tikzpicture}[overlay,remember picture,inner sep=0pt, outer sep=0pt]
\node[anchor=south,yshift=-1ex] at (current page text area.south) {% 
\begin{minipage}{\textwidth}%%%%
\begin{tcolorbox}[colframe=white,opacityback=0]
\begin{tcolorbox}[enhanced,colframe=black,fonttitle=\large\bfseries\sffamily,sidebyside=true, nobeforeafter,before=\vfil,after=\vfil,colupper=black,sidebyside align=top, lefthand width=.95\textwidth,opacitybacktitle=1, opacitytext=1,
segmentation style={black!55,solid,opacity=0,line width=3pt},
title=Summary
]
#1
\end{tcolorbox}
\end{tcolorbox}
\end{minipage}
};
\end{tikzpicture}
}

\usepackage{graphicx}
\usepackage{physics}
\usepackage{amsmath}
\usepackage{tikz}
\usepackage{mathdots}
\usepackage{yhmath}
\usepackage{cancel}
\usepackage{color}
\usepackage{siunitx}
\usepackage{array}
\usepackage{multirow}
\usepackage{amssymb}
\usepackage{gensymb}
\usepackage{tabularx}
\usepackage{booktabs}
\usetikzlibrary{fadings}
\usetikzlibrary{patterns}
\usetikzlibrary{shadows.blur}
\usetikzlibrary{shapes}

\begin{document} 
\maketitle

\topic{Did the Tainos understand that they were being held captive? Did they want to travel to a different land?}{Because Columbus wanted to teach them Spanish and show off to his sponsors, he took some of the Tainos back to Spain. In addition to this, Columbus forced many Tainos to find gold. When they returned with nothing, he would force them to labor until either they found some or died trying.}%

\topic{Did Columbus exert forces of cruelty and violence on the natives because he wanted to conquer them, or did he want to get gold and riches, no matter the price (or both)?}{Upon his return to Hispaniola in 1494, Columbus discovered that the men who were left behind were murdered. Although the local chief at Hispaniola firstly wanted to offer protection and kindness to the Europeans, the exploitation of his people forced them to rebel. News of newcomers spread among tribes, and attitudes towards the Europeans became hostile. This resulted in some deserted villages.}%

\topic{Was this signed as a result of competition (and possibly violence) or to prevent any in the future?}{\underline{\textbf{Treaty of Tordesillas}} $-$ An agreement signed between Spain and Portugal in 1494. The pope drew a line in South America to split it between Spain and Portugal (in Brazil)}%

\topic{Why wasn't it named like other cities (new followed by the country it was discovered by)?}{\underline{\textbf{America}} $-$ The new continent named after \underline{Amerigo} Vespucci. It was named by Martin Waldseem\"uller.}%

\topic{Did the Europeans view the Native Americans as a homogenized, whole tribe, or did they understand and differentiate between the different tribes?}{Although Ferdinand and Isabella yearned for gold, they released an order to treat the natives well, and not as slaves. Despite this, the Spanish conquerors ignored the command and continued to decimate the native population.}%

\topic{What did the Europeans think about this? What about the royalty?}{Unlike the English, the Spanish and French participated in intermarriage with the natives. In the long run, this would serve as a benefit for the Spanish and French, as it strengthened ties to the natives.}%

\topic{Did the Europeans realize it was their fault that a plague had been unleashed on the natives?}{Along with the influx of Europeans came their diseases. Although the Europeans had built up a tolerance for them over centuries, the tribes had not encountered them. As a result, this further decimated the dwindling numbers of the natives. Prior to the arrival of Europeans, the native population is estimated to have been around 75 million. Following the arrival of the Europeans, the native populations dropped to 4.5 million.}%

\topic{Was there trade between Asia and Europe before it existed?}{\underline{\textbf{Silk Road}} $-$ A path leading from China to Venice, discovered by Marco Polo between 1271 and 1295, and primarily used for trade.}%

\topic{Were the items that were being funneled into the Americas intended for trade with the natives, or to aid the explorers in their journey (or both)?}{\underline{\textbf{Columbian Exchange}} $-$ The cycle of trade, between the Old and New world, which began following 1492, and carried diseases, plants, animals, and cultures.}%

\topic{Was this method of trade considered more efficient?}{As a result of the increase in sailing, trade shifted more to waterways, rather than land routes. It had its benefits and shortcomings, but the most important result was the shift in European superpowers. Spain and Portugal replaced Genoa and Venice as the primary links to Asia. This meant that Spanish and Portuguese ships dominated the Indian Ocean.}%

\topic{Was creating a new, autonomous government intentional, or did Cort\'es decide to conquer the natives during the trip?}{\underline{\textbf{New Spain}} $-$ The title given by the Spanish Empire to their new colonies in the Americas.}%

\topic{What was the general sentiment of the people both, in the Americas and in Europe. Was it supportive of the conquest, unsupportive, or did most people not know of the way the natives were being treated?}{Bartolom\'e de Las Casas wrote to the Spanish crown of the poor treatment of the natives. This included forced religious conversions, beatings, working to death, etc. Before her death, Isabella ordered for the natives to be treated as free men; however, this did not help the work conditions for the natives. Most people ignored this order.}%

\topic{What needed to be done to be awarded one?}{\underline{\textbf{Encomienda}} $-$ A large ranch worked by Indian slaves.}%

\topic{Were all explorers conquistadores, or vice versa?}{\underline{\textbf{Conquistadores}} $-$ The name given to the Spanish conquerors of the Americas.}

\topic{Who had a greater benefit from imports, the New or Old world?}{\begin{tabular}{|c|c|} \hline America $\rightarrow$ Europe & Europe $\rightarrow$ America\\ \hline Corn & Wheat \\ Potatoes & Barley \\ Sweet Potatoes & Rye \\ Peanuts & Oats\\ Pumpkins & Apples \\ Pineapples & Peaches \\ Guava & Pears \\ Squash & Plums \\ Tomatoes & Apricots \\ Peppers & Cherries \\ Papayas & Bananas \\ Avocados & Coffee \\ Beans & Tea \\ Cassava & Sugar Cane \\ Blueberries & Melons \\ Tobacco & Lemons \\ Cocoa & Oranges \\ Vanilla & Cabbage \\ & Carrots \\ & Grapes \\ & Lettuce \\ & Onions \\ & Garlic \\ \hline Turkeys & Chickens \\ Llamas & Donkeys \\ & Cattle \\ & Goats \\ & Horses \\ & Pigs \\ & Sheep \\ & Cats \\ \hline \end{tabular}}%

\topic{Did France have any other motives to enter the age of exploration?}{King Francis I's main reason to enter the conquest of the Americas was that he simply did not want to leave it to Portugal and Spain. Francis commissioned the Italian, Giovanni da Verrazano, to explore the Americas.}%

\topic{Was Verrazano the first to explore all of the Atlantic coast?}{Giovanni da Verrazano, on his ship \textit{La Dauphine}, was hired to explore the Atlantic coast, from Florida to Newfoundland in 1524. Jacques Cartier followed in 1534. Both of them were contracted by the French throne.}%

\topic{Were there any explorers who went through this area before Cartier?}{On his second voyage, which lasted from 1535$-$1536, Cartier became among the first people to explore the area of modern-day Canada. He traveled far up the St. Lawrence river. He arrived near the modern areas of Quebec and Montreal. He set up a trade relationship with more than a thousand natives. This would be the basis for future territorial claims of France (Louisiana).}%

\topic{Are these acts a callback to the Act of Supremacy?}{King Henry VIII passed acts through parliament that declared his control of the church over the Pope's so that Henry could divorce Catherine. This formed the Puritan an Anglican churches.}%

\topic{Would this be the cause of a lot of the emigration into the United States?}{\underline{\textbf{Anglicans}} were those that followed Henry VIII's new church, the church of England. They agreed that the church should have been split with the pope, in order to preserve traditional values. \underline{\textbf{Puritans}} were religious extremists. They wanted to radically purify the English church, by refining the priesthood system, especially the higher-up bishops.}%

\topic{Did this event mark the rise of the English navy?}{In an attempt to overthrow the rule of Elizabeth I, Phillip II organized an armada to attack England. This attack failed greatly, mostly due to the storms the ships faced.}%

\newpage

\topic{Were all pirates aligned with certain countries, or did some work for themselves?}{Elizabeth I hired pirates in order to harass Spanish ships. In return, the pirates kept 80\% of their loot from the ships, as well as getting paid by England.}%

\topic{Is this the famous first colony that disappeared?}{Walter Raleigh created a colony near the Roanoke tribes, however, shortages, droughts, and an absence of resupply ships caused the colonists to disappear.}%

\summary{Although many expeditions left and explored much of the coastline of North and South America, these expeditions were failures, as very few settlements were built, aside from those that were built to mine for precious metals. The two most important reasons for exploration, gold and a route to China, had not been found, thus adding to the failures of these expeditions. There would be very few attempts at colonization, due to the low interest in the area, for more than a century following Columbus's voyage. For over a century, the only effect on the natives would be the sharp decline in population. Other than that, their lives remained relatively the same, other than the knowledge of other tribes on a different continent. The peak of the scramble for America would not begin until the \nth{17} century, when France would explore modern-day Canada and Central America, England would explore the East Coast (New England), and Spain would explore Mexico and the West Coast.}

%\topic{Here's another question to begin the new page.}{\lipsum[3]}%
%{\lipsum[4]}%
%{\lipsum[5]}%

%\summary{And another summary that will float to the bottom of the next page.}

\end{document}
