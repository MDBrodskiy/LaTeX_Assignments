\documentclass[a4paper]{article} 
\usepackage{tcolorbox}
\tcbuselibrary{skins}

\title{
\vspace{-3em}
\begin{tcolorbox}[colframe=white,opacityback=0]
\begin{tcolorbox}
\Huge\sffamily\centering AP US History Chapter 3 Notes
\end{tcolorbox}
\end{tcolorbox}
\vspace{-3em}
}

\date{}

\usepackage{background}
\SetBgScale{1}
\SetBgAngle{0}
\SetBgColor{grey}
\SetBgContents{\rule[0em]{4pt}{\textheight}}
\SetBgHshift{-2.3cm}
\SetBgVshift{0cm}

\usepackage{lipsum}% just to generate filler text for the example
\usepackage[margin=2cm]{geometry}

\usepackage{tikz}
\usepackage{tikzpagenodes}

\parindent=0pt

\usepackage{xparse}
\DeclareDocumentCommand\topic{ m m g g g g g}
{
\begin{tcolorbox}[sidebyside,sidebyside align=top,opacityframe=0,opacityback=0,opacitybacktitle=0, opacitytext=1,lefthand width=.3\textwidth]
\begin{tcolorbox}[colback=red!05,colframe=red!25,sidebyside align=top,width=\textwidth,before skip=0pt]
#1\end{tcolorbox}%
\tcblower
\begin{tcolorbox}[colback=blue!05,colframe=blue!10,width=\textwidth,before skip=0pt]
#2
\end{tcolorbox}
\IfNoValueF {#3}{
\begin{tcolorbox}[colback=blue!05,colframe=blue!10,width=\textwidth]
#3
\end{tcolorbox}
}
\IfNoValueF {#4}{
\begin{tcolorbox}[colback=blue!05,colframe=blue!10,width=\textwidth]
#4
\end{tcolorbox}
}
\IfNoValueF {#5}{
\begin{tcolorbox}[colback=blue!05,colframe=blue!10,width=\textwidth]
#5
\end{tcolorbox}
}
\IfNoValueF {#6}{
\begin{tcolorbox}[colback=blue!05,colframe=blue!10,width=\textwidth]
#6
\end{tcolorbox}
}
\IfNoValueF {#7}{
\begin{tcolorbox}[colback=blue!05,colframe=blue!10,width=\textwidth]
#7
\end{tcolorbox}
}
\end{tcolorbox}
}

\def\summary#1{
\begin{tikzpicture}[overlay,remember picture,inner sep=0pt, outer sep=0pt]
\node[anchor=south,yshift=-1ex] at (current page text area.south) {% 
\begin{minipage}{\textwidth}%%%%
\begin{tcolorbox}[colframe=white,opacityback=0]
\begin{tcolorbox}[enhanced,colframe=black,fonttitle=\large\bfseries\sffamily,sidebyside=true, nobeforeafter,before=\vfil,after=\vfil,colupper=black,sidebyside align=top, lefthand width=.95\textwidth,opacitybacktitle=1, opacitytext=1,
segmentation style={black!55,solid,opacity=0,line width=3pt},
title=Summary
]
#1
\end{tcolorbox}
\end{tcolorbox}
\end{minipage}
};
\end{tikzpicture}
}

\usepackage[super]{nth}
\usepackage{graphicx}
\usepackage{physics}
\usepackage{amsmath}
\usepackage{tikz}
\usepackage{mathdots}
\usepackage{yhmath}
\usepackage{cancel}
\usepackage{color}
\usepackage{siunitx}
\usepackage{array}
\usepackage{multirow}
\usepackage{amssymb}
\usepackage{gensymb}
\usepackage{tabularx}
\usepackage{booktabs}
\usetikzlibrary{fadings}
\usetikzlibrary{patterns}
\usetikzlibrary{shadows.blur}
\usetikzlibrary{shapes}

\begin{document} 
\maketitle

\topic{For what reason were the Americas avoided? Was a route to Asia considered more important?}{Following Columbus, Europeans spent more time trying to find a route to Asia past the Americas than exploring the Americas themselves.}%

\topic{Were the Spanish right about North America being barren and icy? Does this refer to Canada? What about Alaska?}{The Spanish found Mexico, Central and South America, and the Caribbean more rewarding than Northern America.}%

\topic{Why would someone travel for over 2 months just to fish for cod? Was codfish trading a lucrative occupation? }{Most of European contact with North America revolved around a codfish trade, which occurred during summers. This lasted for decades, as French, English, and Spanish fisherman spent their summers fishing off the coast of Canada and Maine. \begin{itemize} \item Note: No permanent settlements were established \end{itemize}}%

\topic{Was this a pivoting point where the English decided to force conversions? Was combating Spain their only motive?}{Richard Hakluyt wrote \textit{Pamphlet for the Virginia Enterprise}. In it, he attempted to convince his countrymen to establish permanent villages and plantations. Concerning the natives, Hakluyt considered it profitable to force conversions, as it meant riches and bases to combat Spain.}%

\topic{The book does not mention James I's motives for ending legalized piracy. Why is that?}{As king in 1603, James I outlawed legalized piracy (that is to say, it continued throughout the \nth{17} century, but not with the support of England). As a result of this, the English needed to establish permanent colonies to replace the wealth lost from outlawing piracy. This led to the formation of the Virginia Company, which told its settlers to search for minerals and find a route to China. Settlers were also told to build towns away from the coast, to prevent Spanish attack.}%

\topic{Why didn't the colonists just move to a different location?}{After the establishment of the Jamestown colony in 1607, the colonists dealt with drought and famine. This was because the region in which they settled had a volatile tide. At low tide, the water was filthy. At high tide, the water was very salty. This led to crop failures and poor trade.}%

\topic{If the ventures in the Americas turned out to be a failure and a waste of funds, why did the colonists remain there?}{The droughts and famines of the Jamestown colony in 1607$-$1608 were followed by more droughts and famines from 1609$-$1610. In addition to this, a war between the native tribes and the colonists broke out. This further thinned their nearly nonexistent supplies. Jamestown survived these violent years, but it remained in a dangerous position.}%

\topic{Was the monopoly created by King James withing England, or did it effect all of Europe?}{Although the colonies began as unprofitable, the growing of tobacco began to become a lucrative business. King James saw this as a possibility for great profits, and, as such, he created a royal monopoly.}%

\topic{Did the Separatists leave by their own will, or were they forced to leave?}{\underline{\bold{Pilgrims}} $-$ Separatists (a certain type of Puritans who wanted to separate from the English church completely), better known as Pilgrims, founded the second permanent English colony in North America, at a place known as Plymouth, Massachusetts, in 1620. This consisted of a small group of Separatists who migrated from England to Holland in 1607, and finally to the Plymouth colony.}%

\topic{Was the colony founded by the Pilgrims ruled by the English government, or were they granted self-rule?}{\underline{\bold{Mayflower Compact}} $-$ After approximately two months of sailing, the Pilgrims landed at Plymouth rock. They decided that it was necessary for them to ``combine ourselves together into a civil body politic, for our better ordering and preservation.'' These words signified that they would create a document of agreements, which would later become known as the Mayflower Compact. This served as the beginning of governing by consent of the governed.}%

\topic{What prompted the natives to be peaceful with the settlers of Plymouth, especially with one of their own being captured and taken to England?}{Samoset, a native from the area that is now Maine, strolled into Plymouth village and welcomed the settlers. Soon after, the tribe leader visited with a translator who had previously been captured and taken to England. The tribe showed the settlers how to do many things vital to their survival, such as farming, fishing, and hunting. As a result, the tribe's leader and the villagers agreed to peace, which lasted a total of 54 years. This peace led to the formation of what we now call Thanksgiving, where the villagers sat down with the tribe to a huge feast.}%

\topic{How long did it take this colony to develop into a successful village?}{The Massachusetts Bay Company, which was much like the Virginia Company, but led by Puritans, decided to move out of England, and into North America. In spring of 1630, the whole company moved to Salem, which was north of present-day Boston. By summer, over 1,000 people and 200 cattle had been brought to the settlement. Over the next decade, more than 20,000 would follow.}%

\topic{Did the people who drafted this know about the Mayflower Compact? If so, did they get inspiration from it?}{\underline{\bold{Charter of the Massachusetts Bay Company}} $-$ Much alike the Pilgrims in Plymouth, the other Puritans wanted to form a local government. As such, they drafted the Charter of the Massachusetts Bay Company. This document was much more detailed and thorough. This document established basis for their government, such as a general court for the people. This transformed what was intended to be a business into a government in Massachusetts.}%

\topic{What did the Hartford colonists do that was different from the Boston colonists? Why does the book not mention this?}{Some Puritans found the Boston government too restrictive. As such, they decided to form their own colony, in Hartford, Connecticut. At Hartford, the colonists established the \underline{\bold{Fundamental Orders of Connecticut}}. This document permitted more men to vote than did the Boston charter. There were, however, other Puritans, who believed that the Boston government was not strict enough. As a result, they made the government more theocratic in New Haven the following year.}%

\topic{Was this meant to form a sort of ``legacy'' system?}{In Massachusetts, the church members decided to pass the \underline{\bold{Halfway Covenant}}. This allowed for adults who had been baptized as kids, due to their parents being church members, to baptize their own children in the same congregation. This led to an increase in the people who were allowed to vote, as voting depended on church membership.}%

\topic{Does the capitol come from the name Lord Baltimore? Was King Charles I the first person to start a proprietary system?}{\underline{\bold{Proprietary Colony}} $-$ A colony which was owned by one person. This colony was regarded as private property, and, as such, could be passed down to heirs. The first instance of such a system was in Maryland, where King Charles I gifted the land to George Calvert, the first Lord Baltimore. Calvert was a catholic, and, therefore, he used the land to create a place for catholics to worship. Following the death of George Calvert, his son, Cecil, wanted more people to live in the colony. As such, he granted religious freedom to any christian, including protestants.}%

\topic{Was such a system used by any other colonies? If not, how would those colonies entice people to stay there?}{\underline{\bold{Headright System}} $-$ To entice more people to settle in Maryland, Cecil Calvert, the second Lord Baltimore set this system up. It gave any European settler in Maryland 100 acres of land, with an extra 100 for every adult family member, and 50 for every child.}%

\topic{Which colony was the most profitable?}{\begin{tabular}{p{.21\textwidth}|p{.1\textwidth}|p{.15\textwidth}|p{.15\textwidth}|p{.15\textwidth}} Colony & Founded & Religion & Crop & Government \\ \hline Virginia & 1607 & Anglican & Tobacco & Corporation, 1625 $\rightarrow$ Royal \\ \hline Bermuda & 1612 & Anglican & Mixed & Corporation \\ \hline  Plymouth & 1620 & Puritan & Farming & Corporation \\ \hline  St. Christopher & 1624 & Anglican & Royal \\ \hline  Barbados & 1627 & Anglican & Sugar & Royal \\ \hline  Nevis & 1628 & Anglican & Sugar & Royal \\ \hline  Mass. \& Maine & 1630 & Puritan & Fishing & Corporation \\ \hline  New Hampshire & 1630 & Puritan & Farming & Corporation, 1679 $\rightarrow$ Royal \\ \hline  Antigua & 1632 & Anglican & Sugar & Royal \\ \hline  Montserrat & 1632 & Anglican & Sugar & Royal \\ \hline  Maryland & 1634 &  Catholic (free) & Tobacco & Proprietary, 1690 $\rightarrow$ Royal, 1715 $\rightarrow$ Proprietary \\ \hline  Rhode Island & 1636 & None & Farming & Corporation \\ \hline  Connecticut & 1636 & Puritan & Farming & Corporation \\ \hline  New Haven & 1638 & Puritan & Farming & Corporation \\ \hline  Jamaica & 1655 & Anglican & Sugar & Royal \\ \hline  Carolina & 1663, Split 1729 & Anglican & Rice, Tobacco & Proprietary \\ \hline  New York & 1624 & None & Furs, Trade & Royal \\ \hline  New Jersey & 1664 & None & Farming & Proprietary \\ \hline  Pennsylvania & 1681 & None & Farming & Proprietary \\ \hline  Delaware & 1701 & None & Farming & Proprietary \\ \hline  Georgia & 1732 & None & Farming & Proprietary, 1751 $\rightarrow$ Royal \\ \end{tabular}}%

\topic{Where did this term come from?}{\underline{\bold{Patroons}} $-$ Wealthy Dutch citizens. They were offered large amounts of land along the Hudson river.}%

\topic{Who would these servants pay for their travel to America?}{\underline{\bold{Indentured Servants}} $-$ Colonists who worked in the colonies to pay off their travel to America. They were much like the slaves: they could be of any race, were often whipped, and did not have their own property. The only difference between slaves and indentured servants was that the indentured servants were set free after they paid off their debt.}%

\topic{How would people identify a slave?}{During the \nth{17} century, many slaves, regardless of race, rose to positions of landed gentry. One example is the Johnson family. They came to own over 900 acres of land, after being slaves. Near the Johnsons, lived roughly 300 blacks in 1680. Of the 300, 40 were free. About 30 percent of people of African descent were free in 1668. Although race was not insignificant, slaves of all races existed.}%

\topic{Were the Pequots actually responsible for killing the captain of the trading vessel? The book doesn't say\dots}{\underline{\bold{Pequot War}} $-$ Following the murder of a trading vessel captain, the Puritans aligned themselves with the Mohegans and Narrangansetts. Together, they attacked a Pequot fortress. Although it was a short war, over 400 Pequots were killed, and the village was destroyed. Seeing this bloodshed, the Mohegans saw an opportunity to expand their power. On the other hand, the Narrangansetts were horrified. They were used to skill-based, skirmish warfare, whereas the Europeans wanted absolute annihilation of their opponent.}%

\topic{How did the European leaders view this?}{\underline{\bold{Mestizo}} $-$ Many of the settlers in New Spain and New France were unmarried men. As a result, many of them intermarried with the community. This resulted in the births of children with mixed European and Native American blood, or mestizos. This was beneficial for the colonists because it reduced tensions with the natives.}%

\topic{How did Metacom translate to King Phillip?}{\underline{\bold{King Phillip's War}} $-$ Metacom, the leader of the Wampanoags, was named King Phillip in english. It was suspected that he was planning an attack on the Pilgrims. John Sassamon was a christian Indian. He told the Pilgrims that Metacom was planning an attack, but he was not believed. Later, he was found murdered. The Pilgrims, of course, pointed fingers at Metacom, and hanged three of his associates. This led to the beginning of King Phillip's War. The first skirmishes were fought out two weeks following the executions. Pilgrims moved to fortresses, while their abandoned homes were burned. Some time into the burning of houses, the English fired back. The natives were waiting for them to fire first, and fought ferociously thereafter. This turned into a large scale conflict when it began to effect all colonists and all tribes. The war ended with the death of King Phillip. He was killed and his head was displayed at Plymouth for over 20 years.}%

\topic{Were these conflicts mostly fueled by the instability of the colonies, or actual aggression?}{\underline{\bold{Bacon's Rebellion}} $-$ Following an Indian raid on Thomas Mathew's property (as he had not paid the natives for items), a rich landowner by the name of Nathan Bacon decided to retaliate. He called for an attack against the natives. In addition to this, he believed that they should be attacked indiscriminately, whether peaceful or not. This is because many of the colonists had heard about the war with the Wampanoags, and they did not want the tribes to form a coalition. Governor Berkeley, however, did not agree with Bacon. He believed that they should ally with the friendly natives to defeat the hostile ones. As a result, Bacon burned Jamestown in an act of rebellion. Berkeley requested help from the English crown, and the rebellion was promptly put down.}%

\topic{Would this new friendship with the Montagnais, Huron, and the Checagou down the Mississippi be an important factor in the French and Indian War?}{Champlain, a French explorer, founded Quebec in 1608. He knew that alliances were necessary for it to survive, and, thus, he allied with the Montagnais and Huron to defeat the Iroquois. Furthermore, two explorers, Joliet and Marquette, were the first Europeans to boat down the Mississippi river. On their way, they met the Illinois tribe, in their village named Checagou (Chicago).}%

\topic{The book says nearly \textit{all} Spanish were killed. How many Spanish lived in the area?}{\underline{\bold{Pueblo Revolt}} $-$ Led by Pop\'e, the Pueblo Indians of New Mexico revolted. They killed nearly all Spanish settlers who lived on isolated ranches and farms. The Spanish would regain New Mexico, but only 12 years later. They would later begin moving North, into California, while establishing the mission system.}%

\summary{Europeans were not very interested in the North American colonies in the begin of the \nth{16} century. For decades, the east coast of modern day America was used for only one item: cod. Following Richard Hakluyt's pamphlet, English investors began to form some colonies. Most of the first colonies would fail. The first semi-successful colony would be that of Jamestown, Virginia. Unlike Virginia, colonies like Massachusetts and Pennsylvania were formed for religious freedom. For half a century, life with the natives was quite peaceful; however, the English did not marry with the natives, unlike the Spanish and French. This lack of intermarriage and unity led to a rise in tensions, which eventually erupted as King Phillip's war. At the same time, dissidence took place in Jamestown, as Nathan Bacon rebelled and rioted. Following the initial colonies in New England and Virginia, the amount of colonies began to grow exponentially. The last of the thirteen colonies, Georgia, was established for reasons much different from that of the other colonies: it was a safe haven for debtors. Aside from everything else, one thing was permanent: the new colonies that were being formed would lead to permanent homes to generations of Europeans.}

%\topic{Here's another question to begin the new page.}{\lipsum[3]}%

%\summary{And another summary that will float to the bottom of the next page.}

\end{document}
