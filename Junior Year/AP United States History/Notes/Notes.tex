\documentclass[a4paper]{article} 
\usepackage{tcolorbox}
\tcbuselibrary{skins}

\title{
\vspace{-3em}
\begin{tcolorbox}[colframe=white,opacityback=0]
\begin{tcolorbox}
\Huge\sffamily AP US History Chapter 2 Notes
\end{tcolorbox}
\end{tcolorbox}
\vspace{-3em}
}

\date{}

\usepackage{background}
\SetBgScale{1}
\SetBgAngle{0}
\SetBgColor{grey}
\SetBgContents{\rule[0em]{4pt}{\textheight}}
\SetBgHshift{-2.3cm}
\SetBgVshift{0cm}

\usepackage{lipsum}% just to generate filler text for the example
\usepackage[margin=2cm]{geometry}

\usepackage{tikz}
\usepackage{tikzpagenodes}

\parindent=0pt

\usepackage{xparse}
\DeclareDocumentCommand\topic{ m m g g g g g}
{
\begin{tcolorbox}[sidebyside,sidebyside align=top,opacityframe=0,opacityback=0,opacitybacktitle=0, opacitytext=1,lefthand width=.3\textwidth]
\begin{tcolorbox}[colback=red!05,colframe=red!25,sidebyside align=top,width=\textwidth,before skip=0pt]
#1.\end{tcolorbox}%
\tcblower
\begin{tcolorbox}[colback=blue!05,colframe=blue!10,width=\textwidth,before skip=0pt]
#2
\end{tcolorbox}
\IfNoValueF {#3}{
\begin{tcolorbox}[colback=blue!05,colframe=blue!10,width=\textwidth]
#3
\end{tcolorbox}
}
\IfNoValueF {#4}{
\begin{tcolorbox}[colback=blue!05,colframe=blue!10,width=\textwidth]
#4
\end{tcolorbox}
}
\IfNoValueF {#5}{
\begin{tcolorbox}[colback=blue!05,colframe=blue!10,width=\textwidth]
#5
\end{tcolorbox}
}
\IfNoValueF {#6}{
\begin{tcolorbox}[colback=blue!05,colframe=blue!10,width=\textwidth]
#6
\end{tcolorbox}
}
\IfNoValueF {#7}{
\begin{tcolorbox}[colback=blue!05,colframe=blue!10,width=\textwidth]
#7
\end{tcolorbox}
}
\end{tcolorbox}
}

\def\summary#1{
\begin{tikzpicture}[overlay,remember picture,inner sep=0pt, outer sep=0pt]
\node[anchor=south,yshift=-1ex] at (current page text area.south) {% 
\begin{minipage}{\textwidth}%%%%
\begin{tcolorbox}[colframe=white,opacityback=0]
\begin{tcolorbox}[enhanced,colframe=black,fonttitle=\large\bfseries\sffamily,sidebyside=true, nobeforeafter,before=\vfil,after=\vfil,colupper=black,sidebyside align=top, lefthand width=.95\textwidth,opacitybacktitle=1, opacitytext=1,
segmentation style={black!55,solid,opacity=0,line width=3pt},
title=Summary
]
#1
\end{tcolorbox}
\end{tcolorbox}
\end{minipage}
};
\end{tikzpicture}
}

\usepackage{graphicx}
\usepackage{physics}
\usepackage{amsmath}
\usepackage{tikz}
\usepackage{mathdots}
\usepackage{yhmath}
\usepackage{cancel}
\usepackage{color}
\usepackage{siunitx}
\usepackage{array}
\usepackage{multirow}
\usepackage{amssymb}
\usepackage{gensymb}
\usepackage{tabularx}
\usepackage{booktabs}
\usetikzlibrary{fadings}
\usetikzlibrary{patterns}
\usetikzlibrary{shadows.blur}
\usetikzlibrary{shapes}

\begin{document} 
\maketitle

\topic{Was there trade between Asia and Europe before it existed?}{\underline{\textbf{Silk Road}} $-$ A path leading from China to Venice, discovered by Marco Polo between 1271 and 1295, and primarily used for trade.}%

\topic{Were the items that were being funneled into the Americas intended for trade with the natives, or to aid the explorers in their journey (or both)?}{\underline{\textbf{Columbian Exchange}} $-$ The cycle of trade, between the Old and New world, which began following 1492, and carried diseases, plants, animals, and cultures.}%

\topic{Was creating a new, autonomous government intentional, or did Cort\'es decide to conquer the natives during the trip?}{\underline{\textbf{New Spain}} $-$ The title given by the Spanish Empire to their new colonies in the Americas.}%

\topic{What needed to be done to be awarded one?}{\underline{\textbf{Encomienda}} $-$ A large ranch worked by Indian slaves.}%

\topic{Were all explorers conquistadores, or vice versa?}{\underline{\textbf{Conquistadores}} $-$ The name given to the Spanish conquerors of the Americas.}

\summary{2.1 $-$ Overall, the discovery of the New World would have a more profound effect on life in the Old World than the Old World had on the New World.}

\newpage

\topic{Was Verrazano the first to explore all of the Atlantic coast?}{Giovanni da Verrazano was hired to explore the Atlantic coast, from Florida to Newfoundland in 1524. Jacques Cartier followed in 1534.}%

\topic{Are these acts a callback to the Act of Supremacy?}{King Henry VIII passed acts through parliament that declared his control of the church over the Pope's so that Henry could divorce Catherine. This formed the Puritan an Anglican churches.}%

\topic{Did this event mark the rise of the English navy?}{In an attempt to overthrow the rule of Elizabeth I, Phillip II organized an armada to attack England. This attack failed greatly, mostly due to the storms the ships faced.}%

\topic{Were all pirates aligned with certain countries, or did some work for themselves?}{Elizabeth I hired pirates in order to harass Spanish ships. In return, the pirates kept 80\% of their loot from the ships, as well as getting paid by England.}%

\topic{Is this the famous first colony that disappeared?}{Walter Raleigh created a colony near the Roanoke tribes, however, shortages, droughts, and an absence of resupply ships caused the colonists to disappear.}%

\summary{2.4 \& 2.5 $-$ Although many expeditions left and explored much of the coastline of North and South America, these expeditions were failures, as very few settlements were built, aside from those that were built to mine for precious metals.}

%\topic{Here's another question to begin the new page.}{\lipsum[3]}%
%{\lipsum[4]}%
%{\lipsum[5]}%

%\summary{And another summary that will float to the bottom of the next page.}

\end{document}
