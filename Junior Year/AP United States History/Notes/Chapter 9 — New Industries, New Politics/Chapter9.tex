\documentclass[a4paper]{article} 
\usepackage{tcolorbox}
\tcbuselibrary{skins}

\title{
\vspace{-3em}
\begin{tcolorbox}[colframe=white,opacityback=0]
\begin{tcolorbox}
\Huge\sffamily\centering AP US History Chapter 9 Notes
\end{tcolorbox}
\end{tcolorbox}
\vspace{-3em}
}

\date{}

\usepackage{background}
\SetBgScale{1}
\SetBgAngle{0}
\SetBgColor{grey}
\SetBgContents{\rule[0em]{4pt}{\textheight}}
\SetBgHshift{-2.3cm}
\SetBgVshift{0cm}

\usepackage{lipsum}% just to generate filler text for the example
\usepackage[margin=2cm]{geometry}
\usepackage{hyperref}
\hypersetup{
colorlinks=true,
linkcolor=blue,
filecolor=magenta,      
urlcolor=blue,
citecolor=blue,
}
%\usepackage{manyfoot}
%\DeclareNewFootnote{A}[arabic]
\urlstyle{same}

\usepackage{tikz}
\usepackage{tikzpagenodes}

\parindent=0pt

\usepackage{xparse}
\DeclareDocumentCommand\topic{ m m g g g g g}
{
\begin{tcolorbox}[sidebyside,sidebyside align=top,opacityframe=0,opacityback=0,opacitybacktitle=0, opacitytext=1,lefthand width=.3\textwidth]
\begin{tcolorbox}[colback=red!05,colframe=red!25,sidebyside align=top,width=\textwidth,before skip=0pt]
#1\end{tcolorbox}%
\tcblower
\begin{tcolorbox}[colback=blue!05,colframe=blue!10,width=\textwidth,before skip=0pt]
#2
\end{tcolorbox}
\IfNoValueF {#3}{
\begin{tcolorbox}[colback=blue!05,colframe=blue!10,width=\textwidth]
#3
\end{tcolorbox}
}
\IfNoValueF {#4}{
\begin{tcolorbox}[colback=blue!05,colframe=blue!10,width=\textwidth]
#4
\end{tcolorbox}
}
\IfNoValueF {#5}{
\begin{tcolorbox}[colback=blue!05,colframe=blue!10,width=\textwidth]
#5
\end{tcolorbox}
}
\IfNoValueF {#6}{
\begin{tcolorbox}[colback=blue!05,colframe=blue!10,width=\textwidth]
#6
\end{tcolorbox}
}
\IfNoValueF {#7}{
\begin{tcolorbox}[colback=blue!05,colframe=blue!10,width=\textwidth]
#7
\end{tcolorbox}
}
\end{tcolorbox}
}

\def\summary#1{
\begin{tikzpicture}[overlay,remember picture,inner sep=0pt, outer sep=0pt]
\node[anchor=south,yshift=-1ex] at (current page text area.south) {% 
\begin{minipage}{\textwidth}%%%%
\begin{tcolorbox}[colframe=white,opacityback=0]
\begin{tcolorbox}[enhanced,colframe=black,fonttitle=\large\bfseries\sffamily,sidebyside=true, nobeforeafter,before=\vfil,after=\vfil,colupper=black,sidebyside align=top, lefthand width=.95\textwidth,opacitybacktitle=1, opacitytext=1,
segmentation style={black!55,solid,opacity=0,line width=3pt},
title=Summary
]
#1
\end{tcolorbox}
\end{tcolorbox}
\end{minipage}
};
\end{tikzpicture}
}
\usepackage{color, colortbl}
\definecolor{Gray}{gray}{.6}
\definecolor{BurntOrange}{rgb}{0.85, 0.6, 0.3}
\definecolor{White}{rgb}{1.0, 1.0, 1.0}
\usepackage[super]{nth}
\usepackage{graphicx}
\usepackage{physics}
\usepackage{amsmath}
\usepackage{tikz}
\usepackage{mathdots}
\usepackage{yhmath}
\usepackage{cancel}
\usepackage{color}
\usepackage{siunitx}
\usepackage{array}
\usepackage{multirow}
\usepackage{amssymb}
\usepackage{gensymb}
\usepackage{xcolor}
\usepackage{tabularx}
\usepackage{booktabs}
\usepackage[normalem]{ulem}
\usetikzlibrary{fadings}
\usetikzlibrary{patterns}
\usetikzlibrary{shadows.blur}
\usetikzlibrary{shapes}
\usepackage{fancyhdr}
\pagestyle{fancy}
\lfoot[\vspace{-15pt} \hline]{\vspace{-15pt} \hline}
\rfoot[\vspace{-15pt} \hline]{\vspace{-15pt} \hline}
\cfoot[\thepage]{\thepage}
\lhead[\copyright 2020 $-$ \textit{All Rights Reserved} ]{\copyright 2020 $-$ \textit{All Rights Reserved}}
\chead[AP United States History]{AP United States History}
\rhead[Michael Brodskiy]{Michael Brodskiy}

\begin{document} 
\maketitle

\topic{If other crops, such as rice, indigo, tobacco, and sugar were considered better cash crops than cotton, then what was the purpose of growing cotton? Wouldn't farm owners want to maximize profit, and, thus, sell more expensive crops?}{\begin{itemize} \item Although the cotton industry was already large and growing quickly, many factors would lead to quicker expansion. \item The first factor that contributed to accelerated growth of the industry was preference. Many people began to prefer cotton clothes, as they were warm, yet much more breathable as compared to wool or linen. In addition to this, the cotton crop itself was becoming more and more cheap. \item The second, and more important factor, was the age of industrialization. New inventions were coming out rapidly. One invention was the flying shuttle, which was able to weave cotton at a much faster rate, and required only one person to operate it. Next, the spinning jenny was a multi-spindled machine, which spun multiple strands of cotton at once. In addition to this, the water frame was patented. This machine was one of the earliest examples of mass production, as it used the power of moving water to spin cotton. In addition to this, the steam engine added much more power and automation to factories. \item Ultimately, there would be one most important invention. This invention would involve the first step to all of the cotton processing: actually picking cotton. Eli Whitney came along and made the cotton gin (short for engine). At first, it was made of wooden pins which were supposed to separate the seed from the cotton. The wood, however, was too weak and would break often. Next, Whitney tried wire from a chicken coup. This worked quite well, especially when combined with a brush that cleaned out the seeds that could get stuck. This new cotton gin would increase productivity more than 50 times what it had been. Although Whitney was famous for initially inventing the gin, this invention did not bring him much wealth, as people were creating better and more efficient versions of it daily.  \end{itemize}}%

\topic{Did people flood into Alabama because they had heard of the rich soil, or did they move because it was recently discovered? Overall, what was the motive for moving west?}{\begin{itemize} \item There were two main types of cotton being grown. First was long-stem, which was grown primarily on the Atlantic coast. Second was the green-seed, which would soon overtake the long-stem production. \item As the United States acquired more territory, the production of green-seed cotton rose. This was due to the \textbf{Black Belt}, a stretch of fertile, black soil, which stretched from Georgia to Louisiana. This soil was great for growing cotton. \item As more western lands opened up, more people began to move there. When Alabama was admitted as a state in 1819, it had five times the population as it had in 1810. This move, which followed the War of 1812, was known as Alabama Fever. In 1801, America produced about nine percent of the world's cotton, and India produced about sixty. By 1820, America overtook India, and by 1850, America grew more than two-thirds of the worlds cotton.  \end{itemize}}%

\topic{On what kind of farms were slaves treated the worst? Indigo, rice, sugar, or cotton? Also, in what state were they treated the worst?}{\begin{itemize} \item As plantations spread throughout the black belt, into the interior of the country, more and more slave labor was moved with them. With the move came more problems, such as worsening conditions for slaves. Slaves were now even more highly valued, which meant they were treated even worse. For example, many slaves near the coast, who had developed families over the years, would be divided from their family and sent to a different, more southwest state. \item Many slave owners would use this as a threat$-$they would tell their slaves to work harder, or risk being sent off to the expanding colonies. Slaves, especially those who had families, dreaded this possibility, and revolts came along with it. As treatment of the slaves worsened, revolts and runaways rose proportionally. Slaves were, however, given some limited rights by their owners. Some owners would permit slaves to keep their own small gardens in order to grow themselves food and fabric materials for clothes, and sell the surplus.  \end{itemize}}%

\topic{Was the South generally for or against the new industrial age?}{\begin{itemize} \item When coming back to America from Britain, Francis Cabot Lowell, alongside new technology and innovations, brought with him a new business model and system. Lowell formed one of the early multi-shareholder companies. \item Within his new factories, Lowell needed workers, and fast. He looked to single females for employment, as the migration of many men to the west left many women behind. Thus, the factories employed many women, who were nicknamed Lowell Girls. Many press releases were issued, showing employees working happily in the factories, and talking about the perfect conditions. Although the conditions here were significantly better than England, the still were not ideal. \item When one of the Lowell Mills released news to their employees that wages needed to be cut to, ``meet the unusual pressure of the times,'' some of the first strikes erupted. These strikes, although they would not complete much, would place a mark which showed that workers should have the right to protest.  \end{itemize}}%

\topic{What other coastal cities would become powerful commercial centers? The book only mentions New York.}{\begin{itemize} \item As commerce began to grow during the industrial age in America, New York began to grow as the biggest center of trade. New York led the United States for three reasons: first, ships could dock in one of the many docks available in New York. Second, New York was already powerful enough to sustain a high level of commerce. Third, New York already had many experience longshoremen who could unload the shipments with ease due to their experience. \item The whole industrial era, however, did not come without failures. In 1819, the price of cotton decreased drastically, to less than half the price. This resulted in one of the first depressions in American history. Many farms were foreclosed, and westward movement stopped. In the long run, the price of cotton would rebound.  \end{itemize}}%

\topic{Did the construction of the Erie Canal have any adverse effects? How much did it cost the state of New York?}{\begin{itemize} \item Along with the industrial advances in manufacturing came new feats of engineering. One such example is that of canals, the biggest one being the Erie Canal. The construction of this canal was extremely difficult, and had not been done before. Locks were required to account for changes in height of hundreds of feet. Once it was finished, the New York economy boomed. Shipping prices dropped to one-tenth of the former price. Many goods that were too heavy or perishable were now shipped farther into the United States. People enjoyed items previously considered luxurious, such as oysters. \item Along with canals came roads. The federal government began developing extensive areas of roads, which would be constantly expanded. Toll booths would be placed in various locations on roads in order to charge for using the road, therefore funding more projects. This new network of gravel roads was being developed quickly, and becoming more and more extensive. These roads would also be used by the up and coming federal post office system.  \item In addition to other advances, steam ships appeared around this time. These ships made coastal travel much more efficient than it had ever been. In 1817, a steam ship could travel from New Orleans to Kentucky in 25 days, and, in less than a decade, 8 days. Railroads would soon follow too, but that would not occur until late 1840s to 1850s.  \end{itemize}}%

\topic{The table shows the change in mean travel time between certain locations in different years}{\hspace{-5pt}\begin{tabular}{p{.25\textwidth}!{\color{Gray}\vrule}p{.2\textwidth}!{\color{Gray}\vrule}p{.2\textwidth}!{\color{Gray}\vrule}p{.2\textwidth}} \rowcolor{BurntOrange} \textcolor{White}{Route} & \textcolor{White}{1800} & \textcolor{White}{1830} & \textcolor{White}{1860} \\ NY$\to$Phil. & 2 Days & 1 Day & $<1$ Day\\ \rowcolor{red!15} NY$\to$SC & $>1$ Week & 5 Days & 2 Days\\ NY$\to$IL & 6 Weeks & 3 Weeks & 2 Days\\ \rowcolor{red!15} NY$\to$New Orleans & 4 Weeks & 2 Weeks & 6 Days\\ NY$\to$FL & 2/3 Weeks & 1/2 Weeks & 3 Days \\ \rowcolor{red!15} NY$\to$OH & 2/3 Weeks & 1 Week & 2 Days\\  \hline \end{tabular}}%

\topic{If people were quite distrustful of large corporations, what caused them to change their minds?}{\begin{itemize} \item As larger companies and business ventures began to form, businesses became larger and larger. Much of the public was quite distrustful of large businesses. This would change, however, when multi-shareholder companies formed. This way, a business could get multiple investors, therefore raising more money. \item Writers began to publish stories about the change in culture. For example, \textit{Rip Van Winkle} demonstrated the change in American attitudes from before the revolution, to industrial times. \item With the new, fast-paced business culture that formed in America came the abundance of new clocks. Older clocks only showed the passage of hours, whereas newer models had a minute hand too.  \end{itemize}}%

\topic{Why would justices almost always agree with Marshall, regardless of their prior views? Was he just that good at convincing people, or did the judges want to create a sense of solidarity?}{\begin{itemize} \item Under Marshall, the Supreme Court would expand its powers, which were quite limited under the Constitution. First of all, in \textit{Dartmouth v. Woodward} (1819), the Supreme Court would establish that, once a contract was agreed to, it could not be overturned by legislative action. \item Also, in \textit{McCulloch v. Maryland} (1819), the Supreme Court, still under Marshall, would decide that states were not allowed to interfere with the federal government. \item In addition to these, the Marshall Supreme Court, in \textit{Gibbons v. Ogden} (1824), would prohibit the state of New York from giving Ogden a monopoly on the ferry service. Finally, in \textit{Worcester v. Georgia} (1832), the Supreme Court forbid Georgia from regulating private dealings in Cherokee territory, as, by their law, this land was controlled by the federal government.  \end{itemize}}%

\topic{Couldn't the northerners argue that banning slavery could be good for the south, as setting the slaves free would cause about a two-fifths increase in the counted population of a state, meaning that state had more power.}{\begin{itemize} \item In 1819, when Missouri applied to be a state, James Tallmadge, Jr. of New York would attempt to pass an amendment which banned slavery in the new state. This would cause fiery debates between the north and the south, and would be one of the first times America had come close to civil war. Ultimately, Henry Clay (later known as the ``Great Compromiser'') came up with the \textbf{Missouri Compromise}. This stated that, if Missouri was to be accepted as a slave state, Maine would be accepted as a free state. Furthermore, no new slave states would be admitted above the southern border of Missouri.   \end{itemize}}%

\topic{What was the reaction of the public when, even though he got more electoral votes than anyone else, Jackson did not win the presidency?}{\begin{itemize} \item The election of 1824 was one of the most confusing and difficult elections of all time. Many candidates, such as Adams, Jackson, Clay, and Crawford would run against each other, although they were all from one political party. In the end, no one would get the majority of electoral votes, and, thus, the House had to decide who was to be president. There was one problem though$-$Clay was the speaker of the House. As a result, even though Jackson received the most electoral votes, he did not get the presidency. The house decided on Adams. \item\begin{tabular}{p{.4\textwidth}!{\color{Gray}\vrule}p{.4\textwidth}} \rowcolor{BurntOrange} \textcolor{White}{Candidate} & \textcolor{White}{Electoral Votes} \\ Jackson & 99 \\ \rowcolor{red!15} Adams & 84 \\ Crawford & 41 & \rowcolor{red!15} Clay & 37  \\ \hline \end{tabular} \item Adam's term was quite an important one. He wanted to work to improve the United States, in industry, infrastructure, and more. He would be the last president to oppose a multi-party system. Secretary of State Clay came up with the \textbf{American System}. This was a system in which tariffs were raised to protect American industry. This was beneficial to the commercial cities, however, this made the south spend more money on imports, which meant less net gain. This would cause a bitter divide between the politics of the north and the south.   \end{itemize}}%

\newpage

\topic{Was the bitterness and nastiness between the split political parties not as bad as, equally as bad, or worse than modern times?}{\begin{itemize} \item Following the midterm election in 1826, the Republican-Democrat Party would split. They would turn into the National Democrats (Whigs) who supported the government, and the Democrat Party, who were said to support the common man. In general, the slave-holding south sided with the Democrat party. On the other hand, the north supported the growing Whig party. \item Ultimately, Jackson would win the election in 1828, which would further the existing split between the north and south, and would help lead to the civil war. The slander and libel which occurred during the 1828 election would lead to the exhaustion of resources and energy on both sides.  \end{itemize}}%

\summary{The industrial era caused a huge boom in the American economy. The South, now expanding into southwest territories in the black belt, would create extensive slave plantations, which usually grew cotton, but occasionally had indigo, rice, and sugar. The increase in farms and output of crops caused a sharp rise in the value of slaves. Ultimately, this would cause the treatment of the slaves to worsen. In the North, factories and multi-shareholder companies would form, increasing the average size of a corporation. Inventions in transportation allowed for better flow of goods and better connection between different parts of America. Some of the most important inventions were: canals, trains, fabric-spinning factories, and the cotton gin. The interconnectedness of the individual states, along with the new, faster ships that traveled to Europe created the beginnings of a worldwide economy, which is still used today. The election of Adams, and, subsequently Jackson, would lead to a divide$-$one which was clearly existent at the formation of the country$-$ to come to light, and, ultimately, lead to the civil war.}

%\topic{Here's another question to begin the new page.}{\lipsum[3]}%

%\summary{And another summary that will float to the bottom of the next page.}

\end{document}
