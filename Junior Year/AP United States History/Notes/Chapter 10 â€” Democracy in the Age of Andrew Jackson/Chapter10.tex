\documentclass[a4paper]{article} 
\usepackage{tcolorbox}
\tcbuselibrary{skins}

\title{
\vspace{-3em}
\begin{tcolorbox}[colframe=white,opacityback=0]
\begin{tcolorbox}
\Huge\sffamily\centering AP US History Chapter 10 Notes
\end{tcolorbox}
\end{tcolorbox}
\vspace{-3em}
}

\date{}

\usepackage{background}
\SetBgScale{1}
\SetBgAngle{0}
\SetBgColor{grey}
\SetBgContents{\rule[0em]{4pt}{\textheight}}
\SetBgHshift{-2.3cm}
\SetBgVshift{0cm}

\usepackage{lipsum}% just to generate filler text for the example
\usepackage[margin=2cm]{geometry}
\usepackage{hyperref}
\hypersetup{
colorlinks=true,
linkcolor=blue,
filecolor=magenta,      
urlcolor=blue,
citecolor=blue,
}
%\usepackage{manyfoot}
%\DeclareNewFootnote{A}[arabic]
\urlstyle{same}

\usepackage{tikz}
\usepackage{tikzpagenodes}

\parindent=0pt

\usepackage{xparse}
\DeclareDocumentCommand\topic{ m m g g g g g}
{
\begin{tcolorbox}[sidebyside,sidebyside align=top,opacityframe=0,opacityback=0,opacitybacktitle=0, opacitytext=1,lefthand width=.3\textwidth]
\begin{tcolorbox}[colback=red!05,colframe=red!25,sidebyside align=top,width=\textwidth,before skip=0pt]
#1\end{tcolorbox}%
\tcblower
\begin{tcolorbox}[colback=blue!05,colframe=blue!10,width=\textwidth,before skip=0pt]
#2
\end{tcolorbox}
\IfNoValueF {#3}{
\begin{tcolorbox}[colback=blue!05,colframe=blue!10,width=\textwidth]
#3
\end{tcolorbox}
}
\IfNoValueF {#4}{
\begin{tcolorbox}[colback=blue!05,colframe=blue!10,width=\textwidth]
#4
\end{tcolorbox}
}
\IfNoValueF {#5}{
\begin{tcolorbox}[colback=blue!05,colframe=blue!10,width=\textwidth]
#5
\end{tcolorbox}
}
\IfNoValueF {#6}{
\begin{tcolorbox}[colback=blue!05,colframe=blue!10,width=\textwidth]
#6
\end{tcolorbox}
}
\IfNoValueF {#7}{
\begin{tcolorbox}[colback=blue!05,colframe=blue!10,width=\textwidth]
#7
\end{tcolorbox}
}
\end{tcolorbox}
}

\def\summary#1{
\begin{tikzpicture}[overlay,remember picture,inner sep=0pt, outer sep=0pt]
\node[anchor=south,yshift=-1ex] at (current page text area.south) {% 
\begin{minipage}{\textwidth}%%%%
\begin{tcolorbox}[colframe=white,opacityback=0]
\begin{tcolorbox}[enhanced,colframe=black,fonttitle=\large\bfseries\sffamily,sidebyside=true, nobeforeafter,before=\vfil,after=\vfil,colupper=black,sidebyside align=top, lefthand width=.95\textwidth,opacitybacktitle=1, opacitytext=1,
segmentation style={black!55,solid,opacity=0,line width=3pt},
title=Summary
]
#1
\end{tcolorbox}
\end{tcolorbox}
\end{minipage}
};
\end{tikzpicture}
}
\usepackage{color, colortbl}
\definecolor{Gray}{gray}{.6}
\definecolor{BurntOrange}{rgb}{0.85, 0.6, 0.3}
\definecolor{White}{rgb}{1.0, 1.0, 1.0}
\usepackage[super]{nth}
\usepackage{graphicx}
\usepackage{physics}
\usepackage{amsmath}
\usepackage{tikz}
\usepackage{mathdots}
\usepackage{yhmath}
\usepackage{cancel}
\usepackage{color}
\usepackage{siunitx}
\usepackage{array}
\usepackage{multirow}
\usepackage{amssymb}
\usepackage{gensymb}
\usepackage{xcolor}
\usepackage{tabularx}
\usepackage{booktabs}
\usepackage[normalem]{ulem}
\usetikzlibrary{fadings}
\usetikzlibrary{patterns}
\usetikzlibrary{shadows.blur}
\usetikzlibrary{shapes}
\usepackage{fancyhdr}
\pagestyle{fancy}
\lfoot[\vspace{-15pt} \hline]{\vspace{-15pt} \hline}
\rfoot[\vspace{-15pt} \hline]{\vspace{-15pt} \hline}
\cfoot[\thepage]{\thepage}
\lhead[\copyright 2020 $-$ \textit{All Rights Reserved} ]{\copyright 2020 $-$ \textit{All Rights Reserved}}
\chead[AP United States History]{AP United States History}
\rhead[Michael Brodskiy]{Michael Brodskiy}

\begin{document} 
\maketitle

\topic{If Jackson was such an enemy of the national bank, why did he end up as the president on the twenty dollar bill?}{\begin{itemize} \item Jackson came into office with a clear plan in mind. First of all, he wanted to change personnel in many government positions, unlike his predecessors had done. Jackson used a system which was nicknamed as the \textbf{Spoils System}, in which the victor won the spoils of war, or, in other words, the person who won the executive office got the ``spoils'' of choosing their staff. Among Jackson's staff was Martin Van Buren, the man who would aid in Jackson's 1828 campaign, became the Secretary of State. Jackson would place many informal friends as his advisors, in, what he called, the ``Kitchen Cabinet.'' Jackson's modifications to many services, including the postal service, would harm those services until around 1880, when civil service reform took place. \item Jackson had many goals, not all of them related. He had an extreme distrust of government. One of his most important campaign promises was to remove Indians from northern Georgia, Alabama, and Mississippi. Furthermore, he hated the Bank of the United States and wanted to pare it down, if not remove it completely. His presidency was during a period in which \textbf{Nullification}, or the right of a state to decide not to follow a federal law within its borders, was becoming popular. Jackson, although as a citizen he might have supported this, was against any dissidence from his rule as a president.   \end{itemize}}%

\topic{Were the tribes in communication with each other? If so, what tribes communicated with who? Did the tribes communicate their response plan to the Indian Removal Act?}{\begin{itemize} \item Among the biggest tribes that were displaced were the \textbf{Five Civilized Tribes}: the Cherokees, Choctaws, Chickasaws, Creeks, and Seminoles. The tribes had adopted and assimilated into the white culture. Sequoyah, a Cherokee warrior, was intrigued by written language, which would lead him to establish a written language for the Cherokees and Choctaws. Journals began to publish in English and Cherokee. Furthermore, many natives married with whites, further mixing the cultures. In addition to this, the tribes had developed crop growing techniques like the whites, and they grew cotton, as well as used slaves. \item Jackson passed the \textbf{Indian Removal Act}, claiming it would protect the Cherokees from Georgia laws. Jackson gave the natives a choice: move west, or submit to the laws of Georgia. This wasn't much of a choice, as Jackson himself said. The government treated the land, which they proclaimed ``Indian Territory,'' as though it were empty. This was untrue, as the land, which is modern-day Tennessee, was filled with numerous other tribes, with whom the new tribes would conflict. \item Contrary to popular belief, many whites did come to aid the natives. They claimed that these acts were brutal and unjust, though this would make little difference after the act was actually passed. The act would pass with a vote of 28:19 in the senate and 102:98 in the house. Attacks were made on the bill, such as calling it, ``Oppression with a vengeance.'' Not long after the passage of the act, many Indians came to court to fight it. The natives would win in cases such as \textit{Cherokee Nation v. Georgia} (1831) and \textit{Worcester v. Georgia} (1832). Chief Justice John Marshall ruled that the Cherokees were an independent nation, and, as such, could not be displaced. Jackson blatantly disregarded this, and continued to force migration of the natives. \item Of the five civilized tribes, the Seminoles were the only that did not migrate. Not all of the Creeks moved, though, and some remained to fight. The Second Creek War was put down easily, and lasted from 1836$-$1837, when the Indians were expelled by federal troops. Many of the remaining members of tribes were put in detention camps, where, after some time, 2,000 Cherokee men, women, and children were forced to march west. Around a third of the marching natives died in what is known as the \textbf{Trail of Tears}. During this period, around 2000 members of the Sac and Fox tribes moved east across the Mississippi to escape hostile Sioux tribes. This was seen as war by the Americans, who responded promptly and killed these natives. Overall, about 46,000 natives were removed during Jackson's administration. Today, around 300,000 Cherokee live in the United states, and they are the biggest federally recognized group.  \end{itemize}}%

\topic{Did Jackson realize it was his fault for the panic in 1837? If so, did he feel bad about it, or did he continue happily knowing he caused the destruction of the bank?}{\begin{itemize} \item Jackson absolutely despised the \textbf{Second Bank of the United States}. He made it his duty to dissolve this bank, although he had many hurdles to jump through. The First Bank of the United States was shut down by Jefferson in 1811. The War of 1812 meant a strong economy was essential for the United States, and, as such, the Second Bank of the United States was created. Some people who wanted less government involvement in economy, including Andrew Jackson, argued that the Panic of 1819 was caused by the bank. \item Nicholas Biddle, the president of the United States Bank, would battle with Jackson. Ultimately, Jackson would be the victor. Because Jackson was a populist, he could not directly remove the bank, as many people supported it. In the elction of 1832, a modern political split occured. The Democrat-Republicans split into the democrats and the whigs. The democrats supported Jackson, while the whigs supported Henry Clay. Clay would team up with Biddle to have a platform against Jackson, however Jackson still easily won the majority. Biddle was trying to make the national bank more appealing to gain popular support. One of his moves was to loosen restrictions on credit. After this Jackson maded his move. He devised 23 state banks, into which he wanted to deposit the federal funds. He asked the treasury secretary, McLane, to move the funds, but McLane refused because it was illegal unless the funds were unsafe. Jackson promoted him to Secretary of State, and appointed a new Secretary of Treasury, who also refused to move the funds, and would later be fired. Finally, Roger B. Taney, a third secretary of state came up with a comrpomise: the funds would remain, but new funds would be deposited in state banks. This way, the national bank would bleed out, as bills were paid from there, with no new funds added. Jackson had got the victory he wanted, although he caused another panic in 1837. A new central bank would not be established until 1913.  \end{itemize}}%

\topic{Prior to any personal or political hatred, what were relations between Calhoun and Jackson like?}{\begin{itemize} \item What began as a personal hatred would soon bloom into a debate on states' rights and, ultimately, the Civil War. John C. Calhoun, who had been elected vice president alongside Adams in 1824, was elected vice president again in 1828. As it can be predicted, relations between Jackson and Adam's former vice president were not very good. First of all, they had personal strife which occured as a result of feuding between their wives. This turned into a debate on tariffs, as Jackson, although he personally opposed them, supported them for the campaign to gain popular support. Calhoun, on the other hand, a South Carolinian, was strongly against tariffs, as it caused increase in prices for products in slave states, which did not have any industrial output. \item The tariff which would be passed in 1828 would become known as the \textbf{Tariff of Abominations}, as it was clearly in favor of the industrial north. This would be an early marker for the argument of states' rights, as South Carolina nullified the unfair tariff, for fear of banishment of slavery and federal power. When the federal government said it was to use force, South Carolina responded by threatening secession from the Union. South Carolina had threatened to secede nearly three decades before the Civil war. \item This tension between South Carolina and the government only worsened with Calhoun being the vice president, as Calhoun supported the Carolinians, while Jackson opposed them. Some modifications in the taxing rate for different products was made, however, Jackson refused to back down with the tariff. This stand off would finally end with South Carolina coming to terms, as a bill was passed in Congress which compromised and allowed Jackson to put down the rebellion. South Carolina, though, did not want to seem weak, and, as such, they nullfied this new government bill, which they called the ``Force Bill,'' although they did accept the terms of tariffs.   \end{itemize}}%

\topic{What was the most common religious denomination at this time?}{\begin{itemize} \item In July, 1824, Charles Grandison Finney, a former lawyer, was ordained a Presbyterian minister. He preached with appeals to logos, as a lawyer would. Furthermore, in Connecticut, Lyman Beecher embraced religion, and he helped form interdenominational congregations. In addition to this, Beecher established an education and religious system in Ohio. This marked the beginning of the Second Great Awakening. \item With this awakening came many new social reforms. Unlike before, the revived faith told women to preach to their local communities. Such a call to action would result in the early feminist movements of the 1840s. In addition to this, the reformed church preached for the abolition of slavery, as they saw it as infringement on human rights. More reforms followed with the new sweep of religion, which included prison reform. This made prisons places for reflection and learning, rather than punishment. Also, reforms for mental hospitals were made, and the first prohibition movements were formulated. These reforms in alcohol prompted many congressmen to swear an oath to not consume alcohol. On top of this, the army banned the usual alcohol ration.   \end{itemize}}%

\topic{Were the newfound communities somewhat like early cults, where they have a revered leader that people follow without question?}{\begin{itemize} \item With new freedoms came the exploration of new ideas. These ideas included, but were not limited to, new types of communities and religions. One such community was the Shakers, who were believers in Christ's second coming. In these communities, people practiced celibacy and were required to give their belongings to the community, as all the property was communal. The celibacy and communal property rules were requirements, and, as such, not many people were followers of the movement. Another community is that of Noyes in Oneida, New York. This community followed polygamous rules and sold silverware to stay alive. \item One important early experiment in a utopian system was that of Robert Owen. Owen created a community in New Harmony, Indiana, where he wanted to give universal welfare. The colony failed in less than a year and Owen squandered lots of money. Finally, likely the most important formation was that of the Church of Latter day Saints. Founded by Joseph Smith, these people were searching for religious truth. They established themselves as polygamists, and would later become known as Mormons. Adversity and mistreatment would lead to their migration to Salt Lake City.  \end{itemize}}%

\topic{Is there a modern-day equivalent to the Transcendentalist movement?}{\begin{itemize} \item Ralph Waldo Emerson gave a speech at Harvard Divinity School in Cambridge, Massachusetts, in which he preached, to many Unitarians (himself included) that often, when praying, the soul is not preached. He stated that people should preach ``throbs of desire and hope.'' He was never invited back to Harvard Divinity School. In 1841, George Ripley created a ``Transcendental Club,'' which sought to support residents through manual labor. The community fell apart fairly quickly.  \end{itemize}}%

\topic{Was most of the female population interested in getting a job, such as that of a teacher, or did they prefer to stay home?}{\begin{itemize} \item Catherine Beecher, daughter of Lyman Beecher, was a strong supporter of education, especially with females as teachers. At a time in which men were primarily the teachers, Beecher argued that women were even better than men, and that, with more schools, women could gain employment oppurtunities. \item Furthermore, the first public school systems began to be made. Horace Mann was the greatest supporter of this movement, as he wanted a state board of education, with standardized education in states, and preparation for teachers. This preparation would be state-funded for a year, which increased the amount of teachers available. Many Jacksonian-democrats, though, opposed this, as they wanted to remove the use of government funds, rather than increase it. Many people also rejected this new public school system, as it was anti-Catholic. \item With the push for standardized education came new standardized books. The first was called the \textit{McGuffey Reader}. This offered lessons in reading, writing, spelling, public speaking, social norms, and much more. Using these books, there was a new standard for the American education system, which did exactly what it was supposed to: educated children, while instilling ethical norms.  \end{itemize}}%

\summary{Overall, Andrew Jackson set radical precedents for his successors. He was the first president to appoint his own federal staff. Jackson was quite tenacious in his pursuits, especially that of destroying the national bank and removal of natives. First of all, Jackson passed the important Indian Removal Act, which caused a mass exodus and many, many deaths. This forceful migration, which resulted in the Trail of Tears effectively nullified half a century of treaties. In addition to this, Jackson targeted Nicholas Biddle, the head of the national bank. Ultimately, Jackson would win in his battle against the bank, by making the federal funds bleed out. In addition to this, the Second Great Awakening swept through the states, causing social and ethical reform, as well as abolitionist and feminist movements. Along with the new reforms came experimental communities, like Robert Owen's New Harmony Utopia, or the Mormon movement to Utah. Finally, one of the most important developments was that of the ``factory'' public education system. This system set state-wide standards for education, as well as funded preparaton and training for teachers.}

%\topic{Here's another question to begin the new page.}{\lipsum[3]}%

%\summary{And another summary that will float to the bottom of the next page.}

\end{document}
