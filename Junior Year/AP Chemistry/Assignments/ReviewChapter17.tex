%%%%%%%%%%%%%%%%%%%%%%%%%%%%%%%%%%%%%%%%%%%%%%%%%%%%%%%%%%%%%%%%%%%%%%%%%%%%%%%%%%%%%%%%%%%%%%%%%%%%%%%%%%%%%%%%%%%%%%%%%%%%%%%%%%%%%%%%%%%%%%%%%%%%%%%%%%%%%%%%%%%%%%%%%%%%%%%%%%%%%%%%%%%%
% Written By Michael Brodskiy
% Class: AP Chemistry
% Professor: J. Morgan
%%%%%%%%%%%%%%%%%%%%%%%%%%%%%%%%%%%%%%%%%%%%%%%%%%%%%%%%%%%%%%%%%%%%%%%%%%%%%%%%%%%%%%%%%%%%%%%%%%%%%%%%%%%%%%%%%%%%%%%%%%%%%%%%%%%%%%%%%%%%%%%%%%%%%%%%%%%%%%%%%%%%%%%%%%%%%%%%%%%%%%%%%%%%

\documentclass[12pt]{article} 
\usepackage{alphalph}
\usepackage[utf8]{inputenc}
\usepackage[russian,english]{babel}
\usepackage{titling}
\usepackage{amsmath}
\usepackage{graphicx}
\usepackage{enumitem}
\usepackage{amssymb}
\usepackage[super]{nth}
\usepackage{expl3}
\usepackage[version=4]{mhchem}
\usepackage{hpstatement}
\usepackage{rsphrase}
\usepackage{everysel}
\usepackage{ragged2e}
\usepackage{geometry}
\usepackage{fancyhdr}
\usepackage{cancel}
\usepackage{siunitx}
\usepackage{chemfig}
\usepackage{multicol}
\geometry{top=1.0in,bottom=1.0in,left=1.0in,right=1.0in}
\newcommand{\subtitle}[1]{%
  \posttitle{%
    \par\end{center}
    \begin{center}\large#1\end{center}
    \vskip0.5em}%

}
\newcommand{\orbital}[2]{{%
    \def\+{\big|\hspace{-2pt}\overline{\underline{\hspace{2pt}\upharpoonleft}}}%
    \def\-{\overline{\underline{\downharpoonright\hspace{2pt}}}\hspace{-2pt}\big|}%
    \def\0{\big|\hspace{-2pt}\overline{\underline{\phantom{\hspace{2pt}\downharpoonright}}}}%
    \def\1{\overline{\underline{\phantom{\downharpoonright\hspace{2pt}}}}\hspace{-2pt}\big|}%
  \setlength\tabcolsep{0pt}% remove extra horizontal space from tabular
  \begin{tabular}{c}$#2$\\[2pt]#1\end{tabular}%
}}
\DeclareSIUnit\Molar{\textsc{M}}
\DeclareSIUnit\Molal{\textsc{m}}
\DeclareSIUnit\atm{\textsc{atm}}
\DeclareSIUnit\torr{\textsc{torr}}
\DeclareSIUnit\psi{\textsc{psi}}
\DeclareSIUnit\bar{\textsc{bar}}
\DeclareSIUnit\Celsius{C}
\DeclareSIUnit\degree{$^{\circ}$}
\DeclareSIUnit\calorie{cal}
\usepackage{hyperref}
\hypersetup{
colorlinks=true,
linkcolor=blue,
filecolor=magenta,      
urlcolor=blue,
citecolor=blue,
}

\urlstyle{same}


\title{Review Chapter 17}
\date{May 21, 2020}
\author{Michael Brodskiy\\ \small Instructor: Mr. Morgan}

% Mathematical Operations:

% Sum: $$\sum_{n=a}^{b} f(x) $$
% Integral: $$\int_{lower}^{upper} f(x) dx$$
% Limit: $$\lim_{x\to\infty} f(x)$$

\begin{document}

\maketitle

\begin{enumerate}

  \item

    \begin{enumerate}

      \item \ce{Mg} is oxidized because the oxidation number goes from 0 to 2, and \ce{O} is reduced because the oxidation number goes from 0 to -2

      \item \ce{Al} is oxidized because the oxidation number goes from 0 to 3, and \ce{I} is reduced because the oxidation number goes from 0 to -1

    \end{enumerate}

  \item

    \begin{enumerate}

      \item \ce{Cu^2+ + 2e- -> Cu}

      \item \ce{Cl2 + 2e- -> 2Cl-}

      \item \ce{Zn -> Zn^2+ + 2e-}

      \item \ce{Br2 + 2e- -> 2Br-}

      \item \ce{Cd -> Cd^2+ + 2e-}

      \item \ce{I3^- + 2e- -> 3I-}

    \end{enumerate}

  \item 

    \begin{enumerate}

      \item 2(\ce{Fe^2+ -> Fe^3+ + e-}) $+$ \ce{Sn^4+ + 2e- -> Sn^2+} $=$\\ \ce{2Fe^2+ + Sn^4+ -> 2Fe^3+ + Sn^2+}

      \item \ce{Zn -> Zn^2+ + 2e-} $+$ \ce{Cu^2+ + 2e- -> Cu} $=$ \ce{Zn + Cu^2+ -> Zn^2+ + Cu}

    \end{enumerate}

  \item

    \begin{enumerate}

      \item $-1.077+1.33=.253[\si{\volt}]$

      \item $1.077-1.36=-.283[\si{\volt}]$

    \end{enumerate}

  \item

    \begin{enumerate}

      \item $E^o=.799+.127=.926[\si{\volt}]\Rightarrow -(2)(96485)(.926)=-178.7[\si{\kilo\joule}]$

      \item $E^o=-.614[\si{\volt}]\Rightarrow -(6)(96485)(-.614)=355.45[\si{\kilo\joule}]$

    \end{enumerate}

  \item $\ln(K)=\frac{n\mathcal{F}E^o}{RT}\Rightarrow E^o=1.229-1.36=-.131[\si{\volt}]\Rightarrow \frac{(2)(96485)(-.131)}{(8.314)(298)}=-10.2\Rightarrow K=3.71\cdot10^{-5}$

  \item $E=E^o-\frac{RT}{n\mathcal{F}}\ln(Q)\Rightarrow Q=\frac{1.33}{(.0015)^2}=591111\Rightarrow E^o=.409+.799=1.208[\si{\volt}]\Rightarrow 1.208-\frac{(8.314)(298)}{(2)(96485)}\ln(591111)=1.037[\si{\volt}]$

\end{enumerate}

\end{document}

