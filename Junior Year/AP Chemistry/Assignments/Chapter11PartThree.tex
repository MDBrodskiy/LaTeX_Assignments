%%%%%%%%%%%%%%%%%%%%%%%%%%%%%%%%%%%%%%%%%%%%%%%%%%%%%%%%%%%%%%%%%%%%%%%%%%%%%%%%%%%%%%%%%%%%%%%%%%%%%%%%%%%%%%%%%%%%%%%%%%%%%%%%%%%%%%%%%%%%%%%%%%%%%%%%%%%%%%%%%%%%%%%%%%%%%%%%%%%%%%%%%%%%
% Written By Michael Brodskiy
% Class: AP Chemistry
% Professor: J. Morgan
%%%%%%%%%%%%%%%%%%%%%%%%%%%%%%%%%%%%%%%%%%%%%%%%%%%%%%%%%%%%%%%%%%%%%%%%%%%%%%%%%%%%%%%%%%%%%%%%%%%%%%%%%%%%%%%%%%%%%%%%%%%%%%%%%%%%%%%%%%%%%%%%%%%%%%%%%%%%%%%%%%%%%%%%%%%%%%%%%%%%%%%%%%%%

\documentclass[12pt]{article} 
\usepackage{alphalph}
\usepackage[utf8]{inputenc}
\usepackage[russian,english]{babel}
\usepackage{titling}
\usepackage{amsmath}
\usepackage{graphicx}
\usepackage{enumitem}
\usepackage{amssymb}
\usepackage[super]{nth}
\usepackage{expl3}
\usepackage[version=4]{mhchem}
\usepackage{hpstatement}
\usepackage{rsphrase}
\usepackage{everysel}
\usepackage{ragged2e}
\usepackage{geometry}
\usepackage{fancyhdr}
\usepackage{cancel}
\usepackage{siunitx}
\usepackage{chemfig}
\usepackage{multicol}
\geometry{top=1.0in,bottom=1.0in,left=1.0in,right=1.0in}
\newcommand{\subtitle}[1]{%
  \posttitle{%
    \par\end{center}
    \begin{center}\large#1\end{center}
    \vskip0.5em}%

}
\newcommand{\orbital}[2]{{%
    \def\+{\big|\hspace{-2pt}\overline{\underline{\hspace{2pt}\upharpoonleft}}}%
    \def\-{\overline{\underline{\downharpoonright\hspace{2pt}}}\hspace{-2pt}\big|}%
    \def\0{\big|\hspace{-2pt}\overline{\underline{\phantom{\hspace{2pt}\downharpoonright}}}}%
    \def\1{\overline{\underline{\phantom{\downharpoonright\hspace{2pt}}}}\hspace{-2pt}\big|}%
  \setlength\tabcolsep{0pt}% remove extra horizontal space from tabular
  \begin{tabular}{c}$#2$\\[2pt]#1\end{tabular}%
}}
\DeclareSIUnit\Molar{\textsc{M}}
\DeclareSIUnit\Molal{\textsc{m}}
\DeclareSIUnit\atm{\textsc{atm}}
\DeclareSIUnit\torr{\textsc{torr}}
\DeclareSIUnit\psi{\textsc{psi}}
\DeclareSIUnit\bar{\textsc{bar}}
\DeclareSIUnit\Celsius{C}
\DeclareSIUnit\degree{$^{\circ}$}
\DeclareSIUnit\calorie{cal}
\usepackage{hyperref}
\hypersetup{
colorlinks=true,
linkcolor=blue,
filecolor=magenta,      
urlcolor=blue,
citecolor=blue,
}

\urlstyle{same}


\title{Chapter 11 $-$ Problems 78, 80}
\date{February 2, 2020}
\author{Michael Brodskiy\\ \small Instructor: Mr. Morgan}

% Mathematical Operations:

% Sum: $$\sum_{n=a}^{b} f(x) $$
% Integral: $$\int_{lower}^{upper} f(x) dx$$
% Limit: $$\lim_{x\to\infty} f(x)$$

\begin{document}

\maketitle

\begin{enumerate}

    \setcounter{enumi}{77}

  \item For the following reaction, the experimental rate expression is rate $=k[\ce{NO}]^2[\ce{H2}]$. The following mechanism is proposed. Is this mechanism consistent with the rate expression?

    \begin{center}
      \ce{2H2(g) + 2NO(g) &-> N2(g) + 2H2O(g)}\\
    \end{center}

    \begin{equation}
      \begin{array}{c r}
        \ce{2NO <=> N2O2} & \text{(fast)} \\
      \ce{N2O2 + H2 -> H2O + N2O} & \text{(slow)} \\
        \ce{N2O + H2 -> N2 + H2O} & \text{(fast)}
      \end{array}
      \label{1}
    \end{equation}

    \begin{center}
      Yes, it is consistent, because \ce{N2O2} may be substituted for \ce{2NO}, which, in turn, can be changed into $[\ce{NO}]^2$.
    \end{center}

    \setcounter{enumi}{79}

  \item Two mechanisms are proposed for the following reaction. Show that each of these mechanisms is consistent with the observed rate law: rate $=k[\ce{NO}]^2[\ce{O2}]$

    \begin{center}
      \ce{2NO(g) + O2(g) -> 2NO2(g)}
    \end{center}

    \begin{equation}
      \begin{array}{l c r}
        \text{Mechanism 1:} & \ce{NO + O2 <=> NO3} & \text{(fast)}\\
        & \ce{NO3 + NO -> 2NO2} & \text{(slow)}\\
        \text{Mechanism 2:} & \ce{NO + NO <=> N2O2} & \text{(fast)}\\
        & \ce{N2O2 + O2 -> 2NO2} & \text{(slow)}\\
      \end{array}
      \label{2}
    \end{equation}

    \begin{equation}
      \begin{split}
        \ce{NO3} = \ce{NO + O2}\\
        \ce{NO3 + NO} = \ce{2NO2}\\
        \ce{O2 + 2NO} = \ce{2NO2}\\
        \therefore \text{it is consistent}
      \end{split}
      \label{3}
    \end{equation}

    \begin{equation}
      \begin{split}
        \ce{N2O2} = \ce{2NO}\\
        \ce{N2O2 + O2} = \ce{2NO2}\\
        \ce{2NO + O2} = \ce{2NO2}\\
        \therefore \text{it is consistent}
      \end{split}
      \label{4}
    \end{equation}

\end{enumerate}

\end{document}

