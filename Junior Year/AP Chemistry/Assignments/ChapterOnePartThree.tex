%%%%%%%%%%%%%%%%%%%%%%%%%%%%%%%%%%%%%%%%%%%%%%%%%%%%%%%%%%%%%%%%%%%%%%%%%%%%%%%%%%%%%%%%%%%%%%%%%%%%%%%%%%%%%%%%%%%%%%%%%%%%%%%%%%%%%%%%%%%%%%%%%%%%%%%%%%%%%%%%%%%%%%%%%%%%%%%%%%%%%%%%%%%%
% Written By Michael Brodskiy
% Class: AP Chemistry
% Professor: J. Morgan
%%%%%%%%%%%%%%%%%%%%%%%%%%%%%%%%%%%%%%%%%%%%%%%%%%%%%%%%%%%%%%%%%%%%%%%%%%%%%%%%%%%%%%%%%%%%%%%%%%%%%%%%%%%%%%%%%%%%%%%%%%%%%%%%%%%%%%%%%%%%%%%%%%%%%%%%%%%%%%%%%%%%%%%%%%%%%%%%%%%%%%%%%%%%

\documentclass[12pt]{article} 
\usepackage{alphalph}
\usepackage[utf8]{inputenc}
\usepackage[russian,english]{babel}
\usepackage{titling}
\usepackage{amsmath}
\usepackage{graphicx}
\usepackage{enumitem}
\usepackage{amssymb}
\usepackage[super]{nth}
\usepackage{everysel}
\usepackage{ragged2e}
\usepackage{geometry}
\usepackage{fancyhdr}
\usepackage{cancel}
\usepackage{siunitx}
\geometry{top=1.0in,bottom=1.0in,left=1.0in,right=1.0in}
\newcommand{\subtitle}[1]{%
  \posttitle{%
    \par\end{center}
    \begin{center}\large#1\end{center}
    \vskip0.5em}%

}
\usepackage{hyperref}
\hypersetup{
colorlinks=true,
linkcolor=blue,
filecolor=magenta,      
urlcolor=blue,
citecolor=blue,
}

\urlstyle{same}


\title{Chapter One $-$ Problems: 58}
\date{August 24, 2020}
\author{Michael Brodskiy\\ \small Instructor: Mr. Morgan}

% Mathematical Operations:

% Sum: $$\sum_{n=a}^{b} f(x) $$
% Integral: $$\int_{lower}^{upper} f(x) dx$$
% Limit: $$\lim_{x\to\infty} f(x)$$

\begin{document}

\maketitle

\begin{enumerate}
    \setcounter{enumi}{58}

  \item Radiation exposure to human beings is usually given in rems (radiations equivalent for man). In SI units, the exposure is measured in sieverts ($\si{\sievert}$). One rem equals .0100[$\si{\sievert}$]. At one time, the exposure due to the nuclear reactors in Japan was measured to be 8217[$\si{\milli\sievert\per\hour}$]. How many rems would a person exposed to the radiation for 35[$\si{\minute}$] have absorbed? If one mammogram gives off .30[rem], how many mammograms would that exposure be equivalent to?

    $$\frac{1[rem]}{.01[\cancel{\si{\sievert}}]}\cdot\frac{8.217[\cancel{\si{\sievert}}]}{60[\cancel{\si{\minute}}]}\cdot35[\cancel{\si{\minute}}]=479[rem]$$
    \begin{center} The dose would be equal to 479[$rem$], or $479[\cancel{rem}]\cdot\frac{1[mam]}{.3[\cancel{rem}]}=1.6\cdot10^3$ mammograms \end{center}

\end{enumerate}

\end{document}

