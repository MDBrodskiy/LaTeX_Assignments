%%%%%%%%%%%%%%%%%%%%%%%%%%%%%%%%%%%%%%%%%%%%%%%%%%%%%%%%%%%%%%%%%%%%%%%%%%%%%%%%%%%%%%%%%%%%%%%%%%%%%%%%%%%%%%%%%%%%%%%%%%%%%%%%%%%%%%%%%%%%%%%%%%%%%%%%%%%%%%%%%%%%%%%%%%%%%%%%%%%%%%%%%%%%
% Written By Michael Brodskiy
% Class: AP Chemistry
% Professor: J. Morgan
%%%%%%%%%%%%%%%%%%%%%%%%%%%%%%%%%%%%%%%%%%%%%%%%%%%%%%%%%%%%%%%%%%%%%%%%%%%%%%%%%%%%%%%%%%%%%%%%%%%%%%%%%%%%%%%%%%%%%%%%%%%%%%%%%%%%%%%%%%%%%%%%%%%%%%%%%%%%%%%%%%%%%%%%%%%%%%%%%%%%%%%%%%%%

\documentclass[12pt]{article} 
\usepackage{alphalph}
\usepackage[utf8]{inputenc}
\usepackage[russian,english]{babel}
\usepackage{titling}
\usepackage{amsmath}
\usepackage{graphicx}
\usepackage{enumitem}
\usepackage{amssymb}
\usepackage[super]{nth}
\usepackage{expl3}
\usepackage[version=4]{mhchem}
\usepackage{hpstatement}
\usepackage{rsphrase}
\usepackage{everysel}
\usepackage{ragged2e}
\usepackage{geometry}
\usepackage{fancyhdr}
\usepackage{cancel}
\usepackage{siunitx}
\usepackage{chemfig}
\usepackage{multicol}
\usepackage{xcolor}
\usepackage{array}
\usepackage{color, colortbl}
\definecolor{cadetgrey}{rgb}{0.57, 0.64, 0.69}
\geometry{top=1.0in,bottom=1.0in,left=1.0in,right=1.0in}
\newcommand{\subtitle}[1]{%
  \posttitle{%
    \par\end{center}
    \begin{center}\large#1\end{center}
    \vskip0.5em}%

}
\newcommand{\orbital}[2]{{%
    \def\+{\big|\hspace{-2pt}\overline{\underline{\hspace{2pt}\upharpoonleft}}}%
    \def\-{\overline{\underline{\downharpoonright\hspace{2pt}}}\hspace{-2pt}\big|}%
    \def\0{\big|\hspace{-2pt}\overline{\underline{\phantom{\hspace{2pt}\downharpoonright}}}}%
    \def\1{\overline{\underline{\phantom{\downharpoonright\hspace{2pt}}}}\hspace{-2pt}\big|}%
  \setlength\tabcolsep{0pt}% remove extra horizontal space from tabular
  \begin{tabular}{c}$#2$\\[2pt]#1\end{tabular}%
}}
\DeclareSIUnit\Molar{\textsc{M}}
\DeclareSIUnit\Molal{\textsc{m}}
\DeclareSIUnit\atm{\textsc{atm}}
\DeclareSIUnit\torr{\textsc{torr}}
\DeclareSIUnit\psi{\textsc{psi}}
\DeclareSIUnit\bar{\textsc{bar}}
\DeclareSIUnit\Celsius{C}
\DeclareSIUnit\degree{$^{\circ}$}
\DeclareSIUnit\calorie{cal}
\usepackage{hyperref}
\hypersetup{
colorlinks=true,
linkcolor=blue,
filecolor=magenta,      
urlcolor=blue,
citecolor=blue,
}

\urlstyle{same}


\title{Chapter 14 $-$ Practice FRQ 2}
\date{March 12, 2020}
\author{Michael Brodskiy\\ \small Instructor: Mr. Morgan}

% Mathematical Operations:

% Sum: $$\sum_{n=a}^{b} f(x) $$
% Integral: $$\int_{lower}^{upper} f(x) dx$$
% Limit: $$\lim_{x\to\infty} f(x)$$

\begin{document}

\maketitle

\begin{enumerate}

  \item 

    \begin{enumerate}

        \setcounter{enumii}{3}

      \item 

        \begin{enumerate}

          \item \textbf{ }\\

            \begin{center}
            \begin{figure}[h]
              \centering
              \tikzset{every picture/.style={line width=0.75pt}} %set default line width to 0.75pt        

\begin{tikzpicture}[x=0.75pt,y=0.75pt,yscale=-1,xscale=1]
%uncomment if require: \path (0,300); %set diagram left start at 0, and has height of 300

%Shape: Rectangle [id:dp6473998621396935] 
\draw   (6,5) -- (257,5) -- (257,113) -- (6,113) -- cycle ;
%Shape: Ellipse [id:dp6232586844956738] 
\draw  [fill={rgb, 255:red, 0; green, 0; blue, 0 }  ,fill opacity=1 ] (71.07,66.61) .. controls (71.07,64.97) and (72.46,63.64) .. (74.17,63.64) .. controls (75.88,63.64) and (77.27,64.97) .. (77.27,66.61) .. controls (77.27,68.25) and (75.88,69.58) .. (74.17,69.58) .. controls (72.46,69.58) and (71.07,68.25) .. (71.07,66.61) -- cycle ;
%Shape: Ellipse [id:dp007831122826669423] 
\draw  [fill={rgb, 255:red, 0; green, 0; blue, 0 }  ,fill opacity=1 ] (71.07,50.65) .. controls (71.07,49.01) and (72.46,47.68) .. (74.17,47.68) .. controls (75.88,47.68) and (77.27,49.01) .. (77.27,50.65) .. controls (77.27,52.29) and (75.88,53.62) .. (74.17,53.62) .. controls (72.46,53.62) and (71.07,52.29) .. (71.07,50.65) -- cycle ;
%Shape: Ellipse [id:dp7277705854571324] 
\draw  [fill={rgb, 255:red, 0; green, 0; blue, 0 }  ,fill opacity=1 ] (125.3,66.61) .. controls (125.3,64.97) and (126.69,63.64) .. (128.4,63.64) .. controls (130.11,63.64) and (131.5,64.97) .. (131.5,66.61) .. controls (131.5,68.25) and (130.11,69.58) .. (128.4,69.58) .. controls (126.69,69.58) and (125.3,68.25) .. (125.3,66.61) -- cycle ;
%Shape: Ellipse [id:dp8720379357467392] 
\draw  [fill={rgb, 255:red, 0; green, 0; blue, 0 }  ,fill opacity=1 ] (125.3,50.65) .. controls (125.3,49.01) and (126.69,47.68) .. (128.4,47.68) .. controls (130.11,47.68) and (131.5,49.01) .. (131.5,50.65) .. controls (131.5,52.29) and (130.11,53.62) .. (128.4,53.62) .. controls (126.69,53.62) and (125.3,52.29) .. (125.3,50.65) -- cycle ;
%Shape: Ellipse [id:dp3666926886315627] 
\draw  [fill={rgb, 255:red, 0; green, 0; blue, 0 }  ,fill opacity=1 ] (172.56,66.61) .. controls (172.56,64.97) and (173.95,63.64) .. (175.66,63.64) .. controls (177.37,63.64) and (178.76,64.97) .. (178.76,66.61) .. controls (178.76,68.25) and (177.37,69.58) .. (175.66,69.58) .. controls (173.95,69.58) and (172.56,68.25) .. (172.56,66.61) -- cycle ;
%Shape: Ellipse [id:dp18933030338258106] 
\draw  [fill={rgb, 255:red, 0; green, 0; blue, 0 }  ,fill opacity=1 ] (172.56,50.65) .. controls (172.56,49.01) and (173.95,47.68) .. (175.66,47.68) .. controls (177.37,47.68) and (178.76,49.01) .. (178.76,50.65) .. controls (178.76,52.29) and (177.37,53.62) .. (175.66,53.62) .. controls (173.95,53.62) and (172.56,52.29) .. (172.56,50.65) -- cycle ;
%Shape: Ellipse [id:dp32825413049455476] 
\draw  [fill={rgb, 255:red, 0; green, 0; blue, 0 }  ,fill opacity=1 ] (185.73,66.61) .. controls (185.73,64.97) and (187.12,63.64) .. (188.83,63.64) .. controls (190.54,63.64) and (191.93,64.97) .. (191.93,66.61) .. controls (191.93,68.25) and (190.54,69.58) .. (188.83,69.58) .. controls (187.12,69.58) and (185.73,68.25) .. (185.73,66.61) -- cycle ;
%Shape: Ellipse [id:dp5415816227225667] 
\draw  [fill={rgb, 255:red, 0; green, 0; blue, 0 }  ,fill opacity=1 ] (185.73,50.65) .. controls (185.73,49.01) and (187.12,47.68) .. (188.83,47.68) .. controls (190.54,47.68) and (191.93,49.01) .. (191.93,50.65) .. controls (191.93,52.29) and (190.54,53.62) .. (188.83,53.62) .. controls (187.12,53.62) and (185.73,52.29) .. (185.73,50.65) -- cycle ;
%Shape: Ellipse [id:dp22849439309280983] 
\draw  [fill={rgb, 255:red, 0; green, 0; blue, 0 }  ,fill opacity=1 ] (109.91,85.57) .. controls (108.2,85.52) and (106.86,84.16) .. (106.9,82.52) .. controls (106.95,80.88) and (108.37,79.59) .. (110.08,79.63) .. controls (111.79,79.68) and (113.14,81.04) .. (113.1,82.68) .. controls (113.05,84.32) and (111.63,85.61) .. (109.91,85.57) -- cycle ;
%Shape: Ellipse [id:dp13356299089169932] 
\draw  [fill={rgb, 255:red, 0; green, 0; blue, 0 }  ,fill opacity=1 ] (93.27,85.13) .. controls (91.55,85.09) and (90.21,83.72) .. (90.25,82.08) .. controls (90.3,80.44) and (91.72,79.15) .. (93.43,79.2) .. controls (95.14,79.24) and (96.49,80.61) .. (96.45,82.24) .. controls (96.4,83.88) and (94.98,85.18) .. (93.27,85.13) -- cycle ;
%Shape: Ellipse [id:dp8094827940039651] 
\draw  [fill={rgb, 255:red, 0; green, 0; blue, 0 }  ,fill opacity=1 ] (108.75,36.21) .. controls (107.04,36.16) and (105.69,34.8) .. (105.74,33.16) .. controls (105.79,31.52) and (107.21,30.23) .. (108.92,30.27) .. controls (110.63,30.31) and (111.98,31.68) .. (111.93,33.32) .. controls (111.89,34.96) and (110.46,36.25) .. (108.75,36.21) -- cycle ;
%Shape: Ellipse [id:dp47398550266375894] 
\draw  [fill={rgb, 255:red, 0; green, 0; blue, 0 }  ,fill opacity=1 ] (92.1,35.77) .. controls (90.39,35.73) and (89.04,34.36) .. (89.09,32.72) .. controls (89.14,31.08) and (90.56,29.79) .. (92.27,29.84) .. controls (93.98,29.88) and (95.33,31.24) .. (95.29,32.88) .. controls (95.24,34.52) and (93.81,35.82) .. (92.1,35.77) -- cycle ;
%Shape: Ellipse [id:dp37206565277793735] 
\draw  [fill={rgb, 255:red, 0; green, 0; blue, 0 }  ,fill opacity=1 ] (162.59,36.58) .. controls (160.88,36.53) and (159.53,35.17) .. (159.58,33.53) .. controls (159.63,31.89) and (161.05,30.6) .. (162.76,30.64) .. controls (164.47,30.69) and (165.82,32.05) .. (165.78,33.69) .. controls (165.73,35.33) and (164.3,36.62) .. (162.59,36.58) -- cycle ;
%Shape: Ellipse [id:dp3236473150067467] 
\draw  [fill={rgb, 255:red, 0; green, 0; blue, 0 }  ,fill opacity=1 ] (145.94,36.14) .. controls (144.23,36.1) and (142.88,34.73) .. (142.93,33.09) .. controls (142.98,31.45) and (144.4,30.16) .. (146.11,30.21) .. controls (147.82,30.25) and (149.17,31.62) .. (149.13,33.26) .. controls (149.08,34.89) and (147.65,36.19) .. (145.94,36.14) -- cycle ;
%Shape: Ellipse [id:dp2135368861770648] 
\draw  [fill={rgb, 255:red, 0; green, 0; blue, 0 }  ,fill opacity=1 ] (217.21,36.58) .. controls (215.5,36.53) and (214.15,35.17) .. (214.2,33.53) .. controls (214.24,31.89) and (215.67,30.6) .. (217.38,30.64) .. controls (219.09,30.69) and (220.44,32.05) .. (220.39,33.69) .. controls (220.34,35.33) and (218.92,36.62) .. (217.21,36.58) -- cycle ;
%Shape: Ellipse [id:dp6232008915302696] 
\draw  [fill={rgb, 255:red, 0; green, 0; blue, 0 }  ,fill opacity=1 ] (200.56,36.14) .. controls (198.85,36.1) and (197.5,34.73) .. (197.55,33.09) .. controls (197.59,31.45) and (199.02,30.16) .. (200.73,30.21) .. controls (202.44,30.25) and (203.79,31.62) .. (203.74,33.26) .. controls (203.7,34.89) and (202.27,36.19) .. (200.56,36.14) -- cycle ;
%Shape: Ellipse [id:dp9020235910049739] 
\draw  [fill={rgb, 255:red, 0; green, 0; blue, 0 }  ,fill opacity=1 ] (227.56,66.61) .. controls (227.56,64.97) and (228.95,63.64) .. (230.66,63.64) .. controls (232.37,63.64) and (233.76,64.97) .. (233.76,66.61) .. controls (233.76,68.25) and (232.37,69.58) .. (230.66,69.58) .. controls (228.95,69.58) and (227.56,68.25) .. (227.56,66.61) -- cycle ;
%Shape: Ellipse [id:dp5761848589790346] 
\draw  [fill={rgb, 255:red, 0; green, 0; blue, 0 }  ,fill opacity=1 ] (227.56,50.65) .. controls (227.56,49.01) and (228.95,47.68) .. (230.66,47.68) .. controls (232.37,47.68) and (233.76,49.01) .. (233.76,50.65) .. controls (233.76,52.29) and (232.37,53.62) .. (230.66,53.62) .. controls (228.95,53.62) and (227.56,52.29) .. (227.56,50.65) -- cycle ;

% Text Node
\draw (45.88,72.63) node [anchor=south] [inner sep=0.75pt]   [align=left] {{\Huge H}};
% Text Node
\draw (100.96,58.63) node   [align=left] {{\Huge O}};
% Text Node
\draw (154.13,72.26) node [anchor=south] [inner sep=0.75pt]   [align=left] {{\Huge N}};
% Text Node
\draw (209.4,71.63) node [anchor=south] [inner sep=0.75pt]   [align=left] {{\Huge O}};


\end{tikzpicture}

              \caption{\ce{HNO2} Lewis Dot Diagram}
              \label{fig:1}
            \end{figure}
          \end{center}

        \item Because there are four electron pairs, it must share \ce{sp^3} hybridization

        \end{enumerate}

      \item

        \begin{enumerate}

          \item 

            \begin{equation}
              \begin{split}
                .02\cdot.1=.002[\si{\mole}]\\
                \frac{.002}{.1}=.02[\si{\Molar}_{\ce{HNO2}}]
              \end{split}
              \label{1}
            \end{equation}

          \item

            \begin{equation}
              \begin{split}
                pK_a\approx p\ce{H}\text{ at }10[\si{\milli\liter}]\\
                pK_a\approx 3.2
              \end{split}
              \label{2}
            \end{equation}

        \end{enumerate}

      \item There is a higher concentration of \ce{NO2-} at this point. This is because, past $10[\si{\milli\liter}]$ of \ce{KOH} added, the concentration of \ce{NO2-} is greater.

    \end{enumerate}

\end{enumerate}

\end{document}

