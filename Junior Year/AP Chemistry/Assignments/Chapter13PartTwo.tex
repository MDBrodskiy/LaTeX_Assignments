%%%%%%%%%%%%%%%%%%%%%%%%%%%%%%%%%%%%%%%%%%%%%%%%%%%%%%%%%%%%%%%%%%%%%%%%%%%%%%%%%%%%%%%%%%%%%%%%%%%%%%%%%%%%%%%%%%%%%%%%%%%%%%%%%%%%%%%%%%%%%%%%%%%%%%%%%%%%%%%%%%%%%%%%%%%%%%%%%%%%%%%%%%%%
% Written By Michael Brodskiy
% Class: AP Chemistry
% Professor: J. Morgan
%%%%%%%%%%%%%%%%%%%%%%%%%%%%%%%%%%%%%%%%%%%%%%%%%%%%%%%%%%%%%%%%%%%%%%%%%%%%%%%%%%%%%%%%%%%%%%%%%%%%%%%%%%%%%%%%%%%%%%%%%%%%%%%%%%%%%%%%%%%%%%%%%%%%%%%%%%%%%%%%%%%%%%%%%%%%%%%%%%%%%%%%%%%%

\documentclass[12pt]{article} 
\usepackage{alphalph}
\usepackage[utf8]{inputenc}
\usepackage[russian,english]{babel}
\usepackage{titling}
\usepackage{amsmath}
\usepackage{graphicx}
\usepackage{enumitem}
\usepackage{amssymb}
\usepackage[super]{nth}
\usepackage{expl3}
\usepackage[version=4]{mhchem}
\usepackage{hpstatement}
\usepackage{rsphrase}
\usepackage{everysel}
\usepackage{ragged2e}
\usepackage{geometry}
\usepackage{fancyhdr}
\usepackage{cancel}
\usepackage{siunitx}
\usepackage{chemfig}
\usepackage{multicol}
\usepackage{xcolor}
\usepackage{array}
\usepackage{color, colortbl}
\definecolor{cadetgrey}{rgb}{0.57, 0.64, 0.69}
\geometry{top=1.0in,bottom=1.0in,left=1.0in,right=1.0in}
\newcommand{\subtitle}[1]{%
  \posttitle{%
    \par\end{center}
    \begin{center}\large#1\end{center}
    \vskip0.5em}%

}
\newcommand{\orbital}[2]{{%
    \def\+{\big|\hspace{-2pt}\overline{\underline{\hspace{2pt}\upharpoonleft}}}%
    \def\-{\overline{\underline{\downharpoonright\hspace{2pt}}}\hspace{-2pt}\big|}%
    \def\0{\big|\hspace{-2pt}\overline{\underline{\phantom{\hspace{2pt}\downharpoonright}}}}%
    \def\1{\overline{\underline{\phantom{\downharpoonright\hspace{2pt}}}}\hspace{-2pt}\big|}%
  \setlength\tabcolsep{0pt}% remove extra horizontal space from tabular
  \begin{tabular}{c}$#2$\\[2pt]#1\end{tabular}%
}}
\DeclareSIUnit\Molar{\textsc{M}}
\DeclareSIUnit\Molal{\textsc{m}}
\DeclareSIUnit\atm{\textsc{atm}}
\DeclareSIUnit\torr{\textsc{torr}}
\DeclareSIUnit\psi{\textsc{psi}}
\DeclareSIUnit\bar{\textsc{bar}}
\DeclareSIUnit\Celsius{C}
\DeclareSIUnit\degree{$^{\circ}$}
\DeclareSIUnit\calorie{cal}
\usepackage{hyperref}
\hypersetup{
colorlinks=true,
linkcolor=blue,
filecolor=magenta,      
urlcolor=blue,
citecolor=blue,
}

\urlstyle{same}


\title{Chapter 13 $-$ Problems 34, 48, 54, 72}
\date{February 22, 2020}
\author{Michael Brodskiy\\ \small Instructor: Mr. Morgan}

% Mathematical Operations:

% Sum: $$\sum_{n=a}^{b} f(x) $$
% Integral: $$\int_{lower}^{upper} f(x) dx$$
% Limit: $$\lim_{x\to\infty} f(x)$$

\begin{document}

\maketitle

\begin{enumerate}

    \setcounter{enumi}{33}

  \item

    \begin{enumerate}

      \item 
        
        \begin{equation}
          \begin{split}
            \ce{HSO3-(aq) <=> SO3^2-(aq) + H+(aq)}\\
            k_a=\frac{[\ce{H+}][\ce{SO3^2-}]}{[\ce{HSO3-}]}
          \end{split}
          \label{1}
        \end{equation}

      \item

        \begin{equation}
          \begin{split}
            \ce{HPO4^2-(aq) <=> PO4^3-(aq) + H+(aq)}\\
            k_a=\frac{[\ce{H+}][\ce{PO4^3-}]}{[\ce{HPO4^2-}]}
          \end{split}
          \label{2}
        \end{equation}

      \item

        \begin{equation}
          \begin{split}
            \ce{HNO2(aq) <=> NO2-(aq) + H+(aq)}\\
            k_a=\frac{[\ce{H+}][\ce{NO2-}]}{[\ce{HNO2}]}
          \end{split}
          \label{3}
        \end{equation}

    \end{enumerate}

    \setcounter{enumi}{47}

  \item

    \begin{equation}
      \begin{split}
        \ce{HC4H3N2O3(aq) <=> C4H3N2O3-(aq) + H+(aq)}\\
        k_a=\frac{[\ce{C4H3N2O3-}][\ce{H+}]}{[\ce{HC4H3N2O3}]}\\
        2.34=-\log_{10}\left( [\ce{H+}] \right)\\
        [\ce{H+}]=.0046\left[ \si{\Molar} \right]\\
        \frac{9}{128}=.07[\si{\mole}_{\ce{HC4H3N2O3}}]\\
        \frac{.07}{.325}=.216[\si{\Molar}]\\
      \begin{array}{c | c c c} & [\ce{HC4H3N2O3}] & [\ce{C4H3N2O3-}] & [\ce{H+}]\\ \hline  \text{I} & .216 & 0 & 0\\ \text{C} & -.0046 & .0046 & .0046\\ \text{E} & .2114 & .0046 & .0046  \end{array}\\
      \frac{.0046^2}{.2114}=1\cdot10^{-4}
      \end{split}
      \label{4}
    \end{equation}

    \setcounter{enumi}{53}

  \item

    \begin{enumerate}

      \item 

    \begin{equation}
      \begin{split}
        \frac{x^2}{.279}=3.38\cdot10^{-5}\\
        x=.00307\left[ \si{\Molar} \right]\\
      \end{split}
      \label{5}
    \end{equation}

      \item

    \begin{equation}
      \begin{split}
        \frac{10^{-14}}{.00307}=3.26\cdot10^{-12}\\
      \end{split}
      \label{6}
    \end{equation}

      \item

    \begin{equation}
      \begin{split}
        -\log_{10}\left( .00307 \right)=2.52\\
      \end{split}
      \label{7}
    \end{equation}

      \item

    \begin{equation}
      \begin{split}
        \frac{.00307}{.279}\cdot100=1.1\%
      \end{split}
      \label{8}
    \end{equation}

    \end{enumerate}

    \setcounter{enumi}{71}

  \item

    \begin{equation}
      \begin{split}
        \begin{array}{c | c c c} & [\ce{NaHCO3}] & [\ce{NaCO3-}] & [\ce{H+}]\\ \hline  \text{I} & .72 & 0 & 0\\ \text{C} & -x & x & x\\ \text{E} & .72-x & x & x  \end{array}\\
        k_a=8.369\cdot10^{-21}\\
        \frac{x^2}{.72-x}=8.369\cdot10^{-21}\\
        x=7.7625\cdot10^{-11}\\
        \text{p\ce{H}}=-\log_{10}\left( x \right)=10.11\\
        [\ce{OH-}]=10^{-(14-10.11)}=1.3\cdot10^{-4}
      \end{split}
      \label{9}
    \end{equation}

\end{enumerate}

\end{document}

