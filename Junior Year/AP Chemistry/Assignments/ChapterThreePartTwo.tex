%%%%%%%%%%%%%%%%%%%%%%%%%%%%%%%%%%%%%%%%%%%%%%%%%%%%%%%%%%%%%%%%%%%%%%%%%%%%%%%%%%%%%%%%%%%%%%%%%%%%%%%%%%%%%%%%%%%%%%%%%%%%%%%%%%%%%%%%%%%%%%%%%%%%%%%%%%%%%%%%%%%%%%%%%%%%%%%%%%%%%%%%%%%%
% Written By Michael Brodskiy
% Class: AP Chemistry
% Professor: J. Morgan
%%%%%%%%%%%%%%%%%%%%%%%%%%%%%%%%%%%%%%%%%%%%%%%%%%%%%%%%%%%%%%%%%%%%%%%%%%%%%%%%%%%%%%%%%%%%%%%%%%%%%%%%%%%%%%%%%%%%%%%%%%%%%%%%%%%%%%%%%%%%%%%%%%%%%%%%%%%%%%%%%%%%%%%%%%%%%%%%%%%%%%%%%%%%

\documentclass[12pt]{article} 
\usepackage{alphalph}
\usepackage[utf8]{inputenc}
\usepackage[russian,english]{babel}
\usepackage{titling}
\usepackage{amsmath}
\usepackage{graphicx}
\usepackage{enumitem}
\usepackage{amssymb}
\usepackage[super]{nth}
\usepackage{everysel}
\usepackage{ragged2e}
\usepackage{geometry}
\usepackage{fancyhdr}
\usepackage{cancel}
\usepackage{siunitx}
\geometry{top=1.0in,bottom=1.0in,left=1.0in,right=1.0in}
\newcommand{\subtitle}[1]{%
  \posttitle{%
    \par\end{center}
    \begin{center}\large#1\end{center}
    \vskip0.5em}%

}
\usepackage{hyperref}
\hypersetup{
colorlinks=true,
linkcolor=blue,
filecolor=magenta,      
urlcolor=blue,
citecolor=blue,
}

\urlstyle{same}


\title{Chapter 3 $-$ Problems 6, 60}
\date{September 14, 2020}
\author{Michael Brodskiy\\ \small Instructor: Mr. Morgan}

% Mathematical Operations:

% Sum: $$\sum_{n=a}^{b} f(x) $$
% Integral: $$\int_{lower}^{upper} f(x) dx$$
% Limit: $$\lim_{x\to\infty} f(x)$$

\begin{document}

\maketitle

\begin{enumerate}

    \setcounter{enumi}{5}

  \item A solid circular cone made of pure platinum ($d=21.45[\si{\gram\per\centi\meter\cubed}]$) has a diameter of $2.75[\si{\centi\meter}]$ and a height of $3[in]$. (a) How many moles of platinum are in the cone? (b) How many electrons are in the cone?

    \begin{enumerate}

      \item $3[in]\rightarrow7.62[\si{\centi\meter}]$
        $$m=V\rho\rightarrow \frac{1}{3}\pi\cdot 1.375^2\cdot 7.62\cdot21.45=323.61[\si{\gram}]$$
        $$\si{\mol}_{Pt}=\frac{323.61[\si{\gram}]}{195[\si{\gram\per\mole}]}=1.66[\si{\mole}]_{Pt}$$

      \item $1.66[\si{\mole}]\cdot78\cdot6.02\cdot10^{23}=7.79\cdot10^25$

    \end{enumerate}
    \setcounter{enumi}{59}

  \item When tin comes in contact with the oxygen in the air, tin (IV) oxide, $SnO_2$, is formed. A piece of tin foil, $8.25[\si{\centi\meter}]\cdot21.5[\si{\centi\meter}]\cdot.6[\si{\milli\meter}],\, (d=7.28[\si{\gram\per\centi\meter\cubec}]$, is exposed to oxygen. (a) Assuming that all the tin has reacted, what is the mass of the oxidized tin foil? (b) Air is about 21\% oxygen by volume ($d=1.309[\si{\gram\per\liter}]$). How many liters of air are required to completely react with the tin foil?
    $$Sn+O_2\rightarrow SnO_2$$

    \begin{enumerate}

      \item $.6[\si{\milli\liter}]=.06[\si{\centi\meter}]$
        $$V=8.25\cdot21.5\cdot.06=10.65[\si{\centi\meter\cubed}]\rightarrow m=\rho V=10.65\cdot7.28=77.5[\si{\gram}]$$
        $$\frac{77.5}{119}=.65[\si{\mol}_{Sn}]=1.3[\si{\mol}_{O}\cdot 16=20.8[\si{\gram}] +77.5[\si{\gram}]=98.3[\si{\gram}]$$

        \item $.21\cdot1.309=.275[\si{\gram\per\liter}_O]$
          $$\frac{1}{.275}\cdot20.8=76[\si{\liter}]$$

    \end{enumerate}



\end{enumerate}

\end{document}

