%%%%%%%%%%%%%%%%%%%%%%%%%%%%%%%%%%%%%%%%%%%%%%%%%%%%%%%%%%%%%%%%%%%%%%%%%%%%%%%%%%%%%%%%%%%%%%%%%%%%%%%%%%%%%%%%%%%%%%%%%%%%%%%%%%%%%%%%%%%%%%%%%%%%%%%%%%%%%%%%%%%%%%%%%%%%%%%%%%%%%%%%%%%%
% Written By Michael Brodskiy
% Class: AP Chemistry
% Professor: J. Morgan
%%%%%%%%%%%%%%%%%%%%%%%%%%%%%%%%%%%%%%%%%%%%%%%%%%%%%%%%%%%%%%%%%%%%%%%%%%%%%%%%%%%%%%%%%%%%%%%%%%%%%%%%%%%%%%%%%%%%%%%%%%%%%%%%%%%%%%%%%%%%%%%%%%%%%%%%%%%%%%%%%%%%%%%%%%%%%%%%%%%%%%%%%%%%

\documentclass[12pt]{article} 
\usepackage{alphalph}
\usepackage[utf8]{inputenc}
\usepackage[russian,english]{babel}
\usepackage{titling}
\usepackage{amsmath}
\usepackage{graphicx}
\usepackage{enumitem}
\usepackage{amssymb}
\usepackage[super]{nth}
\usepackage{expl3}
\usepackage[version=4]{mhchem}
\usepackage{hpstatement}
\usepackage{rsphrase}
\usepackage{everysel}
\usepackage{ragged2e}
\usepackage{geometry}
\usepackage{fancyhdr}
\usepackage{cancel}
\usepackage{siunitx}
\usepackage{chemfig}
\usepackage{multicol}
\geometry{top=1.0in,bottom=1.0in,left=1.0in,right=1.0in}
\newcommand{\subtitle}[1]{%
  \posttitle{%
    \par\end{center}
    \begin{center}\large#1\end{center}
    \vskip0.5em}%

}
\DeclareSIUnit\Molar{\textsc{m}}
\DeclareSIUnit\atm{\textsc{atm}}
\DeclareSIUnit\torr{\textsc{torr}}
\DeclareSIUnit\psi{\textsc{psi}}
\DeclareSIUnit\bar{\textsc{bar}}
\usepackage{hyperref}
\hypersetup{
colorlinks=true,
linkcolor=blue,
filecolor=magenta,      
urlcolor=blue,
citecolor=blue,
}

\urlstyle{same}


\title{Problem Set 2 $-$ Chapter 7}
\date{November 20, 2020}
\author{Michael Brodskiy\\ \small Instructor: Mr. Morgan}

% Mathematical Operations:

% Sum: $$\sum_{n=a}^{b} f(x) $$
% Integral: $$\int_{lower}^{upper} f(x) dx$$
% Limit: $$\lim_{x\to\infty} f(x)$$

\begin{document}

\maketitle

\begin{enumerate}

  \item For the following, draw the Lewis Structure, predict the molecular structure, state bond angles, and state if the compound is polar or not.

    \begin{enumerate}

      \item \ce{NF3}

        \vspace{5pt}
        \begin{multicols}{2}
          \hspace{100pt}\chemfig{\lewis{2:,N}(-[:-90]\lewis{0:4:6:,F})(-[:-210]\lewis{2:4:6:,F})(-[:-330]\lewis{0:2:6:,F})}\hspace{20pt}
        The molecular structure is most likely a \underline{Tri-pyramid}; however, it follows the exception that the bond angles are $102^{\circ}$. The compound is polar.
      \end{multicols}
        \vspace{5pt}

      \item \ce{RnF4}

        \vspace{5pt}
        \begin{multicols}{2}
          \hspace{110pt}\chemfig{\lewis{2:6:,Rn}(-[:-45]\lewis{0:2:6:,F})(-[:-135]\lewis{2:4:6:,F})(-[:-225]\lewis{2:4:6:,F})(-[:-315]\lewis{0:2:6:,F})}\hspace{20pt}
          The molecular structure is most likely a \underline{square pyramid}. The bond angles are $90^{\circ}$. The compound is polar.
      \end{multicols}
        \vspace{5pt}

      \item \ce{PCl3}

        \vspace{5pt}
        \begin{multicols}{2}
          \hspace{100pt}\chemfig{\lewis{2:,P}(-[:-90]\lewis{0:4:6:,Cl})(-[:-210]\lewis{2:4:6:,Cl})(-[:-330]\lewis{0:2:6:,Cl})}\hspace{20pt}
          The molecular structure is most likely a \underline{Tri-pyramid}; The bond angles are $109.5^{\circ}$. The compound is polar.
      \end{multicols}
        \vspace{5pt}

      \item \ce{KrF2}

        \vspace{5pt}
        \begin{multicols}{2}
          \hspace{110pt}\chemfig{\lewis{0:2:4:,Kr}(-[:-45]\lewis{0:2:6:,F})(-[:-225]\lewis{2:4:6:,F})}\hspace{20pt}
          The molecular structure is most likely \underline{linear}; The bond angles are $180^{\circ}$. The compound is non-polar.
      \end{multicols}
        \vspace{5pt}

      \item \ce{SeCl4}

        \vspace{5pt}
        \begin{multicols}{2}
          \hspace{110pt}\chemfig{\lewis{2:,Se}(-[:-45]\lewis{0:2:6:,Cl})(-[:-135]\lewis{2:4:6:,Cl})(-[:-225]\lewis{2:4:6:,Cl})(-[:-315]\lewis{0:2:6:,Cl})}\hspace{20pt}
          The molecular structure is most likely a \underline{Tri-bipyramid}. The bond angles are $90^{\circ}$. The compound is non-polar.
      \end{multicols}
        \vspace{5pt}

      \item \ce{CH3Cl}

        \vspace{5pt}
        \begin{multicols}{2}
          \hspace{110pt}\chemfig{C(-[:-45]H)(-[:-135]H)(-[:-225]H)(-[:-315]\lewis{0:2:6:,Cl})}\hspace{20pt}
          The molecular structure is most likely \underline{tetrahedral}. The bond angles are $109.5^{\circ}$. The compound is polar.
      \end{multicols}
        \vspace{5pt}

      \item \ce{CH3OH}

        \vspace{5pt}
        \begin{multicols}{2}
          \hspace{110pt}\chemfig{C(-[:-45]H)(-[:-135]H)(-[:-225]H)-[:-315]\lewis{2:6:,O}(-[:-45]H)}\hspace{20pt}
          The molecular structure is most likely \underline{tetrahedral}. The bond angles are $109.5^{\circ}$. The compound is polar.
      \end{multicols}
        \vspace{5pt}

      \item \ce{O3^2-}

        \vspace{5pt}
        \begin{multicols}{2}
          \hspace{110pt}\chemfig{\lewis{2:,O}(=[:-35]\lewis{0:2:6:,O})(-[:-144.5]\lewis{2:4:6:,O})}\hspace{20pt}
          The molecular structure is most likely \underline{bent}. The bond angles are $109.5^{\circ}$. The compound is polar.
      \end{multicols}
        \vspace{5pt}

      \item \ce{IBr2^-}

        \vspace{5pt}
        \begin{multicols}{2}
          \hspace{110pt}\chemfig{\lewis{2:,\ce{I-}}(=[:-35]\lewis{0:2:6:,Br})(=[:-144.5]\lewis{2:4:6:,Br})}\hspace{20pt}
          The molecular structure is most likely \underline{bent}. The bond angles are $109.5^{\circ}$. The compound is polar.
      \end{multicols}
        \vspace{5pt}

      \item \ce{ClO3^-}

        \vspace{5pt}
        \begin{multicols}{2}
          \hspace{100pt}\chemfig{\lewis{2:,Cl}(-[:-90]\lewis{0:4:6:,O})(-[:-210]\lewis{2:4:6:,O})(-[:-330]\lewis{0:2:6:,\ce{O-}})}\hspace{20pt}
          The molecular structure is most likely a \underline{Tri-pyramid}. The bond angles are $109.5^{\circ}$. The compound is polar.
      \end{multicols}
        \vspace{5pt}

      \item \ce{BeF2}

        \vspace{5pt}
        \begin{multicols}{2}
          \hspace{110pt}\chemfig{\lewis{2:,Be}(-[:-45]\lewis{0:2:6:,F})(=[:-225]\lewis{2:4:6:,F})}\hspace{20pt}
          The molecular structure is most likely \underline{linear}. The bond angles are $180^{\circ}$. The compound is non-polar.
      \end{multicols}
        \vspace{5pt}

      \item \ce{CO3^2-}

        \vspace{5pt}
        \begin{multicols}{2}
          \hspace{100pt}\chemfig{\lewis{2:,C}(-[:-90]\lewis{0:4:6:,\ce{O-}})(-[:-210]\lewis{2:4:6:,\ce{O-}})(=[:-330]\lewis{2:6:,O})}\hspace{20pt}
          The molecular structure is most likely a \underline{Tri-pyramid}. The bond angles are $109.5^{\circ}$. The compound is polar.
      \end{multicols}
        \vspace{5pt}

      \item \ce{CSN-}

        \vspace{5pt}
        \begin{multicols}{2}
          \hspace{110pt}\chemfig{C(=[:-144.5]\lewis{2:6:,S})(=[:-35]\lewis{2:6:,\ce{N-}})}\hspace{20pt}
          The molecular structure is most likely \underline{bent}. The bond angles are $109.5^{\circ}$. The compound is polar.
      \end{multicols}
        \vspace{5pt}

      \item \ce{N2O4}

        \vspace{5pt}
        \begin{multicols}{2}
          \hspace{110pt}\chemfig{N(=[:-225]\lewis{2:6:,O})(-[:-135]\lewis{2:4:6:,O})-N(-[:45]\lewis{0:2:6:,O})(=[:-45]\lewis{2:6:,O})}\hspace{20pt}
          The molecular structure is most likely \underline{Triangular planar}. The bond angles are $120^{\circ}$. The compound is non-polar.
      \end{multicols}
        \vspace{5pt}

      \item \ce{N2O}

        \vspace{5pt}
        \begin{multicols}{2}
          \hspace{110pt}\chemfig{O(-[:-45]\lewis{0:2:6:,N})(-[:-225]\lewis{2:4:6:,N})}\hspace{20pt}
          The molecular structure is most likely \underline{linear}. The bond angles are $180^{\circ}$. The compound is non-polar.
      \end{multicols}
        \vspace{5pt}

      \item \ce{PCl5}

        \vspace{5pt}
        \begin{multicols}{2}
          \hspace{110pt}\chemfig{P(-[:-72]\lewis{0:4:6:,Cl})(-[:-144]\lewis{2:4:6:,Cl})(-[:-216]\lewis{2:4:6:,Cl})(-[:-288]\lewis{0:2:4:,Cl})(-[:-0]\lewis{0:2:6:,Cl})}\hspace{20pt}
          The molecular structure is most likely \underline{Tri-bipyramid}. The bond angles are $120^{\circ}$ between \ce{F} and \ce{P} and $180^{\circ}$ between \ce{F} and \ce{F}. The compound is non-polar.
      \end{multicols}
        \vspace{5pt}

    \end{enumerate}
\newpage
  \item Give the bond angles to all carbon atoms in the following molecules:

    \begin{enumerate}

      \item \ce{CH3CCH}

        \begin{itemize}

          \item On the carbon attached to three hydrogen: $109.5^{\circ}$ 

          \item On the carbon bonded with the other carbons: $109.5^{\circ}$

          \item On the carbon triple bonded to another carbon and single bonded to hydrogen: $109.5^{\circ}$

        \end{itemize}

      \item \ce{C2H4}

        \begin{itemize}

          \item On both carbons: $120^{\circ}$

        \end{itemize}

      \item \ce{CHOOH}

        \begin{itemize}

          \item $120^{\circ}$

        \end{itemize}

      \item \ce{CH3CHCHCH3}

        \begin{itemize}

          \item On the carbons attached to three hydrogen: $109.5^{\circ}$

          \item On the carbons double bonded to each other and hydrogen: $120^{\circ}$

        \end{itemize}

    \end{enumerate}

\end{enumerate}

\end{document}

