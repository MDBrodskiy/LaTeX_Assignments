%%%%%%%%%%%%%%%%%%%%%%%%%%%%%%%%%%%%%%%%%%%%%%%%%%%%%%%%%%%%%%%%%%%%%%%%%%%%%%%%%%%%%%%%%%%%%%%%%%%%%%%%%%%%%%%%%%%%%%%%%%%%%%%%%%%%%%%%%%%%%%%%%%%%%%%%%%%%%%%%%%%%%%%%%%%%%%%%%%%%%%%%%%%%
% Written By Michael Brodskiy
% Class: AP Chemistry
% Professor: J. Morgan
%%%%%%%%%%%%%%%%%%%%%%%%%%%%%%%%%%%%%%%%%%%%%%%%%%%%%%%%%%%%%%%%%%%%%%%%%%%%%%%%%%%%%%%%%%%%%%%%%%%%%%%%%%%%%%%%%%%%%%%%%%%%%%%%%%%%%%%%%%%%%%%%%%%%%%%%%%%%%%%%%%%%%%%%%%%%%%%%%%%%%%%%%%%%

\documentclass[12pt]{article} 
\usepackage{alphalph}
\usepackage[utf8]{inputenc}
\usepackage[russian,english]{babel}
\usepackage{titling}
\usepackage{amsmath}
\usepackage{graphicx}
\usepackage{enumitem}
\usepackage{amssymb}
\usepackage[super]{nth}
\usepackage{expl3}
\usepackage[version=4]{mhchem}
\usepackage{hpstatement}
\usepackage{rsphrase}
\usepackage{everysel}
\usepackage{ragged2e}
\usepackage{geometry}
\usepackage{fancyhdr}
\usepackage{cancel}
\usepackage{siunitx}
\usepackage{chemfig}
\usepackage{multicol}
\geometry{top=1.0in,bottom=1.0in,left=1.0in,right=1.0in}
\newcommand{\subtitle}[1]{%
  \posttitle{%
    \par\end{center}
    \begin{center}\large#1\end{center}
    \vskip0.5em}%

}
\newcommand{\orbital}[2]{{%
    \def\+{\big|\hspace{-2pt}\overline{\underline{\hspace{2pt}\upharpoonleft}}}%
    \def\-{\overline{\underline{\downharpoonright\hspace{2pt}}}\hspace{-2pt}\big|}%
    \def\0{\big|\hspace{-2pt}\overline{\underline{\phantom{\hspace{2pt}\downharpoonright}}}}%
    \def\1{\overline{\underline{\phantom{\downharpoonright\hspace{2pt}}}}\hspace{-2pt}\big|}%
  \setlength\tabcolsep{0pt}% remove extra horizontal space from tabular
  \begin{tabular}{c}$#2$\\[2pt]#1\end{tabular}%
}}
\DeclareSIUnit\Molar{\textsc{M}}
\DeclareSIUnit\Molal{\textsc{m}}
\DeclareSIUnit\atm{\textsc{atm}}
\DeclareSIUnit\torr{\textsc{torr}}
\DeclareSIUnit\psi{\textsc{psi}}
\DeclareSIUnit\bar{\textsc{bar}}
\DeclareSIUnit\Celsius{C}
\DeclareSIUnit\degree{$^{\circ}$}
\DeclareSIUnit\calorie{cal}
\usepackage{hyperref}
\hypersetup{
colorlinks=true,
linkcolor=blue,
filecolor=magenta,      
urlcolor=blue,
citecolor=blue,
}

\urlstyle{same}


\title{Chapter 15 $-$ Problem Set 1}
\date{March 26, 2020}
\author{Michael Brodskiy\\ \small Instructor: Mr. Morgan}

% Mathematical Operations:

% Sum: $$\sum_{n=a}^{b} f(x) $$
% Integral: $$\int_{lower}^{upper} f(x) dx$$
% Limit: $$\lim_{x\to\infty} f(x)$$

\begin{document}

\maketitle

\begin{enumerate}

  \item \ce{PbI2 <=> Pb^2+ + 2I-}

    \begin{enumerate}

      \item 

        \begin{equation}
          \begin{split}
            [\ce{Pb^2+}]\left[ \ce{I-}  \right]^2=10^{-8}\\
            (x)(2x)^2=10^{-8}\\
            x=\sqrt[3]{\frac{10^{-8}}{4}}\\
            =.00136[\si{\Molar}]\\
          \end{split}
          \label{1}
        \end{equation}

      \item 

        \begin{equation}
          \begin{split}
            (x)(.01)^2=10^{-8}\\
            x=\frac{10^{-8}}{(.01)^2}\\
            =1\cdot10^{-4}[\si{\Molar}]\\
          \end{split}
          \label{2}
        \end{equation}

      \item 

        \begin{equation}
          \begin{split}
            (.02)(2x)^2=10^{-8}\\
            x=\sqrt{\frac{10^{-8}}{.08}}\\
            =3.54\cdot10^{-4}[\si{\Molar}]
          \end{split}
          \label{3}
        \end{equation}

    \end{enumerate}

    \newpage

  \item \ce{Mg(OH)2 <=> Mg^2+ + 2OH-}

    \begin{enumerate}

      \item 

        \begin{equation}
          \begin{split}
            \left[ \ce{Mg^2+} \right]\left[ \ce{OH-} \right]^2=1.8\cdot10^{-11}\\
            (x)(2x)^2=1.8\cdot10^{-11}\\
            x=\sqrt[3]{\frac{1.8\cdot10^{-11}}{4}}\\
            =1.65\cdot10^{-4}[\si{\Molar}]\\
          \end{split}
          \label{4}
        \end{equation}

      \item 

        \begin{equation}
          \begin{split}
            (.1)(2x)^2=1.8\cdot10^{-11}\\
            x=\sqrt{\frac{1.8\cdot10^{-11}}{.4}}\\
            =6.7\cdot10^{-6}[\si{\Molar}]\\
          \end{split}
          \label{5}
        \end{equation}

      \item 

        \begin{equation}
          \begin{split}
            (x)(.25)^2=1.8\cdot10^{-11}\\
            x=\frac{1.8\cdot10^{-11}}{.125}\\
            =2.88\cdot10^{-10}[\si{\Molar}]
          \end{split}
          \label{6}
        \end{equation}

    \end{enumerate}

  \item

    \begin{enumerate}

      \item 

        \begin{equation}
          \begin{split}
            .1\cdot.00045=.000045[\si{\mole}_{\ce{Ag+}}]\\
            \frac{.000045}{.35}=1.286\cdot10^{-4}[\si{\Molar}]\\
          \end{split}
          \label{7}
        \end{equation}

      \item 

        \begin{equation}
          \begin{split}
            .25\cdot.00075=.0001875[\si{\mole}_{\ce{CrO4^2-}}]\\
            \frac{.0001875}{.35}=5.36\cdot10^{-4}[\si{\Molar}]
          \end{split}
          \label{8}
        \end{equation}

      \item 

        \begin{equation}
          \begin{split}
            \left( 1.286\cdot10^{-4} \right)^2\left( 5.36\cdot10^{-4} \right)=8.86\cdot10^{-12}\\
            8.86>2\\
            \text{So solid \textit{is} formed}
          \end{split}
          \label{9}
        \end{equation}

    \end{enumerate}

  \item

        \begin{equation}
          \begin{split}
            \left( .00105 \right)^2(x)=6\cdot10^{-4}\\
            x=544[\si{\Molar}]
          \end{split}
          \label{10}
        \end{equation}

      \item \ce{PbCl2 <=> Pb^2+ + 2Cl-}

        \begin{equation}
          \begin{split}
            (x)(2x)^2=3.3\cdot10^{-3}\\
            x=.0938[\si{\Molar}]\\
          .0938\cdot.756\cdot278=19.71[\si{\gram}]\text{ at }80[\si{\degree\Celsius}]\\
            \hline
          (x)(2x)^2=1.6\cdot10^{-5}\\
            x=.0159[\si{\Molar}]\\
            .0159\cdot.756\cdot278=3.336[\si{\gram}]\text{ at }25[\si{\degree\Celsius}]\\
            \hline
            19.71-3.336=16.38[\si{\gram}]
          \end{split}
          \label{11}
        \end{equation}

\end{enumerate}

\end{document}

