%%%%%%%%%%%%%%%%%%%%%%%%%%%%%%%%%%%%%%%%%%%%%%%%%%%%%%%%%%%%%%%%%%%%%%%%%%%%%%%%%%%%%%%%%%%%%%%%%%%%%%%%%%%%%%%%%%%%%%%%%%%%%%%%%%%%%%%%%%%%%%%%%%%%%%%%%%%%%%%%%%%%%%%%%%%%%%%%%%%%%%%%%%%%
% Written By Michael Brodskiy
% Class: AP Chemistry
% Professor: J. Morgan
%%%%%%%%%%%%%%%%%%%%%%%%%%%%%%%%%%%%%%%%%%%%%%%%%%%%%%%%%%%%%%%%%%%%%%%%%%%%%%%%%%%%%%%%%%%%%%%%%%%%%%%%%%%%%%%%%%%%%%%%%%%%%%%%%%%%%%%%%%%%%%%%%%%%%%%%%%%%%%%%%%%%%%%%%%%%%%%%%%%%%%%%%%%%

\documentclass[12pt]{article} 
\usepackage{alphalph}
\usepackage[utf8]{inputenc}
\usepackage[russian,english]{babel}
\usepackage{titling}
\usepackage{amsmath}
\usepackage{graphicx}
\usepackage{enumitem}
\usepackage{amssymb}
\usepackage[super]{nth}
\usepackage{expl3}
\usepackage[version=4]{mhchem}
\usepackage{hpstatement}
\usepackage{rsphrase}
\usepackage{everysel}
\usepackage{ragged2e}
\usepackage{geometry}
\usepackage{fancyhdr}
\usepackage{cancel}
\usepackage{siunitx}
\usepackage{chemfig}
\usepackage{multicol}
\geometry{top=1.0in,bottom=1.0in,left=1.0in,right=1.0in}
\newcommand{\subtitle}[1]{%
  \posttitle{%
    \par\end{center}
    \begin{center}\large#1\end{center}
    \vskip0.5em}%

}
\newcommand{\orbital}[2]{{%
    \def\+{\big|\hspace{-2pt}\overline{\underline{\hspace{2pt}\upharpoonleft}}}%
    \def\-{\overline{\underline{\downharpoonright\hspace{2pt}}}\hspace{-2pt}\big|}%
    \def\0{\big|\hspace{-2pt}\overline{\underline{\phantom{\hspace{2pt}\downharpoonright}}}}%
    \def\1{\overline{\underline{\phantom{\downharpoonright\hspace{2pt}}}}\hspace{-2pt}\big|}%
  \setlength\tabcolsep{0pt}% remove extra horizontal space from tabular
  \begin{tabular}{c}$#2$\\[2pt]#1\end{tabular}%
}}
\DeclareSIUnit\Molar{\textsc{m}}
\DeclareSIUnit\atm{\textsc{atm}}
\DeclareSIUnit\torr{\textsc{torr}}
\DeclareSIUnit\psi{\textsc{psi}}
\DeclareSIUnit\bar{\textsc{bar}}
\DeclareSIUnit\Celsius{$^{\circ}$C}
\usepackage{hyperref}
\hypersetup{
colorlinks=true,
linkcolor=blue,
filecolor=magenta,      
urlcolor=blue,
citecolor=blue,
}

\urlstyle{same}


\title{Chapter 9 $-$ Problems 2, 6, 18}
\date{December 17, 2020}
\author{Michael Brodskiy\\ \small Instructor: Mr. Morgan}

% Mathematical Operations:

% Sum: $$\sum_{n=a}^{b} f(x) $$
% Integral: $$\int_{lower}^{upper} f(x) dx$$
% Limit: $$\lim_{x\to\infty} f(x)$$

\begin{document}

\maketitle

\begin{enumerate}

    \setcounter{enumi}{1}

  \item Benzene, a known carcinogen, was once widely used as a solvent. A sample of benzene vapor in a flask of constant volume exerts a pressure of $325[\si{\mmHg}]$ at $80[\si{\Celsius}]$. The flask is slowly cooled.

    \begin{enumerate}

      \item Assuming no condensation, use the ideal gas law to calculate the pressure of the vapor at $50[\si{\Celsius}]$; at $60[\si{\Celsius}]$

        \begin{equation}
          \begin{split}
            P_{50[\si{\Celsius}]}&=\frac{273+50}{80+273}\cdot325=297[\si{\mmHg}]\\
            P_{60[\si{\Celsius}]}&=\frac{60+273}{80+273}\cdot325=307[\si{\mmHg}]\\
          \end{split}
          \label{1}
        \end{equation}

      \item Compare your answers in (a) to the equilibrium pressure of benzene: $269[\si{\mmHg}]$ at $50[\si{\Celsius}]$, $389[\si{\mmHg}]$ at $60[\si{\Celsius}]$

        \begin{equation}
          \begin{split}
            297[\si{\mmHg}]>269[\si{\mmHg}]\\
            307[\si{\mmHg}]<389[\si{\mmHg}]\\
          \end{split}
          \label{2}
        \end{equation}

      \item On the basis of your answers to (a) and (b), predict the pressure exerted by the benzene at $50[\si{\Celsius}]$, at $60[\si{\Celsius}]$

        \begin{equation}
          \begin{split}
            P_{50[\si{\Celsius}]}&=269[\si{\mmHg}]\\
            P_{60[\si{\Celsius}]}&=307[\si{\mmHg}]\\
          \end{split}
          \label{3}
        \end{equation}

    \end{enumerate}

    \setcounter{enumi}{5}

  \item $p$-Dichlorobenzene, \ce{C6H4Cl2}, can be one of the ingredients in mothballs. Its vapor pressure at $20[\si{\Celsius}]$ is $0.40[\si{\mmHg}]$

    \begin{enumerate}

      \item How many milligrams of \ce{C6H4Cl2} will sublime into an evacuated $750[\si{\milli\liter}]$ flask at $20[\si{\Celsius}]$?
        
        \begin{equation}
          \begin{split}
            n=\frac{.75\cdot.0005}{.0821\cdot293}\\
            =.0000164[\si{\mole}]\\
            .0000164\cdot146=2.4[\si{\milli\gram}]\\
          \end{split}
          \label{4}
        \end{equation}

      \item If $5[\si{\milli\gram}]$ of $p$-Dichlorobenzene were put into an evacuated $750[\si{\milli\liter}]$ flask, how many milligrams would remain in the solid phase?

        \begin{equation}
          \begin{split}
            5-2.4=2.6[\si{\milli\gram}]\\
          \end{split}
          \label{5}
        \end{equation}

      \item What is the final pressure in an evacuated $500[\si{\milli\liter}]$ flask at $20[\si{\Celsius}]$ that contains $2[\si{\milli\gram}]$ of $p$-Dichlorobenzene? Will there be any solid in the falsk?

        \begin{equation}
          \begin{split}
            \frac{.0000137\cdot.0821\cdot293}{.5}\cdot760=.5[\si{\mmHg}]\\
            \text{The vapor pressure is: }.4[\si{\mmHg}]\\
              \text{There will be solid}
          \end{split}
          \label{6}
        \end{equation}

    \end{enumerate}

    \setcounter{enumi}{17}
  
  \item Consider the phase diagram of the compound X. Use the phase diagram to answer the following questions.

    \begin{enumerate}

      \item What is the physical state of the compound at $35[\si{\mmHg}]$ and $120[\si{\Celsius}]$?

        \begin{center}

          Vapor
          
        \end{center}

      \item What is the normal freezing point of the compound?

        \begin{center}

          $20[\si{\Celsius}]$
          
        \end{center}

      \item What is the point $A$ called?

        \begin{center}

          Liquid
          
        \end{center}

      \item What is the point $B$ called?

        \begin{center}

          Triple Point
          
        \end{center}

      \item What is the point $C$ called?

        \begin{center}

          Normal Boiling Point
          
        \end{center}

      \item What change occurs when, at a constant pressure of $33[\si{\mmHg}]$, the temperature is decreased from $40[\si{\Celsius}]$ to $-20[\si{\Celsius}]$?

        \begin{center}

          Deposition (Vapor to Solid) 
          
        \end{center}

      \item Will the solid float on the liquid?

        \begin{center}

          No
          
        \end{center}

      \item Can the compound exist as a liquid at $180[\si{\Celsius}]$ and $2[\si{\atm}]$?

        \begin{center}

          Yes
          
        \end{center}


    \end{enumerate}

\end{enumerate}

\end{document}

