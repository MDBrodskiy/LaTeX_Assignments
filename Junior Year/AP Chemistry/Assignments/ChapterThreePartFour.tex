%%%%%%%%%%%%%%%%%%%%%%%%%%%%%%%%%%%%%%%%%%%%%%%%%%%%%%%%%%%%%%%%%%%%%%%%%%%%%%%%%%%%%%%%%%%%%%%%%%%%%%%%%%%%%%%%%%%%%%%%%%%%%%%%%%%%%%%%%%%%%%%%%%%%%%%%%%%%%%%%%%%%%%%%%%%%%%%%%%%%%%%%%%%%
% Written By Michael Brodskiy
% Class: AP Chemistry
% Professor: J. Morgan
%%%%%%%%%%%%%%%%%%%%%%%%%%%%%%%%%%%%%%%%%%%%%%%%%%%%%%%%%%%%%%%%%%%%%%%%%%%%%%%%%%%%%%%%%%%%%%%%%%%%%%%%%%%%%%%%%%%%%%%%%%%%%%%%%%%%%%%%%%%%%%%%%%%%%%%%%%%%%%%%%%%%%%%%%%%%%%%%%%%%%%%%%%%%

\documentclass[12pt]{article} 
\usepackage{alphalph}
\usepackage[utf8]{inputenc}
\usepackage[russian,english]{babel}
\usepackage{titling}
\usepackage{amsmath}
\usepackage{graphicx}
\usepackage{enumitem}
\usepackage{amssymb}
\usepackage[super]{nth}
\usepackage{expl3}
\usepackage[version=4]{mhchem}
\usepackage{hpstatement}
\usepackage{rsphrase}
\usepackage{everysel}
\usepackage{ragged2e}
\usepackage{geometry}
\usepackage{fancyhdr}
\usepackage{cancel}
\usepackage{siunitx}
\geometry{top=1.0in,bottom=1.0in,left=1.0in,right=1.0in}
\newcommand{\subtitle}[1]{%
  \posttitle{%
    \par\end{center}
    \begin{center}\large#1\end{center}
    \vskip0.5em}%

}
\usepackage{hyperref}
\hypersetup{
colorlinks=true,
linkcolor=blue,
filecolor=magenta,      
urlcolor=blue,
citecolor=blue,
}

\urlstyle{same}


\title{Chapter 3 $-$ Problem 90}
\date{September 28, 2020}
\author{Michael Brodskiy\\ \small Instructor: Mr. Morgan}

% Mathematical Operations:

% Sum: $$\sum_{n=a}^{b} f(x) $$
% Integral: $$\int_{lower}^{upper} f(x) dx$$
% Limit: $$\lim_{x\to\infty} f(x)$$

\begin{document}

\maketitle

\begin{enumerate}

    \setcounter{enumi}{89}

  \item A $5.025[\si{\gram}]$ sample of calcium is burned in air to produce a mixture of two ionic compounds, calcium oxide and calcium nitride. Water is added to this mixture. It reacts with calcium oxide to form $4.832[\si{\gram}]$ of calcium hydroxide. How many grams of calcium oxide are formed? \eqref{2} How many grams of calcium nitride? \eqref{1}

    \begin{equation}
      \begin{split}
        \ce{8Ca + O2 + 2N2 -> 2CaO + 2Ca3N2}\\
        \si{\gram}_{\ce{Ca}} & = .0653\cdot 40 \\
        5.025 - 2.612 & = 2.413[\si{\gram}_{\ce{Ca}}] \\
        \si{\mole}_{\ce{Ca}} & = \frac{2.413}{40} \\
        & = .0603[\si{\mole}_{\ce{Ca}}] \\
        m_{\ce{Ca3N2}} & = \frac{.0603}{3} \cdot 148 \\
        & = 2.97[\si{\gram}_{\ce{Ca3N2}}] \\
      \end{split}
      \label{1}
    \end{equation}

    \begin{equation}
      \begin{split}
        \ce{CaO + H2O -> Ca(OH)2} \\
        \si{\mole}_{\ce{Ca(OH)2}} & = \frac{4.832}{74} \\
        & = .0653[\si{\mole}_{\ce{Ca(OH)2}}] \rightarrow .0653[\si{\mole}_{\ce{CaO}}] \\
        m_{\ce{CaO}} & = .0653\cdot56 \\
        & = 3.66[\si{\gram}_{\ce{CaO}}] \\
      \end{split}
      \label{2}
    \end{equation}

\end{enumerate}

\end{document}

