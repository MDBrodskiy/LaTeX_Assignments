%%%%%%%%%%%%%%%%%%%%%%%%%%%%%%%%%%%%%%%%%%%%%%%%%%%%%%%%%%%%%%%%%%%%%%%%%%%%%%%%%%%%%%%%%%%%%%%%%%%%%%%%%%%%%%%%%%%%%%%%%%%%%%%%%%%%%%%%%%%%%%%%%%%%%%%%%%%%%%%%%%%%%%%%%%%%%%%%%%%%%%%%%%%%
% Written By Michael Brodskiy
% Class: AP Chemistry
% Professor: J. Morgan
%%%%%%%%%%%%%%%%%%%%%%%%%%%%%%%%%%%%%%%%%%%%%%%%%%%%%%%%%%%%%%%%%%%%%%%%%%%%%%%%%%%%%%%%%%%%%%%%%%%%%%%%%%%%%%%%%%%%%%%%%%%%%%%%%%%%%%%%%%%%%%%%%%%%%%%%%%%%%%%%%%%%%%%%%%%%%%%%%%%%%%%%%%%%

\documentclass[12pt]{article} 
\usepackage{alphalph}
\usepackage[utf8]{inputenc}
\usepackage[russian,english]{babel}
\usepackage{titling}
\usepackage{amsmath}
\usepackage{graphicx}
\usepackage{enumitem}
\usepackage{amssymb}
\usepackage[super]{nth}
\usepackage{expl3}
\usepackage[version=4]{mhchem}
\usepackage{hpstatement}
\usepackage{rsphrase}
\usepackage{everysel}
\usepackage{ragged2e}
\usepackage{geometry}
\usepackage{fancyhdr}
\usepackage{cancel}
\usepackage{siunitx}
\usepackage{chemfig}
\usepackage{multicol}
\geometry{top=1.0in,bottom=1.0in,left=1.0in,right=1.0in}
\newcommand{\subtitle}[1]{%
  \posttitle{%
    \par\end{center}
    \begin{center}\large#1\end{center}
    \vskip0.5em}%

}
\newcommand{\orbital}[2]{{%
    \def\+{\big|\hspace{-2pt}\overline{\underline{\hspace{2pt}\upharpoonleft}}}%
    \def\-{\overline{\underline{\downharpoonright\hspace{2pt}}}\hspace{-2pt}\big|}%
    \def\0{\big|\hspace{-2pt}\overline{\underline{\phantom{\hspace{2pt}\downharpoonright}}}}%
    \def\1{\overline{\underline{\phantom{\downharpoonright\hspace{2pt}}}}\hspace{-2pt}\big|}%
  \setlength\tabcolsep{0pt}% remove extra horizontal space from tabular
  \begin{tabular}{c}$#2$\\[2pt]#1\end{tabular}%
}}
\DeclareSIUnit\Molar{\textsc{M}}
\DeclareSIUnit\Molal{\textsc{m}}
\DeclareSIUnit\atm{\textsc{atm}}
\DeclareSIUnit\torr{\textsc{torr}}
\DeclareSIUnit\psi{\textsc{psi}}
\DeclareSIUnit\bar{\textsc{bar}}
\DeclareSIUnit\Celsius{C}
\DeclareSIUnit\degree{$^{\circ}$}
\DeclareSIUnit\calorie{cal}
\usepackage{hyperref}
\hypersetup{
colorlinks=true,
linkcolor=blue,
filecolor=magenta,      
urlcolor=blue,
citecolor=blue,
}

\urlstyle{same}


\title{Chapter 11 $-$ Problem Set 2}
\date{February 1, 2020}
\author{Michael Brodskiy\\ \small Instructor: Mr. Morgan}

% Mathematical Operations:

% Sum: $$\sum_{n=a}^{b} f(x) $$
% Integral: $$\int_{lower}^{upper} f(x) dx$$
% Limit: $$\lim_{x\to\infty} f(x)$$

\begin{document}

\maketitle

\begin{center}
  \ce{8H+(aq) + 4Cl-(aq) + MnO4-(aq) -> 2Cl2(g) + Mn^3+(aq) + 4H2O(l)}
\end{center}

\begin{enumerate}

  \item \ce{Cl2(g)} can be generated in the laboratory by reacting potassium permanganate with an acidified solution of sodium chloride. The net-ionic equation for the reaction is given above.

    \begin{enumerate}

      \item A $25[\si{\milli\liter}]$ sample of $0.250[\si{\Molar}]$ \ce{NaCl} reacts completely with excess \ce{KMnO4(aq)}. The \ce{Cl2(g)} produced is dried and stored in a sealed container. At $22[\si{\degree\Celsius}]$ the pressure of the \ce{Cl2(g)} in the container is $.95[\si{\atm}]$

        \begin{enumerate}

          \item Calculate the number of moles of \ce{Cl-(aq)} present before any reaction occurs.

            \begin{equation}
              .025\cdot.25=6.25\cdot10^{-3}[\si{\mole}]
              \label{1}
            \end{equation}

          \item Calculate the volume, in $\si{\liter}$, of the \ce{Cl2(g)} in the sealed container.

            \begin{equation}
              \begin{split}
                V=\frac{nRT}{P}\\
                \frac{6.25\cdot10^{-3}\cdot .0821\cdot295}{.95}\cdot\frac{1}{2}=.0797[\si{\liter}]\\
              \end{split}
              \label{2}
            \end{equation}

        \end{enumerate}

        \begin{center}
          An initial-rate study was performed on the reaction system. Data for the experiment are given in the table below.\\
          \vspace{10pt}
          \begin{tabular}[H]{| c | c | c | c | c |}
            \hline
          Trial & $[\ce{Cl-}]$ & $[\ce{MnO4-}]$ & $[\ce{H+}]$ & Rate of Disappearance of \ce{MnO4-} in $\si{\Molar\per\second}$\\
          \hline
          1 & 0.0104 & 0.00400 & 3.00 & $2.25\cdot10^{-8}$\\
          \hline
          2 & 0.0312 & 0.00400 & 3.00 & $2.03\cdot10^{-7}$\\
          \hline
          3 & 0.0312 & 0.00200 & 3.00 & $1.02\cdot10^{-7}$\\
          \hline
          \end{tabular}
        \end{center}

      \item Using the information in the table, determine the order of the reaction with respect to each of the following. Justify your answers.

        \begin{enumerate}

          \item \ce{Cl-} 

            \begin{equation}
              \begin{split}
                \frac{.225}{2.03}=\left( \frac{.0104}{.0312} \right)^m\\
                m=2\\
              \end{split}
              \label{3}
            \end{equation}

          \item \ce{MnO4-}

            \begin{equation}
              \begin{split}
                \frac{1.02}{2.03}=\left( \frac{.02}{.04} \right)^n\\
                n=1\\
              \end{split}
              \label{4}
            \end{equation}

        \end{enumerate}

      \item The reaction is known to be third order with respect to \ce{H+}. Using this information and your answers to part (b) above, complete both of the following:

        \begin{enumerate}

          \item Write the rate law for the reaction.

            \begin{equation}
              rate=k[\ce{Cl-}]^2[\ce{MnO4-}][\ce{H+}]^3
              \label{5}
            \end{equation}

          \item Calculate the value of the rate constant, $k$, for the reaction, including appropriate units.

            \begin{equation}
              \begin{split}
                1.02\cdot10^{-7}=k[.0312]^2[.002][3]^3\\
              k=.00194\left[ \frac{1}{\si{\Molar^5\second}} \right]
              \end{split}
              \label{6}
            \end{equation}

        \end{enumerate}

      \item Is it likely that the reaction occurs in a single elementary step? Justify your answer.

        \begin{center}
          No, because the orders of \ce{Cl-} and \ce{H+} differ from those in the equation
        \end{center}

    \end{enumerate}

  \item \begin{center}
      $\begin{array}{l r} \text{Step I} & \ce{O3 + Cl -> O2 + ClO} \\ \text{Step II} & \ce{ClO + O -> Cl + O2}  \end{array}$
    \end{center}

    \begin{enumerate}

      \item Write a balanced equation for the overall reaction represented by Step I and Step II above.

        \begin{center}
          \ce{O3 + O -> 2O2}
        \end{center}

      \item Clearly identify the catalyst in the mechanism above. Justify your answer.

        \begin{center}
          \ce{Cl} drives the whole process, as it is not used up throughout the process
        \end{center}

      \item Clearly identify the intermediate in the mechanism above. Justify your answer.

        \begin{center}
          \ce{ClO} is the intermediate, as it is formed in the second step, and is used up throughout the process
        \end{center}

      \item If the rate law for the overall reaction is found to be $rate=k[\ce{O3}][\ce{Cl}]$, determine the following: 

        \begin{enumerate}

          \item The overall order of the reaction

            \begin{center}
              The overall order is 2
            \end{center}

          \item Appropriate units for the rate constant, $k$

            \begin{center}
            The units are then $\left[ \frac{1}{\si{\Molar\cdot time}} \right]$
            \end{center}

          \item The rate-determining step of the reaction, along with justification for your answer.

        \begin{center}
          Step I is most likely the determining step, because the orders of \ce{O3} and \ce{Cl} match that of the rate formula
        \end{center}

        \end{enumerate}

    \end{enumerate}

\end{enumerate}

\end{document}

