%%%%%%%%%%%%%%%%%%%%%%%%%%%%%%%%%%%%%%%%%%%%%%%%%%%%%%%%%%%%%%%%%%%%%%%%%%%%%%%%%%%%%%%%%%%%%%%%%%%%%%%%%%%%%%%%%%%%%%%%%%%%%%%%%%%%%%%%%%%%%%%%%%%%%%%%%%%%%%%%%%%%%%%%%%%%%%%%%%%%%%%%%%%%
% Written By Michael Brodskiy
% Class: AP Chemistry
% Professor: J. Morgan
%%%%%%%%%%%%%%%%%%%%%%%%%%%%%%%%%%%%%%%%%%%%%%%%%%%%%%%%%%%%%%%%%%%%%%%%%%%%%%%%%%%%%%%%%%%%%%%%%%%%%%%%%%%%%%%%%%%%%%%%%%%%%%%%%%%%%%%%%%%%%%%%%%%%%%%%%%%%%%%%%%%%%%%%%%%%%%%%%%%%%%%%%%%%

\documentclass[12pt]{article} 
\usepackage{alphalph}
\usepackage[utf8]{inputenc}
\usepackage[russian,english]{babel}
\usepackage{titling}
\usepackage{amsmath}
\usepackage{graphicx}
\usepackage{enumitem}
\usepackage{amssymb}
\usepackage[super]{nth}
\usepackage{expl3}
\usepackage[version=4]{mhchem}
\usepackage{hpstatement}
\usepackage{rsphrase}
\usepackage{everysel}
\usepackage{ragged2e}
\usepackage{geometry}
\usepackage{fancyhdr}
\usepackage{cancel}
\usepackage{siunitx}
\usepackage{chemfig}
\usepackage{multicol}
\usepackage{xcolor}
\usepackage{array}
\usepackage{color, colortbl}
\definecolor{cadetgrey}{rgb}{0.57, 0.64, 0.69}
\geometry{top=1.0in,bottom=1.0in,left=1.0in,right=1.0in}
\newcommand{\subtitle}[1]{%
  \posttitle{%
    \par\end{center}
    \begin{center}\large#1\end{center}
    \vskip0.5em}%

}
\newcommand{\orbital}[2]{{%
    \def\+{\big|\hspace{-2pt}\overline{\underline{\hspace{2pt}\upharpoonleft}}}%
    \def\-{\overline{\underline{\downharpoonright\hspace{2pt}}}\hspace{-2pt}\big|}%
    \def\0{\big|\hspace{-2pt}\overline{\underline{\phantom{\hspace{2pt}\downharpoonright}}}}%
    \def\1{\overline{\underline{\phantom{\downharpoonright\hspace{2pt}}}}\hspace{-2pt}\big|}%
  \setlength\tabcolsep{0pt}% remove extra horizontal space from tabular
  \begin{tabular}{c}$#2$\\[2pt]#1\end{tabular}%
}}
\DeclareSIUnit\Molar{\textsc{M}}
\DeclareSIUnit\Molal{\textsc{m}}
\DeclareSIUnit\atm{\textsc{atm}}
\DeclareSIUnit\torr{\textsc{torr}}
\DeclareSIUnit\psi{\textsc{psi}}
\DeclareSIUnit\bar{\textsc{bar}}
\DeclareSIUnit\Celsius{C}
\DeclareSIUnit\degree{$^{\circ}$}
\DeclareSIUnit\calorie{cal}
\usepackage{hyperref}
\hypersetup{
colorlinks=true,
linkcolor=blue,
filecolor=magenta,      
urlcolor=blue,
citecolor=blue,
}

\urlstyle{same}


\title{Chapter 12 $-$ Practice FRQ}
\date{February 23, 2020}
\author{Michael Brodskiy\\ \small Instructor: Mr. Morgan}

% Mathematical Operations:

% Sum: $$\sum_{n=a}^{b} f(x) $$
% Integral: $$\int_{lower}^{upper} f(x) dx$$
% Limit: $$\lim_{x\to\infty} f(x)$$

\begin{document}

\maketitle

\begin{enumerate}

  \item 

    \begin{enumerate}

      \item
        
        \begin{equation}
          \begin{split}
            k_p=\frac{\left[ \ce{CO} \right]^2}{\left[ \ce{CO2} \right]}
          \end{split}
          \label{1}
        \end{equation}

      \item

        \begin{equation}
          \begin{split}
            PV=nRT\rightarrow n=\frac{PV}{RT}\\
            n=\frac{5\cdot2}{.0821\cdot1160}\\
            n=.105\left[ \si{\mole} \right]
          \end{split}
          \label{2}
        \end{equation}

      \item

        \begin{enumerate}

          \item 

        \begin{equation}
          \begin{split}
            \begin{array}{c | c c} & [\ce{CO2}] & [\ce{CO}]\\ \hline  \text{I} & 5 & 0\\ \text{C} & -3.37 & 2x\\ \text{E} & 1.63 & 2x \end{array}\\
            x=3.37\\
            (5+3.37)-1.63=6.74\left[ \si{\atm} \right]
          \end{split}
          \label{3}
        \end{equation}

      \item

        \begin{equation}
          \begin{split}
            k_p=\frac{(6.74)^2}{1.63}\\
            k_p=27.87
          \end{split}
          \label{4}
        \end{equation}



        \end{enumerate}

      \item The catalyst would make no difference. Because the volume is said to be negligible, there would be no difference, as a catalyst only makes a reaction occur more rapidly, rather than generate more or less product. In this manner, the amount of moles, and, therefore, the volume of the gases generated does not change with a catalyst, meaning that the pressure remains the same as well.

      \item

        \begin{equation}
          \begin{split}
            Q=\frac{2^2}{2}=4\\
            4<27.87\\
            \therefore \ce{CO2}\text{ pressure will decrease, as more \ce{CO} needs to be generated}
          \end{split}
          \label{5}
        \end{equation}

    \end{enumerate}

\end{enumerate}

\end{document}

