%%%%%%%%%%%%%%%%%%%%%%%%%%%%%%%%%%%%%%%%%%%%%%%%%%%%%%%%%%%%%%%%%%%%%%%%%%%%%%%%%%%%%%%%%%%%%%%%%%%%%%%%%%%%%%%%%%%%%%%%%%%%%%%%%%%%%%%%%%%%%%%%%%%%%%%%%%%%%%%%%%%%%%%%%%%%%%%%%%%%%%%%%%%%
% Written By Michael Brodskiy
% Class: AP Chemistry
% Professor: J. Morgan
%%%%%%%%%%%%%%%%%%%%%%%%%%%%%%%%%%%%%%%%%%%%%%%%%%%%%%%%%%%%%%%%%%%%%%%%%%%%%%%%%%%%%%%%%%%%%%%%%%%%%%%%%%%%%%%%%%%%%%%%%%%%%%%%%%%%%%%%%%%%%%%%%%%%%%%%%%%%%%%%%%%%%%%%%%%%%%%%%%%%%%%%%%%%

\documentclass[12pt]{article} 
\usepackage{alphalph}
\usepackage[utf8]{inputenc}
\usepackage[russian,english]{babel}
\usepackage{titling}
\usepackage{amsmath}
\usepackage{graphicx}
\usepackage{enumitem}
\usepackage{amssymb}
\usepackage[super]{nth}
\usepackage{expl3}
\usepackage[version=4]{mhchem}
\usepackage{hpstatement}
\usepackage{rsphrase}
\usepackage{everysel}
\usepackage{ragged2e}
\usepackage{geometry}
\usepackage{fancyhdr}
\usepackage{cancel}
\usepackage{siunitx}
\usepackage{chemfig}
\usepackage{multicol}
\usepackage{xcolor}
\usepackage{array}
\usepackage{color, colortbl}
\definecolor{cadetgrey}{rgb}{0.57, 0.64, 0.69}
\geometry{top=1.0in,bottom=1.0in,left=1.0in,right=1.0in}
\newcommand{\subtitle}[1]{%
  \posttitle{%
    \par\end{center}
    \begin{center}\large#1\end{center}
    \vskip0.5em}%

}
\newcommand{\orbital}[2]{{%
    \def\+{\big|\hspace{-2pt}\overline{\underline{\hspace{2pt}\upharpoonleft}}}%
    \def\-{\overline{\underline{\downharpoonright\hspace{2pt}}}\hspace{-2pt}\big|}%
    \def\0{\big|\hspace{-2pt}\overline{\underline{\phantom{\hspace{2pt}\downharpoonright}}}}%
    \def\1{\overline{\underline{\phantom{\downharpoonright\hspace{2pt}}}}\hspace{-2pt}\big|}%
  \setlength\tabcolsep{0pt}% remove extra horizontal space from tabular
  \begin{tabular}{c}$#2$\\[2pt]#1\end{tabular}%
}}
\DeclareSIUnit\Molar{\textsc{M}}
\DeclareSIUnit\Molal{\textsc{m}}
\DeclareSIUnit\atm{\textsc{atm}}
\DeclareSIUnit\torr{\textsc{torr}}
\DeclareSIUnit\psi{\textsc{psi}}
\DeclareSIUnit\bar{\textsc{bar}}
\DeclareSIUnit\Celsius{C}
\DeclareSIUnit\degree{$^{\circ}$}
\DeclareSIUnit\calorie{cal}
\usepackage{hyperref}
\hypersetup{
colorlinks=true,
linkcolor=blue,
filecolor=magenta,      
urlcolor=blue,
citecolor=blue,
}

\urlstyle{same}


\title{Chapter 14 $-$ Practice FRQ 3}
\date{March 19, 2020}
\author{Michael Brodskiy\\ \small Instructor: Mr. Morgan}

% Mathematical Operations:

% Sum: $$\sum_{n=a}^{b} f(x) $$
% Integral: $$\int_{lower}^{upper} f(x) dx$$
% Limit: $$\lim_{x\to\infty} f(x)$$

\begin{document}

\maketitle

\begin{enumerate}

  \item

    \begin{enumerate}

      \item 

        \begin{equation}
          \begin{split}
            [\ce{H+}]&= 10^{-4.95}\\
            &= 1.12\cdot10^{-5}
          \end{split}
          \label{1}
        \end{equation}

      \item 

        \begin{equation}
          \begin{split}
            K_a=\frac{\left[ \ce{H+} \right]\left[ \ce{OBr-} \right]}{\left[ \ce{HOBr} \right]}\\
            \left( \frac{2.3\cdot10^{-9}}{\left( 1.8\cdot10^{-5} \right)^2} \right)=.14[\si{\Molar}]\\
          \end{split}
          \label{2}
        \end{equation}

      \item 

        \begin{enumerate}

          \item 

            \begin{equation}
              \begin{split}
                V&=\frac{2\cdot.065\cdot.146}{.115}\\
                &= .0412[\si{\liter}]
              \end{split}
              \label{3}
            \end{equation}

          \item 

            \begin{equation}
              \begin{split}
                .0412\cdot.115=.004738[\si{\mole}]\\
                \frac{.004738}{.0412+.065}=.0446[\si{\Molar}]\\
                K_b=\frac{10^{-14}}{2.3\cdot10^{-9}}\\
                =4.35\cdot10^{-6}\\
                x=\sqrt{.0446\cdot4.35\cdot10^{-6}}\\
                =4.4\cdot10^{-4}\\
                14+\log_{10}\left( 4.4\cdot10^{-4} \right)=10.65\\
                \text{The \ce{pH} is greater than 7}
              \end{split}
              \label{4}
            \end{equation}

        \end{enumerate}


      \item 

        \begin{equation}
          \begin{split}
            K_a=\frac{\left[ \ce{OBr-} \right]\left[ \ce{H+} \right]}{\left[ \ce{HOBr} \right]}=\frac{\frac{x}{.125}\cdot5\cdot10^{-9}}{\frac{.02}{.125}}\\
            \frac{2.3}{5}\left( \frac{.02}{\cancel{.125}} \right)=\frac{x}{\cancel{.125}}\\
            x=.0092\left[ \si{\mole} \right]
          \end{split}
          \label{4}
        \end{equation}

      \item Because \ce{HBrO3} has more molecules, it is greater in size. Due to a larger size, the intermolecular forces are weaker, which makes it dissociates more easily. Because it dissociates more easily, the $K_a$ value is greater, which means it is a stronger acid.

    \end{enumerate}

\end{enumerate}

\end{document}

