%%%%%%%%%%%%%%%%%%%%%%%%%%%%%%%%%%%%%%%%%%%%%%%%%%%%%%%%%%%%%%%%%%%%%%%%%%%%%%%%%%%%%%%%%%%%%%%%%%%%%%%%%%%%%%%%%%%%%%%%%%%%%%%%%%%%%%%%%%%%%%%%%%%%%%%%%%%%%%%%%%%%%%%%%%%%%%%%%%%%%%%%%%%%
% Written By Michael Brodskiy
% Class: AP Chemistry
% Professor: J. Morgan
%%%%%%%%%%%%%%%%%%%%%%%%%%%%%%%%%%%%%%%%%%%%%%%%%%%%%%%%%%%%%%%%%%%%%%%%%%%%%%%%%%%%%%%%%%%%%%%%%%%%%%%%%%%%%%%%%%%%%%%%%%%%%%%%%%%%%%%%%%%%%%%%%%%%%%%%%%%%%%%%%%%%%%%%%%%%%%%%%%%%%%%%%%%%

\documentclass[12pt]{article} 
\usepackage{alphalph}
\usepackage[utf8]{inputenc}
\usepackage[russian,english]{babel}
\usepackage{titling}
\usepackage{amsmath}
\usepackage{graphicx}
\usepackage{enumitem}
\usepackage{amssymb}
\usepackage[super]{nth}
\usepackage{expl3}
\usepackage[version=4]{mhchem}
\usepackage{hpstatement}
\usepackage{rsphrase}
\usepackage{everysel}
\usepackage{ragged2e}
\usepackage{geometry}
\usepackage{fancyhdr}
\usepackage{cancel}
\usepackage{siunitx}
\usepackage{chemfig}
\usepackage{multicol}
\usepackage{xcolor}
\usepackage{array}
\usepackage{color, colortbl}
\definecolor{cadetgrey}{rgb}{0.57, 0.64, 0.69}
\geometry{top=1.0in,bottom=1.0in,left=1.0in,right=1.0in}
\newcommand{\subtitle}[1]{%
  \posttitle{%
    \par\end{center}
    \begin{center}\large#1\end{center}
    \vskip0.5em}%

}
\newcommand{\orbital}[2]{{%
    \def\+{\big|\hspace{-2pt}\overline{\underline{\hspace{2pt}\upharpoonleft}}}%
    \def\-{\overline{\underline{\downharpoonright\hspace{2pt}}}\hspace{-2pt}\big|}%
    \def\0{\big|\hspace{-2pt}\overline{\underline{\phantom{\hspace{2pt}\downharpoonright}}}}%
    \def\1{\overline{\underline{\phantom{\downharpoonright\hspace{2pt}}}}\hspace{-2pt}\big|}%
  \setlength\tabcolsep{0pt}% remove extra horizontal space from tabular
  \begin{tabular}{c}$#2$\\[2pt]#1\end{tabular}%
}}
\DeclareSIUnit\Molar{\textsc{M}}
\DeclareSIUnit\Molal{\textsc{m}}
\DeclareSIUnit\atm{\textsc{atm}}
\DeclareSIUnit\torr{\textsc{torr}}
\DeclareSIUnit\psi{\textsc{psi}}
\DeclareSIUnit\bar{\textsc{bar}}
\DeclareSIUnit\Celsius{C}
\DeclareSIUnit\degree{$^{\circ}$}
\DeclareSIUnit\calorie{cal}
\usepackage{hyperref}
\hypersetup{
colorlinks=true,
linkcolor=blue,
filecolor=magenta,      
urlcolor=blue,
citecolor=blue,
}

\urlstyle{same}


\title{Chapter 12 $-$ Problems 6, 16, 18, 22}
\date{February 11, 2020}
\author{Michael Brodskiy\\ \small Instructor: Mr. Morgan}

% Mathematical Operations:

% Sum: $$\sum_{n=a}^{b} f(x) $$
% Integral: $$\int_{lower}^{upper} f(x) dx$$
% Limit: $$\lim_{x\to\infty} f(x)$$

\begin{document}

\maketitle

\begin{enumerate}

    \setcounter{enumi}{5}

  \item Write equilibrium constant ($k$) expressions for the following reactions:

    \begin{enumerate}

      \item \ce{Na2CO3(s) <=> 2NaO(s) + CO2(g)}

        \begin{equation}
          k=[P_{\ce{CO2}}]
          \label{1}
        \end{equation}

      \item \ce{C2H6(g) + 2H2O(l) <=> 2CO(g) + 5H2(g)}

        \begin{equation}
          k=\left[ \frac{\left( P_{\ce{CO}} \right)^2\left( P_{\ce{H2}} \right)^5}{P_{\ce{C2H6}}} \right]
          \label{2}
        \end{equation}

      \item \ce{4NO(g) + 6H2O(g) <=> 4NH3(g) + 5O2(g)}

        \begin{equation}
          k=\left[ \frac{\left( P_{\ce{O2}} \right)^5\left( P_{\ce{NH3}} \right)^4}{\left(P_{\ce{NO}}\right)^4\left( P_{\ce{H2O}} \right)^6} \right]
          \label{3}
        \end{equation}

      \item \ce{NH3(g) + HI(l) <=> NH4I(s)}

        \begin{equation}
          k=\left[ \frac{1}{\left( P_{\ce{NH3}} \right)} \right]
          \label{4}
        \end{equation}

    \end{enumerate}

    \setcounter{enumi}{15}

  \item At $800[\si{\degree\Celsius}]$, $k=2.2\cdot10^{-4}$ for the following reaction. Calculate $k$ at $800[\si{\degree\Celsius}]$ for:

    \begin{center}
      \ce{2H2S(g) <=> 2H2(g) + S2(g)}
    \end{center}

    \begin{enumerate}

      \item the synthesis of one mole of \ce{H2S} from \ce{H2} and \ce{S2} gases.

        \begin{equation}
          \begin{split}
            k_i=\frac{1}{k_f}\\
            k_i=k_f^{\frac{1}{2}}\\
            k=\frac{1}{k^{\frac{1}{2}}}\\
            =67.42
          \end{split}
          \label{5}
        \end{equation}

      \item the decomposition of one mole of \ce{H2S} gas

        \begin{equation}
          \begin{split}
            k_i=k^{\frac{1}{2}}\\
            \left(2.2\cdot10^{-4}\right)^{.5}=.015
          \end{split}
          \label{6}
        \end{equation}

    \end{enumerate}

    \setcounter{enumi}{17}

  \item Given the following data at $25[\si{\degree\Celsius}]$, calculate $k$ for the formation of one mole of \ce{NOBr} from its elements in the gaseous state

    \begin{center}
      $\begin{array}{l r}
        \ce{2NO(g) <=> N2(g) + O2(g)} & k=1\cdot10^{-30}\\
        \ce{2NO(g) + Br2(g) <=> 2NOBr(g)} & k=8\cdot10\\
      \end{array}$
    \end{center}

    \begin{equation}
      \begin{split}
        k_1=\left(\frac{1}{10^{-30}}\right)^(.5)\\
        k_1=10^{15}\\
        k_2=\sqrt{80}\\
        k_2=8.94\\
        k_{total}=8.94\cdot10^15
      \end{split}
      \label{7}
    \end{equation}

    \setcounter{enumi}{21}

  \item Calculate $k$ for the formation of methyl alcohol at $100[\si{\degree\Celsius}]$, given that at equilibrium, the partial pressures of the gases are $P_{\ce{CO}}=.814[\si{\atm}]$, $P_{\ce{H2}}=.274[\si{\atm}]$, and $P_{\ce{CH3OH}}=.0512[\si{\atm}]$

    \begin{center}
      \ce{CO(g) + 2H2(g) <=> CH3OH(g)}
    \end{center}

    \begin{equation}
      \begin{split}
        k=\frac{.0512}{.814\cdot(.274)^2}\\
        k=.838
      \end{split}
      \label{8}
    \end{equation}

\end{enumerate}

\end{document}

