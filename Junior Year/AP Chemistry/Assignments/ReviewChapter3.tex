%%%%%%%%%%%%%%%%%%%%%%%%%%%%%%%%%%%%%%%%%%%%%%%%%%%%%%%%%%%%%%%%%%%%%%%%%%%%%%%%%%%%%%%%%%%%%%%%%%%%%%%%%%%%%%%%%%%%%%%%%%%%%%%%%%%%%%%%%%%%%%%%%%%%%%%%%%%%%%%%%%%%%%%%%%%%%%%%%%%%%%%%%%%%
% Written By Michael Brodskiy
% Class: AP Chemistry
% Professor: J. Morgan
%%%%%%%%%%%%%%%%%%%%%%%%%%%%%%%%%%%%%%%%%%%%%%%%%%%%%%%%%%%%%%%%%%%%%%%%%%%%%%%%%%%%%%%%%%%%%%%%%%%%%%%%%%%%%%%%%%%%%%%%%%%%%%%%%%%%%%%%%%%%%%%%%%%%%%%%%%%%%%%%%%%%%%%%%%%%%%%%%%%%%%%%%%%%

\documentclass[12pt]{article} 
\usepackage{alphalph}
\usepackage[utf8]{inputenc}
\usepackage[russian,english]{babel}
\usepackage{titling}
\usepackage{amsmath}
\usepackage{graphicx}
\usepackage{enumitem}
\usepackage{amssymb}
\usepackage[super]{nth}
\usepackage{expl3}
\usepackage[version=4]{mhchem}
\usepackage{hpstatement}
\usepackage{rsphrase}
\usepackage{everysel}
\usepackage{ragged2e}
\usepackage{geometry}
\usepackage{fancyhdr}
\usepackage{cancel}
\usepackage{siunitx}
\geometry{top=1.0in,bottom=1.0in,left=1.0in,right=1.0in}
\newcommand{\subtitle}[1]{%
  \posttitle{%
    \par\end{center}
    \begin{center}\large#1\end{center}
    \vskip0.5em}%

}
\usepackage{hyperref}
\hypersetup{
colorlinks=true,
linkcolor=blue,
filecolor=magenta,      
urlcolor=blue,
citecolor=blue,
}

\urlstyle{same}


\title{Chapter 3 Review Sheet}
\date{September 21, 2020}
\author{Michael Brodskiy\\ \small Instructor: Mr. Morgan}

% Mathematical Operations:

% Sum: $$\sum_{n=a}^{b} f(x) $$
% Integral: $$\int_{lower}^{upper} f(x) dx$$
% Limit: $$\lim_{x\to\infty} f(x)$$

\begin{document}

\maketitle

\begin{enumerate}

  \item Answer the following for $25[\si{\gram}]$ of sulfuric acid (\ce{H2SO4}):

    \begin{enumerate}

      \item The number of grams of oxygen \eqref{1}

        \begin{equation}
          \begin{split}
        \frac{64}{98} & = .653\\
        .653\cdot 25[\si{\gram}] & = 16.325[\si{\gram}_{\ce{O}}]
      \end{split}
        \label{1}
      \end{equation}

    \item The number of molecules of sulfuric acid \eqref{2}

      \begin{equation}
        \begin{split}
        \frac{25}{98} & = .255[\si{\mole}_{\ce{H2SO4}}] \\
        \ce{H2SO4}_{molecules} & = .255\cdot6.022\cdot10^{23} \\
        & = 1.54\cdot10^{23}[\ce{H2SO4}]
      \end{split}
        \label{2}
      \end{equation}

    \item  The number of atoms of hydrogen \eqref{3} 

      \begin{equation}
        \begin{split}
          \text{2 \ce{H} atoms in \ce{H2SO4}} \\
          \ce{H}_{atoms} & = 2\cdot1.54\cdot10^{23} \\
          & = 3.08\cdot10^{23}[\ce{H}_{atoms}]
      \end{split}
        \label{3}
      \end{equation}

    \end{enumerate}

  \item Calculate the mass percent of hydrogen in \ce{(NH4)2SO4} \eqref{4}

    \begin{equation}
      \begin{split}
      \frac{8}{132} \cdot 100\% & = 6\%
    \end{split}
      \label{4}
    \end{equation}

  \item A sample of a compound that contains \ce{Cl} and \ce{O} reacts with excess hydrogen to give $0.233[\si{\gram}]$ of \ce{HCl} and $0.403[\si{\gram}]$ of water. Determine the empirical formula \eqref{5}

    \begin{equation}
      \begin{split}
        \si{\mole}_{\ce{Cl}} & = \frac{.233}{36} \\
        & = .0064[\si{\mole}_{\ce{Cl}}] \\
        \si{\mole}_{\ce{O}} & = \frac{.403}{18} \\
      & = .022[\si{\mole}_{\ce{O}}] \\
      \ce{Cl}_{\frac{.022}{.0064}}\ce{O}_{\frac{.0064}{.0064}} & = \ce{Cl2O7}
      \end{split}
      \label{5}
    \end{equation}

\end{enumerate}

\end{document}

