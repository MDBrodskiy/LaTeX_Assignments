%%%%%%%%%%%%%%%%%%%%%%%%%%%%%%%%%%%%%%%%%%%%%%%%%%%%%%%%%%%%%%%%%%%%%%%%%%%%%%%%%%%%%%%%%%%%%%%%%%%%%%%%%%%%%%%%%%%%%%%%%%%%%%%%%%%%%%%%%%%%%%%%%%%%%%%%%%%%%%%%%%%%%%%%%%%%%%%%%%%%%%%%%%%%
% Written By Michael Brodskiy
% Class: AP Chemistry
% Professor: J. Morgan
%%%%%%%%%%%%%%%%%%%%%%%%%%%%%%%%%%%%%%%%%%%%%%%%%%%%%%%%%%%%%%%%%%%%%%%%%%%%%%%%%%%%%%%%%%%%%%%%%%%%%%%%%%%%%%%%%%%%%%%%%%%%%%%%%%%%%%%%%%%%%%%%%%%%%%%%%%%%%%%%%%%%%%%%%%%%%%%%%%%%%%%%%%%%

\documentclass[12pt]{article} 
\usepackage{alphalph}
\usepackage[utf8]{inputenc}
\usepackage[russian,english]{babel}
\usepackage{titling}
\usepackage{amsmath}
\usepackage{graphicx}
\usepackage{enumitem}
\usepackage{amssymb}
\usepackage[super]{nth}
\usepackage{expl3}
\usepackage[version=4]{mhchem}
\usepackage{hpstatement}
\usepackage{rsphrase}
\usepackage{everysel}
\usepackage{ragged2e}
\usepackage{geometry}
\usepackage{fancyhdr}
\usepackage{cancel}
\usepackage{siunitx}
\geometry{top=1.0in,bottom=1.0in,left=1.0in,right=1.0in}
\newcommand{\subtitle}[1]{%
  \posttitle{%
    \par\end{center}
    \begin{center}\large#1\end{center}
    \vskip0.5em}%

}
\usepackage{hyperref}
\hypersetup{
colorlinks=true,
linkcolor=blue,
filecolor=magenta,      
urlcolor=blue,
citecolor=blue,
}

\urlstyle{same}


\title{Chapter 3 Review Sheet}
\date{September 21, 2020}
\author{Michael Brodskiy\\ \small Instructor: Mr. Morgan}

% Mathematical Operations:

% Sum: $$\sum_{n=a}^{b} f(x) $$
% Integral: $$\int_{lower}^{upper} f(x) dx$$
% Limit: $$\lim_{x\to\infty} f(x)$$

\begin{document}

\maketitle

\begin{enumerate}

  \item Answer the following for $25[\si{\gram}]$ of sulfuric acid (\ce{H2SO4}):

    \begin{enumerate}

      \item The number of grams of oxygen \eqref{1}

        \begin{equation}
          \begin{split}
        \frac{64}{98} & = .653\\
        .653\cdot 25[\si{\gram}] & = 16.325[\si{\gram}_{\ce{O}}]
      \end{split}
        \label{1}
      \end{equation}

    \item The number of molecules of sulfuric acid \eqref{2}

      \begin{equation}
        \begin{split}
        \frac{25}{98} & = .255[\si{\mole}_{\ce{H2SO4}}] \\
        \ce{H2SO4}_{molecules} & = .255\cdot6.022\cdot10^{23} \\
        & = 1.54\cdot10^{23}[\ce{H2SO4}]
      \end{split}
        \label{2}
      \end{equation}

    \item  The number of atoms of hydrogen \eqref{3} 

      \begin{equation}
        \begin{split}
          \text{2 \ce{H} atoms in \ce{H2SO4}} \\
          \ce{H}_{atoms} & = 2\cdot1.54\cdot10^{23} \\
          & = 3.08\cdot10^{23}[\ce{H}_{atoms}]
      \end{split}
        \label{3}
      \end{equation}

    \end{enumerate}

  \item Calculate the mass percent of hydrogen in \ce{(NH4)2SO4} \eqref{4}

    \begin{equation}
      \begin{split}
      \frac{8}{132} \cdot 100\% & = 6\%
    \end{split}
      \label{4}
    \end{equation}

  \item A sample of a compound that contains \ce{Cl} and \ce{O} reacts with excess hydrogen to give $0.233[\si{\gram}]$ of \ce{HCl} and $0.403[\si{\gram}]$ of water. Determine the empirical formula \eqref{5}

    \begin{equation}
      \begin{split}
        \si{\mole}_{\ce{Cl}} & = \frac{.233}{36} \\
        & = .0064[\si{\mole}_{\ce{Cl}}] \\
        \si{\mole}_{\ce{O}} & = \frac{.403}{18} \\
      & = .022[\si{\mole}_{\ce{O}}] \\
      \ce{Cl}_{\frac{.022}{.0064}}\ce{O}_{\frac{.0064}{.0064}} & = \ce{Cl2O7}
      \end{split}
      \label{5}
    \end{equation}

  \item When $0.273[\si{\gram}]$ of magnesium is heated in nitrogen gas a compound forms that weights $0.378[\si{\gram}]$. Calculate the empirical formula \eqref{6}.

    \begin{equation}
      \begin{split}
        \ce{Mg + N2 -> MgN2} \\
        m_{\ce{N2}} & = .378-.273\\
        & = .105[\si{\gram}_{\ce{N2}}] \\
        \si{\mole}_{\ce{N2}} & = \frac{.105}{28} \\
        & = .00375[\si{\mole}_{\ce{N2}}] \\
        \si{\mole}_{\ce{Mg}} & = \frac{.273}{24} \\
        & = .0114[\si{\mole}_{\ce{Mg}}] \\
      \ce{Mg}_\frac{.0114}{.00375}\ce{N2}_\frac{.00375}{.00375} & \rightarrow \ce{Mg3N2}
      \end{split}
      \label{6}
    \end{equation}

  \item Find the formula of the hydrated compound \ce{MgCl2$n$(H2O)} with the following data when it is heated \eqref{7}: Mass of empty dish = $22.347[\si{\gram}]$; Initial mass of sample and dish = $25.825[\si{\gram}]$; Mass of sample and dish after heating = $23.976[\si{\gram}]$.

    \begin{equation}
      \begin{split}
        m_{\ce{MgCl2}} & = 23.976-22.347 \\
        & = 1.629[\si{\gram}_{\ce{MgCl2}}] \\
        \si{\mole}_{\ce{MgCl2}} & = \frac{1.629}{94} \\
        & = .0173[\si{\mole}_{\ce{MgCl2}}] \\ 
        m_{\ce{MgCl2$n$(H2O)}} & = 25.825-22.347\\
        & = 3.478[\si{\gram}_{\ce{MgCl2$n$(H2O)}}]\\
        m_{\ce{$n$(H2O)}} & = 3.478-1.629 \\
        & = 1.849[\si{\gram}_{\ce{H2O}}] \\
        \si{\mole}_{\ce{H2O}} = \frac{1.849}{18} \\
        & = .102[\si{\mole}_{\ce{H2O}}] \\
        \left( \frac{.0173}{.0173} \right)\ce{MgCl2}\left( \frac{.102}{.0173} \right)\ce{H2O} & \rightarrow \ce{MgCl2*6H2O}
      \end{split}
      \label{7}
    \end{equation}

  \item \ce{NaNO2} and carbon dioxide are prepared by passing nitrogen monoxide and oxygen into a solution of sodium carbonate. How many grams of \ce{NaNO2} are produced if you start with 50 grams of each reactant? \eqref{8}

    \begin{equation}
      \begin{split}
        \ce{4NO + O2 + 2Na2CO3 ->  4NaNO2 + 2CO2}\\
        \si{\mole}_{\ce{NO}} & = \frac{50}{30}\cdot.25\\
        & = .417[\si{\mole}_{\ce{NO}}]\\
        \si{\mole}_{\ce{Na2CO3}} & = \frac{50}{106} \cdot .5 \\
        & = .236[\si{\mole}_{\ce{Na2CO3}}]\\
        \si{\mole}_{\ce{O2}} & = \frac{50}{32} \\
        & = 1.5625[\si{\mole}_{\ce{O2}}]\\
        m_{\ce{NaNO2}} & = 70\cdot4\cdot.236 \\
        & = 66[\si{\gram}_{\ce{NaNO2}}]  
      \end{split}
      \label{8}
    \end{equation}

  \item Calculate the theoretical yield of \ce{ZnS}, in grams, that can be made from $0.488[\si{\gram}]$ of \ce{Zn} and $0.503[\si{\gram}]$ of sulfur \eqref{9}.

    \begin{equation}
      \begin{split}
        \ce{8Zn +S8 -> 8ZnS}\\
        \si{\mole}_{\ce{Zn}} & = \frac{.488}{65} \cdot\frac{1}{8} \\
        & = .00094[\si{\mole}_{\ce{Zn}}] \\
        \si{\mole}_{\ce{S8}} & = \frac{.503}{256} \\
        & = .002[\si{\mole}_{\ce{S8}}] \\
        m_{\ce{ZnS}} & = 97\cdot 8 \cdot .00094 \\
        & = .729[\si{\gram}_{\ce{ZnS}}]
      \end{split}
      \label{9}
    \end{equation}

  \item After $2.02[\si{\gram}]$ of \ce{Al} has reacted with $0.4[\si{\liter}]$ of \ce{HCl} ($\rho = 1.12[\si{\gram\per\milli\liter}]\rightarrow 1.12[\si{\kilo\gram\per\liter}]$), what is the mass of the remaining \ce{HCl}? \eqref{10}

      \begin{equation}
        \begin{split}
          \ce{2Al + 6HCl -> 2AlCl3 + 3H2} \\
          m_{\ce{HCl}} & = .4\cdot1.12\cdot1000\\
          & = 448[\si{\gram}_{\ce{HCl}}] \\
          \si{\mole}_{\ce{Al}} & = .5\cdot\frac{2.02}{27}\\
          & = .0374[\si{\mole}_{\ce{Al}}] \\
          \si{\mole}_{\ce{HCl}} & = 6\cdot .0374 \\
          & = .2244[\si{\mole}_{\ce{HCl}}] \\
          m_{\ce{HCl}} & = .2244\cdot 36 \\
          & = 8[\si{\gram}_{\ce{HCl}}] \text{ Used} \\
          \Delta m_{\ce{HCl}} & = 448 - 8 \\
          & = 440[\si{\gram}_{\ce{HCl}}]
        \end{split}
        \label{10}
      \end{equation}

\end{enumerate}

\end{document}

