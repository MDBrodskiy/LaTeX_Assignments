%%%%%%%%%%%%%%%%%%%%%%%%%%%%%%%%%%%%%%%%%%%%%%%%%%%%%%%%%%%%%%%%%%%%%%%%%%%%%%%%%%%%%%%%%%%%%%%%%%%%%%%%%%%%%%%%%%%%%%%%%%%%%%%%%%%%%%%%%%%%%%%%%%%%%%%%%%%%%%%%%%%%%%%%%%%%%%%%%%%%%%%%%%%%
% Written By Michael Brodskiy
% Class: AP Chemistry
% Professor: J. Morgan
%%%%%%%%%%%%%%%%%%%%%%%%%%%%%%%%%%%%%%%%%%%%%%%%%%%%%%%%%%%%%%%%%%%%%%%%%%%%%%%%%%%%%%%%%%%%%%%%%%%%%%%%%%%%%%%%%%%%%%%%%%%%%%%%%%%%%%%%%%%%%%%%%%%%%%%%%%%%%%%%%%%%%%%%%%%%%%%%%%%%%%%%%%%%

\documentclass[12pt]{article} 
\usepackage{alphalph}
\usepackage[utf8]{inputenc}
\usepackage[russian,english]{babel}
\usepackage{titling}
\usepackage{amsmath}
\usepackage{graphicx}
\usepackage{enumitem}
\usepackage{amssymb}
\usepackage[super]{nth}
\usepackage{expl3}
\usepackage[version=4]{mhchem}
\usepackage{hpstatement}
\usepackage{rsphrase}
\usepackage{everysel}
\usepackage{ragged2e}
\usepackage{geometry}
\usepackage{fancyhdr}
\usepackage{cancel}
\usepackage{siunitx} 
\geometry{top=1.0in,bottom=1.0in,left=1.0in,right=1.0in}
\newcommand{\subtitle}[1]{%
  \posttitle{%
    \par\end{center}
    \begin{center}\large#1\end{center}
    \vskip0.5em}%

}
\DeclareSIUnit\Molar{\textsc{m}}
\DeclareSIUnit\atm{\textsc{atm}}
\DeclareSIUnit\Celsius{^{\circ}\textsc{C}}
\DeclareSIUnit\torr{\textsc{torr}}
\DeclareSIUnit\mmHg{\textsc{mmHg}}
\usepackage{hyperref}
\hypersetup{
colorlinks=true,
linkcolor=blue,
filecolor=magenta,      
urlcolor=blue,
citecolor=blue,
}

\urlstyle{same}


\title{Chapter 5 $-$ Problem Set 2}
\date{October 22, 2020}
\author{Michael Brodskiy\\ \small Instructor: Mr. Morgan}

% Mathematical Operations:

% Sum: $$\sum_{n=a}^{b} f(x) $$
% Integral: $$\int_{lower}^{upper} f(x) dx$$
% Limit: $$\lim_{x\to\infty} f(x)$$

\begin{document}

\maketitle

\begin{enumerate}

  \item Calculate the molecular mass of a liquid that, when vaporized at $99[\si{\Celsius}]$ and $716[\si{\torr}]$ gave $225[\si{\milli\liter}]$ of vapor with a mass of $0.773[\si{\gram}]$. \eqref{1}

    \begin{equation}
      \begin{split}
        n&=\frac{PV}{RT}\\
        &=.00694[\si{\mole}]\\
        m_{molar}&=\frac{.773}{.00694}\\
        &=111.4\left[ \frac{\si{\gram}}{\si{\mole}} \right]
      \end{split}
      \label{1}
    \end{equation}

  \item Calculate the density of ammonium dichromate at STP. \eqref{2}

    \begin{equation}
      \begin{split}
        m_{molar}&=252\left[ \frac{\si{\gram}}{\si{\mole}} \right]\\
        \frac{n}{V}&=\frac{1}{.0821\cdot273}\\
        &=.0446\left[ \frac{\si{\mole}}{\si{\liter}} \right]\\
        252\cdot.0446&=11.2\left[ \frac{\si{\gram}}{\si{\liter}} \right]
      \end{split}
      \label{2}
    \end{equation}

  \item At what pressure will nitrogen have a density of $0.985\left[ \frac{\si{\gram}}{\si{\liter}} \right]$ at $25[\si{\Celsius}]$. \eqref{3}

    \begin{equation}
      \begin{split}
        \frac{1}{14}\cdot.985&=\frac{P}{.0821\cdot298}\\
        P&=.86[\si{\atm}]
      \end{split}
      \label{3}
    \end{equation}

  \item How many liters of \ce{CO2} measured at $26[\si{\Celsius}]$ and $767[\si{\torr}]$ are produced in the combustion of $125[\si{\milli\liter}]$ of propanol ($d = 0.804\left[ \frac{\si{\gram}}{\si{\milli\liter}} \right]$)? \eqref{4}

      \begin{equation}
        \begin{split}
          \ce{2C3H8O + 9O2}& \ce{-> 6CO2 + 8H2O}\\
          125\cdot.804&=100.5[\si{\gram}_{\ce{C3H8O}}]\\
          \frac{100.5}{60}&=1.68[\si{\mole}_{\ce{C3H8O}}]\\
          1.68\cdot3&=5.04[\si{\mole}_{\ce{CO2}}]\\
          V&=\frac{5.04\cdot.0821\cdot299}{1.009}\\
          &=123[\si{\liter}_{\ce{CO2}}]\\
        \end{split}
        \label{4}
      \end{equation}

    \item Oxygen is collected over water (vapor pressure of water = $31.8[\si{\mmHg}]$) at $30[\si{\Celsius}]$ and a barometric pressure of $742[\si{\torr}]$.  What is the partial pressure and mole fraction of oxygen? \eqref{5}

    \begin{equation}
      \begin{split}
        31.8[\si{\mmHg}]&=.0418[\si{\atm}]\\
        742[\si{\torr}]&=.976[\si{\atm}]\\
        .976-.0418&=.934[\si{\atm}]\\
        \si{\mole}_f&=\frac{.934}{.976}\\
        &=.96
      \end{split}
      \label{5}
    \end{equation}

  \item What volume is occupied by $1.25[\si{\gram}]$ of oxygen saturated with water vapor at $25[\si{\Celsius}]$ (vp water = $23.8[\si{\mmHg}]$) and a total pressure of $749[\si{\mmHg}]$? \eqref{6}

    \begin{equation}
      \begin{split}
        23.8[\si{\mmHg}]&=.0313[\si{\atm}]\\
        749[\si{\torr}]&=.986[\si{\atm}]\\
        .986-.0313&=.9547[\si{\atm}]\\
        \frac{1.25}{32}&=.0391[\si{\mole}_{\ce{O}}]\\
        V&=\frac{.0391\cdot.0821\cdot298}{.9547}\\
        &=1[\si{\liter}_{\ce{O}}]
      \end{split}
      \label{6}
    \end{equation}

  \item A quantity of nitrogen gas originally held at $3.8[\si{\atm}]$ in $1.0[\si{\liter}]$ container at $25[\si{\Celsius}]$ is transferred to a $10.0[\si{\liter}]$ container at $20[\si{\Celsius}]$.  A quantity of oxygen gas originally at $4.75[\si{\atm}]$ and $26[\si{\Celsius}]$ in a $5.0[\si{\liter}]$ container is transferred to the same container.  What is the total pressure in the new container? \eqref{7}

    \begin{equation}
      \begin{split}
        n_{\ce{O}}&=\frac{3.8\cdot1}{.0821\cdot298}\\
        &=.155[\si{\mole}_{\ce{O}}]\\
        n_{\ce{N}}&=\frac{4.75\cdot5}{.0821\cdot299}\\
        &=.967[\si{\mole}_{\ce{N}}]\\
        P_{\ce{O}}&=\frac{.155\cdot.0821\cdot293}{10}\\
        &=.373[\si{\atm}]\\
        P_{\ce{N}}&=\frac{.967\cdot.0821\cdot293}{10}\\
        &=2.33[\si{\atm}]\\
        P_{total}&=2.33+.373\\
        &=2.703[\si{\atm}]
      \end{split}
      \label{7}
    \end{equation}

  \item Nitrogen gas is held in a $2.0[\si{\liter}]$ container at $1.0[\si{\atm}]$ and $25[\si{\Celsius}]$.  Oxygen gas is held in another $3.0[\si{\liter}]$ container at $2.0[\si{\atm}]$ and $25[\si{\Celsius}]$.  The containers are then put together to allow both gases to mix.  What is the partial pressure of each gas and the total pressure in the combined container? \eqref{8}

    \begin{equation}
      \begin{split}
        n_{\ce{N}}&=\frac{2\cdot1}{.0821\cdot298}\\
        &=.0817[\si{\mole}_{\ce{N}}]\\
        n_{\ce{O}}&=\frac{3\cdot2}{.0821\cdot298}\\
        &=.245[\si{\mole}_{\ce{N}}]\\
        P_{\ce{N}}&=\frac{.0817\cdot.0821\cdot298}{5}\\
        &=.4[\si{\atm}]\\
        P_{\ce{O}}&=\frac{.245\cdot.0821\cdot298}{5}\\
        &=1.2[\si{\atm}]\\
        P_{total}&=1.2+.4\\
        &=1.6[\si{\atm}]
      \end{split}
      \label{8}
    \end{equation}

\end{enumerate}

\end{document}

