%%%%%%%%%%%%%%%%%%%%%%%%%%%%%%%%%%%%%%%%%%%%%%%%%%%%%%%%%%%%%%%%%%%%%%%%%%%%%%%%%%%%%%%%%%%%%%%%%%%%%%%%%%%%%%%%%%%%%%%%%%%%%%%%%%%%%%%%%%%%%%%%%%%%%%%%%%%%%%%%%%%%%%%%%%%%%%%%%%%%%%%%%%%%
% Written By Michael Brodskiy
% Class: AP Chemistry
% Professor: J. Morgan
%%%%%%%%%%%%%%%%%%%%%%%%%%%%%%%%%%%%%%%%%%%%%%%%%%%%%%%%%%%%%%%%%%%%%%%%%%%%%%%%%%%%%%%%%%%%%%%%%%%%%%%%%%%%%%%%%%%%%%%%%%%%%%%%%%%%%%%%%%%%%%%%%%%%%%%%%%%%%%%%%%%%%%%%%%%%%%%%%%%%%%%%%%%%

\documentclass[12pt]{article} 
\usepackage{alphalph}
\usepackage[utf8]{inputenc}
\usepackage[russian,english]{babel}
\usepackage{titling}
\usepackage{amsmath}
\usepackage{graphicx}
\usepackage{enumitem}
\usepackage{amssymb}
\usepackage[super]{nth}
\usepackage{expl3}
\usepackage[version=4]{mhchem}
\usepackage{hpstatement}
\usepackage{rsphrase}
\usepackage{everysel}
\usepackage{ragged2e}
\usepackage{geometry}
\usepackage{fancyhdr}
\usepackage{cancel}
\usepackage{siunitx}
\usepackage{chemfig}
\usepackage{multicol}
\usepackage{xcolor}
\usepackage{array}
\usepackage{color, colortbl}
\definecolor{cadetgrey}{rgb}{0.57, 0.64, 0.69}
\geometry{top=1.0in,bottom=1.0in,left=1.0in,right=1.0in}
\newcommand{\subtitle}[1]{%
  \posttitle{%
    \par\end{center}
    \begin{center}\large#1\end{center}
    \vskip0.5em}%

}
\newcommand{\orbital}[2]{{%
    \def\+{\big|\hspace{-2pt}\overline{\underline{\hspace{2pt}\upharpoonleft}}}%
    \def\-{\overline{\underline{\downharpoonright\hspace{2pt}}}\hspace{-2pt}\big|}%
    \def\0{\big|\hspace{-2pt}\overline{\underline{\phantom{\hspace{2pt}\downharpoonright}}}}%
    \def\1{\overline{\underline{\phantom{\downharpoonright\hspace{2pt}}}}\hspace{-2pt}\big|}%
  \setlength\tabcolsep{0pt}% remove extra horizontal space from tabular
  \begin{tabular}{c}$#2$\\[2pt]#1\end{tabular}%
}}
\DeclareSIUnit\Molar{\textsc{M}}
\DeclareSIUnit\Molal{\textsc{m}}
\DeclareSIUnit\atm{\textsc{atm}}
\DeclareSIUnit\torr{\textsc{torr}}
\DeclareSIUnit\psi{\textsc{psi}}
\DeclareSIUnit\bar{\textsc{bar}}
\DeclareSIUnit\Celsius{C}
\DeclareSIUnit\degree{$^{\circ}$}
\DeclareSIUnit\calorie{cal}
\usepackage{hyperref}
\hypersetup{
colorlinks=true,
linkcolor=blue,
filecolor=magenta,      
urlcolor=blue,
citecolor=blue,
}

\urlstyle{same}


\title{Chapter 14 $-$ Practice FRQ}
\date{March 5, 2020}
\author{Michael Brodskiy\\ \small Instructor: Mr. Morgan}

% Mathematical Operations:

% Sum: $$\sum_{n=a}^{b} f(x) $$
% Integral: $$\int_{lower}^{upper} f(x) dx$$
% Limit: $$\lim_{x\to\infty} f(x)$$

\begin{document}

\maketitle

\begin{enumerate}

  \item 

    \begin{enumerate}

      \item 

        \begin{equation}
          k_b=\frac{[\ce{C6H5NH3+}][\ce{OH-}]}{[\ce{C6H5NH2}]}
          \label{1}
        \end{equation}

      \item 

        \begin{equation}
          \begin{split}
            [\ce{OH-}]&=10^{8.82-14}=6.61\cdot10^{-6}\\
            k_b&=\frac{\left(6.61\cdot10^{-6}\right)^2}{.1}\\
            &=4.37\cdot10^{-10}
          \end{split}
          \label{2}
        \end{equation}

      \item \ce{C6H5NH2(aq) + H+(aq) -> C6H5NH3+(aq)}

        \begin{equation}
          \begin{split}
            \begin{tabular}{l | c c c}
              & \ce{C6H5NH2} & \ce{H+} & \ce{C6H5NH3+}\\
              \hline
              \text{I} & .0025 & .0005 & 0\\
              \text{C} & -.0005 & -.0005 & .0005\\
              \text{E} & .002 & 0 & .0005\\
            \end{tabular}\\
            [\ce{C6H5NH2}]=\frac{.002}{.03}=.066\bar{6}\\
            [\ce{C6H5NH3+}]=\frac{.0005}{.03}=.016\bar{6}\\
            p\ce{OH}=-\log_{10}\left( 4.37\cdot10^{-10}  \right)+\log_{10}\left( \frac{.016\bar{6}}{.066\bar{6}} \right)\\
            p\ce{H}=14-8.757=5.24
          \end{split}
          \label{3}
        \end{equation}

      \item 

        \begin{equation}
          \begin{split}
            \begin{tabular}{l | c c c}
              & \ce{C6H5NH2} & \ce{H+} & \ce{C6H5NH3+}\\
              \hline
              \text{I} & .0025 & .0025 & 0\\
              \text{C} & -.0025 & -.0025 & .0025\\
              \text{E} & 0 & 0 & .0025\\
            \end{tabular}\\
            [\ce{C6H5NH2}]=\frac{.0025}{.05}=.05\\
            k_a=\frac{k_w}{k_b}=2.29\cdot10^{-5}\\
            \frac{x^2}{.05}=2.29\cdot10^{-5}\\
            x=.00107\\
            -\log_{10}\left( .00107  \right)=2.97
          \end{split}
          \label{4}
        \end{equation}

      \item Erythrosine is the best option because its p$k_a$ is closest to the p\ce{H} (3 is close to 2.97)

    \end{enumerate}

\end{enumerate}

\end{document}

