%%%%%%%%%%%%%%%%%%%%%%%%%%%%%%%%%%%%%%%%%%%%%%%%%%%%%%%%%%%%%%%%%%%%%%%%%%%%%%%%%%%%%%%%%%%%%%%%%%%%%%%%%%%%%%%%%%%%%%%%%%%%%%%%%%%%%%%%%%%%%%%%%%%%%%%%%%%%%%%%%%%%%%%%%%%%%%%%%%%%%%%%%%%%
% Written By Michael Brodskiy
% Class: AP Chemistry
% Professor: J. Morgan
%%%%%%%%%%%%%%%%%%%%%%%%%%%%%%%%%%%%%%%%%%%%%%%%%%%%%%%%%%%%%%%%%%%%%%%%%%%%%%%%%%%%%%%%%%%%%%%%%%%%%%%%%%%%%%%%%%%%%%%%%%%%%%%%%%%%%%%%%%%%%%%%%%%%%%%%%%%%%%%%%%%%%%%%%%%%%%%%%%%%%%%%%%%%

\documentclass[12pt]{article} 
\usepackage{alphalph}
\usepackage[utf8]{inputenc}
\usepackage[russian,english]{babel}
\usepackage{titling}
\usepackage{amsmath}
\usepackage{graphicx}
\usepackage{enumitem}
\usepackage{amssymb}
\usepackage[super]{nth}
\usepackage{everysel}
\usepackage{ragged2e}
\usepackage{geometry}
\usepackage{fancyhdr}
\usepackage{cancel}
\usepackage{siunitx}
\geometry{top=1.0in,bottom=1.0in,left=1.0in,right=1.0in}
\newcommand{\subtitle}[1]{%
  \posttitle{%
    \par\end{center}
    \begin{center}\large#1\end{center}
    \vskip0.5em}%

}
\usepackage{hyperref}
\hypersetup{
colorlinks=true,
linkcolor=blue,
filecolor=magenta,      
urlcolor=blue,
citecolor=blue,
}

\urlstyle{same}


\title{Problem Set Chapter 1 \& 2}
\date{August 27, 2020}
\author{Michael Brodskiy\\ \small Instructor: Mr. Morgan}

% Mathematical Operations:

% Sum: $$\sum_{n=a}^{b} f(x) $$
% Integral: $$\int_{lower}^{upper} f(x) dx$$
% Limit: $$\lim_{x\to\infty} f(x)$$

\begin{document}

\maketitle

\begin{enumerate}

  \item Indicate the number of protons, electrons, and neutrons in the following:

    \begin{enumerate}

      \item Iron $-$ 26 Protons, 26 Electrons, and 30 Neutrons

      \item Tin $-$ 50 Protons, 50 Electrons, and 69 Neutrons

      \item Al$^{+3}$ $-$  13 Protons, 10 Electrons, 14 Neutrons

    \end{enumerate}

  \item Perform the conversions:

    \begin{enumerate}

      \item 6.23 $\si{\gram\per\milli\liter} \rightarrow lb\,in^{-3}$

                            $$1[in^3]\rightarrow16.387[\si{\centi\meter\cubed}]$$
                            $$1[lb]\rightarrow453.592[\si{\gram}]$$
                            $$\frac{6.23[\cancel{\si{\gram}}]}{1[\cancel{\si{\centi\meter\cubed}}]}\cdot\frac{1[lb]}{453.592[\cancel{\si{\gram}}]}\cdot\frac{16.387[\cancel{\si{\centi\meter\cubed}}]}{1[in^3]}=.225\left[\frac{lb}{in^3}\right]$$


                          \item 55 $mi/hr \rightarrow \si{\meter}/\si{\second}$

                            $$1[\si{\meter}]\rightarrow6.21\cdot10^{-4}[mi]$$
                            $$1[\si{\second}]\rightarrow2.78\cdot10^{-4}[hr]$$
                            $$\frac{55[\cancel{mi}]}{1[\cancel{hr}]}\cdot\frac{1[\si{\meter}]}{6.21\cdot10^{-4}[\cancel{mi}]}\cdot\frac{2.78\cdot10^{-4}[\cancel{hr}]}{1[\si{\second}]}=25\left[\frac{\si{\meter}}{\si{\second}}\right]$$

    \end{enumerate}

  \item For the following pairs, write the formula for the compound that would form and name it:
    
    \begin{enumerate}

        \item $S^{-2}\text{ and }Al^{+3} \rightarrow Al_2S_3 \rightarrow$ Aluminum Sulfide 

        \item $Fe^{+2}\text{ and }I^{-1} \rightarrow FeI_2 \rightarrow$ Iron Iodide

        \item $SO_3^{-2}\text{ and }Ni^{+3} \rightarrow Ni_2(SO_3)_3 \rightarrow$ Nickel (III) Sulfite

        \item $O^{-2}\text{ and }Cu^{+4} \rightarrow CuO_2 \rightarrow$ Copper (IV) Oxide

        \item $PO_4^{-3}\text{ and }NH_4^{+} \rightarrow (NH_4)_3PO_4 \rightarrow$ Ammonium Phosphate

        \item $HPO_4^{-2}\text{ and }Al^{+3} \rightarrow Al_2(HPO_4)_3 \rightarrow$ Aluminum Hydrogen Phosphate

        \item $CO_3^{-2}\text{ and }NH_4^{+} \rightarrow (NH_4)_2CO_3 \rightarrow$ Ammonium Carbonate

        \item $PO_4^{-3}\text{ and }Hg^{+2} \rightarrow Hg_3(PO_4)_2 \rightarrow$ Mercury (II) Phosphate

        \item $S^{-2}\text{ and }V^{+6} \rightarrow VS_3 \rightarrow$ Vanadium (VI) Sulfide

    \end{enumerate}

  \item An empty 3.0[$\si{\liter}$] bottle weights 1.7[$\si{\kilo\gram}$]. Filled with wine, the bottle weights 4.72[$\si{\kilo\gram}$]. The wine contains 11\% ethyl alcohol by mass. How many ounces of ethyl alcohol are present in 275[$\si{\milli\liter}$] of the wine? 

    \begin{enumerate}

      \item $4.72[\si{\kilo\gram}]-1.7[\si{\kilo\gram}]=3.02[\si{\kilo\gram}]$ of Wine

        $$\frac{275[\cancel{\si{\milli\liter}}]}{3000[\cancel{\si{\milli\liter}}]}\cdot106.527[oz]\cdot.11=1.1[oz]$$

    \end{enumerate}

  \item A lab experiment requires $0.5[\si{gram}]$ of copper wire ($d = 8.94[\si{\gram\per\milli\liter}]$). The diameter of the wire is $0.0179[in]$. Determine the length of the wire (in $\si{\centi\meter}$) to used for this experiment.Volume of cylinder = $\pi r^2 l$ 

    \begin{enumerate}

      \item $.0179[in]\rightarrow.0455[\si{\centi\meter}]\,, V=.00162l$

        $$V=\frac{m}{\rho}\Rightarrow l=\frac{.5[\cancel{\si{\gram}}]}{.00162[\cancel{\si{\meter\squared}}]8.94[\si{\gram\per\milli\liter}]}$$
        $$l=34.5[\si{\centi\meter}]$$

    \end{enumerate}

\end{enumerate}

\end{document}

