%%%%%%%%%%%%%%%%%%%%%%%%%%%%%%%%%%%%%%%%%%%%%%%%%%%%%%%%%%%%%%%%%%%%%%%%%%%%%%%%%%%%%%%%%%%%%%%%%%%%%%%%%%%%%%%%%%%%%%%%%%%%%%%%%%%%%%%%%%%%%%%%%%%%%%%%%%%%%%%%%%%%%%%%%%%%%%%%%%%%%%%%%%%%
% Written By Michael Brodskiy
% Class: AP Chemistry
% Professor: J. Morgan
%%%%%%%%%%%%%%%%%%%%%%%%%%%%%%%%%%%%%%%%%%%%%%%%%%%%%%%%%%%%%%%%%%%%%%%%%%%%%%%%%%%%%%%%%%%%%%%%%%%%%%%%%%%%%%%%%%%%%%%%%%%%%%%%%%%%%%%%%%%%%%%%%%%%%%%%%%%%%%%%%%%%%%%%%%%%%%%%%%%%%%%%%%%%

\documentclass[12pt]{article} 
\usepackage{alphalph}
\usepackage[utf8]{inputenc}
\usepackage[russian,english]{babel}
\usepackage{titling}
\usepackage{amsmath}
\usepackage{graphicx}
\usepackage{enumitem}
\usepackage{amssymb}
\usepackage[super]{nth}
\usepackage{expl3}
\usepackage[version=4]{mhchem}
\usepackage{hpstatement}
\usepackage{rsphrase}
\usepackage{everysel}
\usepackage{ragged2e}
\usepackage{geometry}
\usepackage{fancyhdr}
\usepackage{cancel}
\usepackage{siunitx}
\usepackage{chemfig}
\usepackage{multicol}
\usepackage{xcolor}
\usepackage{array}
\usepackage{color, colortbl}
\definecolor{cadetgrey}{rgb}{0.57, 0.64, 0.69}
\geometry{top=1.0in,bottom=1.0in,left=1.0in,right=1.0in}
\newcommand{\subtitle}[1]{%
  \posttitle{%
    \par\end{center}
    \begin{center}\large#1\end{center}
    \vskip0.5em}%

}
\newcommand{\orbital}[2]{{%
    \def\+{\big|\hspace{-2pt}\overline{\underline{\hspace{2pt}\upharpoonleft}}}%
    \def\-{\overline{\underline{\downharpoonright\hspace{2pt}}}\hspace{-2pt}\big|}%
    \def\0{\big|\hspace{-2pt}\overline{\underline{\phantom{\hspace{2pt}\downharpoonright}}}}%
    \def\1{\overline{\underline{\phantom{\downharpoonright\hspace{2pt}}}}\hspace{-2pt}\big|}%
  \setlength\tabcolsep{0pt}% remove extra horizontal space from tabular
  \begin{tabular}{c}$#2$\\[2pt]#1\end{tabular}%
}}
\DeclareSIUnit\Molar{\textsc{M}}
\DeclareSIUnit\Molal{\textsc{m}}
\DeclareSIUnit\atm{\textsc{atm}}
\DeclareSIUnit\torr{\textsc{torr}}
\DeclareSIUnit\psi{\textsc{psi}}
\DeclareSIUnit\bar{\textsc{bar}}
\DeclareSIUnit\Celsius{C}
\DeclareSIUnit\degree{$^{\circ}$}
\DeclareSIUnit\calorie{cal}
\usepackage{hyperref}
\hypersetup{
colorlinks=true,
linkcolor=blue,
filecolor=magenta,      
urlcolor=blue,
citecolor=blue,
}

\urlstyle{same}


\title{Chapter 12 $-$ Problems 26, 28, 50, 52}
\date{February 16, 2020}
\author{Michael Brodskiy\\ \small Instructor: Mr. Morgan}

% Mathematical Operations:

% Sum: $$\sum_{n=a}^{b} f(x) $$
% Integral: $$\int_{lower}^{upper} f(x) dx$$
% Limit: $$\lim_{x\to\infty} f(x)$$

\begin{document}

\maketitle

\begin{enumerate}

    \setcounter{enumi}{25}

  \item

    \begin{center}
      \begin{tabular}[H]{c c | c | c}
         & \ce{SO3} & \ce{SO2} & \ce{O2}\\
        \hline
        I & $.541[\si{\atm}]$ & $0[\si{\atm}]$ & $0[\si{\atm}]$\\
        \hline
        C & $-.432[\si{\atm}]$  & $.432[\si{\atm}]$ & $.216[\si{\atm}]$ \\
        \hline
        E & $.109[\si{\atm}]$  & $.432[\si{\atm}$  & $.216[\si{\atm}]$
      \end{tabular}
    \end{center}

    \begin{equation}
      \begin{split}
        k&=\frac{[.216]\cdot[.432]^2}{[.109]^2}\\
        &=3.39\\
      \end{split}
      \label{1}
    \end{equation}

    \setcounter{enumi}{27}

  \item

    \begin{enumerate}

      \item 

        \begin{equation}
          \begin{split}
            Q&=\frac{.33\cdot.65}{.026}\\
            &= 8.25\\
            Q\neq k\\
            \therefore \text{It is not at equilibrium}
          \end{split}
          \label{2}
        \end{equation}

      \item Because $Q<k$, more products need to be formed

    \end{enumerate}

    \setcounter{enumi}{49}

  \item

    \begin{enumerate}
        
      \item 

        \begin{enumerate}

      \item If \ce{O2(g)} is removed, then there will be more reactants, and, therefore, more ammonia

      \item If \ce{N2(g)} is added, then there will be more reactants, and, therefore, more ammonia

      \item If water is added, there is no effect on ammonia

      \item Because there are more gas molecules on the left, and the volume is increased, then ammonia will increase

      \item Because the reaction is exothermic, and the temperature is increased, there will be more ammonia

        \end{enumerate}

      \item $k$ is decreased in 5, but left the same in all others

    \end{enumerate}

    \setcounter{enumi}{51}

  \item

    \begin{enumerate}

      \item The equilibrium will shift to the right because it has more gas molecules

      \item The equilibrium will shift to the right because there are more gas molecules

      \item The equilibrium will shift to the right because there are more gas molecules

    \end{enumerate}

\end{enumerate}

\end{document}

