%%%%%%%%%%%%%%%%%%%%%%%%%%%%%%%%%%%%%%%%%%%%%%%%%%%%%%%%%%%%%%%%%%%%%%%%%%%%%%%%%%%%%%%%%%%%%%%%%%%%%%%%%%%%%%%%%%%%%%%%%%%%%%%%%%%%%%%%%%%%%%%%%%%%%%%%%%%%%%%%%%%%%%%%%%%%%%%%%%%%%%%%%%%%
% Written By Michael Brodskiy
% Class: AP Chemistry
% Professor: J. Morgan
%%%%%%%%%%%%%%%%%%%%%%%%%%%%%%%%%%%%%%%%%%%%%%%%%%%%%%%%%%%%%%%%%%%%%%%%%%%%%%%%%%%%%%%%%%%%%%%%%%%%%%%%%%%%%%%%%%%%%%%%%%%%%%%%%%%%%%%%%%%%%%%%%%%%%%%%%%%%%%%%%%%%%%%%%%%%%%%%%%%%%%%%%%%%

\documentclass[12pt]{article} 
\usepackage{alphalph}
\usepackage[utf8]{inputenc}
\usepackage[russian,english]{babel}
\usepackage{titling}
\usepackage{amsmath}
\usepackage{graphicx}
\usepackage{enumitem}
\usepackage{amssymb}
\usepackage[super]{nth}
\usepackage{expl3}
\usepackage[version=4]{mhchem}
\usepackage{hpstatement}
\usepackage{rsphrase}
\usepackage{everysel}
\usepackage{ragged2e}
\usepackage{geometry}
\usepackage{fancyhdr}
\usepackage{cancel}
\usepackage{siunitx}
\usepackage{chemfig}
\usepackage{multicol}
\geometry{top=1.0in,bottom=1.0in,left=1.0in,right=1.0in}
\newcommand{\subtitle}[1]{%
  \posttitle{%
    \par\end{center}
    \begin{center}\large#1\end{center}
    \vskip0.5em}%

}
\newcommand{\orbital}[2]{{%
    \def\+{\big|\hspace{-2pt}\overline{\underline{\hspace{2pt}\upharpoonleft}}}%
    \def\-{\overline{\underline{\downharpoonright\hspace{2pt}}}\hspace{-2pt}\big|}%
    \def\0{\big|\hspace{-2pt}\overline{\underline{\phantom{\hspace{2pt}\downharpoonright}}}}%
    \def\1{\overline{\underline{\phantom{\downharpoonright\hspace{2pt}}}}\hspace{-2pt}\big|}%
  \setlength\tabcolsep{0pt}% remove extra horizontal space from tabular
  \begin{tabular}{c}$#2$\\[2pt]#1\end{tabular}%
}}
\DeclareSIUnit\Molar{\textsc{M}}
\DeclareSIUnit\Molal{\textsc{m}}
\DeclareSIUnit\atm{\textsc{atm}}
\DeclareSIUnit\torr{\textsc{torr}}
\DeclareSIUnit\psi{\textsc{psi}}
\DeclareSIUnit\bar{\textsc{bar}}
\DeclareSIUnit\Celsius{C}
\DeclareSIUnit\degree{$^{\circ}$}
\DeclareSIUnit\calorie{cal}
\usepackage{hyperref}
\hypersetup{
colorlinks=true,
linkcolor=blue,
filecolor=magenta,      
urlcolor=blue,
citecolor=blue,
}

\urlstyle{same}


\title{Chapter 16 $-$ Problem Set 1}
\date{April 9, 2020}
\author{Michael Brodskiy\\ \small Instructor: Mr. Morgan}

% Mathematical Operations:

% Sum: $$\sum_{n=a}^{b} f(x) $$
% Integral: $$\int_{lower}^{upper} f(x) dx$$
% Limit: $$\lim_{x\to\infty} f(x)$$

\begin{document}

\maketitle

\begin{enumerate}

  \item 

    \begin{equation}
      \begin{split}
        \ce{2CH3OH -> 2CH4 + O2}\\
        \Delta G_{\ce{CH3OH}} =-166.3\left[ \frac{\si{\kilo\joule}}{\si{\mole}} \right]\\
        \Delta G_{\ce{CH4}} =-50.7\left[ \frac{\si{\kilo\joule}}{\si{\mole}} \right]\\
        \Delta G_{\ce{O2}} =0\left[ \frac{\si{\kilo\joule}}{\si{\mole}} \right]\\
        \Delta G=2(-50.7)-2(-166.3)=231.2[\si{\kilo\joule}]\\
        \text{Not Thermodynamically Feasible}
      \end{split}
      \label{1}
    \end{equation}

  \item

    \begin{enumerate}

      \item 

        \begin{equation}
          \begin{split}
            \text{Formed: }0\\
            \text{Broken: }\chemfig{\ce{N}~\ce{N}}\\
            0-950=950\left[ \frac{\si{\kilo\joule}}{\si{\mole}} \right]\\
          \end{split}
          \label{2}
        \end{equation}

      \item It is becoming more orderly, so negative

      \item It is this way because the sign of $\Delta S$ is negative. Because the sign is negative, $\Delta H$ is the only negative part of the formula. In this manner, the greater the temperature $\Delta S$ is multiplied by, the greater the $\Delta G$

      \item The kinetic energy of the molecules is too low to spontaneously collide at this low temperature.

    \end{enumerate}

  \item 

    \begin{enumerate}

      \item 

        \begin{enumerate}

          \item 

            \begin{equation}
              \begin{split}
                \frac{75}{56}=1.34[\si{\mole}]\\
              \end{split}
              \label{3}
            \end{equation}

          \item 

            \begin{equation}
              \begin{split}
                n=\frac{PV}{RT}\\
                \frac{2.66\cdot11.5}{298\cdot.0821}=1.25[\si{\mole}]\\
              \end{split}
              \label{4}
            \end{equation}

        \end{enumerate}

      \item 

        \begin{equation}
          \begin{split}
            \frac{1.34}{2}=.67[\si{\mole}_{\ce{Fe}}]\\
            \frac{1.25}{1.5}=.83[\si{\mole}_{\ce{O2}}]\\
            \si{\mole}_{\ce{Fe}}<\si{\mole}_{\ce{O2}}\\
            \text{So \ce{Fe} is limiting}
          \end{split}
          \label{5}
        \end{equation}
      
      \item 

        \begin{equation}
          \begin{split}
            \si{\mole}_{\ce{Fe2O3}}=\frac{\si{\mole}_{\ce{Fe}}}{2}\\
            \frac{1.34}{2}=.67[\si{\mole}_{\ce{Fe2O3}}]
          \end{split}
          \label{6}
        \end{equation}

      \item 

        \begin{enumerate}

          \item 

            \begin{equation}
              \begin{split}
                \Delta G^{\circ}_f=\Delta H^{\circ}_f - T\Delta S^{\circ}_f\\
                -740=-824-298(\Delta S^{\circ}_f)\\
                \Delta S^{\circ}_f=-.282\left[ \frac{\si{\kilo\joule}}{\si{\mole\kelvin}} \right]
              \end{split}
              \label{7}
            \end{equation}

          \item Enthalpy because the magnitude of enthalpy $>>$ the magnitude of entropy

        \end{enumerate}

      \item 

        \begin{equation}
          \begin{split}
            -280=\Delta H_{\ce{Fe2O3}}-2\Delta H_{\ce{FeO}}\\
            \Delta H_{\ce{FeO}}=\frac{-280+824}{-2}\\
            =-272\left[ \frac{\si{\kilo\joule}}{\si{\mole}} \right]
          \end{split}
          \label{8}
        \end{equation}

    \end{enumerate}

\end{enumerate}

\end{document}

