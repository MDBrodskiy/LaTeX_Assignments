%%%%%%%%%%%%%%%%%%%%%%%%%%%%%%%%%%%%%%%%%%%%%%%%%%%%%%%%%%%%%%%%%%%%%%%%%%%%%%%%%%%%%%%%%%%%%%%%%%%%%%%%%%%%%%%%%%%%%%%%%%%%%%%%%%%%%%%%%%%%%%%%%%%%%%%%%%%%%%%%%%%%%%%%%%%%%%%%%%%%%%%%%%%%
% Written By Michael Brodskiy
% Class: AP Chemistry
% Professor: J. Morgan
%%%%%%%%%%%%%%%%%%%%%%%%%%%%%%%%%%%%%%%%%%%%%%%%%%%%%%%%%%%%%%%%%%%%%%%%%%%%%%%%%%%%%%%%%%%%%%%%%%%%%%%%%%%%%%%%%%%%%%%%%%%%%%%%%%%%%%%%%%%%%%%%%%%%%%%%%%%%%%%%%%%%%%%%%%%%%%%%%%%%%%%%%%%%

\documentclass[12pt]{article} 
\usepackage{alphalph}
\usepackage[utf8]{inputenc}
\usepackage[russian,english]{babel}
\usepackage{titling}
\usepackage{amsmath}
\usepackage{graphicx}
\usepackage{enumitem}
\usepackage{amssymb}
\usepackage[super]{nth}
\usepackage{expl3}
\usepackage[version=4]{mhchem}
\usepackage{hpstatement}
\usepackage{rsphrase}
\usepackage{everysel}
\usepackage{ragged2e}
\usepackage{geometry}
\usepackage{fancyhdr}
\usepackage{cancel}
\usepackage{siunitx}
\geometry{top=1.0in,bottom=1.0in,left=1.0in,right=1.0in}
\newcommand{\subtitle}[1]{%
  \posttitle{%
    \par\end{center}
    \begin{center}\large#1\end{center}
    \vskip0.5em}%

}
\DeclareSIUnit\Molar{\textsc{m}}
\usepackage{hyperref}
\hypersetup{
colorlinks=true,
linkcolor=blue,
filecolor=magenta,      
urlcolor=blue,
citecolor=blue,
}

\urlstyle{same}


\title{Review Set Chapter 4}
\date{October 15, 2020}
\author{Michael Brodskiy\\ \small Instructor: Mr. Morgan}

% Mathematical Operations:

% Sum: $$\sum_{n=a}^{b} f(x) $$
% Integral: $$\int_{lower}^{upper} f(x) dx$$
% Limit: $$\lim_{x\to\infty} f(x)$$

\begin{document}

\maketitle

\begin{enumerate}

  \item What volume of $.08[\si{\Molar}]$ solution of copper (II) sulfate is needed to react with $.02[\si{\liter}]$ of $.2[\si{\Molar}]$ solution of sodium hydroxide? \eqref{1}

    \begin{equation}
      \begin{split}
        \ce{CuSO4 + 2NaOH } & \ce{-> Cu(OH)2 + Na2SO4}\\
        .2\cdot.02 & = .004\\
        \frac{.004}{.08}\cdot .5 & = .025[\si{\liter}]
      \end{split}
      \label{1}
    \end{equation}

  \item Give the oxidation number of each atom in the following:

    \begin{enumerate}

      \item \ce{N2H4} $\rightarrow \ce{N}=-2,\,\ce{H}=1$

      \item \ce{NOF} $\rightarrow \ce{N}=3,\,\ce{F}=-1,\,\ce{O}=-2$

      \item \ce{Sb4O10} $\rightarrow \ce{O}=-2,\,\ce{Sb}=5$

      \item \ce{CaC2O4} $\rightarrow \ce{Ca}=2,\,\ce{O}=-2,\,\ce{C}=3$

      \item \ce{HSO4} $\rightarrow \ce{S}=6,\,\ce{H}=1,\,\ce{O}=-2$

      \item \ce{Sn^4+} $\rightarrow \ce{Sn}=4$

    \end{enumerate}

  \item State which reactant is oxidized and which is reduced:

  \item How many grams of solid is produced when $13[\si{\milli\liter}]$ of $.164[\si{\Molar}]$ zinc (II) sulfate is mixed with excess ammonium sulfide:

  \item Complete each of the following, indicate the physical state of each product:

  \item Given the following reactants, write the corresponding balanced complete ionic equation. Include physical states and any charges:

  \item How many grams of solid is produced when $50[\si{\milli\liter}]$ of $.2[\si{\Molar}]$ \ce{Na2CO3} is mixed with $50[\si{\milli\liter}]$ of $.158[\si{\Molar}]$ of \ce{BaCl2}

  \item Balance the following Redox equations:

\end{enumerate}

\end{document}

