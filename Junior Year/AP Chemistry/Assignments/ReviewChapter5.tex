%%%%%%%%%%%%%%%%%%%%%%%%%%%%%%%%%%%%%%%%%%%%%%%%%%%%%%%%%%%%%%%%%%%%%%%%%%%%%%%%%%%%%%%%%%%%%%%%%%%%%%%%%%%%%%%%%%%%%%%%%%%%%%%%%%%%%%%%%%%%%%%%%%%%%%%%%%%%%%%%%%%%%%%%%%%%%%%%%%%%%%%%%%%%
% Written By Michael Brodskiy
% Class: AP Chemistry
% Professor: J. Morgan
%%%%%%%%%%%%%%%%%%%%%%%%%%%%%%%%%%%%%%%%%%%%%%%%%%%%%%%%%%%%%%%%%%%%%%%%%%%%%%%%%%%%%%%%%%%%%%%%%%%%%%%%%%%%%%%%%%%%%%%%%%%%%%%%%%%%%%%%%%%%%%%%%%%%%%%%%%%%%%%%%%%%%%%%%%%%%%%%%%%%%%%%%%%%

\documentclass[12pt]{article} 
\usepackage{alphalph}
\usepackage[utf8]{inputenc}
\usepackage[russian,english]{babel}
\usepackage{titling}
\usepackage{amsmath}
\usepackage{physics}
\usepackage{tikz}
\usepackage{mathdots}
\usepackage{yhmath}
\usepackage{cancel}
\usepackage{color}
\usepackage{siunitx}
\usepackage{array}
\usepackage{multirow}
\usepackage{amssymb}
\usepackage{gensymb}
\usepackage{tabularx}
\usepackage{booktabs}
\usetikzlibrary{fadings}
\usetikzlibrary{patterns}
\usetikzlibrary{shadows.blur}
\usetikzlibrary{shapes}
\usepackage{graphicx}
\usepackage{enumitem}
\usepackage[super]{nth}
\usepackage{expl3}
\usepackage[version=4]{mhchem}
\usepackage{hpstatement}
\usepackage{rsphrase}
\usepackage{everysel}
\usepackage{ragged2e}
\usepackage{geometry}
\usepackage{fancyhdr}
\usepackage{cancel}
\usepackage{siunitx} 
\geometry{top=1.0in,bottom=1.0in,left=1.0in,right=1.0in}
\newcommand{\subtitle}[1]{%
  \posttitle{%
    \par\end{center}
    \begin{center}\large#1\end{center}
    \vskip0.5em}%

}
\DeclareSIUnit\Molar{\textsc{m}}
\DeclareSIUnit\atm{\textsc{atm}}
\DeclareSIUnit\Celsius{^{\circ}\textsc{C}}
\DeclareSIUnit\torr{\textsc{torr}}
\DeclareSIUnit\mmHg{\textsc{mmHg}}
\usepackage{hyperref}
\hypersetup{
colorlinks=true,
linkcolor=blue,
filecolor=magenta,      
urlcolor=blue,
citecolor=blue,
}

\urlstyle{same}


\title{Chapter 5 $-$ Problems 12}
\date{November 3, 2020}
\author{Michael Brodskiy\\ \small Instructor: Mr. Morgan}

% Mathematical Operations:

% Sum: $$\sum_{n=a}^{b} f(x) $$
% Integral: $$\int_{lower}^{upper} f(x) dx$$
% Limit: $$\lim_{x\to\infty} f(x)$$

\begin{document}

\maketitle

\begin{enumerate}

  \item Calculate the density of water at STP.

    \begin{equation}
      \begin{split}
        \text{At STP: }& n=1[\si{\mole}],\,V=22.4[\si{\liter}]\\
        \rho&=\frac{18\cdot1}{22.4}\\
        &=.804\left[ \frac{\si{\gram}}{\si{\liter}} \right]
      \end{split}
      \label{1}
    \end{equation}

  \item What volume of oxygen at $50[\si{\Celsius}]$ and $1[\si{\atm}]$ is required to combust with $2[\si{\gram}]$ of \ce{C8H18}?

    \begin{equation}
      \begin{split}
        \ce{C8H18 + O2 &-> CO2 + H2O}\\
        \ce{2C8H18 + 25O2 &-> 16CO2 + 18H2O}\\
        \ce{C8H18 &->} 114\left[ \frac{\si{\gram}}{\si{\mole}} \right]\\
        \frac{2}{114}&=.0175[\si{\mole}]\\
        .0175\cdot\frac{25}{2}&=.219[\si{\mole}]\\
        V&=\frac{nRT}{P}\\
        \frac{.219\cdot.0821\cdot323}{1}&=5.808[\si{\liter}]
      \end{split}
      \label{2}
    \end{equation}
    
  \item Explain in terms of Kinetic Molecular Theory what happens to a container with gas in it when the volume is increased.

    \begin{justifying}
      When the volume is increased, the pressure decreases because the molecules make contact with the walls less often.
    \end{justifying}

  \item What volume of oxygen is collected with water vapor at $23[\si{\Celsius}]$ by reacting $2.3[\si{\gram}]$ of \ce{KClO3} if the total pressure is $742[\si{\mmHg}]$ and water vapor pressure is $21.07[\si{\mmHg}]$?

    \begin{equation}
      \ce{2KClO3 -> 2KCl + 3O2}
      \label{3}
    \end{equation}

    \begin{equation}
      \begin{split}
        742[\si{\mmHg}]&=.976[\si{\atm}]\\
        21.07[\si{\mmHg}]&=.0277[\si{\atm}]\\
        m_{\ce{KClO3}}&=123\left[ \frac{\si{\gram}}{\si{\mole}} \right]\\
        \frac{2.3}{123}&=.0187[\si{\mole}]\\
        \frac{3}{2}\cdot.0187&=.028[\si{\mole}_{\ce{O2}}]\\
        V&=\frac{nRT}{P}\\
        \frac{.028\cdot.0821\cdot296}{.976-.0277}&=.718[\si{\liter}]\\
      \end{split}
      \label{4}
    \end{equation}

  \item A refrigerant system uses $5.0[in^3]$ of \ce{He} compressed to 195[psi] at $20[\si{\Celsius}]$. What mass of \ce{He}, in grams, is needed for such a system? ($1[in]=2.54[\si{\centi\meter}]$)

    \begin{equation}
      \begin{split}
        195[psi]&=13.27[\si{\atm}]\\
        5[in^3]&=.082[\si{\liter}]\\
        n&=\frac{13.27\cdot.082}{.0821\cdot293}\\
        &=.045[\si{\mole}]\\
        4\cdot.045&=.18[\si{\gram}]
      \end{split}
      \label{5}
    \end{equation}

  \item A balloon is filled with $1[\si{\liter}]$ of helium at $1[\si{\atm}]$ and a starting temperature. The balloon rises to a point where the pressure is $220[\si{\torr}]$, temperature is $-31[\si{\Celsius}]$, and the volume of the balloon increases $2.8[\si{\liter}]$. What is the starting temperature of the balloon?

    \begin{equation}
      \begin{split}
        T&=\frac{PV}{nR}\\
        \frac{1\cdot1}{.0821}&=\frac{12.18}{n}[^{\circ}\si{\kelvin}]\\
        n&=\frac{PV}{RT}\\
        220[\si{\torr}]&=.289[\si{\atm}]\\
        \frac{2.8\cdot.289}{.0821\cdot242}&=.0407[\si{\mole}]\\
        \frac{12.18}{.0407}&=299.26[^{\circ}\si{\kelvin}]
      \end{split}
      \label{6}
    \end{equation}

  \item If $7.75[\si{\gram}]$ of \ce{SO2} is added, at constant temp and volume, to $5.87[\si{\liter}]$ of \ce{SO2} at $705[\si{\mmHg}]$ and $26.5[\si{\Celsius}]$, what will be the new gas pressure?

    \begin{equation}
      \begin{split}
        705[\si{\mmHg}]&=.928[\si{\atm}]\\
        \ce{SO2 &->} 64\left[ \frac{\si{\gram}}{\si{\mole}} \right]\\
        \frac{7.75}{64}&=.121[\si{\mole}]\\
        n&=\frac{5.87\cdot.928}{.0821\cdot299.5}\\
        &=.222[\si{\mole}]\\
        P&=\frac{(.222+.121)\cdot.0821\cdot299.5}{5.87}\\
        &=1.44[\si{\atm}]
      \end{split}
      \label{7}
    \end{equation}

\end{enumerate}

\end{document}

