%%%%%%%%%%%%%%%%%%%%%%%%%%%%%%%%%%%%%%%%%%%%%%%%%%%%%%%%%%%%%%%%%%%%%%%%%%%%%%%%%%%%%%%%%%%%%%%%%%%%%%%%%%%%%%%%%%%%%%%%%%%%%%%%%%%%%%%%%%%%%%%%%%%%%%%%%%%%%%%%%%%%%%%%%%%%%%%%%%%%%%%%%%%%
% Written By Michael Brodskiy
% Class: AP Chemistry
% Professor: J. Morgan
%%%%%%%%%%%%%%%%%%%%%%%%%%%%%%%%%%%%%%%%%%%%%%%%%%%%%%%%%%%%%%%%%%%%%%%%%%%%%%%%%%%%%%%%%%%%%%%%%%%%%%%%%%%%%%%%%%%%%%%%%%%%%%%%%%%%%%%%%%%%%%%%%%%%%%%%%%%%%%%%%%%%%%%%%%%%%%%%%%%%%%%%%%%%

\documentclass[12pt]{article} 
\usepackage{alphalph}
\usepackage[utf8]{inputenc}
\usepackage[russian,english]{babel}
\usepackage{titling}
\usepackage{amsmath}
\usepackage{graphicx}
\usepackage{enumitem}
\usepackage{amssymb}
\usepackage[super]{nth}
\usepackage{everysel}
\usepackage{ragged2e}
\usepackage{geometry}
\usepackage{fancyhdr}
\usepackage{cancel}
\usepackage{siunitx}
\geometry{top=1.0in,bottom=1.0in,left=1.0in,right=1.0in}
\newcommand{\subtitle}[1]{%
  \posttitle{%
    \par\end{center}
    \begin{center}\large#1\end{center}
    \vskip0.5em}%

}
\usepackage{hyperref}
\hypersetup{
colorlinks=true,
linkcolor=blue,
filecolor=magenta,      
urlcolor=blue,
citecolor=blue,
}

\urlstyle{same}


\title{Problem Set Chapter 3, Part 2}
\date{September 15, 2020}
\author{Michael Brodskiy\\ \small Instructor: Mr. Morgan}

% Mathematical Operations:

% Sum: $$\sum_{n=a}^{b} f(x) $$
% Integral: $$\int_{lower}^{upper} f(x) dx$$
% Limit: $$\lim_{x\to\infty} f(x)$$

\begin{document}

\maketitle

\begin{enumerate}

  \item Balance the following:

    \begin{enumerate}

      \item $Zn(C_2H_3O_2)_2+Na_3PO_4\rightarrow NaC_2H_3O_2 + Zn_3(PO_4)_2$
       $$3Zn(C_2H_3O_2)_2+2Na_3PO_4\rightarrow 6NaC_2H_3O_2 + Zn_3(PO_4)_2$$

     \item $Ca_{10}F_2(PO_4)_6+H_2SO_4\rightarrow HF+Ca(H_2PO_4)_2+CaSO_4$
      $$Ca_{10}F_2(PO_4)_6+7H_2SO_4\rightarrow 2HF+3Ca(H_2PO_4)_2+7CaSO_4$$

     \item $C_2H_6+O_2\rightarrow CO_2+H_2O$
      $$2C_2H_6+7O_2\rightarrow 4CO_2+6H_2O$$


    \end{enumerate}

  \item Calculate the number of grams of both products when $17.8[\si{\gram}]$ of $C_3H_8$ is combusted.

    \begin{enumerate}

      \item $C_3H_8+5O_2\rightarrow 4H_2O+3CO_2$ 
        $$\frac{17.8[\si{\gram}]}{44[\si{\gram\per\mole}]}=.4[\si{\mole}_{C_3H_8}]\rightarrow 2[\si{\mole}_{O_2}],\,1.6[\si{\mole}_{H_2O}],\,1.2[\si{\mole}_{CO_2}]$$
        $$1.6[\si{\mole}]\cdot18[\si{\gram\per\mole}]=28.8[\si{\gram}]$$
        $$1.2[\si{\mole}]\cdot44[\si{\gram\per\mole}]=52.8[\si{\gram}]$$

    \end{enumerate}

  \item A $0.1204[\si{\gram}]$ sample of carboxylic acid (containing C, O, and H) is burned in oxygen to yield $0.2147[\si{\gram}]$ of carbon dioxide and $0.0884[\si{\gram}]$ of water. Calculate the empirical formula.

    $$.2147\cdot\frac{12}{44}=.059[\si{\gram}_C],\,.0884\cdot\frac{2}{18}=.0098[\si{\gram}_H],\,.1204-.0098-.059=.0516[\si{\gram}_O]$$
    $$\frac{.059}{12}=.0049[\si{\mole}_C],\,\frac{.0098}{1}=.0098[\si{\mole}_H],\,\frac{.0516}{16}=.0032[\si{\mole}_O]$$

    $$C_3H_6O_2$$

  \item Phenol contains C, H, and O. Combustion of $2.136[\si{\milli\gram}]$ of phenol gives $5.993[\si{\milli\gram}]$ of $CO_2$ and $1.227[\si{\milli\gram}_{H_2O}]$. What is the simplest formula?

  \item Kerosene $(C_{14}H_{30})$ has a density of $0.763[\si{\gram\per\milli\liter}]$. How many grams of carbon dioxide are produced by the combustion of $3.785[\si{\liter}]$ of kerosene? 

  \item  How many liters of $CH_3CH_2OH$ (density = $0.789[\si{\gram\per\milli\liter}]$) must be consumed to produce $25[\si{\liter}]]$ of $CH_3CHO$ (density = $0.788[\si{\gram\per\milli\liter}]$)?
  $$CH_3CH_2OH+O_2\rightarrowCH_3CHO+H_2O$$

\end{enumerate}

\end{document}

