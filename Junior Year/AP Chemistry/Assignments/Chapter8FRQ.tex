%%%%%%%%%%%%%%%%%%%%%%%%%%%%%%%%%%%%%%%%%%%%%%%%%%%%%%%%%%%%%%%%%%%%%%%%%%%%%%%%%%%%%%%%%%%%%%%%%%%%%%%%%%%%%%%%%%%%%%%%%%%%%%%%%%%%%%%%%%%%%%%%%%%%%%%%%%%%%%%%%%%%%%%%%%%%%%%%%%%%%%%%%%%%
% Written By Michael Brodskiy
% Class: AP Chemistry
% Professor: J. Morgan
%%%%%%%%%%%%%%%%%%%%%%%%%%%%%%%%%%%%%%%%%%%%%%%%%%%%%%%%%%%%%%%%%%%%%%%%%%%%%%%%%%%%%%%%%%%%%%%%%%%%%%%%%%%%%%%%%%%%%%%%%%%%%%%%%%%%%%%%%%%%%%%%%%%%%%%%%%%%%%%%%%%%%%%%%%%%%%%%%%%%%%%%%%%%

\documentclass[12pt]{article} 
\usepackage{alphalph}
\usepackage[utf8]{inputenc}
\usepackage[russian,english]{babel}
\usepackage{titling}
\usepackage{amsmath}
\usepackage{graphicx}
\usepackage{enumitem}
\usepackage{amssymb}
\usepackage[super]{nth}
\usepackage{expl3}
\usepackage[version=4]{mhchem}
\usepackage{hpstatement}
\usepackage{rsphrase}
\usepackage{everysel}
\usepackage{ragged2e}
\usepackage{geometry}
\usepackage{fancyhdr}
\usepackage{cancel}
\usepackage{siunitx}
\usepackage{chemfig}
\usepackage{multicol}
\geometry{top=1.0in,bottom=1.0in,left=1.0in,right=1.0in}
\newcommand{\subtitle}[1]{%
  \posttitle{%
    \par\end{center}
    \begin{center}\large#1\end{center}
    \vskip0.5em}%

}
\newcommand{\orbital}[2]{{%
    \def\+{\big|\hspace{-2pt}\overline{\underline{\hspace{2pt}\upharpoonleft}}}%
    \def\-{\overline{\underline{\downharpoonright\hspace{2pt}}}\hspace{-2pt}\big|}%
    \def\0{\big|\hspace{-2pt}\overline{\underline{\phantom{\hspace{2pt}\downharpoonright}}}}%
    \def\1{\overline{\underline{\phantom{\downharpoonright\hspace{2pt}}}}\hspace{-2pt}\big|}%
  \setlength\tabcolsep{0pt}% remove extra horizontal space from tabular
  \begin{tabular}{c}$#2$\\[2pt]#1\end{tabular}%
}}
\DeclareSIUnit\Molar{\textsc{m}}
\DeclareSIUnit\atm{\textsc{atm}}
\DeclareSIUnit\torr{\textsc{torr}}
\DeclareSIUnit\psi{\textsc{psi}}
\DeclareSIUnit\bar{\textsc{bar}}
\DeclareSIUnit\Celsius{C}
\DeclareSIUnit\degree{$^{\circ}$}
\DeclareSIUnit\calorie{cal}
\usepackage{hyperref}
\hypersetup{
colorlinks=true,
linkcolor=blue,
filecolor=magenta,      
urlcolor=blue,
citecolor=blue,
}

\urlstyle{same}


\title{Chapter 8 $-$ Practice FRQ}
\date{January 19, 2020}
\author{Michael Brodskiy\\ \small Instructor: Mr. Morgan}

% Mathematical Operations:

% Sum: $$\sum_{n=a}^{b} f(x) $$
% Integral: $$\int_{lower}^{upper} f(x) dx$$
% Limit: $$\lim_{x\to\infty} f(x)$$

\begin{document}

\maketitle

\begin{enumerate}
  
  \item Aluminum metal can be recycled from scrap metal by melting the metal to evaporate impurities.

    \begin{enumerate}

      \item Calculate the amount of heat needed to purify $1[\si{\mole}]$ of \ce{Al} originally at $298[\si{\kelvin}]$ by melting it. The melting point of \ce{Al} is $933[\si{\kelvin}]$. The molar heat capacity of \ce{Al} is $24\left[ \frac{\si{\joule}}{\si{\mole\kelvin}} \right]$, and the heat of fusion of \ce{Al} is $10.7\left[ \frac{\si{\kilo\joule}}{\si{\mole}} \right]$

        \begin{equation}
          \begin{split}
            Q=cm\Delta T\\
            24\cdot1\cdot(933-298)=15240[\si{\joule}]\\
            15.42+10.7=25.94[\si{\kilo\joule}]
            \end{split}
          \label{1}
        \end{equation}

      \item The equation for the overall process of extracting \ce{Al} from \ce{Al2O3} is shown below. Which requires less energy, recycling existing \ce{Al} or extracting \ce{Al} from \ce{Al2O3}? Justify your answer with a calculation.

        \begin{center}
          \ce{Al2O3(s) -> 2Al(s) + 3/2O2(g)}\quad\quad$\Delta H^{\circ}=1675\left[ \frac{\si{\kilo\joule}}{\si{\mole}} \right]$
        \end{center}

        \begin{equation}
          \begin{split}
            .5\cdot1675=837.5[\si{\kilo\joule}]\\
            837.5>25.94\\
          \therefore\text{ It makes more sense to recycle}\\
          \end{split}
          \label{2}
        \end{equation}

    \end{enumerate}

\end{enumerate}

\end{document}

