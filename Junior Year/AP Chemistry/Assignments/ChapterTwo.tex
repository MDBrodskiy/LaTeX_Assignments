%%%%%%%%%%%%%%%%%%%%%%%%%%%%%%%%%%%%%%%%%%%%%%%%%%%%%%%%%%%%%%%%%%%%%%%%%%%%%%%%%%%%%%%%%%%%%%%%%%%%%%%%%%%%%%%%%%%%%%%%%%%%%%%%%%%%%%%%%%%%%%%%%%%%%%%%%%%%%%%%%%%%%%%%%%%%%%%%%%%%%%%%%%%%
% Written By Michael Brodskiy
% Class: AP Chemistry
% Professor: J. Morgan
%%%%%%%%%%%%%%%%%%%%%%%%%%%%%%%%%%%%%%%%%%%%%%%%%%%%%%%%%%%%%%%%%%%%%%%%%%%%%%%%%%%%%%%%%%%%%%%%%%%%%%%%%%%%%%%%%%%%%%%%%%%%%%%%%%%%%%%%%%%%%%%%%%%%%%%%%%%%%%%%%%%%%%%%%%%%%%%%%%%%%%%%%%%%

\documentclass[12pt]{article} 
\usepackage{alphalph}
\usepackage[utf8]{inputenc}
\usepackage[russian,english]{babel}
\usepackage{titling}
\usepackage{amsmath}
\usepackage{graphicx}
\usepackage{enumitem}
\usepackage{amssymb}
\usepackage[super]{nth}
\usepackage{everysel}
\usepackage{ragged2e}
\usepackage{geometry}
\usepackage{fancyhdr}
\usepackage{cancel}
\usepackage{siunitx}
\geometry{top=1.0in,bottom=1.0in,left=1.0in,right=1.0in}
\newcommand{\subtitle}[1]{%
  \posttitle{%
    \par\end{center}
    \begin{center}\large#1\end{center}
    \vskip0.5em}%

}
\usepackage{hyperref}
\hypersetup{
colorlinks=true,
linkcolor=blue,
filecolor=magenta,      
urlcolor=blue,
citecolor=blue,
}

\urlstyle{same}


\title{Chapter Two $-$ Problems: 14, 50, 52}
\date{August 25, 2020}
\author{Michael Brodskiy\\ \small Instructor: Mr. Morgan}

% Mathematical Operations:

% Sum: $$\sum_{n=a}^{b} f(x) $$
% Integral: $$\int_{lower}^{upper} f(x) dx$$
% Limit: $$\lim_{x\to\infty} f(x)$$

\begin{document}

\maketitle

\begin{enumerate}
    \setcounter{enumi}{13}

  \item How many neutrons, protons, and electrons does each have, and what element does each represent?

    \begin{enumerate}

      \item $^{75}_{33}A\Rightarrow 33p^+, 33e^-, 42n$ This element is arsenic

      \item $^{51}_{23}L\Rightarrow 23p^+, 23e^-, 28n$ This element is vanadium

      \item $^{131}_{54}Z\Rightarrow 54p^+, 54e^-, 77n$ This element is xenon

    \end{enumerate}

    \setcounter{enumi}{49}

  \item Give the number of protons and electrons in the following:

    \begin{enumerate}

    \item $S_8\Rightarrow 128p^+, 128e^-$

    \item $SO_4^{2-}\Rightarrow 48p^+, 50e^-$

    \item $H_2S \Rightarrow 18p^+, 18e^-$

    \item $S^{2-}\Rightarrow 16p^+, 18e^-$

    \end{enumerate}

    \setcounter{enumi}{51}

  \item Complete the table:


    \begin{center}

      \begin{tabular}{|c|c|c|c|c|}
\hline
Nuclear & Metal, Nonmetal & & & Number of \\
Symbol & Metalloid & Group & Period & Neutrons\\
\hline
Al-27 & Metal & 13 & 3 & 14 \\
\hline
Te-128 & Metalloid & 16 & 5 & 76 \\
\hline
Xe-134 & Nonmetal & 18 & 5 & 80 \\ 
\hline
C-12 & Nonmetal & 14 & 2 & 8 \\
\hline

      \end{tabular}

    \end{center}

\end{enumerate}

\end{document}

