%%%%%%%%%%%%%%%%%%%%%%%%%%%%%%%%%%%%%%%%%%%%%%%%%%%%%%%%%%%%%%%%%%%%%%%%%%%%%%%%%%%%%%%%%%%%%%%%%%%%%%%%%%%%%%%%%%%%%%%%%%%%%%%%%%%%%%%%%%%%%%%%%%%%%%%%%%%%%%%%%%%%%%%%%%%%%%%%%%%%%%%%%%%%
% Written By Michael Brodskiy
% Class: AP Chemistry
% Professor: J. Morgan
%%%%%%%%%%%%%%%%%%%%%%%%%%%%%%%%%%%%%%%%%%%%%%%%%%%%%%%%%%%%%%%%%%%%%%%%%%%%%%%%%%%%%%%%%%%%%%%%%%%%%%%%%%%%%%%%%%%%%%%%%%%%%%%%%%%%%%%%%%%%%%%%%%%%%%%%%%%%%%%%%%%%%%%%%%%%%%%%%%%%%%%%%%%%

\documentclass[12pt]{article} 
\usepackage{alphalph}
\usepackage[utf8]{inputenc}
\usepackage[russian,english]{babel}
\usepackage{titling}
\usepackage{amsmath}
\usepackage{graphicx}
\usepackage{enumitem}
\usepackage{amssymb}
\usepackage[super]{nth}
\usepackage{everysel}
\usepackage{ragged2e}
\usepackage{geometry}
\usepackage{fancyhdr}
\usepackage{cancel}
\usepackage{siunitx}
\geometry{top=1.0in,bottom=1.0in,left=1.0in,right=1.0in}
\newcommand{\subtitle}[1]{%
  \posttitle{%
    \par\end{center}
    \begin{center}\large#1\end{center}
    \vskip0.5em}%

}
\usepackage{hyperref}
\hypersetup{
colorlinks=true,
linkcolor=blue,
filecolor=magenta,      
urlcolor=blue,
citecolor=blue,
}

\urlstyle{same}


\title{Chapter One $-$ Problems: 37, 50, 54}
\date{August 24, 2020}
\author{Michael Brodskiy\\ \small Professor: Mr. Morgan}

% Mathematical Operations:

% Sum: $$\sum_{n=a}^{b} f(x) $$
% Integral: $$\int_{lower}^{upper} f(x) dx$$
% Limit: $$\lim_{x\to\infty} f(x)$$

\begin{document}

\maketitle

\begin{enumerate}
    \setcounter{enumi}{36}

\item 2 Acres $\rightarrow$ Hectares

  $$2[\cancel{ac}] \cdot \frac{4.356\cdot10^4[\cancel{ft^2}]}{1[\cancel{ac}]}\cdot\frac{.0929[\cancel{\si{\meter\squared}}]}{1[\cancel{ft^2}]}\cdot\frac{1[\si{\hectare}]}{10000[\cancel{\si{\meter\squared}}]}=.8093[\si{\hectare}]$$

  \setcounter{enumi}{49}

\item How long is a 10[$lb$] spool of 12-gauge (diameter of .0808[$in$]), with density 2.70 [$\si{\gram\per\centi\meter\cubed}$]

  $$10[lb]\rightarrow [\si{\kilo\gram}]=4.536[\si{\kilo\gram}]$$
  $$2.7[\si{\gram\per\centi\meter\cubed}]=2700[\si{\kilo\gram\per\meter\cubed}]$$
  $$.0404[in]=.001026[\si{\meter}]$$
  $$\pi r^2 l= \frac{4.536[\si{\kilo\gram}]}{1000[\si{\kilo\gram\per\meter\cubed}]}$$
  $$l=\frac{4.536[\si{\kilo\gram}]}{2700[\si{\kilo\gram\per\meter\cubed}]\cdot\pi\cdot(.001026[\si{\meter}])^2}$$
    $$l=508[\si{\meter}]$$


	\setcounter{enumi}{53}

  \item Potassium sulfate has a solubility of 15[$\si{\gram}$]/100[$\si{\gram}$] of water at 40[$\si{\celsius}$]. A solution is prepared by adding 39[$\si{\gram}$] to 225[$\si{\gram}$] of water. Is the solution unsaturated, saturated, or supersaturated? If precipitation occurs, how many grams would you expect to crystallize out?

    $$\frac{39}{225} > \frac{15}{100}$$
    \begin{center} This means that the solution is supersaturated. Using proportionality, 33.75[$\si{\gram}$] of potassium sulfate would be needed to achieve a saturated solution in 225[$\si{\gram}$] of water. This means that $39-33.75=5.25[\si{\gram}]$ of potassium sulfate would crystallize.\end{center}


\end{enumerate}

\end{document}

