%%%%%%%%%%%%%%%%%%%%%%%%%%%%%%%%%%%%%%%%%%%%%%%%%%%%%%%%%%%%%%%%%%%%%%%%%%%%%%%%%%%%%%%%%%%%%%%%%%%%%%%%%%%%%%%%%%%%%%%%%%%%%%%%%%%%%%%%%%%%%%%%%%%%%%%%%%%%%%%%%%%%%%%%%%%%%%%%%%%%%%%%%%%%
% Written By Michael Brodskiy
% Class: AP Chemistry
% Professor: J. Morgan
%%%%%%%%%%%%%%%%%%%%%%%%%%%%%%%%%%%%%%%%%%%%%%%%%%%%%%%%%%%%%%%%%%%%%%%%%%%%%%%%%%%%%%%%%%%%%%%%%%%%%%%%%%%%%%%%%%%%%%%%%%%%%%%%%%%%%%%%%%%%%%%%%%%%%%%%%%%%%%%%%%%%%%%%%%%%%%%%%%%%%%%%%%%%

\documentclass[12pt]{article} 
\usepackage{alphalph}
\usepackage[utf8]{inputenc}
\usepackage[russian,english]{babel}
\usepackage{titling}
\usepackage{amsmath}
\usepackage{graphicx}
\usepackage{enumitem}
\usepackage{amssymb}
\usepackage[super]{nth}
\usepackage{everysel}
\usepackage{ragged2e}
\usepackage{geometry}
\usepackage{fancyhdr}
\usepackage{cancel}
\usepackage{siunitx}
\geometry{top=1.0in,bottom=1.0in,left=1.0in,right=1.0in}
\newcommand{\subtitle}[1]{%
  \posttitle{%
    \par\end{center}
    \begin{center}\large#1\end{center}
    \vskip0.5em}%

}
\usepackage{hyperref}
\hypersetup{
colorlinks=true,
linkcolor=blue,
filecolor=magenta,      
urlcolor=blue,
citecolor=blue,
}

\urlstyle{same}


\title{Chapter 3 $-$ Problems 36, 48, \& 62}
\date{September 10, 2020}
\author{Michael Brodskiy\\ \small Instructor: Mr. Morgan}

% Mathematical Operations:

% Sum: $$\sum_{n=a}^{b} f(x) $$
% Integral: $$\int_{lower}^{upper} f(x) dx$$
% Limit: $$\lim_{x\to\infty} f(x)$$

\begin{document}

\maketitle

\begin{enumerate}
    \setcounter{enumi}{35}

  \item Nickel reacts with sulfur to form a sulfide. If $2.986[\si{\gram}]$ of nickel reacts with enough sulfur to form $5.433[\si{\gram}]$ of nickel sulfide, what is the simplest formula of the sulfide? Name the sulfide.

    $$\frac{2.986}{59}=.051[\si{\mole}]Ni\cdot{1}{.051}=1\cdot2=Ni_2$$
    $$5.433-2.986=2.447[\si{\gram}]S\rightarrow\frac{2.447}{32}=.076[\si{\mol}]\cdot{1}{.051}=1.5\cdot2=S_3$$
    $$Ni_2S_3\rightarrow\text{ Nickel (III) Sulfide }$$

    \setcounter{enumi}{47}

  \item Balance the following equations:

    \begin{enumerate}

      \item $C_6H_{12}O_6+O_2\rightarrow CO_2+H_2O$
       $$C_6H_{12}O_6+O_2\rightarrow CO_2+6H_2O$$
       $$C_6H_{12}O_6+6O_2\rightarrow 6CO_2+6H_2O$$

     \item $XeF_4+H_2O\rightarrow Xe+O_2+HF$
      $$XeF_4+2H_2O\rightarrow Xe+O_2+4HF$$

    \item $NaCl+H_2O+SO_2+O_2\rightarrow Na_2SO_4 + HCl$
     $$NaCl+H_2O+SO_2+O_2\rightarrow Na_2SO_4 + 4HCl$$
     $$4NaCl+H_2O+SO_2+O_2\rightarrow Na_2SO_4 + 4HCl$$
     $$4NaCl+2H_2O+2SO_2+O_2\rightarrow Na_2SO_4 + 4HCl$$
     $$4NaCl+2H_2O+2SO_2+O_2\rightarrow 2Na_2SO_4 + 4HCl$$

    \end{enumerate}

    \setcounter{enumi}{61}

  \item When corn is allowed to ferment, the fructose is converted to ethyl alcohol according to the following reaction:

    $$C_6H_12O_6\rightarrow 2C_2H_5OH+2CO_2$$

    \begin{enumerate}

      \item What volume of ethyl alcohol ($\rho=0.789[\si{\gram\per\milli\liter}]$) is produced from one pound of fructose?

        $$1[lb]=453.592[\si{\gram}]\rightarrow\frac{453.592}{180}=2.51[\si{\mole}]$$ 
        $$2\cdot2.51[\si{\mole}]=5.02[\si{\mole}]\text{ ethyl alcohol }\cdot46[\si{\gram\per\mole}]=230.9[\si{\gram}]\rightarrow\frac{230.9}{.789}=293[\si{\milli\liter}]$$

      \item Gasohol can be a mixture of $10[\si{\milli\liter}]$ ethyl alcohol and $90[\si{\milli\liter}]$ of gasoline. How many grams of fructose are required to produce the ethyl alcohol in one gallon of gasohol.

        $$1[gal]=3785[\si{\milli\liter}]\rightarrow378.5[\si{\milli\liter}]\text{ ethyl alcohol }\cdot.789=298.64[\si{\gram}]$$
        $$\frac{298.64}{46}=6.49[\si{\mole}]\cdot .5=3.25[\si{\mole}]\text{ fructose }\cdot 180=585[\si{\gram}]\text{ fructose }$$

    \end{enumerate}

\end{enumerate}

\end{document}

