%%%%%%%%%%%%%%%%%%%%%%%%%%%%%%%%%%%%%%%%%%%%%%%%%%%%%%%%%%%%%%%%%%%%%%%%%%%%%%%%%%%%%%%%%%%%%%%%%%%%%%%%%%%%%%%%%%%%%%%%%%%%%%%%%%%%%%%%%%%%%%%%%%%%%%%%%%%%%%%%%%%%%%%%%%%%%%%%%%%%%%%%%%%%
% Written By Michael Brodskiy
% Class: AP Chemistry
% Professor: J. Morgan
%%%%%%%%%%%%%%%%%%%%%%%%%%%%%%%%%%%%%%%%%%%%%%%%%%%%%%%%%%%%%%%%%%%%%%%%%%%%%%%%%%%%%%%%%%%%%%%%%%%%%%%%%%%%%%%%%%%%%%%%%%%%%%%%%%%%%%%%%%%%%%%%%%%%%%%%%%%%%%%%%%%%%%%%%%%%%%%%%%%%%%%%%%%%

\documentclass[12pt]{article} 
\usepackage{alphalph}
\usepackage[utf8]{inputenc}
\usepackage[russian,english]{babel}
\usepackage{titling}
\usepackage{amsmath}
\usepackage{graphicx}
\usepackage{enumitem}
\usepackage{amssymb}
\usepackage[super]{nth}
\usepackage{expl3}
\usepackage[version=4]{mhchem}
\usepackage{hpstatement}
\usepackage{rsphrase}
\usepackage{everysel}
\usepackage{ragged2e}
\usepackage{geometry}
\usepackage{fancyhdr}
\usepackage{cancel}
\usepackage{siunitx}
\usepackage{chemfig}
\usepackage{multicol}
\usepackage{xcolor}
\usepackage{array}
\usepackage{color, colortbl}
\definecolor{cadetgrey}{rgb}{0.57, 0.64, 0.69}
\geometry{top=1.0in,bottom=1.0in,left=1.0in,right=1.0in}
\newcommand{\subtitle}[1]{%
  \posttitle{%
    \par\end{center}
    \begin{center}\large#1\end{center}
    \vskip0.5em}%

}
\newcommand{\orbital}[2]{{%
    \def\+{\big|\hspace{-2pt}\overline{\underline{\hspace{2pt}\upharpoonleft}}}%
    \def\-{\overline{\underline{\downharpoonright\hspace{2pt}}}\hspace{-2pt}\big|}%
    \def\0{\big|\hspace{-2pt}\overline{\underline{\phantom{\hspace{2pt}\downharpoonright}}}}%
    \def\1{\overline{\underline{\phantom{\downharpoonright\hspace{2pt}}}}\hspace{-2pt}\big|}%
  \setlength\tabcolsep{0pt}% remove extra horizontal space from tabular
  \begin{tabular}{c}$#2$\\[2pt]#1\end{tabular}%
}}
\DeclareSIUnit\Molar{\textsc{M}}
\DeclareSIUnit\Molal{\textsc{m}}
\DeclareSIUnit\atm{\textsc{atm}}
\DeclareSIUnit\torr{\textsc{torr}}
\DeclareSIUnit\psi{\textsc{psi}}
\DeclareSIUnit\bar{\textsc{bar}}
\DeclareSIUnit\Celsius{C}
\DeclareSIUnit\degree{$^{\circ}$}
\DeclareSIUnit\calorie{cal}
\usepackage{hyperref}
\hypersetup{
colorlinks=true,
linkcolor=blue,
filecolor=magenta,      
urlcolor=blue,
citecolor=blue,
}

\urlstyle{same}


\title{Chapter 14 $-$ Problem Set 1}
\date{March 1, 2020}
\author{Michael Brodskiy\\ \small Instructor: Mr. Morgan}

% Mathematical Operations:

% Sum: $$\sum_{n=a}^{b} f(x) $$
% Integral: $$\int_{lower}^{upper} f(x) dx$$
% Limit: $$\lim_{x\to\infty} f(x)$$

\begin{document}

\maketitle

\begin{enumerate}

  \item

    \begin{enumerate}

      \item 

        \begin{equation}
          \begin{split}
            [\ce{H+}]&=4\left( 4.7\cdot10^{-11} \right)\\
            &=1.88\cdot10^{-10}\\
            -\log_{10}\left( 1.88\cdot10^{-10} \right)&=9.725
        \end{split}
          \label{1}
        \end{equation}

      \item

        \begin{equation}
          \begin{split}
            \frac{[\ce{HCO3-}]}{[\ce{H+}][\ce{CO3^2-}]}&=\frac{1}{4.7\cdot10^{-11}}\\
            \frac{[\ce{HCO3-}]}{[\ce{CO3^2-}]}&=\frac{10^{-10.83}}{4.7\cdot10^{-11}}\\
            &=.315
        \end{split}
          \label{2}
        \end{equation}

    \end{enumerate}

  \item \ce{H+(aq) + OH-(aq) -> H2O}

    \begin{enumerate}

      \item 

        \begin{equation}
          \begin{split}
            .02\cdot.5=.01[\si{\mole_{\ce{HCl}}}]\\
          .00745\cdot.5=.003725[\si{\mole}_{\ce{OH-}}]\\
            .01-.003725=.006275[\si{\mole}_{\ce{HCl}}]\\
            -\log_{10}\left( \frac{.006275}{.02745} \right)=.641
          \end{split}
          \label{3}
        \end{equation}

      \item

        \begin{equation}
          \begin{split}
            .02\cdot.5=.01[\si{\mole_{\ce{HCl}}}]\\
            .0185\cdot.5=.00925[\si{\mole}_{\ce{OH-}}]\\
            .01-.00925=.00075[\si{\mole}_{\ce{HCl}}]\\
            -\log_{10}\left( \frac{.00075}{.0385} \right)=1.71
          \end{split}
          \label{4}
        \end{equation}

      \item

        \begin{equation}
          \begin{split}
            .02\cdot.5=.01[\si{\mole_{\ce{HCl}}}]\\
            .02035-.02=.00035[\si{\milli\liter}_{\ce{OH-}}]\\
            .00035\cdot.5=.000175[\si{\mole}_{\ce{OH-}}]\\
            14+\log_{10}\left( \frac{.000175}{.04035} \right)=11.64
          \end{split}
          \label{5}
        \end{equation}

    \end{enumerate}

  \item 

    \begin{enumerate}

      \item 

        \begin{equation}
          \begin{split}
            -\log_{10}\left( \frac{4}{5}\cdot1.8\cdot10^{-5} \right)=4.84
          \end{split}
          \label{6}
        \end{equation}

      \item

        \begin{equation}
          \begin{split}
            4.84-\log_{10}\left( \frac{.04-.01}{.05+.01}  \right)=5.14
          \end{split}
          \label{7}
        \end{equation}

      \item

        \begin{equation}
          \begin{split}
            4.84-\log_{10}\left( \frac{.04+.01}{.05-.01}  \right)=4.74
          \end{split}
          \label{8}
        \end{equation}

    \end{enumerate}

  \item

    \begin{enumerate}

      \item 

        \begin{equation}
          \begin{split}
            -\log_{10}\left( \frac{.025(.1)-.0113(.2)}{.025+.0113} \right)=2.18
          \end{split}
          \label{9}
        \end{equation}

      \item

        \begin{equation}
          \begin{split}
            -\log_{10}\left( \frac{.025(.1)-.0125(.2)}{.025+.0125} \right)\text{ is undefined}\\
            \text{This means p\ce{H} } = 7\\
          \end{split}
          \label{10}
        \end{equation}

      \item

        \begin{equation}
          \begin{split}
            .0138(.2)-.025(.1)=.00026\left[ \si{\mole}_{\ce{OH-}} \right]\\
            14+\log_{10}\left( \frac{.00026}{.025+.0138} \right)= 11.82
          \end{split}
          \label{11}
        \end{equation}

    \end{enumerate}

  \item

    \begin{equation}
      \begin{split}
        .132\cdot500\cdot.943=62.238[\si{\gram}_{\ce{NH3}}]\\
        \frac{62.238}{17}=3.66[\si{\mole}_{\ce{NH3}}]\\
        \frac{x}{3.66}\cdot5.6\cdot10^{-10}=10^{-9.45}\\
        \frac{x}{3.66}=.634\\
        x=3.66\cdot.634\\
        =2.32[\si{\mole}_{\ce{NH4Cl}}]\\
        2.32\cdot53=123[\si{\gram}]\\
      \end{split}
      \label{12}
    \end{equation}

\end{enumerate}

\end{document}

