%%%%%%%%%%%%%%%%%%%%%%%%%%%%%%%%%%%%%%%%%%%%%%%%%%%%%%%%%%%%%%%%%%%%%%%%%%%%%%%%%%%%%%%%%%%%%%%%%%%%%%%%%%%%%%%%%%%%%%%%%%%%%%%%%%%%%%%%%%%%%%%%%%%%%%%%%%%%%%%%%%%%%%%%%%%%%%%%%%%%%%%%%%%%
% Written By Michael Brodskiy
% Class: AP Chemistry
% Professor: J. Morgan
%%%%%%%%%%%%%%%%%%%%%%%%%%%%%%%%%%%%%%%%%%%%%%%%%%%%%%%%%%%%%%%%%%%%%%%%%%%%%%%%%%%%%%%%%%%%%%%%%%%%%%%%%%%%%%%%%%%%%%%%%%%%%%%%%%%%%%%%%%%%%%%%%%%%%%%%%%%%%%%%%%%%%%%%%%%%%%%%%%%%%%%%%%%%

\documentclass[12pt]{article} 
\usepackage{alphalph}
\usepackage[utf8]{inputenc}
\usepackage[russian,english]{babel}
\usepackage{titling}
\usepackage{amsmath}
\usepackage{graphicx}
\usepackage{enumitem}
\usepackage{amssymb}
\usepackage[super]{nth}
\usepackage{expl3}
\usepackage[version=4]{mhchem}
\usepackage{hpstatement}
\usepackage{rsphrase}
\usepackage{everysel}
\usepackage{ragged2e}
\usepackage{geometry}
\usepackage{fancyhdr}
\usepackage{cancel}
\usepackage{siunitx}
\usepackage{chemfig}
\usepackage{multicol}
\geometry{top=1.0in,bottom=1.0in,left=1.0in,right=1.0in}
\newcommand{\subtitle}[1]{%
  \posttitle{%
    \par\end{center}
    \begin{center}\large#1\end{center}
    \vskip0.5em}%

}
\newcommand{\orbital}[2]{{%
    \def\+{\big|\hspace{-2pt}\overline{\underline{\hspace{2pt}\upharpoonleft}}}%
    \def\-{\overline{\underline{\downharpoonright\hspace{2pt}}}\hspace{-2pt}\big|}%
    \def\0{\big|\hspace{-2pt}\overline{\underline{\phantom{\hspace{2pt}\downharpoonright}}}}%
    \def\1{\overline{\underline{\phantom{\downharpoonright\hspace{2pt}}}}\hspace{-2pt}\big|}%
  \setlength\tabcolsep{0pt}% remove extra horizontal space from tabular
  \begin{tabular}{c}$#2$\\[2pt]#1\end{tabular}%
}}
\DeclareSIUnit\Molar{\textsc{m}}
\DeclareSIUnit\atm{\textsc{atm}}
\DeclareSIUnit\torr{\textsc{torr}}
\DeclareSIUnit\psi{\textsc{psi}}
\DeclareSIUnit\bar{\textsc{bar}}
\usepackage{hyperref}
\hypersetup{
colorlinks=true,
linkcolor=blue,
filecolor=magenta,      
urlcolor=blue,
citecolor=blue,
}

\urlstyle{same}


\title{Review Set $-$ Chapter 6 \& 7}
\date{December 10, 2020}
\author{Michael Brodskiy\\ \small Instructor: Mr. Morgan}

% Mathematical Operations:

% Sum: $$\sum_{n=a}^{b} f(x) $$
% Integral: $$\int_{lower}^{upper} f(x) dx$$
% Limit: $$\lim_{x\to\infty} f(x)$$

\begin{document}

\maketitle

\begin{enumerate}

  \item Write the complete electron configuration for the following atoms:

    \begin{enumerate}

      \item Boron $-$ \ce{1s^2 2s^2 2p^1}

      \item Silver $-$ \ce{1s^2 2s^2 2p^6 3s^2 3p^6 4s^2 3d^10 4p^6 5s^2 4d^9}

    \end{enumerate}

  \item Write the box diagram for the following:

    \begin{enumerate}

      \item Barium:

        \begin{center}
          \orbital{1s}{\+\-} \quad \orbital{2s}{\+\-} \quad \orbital{2p}{\+\- \+\- \+\-} \quad \orbital{3s}{\+\-} \quad \orbital{3p}{\+\- \+\- \+\-} \quad \orbital{4s}{\+\-} \quad \orbital{3d}{\+\- \+\- \+\- \+\- \+\-} \quad \orbital{4p}{\+\- \+\- \+\-} \quad \orbital{5s}{\+\-} \quad \orbital{4d}{\+\- \+\- \+\- \+\- \+\-} \quad \orbital{5p}{\+\- \+\- \+\-} \quad \orbital{6s}{\+\-}
        \end{center}

      \item Potassium:

        \begin{center}
          \orbital{1s}{\+\-} \quad \orbital{2s}{\+\-} \quad \orbital{2p}{\+\- \+\- \+\-} \quad \orbital{3s}{\+\-} \quad \orbital{3p}{\+\- \+\- \+\-} \quad \orbital{4s}{\+\1} 
        \end{center}

    \end{enumerate}

  \item Write the noble gas short-hand for the following atoms:

    \begin{enumerate}

      \item Calcium $-$ \ce{[Ar] 4s^2}

      \item Tin $-$ \ce{[Kr] 5s^2 4d^10 5p^2}

      \item Iodine $-$ \ce{[Kr] 5s^2 4d^10 5p^5}

      \item Bismuth $-$ \ce{[Xe] 6s^2 4f^14 5d^10 6p^3} 

    \end{enumerate}

  \item Write the four quantum numbers for the last electron in the following atom:

    \begin{enumerate}

      \item Vanadium $-$ $n=3,\,l=2,\,m_l=0,\,m_s=\frac{1}{2}$

      \item Calcium $-$ $n=4,\,l=0,\,m_l=0,\,m_s=-\frac{1}{2}$ 

      \item Uranium $-$ $n=6,\,l=2,\,m_l=-2,\,m_s=\frac{1}{2}$ 

      \item Copper $-$ $n=3,\,l=2,\,m_l=1,\,m_s=-\frac{1}{2}$ 

    \end{enumerate}

  \item State the number of valence electrons in the following: 

    \begin{enumerate}

      \item Antimony $-$ 5

      \item Neon $-$ 8

      \item Mercury $-$ 2 

      \item \ce{[Kr] 5s^2 4d^10 5p^2} $-$ 4

      \item Sodium $-$ 1

      \item \ce{[Xe] 6s^1} $-$ 1

    \end{enumerate}

  \item Draw the Lewis Structure for the following molecules. For those molecules that exhibit resonance, draw all possible resonance forms.

    \begin{enumerate}

      \item \ce{C2H4} 

        \begin{center}
          \chemfig{C(-[:-135]H)(-[:-225]H)=C(-[:-45]H)(-[:-315]H)}
        \end{center}

      \item \ce{HSO4^-}

        \begin{center}
          \chemfig{S(-[0]\lewis{2:6:,O}(-[:45]H))(=[2]\lewis{0:4:,O})(-[4]\lewis{2:4:6:,\ce{O-}})(=[6]\lewis{0:4:,O})}\\\vspace{15pt}
          \chemfig{S(-[2]\lewis{0:4:,O}(-[:135]H))(=[0]\lewis{2:6:,O})(-[6]\lewis{0:4:6:,\ce{O-}})(=[4]\lewis{2:6:,O})}\\\vspace{15pt}
          \chemfig{S(-[4]\lewis{2:6:,O}(-[:225]H))(=[2]\lewis{0:4:,O})(-[0]\lewis{0:2:6:,\ce{O-}})(=[6]\lewis{0:4:,O})}\\\vspace{15pt}
          \chemfig{S(-[6]\lewis{2:6:,O}(-[:315]H))(=[0]\lewis{0:4:,O})(-[2]\lewis{0:2:4:,\ce{O-}})(=[4]\lewis{0:4:,O})}\\\vspace{15pt}
        \end{center}

      \item \ce{N2O4} 

        \begin{center}
          \chemfig{N(-[:-135]\lewis{2:4:6:,O})(-[:-225]\lewis{2:4:6:,O})-N(-[:-45]\lewis{0:2:6:,O})(-[:-315]\lewis{0:2:6:,O})}\\\vspace{15pt}
        \end{center}

      \item \ce{NH2OH}

        \begin{center}
          \chemfig{\lewis{2:,N}(-[:-90]H)(-[:-210]H)(-[:-330]\lewis{2:6:,O}(-[:-45]H))}
        \end{center}

    \end{enumerate}

\item For the following molecules: a) predict the molecular structure; b) predict the bond angle; c) state the polarity of the molecule.

\begin{enumerate}

  \item \ce{N2O}

    \begin{enumerate}

      \item Linear

      \item $180^{\circ}$

      \item Non-polar

    \end{enumerate}

  \item \ce{BF3}

    \begin{enumerate}

      \item Triangular Planar

      \item $120^{\circ}$

      \item Non-polar

    \end{enumerate}

  \item \ce{Cl2O}

    \begin{enumerate}

      \item Bent

      \item $109.5^{\circ}$

      \item Polar

    \end{enumerate}

  \item \ce{NI3}

    \begin{enumerate}

      \item Tri-pyramid

      \item $109.5^{\circ}$

      \item Polar

    \end{enumerate}

\end{enumerate}

\item State what hybridization the center atom must use.

\begin{enumerate}

\item \ce{N2}

  \begin{center}
    \ce{sp^3}
  \end{center}

\item \ce{F2CO}

  \begin{center}
    \ce{sp^3}
  \end{center}

\item \ce{SeCl4}

  \begin{center}
    \ce{sp^3d}
  \end{center}

\item \ce{IF3}

  \begin{center}
    \ce{sp^3d}
  \end{center}

\item \ce{PCl5}

  \begin{center}
    \ce{sp^3d}
  \end{center}

\item \ce{SF6}

  \begin{center}
    \ce{sp^3d^2}
  \end{center}

  \end{enumerate}

\end{enumerate}

\end{document}

