%%%%%%%%%%%%%%%%%%%%%%%%%%%%%%%%%%%%%%%%%%%%%%%%%%%%%%%%%%%%%%%%%%%%%%%%%%%%%%%%%%%%%%%%%%%%%%%%%%%%%%%%%%%%%%%%%%%%%%%%%%%%%%%%%%%%%%%%%%%%%%%%%%%%%%%%%%%%%%%%%%%%%%%%%%%%%%%%%%%%%%%%%%%%
% Written By Michael Brodskiy
% Class: AP Chemistry
% Professor: J. Morgan
%%%%%%%%%%%%%%%%%%%%%%%%%%%%%%%%%%%%%%%%%%%%%%%%%%%%%%%%%%%%%%%%%%%%%%%%%%%%%%%%%%%%%%%%%%%%%%%%%%%%%%%%%%%%%%%%%%%%%%%%%%%%%%%%%%%%%%%%%%%%%%%%%%%%%%%%%%%%%%%%%%%%%%%%%%%%%%%%%%%%%%%%%%%%

\documentclass[12pt]{article} 
\usepackage{alphalph}
\usepackage[utf8]{inputenc}
\usepackage[russian,english]{babel}
\usepackage{titling}
\usepackage{amsmath}
\usepackage{graphicx}
\usepackage{enumitem}
\usepackage{amssymb}
\usepackage[super]{nth}
\usepackage{expl3}
\usepackage[version=4]{mhchem}
\usepackage{hpstatement}
\usepackage{rsphrase}
\usepackage{everysel}
\usepackage{ragged2e}
\usepackage{geometry}
\usepackage{fancyhdr}
\usepackage{cancel}
\usepackage{siunitx}
\usepackage{chemfig}
\usepackage{multicol}
\geometry{top=1.0in,bottom=1.0in,left=1.0in,right=1.0in}
\newcommand{\subtitle}[1]{%
  \posttitle{%
    \par\end{center}
    \begin{center}\large#1\end{center}
    \vskip0.5em}%

}
\newcommand{\orbital}[2]{{%
    \def\+{\big|\hspace{-2pt}\overline{\underline{\hspace{2pt}\upharpoonleft}}}%
    \def\-{\overline{\underline{\downharpoonright\hspace{2pt}}}\hspace{-2pt}\big|}%
    \def\0{\big|\hspace{-2pt}\overline{\underline{\phantom{\hspace{2pt}\downharpoonright}}}}%
    \def\1{\overline{\underline{\phantom{\downharpoonright\hspace{2pt}}}}\hspace{-2pt}\big|}%
  \setlength\tabcolsep{0pt}% remove extra horizontal space from tabular
  \begin{tabular}{c}$#2$\\[2pt]#1\end{tabular}%
}}
\DeclareSIUnit\Molar{\textsc{M}}
\DeclareSIUnit\Molal{\textsc{m}}
\DeclareSIUnit\atm{\textsc{atm}}
\DeclareSIUnit\torr{\textsc{torr}}
\DeclareSIUnit\psi{\textsc{psi}}
\DeclareSIUnit\bar{\textsc{bar}}
\DeclareSIUnit\Celsius{C}
\DeclareSIUnit\degree{$^{\circ}$}
\DeclareSIUnit\calorie{cal}
\usepackage{hyperref}
\hypersetup{
colorlinks=true,
linkcolor=blue,
filecolor=magenta,      
urlcolor=blue,
citecolor=blue,
}

\urlstyle{same}


\title{Review Chapter 16}
\date{May 14, 2020}
\author{Michael Brodskiy\\ \small Instructor: Mr. Morgan}

% Mathematical Operations:

% Sum: $$\sum_{n=a}^{b} f(x) $$
% Integral: $$\int_{lower}^{upper} f(x) dx$$
% Limit: $$\lim_{x\to\infty} f(x)$$

\begin{document}

\maketitle

\begin{enumerate}

  \item

    \begin{equation}
      \begin{split}
        \Delta G=\Delta H-T\Delta S\\
        \sum \Delta H= 3(-285.8)+824.2=-33.2[\si{\kilo\joule}]\\
        T\sum \Delta S= 308(3(.0699)+2(.0273)-.0874-3(.1306))=-66.19[\si{\kilo\joule}]\\
        \Delta H-T\Delta S=33[\si{\kilo\joule}]
      \end{split}
      \label{1}
    \end{equation}

  \item

    \begin{equation}
      \begin{split}
        T=\frac{\Delta H}{\Delta S}\\
        \Delta H= -635.1-393.5+1206.9=178.3[\si{\kilo\joule}]\\
        \Delta S= .0398+.2136-.0929=.1605\left[ \frac{\si{\kilo\joule}}{\si{\kelvin}} \right]\\
        \frac{178.3}{.1605}=1110[\si{\kelvin}]
      \end{split}
      \label{2}
    \end{equation}

  \item

    \begin{equation}
      \begin{split}
        10^{-3.12}=.00076[\si{\Molar}]\\
        K=\frac{[\ce{Fe^3+}]^4}{[\ce{H+}]^4[\ce{Fe^2+}]^4[\ce{O2}]}\\
        \frac{(.25)^4}{(.00076)^4(.25)^4(.755)}=4\cdot10^{12}\\
        -8.314\cdot298\cdot\ln(4\cdot10^{12})=-71.9[\si{\kilo\joule}]
      \end{split}
      \label{3}
    \end{equation}

  \item

    \begin{equation}
      \begin{split}
        1000=-8.314\cdot298\cdot\ln\left( \frac{(.4)^2}{(.4)^2\cdot x} \right)
        x=e^{\frac{1000}{8.314\cdot298}}=1.5[\si{\atm}]
      \end{split}
      \label{4}
    \end{equation}

  \item

    \begin{equation}
      \begin{split}
        \Delta G=-RT\ln(K)\\
        -8.314\cdot298\cdot\ln(1\cdot10^{-37})=91.7[\si{\kilo\joule}]
      \end{split}
      \label{5}
    \end{equation}

  \item

    \begin{equation}
      \begin{split}
        \Delta G =-RT\ln(K)\\
        -8.314\cdot296\cdot\ln(3.5\cdot10^{-6})=30.9[\si{\kilo\joule}]\\
      \end{split}
      \label{6}
    \end{equation}

  \item

    \begin{equation}
      \begin{split}
        \Delta G=-RT\ln(K)\\
        K=\frac{[\ce{H+}][\ce{F-}]}{[\ce{HF}]}\\
        \frac{[10^{-1.89}]^2}{.12}=.001383\\
        -8.314\cdot300\cdot\ln\left(  .001383\right)=16.4[\si{\kilo\joule}]
      \end{split}
      \label{7}
    \end{equation}

  \item

    \begin{enumerate}

      \item Entropy is negative in such a case because the entropy of the system is decreasing. The less moles of gas present, the less the disorder.

      \item Entropy is taken separate from temperature because it is so small, that it is treated independently

      \item A solid has lower entropy because they are extremely stable. The order of the molecules is much greater in a solid than a liquid.

    \end{enumerate}

\end{enumerate}

\end{document}

