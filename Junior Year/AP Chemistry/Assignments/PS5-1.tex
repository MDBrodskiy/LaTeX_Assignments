%%%%%%%%%%%%%%%%%%%%%%%%%%%%%%%%%%%%%%%%%%%%%%%%%%%%%%%%%%%%%%%%%%%%%%%%%%%%%%%%%%%%%%%%%%%%%%%%%%%%%%%%%%%%%%%%%%%%%%%%%%%%%%%%%%%%%%%%%%%%%%%%%%%%%%%%%%%%%%%%%%%%%%%%%%%%%%%%%%%%%%%%%%%%
% Written By Michael Brodskiy
% Class: AP Chemistry
% Professor: J. Morgan
%%%%%%%%%%%%%%%%%%%%%%%%%%%%%%%%%%%%%%%%%%%%%%%%%%%%%%%%%%%%%%%%%%%%%%%%%%%%%%%%%%%%%%%%%%%%%%%%%%%%%%%%%%%%%%%%%%%%%%%%%%%%%%%%%%%%%%%%%%%%%%%%%%%%%%%%%%%%%%%%%%%%%%%%%%%%%%%%%%%%%%%%%%%%

\documentclass[12pt]{article} 
\usepackage{alphalph}
\usepackage[utf8]{inputenc}
\usepackage[russian,english]{babel}
\usepackage{titling}
\usepackage{amsmath}
\usepackage{graphicx}
\usepackage{enumitem}
\usepackage{amssymb}
\usepackage[super]{nth}
\usepackage{expl3}
\usepackage[version=4]{mhchem}
\usepackage{hpstatement}
\usepackage{rsphrase}
\usepackage{everysel}
\usepackage{ragged2e}
\usepackage{geometry}
\usepackage{fancyhdr}
\usepackage{cancel}
\usepackage{siunitx}
\geometry{top=1.0in,bottom=1.0in,left=1.0in,right=1.0in}
\newcommand{\subtitle}[1]{%
  \posttitle{%
    \par\end{center}
    \begin{center}\large#1\end{center}
    \vskip0.5em}%

}
\DeclareSIUnit\Molar{\textsc{m}}
\DeclareSIUnit\atm{\textsc{atm}}
\DeclareSIUnit\Celsius{\textsc{C}}
\DeclareSIUnit\torr{\textsc{torr}}
\usepackage{hyperref}
\hypersetup{
colorlinks=true,
linkcolor=blue,
filecolor=magenta,      
urlcolor=blue,
citecolor=blue,
}

\urlstyle{same}


\title{Chapter 5 $-$ Problem Set 1}
\date{October 22, 2020}
\author{Michael Brodskiy\\ \small Instructor: Mr. Morgan}

% Mathematical Operations:

% Sum: $$\sum_{n=a}^{b} f(x) $$
% Integral: $$\int_{lower}^{upper} f(x) dx$$
% Limit: $$\lim_{x\to\infty} f(x)$$

\begin{document}

\maketitle

\begin{enumerate}

  \item How many grams of hydrogen is needed to fill a $80[\si{\liter}]$ tank to a pressure of $150[\si{\atm}]$ at $27[^{\circ}\si{\Celsius}]$? \eqref{1}

    \begin{equation}
      \begin{split}
        n&=\frac{PV}{RT}\\
        \frac{150\cdot80}{.0821\cdot300}&=487.211[\si{\mole}]\\
        2\cdot487&=974[\si{\gram}_{\ce{H}}]\\
      \end{split}
      \label{1}
    \end{equation}

  \item At what temperature does $16.3[\si{\gram}]$ of nitrogen have a pressure of $725[\si{\torr}]$ in a $25[\si{\liter}]$ tank? \eqref{2}

    \begin{equation}
      \begin{split}
      725[\si{\torr}]&=.954[\si{\atm}]\\
      \frac{16.3}{28}&=.582[\si{\mole}_{\ce{N}}]\\
      T&=\frac{25\cdot.954}{.582\cdot.0821}\\
      &=499[^{\circ}\si{\kelvin}]
    \end{split}
      \label{2}
    \end{equation}

  \item What is the volume, in $[\si{\milli\liter}]$, of $837[\si{\milli\gram}]$ of xenon gas at STP? \eqref{3}

      \begin{equation}
        \begin{split}
          837[\si{\milli\gram}]&=.837[\si{\gram}]\\
          V_{\si{\milli\liter}}&=1000\cdot\frac{.00639\cdot.0821\cdot273}{1}\\
          &=143.4\cdot[\si{\milli\liter}]
        \end{split}
        \label{3}
      \end{equation}

    \item A gas at STP is in a $25[\si{\liter}]$ container.  The volume is increased to $50[\si{\liter}]$ and pressure is increased to $1.5[\si{\atm}]$.  What is new temperature?  \eqref{4}

      \begin{equation}
        \begin{split}
          P\rightarrow1.5P,\,&V\rightarrow2V\\
          2V\cdot1.5P&=3T\\
          3\cdot273&=819[^{\circ}\si{\kelvin}]
        \end{split}
        \label{4}
      \end{equation}

    \item A balloon is filled with $1.0[\si{\liter}]$ of helium at $1.0[\si{\atm}]$ and a starting temp.  The balloon rises to a point where the pressure is $220[\si{\torr}]$, temp is $-31[^{\circ}\si{\Celsius}]$, and the volume increases to $2.8[\si{\liter}]$.  What is the starting temp of the balloon? \eqref{5}

      \begin{equation}
        \begin{split}
          T_1&=T_2\frac{P_1V_1}{P_2V_2}\\
          &=242\cdot\frac{1}{.289\cdot2.8}\\
          &=299[^{\circ}\si{\kelvin}]
        \end{split}
        \label{5}
      \end{equation}

    \item How many grams of gas must be released from a $45.2[\si{\liter}]$ sample of nitrogen at STP to reduce the volume to $45[\si{\liter}]$ at STP? \eqref{6}

      \begin{equation}
        \begin{split}
          n_1&=\frac{45.2}{.0821\cdot273}\\
          n_2&=\frac{45}{.0821\cdot273}\\
          n_1-n_2&=.0089[\si{\mole}]\\
          .0089\cdot28&=.25[\si{\gram}_{\ce{N}}]
        \end{split}
        \label{6}
      \end{equation}

    \item A neon sign is made of glass tubing whose inside diameter is $2.0[\si{\centi\meter}]$ and whose length is $4.0[\si{\centi\meter}]$.  If the sign contains neon at a pressure of $1.5[\si{\torr}]$ at $35[^{\circ}\si{\Celsius}]$, how many grams of neon are in the sign? ($V=\pi r^2 h$) \eqref{7}

      \begin{equation}
        \begin{split}
          V&=\pi(1)^24\\
          &=12.57[\si{\milli\liter}]\\
          &=.01257[\si{\liter}]\\
          1.5[\si{\torr}]&=.002[\si{\atm}]\\
          n&=\frac{.002\cdot.01257}{.0821\cdot308}\\
          &=9.94\cdot10^{-7}[\si{\mole}]\\
          20\cdot9.94\cdot10^{-7}&=1.99\cdot10^{-5}[\si{\gram}]
        \end{split}
        \label{7}
      \end{equation}

    \item Calculate the number of molecules in a deep breath of air whose volume is $2.55[\si{\liter}]$ at body temp of $37[^{\circ}\si{\Celsius}]$, and a pressure of $740[\si{\torr}]$. \eqref{8}

      \begin{equation}
        \begin{split}
          n&=\frac{.974\cdot2.55}{.0821\cdot310}\\
          &=.0976[\si{\mole}]\\
          6.022\cdot10^{23}\cdot.0976&=5.88\cdot10^{22}[\text{molecules}]
        \end{split}
        \label{8}
      \end{equation}

\end{enumerate}

\end{document}

