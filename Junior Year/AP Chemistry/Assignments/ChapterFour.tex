%%%%%%%%%%%%%%%%%%%%%%%%%%%%%%%%%%%%%%%%%%%%%%%%%%%%%%%%%%%%%%%%%%%%%%%%%%%%%%%%%%%%%%%%%%%%%%%%%%%%%%%%%%%%%%%%%%%%%%%%%%%%%%%%%%%%%%%%%%%%%%%%%%%%%%%%%%%%%%%%%%%%%%%%%%%%%%%%%%%%%%%%%%%%
% Written By Michael Brodskiy
% Class: AP Chemistry
% Professor: J. Morgan
%%%%%%%%%%%%%%%%%%%%%%%%%%%%%%%%%%%%%%%%%%%%%%%%%%%%%%%%%%%%%%%%%%%%%%%%%%%%%%%%%%%%%%%%%%%%%%%%%%%%%%%%%%%%%%%%%%%%%%%%%%%%%%%%%%%%%%%%%%%%%%%%%%%%%%%%%%%%%%%%%%%%%%%%%%%%%%%%%%%%%%%%%%%%

\documentclass[12pt]{article} 
\usepackage{alphalph}
\usepackage[utf8]{inputenc}
\usepackage[russian,english]{babel}
\usepackage{titling}
\usepackage{amsmath}
\usepackage{graphicx}
\usepackage{enumitem}
\usepackage{amssymb}
\usepackage[super]{nth}
\usepackage{expl3}
\usepackage[version=4]{mhchem}
\usepackage{hpstatement}
\usepackage{rsphrase}
\usepackage{everysel}
\usepackage{ragged2e}
\usepackage{geometry}
\usepackage{fancyhdr}
\usepackage{cancel}
\usepackage{siunitx}
\geometry{top=1.0in,bottom=1.0in,left=1.0in,right=1.0in}
\newcommand{\subtitle}[1]{%
  \posttitle{%
    \par\end{center}
    \begin{center}\large#1\end{center}
    \vskip0.5em}%

}
\DeclareSIUnit\Molar{\textsc{m}}
\usepackage{hyperref}
\hypersetup{
colorlinks=true,
linkcolor=blue,
filecolor=magenta,      
urlcolor=blue,
citecolor=blue,
}

\urlstyle{same}


\title{Chapter 4 $-$ Problems 50, 56}
\date{September 29, 2020}
\author{Michael Brodskiy\\ \small Instructor: Mr. Morgan}

% Mathematical Operations:

% Sum: $$\sum_{n=a}^{b} f(x) $$
% Integral: $$\int_{lower}^{upper} f(x) dx$$
% Limit: $$\lim_{x\to\infty} f(x)$$

\begin{document}

\maketitle

\begin{enumerate}

    \setcounter{enumi}{49}
  \item The Vanadium (\ce{V}) ion in a $.5000[\si{\gram}]$ sample of ore is converted to \ce{VO2+} ions. The amount of \ce{VO2+} in solution can be determined by reaction with an acid solution of \ce{KMnO4}. The balanced equation for the reaction is \eqref{1}. What is the mass percent of vanadium in the ore if $26.45[\si{\milli\liter}]$ of $.02250[\si{\Molar}]$ permanganate solution is required to complete the reaction? \eqref{2}

    \begin{equation}
      \ce{5VO^{2+} + MnO4^- + 11H2O -> Mn^{2+} + 5V(OH4)^+ + 2H^+}
      \label{1}
    \end{equation}

    \begin{equation}
      \begin{split}
        26.45[\si{\milli\liter}]&\rightarrow .02645[\si{\liter}] \\
          .0225\cdot.02645=.000595[\si{\mole}_{\ce{MnO4^-}}]&\rightarrow .002975[\si{\mole}_{\ce{VO^{2+}}}] \\
          .002975\cdot51&=.1517[\si{\gram}_{\ce{V}}]\\
          \%_{\ce{V}}&=\frac{.1517}{.5}\cdot100\%=30.3\%\\
      \end{split}
      \label{2}
    \end{equation}


    \setcounter{enumi}{55}
  \item Laws passed in some states define a drunk driver as one who drives with a blood alcohol level of .10\% by mass or higher. The level of alcohol can be determined by titrating blood plasma with potassium dichromate according to the equation: \eqref{3}. Assuming that the only substance that reacts with dichromate in blood plasma is alcohol, is a person legally drunk if $38.94[\si{\milli\liter}]$ of $.0723[\si{\Molar}]$ potassium dichromate is required to titrate a $50.0[\si{\gram}]$ sample of blood plasma? \eqref{4}

    \begin{equation}
      \ce{16H^+ + 2Cr2O7^{2-} + C2H5OH -> 4Cr^{3+} + 2CO2 + 11H2O}
      \label{3}
    \end{equation}

    \begin{equation}
      \begin{split}
        38.94[\si{\milli\liter}]&\rightarrow.03894[\si{\liter}]\\
        .0723\cdot.03894=.00281[\si{\mole}_{\ce{Cr2O7^{2-}}}]&\rightarrow.00141[\si{\mole}_{\ce{C2H5OH}}]\\
        .00141\cdot46&=.0649[\si{\gram}_{\ce{C2H5OH}}]\\
        \%_{\ce{C2H5OH}}=\frac{.0649}{50}\cdot100\%&=.129\%\\
        \text{Yes, the person} & \text{ is legally drunk}\\
      \end{split}
      \label{4}
    \end{equation}

\end{enumerate}

\end{document}

