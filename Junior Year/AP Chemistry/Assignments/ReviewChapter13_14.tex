%%%%%%%%%%%%%%%%%%%%%%%%%%%%%%%%%%%%%%%%%%%%%%%%%%%%%%%%%%%%%%%%%%%%%%%%%%%%%%%%%%%%%%%%%%%%%%%%%%%%%%%%%%%%%%%%%%%%%%%%%%%%%%%%%%%%%%%%%%%%%%%%%%%%%%%%%%%%%%%%%%%%%%%%%%%%%%%%%%%%%%%%%%%%
% Written By Michael Brodskiy
% Class: AP Chemistry
% Professor: J. Morgan
%%%%%%%%%%%%%%%%%%%%%%%%%%%%%%%%%%%%%%%%%%%%%%%%%%%%%%%%%%%%%%%%%%%%%%%%%%%%%%%%%%%%%%%%%%%%%%%%%%%%%%%%%%%%%%%%%%%%%%%%%%%%%%%%%%%%%%%%%%%%%%%%%%%%%%%%%%%%%%%%%%%%%%%%%%%%%%%%%%%%%%%%%%%%

\documentclass[12pt]{article} 
\usepackage{alphalph}
\usepackage[utf8]{inputenc}
\usepackage[russian,english]{babel}
\usepackage{titling}
\usepackage{amsmath}
\usepackage{graphicx}
\usepackage{enumitem}
\usepackage{amssymb}
\usepackage[super]{nth}
\usepackage{expl3}
\usepackage[version=4]{mhchem}
\usepackage{hpstatement}
\usepackage{rsphrase}
\usepackage{everysel}
\usepackage{ragged2e}
\usepackage{geometry}
\usepackage{fancyhdr}
\usepackage{cancel}
\usepackage{siunitx}
\usepackage{chemfig}
\usepackage{multicol}
\geometry{top=1.0in,bottom=1.0in,left=1.0in,right=1.0in}
\newcommand{\subtitle}[1]{%
  \posttitle{%
    \par\end{center}
    \begin{center}\large#1\end{center}
    \vskip0.5em}%

}
\newcommand{\orbital}[2]{{%
    \def\+{\big|\hspace{-2pt}\overline{\underline{\hspace{2pt}\upharpoonleft}}}%
    \def\-{\overline{\underline{\downharpoonright\hspace{2pt}}}\hspace{-2pt}\big|}%
    \def\0{\big|\hspace{-2pt}\overline{\underline{\phantom{\hspace{2pt}\downharpoonright}}}}%
    \def\1{\overline{\underline{\phantom{\downharpoonright\hspace{2pt}}}}\hspace{-2pt}\big|}%
  \setlength\tabcolsep{0pt}% remove extra horizontal space from tabular
  \begin{tabular}{c}$#2$\\[2pt]#1\end{tabular}%
}}
\DeclareSIUnit\Molar{\textsc{M}}
\DeclareSIUnit\Molal{\textsc{m}}
\DeclareSIUnit\atm{\textsc{atm}}
\DeclareSIUnit\torr{\textsc{torr}}
\DeclareSIUnit\psi{\textsc{psi}}
\DeclareSIUnit\bar{\textsc{bar}}
\DeclareSIUnit\Celsius{C}
\DeclareSIUnit\degree{$^{\circ}$}
\DeclareSIUnit\calorie{cal}
\usepackage{hyperref}
\hypersetup{
colorlinks=true,
linkcolor=blue,
filecolor=magenta,      
urlcolor=blue,
citecolor=blue,
}

\urlstyle{same}


\title{Chapter 13 \& 14 $-$ Review Set}
\date{March 16, 2020}
\author{Michael Brodskiy\\ \small Instructor: Mr. Morgan}

% Mathematical Operations:

% Sum: $$\sum_{n=a}^{b} f(x) $$
% Integral: $$\int_{lower}^{upper} f(x) dx$$
% Limit: $$\lim_{x\to\infty} f(x)$$

\begin{document}

\maketitle

\begin{enumerate}

  \item Calculate:

    \begin{enumerate}

      \item \ce{pH} when $\left[ \ce{OH-} \right]=2.3\cdot10^{-6}$

        \begin{equation}
          \begin{split}
            -\log_{10}\left( \frac{1\cdot10^{-14}}{2.3\cdot10^{-6}} \right)=8.36
          \end{split}
          \label{1}
        \end{equation}

      \item \ce{pOH} when $\left[ \ce{H3O+} \right]=2.8\cdot10^{-8}$

        \begin{equation}
          \begin{split}
            -\log_{10}\left( \frac{1\cdot10^{-14}}{2.8\cdot10^{-8}} \right)=6.45
          \end{split}
          \label{2}
        \end{equation}

      \item $[\ce{H3O+}]$ when \ce{pH} is $8.53$

        \begin{equation}
          \begin{split}
            10^{-8.53}=2.95\cdot10^{-9}\left[ \si{\Molar} \right]
          \end{split}
          \label{3}
        \end{equation}

      \item $[\ce{OH-}]$ when \ce{pH} is $2.36$

        \begin{equation}
          \begin{split}
            \frac{1\cdot10^{-14}}{10^{-2.36}}=2.29\cdot10^{-12}\left[ \si{\Molar} \right]
          \end{split}
          \label{4}
        \end{equation}

    \end{enumerate}

  \item Write the dissociation equation:

    \begin{enumerate}

      \item \ce{HBr -> H+ + Br-}

      \item \ce{F- + H2O -> HF + OH-}

      \item \ce{HC2H5O -> H+ + C2H5O-}

      \item \ce{HClO4 -> H+ + ClO4-}

      \item \ce{HNO2 -> H+ + NO2-}

      \item \ce{PO4^3- + H2O -> HPO4^2- + OH-}

    \end{enumerate}

  \item

    \begin{equation}
      \begin{split}
        \frac{x^2}{1.5}=.00014\\
        x=.0145[\si{\Molar}]
      \end{split}
      \label{5}
    \end{equation}

  \item

    \begin{equation}
      \begin{split}
        \frac{x^2}{.126}=1.5\cdot10^{-9}\\
        x=\sqrt{.126\cdot1.5\cdot10^{-9}}\\
        14+\log_{10}\left( x  \right)=9.14
      \end{split}
      \label{6}
    \end{equation}

  \item

    \begin{equation}
      \begin{split}
        [\ce{H+}]=10^{-9.8}=1.585\cdot10^{-10}\\
        [\ce{OH-}]=6.31\cdot10^{-5}\\
        \frac{\left( 6.31\cdot10^{-5} \right)^2}{.0278}=1.432\cdot10^{-7}\left[ \si{\Molar} \right]
      \end{split}
      \label{7}
    \end{equation}

  \item \ce{H2PO4- + OH- -> HPO4^2- + H2O}

    \begin{equation}
      \begin{split}
        \text{Lost: } .5\cdot.3-.137\cdot.42=.09246[\si{\mole}]\\
        \text{Gain: } .137\cdot.42+.05=.1075[\si{\mole}]\\
        [\ce{H+}]=\frac{\left(6.2\cdot10^{-8}\right)\left( \frac{.09246}{.3+.42} \right)}{\frac{.1075}{.3+42}}\\
          =5.33\cdot10^{-8}\\
          -\log_{10}\left( 5.33\cdot10^{-8} \right)=7.27
      \end{split}
      \label{8}
    \end{equation}

  \item

    \begin{enumerate}

      \item 

    \begin{equation}
      \begin{split}
        .006\cdot.532=.003192[\si{\mole}]\\
        .032\cdot.201=.006432-.003192=.00324[\si{\mole}]\\
        [\ce{H+}]&=\frac{(1.5\cdot10^{-5})(.00324)}{.003192}\\
        &= 1.52\cdot10^{-5}\\
        \ce{pH}=4.817
      \end{split}
      \label{9}
    \end{equation}

      \item 

    \begin{equation}
      \begin{split}
        V_b&= \frac{.032\cdot.201}{.532}\\
        &= .0121[\si{\liter}]\\
        .0121\cdot.532=.006432[\si{\mole}]\\
        \frac{.006432}{.0121+.032}=.14585[\si{\Molar}]\\
        \ce{Kb}=\frac{10^{-14}}{1.5\cdot10^{-5}}=6.67\cdot10^{-10}\\
        x=\sqrt{.14585\cdot6.67\cdot10^{-10}}=9.86\cdot10^{-6}[\si{\Molar}]\\
        \ce{pH}=14+\log_{10}\left( 9.86\cdot10^{-6} \right)=9
      \end{split}
      \label{10}
    \end{equation}

    \end{enumerate}

  \item

    \begin{enumerate}

      \item 

    \begin{equation}
      \begin{split}
        .094\cdot.1035=.009729[\si{\mole}]\\
        .044\cdot.332=.014608-.009729=.004879[\si{\mole}]\\
        [\ce{OH-}]&=\frac{(5.6\cdot10^{-10})(.004879)}{.009729}\\
        &= 2.808\cdot10^{-10}\\
        \ce{pH}=4.45
      \end{split}
      \label{11}
    \end{equation}

      \item 

    \begin{equation}
      \begin{split}
        V_b&= \frac{.044\cdot.332}{.1035}\\
        &= .141[\si{\liter}]\\
        .141\cdot.1035=.0146[\si{\mole}]\\
        \frac{.0146}{.141+.044}=.0789[\si{\Molar}]\\
        \ce{Ka}=\frac{10^{-14}}{5.6\cdot10^{-10}}=1.786\cdot10^{-5}\\
        x=\sqrt{.0789\cdot1.786\cdot10^{-5}}=.00119[\si{\Molar}]\\
        \ce{pH}=-\log_{10}\left( .00119 \right)=2.924
      \end{split}
      \label{12}
    \end{equation}

    \end{enumerate}

  \item

    \begin{enumerate}

      \item 

    \begin{equation}
      \begin{split}
        x&=\sqrt{.178\cdot1.9\cdot10^{-4}}\\
        &= .00582[\si{\Molar}]\\
        -\log_{10}\left( .00582 \right)=2.24\\
      \end{split}
      \label{13}
    \end{equation}

      \item 

    \begin{equation}
      \begin{split}
        .0048\cdot.523=.00251[\si{\mole}]\\
        .03\cdot.178=.00534-.00251=.00283[\si{\mole}]\\
        [\ce{H+}]=\frac{\left(1.9\cdot10^{-4}\right)\left( .00283 \right)}{.00251}\\
        \ce{pH}=3.67
      \end{split}
      \label{14}
    \end{equation}

    \end{enumerate}

  \item

    \begin{enumerate}

      \item 

    \begin{equation}
      \begin{split}
        x=\sqrt{.244\cdot7.14\cdot10^{-11}}\\
        =4.174\cdot10^{-6}\\
        14+\log_{10}\left( 4.174\cdot10^{-6}  \right)=8.621
      \end{split}
      \label{15}
    \end{equation}

      \item 

    \begin{equation}
      \begin{split}
        .078\cdot.1033=.00806[\si{\mole}]\\
        .05\cdot.244=.0122-.00806=.00414[\si{\mole}]\\
        [\ce{H+}]=\frac{\left( 7.14\cdot10^{-11} \right)\left( .00414 \right)}{.00806}\\
        \ce{pH}=3.56
      \end{split}
      \label{16}
    \end{equation}

      \item 

    \begin{equation}
      \begin{split}
        V_b=\frac{.05\cdot.244}{.1033}\\
        =.1181[\si{\liter}]\\
        .1181\cdot.1033=.0122[\si{\mole}]\\
        \frac{.0122}{.1181+.05}=.0726[\si{\Molar}]\\
        \ce{Ka}=\frac{10^{-14}}{7.14\cdot10^{-11}}\\
        =1.4\cdot10^{-4}\\
        x=\sqrt{.0726\cdot1.4\cdot10^{-4}}\\
        =.003189\\
        -\log_{10}\left( .003189 \right)=2.5
      \end{split}
      \label{17}
    \end{equation}

    \end{enumerate}

    \end{enumerate}

\end{enumerate}

\end{document}

