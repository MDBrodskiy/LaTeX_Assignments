%%%%%%%%%%%%%%%%%%%%%%%%%%%%%%%%%%%%%%%%%%%%%%%%%%%%%%%%%%%%%%%%%%%%%%%%%%%%%%%%%%%%%%%%%%%%%%%%%%%%%%%%%%%%%%%%%%%%%%%%%%%%%%%%%%%%%%%%%%%%%%%%%%%%%%%%%%%%%%%%%%%%%%%%%%%%%%%%%%%%%%%%%%%%
% Written By Michael Brodskiy
% Class: AP Chemistry
% Professor: J. Morgan
%%%%%%%%%%%%%%%%%%%%%%%%%%%%%%%%%%%%%%%%%%%%%%%%%%%%%%%%%%%%%%%%%%%%%%%%%%%%%%%%%%%%%%%%%%%%%%%%%%%%%%%%%%%%%%%%%%%%%%%%%%%%%%%%%%%%%%%%%%%%%%%%%%%%%%%%%%%%%%%%%%%%%%%%%%%%%%%%%%%%%%%%%%%%

\documentclass[12pt]{article} 
\usepackage{alphalph}
\usepackage[utf8]{inputenc}
\usepackage[russian,english]{babel}
\usepackage{titling}
\usepackage{amsmath}
\usepackage{graphicx}
\usepackage{enumitem}
\usepackage{amssymb}
\usepackage[super]{nth}
\usepackage{expl3}
\usepackage[version=4]{mhchem}
\usepackage{hpstatement}
\usepackage{rsphrase}
\usepackage{everysel}
\usepackage{ragged2e}
\usepackage{geometry}
\usepackage{fancyhdr}
\usepackage{cancel}
\usepackage{siunitx}
\usepackage{chemfig}
\usepackage{multicol}
\geometry{top=1.0in,bottom=1.0in,left=1.0in,right=1.0in}
\newcommand{\subtitle}[1]{%
  \posttitle{%
    \par\end{center}
    \begin{center}\large#1\end{center}
    \vskip0.5em}%

}
\newcommand{\orbital}[2]{{%
    \def\+{\big|\hspace{-2pt}\overline{\underline{\hspace{2pt}\upharpoonleft}}}%
    \def\-{\overline{\underline{\downharpoonright\hspace{2pt}}}\hspace{-2pt}\big|}%
    \def\0{\big|\hspace{-2pt}\overline{\underline{\phantom{\hspace{2pt}\downharpoonright}}}}%
    \def\1{\overline{\underline{\phantom{\downharpoonright\hspace{2pt}}}}\hspace{-2pt}\big|}%
  \setlength\tabcolsep{0pt}% remove extra horizontal space from tabular
  \begin{tabular}{c}$#2$\\[2pt]#1\end{tabular}%
}}
\DeclareSIUnit\Molar{\textsc{M}}
\DeclareSIUnit\Molal{\textsc{m}}
\DeclareSIUnit\atm{\textsc{atm}}
\DeclareSIUnit\torr{\textsc{torr}}
\DeclareSIUnit\psi{\textsc{psi}}
\DeclareSIUnit\bar{\textsc{bar}}
\DeclareSIUnit\Celsius{C}
\DeclareSIUnit\degree{$^{\circ}$}
\DeclareSIUnit\calorie{cal}
\usepackage{hyperref}
\hypersetup{
colorlinks=true,
linkcolor=blue,
filecolor=magenta,      
urlcolor=blue,
citecolor=blue,
}

\urlstyle{same}


\title{Chapter 13 \& 14 $-$ Review Set}
\date{March 16, 2020}
\author{Michael Brodskiy\\ \small Instructor: Mr. Morgan}

% Mathematical Operations:

% Sum: $$\sum_{n=a}^{b} f(x) $$
% Integral: $$\int_{lower}^{upper} f(x) dx$$
% Limit: $$\lim_{x\to\infty} f(x)$$

\begin{document}

\maketitle

\begin{enumerate}

  \item Calculate:

    \begin{enumerate}

      \item \ce{pH} when $\left[ \ce{OH-} \right]=2.3\cdot10^{-6}$

        \begin{equation}
          \begin{split}
            -\log_{10}\left( \frac{1\cdot10^{-14}}{2.3\cdot10^{-6}} \right)=8.36
          \end{split}
          \label{1}
        \end{equation}

      \item \ce{pOH} when $\left[ \ce{H3O+} \right]=2.8\cdot10^{-8}$

        \begin{equation}
          \begin{split}
            -\log_{10}\left( \frac{1\cdot10^{-14}}{2.8\cdot10^{-8}} \right)=6.45
          \end{split}
          \label{2}
        \end{equation}

      \item $[\ce{H3O+}]$ when \ce{pH} is $8.53$

        \begin{equation}
          \begin{split}
            10^{-8.53}=2.95\cdot10^{-9}\left[ \si{\Molar} \right]
          \end{split}
          \label{3}
        \end{equation}

      \item $[\ce{OH-}]$ when \ce{pH} is $2.36$

        \begin{equation}
          \begin{split}
            \frac{1\cdot10^{-14}}{10^{-2.36}}=2.29\cdot10^{-12}\left[ \si{\Molar} \right]
          \end{split}
          \label{4}
        \end{equation}

    \end{enumerate}

  \item Write the dissociation equation:

    \begin{enumerate}

      \item \ce{HBr -> H+ + Br-}

      \item \ce{F- + H2O -> HF + OH-}

      \item \ce{HC2H5O -> H+ + C2H5O-}

      \item \ce{HClO4 -> H+ + ClO4-}

      \item \ce{HNO2 -> H+ + NO2-}

      \item \ce{PO4^3- + H2O -> HPO4^2- + OH-}

    \end{enumerate}

  \item

    \begin{equation}
      \begin{split}
        \frac{x^2}{1.5}=.00014\\
        x=.0145[\si{\Molar}]
      \end{split}
      \label{5}
    \end{equation}

  \item

    \begin{equation}
      \begin{split}
        \frac{x^2}{.126}=1.5\cdot10^{-9}\\
        x=\sqrt{.126\cdot1.5\cdot10^{-9}}\\
        14+\log_{10}\left( x  \right)=9.14
      \end{split}
      \label{6}
    \end{equation}

  \item

    \begin{equation}
      \begin{split}
        [\ce{H+}]=10^{-9.8}=1.585\cdot10^{-10}\\
        [\ce{OH-}]=6.31\cdot10^{-5}\\
        \frac{\left( 6.31\cdot10^{-5} \right)^2}{.0278}=1.432\cdot10^{-7}\left[ \si{\Molar} \right]
      \end{split}
      \label{7}
    \end{equation}

  \item

    \begin{equation}
      \begin{split}
        \frac{.3\cdot.5+.42\cdot.137}{.72}=.28825[\si{\Molar}]\\
      \end{split}
      \label{8}
    \end{equation}

  \item

    \begin{enumerate}

      \item 

    \begin{equation}
      \begin{split}
      \end{split}
      \label{9}
    \end{equation}

      \item 

    \begin{equation}
      \begin{split}
      \end{split}
      \label{10}
    \end{equation}

    \end{enumerate}

  \item

    \begin{enumerate}

      \item 

    \begin{equation}
      \begin{split}
      \end{split}
      \label{11}
    \end{equation}

      \item 

    \begin{equation}
      \begin{split}
      \end{split}
      \label{12}
    \end{equation}

    \end{enumerate}

  \item

    \begin{equation}
      \begin{split}
      \end{split}
      \label{13}
    \end{equation}

  \item

    \begin{equation}
      \begin{split}
      \end{split}
      \label{14}
    \end{equation}

\end{enumerate}

\end{document}

