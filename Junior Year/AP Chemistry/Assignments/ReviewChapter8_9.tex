%%%%%%%%%%%%%%%%%%%%%%%%%%%%%%%%%%%%%%%%%%%%%%%%%%%%%%%%%%%%%%%%%%%%%%%%%%%%%%%%%%%%%%%%%%%%%%%%%%%%%%%%%%%%%%%%%%%%%%%%%%%%%%%%%%%%%%%%%%%%%%%%%%%%%%%%%%%%%%%%%%%%%%%%%%%%%%%%%%%%%%%%%%%%
% Written By Michael Brodskiy
% Class: AP Chemistry
% Professor: J. Morgan
%%%%%%%%%%%%%%%%%%%%%%%%%%%%%%%%%%%%%%%%%%%%%%%%%%%%%%%%%%%%%%%%%%%%%%%%%%%%%%%%%%%%%%%%%%%%%%%%%%%%%%%%%%%%%%%%%%%%%%%%%%%%%%%%%%%%%%%%%%%%%%%%%%%%%%%%%%%%%%%%%%%%%%%%%%%%%%%%%%%%%%%%%%%%

\documentclass[12pt]{article} 
\usepackage{alphalph}
\usepackage[utf8]{inputenc}
\usepackage[russian,english]{babel}
\usepackage{titling}
\usepackage{amsmath}
\usepackage{graphicx}
\usepackage{enumitem}
\usepackage{amssymb}
\usepackage[super]{nth}
\usepackage{expl3}
\usepackage[version=4]{mhchem}
\usepackage{hpstatement}
\usepackage{rsphrase}
\usepackage{everysel}
\usepackage{ragged2e}
\usepackage{geometry}
\usepackage{fancyhdr}
\usepackage{cancel}
\usepackage{siunitx}
\usepackage{chemfig}
\usepackage{multicol}
\geometry{top=1.0in,bottom=1.0in,left=1.0in,right=1.0in}
\newcommand{\subtitle}[1]{%
  \posttitle{%
    \par\end{center}
    \begin{center}\large#1\end{center}
    \vskip0.5em}%

}
\newcommand{\orbital}[2]{{%
    \def\+{\big|\hspace{-2pt}\overline{\underline{\hspace{2pt}\upharpoonleft}}}%
    \def\-{\overline{\underline{\downharpoonright\hspace{2pt}}}\hspace{-2pt}\big|}%
    \def\0{\big|\hspace{-2pt}\overline{\underline{\phantom{\hspace{2pt}\downharpoonright}}}}%
    \def\1{\overline{\underline{\phantom{\downharpoonright\hspace{2pt}}}}\hspace{-2pt}\big|}%
  \setlength\tabcolsep{0pt}% remove extra horizontal space from tabular
  \begin{tabular}{c}$#2$\\[2pt]#1\end{tabular}%
}}
\DeclareSIUnit\Molar{\textsc{m}}
\DeclareSIUnit\atm{\textsc{atm}}
\DeclareSIUnit\torr{\textsc{torr}}
\DeclareSIUnit\psi{\textsc{psi}}
\DeclareSIUnit\bar{\textsc{bar}}
\DeclareSIUnit\Celsius{C}
\DeclareSIUnit\degree{$^{\circ}$}
\DeclareSIUnit\calorie{cal}
\usepackage{hyperref}
\hypersetup{
colorlinks=true,
linkcolor=blue,
filecolor=magenta,      
urlcolor=blue,
citecolor=blue,
}

\urlstyle{same}


\title{Chapter 8 \& 9 $-$ Review Set}
\date{January 14, 2020}
\author{Michael Brodskiy\\ \small Instructor: Mr. Morgan}

% Mathematical Operations:

% Sum: $$\sum_{n=a}^{b} f(x) $$
% Integral: $$\int_{lower}^{upper} f(x) dx$$
% Limit: $$\lim_{x\to\infty} f(x)$$

\begin{document}

\maketitle

\begin{enumerate}

  \item The $\Delta H$ for the combustion of \ce{C6H12O6} is $�2820[\si{\kilo\joule}]$.  Determine $\Delta H$ if you start with 35.8 g of oxygen.  

    \begin{equation}
      \begin{split}
        \ce{C6H12O6 + 6O2 -> 6CO2 + 6H2O}\\
        \frac{35.8}{32}\cdot\frac{-2820}{6}=-525.81[\si{\kilo\joule}]\\
      \end{split}
      \label{1}
    \end{equation}

  \item The enthalpy change for the combustion of \ce{CH4} is $-891[\si{\kilo\joule}]$.  Calculate the enthalpy change if you end with $52[\si{\gram}]$ of water. 

    \begin{equation}
      \begin{split}
        \ce{CH4 + 2O2 -> CO2 + 2H2O}\\
        \frac{52}{18}=2.\bar{8}[\si{\mole}]\\
        2.\bar{8}\cdot\frac{-891}{2}=-1287[\si{\kilo\joule}]\\
      \end{split}
      \label{2}
    \end{equation}

  \item Calculate the heat of formation of \ce{AlCl3} when the enthalpy change of the reaction is $-2677[\si{\kilo\joule}]$, given \ce{3Al + 3NH4ClO4$(\Delta H=-295[\si{\kilo\joule}])$ -> Al2O3$(\Delta H=-1676[\si{\kilo\joule}])$ + AlCl3 + 3NO$(\Delta H=90[\si{\kilo\joule}])$ + 6H2O$(\Delta H=-242[\si{\kilo\joule}])$}

    \begin{equation}
      \begin{split}
        (-1676) + x + 3(90) + 6(-242) - 3(-295) = -2677[\si{\kilo\joule}]\\
        x=-704[\si{\kilo\joule}]\\
      \end{split}
      \label{3}
    \end{equation}

  \item What are the strongest attractive forces that must be overcome to:

    \begin{enumerate}

      \item Melt Ice $-$ Hydrogen Bonding

      \item Vaporize \ce{CaCl2} $-$ Ionic

      \item Melt \ce{KNO3} $-$ Ionic
        
      \item Dissolve \ce{Br2} in \ce{CCl4} $-$ London

      \item Sublime \ce{CO2} $-$ London

      \item Boil \ce{CH4} $-$ London

      \item Melt Iodine $-$ London

      \item Melt \ce{SiO2} $-$ Network Covalent

      \item Boil \ce{C2H5OH} $-$ Hydrogen Bonding

      \item Melt \ce{NH3} $-$ Hydrogen Bonding

    \end{enumerate}

  \item \ce{C8H18} has a vapor pressure of $45.2\[\si{\mmHg}]$ at $25[\si{\degree\Celsius}]$.  If $10[\si{\milli\liter}]$ ($\rho= 0.692\left[ \frac{\si{\gram}}{\si{\milli\liter}} \right]$) is added to a $15[\si{\liter}]$ container, how many molecules will be left in the liquid phase after equilibrium is established?

    \begin{equation}
      \begin{split}
        10\cdot.692=6.92[\si{\gram}]\\
        \frac{6.92}{130}=.0532[\si{\mole}]\\
        P=\frac{nRT}{V}=\frac{.0532\cdot.0821\cdot 298}{15}\\
        =.0868[\si{\atm}]\rightarrow65.947[\si{\mmHg}]\\
        65.947-45.2=20.747[\si{\mmHg}_{\text{liquid}}]\\
        \frac{20.747}{45.2}\cdot.0532=.0244[\si{\mole}]\\
        .0244\cdot6.22\cdot10^{23}=1.51\cdot10^{22}[\text{molecules}]\\
      \end{split}
      \label{4}
    \end{equation}

  \item Playing tennis for half an hour consumes $225[\si{\kilo\calorie}]$ of energy.  How long would you have to play tennis to lose one pound of body fat?  (one gram of body fat = $32[\si{\kilo\joule}]$ of energy)

    \begin{equation}
      \begin{split}
        225[\si{\kilo\calorie}]=941.4[\si{\kilo\joule}]\\
        2\cdot\frac{941.4}{32}=58.84\left[ \frac{\si{\gram}}{\si{\hour}} \right]\\
        58.84\left[ \frac{\si{\gram}}{\si{\hour}} \right]=.13\left[ \frac{\text{lb}}{\si{\hour}} \right]\\
        .13\cdot x=1\\
        x=7.7[\si{\hour}]\\
      \end{split}
      \label{5}
    \end{equation}

  \item Calculate the $\Delta H$ using bond energies when \ce{C2H4} is combusted.

    \begin{equation}
      \begin{split}
        \ce{C2H4 + 3O2 -> 2CO2 + 2H2O}\\
        \text{Broken: }\\
        3(\chemfig{O=O})=3(498)\\
        \chemfig{C=C}=612\\
        4(\chemfig{C-H})=4(414)\\
        \hline\\
        \text{Made: }\\
        4(\chemfig{C=O})=4(715)\\
        4(\chemfig{O-H})=4(464)\\
        \hline\\
        -4(715)-4(464)+3(498)+612+4(414)=-954[\si{\kilo\joule}]\\
      \end{split}
      \label{6}
    \end{equation}

\end{enumerate}

\end{document}

