%%%%%%%%%%%%%%%%%%%%%%%%%%%%%%%%%%%%%%%%%%%%%%%%%%%%%%%%%%%%%%%%%%%%%%%%%%%%%%%%%%%%%%%%%%%%%%%%%%%%%%%%%%%%%%%%%%%%%%%%%%%%%%%%%%%%%%%%%%%%%%%%%%%%%%%%%%%%%%%%%%%%%%%%%%%%%%%%%%%%%%%%%%%%
% Written By Michael Brodskiy
% Class: AP Chemistry
% Professor: J. Morgan
%%%%%%%%%%%%%%%%%%%%%%%%%%%%%%%%%%%%%%%%%%%%%%%%%%%%%%%%%%%%%%%%%%%%%%%%%%%%%%%%%%%%%%%%%%%%%%%%%%%%%%%%%%%%%%%%%%%%%%%%%%%%%%%%%%%%%%%%%%%%%%%%%%%%%%%%%%%%%%%%%%%%%%%%%%%%%%%%%%%%%%%%%%%%

\documentclass[12pt]{article} 
\usepackage{alphalph}
\usepackage[utf8]{inputenc}
\usepackage[russian,english]{babel}
\usepackage{titling}
\usepackage{amsmath}
\usepackage{graphicx}
\usepackage{enumitem}
\usepackage{amssymb}
\usepackage[super]{nth}
\usepackage{expl3}
\usepackage[version=4]{mhchem}
\usepackage{hpstatement}
\usepackage{rsphrase}
\usepackage{everysel}
\usepackage{ragged2e}
\usepackage{geometry}
\usepackage{fancyhdr}
\usepackage{cancel}
\usepackage{siunitx}
\usepackage{chemfig}
\usepackage{multicol}
\geometry{top=1.0in,bottom=1.0in,left=1.0in,right=1.0in}
\newcommand{\subtitle}[1]{%
  \posttitle{%
    \par\end{center}
    \begin{center}\large#1\end{center}
    \vskip0.5em}%

}
\newcommand{\orbital}[2]{{%
    \def\+{\big|\hspace{-2pt}\overline{\underline{\hspace{2pt}\upharpoonleft}}}%
    \def\-{\overline{\underline{\downharpoonright\hspace{2pt}}}\hspace{-2pt}\big|}%
    \def\0{\big|\hspace{-2pt}\overline{\underline{\phantom{\hspace{2pt}\downharpoonright}}}}%
    \def\1{\overline{\underline{\phantom{\downharpoonright\hspace{2pt}}}}\hspace{-2pt}\big|}%
  \setlength\tabcolsep{0pt}% remove extra horizontal space from tabular
  \begin{tabular}{c}$#2$\\[2pt]#1\end{tabular}%
}}
\DeclareSIUnit\Molar{\textsc{M}}
\DeclareSIUnit\Molal{\textsc{m}}
\DeclareSIUnit\atm{\textsc{atm}}
\DeclareSIUnit\torr{\textsc{torr}}
\DeclareSIUnit\psi{\textsc{psi}}
\DeclareSIUnit\bar{\textsc{bar}}
\DeclareSIUnit\Celsius{C}
\DeclareSIUnit\degree{$^{\circ}$}
\DeclareSIUnit\calorie{cal}
\usepackage{hyperref}
\hypersetup{
colorlinks=true,
linkcolor=blue,
filecolor=magenta,      
urlcolor=blue,
citecolor=blue,
}

\urlstyle{same}


\title{Chapter 11 $-$ Problem Set 1}
\date{January 28, 2020}
\author{Michael Brodskiy\\ \small Instructor: Mr. Morgan}

% Mathematical Operations:

% Sum: $$\sum_{n=a}^{b} f(x) $$
% Integral: $$\int_{lower}^{upper} f(x) dx$$
% Limit: $$\lim_{x\to\infty} f(x)$$

\begin{document}

\maketitle

\begin{enumerate}

  \item Given the following data for the reaction of \ce{NO} with \ce{H2}, calculate $k$

    \begin{center}
      \begin{tabular}[H]{c c c}
        [\ce{NO}] & [\ce{H2}] & Rate \\
        0.0064 & 0.0022 & 0.000026 \\
        0.0128 & 0.0022 & 0.0001 \\
        0.0064 & 0.0045 & 0.000051\\
      \end{tabular}
    \end{center}

    \begin{equation}
      \begin{split}
        \frac{.26}{1}=\left( \frac{64}{128}  \right)^m\\
        m=2\\
        \frac{.26}{.51}=\left( \frac{22}{45}  \right)^n\\
        n=1\\
        .0001=k[.0128]^2[.0022]\\
      k=277\left[ \frac{1}{\si{\Molar\squared\second}} \right]
      \end{split}
      \label{1}
    \end{equation}

  \item How long will it take a first order substance with $k = .27\left[ \frac{1}{\si{\second}} \right]$ to be reduced to $\frac{1}{3}$ the original concentration?

    \begin{equation}
      \begin{split}
        t_{1/3}=\frac{\ln(3)}{.27}\\
        =4.069[\si{\second}]\\
      \end{split}
      \label{2}
    \end{equation}

  \item How long will it take a first order substance with $k = .59\left[ \frac{1}{\si{\second}} \right]$ to be reduced to 25\% of original?

    \begin{equation}
      \begin{split}
        t_{1/4}=\frac{\ln(4)}{.59}\\
        =2.35[\si{\second}]
      \end{split}
      \label{3}
    \end{equation}

  \item The decomposition of \ce{NO2} is second order with a rate of $0.002\left[ \frac{\si{\mole}}{\si{\liter\second}} \right]$ when the concentration is $0.08[\si{\Molar}]$. Calculate the rate when the concentration of \ce{NO2} is $0.02[\si{\Molar}]$.

    \begin{equation}
      \begin{split}
        .002=k[.08]^2\\
        k=.3125\left[ \frac{\si{\liter}}{\si{\mole\second}} \right]\\
        rate=.3125\cdot[.02]^2\\
        =.000125=1.25\cdot10^{-4}
      \end{split}
      \label{4}
    \end{equation}

  \item For the first order reaction \ce{SO2Cl2 -> SO2 + Cl2} the rate constant is $0.000022\left[ \frac{1}{\si{\second}} \right]$. If you start with $0.0248[\si{\Molar}]$ of the reactant, what is the concentration of \ce{SO2Cl2} after $4.5[\si{\hour}]$?

    \begin{equation}
      \begin{split}
        1[\si{\hour}]\longrightarrow3600[\si{\second}]\\
        \ln\left( x_f \right)=\ln\left( .0248 \right)-.000022(4.5)(3600)\\
        x_f=e^{\ln(.0248)-.000022(4.5)(3600)}\\
      =.0174\left[ \si{\Molar} \right]
      \end{split}
      \label{5}
    \end{equation}

  \item The half-life of ethyl bromide at $720[\si{\degree\kelvin}]$ is $650\left[ \si{\second} \right]$ for the first order reaction. Find the time required for the concentration to drop from $0.05[\si{\Molar}]$ to $0.0125[\si{\Molar}]$.

    \begin{equation}
      \begin{split}
        \frac{.0125}{.05}=\frac{1}{4}\\
        650\cdot2=1300[\si{\second}]
      \end{split}
      \label{6}
    \end{equation}

  \item The first order reaction of diazomethane has a half-life of $17.3[\si{\minute}]$ at $873[\si{\degree\kelvin}]$. If the concentration is $0.058[\si{\Molar}]$ after $10[\si{\minute}]$, what is the original concentration?

    \begin{equation}
      \begin{split}
        17.3=\frac{\ln(2)}{k}\\
        k=\frac{\ln(2)}{17.3}\\
        k=.04\left[ \frac{1}{\si{\minute}} \right]\\
        \ln(x_o)=\ln(.058)+.04(10)\\
        x_o=e^{\ln(.058)+.4}\\
        x_o=.087[\si{\Molar}]
      \end{split}
      \label{7}
    \end{equation}

\end{enumerate}

\end{document}

