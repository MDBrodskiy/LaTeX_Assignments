%%%%%%%%%%%%%%%%%%%%%%%%%%%%%%%%%%%%%%%%%%%%%%%%%%%%%%%%%%%%%%%%%%%%%%%%%%%%%%%%%%%%%%%%%%%%%%%%%%%%%%%%%%%%%%%%%%%%%%%%%%%%%%%%%%%%%%%%%%%%%%%%%%%%%%%%%%%%%%%%%%%%%%%%%%%%%%%%%%%%%%%%%%%%
% Written By Michael Brodskiy
% Class: AP Chemistry
% Professor: J. Morgan
%%%%%%%%%%%%%%%%%%%%%%%%%%%%%%%%%%%%%%%%%%%%%%%%%%%%%%%%%%%%%%%%%%%%%%%%%%%%%%%%%%%%%%%%%%%%%%%%%%%%%%%%%%%%%%%%%%%%%%%%%%%%%%%%%%%%%%%%%%%%%%%%%%%%%%%%%%%%%%%%%%%%%%%%%%%%%%%%%%%%%%%%%%%%

\documentclass[12pt]{article} 
\usepackage{alphalph}
\usepackage[utf8]{inputenc}
\usepackage[russian,english]{babel}
\usepackage{titling}
\usepackage{amsmath}
\usepackage{graphicx}
\usepackage{enumitem}
\usepackage{amssymb}
\usepackage[super]{nth}
\usepackage{expl3}
\usepackage[version=4]{mhchem}
\usepackage{hpstatement}
\usepackage{rsphrase}
\usepackage{everysel}
\usepackage{ragged2e}
\usepackage{geometry}
\usepackage{fancyhdr}
\usepackage{cancel}
\usepackage{siunitx}
\usepackage{chemfig}
\usepackage{multicol}
\geometry{top=1.0in,bottom=1.0in,left=1.0in,right=1.0in}
\newcommand{\subtitle}[1]{%
  \posttitle{%
    \par\end{center}
    \begin{center}\large#1\end{center}
    \vskip0.5em}%

}
\newcommand{\orbital}[2]{{%
    \def\+{\big|\hspace{-2pt}\overline{\underline{\hspace{2pt}\upharpoonleft}}}%
    \def\-{\overline{\underline{\downharpoonright\hspace{2pt}}}\hspace{-2pt}\big|}%
    \def\0{\big|\hspace{-2pt}\overline{\underline{\phantom{\hspace{2pt}\downharpoonright}}}}%
    \def\1{\overline{\underline{\phantom{\downharpoonright\hspace{2pt}}}}\hspace{-2pt}\big|}%
  \setlength\tabcolsep{0pt}% remove extra horizontal space from tabular
  \begin{tabular}{c}$#2$\\[2pt]#1\end{tabular}%
}}
\DeclareSIUnit\Molar{\textsc{M}}
\DeclareSIUnit\Molal{\textsc{m}}
\DeclareSIUnit\atm{\textsc{atm}}
\DeclareSIUnit\torr{\textsc{torr}}
\DeclareSIUnit\psi{\textsc{psi}}
\DeclareSIUnit\bar{\textsc{bar}}
\DeclareSIUnit\Celsius{C}
\DeclareSIUnit\degree{$^{\circ}$}
\DeclareSIUnit\calorie{cal}
\usepackage{hyperref}
\hypersetup{
colorlinks=true,
linkcolor=blue,
filecolor=magenta,      
urlcolor=blue,
citecolor=blue,
}

\urlstyle{same}


\title{Chapter 17 $-$ Problem Set 2}
\date{April 22, 2020}
\author{Michael Brodskiy\\ \small Instructor: Mr. Morgan}

% Mathematical Operations:

% Sum: $$\sum_{n=a}^{b} f(x) $$
% Integral: $$\int_{lower}^{upper} f(x) dx$$
% Limit: $$\lim_{x\to\infty} f(x)$$

\begin{document}

\maketitle

\begin{enumerate}

  \item

    \begin{enumerate}

      \item \ce{K} is the better reducing agent

      \item \ce{Al} is the better reducing agent

      \item \ce{Co} is the better reducing agent

      \item \ce{Cr} is the better reducing agent

    \end{enumerate}

  \item

    \begin{enumerate}

      \item \ce{Cl2} is the better oxidizing agent

      \item \ce{Au^3+} is the better oxidizing agent

      \item \ce{F2} is the better oxidizing agent

      \item \ce{Ag+} is the better oxidizing agent

    \end{enumerate}

  \item 

    \begin{enumerate}

      \item \ce{MnO2(s) + 4H+(aq) + 2I-(aq) -> Mn^2+(aq) + 2H2O + I2(s)}

        \begin{equation}
          \begin{split}
            \,\,&1.229\\
            \,\,-&0.534\\
            \hline
            \,\,&0.695[\si{\volt}]
          \end{split}
          \label{1}
        \end{equation}

      \item \ce{I2(s) + 2Cl-(aq) -> 2I-(aq) + Cl2(g)}

        \begin{equation}
          \begin{split}
            \,\,&.534\\
            \,\,-&1.36\\
            \hline
            \,\,-&.826[\si{\volt}]
          \end{split}
          \label{2}
        \end{equation}

      \item \ce{Pt + NO(g) + 2H+(aq) + H2O2(aq) -> NO3-(aq) + 4H+(aq) + Pt}

        \begin{equation}
          \begin{split}
            \,\,&1.763\\
            \,\,&-.964\\
            \hline
            \,\,&0.799[\si{\volt}]
          \end{split}
          \label{3}
        \end{equation}


    \end{enumerate}

  \item

    \begin{enumerate}

      \item \ce{2Cl-(aq) + Br2(g) -> Cl2(g) + 2Br-(aq)} is not spontaneous:

        \begin{equation}
          \begin{split}
            \,\,&1.077\\
            \,\,-&1.36\\
            \hline
            \,\,-&.283[\si{\volt}]
          \end{split}
          \label{4}
        \end{equation}

      \item \ce{Sn^4+(aq) + Cu^2+(s) -> Sn^2+(aq) + Cu(s)} is not spontaneous:

        \begin{equation}
          \begin{split}
            \,\,&.154\\
            \,\,-&.339\\
            \hline
            \,\,-&.185[\si{\volt}]
          \end{split}
          \label{5}
        \end{equation}


    \end{enumerate}

  \item

    \begin{enumerate}

      \item $E^0=.799-(-.141)=.94[\si{\volt}]\Rightarrow -(2)\left( 9.648\cdot10^{4} \right)(.94)=-181.4[\si{\kilo\joule}]$

      \item $E^0=1.229-(.534)=.695[\si{\volt}]\Rightarrow -(2)\left( 9.648\cdot10^{4} \right)\left( .695 \right)=-134.1[\si{\kilo\joule}]$

    \end{enumerate}

  \item 

    \begin{enumerate}

      \item $E^0=-.547-(.004)=-.551[\si{\volt}]\Rightarrow \frac{-.551(2)}{.0257}=-42.88\Rightarrow e^{-42.88}=2.39\cdot10^{-19}$

      \item $E^0=1.001-.799=.202\Rightarrow\frac{.202\cdot3}{.0257}=23.58\Rightarrow e^{23.58}=1.74\cdot10^{10}$

    \end{enumerate}

\end{enumerate}

\end{document}

