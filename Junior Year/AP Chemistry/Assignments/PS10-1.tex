%%%%%%%%%%%%%%%%%%%%%%%%%%%%%%%%%%%%%%%%%%%%%%%%%%%%%%%%%%%%%%%%%%%%%%%%%%%%%%%%%%%%%%%%%%%%%%%%%%%%%%%%%%%%%%%%%%%%%%%%%%%%%%%%%%%%%%%%%%%%%%%%%%%%%%%%%%%%%%%%%%%%%%%%%%%%%%%%%%%%%%%%%%%%
% Written By Michael Brodskiy
% Class: AP Chemistry
% Professor: J. Morgan
%%%%%%%%%%%%%%%%%%%%%%%%%%%%%%%%%%%%%%%%%%%%%%%%%%%%%%%%%%%%%%%%%%%%%%%%%%%%%%%%%%%%%%%%%%%%%%%%%%%%%%%%%%%%%%%%%%%%%%%%%%%%%%%%%%%%%%%%%%%%%%%%%%%%%%%%%%%%%%%%%%%%%%%%%%%%%%%%%%%%%%%%%%%%

\documentclass[12pt]{article} 
\usepackage{alphalph}
\usepackage[utf8]{inputenc}
\usepackage[russian,english]{babel}
\usepackage{titling}
\usepackage{amsmath}
\usepackage{graphicx}
\usepackage{enumitem}
\usepackage{amssymb}
\usepackage[super]{nth}
\usepackage{expl3}
\usepackage[version=4]{mhchem}
\usepackage{hpstatement}
\usepackage{rsphrase}
\usepackage{everysel}
\usepackage{ragged2e}
\usepackage{geometry}
\usepackage{fancyhdr}
\usepackage{cancel}
\usepackage{siunitx}
\usepackage{chemfig}
\usepackage{multicol}
\geometry{top=1.0in,bottom=1.0in,left=1.0in,right=1.0in}
\newcommand{\subtitle}[1]{%
  \posttitle{%
    \par\end{center}
    \begin{center}\large#1\end{center}
    \vskip0.5em}%

}
\newcommand{\orbital}[2]{{%
    \def\+{\big|\hspace{-2pt}\overline{\underline{\hspace{2pt}\upharpoonleft}}}%
    \def\-{\overline{\underline{\downharpoonright\hspace{2pt}}}\hspace{-2pt}\big|}%
    \def\0{\big|\hspace{-2pt}\overline{\underline{\phantom{\hspace{2pt}\downharpoonright}}}}%
    \def\1{\overline{\underline{\phantom{\downharpoonright\hspace{2pt}}}}\hspace{-2pt}\big|}%
  \setlength\tabcolsep{0pt}% remove extra horizontal space from tabular
  \begin{tabular}{c}$#2$\\[2pt]#1\end{tabular}%
}}
\DeclareSIUnit\Molar{\textsc{M}}
\DeclareSIUnit\Molal{\textsc{m}}
\DeclareSIUnit\atm{\textsc{atm}}
\DeclareSIUnit\torr{\textsc{torr}}
\DeclareSIUnit\psi{\textsc{psi}}
\DeclareSIUnit\bar{\textsc{bar}}
\DeclareSIUnit\Celsius{C}
\DeclareSIUnit\degree{$^{\circ}$}
\DeclareSIUnit\calorie{cal}
\usepackage{hyperref}
\hypersetup{
colorlinks=true,
linkcolor=blue,
filecolor=magenta,      
urlcolor=blue,
citecolor=blue,
}

\urlstyle{same}


\title{Chapter 10 $-$ Problem Set}
\date{January 21, 2020}
\author{Michael Brodskiy\\ \small Instructor: Mr. Morgan}

% Mathematical Operations:

% Sum: $$\sum_{n=a}^{b} f(x) $$
% Integral: $$\int_{lower}^{upper} f(x) dx$$
% Limit: $$\lim_{x\to\infty} f(x)$$

\begin{document}

\maketitle

\begin{enumerate}

  \item A solution is made by dissolving $1.25[\si{\gram}]$ of \ce{C2H5OH} in $11.6[\si{\gram}]$ of water ($\rho=1.38\left[ \frac{\si{\gram}}{\si{\milli\liter}} \right]$).  Calculate the molality of the solution.

    \begin{equation}
      \begin{split}
        \frac{1.25}{46}=.0272[\si{\mole}]\\
        \frac{.0272}{.0116}=2.34[\si{\Molal}]\\
        \frac{12.85}{1.38}=9.31[\si{\milli\liter}]\\
        \frac{.0272}{.00931}=2.92[\si{\Molar}]\\
      \end{split}
      \label{1}
    \end{equation}

  \item A solution contains $50[\si{\gram}]$ of \ce{CS2} and $50[\si{\gram}]$ of \ce{CHCl3}.  Calculate the mole fraction of each.

    \begin{equation}
      \begin{split}
        \frac{50}{76}=.658[\si{\mole}]\\
        \frac{50}{119}=.42[\si{\mole}]\\
        \frac{.658}{.42+.658}=.39_{\ce{CHCl3}}\\
        1-.39=.61_{\ce{CS2}}\\
      \end{split}
      \label{2}
    \end{equation}

  \item The molality of a solution of \ce{C12H22O11} is $1.62[\si{\Molal}]$.  Calculate the mass percent.

    \begin{equation}
      \begin{split}
        \ce{C12H22O11} \rightarrow 342\left[ \frac{\si{\gram}}{\si{\mole}} \right]\\
        1.62\cdot342=554[\si{\gram}]\\
        \frac{554}{1000+554}=36\%
      \end{split}
      \label{3}
    \end{equation}

  \item The mole fraction of a solution of \ce{C2H5OH} is $0.0532$. Calculate the molality.

    \begin{equation}
      \begin{split}
        1[\si{\mole}_{total}]\rightarrow.0532[\si{\mole}_{\ce{C2H5OH}}]\\
        1-.0532=.9468[\si{\mole}_{\ce{H2O}}]\\
        .9468\cdot18=17.04[\si{\gram}]\\
        \frac{.0532}{.01704}=3.12[\si{\Molal}]
      \end{split}
      \label{4}
    \end{equation}

  \item Complete the following table for three different solutions of \ce{NaOH}:

    \begin{center}
    \begin{tabular}[H]{|c|c|c|c|c|}
      \hline
      & Density $\left( \frac{\si{\gram}}{\si{\milli\liter}} \right)$ & Molarity $\left( \si{\Molar} \right)$ & Molality $\left( \si{\Molal} \right)$ & Mass ($\%$) \\
      \hline
      Solution 1 & 1.05 & 1.32 & \textcolor{red}{1.32} & \textcolor{red}{5} \\
      \hline
      Solution 2 & 1.22 & \textcolor{red}{6.1} & \textcolor{red}{6.25} & 20.0\\
      \hline
      Solution 3 & 1.35 & \textcolor{red}{10.8} & 11.8 & \textcolor{red}{32} \\
      \hline
    \end{tabular}
    \end{center}

    \begin{equation}
      \begin{split}
        \text{Solution 1:}\\
        1.32[\si{\mole}_{\ce{NaOH}}]\rightarrow1000[\si{\milli\liter}]\\
        1000\cdot1.05=1050[\si{\gram}]\\
        1.32\cdot40=52.8[\si{\gram}]\\
        1050-52.8=997.2[\si{\gram}]\\
        \frac{1.32}{.9972}=1.32[\si{\Molal}]\\
        \frac{52.8}{1050}=5\%\\
      \end{split}
      \label{5}
    \end{equation}
    \begin{equation}
      \begin{split}
        \text{Solution 2:}\\
        1220[\si{\gram}]\rightarrow1000[\si{\milli\liter}]\\
        1220\cdot.2=244[\si{\gram}_{\ce{NaOH}}]\\
        \frac{244}{40}=6.1[\si{\mole}]\\
        \frac{6.1}{1}=6.1[\si{\Molar}]\\
        \frac{6.1}{.976}=6.25[\si{\Molal}]\\
      \end{split}
      \label{6}
    \end{equation}
    \begin{equation}
      \begin{split}
        \text{Solution 3:}\\
        11.8[\si{\mole}]\rightarrow1000[\si{\gram}]\\
        40\cdot11.8=472[\si{\gram}]\\
        1000+472=1472[\si{\gram}]\\
        \frac{1472}{1.35}=1090[\si{\milli\liter}]\\
        \frac{11.8}{1.09}=10.8[\si{\Molar}]\\
        \frac{472}{1472}\cdot100\%=32\%\\
      \end{split}
      \label{7}
    \end{equation}
  
\end{enumerate}

\end{document}

