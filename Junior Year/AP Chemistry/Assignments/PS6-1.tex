%%%%%%%%%%%%%%%%%%%%%%%%%%%%%%%%%%%%%%%%%%%%%%%%%%%%%%%%%%%%%%%%%%%%%%%%%%%%%%%%%%%%%%%%%%%%%%%%%%%%%%%%%%%%%%%%%%%%%%%%%%%%%%%%%%%%%%%%%%%%%%%%%%%%%%%%%%%%%%%%%%%%%%%%%%%%%%%%%%%%%%%%%%%%
% Written By Michael Brodskiy
% Class: AP Chemistry
% Professor: J. Morgan
%%%%%%%%%%%%%%%%%%%%%%%%%%%%%%%%%%%%%%%%%%%%%%%%%%%%%%%%%%%%%%%%%%%%%%%%%%%%%%%%%%%%%%%%%%%%%%%%%%%%%%%%%%%%%%%%%%%%%%%%%%%%%%%%%%%%%%%%%%%%%%%%%%%%%%%%%%%%%%%%%%%%%%%%%%%%%%%%%%%%%%%%%%%%

\documentclass[12pt]{article} 
\usepackage{alphalph}
\usepackage[utf8]{inputenc}
\usepackage[russian,english]{babel}
\usepackage{titling}
\usepackage{amsmath}
\usepackage{physics}
\usepackage{tikz}
\usepackage{mathdots}
\usepackage{yhmath}
\usepackage{cancel}
\usepackage{color}
\usepackage{siunitx}
\usepackage{array}
\usepackage{multirow}
\usepackage{amssymb}
\usepackage{gensymb}
\usepackage{tabularx}
\usepackage{booktabs}
\usetikzlibrary{fadings}
\usetikzlibrary{patterns}
\usetikzlibrary{shadows.blur}
\usetikzlibrary{shapes}
\usepackage{graphicx}
\usepackage{enumitem}
\usepackage[super]{nth}
\usepackage{expl3}
\usepackage[version=4]{mhchem}
\usepackage{hpstatement}
\usepackage{rsphrase}
\usepackage{everysel}
\usepackage{ragged2e}
\usepackage{geometry}
\usepackage{fancyhdr}
\usepackage{cancel}
\usepackage{siunitx} 
\geometry{top=1.0in,bottom=1.0in,left=1.0in,right=1.0in}
\newcommand{\subtitle}[1]{%
  \posttitle{%
    \par\end{center}
    \begin{center}\large#1\end{center}
    \vskip0.5em}%

}
\DeclareSIUnit\Molar{\textsc{m}}
\DeclareSIUnit\atm{\textsc{atm}}
\DeclareSIUnit\Celsius{^{\circ}\textsc{C}}
\DeclareSIUnit\torr{\textsc{torr}}
\DeclareSIUnit\mmHg{\textsc{mmHg}}
\usepackage{hyperref}
\hypersetup{
colorlinks=true,
linkcolor=blue,
filecolor=magenta,      
urlcolor=blue,
citecolor=blue,
}

\urlstyle{same}


\title{Chapter 6 $-$ Problem Set 1}
\date{November 12, 2020}
\author{Michael Brodskiy\\ \small Instructor: Mr. Morgan}

% Mathematical Operations:

% Sum: $$\sum_{n=a}^{b} f(x) $$
% Integral: $$\int_{lower}^{upper} f(x) dx$$
% Limit: $$\lim_{x\to\infty} f(x)$$

\begin{document}

\maketitle

\begin{enumerate}

  \item Write the electron configuration for the following:

    \begin{enumerate}

      \item Nickel

        \begin{center}
          \ce{1s^2 2s^2 2p^6 3s^2 3p^6 4s^2 3d^8}
        \end{center}

      \item Tungsten

      \begin{center}
        \ce{1s^2 2s^2 2p^6 3s^2 3p^6 4s^2 3d^10 4p^6 5s^2 4d^10 4f^14 5p^6 5d^4 6s^2}
      \end{center}

      \item Oxygen

      \item Lead

    \end{enumerate}

  \item Write the box diagram for the following:

    \begin{enumerate}

      \item Fluorine

      \item Vanadium

      \item Bismuth

      \item Silver

    \end{enumerate}

  \item Give the four quantum numbers for the second to last electron in the following:

    \begin{enumerate}

      \item Calcium

      \item Iodine

      \item Tin

      \item Carbon

      \item Radon

      \item Gallium

    \end{enumerate}

  \item State how many electrons are in the following:

    \begin{enumerate}

      \item f orbital

      \item d sublevel

      \item All sublevels where $n=3$

      \item All sublevels where $n=5$

      \item $l=2$

      \item $l=0$

    \end{enumerate}

  \item State the number of unpaired electrons in:

    \begin{enumerate}

      \item Iron

      \item Arsenic

      \item Tin

      \item Silver

    \end{enumerate}

  \item State what atom's electron configuration ends with the following:

    \begin{enumerate}

      \item \ce{3d^3}

      \item \ce{4p^2}
        
      \item \ce{4f^7}

      \item \ce{5s^1}

      \item \ce{6s^2}

      \item \ce{5d^8}

    \end{enumerate}

\end{enumerate}

\end{document}

