%%%%%%%%%%%%%%%%%%%%%%%%%%%%%%%%%%%%%%%%%%%%%%%%%%%%%%%%%%%%%%%%%%%%%%%%%%%%%%%%%%%%%%%%%%%%%%%%%%%%%%%%%%%%%%%%%%%%%%%%%%%%%%%%%%%%%%%%%%%%%%%%%%%%%%%%%%%%%%%%%%%%%%%%%%%%%%%%%%%%%%%%%%%%
% Written By Michael Brodskiy
% Class: AP Chemistry
% Professor: J. Morgan
%%%%%%%%%%%%%%%%%%%%%%%%%%%%%%%%%%%%%%%%%%%%%%%%%%%%%%%%%%%%%%%%%%%%%%%%%%%%%%%%%%%%%%%%%%%%%%%%%%%%%%%%%%%%%%%%%%%%%%%%%%%%%%%%%%%%%%%%%%%%%%%%%%%%%%%%%%%%%%%%%%%%%%%%%%%%%%%%%%%%%%%%%%%%

\documentclass[12pt]{article} 
\usepackage{alphalph}
\usepackage[utf8]{inputenc}
\usepackage[russian,english]{babel}
\usepackage{titling}
\usepackage{amsmath}
\usepackage{graphicx}
\usepackage{enumitem}
\usepackage{amssymb}
\usepackage[super]{nth}
\usepackage{expl3}
\usepackage[version=4]{mhchem}
\usepackage{hpstatement}
\usepackage{rsphrase}
\usepackage{everysel}
\usepackage{ragged2e}
\usepackage{geometry}
\usepackage{fancyhdr}
\usepackage{cancel}
\usepackage{siunitx}
\usepackage{chemfig}
\usepackage{multicol}
\geometry{top=1.0in,bottom=1.0in,left=1.0in,right=1.0in}
\newcommand{\subtitle}[1]{%
  \posttitle{%
    \par\end{center}
    \begin{center}\large#1\end{center}
    \vskip0.5em}%

}
\newcommand{\orbital}[2]{{%
    \def\+{\big|\hspace{-2pt}\overline{\underline{\hspace{2pt}\upharpoonleft}}}%
    \def\-{\overline{\underline{\downharpoonright\hspace{2pt}}}\hspace{-2pt}\big|}%
    \def\0{\big|\hspace{-2pt}\overline{\underline{\phantom{\hspace{2pt}\downharpoonright}}}}%
    \def\1{\overline{\underline{\phantom{\downharpoonright\hspace{2pt}}}}\hspace{-2pt}\big|}%
  \setlength\tabcolsep{0pt}% remove extra horizontal space from tabular
  \begin{tabular}{c}$#2$\\[2pt]#1\end{tabular}%
}}
\DeclareSIUnit\Molar{\textsc{M}}
\DeclareSIUnit\Molal{\textsc{m}}
\DeclareSIUnit\atm{\textsc{atm}}
\DeclareSIUnit\torr{\textsc{torr}}
\DeclareSIUnit\psi{\textsc{psi}}
\DeclareSIUnit\bar{\textsc{bar}}
\DeclareSIUnit\Celsius{C}
\DeclareSIUnit\degree{$^{\circ}$}
\DeclareSIUnit\calorie{cal}
\usepackage{hyperref}
\hypersetup{
colorlinks=true,
linkcolor=blue,
filecolor=magenta,      
urlcolor=blue,
citecolor=blue,
}

\urlstyle{same}


\title{Chapter 17 $-$ Problem Set 3}
\date{April 23, 2020}
\author{Michael Brodskiy\\ \small Instructor: Mr. Morgan}

% Mathematical Operations:

% Sum: $$\sum_{n=a}^{b} f(x) $$
% Integral: $$\int_{lower}^{upper} f(x) dx$$
% Limit: $$\lim_{x\to\infty} f(x)$$

\begin{document}

\maketitle

\begin{enumerate}

  \item

    \begin{enumerate}

      \item 

        \begin{equation}
          \begin{split}
            E^0=.141+.799=.94[\si{\volt}]\\
            -(2)\left( 9.648\cdot10^{4} \right)\left( .94 \right)=-181.4[\si{\kilo\joule}]
          \end{split}
          \label{1}
        \end{equation}

      \item 

        \begin{equation}
          \begin{split}
            E^0=1.229-.534=.695[\si{\volt}]\\
            -(2)\left( 9.648\cdot10^{4} \right)\left( .695 \right)=-134.1[\si{\kilo\joule}]
          \end{split}
          \label{2}
        \end{equation}

    \end{enumerate}

  \item

    \begin{enumerate}

      \item 

        \begin{equation}
          \begin{split}
            E^0=-.547-.004=-.551[\si{\volt}]\\
            \ln(K)=\frac{(-.551)(2)}{.0257}=-42.88\\
            e^{-42.88}=2.39\cdot10^{-19}
          \end{split}
          \label{3}
        \end{equation}

      \item \ce{2Cr^2+(aq) + Sn^4+(aq) -> 2Cr^3+(aq) + Sn^2+(aq)}

        \begin{equation}
          \begin{split}
            E^0=.408+.154=.562[\si{\volt}]\\
            \ln(K)=\frac{(.267)(2)}{.0257}=43.74\\
            e^{43.74}=9.86\cdot10^{18}
          \end{split}
          \label{4}
        \end{equation}

    \end{enumerate}

  \item

    \begin{equation}
      \begin{split}
        E^0=-1.33+1.36=.03[\si{\volt}]\\
        Q=[\ce{H+}]^{14}=(.126)^{14}=2.51\cdot10^{-13}\\
        E=.03-\frac{.0257}{6}\ln(2.51\cdot10^{-13})\\
        =.154[\si{\volt}]
      \end{split}
      \label{5}
    \end{equation}

  \item

    \begin{equation}
      \begin{split}
        \ce{Ag}=108\left[ \frac{\si{\gram}}{\si{\mole}} \right]\\
        \left( \frac{1}{10.5\cdot6.25} \right)\left( 108  \right)\left( \frac{1}{9.648\cdot10^4} \right)\left( 2 \right)(8700)=.297[\si{\centi\meter}]
      \end{split}
      \label{6}
    \end{equation}

\end{enumerate}

\end{document}

