%%%%%%%%%%%%%%%%%%%%%%%%%%%%%%%%%%%%%%%%%%%%%%%%%%%%%%%%%%%%%%%%%%%%%%%%%%%%%%%%%%%%%%%%%%%%%%%%%%%%%%%%%%%%%%%%%%%%%%%%%%%%%%%%%%%%%%%%%%%%%%%%%%%%%%%%%%%%%%%%%%%%%%%%%%%%%%%%%%%%%%%%%%%%
% Written By Michael Brodskiy
% Class: AP Chemistry
% Professor: J. Morgan
%%%%%%%%%%%%%%%%%%%%%%%%%%%%%%%%%%%%%%%%%%%%%%%%%%%%%%%%%%%%%%%%%%%%%%%%%%%%%%%%%%%%%%%%%%%%%%%%%%%%%%%%%%%%%%%%%%%%%%%%%%%%%%%%%%%%%%%%%%%%%%%%%%%%%%%%%%%%%%%%%%%%%%%%%%%%%%%%%%%%%%%%%%%%

\documentclass[12pt]{article} 
\usepackage{alphalph}
\usepackage[utf8]{inputenc}
\usepackage[russian,english]{babel}
\usepackage{titling}
\usepackage{amsmath}
\usepackage{graphicx}
\usepackage{enumitem}
\usepackage{amssymb}
\usepackage[super]{nth}
\usepackage{expl3}
\usepackage[version=4]{mhchem}
\usepackage{hpstatement}
\usepackage{rsphrase}
\usepackage{everysel}
\usepackage{ragged2e}
\usepackage{geometry}
\usepackage{fancyhdr}
\usepackage{cancel}
\usepackage{siunitx}
\usepackage{chemfig}
\usepackage{multicol}
\geometry{top=1.0in,bottom=1.0in,left=1.0in,right=1.0in}
\newcommand{\subtitle}[1]{%
  \posttitle{%
    \par\end{center}
    \begin{center}\large#1\end{center}
    \vskip0.5em}%

}
\DeclareSIUnit\Molar{\textsc{m}}
\DeclareSIUnit\atm{\textsc{atm}}
\DeclareSIUnit\torr{\textsc{torr}}
\DeclareSIUnit\psi{\textsc{psi}}
\DeclareSIUnit\bar{\textsc{bar}}
\usepackage{hyperref}
\hypersetup{
colorlinks=true,
linkcolor=blue,
filecolor=magenta,      
urlcolor=blue,
citecolor=blue,
}

\urlstyle{same}


\title{Chapter 7 $-$ Problems 56, 58, 64}
\date{December 3, 2020}
\author{Michael Brodskiy\\ \small Instructor: Mr. Morgan}

% Mathematical Operations:

% Sum: $$\sum_{n=a}^{b} f(x) $$
% Integral: $$\int_{lower}^{upper} f(x) dx$$
% Limit: $$\lim_{x\to\infty} f(x)$$

\begin{document}

\maketitle

\begin{enumerate}

    \setcounter{enumi}{55}

  \item In each of the following polyatomic ions, the central atom has an expanded octet. Determine the number of electron pairs around the central atom and the hybridization in:

    \begin{enumerate}

      \item \ce{ClF4-}

        \begin{center}
          Electron Pairs: 6\\
          Hybridization: \ce{sp^3d^2}
        \end{center}

      \item \ce{GeCl6^2-}

        \begin{center}
          Electron Pairs: 6\\
          Hybridization: \ce{sp^3d^2}
        \end{center}

      \item \ce{SbCl4-}

        \begin{center}
          Electron Pairs: 5\\
          Hybridization: \ce{sp^3d}
        \end{center}

    \end{enumerate}

    \setcounter{enumi}{57}

  \item Acrylonitrile, \ce{C3H3N}, is the building block of the polymer Orlon. What is the hybridization of nitrogen and of the three numbered carbon atoms?

    \begin{enumerate}

      \item \ce{N} $-$ \ce{sp}

      \item \ce{C} (1) $-$ \ce{sp^2}

      \item \ce{C} (2) $-$ \ce{sp^2}

      \item \ce{C} (3) $-$ \ce{sp}

    \end{enumerate}

    \setcounter{enumi}{63}

  \item Give the number of sigma and pi bonds in the molecule in Question 58

    \begin{center}
      $\sigma\longrightarrow6$\\
      $\pi\longrightarrow3$
    \end{center}

\end{enumerate}

\end{document}

