%%%%%%%%%%%%%%%%%%%%%%%%%%%%%%%%%%%%%%%%%%%%%%%%%%%%%%%%%%%%%%%%%%%%%%%%%%%%%%%%%%%%%%%%%%%%%%%%%%%%%%%%%%%%%%%%%%%%%%%%%%%%%%%%%%%%%%%%%%%%%%%%%%%%%%%%%%%%%%%%%%%%%%%%%%%%%%%%%%%%%%%%%%%%
% Written By Michael Brodskiy
% Class: AP Chemistry
% Professor: J. Morgan
%%%%%%%%%%%%%%%%%%%%%%%%%%%%%%%%%%%%%%%%%%%%%%%%%%%%%%%%%%%%%%%%%%%%%%%%%%%%%%%%%%%%%%%%%%%%%%%%%%%%%%%%%%%%%%%%%%%%%%%%%%%%%%%%%%%%%%%%%%%%%%%%%%%%%%%%%%%%%%%%%%%%%%%%%%%%%%%%%%%%%%%%%%%%

\documentclass[12pt]{article} 
\usepackage{alphalph}
\usepackage[utf8]{inputenc}
\usepackage[russian,english]{babel}
\usepackage{titling}
\usepackage{amsmath}
\usepackage{graphicx}
\usepackage{enumitem}
\usepackage{amssymb}
\usepackage[super]{nth}
\usepackage{expl3}
\usepackage[version=4]{mhchem}
\usepackage{hpstatement}
\usepackage{rsphrase}
\usepackage{everysel}
\usepackage{ragged2e}
\usepackage{geometry}
\usepackage{fancyhdr}
\usepackage{cancel}
\usepackage{siunitx}
\usepackage{chemfig}
\usepackage{multicol}
\geometry{top=1.0in,bottom=1.0in,left=1.0in,right=1.0in}
\newcommand{\subtitle}[1]{%
  \posttitle{%
    \par\end{center}
    \begin{center}\large#1\end{center}
    \vskip0.5em}%

}
\newcommand{\orbital}[2]{{%
    \def\+{\big|\hspace{-2pt}\overline{\underline{\hspace{2pt}\upharpoonleft}}}%
    \def\-{\overline{\underline{\downharpoonright\hspace{2pt}}}\hspace{-2pt}\big|}%
    \def\0{\big|\hspace{-2pt}\overline{\underline{\phantom{\hspace{2pt}\downharpoonright}}}}%
    \def\1{\overline{\underline{\phantom{\downharpoonright\hspace{2pt}}}}\hspace{-2pt}\big|}%
  \setlength\tabcolsep{0pt}% remove extra horizontal space from tabular
  \begin{tabular}{c}$#2$\\[2pt]#1\end{tabular}%
}}
\DeclareSIUnit\Molar{\textsc{m}}
\DeclareSIUnit\atm{\textsc{atm}}
\DeclareSIUnit\torr{\textsc{torr}}
\DeclareSIUnit\psi{\textsc{psi}}
\DeclareSIUnit\bar{\textsc{bar}}
\DeclareSIUnit\Celsius{C}
\DeclareSIUnit\degree{$^{\circ}$}
\usepackage{hyperref}
\hypersetup{
colorlinks=true,
linkcolor=blue,
filecolor=magenta,      
urlcolor=blue,
citecolor=blue,
}

\urlstyle{same}


\title{Chapter 8 $-$ Problem 36}
\date{January 12, 2020}
\author{Michael Brodskiy\\ \small Instructor: Mr. Morgan}

% Mathematical Operations:

% Sum: $$\sum_{n=a}^{b} f(x) $$
% Integral: $$\int_{lower}^{upper} f(x) dx$$
% Limit: $$\lim_{x\to\infty} f(x)$$

\begin{document}

\maketitle

\begin{enumerate}

    \setcounter{enumi}{35}

  \item Given the following thermochemical equations, calculate $\Delta H^{\circ}$ for the decomposition of \ce{B2H6} into its elements:

    \begin{center}
      \ce{4B(s) + 3O2(g) -> 2B2O3(s)} \hspace{30pt} $\Delta H^{\circ}=-2543.8[\si{\kilo\joule}]$\\
      \ce{H2(g) + 1/2O2(g) -> H2O(g)} \hspace{30pt} $\Delta H^{\circ}=-241.8[\si{\kilo\joule}]$\\
      \ce{B2H6(s) + 3O2 -> B2O3(s) + 3H2O(g)} \hspace{25pt}  $\Delta H^{\circ}=-2032.9[\si{\kilo\joule}]$\\
    \end{center}

    \begin{equation}
      \begin{split}
        2(\ce{B2H6(s) + 3O2 -> B2O3(s) + 3H2O(g)})\\
        +\,\,\ce{2B2O3(s) -> 4B(s) + 3O2(g)}\\
        \hline\\
        .2(-2032.9)+2543.8=-1522[\si{\kilo\joule}]\\
        \ce{2B2H6(s) + 3O2 -> 4B(s) + 6H2O(g)}\\
        +\,\,\ce{6H2O(g) -> 3O2(g) + 6H2(g)}\\
        \hline\\
        -1522+6(241.8)=-71.2[\si{\kilo\joule}]\\
        \ce{2B2H6(s) -> 4B(s) + 6H2(g)}\\
        \ce{B2H6(s) -> 2B(s) + 3H2(g)}\\
        -71.2\cdot.5=-35.6[\si{\kilo\joule}]\\
      \end{split}
      \label{1}
    \end{equation}

\end{enumerate}

\end{document}

