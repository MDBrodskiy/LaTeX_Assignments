%%%%%%%%%%%%%%%%%%%%%%%%%%%%%%%%%%%%%%%%%%%%%%%%%%%%%%%%%%%%%%%%%%%%%%%%%%%%%%%%%%%%%%%%%%%%%%%%%%%%%%%%%%%%%%%%%%%%%%%%%%%%%%%%%%%%%%%%%%%%%%%%%%%%%%%%%%%%%%%%%%%%%%%%%%%%%%%%%%%%%%%%%%%%
% Written By Michael Brodskiy
% Class: AP Chemistry
% Professor: J. Morgan
%%%%%%%%%%%%%%%%%%%%%%%%%%%%%%%%%%%%%%%%%%%%%%%%%%%%%%%%%%%%%%%%%%%%%%%%%%%%%%%%%%%%%%%%%%%%%%%%%%%%%%%%%%%%%%%%%%%%%%%%%%%%%%%%%%%%%%%%%%%%%%%%%%%%%%%%%%%%%%%%%%%%%%%%%%%%%%%%%%%%%%%%%%%%

\documentclass[12pt]{article} 
\usepackage{alphalph}
\usepackage[utf8]{inputenc}
\usepackage[russian,english]{babel}
\usepackage{titling}
\usepackage{amsmath}
\usepackage{graphicx}
\usepackage{enumitem}
\usepackage{amssymb}
\usepackage[super]{nth}
\usepackage{expl3}
\usepackage[version=4]{mhchem}
\usepackage{hpstatement}
\usepackage{rsphrase}
\usepackage{everysel}
\usepackage{ragged2e}
\usepackage{geometry}
\usepackage{fancyhdr}
\usepackage{cancel}
\usepackage{siunitx}
\usepackage{chemfig}
\usepackage{multicol}
\geometry{top=1.0in,bottom=1.0in,left=1.0in,right=1.0in}
\newcommand{\subtitle}[1]{%
  \posttitle{%
    \par\end{center}
    \begin{center}\large#1\end{center}
    \vskip0.5em}%

}
\newcommand{\orbital}[2]{{%
    \def\+{\big|\hspace{-2pt}\overline{\underline{\hspace{2pt}\upharpoonleft}}}%
    \def\-{\overline{\underline{\downharpoonright\hspace{2pt}}}\hspace{-2pt}\big|}%
    \def\0{\big|\hspace{-2pt}\overline{\underline{\phantom{\hspace{2pt}\downharpoonright}}}}%
    \def\1{\overline{\underline{\phantom{\downharpoonright\hspace{2pt}}}}\hspace{-2pt}\big|}%
  \setlength\tabcolsep{0pt}% remove extra horizontal space from tabular
  \begin{tabular}{c}$#2$\\[2pt]#1\end{tabular}%
}}
\DeclareSIUnit\Molar{\textsc{M}}
\DeclareSIUnit\Molal{\textsc{m}}
\DeclareSIUnit\atm{\textsc{atm}}
\DeclareSIUnit\torr{\textsc{torr}}
\DeclareSIUnit\psi{\textsc{psi}}
\DeclareSIUnit\bar{\textsc{bar}}
\DeclareSIUnit\Celsius{C}
\DeclareSIUnit\degree{$^{\circ}$}
\DeclareSIUnit\calorie{cal}
\usepackage{hyperref}
\hypersetup{
colorlinks=true,
linkcolor=blue,
filecolor=magenta,      
urlcolor=blue,
citecolor=blue,
}

\urlstyle{same}


\title{Chapter 16 $-$ Problems 60, 70, 72}
\date{April 15, 2020}
\author{Michael Brodskiy\\ \small Instructor: Mr. Morgan}

% Mathematical Operations:

% Sum: $$\sum_{n=a}^{b} f(x) $$
% Integral: $$\int_{lower}^{upper} f(x) dx$$
% Limit: $$\lim_{x\to\infty} f(x)$$

\begin{document}

\maketitle

\begin{enumerate}

    \setcounter{enumi}{59}

  \item

    \begin{enumerate}

      \item 

        \begin{equation}
          \begin{split}
            \Delta G^{\circ} = \Delta H^{\circ} - T\Delta S^{\circ}\\
            \Delta G_{\ce{Ag+}}=77.1\left[ \frac{\si{\kilo\joule}}{\si{\mole}} \right]\\
            \Delta G_{\ce{Cl-}}=-131.2\left[ \frac{\si{\kilo\joule}}{\si{\mole}} \right]\\
            \Delta G_{\ce{AgCl}}=-109.8\left[ \frac{\si{\kilo\joule}}{\si{\mole}} \right]\\
            77.1+(-131.2)-(-109.8)=55.7\left[ \si{\kilo\joule} \right]
          \end{split}
          \label{1}
        \end{equation}

      \item 

        \begin{equation}
          \begin{split}
            55.7=-RT\ln(x^2)\\
            \ln(x^2)=-22.5\\
            x^2=e^{-22.5}\\
            x=1.3\cdot10^{-5}\\
          \end{split}
          \label{2}
        \end{equation}

      \item It does make sense because $K_{sp}=[\ce{Ag+}][\ce{Cl-}]$, which, with the above concentration equals $1.69\cdot10^{-10}$

    \end{enumerate}

    \setcounter{enumi}{69}

  \item

    \begin{equation}
      \begin{split}
        [\ce{H+}]=10^{-10.6}=2.51\cdot10^{-11}[\si{\Molar}]\\
        [\ce{OH-}]=10^{-14+10.6}=4\cdot10^{-4}[\si{\Molar}]\\
        .25-4\cdot10^{-4}=.2496[\si{\Molar}]\\
        K_b=\frac{\left( 4\cdot10^{-4} \right)^2}{.2496}\\
        =6.36\cdot10^{-7}\\
        -(.00831)(298)\ln\left( 6.46\cdot10^{-7} \right)=35.3[\si{\kilo\joule}]
      \end{split}
      \label{3}
    \end{equation}

    \setcounter{enumi}{71}

  \item

    \begin{equation}
      \begin{split}
        \ce{N2O5(g) -> 2NO(g) + 3/2O2(g)} \\
        +\,\,\,2\left( \ce{NO(g) + 1/2O2(g) -> NO2(g)} \right)\\
        \hline
        \ce{N2O5(g) -> 1/2O2(g) + 2NO2(g)}\\
        -\left( -59.2-2(35.6) \right)=130.4[\si{\kilo\joule}]
      \end{split}
      \label{4}
    \end{equation}

\end{enumerate}

\end{document}

