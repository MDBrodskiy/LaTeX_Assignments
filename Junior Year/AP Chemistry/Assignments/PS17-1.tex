%%%%%%%%%%%%%%%%%%%%%%%%%%%%%%%%%%%%%%%%%%%%%%%%%%%%%%%%%%%%%%%%%%%%%%%%%%%%%%%%%%%%%%%%%%%%%%%%%%%%%%%%%%%%%%%%%%%%%%%%%%%%%%%%%%%%%%%%%%%%%%%%%%%%%%%%%%%%%%%%%%%%%%%%%%%%%%%%%%%%%%%%%%%%
% Written By Michael Brodskiy
% Class: AP Chemistry
% Professor: J. Morgan
%%%%%%%%%%%%%%%%%%%%%%%%%%%%%%%%%%%%%%%%%%%%%%%%%%%%%%%%%%%%%%%%%%%%%%%%%%%%%%%%%%%%%%%%%%%%%%%%%%%%%%%%%%%%%%%%%%%%%%%%%%%%%%%%%%%%%%%%%%%%%%%%%%%%%%%%%%%%%%%%%%%%%%%%%%%%%%%%%%%%%%%%%%%%

\documentclass[12pt]{article} 
\usepackage{alphalph}
\usepackage[utf8]{inputenc}
\usepackage[russian,english]{babel}
\usepackage{titling}
\usepackage{amsmath}
\usepackage{graphicx}
\usepackage{enumitem}
\usepackage{amssymb}
\usepackage[super]{nth}
\usepackage{expl3}
\usepackage[version=4]{mhchem}
\usepackage{hpstatement}
\usepackage{rsphrase}
\usepackage{everysel}
\usepackage{ragged2e}
\usepackage{geometry}
\usepackage{fancyhdr}
\usepackage{cancel}
\usepackage{siunitx}
\usepackage{chemfig}
\usepackage{multicol}
\geometry{top=1.0in,bottom=1.0in,left=1.0in,right=1.0in}
\newcommand{\subtitle}[1]{%
  \posttitle{%
    \par\end{center}
    \begin{center}\large#1\end{center}
    \vskip0.5em}%

}
\newcommand{\orbital}[2]{{%
    \def\+{\big|\hspace{-2pt}\overline{\underline{\hspace{2pt}\upharpoonleft}}}%
    \def\-{\overline{\underline{\downharpoonright\hspace{2pt}}}\hspace{-2pt}\big|}%
    \def\0{\big|\hspace{-2pt}\overline{\underline{\phantom{\hspace{2pt}\downharpoonright}}}}%
    \def\1{\overline{\underline{\phantom{\downharpoonright\hspace{2pt}}}}\hspace{-2pt}\big|}%
  \setlength\tabcolsep{0pt}% remove extra horizontal space from tabular
  \begin{tabular}{c}$#2$\\[2pt]#1\end{tabular}%
}}
\DeclareSIUnit\Molar{\textsc{M}}
\DeclareSIUnit\Molal{\textsc{m}}
\DeclareSIUnit\atm{\textsc{atm}}
\DeclareSIUnit\torr{\textsc{torr}}
\DeclareSIUnit\psi{\textsc{psi}}
\DeclareSIUnit\bar{\textsc{bar}}
\DeclareSIUnit\Celsius{C}
\DeclareSIUnit\degree{$^{\circ}$}
\DeclareSIUnit\calorie{cal}
\usepackage{hyperref}
\hypersetup{
colorlinks=true,
linkcolor=blue,
filecolor=magenta,      
urlcolor=blue,
citecolor=blue,
}

\urlstyle{same}


\title{Chapter 17 $-$ Problem Set 1}
\date{April 16, 2020}
\author{Michael Brodskiy\\ \small Instructor: Mr. Morgan}

% Mathematical Operations:

% Sum: $$\sum_{n=a}^{b} f(x) $$
% Integral: $$\int_{lower}^{upper} f(x) dx$$
% Limit: $$\lim_{x\to\infty} f(x)$$

\begin{document}

\maketitle

\begin{enumerate}

  \item

    \begin{enumerate}

      \item \ce{W} $=+4$

      \item \ce{Na} $=+4$; \ce{O} $=-2$

      \item \ce{H} $=+1$; \ce{I} $=+5$; \ce{O} $=-2$

      \item \ce{C} $=+4$; \ce{O} $=-2$

    \end{enumerate}

  \item

    \begin{enumerate}

      \item \ce{NO2-} loses electrons, so it is oxidized, and \ce{CrO4^2-} is reduced

      \item \ce{ClO3-} gains electrons, so it is reduced, and \ce{S^2-} is oxidizes

    \end{enumerate}

  \item

    \begin{enumerate}

      \item \ce{ClO2 + H2O -> ClO3- + 2H+ + e-}

      \item \ce{MnO4- + 4H+ + 3e- -> MnO2 + 2H2O}

    \end{enumerate}

  \item

    \begin{enumerate}

      \item 

        \begin{equation}
          \begin{split}
            5(\ce{2Cl- -> Cl2 + 2e-})\\
            +\,\,\,2(\ce{MnO4- + 8H+ + 5e- -> Mn^2+ + 4H2O})\\
            \hline
            \ce{2MnO4- + 16H+ + 10Cl- -> 2Mn^2+ + 8H2O + 5Cl2}
          \end{split}
          \label{1}
        \end{equation}

      \item 

        \begin{equation}
          \begin{split}
            2(\ce{Sn -> Sn^2+ + 2e-})\\
            +\,\,\,\ce{O2 + 4H+ + 4e- -> H2O}\\
            \hline
            \ce{2Sn + O2 + 4H+ -> 2Sn^2+ + H2O}
          \end{split}
          \label{2}
        \end{equation}

      \item 

        \begin{equation}
          \begin{split}
            3\ce{Se + 2H2O -> SeO2 + 4H+ + 4e-}\\
            +\,\,\,4\ce{NO3- + 4H+ + 3e- -> NO + 2H2O}\\
            \hline
            \ce{3Se + 4NO3- + 4H+ -> 3SeO2 + 4NO + 2H2O}
          \end{split}
          \label{3}
        \end{equation}

    \end{enumerate}

    \newpage

  \item

    \begin{enumerate}

      \item \textbf{ }\\

        \begin{center}
          \begin{figure}[h!]
            \centering
            \include{Figures/ZnAg}
            \caption{Galvanic Cell for \ce{Zn}/\ce{Zn^2+}//\ce{Ag+}/\ce{Ag}}
            \label{fig:1}
          \end{figure}
        \end{center}

      \item \textbf{ }\\

        \begin{center}
          \begin{figure}[h!]
            \centering
            \include{Figures/FeO2}
            \caption{Galvanic Cell for \ce{Fe^2+}/\ce{Fe^3+}//\ce{O2}/\ce{H2O}}
            \label{fig:2}
          \end{figure}
        \end{center}

      \item \textbf{ }\\

        \begin{center}
          \begin{figure}[h!]
            \centering
            \include{Figures/FeH}
            \caption{Galvanic Cell for \ce{Fe}/\ce{Fe(OH)2}//\ce{2H2O}/\ce{H2}}
            \label{fig:3}
          \end{figure}
        \end{center}

        \newpage

      \item \textbf{ }\\

        \begin{center}
          \begin{figure}[h!]
            \centering
            \include{Figures/BrI}
            \caption{Galvanic Cell for \ce{Br2}/\ce{2Br-}//\ce{2I-}/\ce{I2}}
            \label{fig:4}
          \end{figure}
        \end{center}

        \newpage

      \item \textbf{ }\\

        \begin{center}
          \begin{figure}[h!]
            \centering
            \tikzset{every picture/.style={line width=0.75pt}} %set default line width to 0.75pt        

\begin{tikzpicture}[x=0.75pt,y=0.75pt,yscale=-1,xscale=1]
%uncomment if require: \path (0,300); %set diagram left start at 0, and has height of 300

%Shape: Can [id:dp4503664778559091] 
\draw  [fill={rgb, 255:red, 128; green, 128; blue, 128 }  ,fill opacity=1 ] (190,171) -- (190,255) .. controls (190,259.97) and (176.57,264) .. (160,264) .. controls (143.43,264) and (130,259.97) .. (130,255) -- (130,171) .. controls (130,166.03) and (143.43,162) .. (160,162) .. controls (176.57,162) and (190,166.03) .. (190,171) .. controls (190,175.97) and (176.57,180) .. (160,180) .. controls (143.43,180) and (130,175.97) .. (130,171) ;
%Shape: Ellipse [id:dp565979149570075] 
\draw  [fill={rgb, 255:red, 255; green, 255; blue, 255 }  ,fill opacity=1 ] (130,171) .. controls (130,166.03) and (143.43,162) .. (160,162) .. controls (176.57,162) and (190,166.03) .. (190,171) .. controls (190,175.97) and (176.57,180) .. (160,180) .. controls (143.43,180) and (130,175.97) .. (130,171) -- cycle ;
%Straight Lines [id:da3663451366147634] 
\draw [color={rgb, 255:red, 208; green, 2; blue, 27 }  ,draw opacity=1 ]   (160,29.58) -- (160,171) ;

%Shape: Can [id:dp0951927406661861] 
\draw  [fill={rgb, 255:red, 128; green, 128; blue, 128 }  ,fill opacity=1 ] (333,171) -- (333,255) .. controls (333,259.97) and (319.57,264) .. (303,264) .. controls (286.43,264) and (273,259.97) .. (273,255) -- (273,171) .. controls (273,166.03) and (286.43,162) .. (303,162) .. controls (319.57,162) and (333,166.03) .. (333,171) .. controls (333,175.97) and (319.57,180) .. (303,180) .. controls (286.43,180) and (273,175.97) .. (273,171) ;
%Shape: Ellipse [id:dp9301470966524585] 
\draw  [fill={rgb, 255:red, 255; green, 255; blue, 255 }  ,fill opacity=1 ] (273,171) .. controls (273,166.03) and (286.43,162) .. (303,162) .. controls (319.57,162) and (333,166.03) .. (333,171) .. controls (333,175.97) and (319.57,180) .. (303,180) .. controls (286.43,180) and (273,175.97) .. (273,171) -- cycle ;
%Straight Lines [id:da15512893440324627] 
\draw [color={rgb, 255:red, 208; green, 2; blue, 27 }  ,draw opacity=1 ]   (303,29.58) -- (303,171) ;

%Straight Lines [id:da5383287109318899] 
\draw [color={rgb, 255:red, 208; green, 2; blue, 27 }  ,draw opacity=1 ]   (160,29.58) -- (303,29.58) ;
%Straight Lines [id:da19199568942290224] 
\draw [color={rgb, 255:red, 74; green, 144; blue, 226 }  ,draw opacity=1 ]   (174,42) -- (174,171) ;
%Straight Lines [id:da8343730433897103] 
\draw [color={rgb, 255:red, 74; green, 144; blue, 226 }  ,draw opacity=1 ]   (290.6,42) -- (290.6,171) ;
%Straight Lines [id:da020014113786278376] 
\draw [color={rgb, 255:red, 74; green, 144; blue, 226 }  ,draw opacity=1 ]   (290.6,42) -- (174,42) ;
%Curve Lines [id:da7591506959538374] 
\draw    (378,162) .. controls (351.27,180.81) and (367.66,121.21) .. (334.03,169.5) ;
\draw [shift={(333,171)}, rotate = 304.46] [color={rgb, 255:red, 0; green, 0; blue, 0 }  ][line width=0.75]    (10.93,-3.29) .. controls (6.95,-1.4) and (3.31,-0.3) .. (0,0) .. controls (3.31,0.3) and (6.95,1.4) .. (10.93,3.29)   ;
%Curve Lines [id:da17504057504299442] 
\draw    (160,180) .. controls (102.18,191.76) and (89.5,187.19) .. (90.9,155.95) ;
\draw [shift={(91,154)}, rotate = 453.47] [color={rgb, 255:red, 0; green, 0; blue, 0 }  ][line width=0.75]    (10.93,-3.29) .. controls (6.95,-1.4) and (3.31,-0.3) .. (0,0) .. controls (3.31,0.3) and (6.95,1.4) .. (10.93,3.29)   ;
%Shape: Boxed Line [id:dp5443668674906494] 
\draw    (174,16) -- (288.71,15.02) ;
\draw [shift={(290.71,15)}, rotate = 539.51] [color={rgb, 255:red, 0; green, 0; blue, 0 }  ][line width=0.75]    (10.93,-3.29) .. controls (6.95,-1.4) and (3.31,-0.3) .. (0,0) .. controls (3.31,0.3) and (6.95,1.4) .. (10.93,3.29)   ;

% Text Node
\draw (150,208) node [anchor=north west][inner sep=0.75pt]   [align=left] {$\displaystyle Fe$};
% Text Node
\draw (380,165) node [anchor=north west][inner sep=0.75pt]   [align=left] {$\displaystyle Cu^{2+}$};
% Text Node
\draw (293,206) node [anchor=north west][inner sep=0.75pt]   [align=left] {$\displaystyle Cu$};
% Text Node
\draw (91,151) node [anchor=south] [inner sep=0.75pt]   [align=left] {$\displaystyle Fe^{2+}$};
% Text Node
\draw (172,16) node [anchor=east] [inner sep=0.75pt]   [align=left] {$\displaystyle e^{-}$};
% Text Node
\draw (160,267) node [anchor=north] [inner sep=0.75pt]   [align=left] {Anode};
% Text Node
\draw (303,267) node [anchor=north] [inner sep=0.75pt]   [align=left] {Cathode};


\end{tikzpicture}

            \caption{Galvanic Cell for \ce{Fe}/\ce{Fe(OH)2}//\ce{Cu(OH)2}/\ce{Cu}}
            \label{fig:5}
          \end{figure}
        \end{center}

    \end{enumerate}

\end{enumerate}

\end{document}

