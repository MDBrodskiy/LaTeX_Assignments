%%%%%%%%%%%%%%%%%%%%%%%%%%%%%%%%%%%%%%%%%%%%%%%%%%%%%%%%%%%%%%%%%%%%%%%%%%%%%%%%%%%%%%%%%%%%%%%%%%%%%%%%%%%%%%%%%%%%%%%%%%%%%%%%%%%%%%%%%%%%%%%%%%%%%%%%%%%%%%%%%%%%%%%%%%%%%%%%%%%%%%%%%%%%
% Written By Michael Brodskiy
% Class: AP Chemistry
% Professor: J. Morgan
%%%%%%%%%%%%%%%%%%%%%%%%%%%%%%%%%%%%%%%%%%%%%%%%%%%%%%%%%%%%%%%%%%%%%%%%%%%%%%%%%%%%%%%%%%%%%%%%%%%%%%%%%%%%%%%%%%%%%%%%%%%%%%%%%%%%%%%%%%%%%%%%%%%%%%%%%%%%%%%%%%%%%%%%%%%%%%%%%%%%%%%%%%%%

\documentclass[12pt]{article} 
\usepackage{alphalph}
\usepackage[utf8]{inputenc}
\usepackage[russian,english]{babel}
\usepackage{titling}
\usepackage{amsmath}
\usepackage{graphicx}
\usepackage{enumitem}
\usepackage{amssymb}
\usepackage[super]{nth}
\usepackage{everysel}
\usepackage{ragged2e}
\usepackage{geometry}
\usepackage{fancyhdr}
\usepackage{cancel}
\usepackage{siunitx}
\geometry{top=1.0in,bottom=1.0in,left=1.0in,right=1.0in}
\newcommand{\subtitle}[1]{%
  \posttitle{%
    \par\end{center}
    \begin{center}\large#1\end{center}
    \vskip0.5em}%

}
\usepackage{hyperref}
\hypersetup{
colorlinks=true,
linkcolor=blue,
filecolor=magenta,      
urlcolor=blue,
citecolor=blue,
}

\urlstyle{same}


\title{Problem Set Chapter 1 \& 2}
\date{August 27, 2020}
\author{Michael Brodskiy\\ \small Instructor: Mr. Morgan}

% Mathematical Operations:

% Sum: $$\sum_{n=a}^{b} f(x) $$
% Integral: $$\int_{lower}^{upper} f(x) dx$$
% Limit: $$\lim_{x\to\infty} f(x)$$

\begin{document}

\maketitle

\begin{enumerate}

  \item Convert the following:

    \begin{enumerate}

      \item SKIP
        
      \item $800[\si{\gram\per\liter}]\to [\text{lb}\,\text{in}^{-3}]$

        $$1[\si{\gram}]=.002205[\text{lb}]$$
        $$1[\si{\liter}]=61.0237$$
        $$\frac{800[\cancel{\si{\gram}}]}{1[\cancel{\si{\liter}}]}\cdot\frac{.002205[\text{lb}]}{1[\cancel{\si{\gram}}]}\cdot\frac{1[\cancel{\si{\liter}}]}{61.0237[\text{in}^3]}=.0289\left[\frac{\text{lb}}{\text{in}^2}\right]$$

    \end{enumerate}

  \item SKIP

  \item Name the following:

    \begin{enumerate}

      \item $LiOH \rightarrow$ Lithium Hydroxide

      \item $CaF_2 \rightarrow$ Calcium Fluoride

      \item $FeCO_3 \rightarrow$ Iron (II) Carbonate

      \item $S_4N_2 \rightarrow$ Tetrasulfur Dinitride

      \item $Zn(NO_3)_2 \rightarrow$ Zinc Nitrate
        
      \item $K_2SO_4 \rightarrow$ Potassium Sulfate

      \item $NO \rightarrow$ Nitrogen Monoxide

      \item $FeCl_2 \rightarrow$ Iron (II) Chloride

      \item $Na_2O \rightarrow$ Sodium Oxide

      \item $K_2S \rightarrow$ Potassium Sulfide

      \item $Cr_2(SO_4)_3 \rightarrow$ Chromium (III) Sulfate

      \item $Cu(OH)_2 \rightarrow$ Copper (II) Hydroxide

      \item $KOH \rightarrow$ Potassium Hydroxide

      \item $CuI \rightarrow$ Copper (I) Iodide

    \end{enumerate}

  \item Write the formula for the following compound:
    
    \begin{enumerate}

      \item Diselenium Diiodide $\rightarrow Se_2I_2$

      \item Tin (II) Phosphate $\rightarrow Sn_3(PO_4)_2$

      \item Potassium Dichromate $\rightarrow K_2Cr_2O_7$

      \item Gold (I) Sulfide $\rightarrow Au_2S$

      \item Barium Hydroxide $\rightarrow Ba(OH)_2$

      \item Ammonium Phosphate $\rightarrow (NH_4)_3PO_4$

      \item Potassium Sulfate $\rightarrow K_2SO_4$

      \item Calcium Nitrate $\rightarrow Ca(NO_3)_2$

      \item Iron (II) Carbonate $\rightarrow FeCO_3$

      \item Ammonium Dichromate $\rightarrow (NH_4)_2Cr_2O_7$

      \item Potassium Sulfide $\rightarrow K_2S$

      \item Cobalt (II) Nitrate $\rightarrow Co(NO_3)_2$

    \end{enumerate}

  \item Calculate the mass (in grams) of nitric acid that is contained in a 3.5 liter mixture of 69.8\% by weight of nitric acid. Density of mixture is $1.42[\si{\gram\per\centi\meter\cubed}]$

    $$1[\si{\centi\meter\cubed}]=.001[\si{\liter}]$$
    $$\rho=\frac{m}{V}$$
    $$\rho V=m \rightarrow \frac{1.42[\si{\gram}]}{1[\cancel{\si{\centi\meter\cubed}}]}\cdot\frac{1000[\cancel{\si{\centi\meter\cubed}}]}{1[\cancel{\si{\liter}}]}\cdot 3.5[\cancel{\si{\liter}}]\cdot.698=3.47\cdot10^3[\si{\gram}]$$

  \item The density of a piece of silver is $10.5[\si{\gram\per\milli\liter}]$. This piece is placed in a graduated cylinder containing $11.2[\si{\milli\liter}]$ of water, and then the water rises to $11.7[\si{\milli\liter}]$. What is the mass (in grams) of the piece of silver?

    $$V_{Ag}=11.7-11.2=.5[\si{\milli\liter}]$$
    $$m=\rho V\rightarrow m=.5[\cancel{\si{\milli\liter}}]\cdot\frac{10.5[\si{\gram}]}{1[\cancel{\si{\milli\liter}}]}=5.25[\si{\gram}]$$

  \item How long is a cylindrical bar with base area of $1.5[\si{\centi\meter\squared}]$, if it is made of $898[\si{\kilo\gram}]$ of iron with a density of $7.76[\si{\gram\per\centi\meter\cubed}]$.

    $$898[\si{\kilo\gram}]=898000[\si{\gram}]$$
    $$\rho=\frac{m}{V}\rightarrow\frac{\rho}{m}=\frac{1}{1.5l}\rightarrow l=\frac{m}{1.5\rho}$$
    $$l=\frac{898000}{1.5\cdot7.76}=7.7\cdot10^4[\si{\centi\meter}]$$

  \item A square of foil with density $2.7[\si{\gram\per\milli\liter}]$ is $5.1[\si{\centi\meter}]$ on a side and has a mass of $1.762[\si{\gram}]$. Calculate the thickness of the foil.

    $$\rho=\frac{m}{V}\rightarrow V=\frac{m}{\rho}\rightarrow l=\frac{m}{A\rho}$$
    $$l=\frac{1.762}{5.1\cdot5.1\cdot2.7}=.025[\si{\centi\meter}]$$

  \item A certain material has a density of $12.8[\si{\kilo\gram\per\meter\cubed}]$. How many grams of this material are needed to fill a volume of 2[$ft^3$]?

    $$2[ft^3]=.056634[\si{\meter\cubed}]$$
    $$\rho=\frac{m}{V}\rightarrow m=\rho V$$
    $$m=\frac{12.8[\cancel{\si{\kilo\gram}}]}{1[\cancel{\si{\meter\cubed}}]}\cdot.055634[\cancel{\si{\meter\cubed}}]\cdot\frac{1000[\si{\gram}]}{1[\cancel{\si{\kilo\gram}}]}=712[\si{\gram}]$$

\end{enumerate}

\end{document}
