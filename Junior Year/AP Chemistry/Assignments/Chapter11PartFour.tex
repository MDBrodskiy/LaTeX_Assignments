%%%%%%%%%%%%%%%%%%%%%%%%%%%%%%%%%%%%%%%%%%%%%%%%%%%%%%%%%%%%%%%%%%%%%%%%%%%%%%%%%%%%%%%%%%%%%%%%%%%%%%%%%%%%%%%%%%%%%%%%%%%%%%%%%%%%%%%%%%%%%%%%%%%%%%%%%%%%%%%%%%%%%%%%%%%%%%%%%%%%%%%%%%%%
% Written By Michael Brodskiy
% Class: AP Chemistry
% Professor: J. Morgan
%%%%%%%%%%%%%%%%%%%%%%%%%%%%%%%%%%%%%%%%%%%%%%%%%%%%%%%%%%%%%%%%%%%%%%%%%%%%%%%%%%%%%%%%%%%%%%%%%%%%%%%%%%%%%%%%%%%%%%%%%%%%%%%%%%%%%%%%%%%%%%%%%%%%%%%%%%%%%%%%%%%%%%%%%%%%%%%%%%%%%%%%%%%%

\documentclass[12pt]{article} 
\usepackage{alphalph}
\usepackage[utf8]{inputenc}
\usepackage[russian,english]{babel}
\usepackage{titling}
\usepackage{amsmath}
\usepackage{graphicx}
\usepackage{enumitem}
\usepackage{amssymb}
\usepackage[super]{nth}
\usepackage{expl3}
\usepackage[version=4]{mhchem}
\usepackage{hpstatement}
\usepackage{rsphrase}
\usepackage{everysel}
\usepackage{ragged2e}
\usepackage{geometry}
\usepackage{fancyhdr}
\usepackage{cancel}
\usepackage{siunitx}
\usepackage{chemfig}
\usepackage{multicol}
\usepackage{xcolor}
\usepackage{color, colortbl}
\definecolor{cadetgrey}{rgb}{0.57, 0.64, 0.69}
\geometry{top=1.0in,bottom=1.0in,left=1.0in,right=1.0in}
\newcommand{\subtitle}[1]{%
  \posttitle{%
    \par\end{center}
    \begin{center}\large#1\end{center}
    \vskip0.5em}%

}
\newcommand{\orbital}[2]{{%
    \def\+{\big|\hspace{-2pt}\overline{\underline{\hspace{2pt}\upharpoonleft}}}%
    \def\-{\overline{\underline{\downharpoonright\hspace{2pt}}}\hspace{-2pt}\big|}%
    \def\0{\big|\hspace{-2pt}\overline{\underline{\phantom{\hspace{2pt}\downharpoonright}}}}%
    \def\1{\overline{\underline{\phantom{\downharpoonright\hspace{2pt}}}}\hspace{-2pt}\big|}%
  \setlength\tabcolsep{0pt}% remove extra horizontal space from tabular
  \begin{tabular}{c}$#2$\\[2pt]#1\end{tabular}%
}}
\DeclareSIUnit\Molar{\textsc{M}}
\DeclareSIUnit\Molal{\textsc{m}}
\DeclareSIUnit\atm{\textsc{atm}}
\DeclareSIUnit\torr{\textsc{torr}}
\DeclareSIUnit\psi{\textsc{psi}}
\DeclareSIUnit\bar{\textsc{bar}}
\DeclareSIUnit\Celsius{C}
\DeclareSIUnit\degree{$^{\circ}$}
\DeclareSIUnit\calorie{cal}
\usepackage{hyperref}
\hypersetup{
colorlinks=true,
linkcolor=blue,
filecolor=magenta,      
urlcolor=blue,
citecolor=blue,
}

\urlstyle{same}


\title{Chapter 11 $-$ Practice FRQ}
\date{February 8, 2020}
\author{Michael Brodskiy\\ \small Instructor: Mr. Morgan}

% Mathematical Operations:

% Sum: $$\sum_{n=a}^{b} f(x) $$
% Integral: $$\int_{lower}^{upper} f(x) dx$$
% Limit: $$\lim_{x\to\infty} f(x)$$

\begin{document}

\maketitle

\begin{enumerate}

    \setcounter{enumi}{2}

  \item Answer the following questions related to the kinetics of chemical reactions.

    \begin{center}
      \ce{I-(aq) + ClO-(aq) ->[OH-] IO-(aq) + Cl-(aq)}
    \end{center}

    \begin{center}
      Iodide ion, \ce{I-}, is oxidized to hypoiodite ion, \ce{IO-}, by hypochlorite, \ce{ClO-}, in a basic solution according to the equation above. Three initial-rate experiments were conducted; the results are shown in the following table.\\
      \begin{tabular}[H]{|c|c|c|c|}
        \hline
        \rowcolor{cadetgrey} Experiment & $[\ce{I-}]$ & $[\ce{ClO-}]$ & Initial Rate \\
        \hline
        1 & 0.017 & 0.015 & 0.156 \\
        \hline
        2 & 0.052 & 0.015 & 0.476 \\
        \hline
        3 & 0.016 & 0.061 & 0.596 \\
        \hline
      \end{tabular}
    \end{center}

    \begin{enumerate}

      \item Determine the order of the reaction with respect to each reactant listed below. Show your work.

        \begin{enumerate}

          \item \ce{I-(aq)}

            \begin{equation}
              \begin{split}
              \frac{156}{476}=\left(\frac{17}{52}\right)^m\\
              m=1\\
              \end{split}
              \label{1}
            \end{equation}

          \item \ce{ClO-(aq)}

            \begin{equation}
              \begin{split}
                \frac{156}{596}=\frac{17}{16}\cdot\left( \frac{15}{61} \right)^n\\
                n=1
              \end{split}
              \label{2}
            \end{equation}

        \end{enumerate}

      \item For the reaction,

        \begin{enumerate}

          \item write the law that is consistent with the calculations in part (a);

            \begin{equation}
              rate=k[\ce{I-(aq)}][\ce{ClO-(aq)}]
              \label{3}
            \end{equation}

          \item calculate the value of the specific rate constant, $k$, and specify units.

            \begin{equation}
              \begin{split}
                .596=k\left( .016 \right)\left( .061 \right)\\
                \frac{.596}{.016\cdot.061}=610.66\left[ \frac{1}{\si{\Molar\second}} \right]
              \end{split}
              \label{4}
            \end{equation}

        \end{enumerate}

    \end{enumerate}

\end{enumerate}

\end{document}

