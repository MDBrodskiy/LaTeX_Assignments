%%%%%%%%%%%%%%%%%%%%%%%%%%%%%%%%%%%%%%%%%%%%%%%%%%%%%%%%%%%%%%%%%%%%%%%%%%%%%%%%%%%%%%%%%%%%%%%%%%%%%%%%%%%%%%%%%%%%%%%%%%%%%%%%%%%%%%%%%%%%%%%%%%%%%%%%%%%%%%%%%%%%%%%%%%%%%%%%%%%%%%%%%%%%
% Written By Michael Brodskiy
% Class: AP Chemistry
% Professor: J. Morgan
%%%%%%%%%%%%%%%%%%%%%%%%%%%%%%%%%%%%%%%%%%%%%%%%%%%%%%%%%%%%%%%%%%%%%%%%%%%%%%%%%%%%%%%%%%%%%%%%%%%%%%%%%%%%%%%%%%%%%%%%%%%%%%%%%%%%%%%%%%%%%%%%%%%%%%%%%%%%%%%%%%%%%%%%%%%%%%%%%%%%%%%%%%%%

\documentclass[12pt]{article} 
\usepackage{alphalph}
\usepackage[utf8]{inputenc}
\usepackage[russian,english]{babel}
\usepackage{titling}
\usepackage{amsmath}
\usepackage{physics}
\usepackage{tikz}
\usepackage{mathdots}
\usepackage{yhmath}
\usepackage{cancel}
\usepackage{color}
\usepackage{siunitx}
\usepackage{array}
\usepackage{multirow}
\usepackage{amssymb}
\usepackage{gensymb}
\usepackage{tabularx}
\usepackage{booktabs}
\usetikzlibrary{fadings}
\usetikzlibrary{patterns}
\usetikzlibrary{shadows.blur}
\usetikzlibrary{shapes}
\usepackage{graphicx}
\usepackage{enumitem}
\usepackage[super]{nth}
\usepackage{expl3}
\usepackage[version=4]{mhchem}
\usepackage{hpstatement}
\usepackage{rsphrase}
\usepackage{everysel}
\usepackage{ragged2e}
\usepackage{geometry}
\usepackage{fancyhdr}
\usepackage{cancel}
\usepackage{siunitx} 
\geometry{top=1.0in,bottom=1.0in,left=1.0in,right=1.0in}
\newcommand{\subtitle}[1]{%
  \posttitle{%
    \par\end{center}
    \begin{center}\large#1\end{center}
    \vskip0.5em}%

}
\newcommand{\orbital}[2]{{%
    \def\+{\big|\hspace{-2pt}\overline{\underline{\hspace{2pt}\upharpoonleft}}}%
    \def\-{\overline{\underline{\downharpoonright\hspace{2pt}}}\hspace{-2pt}\big|}%
    \def\0{\big|\hspace{-2pt}\overline{\underline{\phantom{\hspace{2pt}\downharpoonright}}}}%
    \def\1{\overline{\underline{\phantom{\downharpoonright\hspace{2pt}}}}\hspace{-2pt}\big|}%
  \setlength\tabcolsep{0pt}% remove extra horizontal space from tabular
  \begin{tabular}{c}$#2$\\[2pt]#1\end{tabular}%
}}
\DeclareSIUnit\Molar{\textsc{m}}
\DeclareSIUnit\atm{\textsc{atm}}
\DeclareSIUnit\Celsius{^{\circ}\textsc{C}}
\DeclareSIUnit\torr{\textsc{torr}}
\DeclareSIUnit\mmHg{\textsc{mmHg}}
\usepackage{hyperref}
\hypersetup{
colorlinks=true,
linkcolor=blue,
filecolor=magenta,      
urlcolor=blue,
citecolor=blue,
}

\urlstyle{same}


\title{Chapter 6 $-$ Problems 56, 58, 60}
\date{November 17, 2020}
\author{Michael Brodskiy\\ \small Instructor: Mr. Morgan}

% Mathematical Operations:

% Sum: $$\sum_{n=a}^{b} f(x) $$
% Integral: $$\int_{lower}^{upper} f(x) dx$$
% Limit: $$\lim_{x\to\infty} f(x)$$

\begin{document}

\maketitle

\begin{enumerate}

    \setcounter{enumi}{55}

  \item Which of the four atoms \ce{Na}, \ce{P}, \ce{Cl}, or \ce{K}:

    \begin{enumerate}

      \item has the largest atomic radius

        \begin{justifying}
          \ce{K}
        \end{justifying}

      \item has the highest ionization energy

        \begin{justifying}
          \ce{Cl}
        \end{justifying}

      \item is the most electronegative

        \begin{justifying}
          \ce{Cl}
        \end{justifying}

    \end{enumerate}

    \setcounter{enumi}{57}

  \item Select the smaller member of each pair

    \begin{enumerate}

      \item \ce{P} and \ce{P^3-}

        \begin{justifying}
          \ce{P}
        \end{justifying}

      \item \ce{V^2+} and \ce{V^4+}

        \begin{justifying}
          \ce{V^4+}
        \end{justifying}

      \item \ce{K} and \ce{K^+}

        \begin{justifying}
          \ce{K^+}
        \end{justifying}

      \item \ce{Co} and \ce{Co^3+}

        \begin{justifying}
          \ce{Co^3+}
        \end{justifying}

    \end{enumerate}

    \setcounter{enumi}{59}

  \item Arrange the following species in order of increasing radius:

    \begin{enumerate}

      \item \ce{Rb}, \ce{K}, \ce{Cs}, \ce{Kr}

        \begin{center}
          \begin{tabular}{l c c c c r}
            Least & \ce{Kr} & \ce{K} & \ce{Rb} & \ce{Cs} & Greatest\\
          \end{tabular}
        \end{center}

      \item \ce{Ar}, \ce{Cs}, \ce{Si}, \ce{Al}

        \begin{center}
          \begin{tabular}{l c c c c r}
            Least & \ce{Ar} & \ce{Si} & \ce{Al} & \ce{Cs} & Greatest\\
          \end{tabular}
        \end{center}

    \end{enumerate}

\end{enumerate}

\end{document}

