%%%%%%%%%%%%%%%%%%%%%%%%%%%%%%%%%%%%%%%%%%%%%%%%%%%%%%%%%%%%%%%%%%%%%%%%%%%%%%%%%%%%%%%%%%%%%%%%%%%%%%%%%%%%%%%%%%%%%%%%%%%%%%%%%%%%%%%%%%%%%%%%%%%%%%%%%%%%%%%%%%%%%%%%%%%%%%%%%%%%%%%%%%%%
% Written By Michael Brodskiy
% Class: AP Chemistry
% Professor: J. Morgan
%%%%%%%%%%%%%%%%%%%%%%%%%%%%%%%%%%%%%%%%%%%%%%%%%%%%%%%%%%%%%%%%%%%%%%%%%%%%%%%%%%%%%%%%%%%%%%%%%%%%%%%%%%%%%%%%%%%%%%%%%%%%%%%%%%%%%%%%%%%%%%%%%%%%%%%%%%%%%%%%%%%%%%%%%%%%%%%%%%%%%%%%%%%%

\documentclass[12pt]{article} 
\usepackage{alphalph}
\usepackage[utf8]{inputenc}
\usepackage[russian,english]{babel}
\usepackage{titling}
\usepackage{amsmath}
\usepackage{physics}
\usepackage{tikz}
\usepackage{mathdots}
\usepackage{yhmath}
\usepackage{cancel}
\usepackage{color}
\usepackage{siunitx}
\usepackage{array}
\usepackage{multirow}
\usepackage{amssymb}
\usepackage{gensymb}
\usepackage{tabularx}
\usepackage{booktabs}
\usetikzlibrary{fadings}
\usetikzlibrary{patterns}
\usetikzlibrary{shadows.blur}
\usetikzlibrary{shapes}
\usepackage{graphicx}
\usepackage{enumitem}
\usepackage[super]{nth}
\usepackage{expl3}
\usepackage[version=4]{mhchem}
\usepackage{hpstatement}
\usepackage{rsphrase}
\usepackage{everysel}
\usepackage{ragged2e}
\usepackage{geometry}
\usepackage{fancyhdr}
\usepackage{cancel}
\usepackage{siunitx} 
\geometry{top=1.0in,bottom=1.0in,left=1.0in,right=1.0in}
\newcommand{\subtitle}[1]{%
  \posttitle{%
    \par\end{center}
    \begin{center}\large#1\end{center}
    \vskip0.5em}%

}
\newcommand{\orbital}[2]{{%
    \def\+{\big|\hspace{-2pt}\overline{\underline{\hspace{2pt}\upharpoonleft}}}%
    \def\-{\overline{\underline{\downharpoonright\hspace{2pt}}}\hspace{-2pt}\big|}%
    \def\0{\big|\hspace{-2pt}\overline{\underline{\phantom{\hspace{2pt}\downharpoonright}}}}%
    \def\1{\overline{\underline{\phantom{\downharpoonright\hspace{2pt}}}}\hspace{-2pt}\big|}%
  \setlength\tabcolsep{0pt}% remove extra horizontal space from tabular
  \begin{tabular}{c}$#2$\\[2pt]#1\end{tabular}%
}}
\DeclareSIUnit\Molar{\textsc{m}}
\DeclareSIUnit\atm{\textsc{atm}}
\DeclareSIUnit\Celsius{^{\circ}\textsc{C}}
\DeclareSIUnit\torr{\textsc{torr}}
\DeclareSIUnit\mmHg{\textsc{mmHg}}
\usepackage{hyperref}
\hypersetup{
colorlinks=true,
linkcolor=blue,
filecolor=magenta,      
urlcolor=blue,
citecolor=blue,
}

\urlstyle{same}


\title{Chapter 6 $-$ Problems 32, 50, 52, 54, 75}
\date{November 12, 2020}
\author{Michael Brodskiy\\ \small Instructor: Mr. Morgan}

% Mathematical Operations:

% Sum: $$\sum_{n=a}^{b} f(x) $$
% Integral: $$\int_{lower}^{upper} f(x) dx$$
% Limit: $$\lim_{x\to\infty} f(x)$$

\begin{document}

\maketitle

\begin{enumerate}

    \setcounter{enumi}{31}

  \item Write the ground state electron configuration for:

    \begin{enumerate}

      \item \ce{Mg}

        \begin{center}
          \ce{[Ne] 3s^2}
        \end{center}

      \item \ce{Os}

        \begin{center}
          \ce{[Xe] 6s^2 4f^14 5d^6}
        \end{center}

      \item \ce{Ge}

        \begin{center}
        \ce{[Ar] 4s^2 3d^10 4p^2}
        \end{center}

      \item \ce{V}

        \begin{center}
          \ce{[Ar] 4s^2 3d^3}
        \end{center}

      \item \ce{At}

        \begin{center}
          \ce{[Xe] 6s^2 5d^10 4f^14 6p^5}
        \end{center}

    \end{enumerate}

    \setcounter{enumi}{49}

  \item Write the ground state electron configuration for:

    \begin{enumerate}

      \item \ce{F}, \ce{F-}

        \begin{center}
          \ce{F}: \ce{1s^2 2s^2 2p^5}  \\
          \ce{F-}: \ce{1s^2 2s^2 2p^6} 
        \end{center}

      \item \ce{Sc}, \ce{Sc^3+}

        \begin{center}
          \ce{Sc}: \ce{1s^2 2s^2 2p^6 3s^2 3p^6 4s^2 3d^1}  \\
          \ce{Sc^3+}: \ce{1s^2 2s^2 2p^6 3s^2 3p^6} 
        \end{center}

      \item \ce{Mn^2+}, \ce{Mn^5+}

        \begin{center}
          \ce{Mn^2+}: \ce{1s^2 2s^2 2p^6 3s^2 3p^6 4s^2 3d^5}  \\
          \ce{Mn^5+}: \ce{1s^2 2s^2 2p^6 3s^2 3p^6 4s^2 3d^2} 
        \end{center}

      \item \ce{O^-}. \ce{O^2-}

        \begin{center}
          \ce{O^-}: \ce{1s^2 2s^2 2p^5}  \\
          \ce{O^2-}: \ce{1s^2 2s^2 2p^6} 
        \end{center}

    \end{enumerate}

    \setcounter{enumi}{51}

  \item How many unpaired electrons are in the following ions:

    \begin{enumerate}

      \item \ce{Al^3+}

        \begin{center}
          0
        \end{center}

      \item \ce{Cl^-}

        \begin{center}
          0
        \end{center}

      \item \ce{Sr^2+}

        \begin{center}
          0
        \end{center}

      \item \ce{Zr^4+}

        \begin{center}
          0
        \end{center}

    \end{enumerate}

    \setcounter{enumi}{53}

  \item Arrange the elements \ce{Mg}, \ce{S}, and \ce{Cl} in order of:

    \begin{enumerate}

      \item Increasing atom radius

        \begin{center}
          \begin{tabular}{l | c c c | r}
            Least & \ce{Cl} & \ce{S} & \ce{Mg} & Greatest\\
          \end{tabular}
        \end{center}

      \item Increasing first ionization energy

        \begin{center}
          \begin{tabular}{l | c c c | r}
            Least & \ce{Mg} & \ce{S} & \ce{Cl} & Greatest\\
          \end{tabular}
        \end{center}

      \item Decreasing Electronegativity

        \begin{center}
          \begin{tabular}{l | c c c | r}
            Least & \ce{Mg} & \ce{S} & \ce{Cl} & Greatest\\
          \end{tabular}
        \end{center}

    \end{enumerate}

    \setcounter{enumi}{74}

  \item Explain why:

    \begin{enumerate}

      \item Negative ions are larger than their corresponding atoms.

        \begin{justifying}
          A negative ion means greater electrons. As such, this means that the repulsion force between the nucleus of the atom and the proton is greater, thereby increasing the distance from nucleus to orbitals.
        \end{justifying}

      \item Scandium, a transition metal, forms an ion with a noble-gas structure.

        \begin{justifying}
          This is because atoms try to take the most stable electron configuration. The noble gases have the most stable electron configurations, which means that Scandium will try to take a similar structure.
        \end{justifying}

      \item Electronegativity decreases down a group in the periodic table

        \begin{justifying}
          This is because, although it should theoretically remain the same, the repulsion forces between the nucleus and the orbitals is greater, meaning that the elements moving down are slightly less stable, and, therefore, are less able to hold an electron.
        \end{justifying}

    \end{enumerate}

\end{enumerate}

\end{document}

