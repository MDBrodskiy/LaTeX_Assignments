%%%%%%%%%%%%%%%%%%%%%%%%%%%%%%%%%%%%%%%%%%%%%%%%%%%%%%%%%%%%%%%%%%%%%%%%%%%%%%%%%%%%%%%%%%%%%%%%%%%%%%%%%%%%%%%%%%%%%%%%%%%%%%%%%%%%%%%%%%%%%%%%%%%%%%%%%%%%%%%%%%%%%%%%%%%%%%%%%%%%%%%%%%%%
% Written By Michael Brodskiy
% Class: AP Chemistry
% Professor: J. Morgan
%%%%%%%%%%%%%%%%%%%%%%%%%%%%%%%%%%%%%%%%%%%%%%%%%%%%%%%%%%%%%%%%%%%%%%%%%%%%%%%%%%%%%%%%%%%%%%%%%%%%%%%%%%%%%%%%%%%%%%%%%%%%%%%%%%%%%%%%%%%%%%%%%%%%%%%%%%%%%%%%%%%%%%%%%%%%%%%%%%%%%%%%%%%%

\documentclass[12pt]{article} 
\usepackage{alphalph}
\usepackage[utf8]{inputenc}
\usepackage[russian,english]{babel}
\usepackage{titling}
\usepackage{amsmath}
\usepackage{graphicx}
\usepackage{enumitem}
\usepackage{amssymb}
\usepackage[super]{nth}
\usepackage{expl3}
\usepackage[version=4]{mhchem}
\usepackage{hpstatement}
\usepackage{rsphrase}
\usepackage{everysel}
\usepackage{ragged2e}
\usepackage{geometry}
\usepackage{fancyhdr}
\usepackage{cancel}
\usepackage{siunitx}
\usepackage{chemfig}
\usepackage{multicol}
\geometry{top=1.0in,bottom=1.0in,left=1.0in,right=1.0in}
\newcommand{\subtitle}[1]{%
  \posttitle{%
    \par\end{center}
    \begin{center}\large#1\end{center}
    \vskip0.5em}%

}
\newcommand{\orbital}[2]{{%
    \def\+{\big|\hspace{-2pt}\overline{\underline{\hspace{2pt}\upharpoonleft}}}%
    \def\-{\overline{\underline{\downharpoonright\hspace{2pt}}}\hspace{-2pt}\big|}%
    \def\0{\big|\hspace{-2pt}\overline{\underline{\phantom{\hspace{2pt}\downharpoonright}}}}%
    \def\1{\overline{\underline{\phantom{\downharpoonright\hspace{2pt}}}}\hspace{-2pt}\big|}%
  \setlength\tabcolsep{0pt}% remove extra horizontal space from tabular
  \begin{tabular}{c}$#2$\\[2pt]#1\end{tabular}%
}}
\DeclareSIUnit\Molar{\textsc{M}}
\DeclareSIUnit\Molal{\textsc{m}}
\DeclareSIUnit\atm{\textsc{atm}}
\DeclareSIUnit\torr{\textsc{torr}}
\DeclareSIUnit\psi{\textsc{psi}}
\DeclareSIUnit\bar{\textsc{bar}}
\DeclareSIUnit\Celsius{C}
\DeclareSIUnit\degree{$^{\circ}$}
\DeclareSIUnit\calorie{cal}
\usepackage{hyperref}
\hypersetup{
colorlinks=true,
linkcolor=blue,
filecolor=magenta,      
urlcolor=blue,
citecolor=blue,
}

\urlstyle{same}


\title{Chapter 10 \& 11 $-$ Review Set}
\date{February 4, 2020}
\author{Michael Brodskiy\\ \small Instructor: Mr. Morgan}

% Mathematical Operations:

% Sum: $$\sum_{n=a}^{b} f(x) $$
% Integral: $$\int_{lower}^{upper} f(x) dx$$
% Limit: $$\lim_{x\to\infty} f(x)$$

\begin{document}

\maketitle

\begin{enumerate}

  \item If the mass percent of \ce{KOH} is 22\% and density is $1.52 \left[ \frac{\si{\gram}}{\si{\milli\liter}} \right]$, what is the molality?   

    \begin{equation}
      \begin{split}
        1000[\si{\milli\liter}]\rightarrow 1520[\si{\gram}_{total}]\\
        .22\cdot1520=334.4[\si{\gram}_{\ce{KOH}}]\\
        1520-334.4=1185.6[\si{\gram}_{\ce{H2O}}]\\
        \frac{334.4}{39+16+1}=5.97[\si{\mole}]\\
        \frac{5.97}{1.186}=5.03[\si{\Molal}]
      \end{split}
      \label{1}
    \end{equation}

  \item The molality of \ce{C12H22O} in water is $1.62\left[ \frac{\si{\mole}}{\si{\kilo\gram}} \right]$.  What is the mass percent of the solute?  

    \begin{equation}
      \begin{split}
        1.62[\si{\mole}_{\ce{C12H22O}}]\rightarrow1[\si{\kilo\gram}]\\
        1.62\cdot182=294.84[\si{\gram}]\\
        1000+294.84=294.84[\si{\gram}_{\ce{H2O}}]\\
        \frac{294.84}{1294.84}=23\%
      \end{split}
      \label{2}
    \end{equation}

  \item Calculate the mass of \ce{HCl} in $5.0[\si{\milli\liter}]$ of concentrated hydrochloric acid (density = 1.19 $\left[ \frac{\si{\gram}}{\si{\milli\liter}} \right]$) containing 37.23\% \ce{HCl} by mass.

    \begin{equation}
      \begin{split}
        5\cdot1.19=5.95\\
        5.95\cdot.3723=2.2[\si{\gram}]
      \end{split}
      \label{3}
    \end{equation}

  \item A dilute sulfuric acid solution that is $3.39[\si{\Molal}]$ \ce{H2SO4} has a density of $1.18\left[ \frac{\si{\gram}}{\si{\milli\liter}} \right]$.  How many moles of \ce{H2SO4} are there in $375[\si{\milli\liter}]$ of this solution?

    \begin{equation}
      \begin{split}
        1.18\cdot375=442.5[\si{\gram}]\\
        .4425\cdot3.39=1.5[\si{\mole}]
      \end{split}
      \label{4}
    \end{equation}

  \item Calculate the mass percent of \ce{H2SO4} in a $6.80[\si{\Molal}]$ solution.

    \begin{equation}
      \begin{split]
      6.8[\si{\mole}]\rightarrow 1[\si{\kilo\gram}]\\
      6.8\cdot98=666.4\\
      \frac{666.4}{1666.4}=40\%
    \end{split}
      \label{5}
    \end{equation}

  \item The mole fraction of \ce{C2H5OH} in a water solution is $0.0532$.  Calculate the molality of \ce{C2H5OH}.

    \begin{equation}
      \begin{split}
        \frac{.0532}{x-.0532}=.0532\\
        x=.9468\left[ \si{\mole} \right]\\
        18\cdot.9468=17.0424[\si{\gram}]\\
        \frac{.0532}{.017}=3.12[\si{\Molal}]
      \end{split}
      \label{6}
    \end{equation}

  \item How long will it take a first order substance with $k$ = $0.45\left[ \frac{1}{\si{\second}} \right]$ to be reduced to 33\% of the original concentration?

    \begin{equation}
      \begin{split}
        t=\frac{\ln(3)}{.45}\\
        =2.44[\si{\second}]
      \end{split}
      \label{7}
    \end{equation}

  \item How long will it take a first order substance with $k$ = $0.88\left[ \frac{1}{\si{\second}} \right]$ to be reduced to 1/8 of original?

    \begin{equation}
      \begin{split}
        t=\frac{\ln(8)}{.88}\\
        =2.36[\si{\second}]
      \end{split}
      \label{8}
    \end{equation}

  \item The decomposition of \ce{CO2} is second order with a rate of $0.008\left[ \frac{\si{\mole}}{\si{\liter\second}} \right]$ when the concentration is $0.12[\si{\Molar}]$.  Calculate the rate when the concentration of \ce{CO2} is $0.25[\si{\Molar}]$.

    \begin{equation}
      \begin{split}
        .008=k[.12]^2\\
        k=.555\left[ \frac{\si{\liter}}{\si{\mole\second}} \right]\\
        rate=.555\cdot[.25]^2\\
        =.0347\left[ \frac{\si{\Molar}}{\si{\second}} \right]
      \end{split}
      \label{9}
    \end{equation}

  \item Given the following data for the reaction of \ce{BF} with \ce{H2}, calculate $k$

    \begin{center}
    \begin{tabular}[H]{c c c}
      [\ce{BF}] & [\ce{H2}] & Rate \\
      0.1 & 0.1 & 0.0341 \\
      0.2 & 0.233 & 0.159 \\
      0.2 & 0.0750 & 0.0512
    \end{tabular}
  \end{center}

  \begin{equation}
    \begin{split}
    \frac{.0512}{.159}=\left( \frac{.075}{.233} \right)^m\\
    m=1\\
    \frac{.0341}{.0512}=\left( \frac{.1}{.075} \right)\left( \frac{.1}{.2} \right)^n\\
    n=1\\
    .0512=k[.075][.2]\\
    k=3.41\left[ \frac{1}{\si{\Molar\second}} \right]
  \end{split}
    \label{10}
  \end{equation}

\item Which step must be the rate-determining step in the following mechanism if the rate law is $Rate=k[\ce{H2}][\ce{NO}]^2$?

  \begin{center}
    \ce{2NO -> N2O2}\\
    \ce{N2O2 + H2 -> N2O + H2O}\\
    \ce{N2O + H2 -> N2 + H2O}
  \end{center}

  \begin{center}
    Step Two
  \end{center}

\end{enumerate}

\end{document}

