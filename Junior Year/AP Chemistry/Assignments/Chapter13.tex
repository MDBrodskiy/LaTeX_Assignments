%%%%%%%%%%%%%%%%%%%%%%%%%%%%%%%%%%%%%%%%%%%%%%%%%%%%%%%%%%%%%%%%%%%%%%%%%%%%%%%%%%%%%%%%%%%%%%%%%%%%%%%%%%%%%%%%%%%%%%%%%%%%%%%%%%%%%%%%%%%%%%%%%%%%%%%%%%%%%%%%%%%%%%%%%%%%%%%%%%%%%%%%%%%%
% Written By Michael Brodskiy
% Class: AP Chemistry
% Professor: J. Morgan
%%%%%%%%%%%%%%%%%%%%%%%%%%%%%%%%%%%%%%%%%%%%%%%%%%%%%%%%%%%%%%%%%%%%%%%%%%%%%%%%%%%%%%%%%%%%%%%%%%%%%%%%%%%%%%%%%%%%%%%%%%%%%%%%%%%%%%%%%%%%%%%%%%%%%%%%%%%%%%%%%%%%%%%%%%%%%%%%%%%%%%%%%%%%

\documentclass[12pt]{article} 
\usepackage{alphalph}
\usepackage[utf8]{inputenc}
\usepackage[russian,english]{babel}
\usepackage{titling}
\usepackage{amsmath}
\usepackage{graphicx}
\usepackage{enumitem}
\usepackage{amssymb}
\usepackage[super]{nth}
\usepackage{expl3}
\usepackage[version=4]{mhchem}
\usepackage{hpstatement}
\usepackage{rsphrase}
\usepackage{everysel}
\usepackage{ragged2e}
\usepackage{geometry}
\usepackage{fancyhdr}
\usepackage{cancel}
\usepackage{siunitx}
\usepackage{chemfig}
\usepackage{multicol}
\usepackage{xcolor}
\usepackage{array}
\usepackage{color, colortbl}
\definecolor{cadetgrey}{rgb}{0.57, 0.64, 0.69}
\geometry{top=1.0in,bottom=1.0in,left=1.0in,right=1.0in}
\newcommand{\subtitle}[1]{%
  \posttitle{%
    \par\end{center}
    \begin{center}\large#1\end{center}
    \vskip0.5em}%

}
\newcommand{\orbital}[2]{{%
    \def\+{\big|\hspace{-2pt}\overline{\underline{\hspace{2pt}\upharpoonleft}}}%
    \def\-{\overline{\underline{\downharpoonright\hspace{2pt}}}\hspace{-2pt}\big|}%
    \def\0{\big|\hspace{-2pt}\overline{\underline{\phantom{\hspace{2pt}\downharpoonright}}}}%
    \def\1{\overline{\underline{\phantom{\downharpoonright\hspace{2pt}}}}\hspace{-2pt}\big|}%
  \setlength\tabcolsep{0pt}% remove extra horizontal space from tabular
  \begin{tabular}{c}$#2$\\[2pt]#1\end{tabular}%
}}
\DeclareSIUnit\Molar{\textsc{M}}
\DeclareSIUnit\Molal{\textsc{m}}
\DeclareSIUnit\atm{\textsc{atm}}
\DeclareSIUnit\torr{\textsc{torr}}
\DeclareSIUnit\psi{\textsc{psi}}
\DeclareSIUnit\bar{\textsc{bar}}
\DeclareSIUnit\Celsius{C}
\DeclareSIUnit\degree{$^{\circ}$}
\DeclareSIUnit\calorie{cal}
\usepackage{hyperref}
\hypersetup{
colorlinks=true,
linkcolor=blue,
filecolor=magenta,      
urlcolor=blue,
citecolor=blue,
}

\urlstyle{same}


\title{Chapter 13 $-$ Problems 2, 6, 10, 16, 30}
\date{February 18, 2020}
\author{Michael Brodskiy\\ \small Instructor: Mr. Morgan}

% Mathematical Operations:

% Sum: $$\sum_{n=a}^{b} f(x) $$
% Integral: $$\int_{lower}^{upper} f(x) dx$$
% Limit: $$\lim_{x\to\infty} f(x)$$

\begin{document}

\maketitle

\begin{enumerate}

    \setcounter{enumi}{1}

      \item 

        \begin{enumerate}

          \item \ce{CN-(aq)} is the base, and the \ce{HCN(aq)} is the conjugate acid pair. \ce{H2O} is the acid, and \ce{OH-(aq)} is the conjugate base pair.

          \item \ce{HCO3-(aq)} is the base, and the \ce{H2CO3(aq)} is the conjugate acid pair. \ce{H3O+(aq)} is the acid, and \ce{H2O} is the conjugate base pair

          \item \ce{HC2H3O2(aq)} is the acid, and \ce{C2H3O2-(aq)} is the conjugate base pair. \ce{HS-(aq)} is the base, and \ce{H2S(aq)} is the conjugate acid pair.

        \end{enumerate}

    \setcounter{enumi}{5}

      \item 

        \begin{enumerate}

          \item \ce{HAsO4^2-} 

          \item \ce{Fe(H2O)4(OH)2}

          \item \ce{ClO3-}

          \item \ce{NH3}

          \item \ce{C2H3O2-}

        \end{enumerate}

    \setcounter{enumi}{9}

      \item 

        \begin{enumerate}

          \item \ce{Zn(H2O)3OH+ + H2O <=> Zn(H2O)2(OH)2 + H3O+}

          \item \ce{HSO4- + H2O <=> SO4^2- + H3O+}

          \item \ce{HNO2 + H2O <=> NO2- + H3O+}

          \item \ce{Fe(H2O)6^2+ + H2O <=> Fe(H2O)5(OH)^+ + H3O+}

          \item \ce{HC2H3O2 + H2O <=> C2H3O2- + H3O+}

          \item \ce{H2PO4- + H2O <=> HPO4^2- + H3O+}

        \end{enumerate}

    \setcounter{enumi}{15}

      \item 

        \begin{enumerate}

          \item 

            \begin{equation*}
              \begin{split}
                -0.76&=-\log_{10}\left( [\ce{H+}] \right)\\
                [\ce{H+}]&=10^{.76}\\
                &= 5.754[\si{\Molar}_{\ce{H+}}]\\
                14+0.76&=-\log_{10}\left( [\ce{OH-}] \right)\\
                [\ce{OH-}]&=10^{-14.76}\\
                &= 1.74\cdot10^{-15}[\si{\Molar}_{\ce{OH-}}]
              \end{split}
              \label{1}
            \end{equation}

          \item

            \begin{equation*}
              \begin{split}
                9.11&=-\log_{10}\left( [\ce{H+}] \right)\\
                [\ce{H+}]&=10^{-9.11}\\
                &= 7.8\cdot10^{-10}[\si{\Molar}_{\ce{H+}}]\\
                14-9.11&=-\log_{10}\left( [\ce{OH-}] \right)\\
                [\ce{OH-}]&=10^{-4.89}\\
                &= 1.3\cdot10^{-5}[\si{\Molar}_{\ce{OH-}}]
              \end{split}
              \label{2}
            \end{equation}

          \item

            \begin{equation*}
              \begin{split}
                3.81&=-\log_{10}\left( [\ce{H+}] \right)\\
                [\ce{H+}]&=10^{-3.81}\\
                &= 1.55\cdot10^{-4}[\si{\Molar}_{\ce{H+}}]\\
                14-3.81&=-\log_{10}\left( [\ce{OH-}] \right)\\
                [\ce{OH-}]&=10^{-10.19}\\
                &= 6.46\cdot10^{-11}[\si{\Molar}_{\ce{OH-}}]
              \end{split}
              \label{3}
            \end{equation}

          \item

            \begin{equation*}
              \begin{split}
                12.08&=-\log_{10}\left( [\ce{H+}] \right)\\
                [\ce{H+}]&=10^{-12.08}\\
                &= 8.3\cdot10^{-13}[\si{\Molar}_{\ce{H+}}]\\
                14-12.08&=-\log_{10}\left( [\ce{OH-}] \right)\\
                [\ce{OH-}]&=10^{-1.92}\\
                &= 1.2\cdot10^{-2}[\si{\Molar}_{\ce{OH-}}]
              \end{split}
              \label{4}
            \end{equation}

        \end{enumerate}

    \setcounter{enumi}{29}

      \item 

        \begin{equation*}
          \begin{split}
            [\ce{H+}]_{\ce{HNO3}}&=10^{-1.39}\\
            &= 4.1\cdot10^{-2}\left[ \si{\Molar}_{\ce{H+}} \right]\\
            .145\cdot.575&=8.34\cdot10^{-2}\left[ \si{\mole} \right]\\
            .493\cdot.041&=2\cdot10^{-2}\left[ \si{\mole} \right]\\
            \frac{10.34\cdot10^{-2}}{.638}&=.162[\si{\Molar}]\\
            -\log_{10}\left( .162 \right)&=.79
          \end{split}
          \label{5}
        \end{equation}

\end{enumerate}

\end{document}

