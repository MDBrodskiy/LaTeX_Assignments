\documentclass{article}
\usepackage[paperletter]{geometry}
\geometry{top=1.0in, bottom=1.0in, left=1.0in, right=1.0in}
\usepackage[utf8]{inputenc}
\usepackage{cancel}
\usepackage{tikz}
\usepackage{siunitx}
\usepackage{amsmath}
\usepackage{mathdots}
\usepackage{yhmath}
\usepackage{color}
\usepackage{array}
\usepackage{multirow}
\usepackage{amssymb}
\usepackage{gensymb}
\usepackage{tabularx}
\usepackage{booktabs}
\usepackage{expl3}
\usepackage[version=4]{mhchem}
\usepackage{hpstatement}
\usepackage{rsphrase}
\usepackage[symbol]{footmisc}
\usepackage{hyperref}
\hypersetup{
colorlinks=true,
linkcolor=blue,
filecolor=magenta,      
urlcolor=blue,
citecolor=blue,
}

\urlstyle{same}

\renewcommand*{\thefootnote}{\fnsymbol{footnote}}
%\usepackage{everysel}
%\usepackage{ragged2e}
%\renewcommand*\familydefault{\ttdefault}
%\EverySelectfont{%
%\fontdimen2\font=0.4em% interword space
%\fontdimen3\font=0.2em% interword stretch
%\fontdimen4\font=0.1em% interword shrink
%\fontdimen7\font=0.1em% extra space
%\hyphenchar\font=`\-% to allow hyphenation
%}

\title{Lab One \\ AP Chemistry}
\author{Michael \textsc{Brodskiy}}
\date{September 21, 2020}

\begin{document}

\maketitle
\begin{center}
\begin{tabular}{l r}
\underline{Date Performed}: & September 14, 2020 \\\\ % Date the experiment was performed
\underline{Partners}: & N/A \\
\underline{Instructor}: & Mr. Morgan \\\\\\\\\\ % Instructor/supervisor
\end{tabular}
\end{center}
\newpage

\tableofcontents

\newpage

\section{Pre-Lab}

\begin{itemize}

  \item This Lab Requires:

    \begin{enumerate}

      \item Question

      \item Data (including calculations)
        
      \item Conclusion

    \end{enumerate}

  \item \ce{NaHCO3 + HCl -> H2O + NaCl + CO2}

    \begin{enumerate}

      \item Calculate how much \ce{NaCl} formed

      \item Calculate Limiting Factor

      \item Determine how much \ce{NaCl} should form

      \item Calculate \% yield

    \end{enumerate}

\end{itemize}

\section{Data}

    \begin{enumerate}

      \item Mass of dish: $36.72[\si{\gram}]$

      \item Mass of \ce{NaHCO3} (Baking Soda): $2.00[\si{\gram}]$

      \item Molarity of \ce{HCl}: $6[\si{\mole\per\liter}]$

      \item Moles of \ce{HCl}: $3[\cancel{\si{\milli\liter}}]\cdot6[\si{\mole\per\liter}]\cdot\frac{1[\cancel{\si{\liter}}]}{1000[\cancel{\si{\milli\liter}}]}=.018[\si{\mole}_{\ce{HCl}}]$

      \item Mass Dish + \ce{NaCl} after procedure: $38.00[\si{\gram}]$

    \end{enumerate}

\section{Question and Statement of Purpose}

    \begin{justify}

      What is the theoretical mass of \ce{NaCl} formed vs the experimental mass? What is the percent yield?

    \end{justify}
    
\newpage

\section{Calculations}

    \begin{justify}

      The amount of \ce{NaCl} formed may be easily determined by finding the change in the mass of the dish ($\Delta m_{dish}$ \eqref{eq1}):
    \end{justify}

\begin{equation} \label{eq1}
\begin{split}
  \Delta m_{dish} & = 38.00[\si{\gram}]-36.72[\si{\gram}] \\
 & =1.28[\si{\gram}_{\ce{NaCl}}]
\end{split}
\end{equation}

    \begin{justify}

      To find the limiting factor, one must first find the mass per mole \eqref{eq2}. Then, it is necessary to find the moles of each molecule by dividing the experimental mass of the molecule by the mass per mole \eqref{eq3}. Since, in this case, the amount of moles of each substance is one, the amount with the least moles is the limiting factor \eqref{eq4}.

    \end{justify}

\begin{equation} \label{eq2}
\begin{split}
  \frac{m_{\ce{NaHCO3}}}{\si{\mole}_{\ce{NaHCO3}}} & = 23 + 1 + 12 + 3(16)  \\
  & = 84\left[ \frac{\si{\gram}}{\si{\mole}}  \right] \\
\end{split}
\end{equation}

\begin{equation} \label{eq3}
\frac{2[\si{\gram}_{\ce{NaHCO3}}]}{84[\si{\gram}_{\ce{NaHCO3}}]} = .024[\si{\mole}_{\ce{NaHCO3}}]
\end{equation}

\begin{equation} \label{eq4}
  \begin{split}
  .018[\si{\mole}_{\ce{HCl}}] < .024[\si{\mole}_{\ce{NaHCO3}}] \\
    \therefore \text{\ce{HCl} is the limiting factor}
  \end{split}
\end{equation}

\begin{justify}

  Because the limiting factor is the $.018[\si{\mole}_{\ce{HCl}}]$, and the chemical equation involves only one mole of each molecule, then $.018[\si{\mole}]$ of each molecule will be formed. To switch this value to grams, multiply the grams per mole by the amount of moles \eqref{eq5}.

\end{justify}

\begin{equation} \label{eq5}
  \begin{split}
    .018[\si{\mole}_{\ce{NaCl}}]\cdot 58\left[ \frac{\si{\gram}}{\si{\mole}} \right] = m_{\ce{NaCl}}\\
    = 1.044[\si{\gram}_{\ce{NaCl}}]
  \end{split}
\end{equation}

\begin{justify}

  Finally, to calculate the percent yield, one must divide the experimental mass by the theoretical mass, while multiplying by 100\% \eqref{eq6}.

\begin{equation} \label{eq6}
  \frac{1.28[\si{\gram}_{\ce{NaCl}}]}{1.044[\si{\gram}_{\ce{NaCl}}]} \cdot 100\% = 123\%
\end{equation}

\end{justify}

\section{Conclusion}

\begin{justify}

  The results of this laboratory experiment demonstrate our percent yield as 123\%. Therefore, it is evident that this experiment yielded a greater amount of Sodium Chloride (\ce{NaCl}) than expected, given the limiting factor of Hydrochloric Acid (\ce{HCl}). 

\end{justify}

\end{document}
