%%%%%%%%%%%%%%%%%%%%%%%%%%%%%%%%%%%%%%%%%%%%%%%%%%%%%%%%%%%%%%%%%%%%%%%%%%%%%%%%%%%%%%%%%%%%%%%%%%%%%%%%%%%%%%%%%%%%%%%%%%%%%%%%%%%%%%%%%%%%%%%%%%%%%%%%%%%%%%%%%%%%%%%%%%%%%%%%%%%%%%%%%%%%
% Written By Michael Brodskiy
% Class: AP Chemistry
% Professor: J. Morgan
%%%%%%%%%%%%%%%%%%%%%%%%%%%%%%%%%%%%%%%%%%%%%%%%%%%%%%%%%%%%%%%%%%%%%%%%%%%%%%%%%%%%%%%%%%%%%%%%%%%%%%%%%%%%%%%%%%%%%%%%%%%%%%%%%%%%%%%%%%%%%%%%%%%%%%%%%%%%%%%%%%%%%%%%%%%%%%%%%%%%%%%%%%%%

\documentclass[12pt]{article} 
\usepackage{alphalph}
\usepackage[utf8]{inputenc}
\usepackage[russian,english]{babel}
\usepackage{titling}
\usepackage{amsmath}
\usepackage{graphicx}
\usepackage{enumitem}
\usepackage{amssymb}
\usepackage[super]{nth}
\usepackage{expl3}
\usepackage[version=4]{mhchem}
\usepackage{hpstatement}
\usepackage{rsphrase}
\usepackage{everysel}
\usepackage{ragged2e}
\usepackage{geometry}
\usepackage{fancyhdr}
\usepackage{cancel}
\usepackage{siunitx}
\usepackage{chemfig}
\geometry{top=1.0in,bottom=1.0in,left=1.0in,right=1.0in}
\newcommand{\subtitle}[1]{%
  \posttitle{%
    \par\end{center}
    \begin{center}\large#1\end{center}
    \vskip0.5em}%

}
\newcommand{\orbital}[2]{{%
    \def\+{\big|\hspace{-2pt}\overline{\underline{\hspace{2pt}\upharpoonleft}}}%
    \def\-{\overline{\underline{\downharpoonright\hspace{2pt}}}\hspace{-2pt}\big|}%
    \def\0{\big|\hspace{-2pt}\overline{\underline{\phantom{\hspace{2pt}\downharpoonright}}}}%
    \def\1{\overline{\underline{\phantom{\downharpoonright\hspace{2pt}}}}\hspace{-2pt}\big|}%
  \setlength\tabcolsep{0pt}% remove extra horizontal space from tabular
  \begin{tabular}{c}$#2$\\[2pt]#1\end{tabular}%
}}
\DeclareSIUnit\Molar{\textsc{m}}
\DeclareSIUnit\atm{\textsc{atm}}
\DeclareSIUnit\torr{\textsc{torr}}
\DeclareSIUnit\psi{\textsc{psi}}
\DeclareSIUnit\bar{\textsc{bar}}
\usepackage{hyperref}
\hypersetup{
colorlinks=true,
linkcolor=blue,
filecolor=magenta,      
urlcolor=blue,
citecolor=blue,
}

\urlstyle{same}


\title{Chapter 7 $-$ Covalent Bonds}
\date{\today}
\author{Michael Brodskiy\\ \small Instructor: Mr. Morgan}

% Mathematical Operations:

% Sum: $$\sum_{n=a}^{b} f(x) $$
% Integral: $$\int_{lower}^{upper} f(x) dx$$
% Limit: $$\lim_{x\to\infty} f(x)$$

\begin{document}

\maketitle

\begin{itemize}

  \item Lewis Structures:

    \begin{enumerate}

      \item Sum Valence Electrons

      \item Connect atoms with lines

      \item Arrange remaining \ce{e-} to satisfy octet rule

    \end{enumerate}

  \item Resonance $-$ Being able to draw more than one Lewis structure

  \item Halogens will almost never have double bonds

  \item Molecular Shapes:

    \begin{enumerate}

      \item Linear

        \begin{itemize}

          \item Bond angle equals $180^{\circ}$

          \item Usually non-polar

          \item Ex:

            \chemfig{\lewis{2:6:,O}=C=\lewis{2:6:,O}}

        \end{itemize}

      \item Triangular Planar

        \begin{itemize}

          \item Bond angle equals $120^{\circ}$

          \item Usually non-polar

          \item Ex:

            \vspace{5pt}
            \chemfig{B(=[6]F)(-[:-210]F)(-[:-330]F)}

        \end{itemize}

      \item Bent

        \begin{itemize}

          \item Bond angle equals $109.5^{\circ}$

          \item Appears as a linear, but bent

          \item Always polar

          \item Ex:

            \chemfig{\lewis{2:6:,O}(-[:-144.5]H)(-[:-35]H)}

        \end{itemize}

      \item Tri-Pyramid

        \begin{itemize}

          \item Bond angle equals $109.5^{\circ}$

          \item Appears as a three dimensional, triangular pyramid

          \item Always polar

          \item Ex:

            \chemfig{\lewis{2:,N}(-[6]H)(-[:-10]H)(-[:-170]H)}

        \end{itemize}

      \item Tetrahedral

        \begin{itemize}

          \item Bond angle equals $109.5^{\circ}$

          \item Appears as a three dimensional, square pyramid

          \item Usually non-polar

          \item Ex:

            \chemfig{C(-[0]H)(-[2]H)(-[4]H)(-[6]H)}

        \end{itemize}

    \end{enumerate}

  \item Expanded Octets: \ce{5e-} pairs

    \begin{enumerate}

      \item Tri-Bipyramid (5 atoms)

        \begin{enumerate}

          \item Usually non-polar

        \end{enumerate}

      \item Seesaw (4 atoms) 

        \begin{enumerate}

          \item Always polar

        \end{enumerate}

      \item T Shape (3 atoms)

        \begin{enumerate}

          \item Always polar

        \end{enumerate}

      \item Linear (2 atoms)

        \begin{enumerate}

          \item Usually non-polar

        \end{enumerate}

    \end{enumerate}

  \item Expanded Octets: \ce{6e-} pairs

    \begin{itemize}

      \item Octahedral (6 atoms)

        \begin{enumerate}

          \item Usually non-polar

        \end{enumerate}

      \item Square Pyramid (5 atoms)

        \begin{enumerate}

          \item Always polar

        \end{enumerate}

      \item Square Planar (4 atoms)

        \begin{enumerate}

          \item Usually non-polar

        \end{enumerate}

    \end{itemize}

  \item VESPR $-$ \ce{e-} pairs repel, and double and triple bonds count as one pair.

  \item Molecular Polarity $-$ The negative side is what repels branching atoms, while the branching atoms are attracted to the positive side.

  \item Exceptions $-$ \ce{NF3} has a bond angle of $102^{\circ}$, \ce{CF4} has a bond angle of $109.5^{\circ}$. This is because lone pairs repel more than bonding pairs.

  \item These types of graphs show the appropriate bond distance between two atoms:

    \begin{figure}[h]
      \centering
      \includegraphics[width=.7\textwidth]{Figures/HappyGraph.png}
      \caption{Best Bonding Distance For \ce{H2}}
      \label{fig:1}
    \end{figure}

    \newpage
  \item The AP Exam will show a graph like the following and ask what is incorrect. Horizontal is determined by protons, while vertical is determined by size of the atom. In this case, \ce{F2} should be left of \ce{I2}

    \begin{figure}[h]
      \centering
      \includegraphics[width=.7\textwidth]{Figures/I2F2.png}
      \caption{Incorrect Energy Required to Break \ce{I2} and \ce{F2} Bonds}
      \label{fig:2}
    \end{figure}

  \item Hybridization of Orbitals $-$ When an atom should be able to bond an $n$ amount of times, but, according to the box diagram, it can not, electrons are moved around to fit the expected number of bonds. (Ex. \ce{sp^3} for \ce{C}, \ce{sp^2} for \ce{B}, or \ce{sp} for \ce{Be})

  \item \textit{Example:}

    \begin{center}
      Carbon:\\
    \orbital{s}{\+\1} \quad \orbital{p}{\+\1 \+\1 \+\1}\\
      Instead of:\\
    \orbital{s}{\+\-} \quad \orbital{p}{\+\1 \+\1 \0\1}
    \end{center}

\newpage

  \item Table for hybridization:

    \begin{tabular}{c | c}
      \underline{# \ce{e-} pairs} & \underline{hybrid}\\
      2 & \ce{sp}\\
      3 & \ce{sp^2}\\
      4 & \ce{sp^3}\\
      5 & \ce{sp^3d}\\
      6 & \ce{sp^3d^2}\\
    \end{tabular}

  \item Sigma and Pi Bonds $-$ Sigma ($\sigma$)-bond on axis, while Pi ($\pi$)-bond not on axis

  \item Coulomb's Law: The force between two charged particles is proportional to the magnitude of each of the two charges, and inversely proportional to the square of the distance between them (further apart, weaker force) ($\frac{kq_1q_2}{r^2}$)

  \item Electron Shielding $-$ Core electrons are generally closer to the nucleus than calence electrons and are considered to ``shield'' the valence electrons from the full electrostatic attraction of the nucleus

  \item Lattice Energy $-$ Energy of an ionic structure

  \item \textit{Example: } Which has more lattice energy, \ce{ZnCl2} or \ce{ZnO}?

    \begin{center}
      \ce{ZnO} has more lattice energy than \ce{ZnCl2} because oxygen has a larger negative charge
    \end{center}

\end{itemize}

\end{document}

