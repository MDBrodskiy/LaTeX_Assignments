%%%%%%%%%%%%%%%%%%%%%%%%%%%%%%%%%%%%%%%%%%%%%%%%%%%%%%%%%%%%%%%%%%%%%%%%%%%%%%%%%%%%%%%%%%%%%%%%%%%%%%%%%%%%%%%%%%%%%%%%%%%%%%%%%%%%%%%%%%%%%%%%%%%%%%%%%%%%%%%%%%%%%%%%%%%%%%%%%%%%%%%%%%%%
% Written By Michael Brodskiy
% Class: AP Chemistry
% Professor: J. Morgan
%%%%%%%%%%%%%%%%%%%%%%%%%%%%%%%%%%%%%%%%%%%%%%%%%%%%%%%%%%%%%%%%%%%%%%%%%%%%%%%%%%%%%%%%%%%%%%%%%%%%%%%%%%%%%%%%%%%%%%%%%%%%%%%%%%%%%%%%%%%%%%%%%%%%%%%%%%%%%%%%%%%%%%%%%%%%%%%%%%%%%%%%%%%%

\documentclass[12pt]{article} 
\usepackage{alphalph}
\usepackage[utf8]{inputenc}
\usepackage[russian,english]{babel}
\usepackage{titling}
\usepackage{amsmath}
\usepackage{graphicx}
\usepackage{enumitem}
\usepackage{amssymb}
\usepackage[super]{nth}
\usepackage{everysel}
\usepackage{ragged2e}
\usepackage{geometry}
\usepackage{fancyhdr}
\usepackage{cancel}
\usepackage{siunitx}
\geometry{top=1.0in,bottom=1.0in,left=1.0in,right=1.0in}
\newcommand{\subtitle}[1]{%
  \posttitle{%
    \par\end{center}
    \begin{center}\large#1\end{center}
    \vskip0.5em}%

}
\usepackage{hyperref}
\hypersetup{
colorlinks=true,
linkcolor=blue,
filecolor=magenta,      
urlcolor=blue,
citecolor=blue,
}

\urlstyle{same}


\title{Chapter 3 $-$ Mass Relationships}
\date{\today}
\author{Michael Brodskiy\\ \small Instructor: Mr. Morgan}

% Mathematical Operations:

% Sum: $$\sum_{n=a}^{b} f(x) $$
% Integral: $$\int_{lower}^{upper} f(x) dx$$
% Limit: $$\lim_{x\to\infty} f(x)$$

\begin{document}

\maketitle

\begin{itemize}

  \item Molar Mass

    \begin{enumerate}

      \item $6.022\cdot10^23$ is one mole (Avogadro's number)

      \item Obtained by adding the atomic mass of each element present

      \item ex. C = $12[\si{\gram}]/\si{\mole}$

    \end{enumerate}

  \item Molarity (M)

    \begin{enumerate}

      \item $\si{\mole}/\si{\liter}$

    \end{enumerate}

  \item Molar Ratio

    \begin{enumerate}

      \item Use $C_12H_22O_11$ for example:

      \item Ratio for Carbon: $\frac{12\si{\mole}_C}{\si{\mole}_{C_{12}H_{22}O_{11}}}$

      \item Ratio for Hydrogen: $\frac{22\si{\mole}_H}{\si{\mole}_{C_{12}H_{22}O_{11}}}$

      \item Ratio for Oxygen: $\frac{11\si{\mole}_C}{\si{\mole}_{C_{12}H_{22}O_{11}}}$

    \end{enumerate}

  \item Mass Percent

    \begin{enumerate}

      \item Mass of Element per Mass of Compound times 100 ($\frac{m_e}{m_c}\cdot100$)

    \end{enumerate}

  \item Chemical Formulas:

    \begin{enumerate}

      \item Empirical Formula $-$ The simplest form, only gives the ratios of atoms

      \item Molecular Formula $-$ The actual formula, gives the exact ratio of atoms (sometimes Molecular can be Empirical, but usually not)

    \end{enumerate}

  \item Calculating the Empirical Formula:

    \begin{enumerate}

      \item Convert to Moles

      \item Divide all by the smallest

      \item Multiply by integer

    \end{enumerate}

  \item Combustion Reactions:

    \begin{enumerate}

      \item Always involves a hydrocarbon (anything involving $CH$)

      \item Ex (simplest reaction):

      $$CH+O_2\rightarrow H_2O +CO_2$$

    \end{enumerate}

  \item Balancing Equations:

    \begin{enumerate}

      \item $N_2H_4+N_2O_4\rightarrow N_2+H_2O$

        $$N_2H_4+N_2O_4\rightarrow N_2+4H_2O$$
        $$2N_2H_4+N_2O_4\rightarrow N_2+4H_2O$$
        $$2N_2H_4+N_2O_4\rightarrow 3N_2+4(H_2O)$$

      \item $NH_3+O_2\rightarrow NO +H_2O$
       $$2NH_3 + O_2\rightarrow NO + 3H_2O$$
       $$4NH_3 + 5O_2\rightarrow 4NO + 6H_2O$$

    \end{enumerate}

  \item What is the Difference? One is three unbound nitrate molecules, while the other is three chemically bound nitrate molecules

    $$3NO_3\text{ vs }(NO_3)_3$$

  \item Stoichiometry $-$ Calculations with balanced equations

  \item Combust $C_3H_8$:

    \begin{enumerate}

      \item $C_3H_8 + 5O_2\rightarrow 3CO_2 + 4H_2O$

      \item Start with $17.8[\si{\gram}]$ of $O_2$, how much $H_2O$ do we have at the end?

        $$\frac{17.8}{32}=.556[\si{\mole}]\cdot\frac{4}{5}=.445[\si{\mole}]\cdot18\frac{[\si{\gram}]}{[\si{\mole}]}=8[\si{\gram}]\,\,H_2O$$

    \end{enumerate}

\end{itemize}

\end{document}

