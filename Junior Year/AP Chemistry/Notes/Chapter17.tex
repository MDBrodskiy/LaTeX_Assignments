%%%%%%%%%%%%%%%%%%%%%%%%%%%%%%%%%%%%%%%%%%%%%%%%%%%%%%%%%%%%%%%%%%%%%%%%%%%%%%%%%%%%%%%%%%%%%%%%%%%%%%%%%%%%%%%%%%%%%%%%%%%%%%%%%%%%%%%%%%%%%%%%%%%%%%%%%%%%%%%%%%%%%%%%%%%%%%%%%%%%%%%%%%%%
% Written By Michael Brodskiy
% Class: AP Chemistry
% Professor: J. Morgan
%%%%%%%%%%%%%%%%%%%%%%%%%%%%%%%%%%%%%%%%%%%%%%%%%%%%%%%%%%%%%%%%%%%%%%%%%%%%%%%%%%%%%%%%%%%%%%%%%%%%%%%%%%%%%%%%%%%%%%%%%%%%%%%%%%%%%%%%%%%%%%%%%%%%%%%%%%%%%%%%%%%%%%%%%%%%%%%%%%%%%%%%%%%%

\documentclass[12pt]{article} 
\usepackage{alphalph}
\usepackage[utf8]{inputenc}
\usepackage[russian,english]{babel}
\usepackage{titling}
\usepackage{amsmath}
\usepackage{graphicx}
\usepackage{enumitem}
\usepackage{amssymb}
\usepackage[super]{nth}
\usepackage{expl3}
\usepackage[version=4]{mhchem}
\usepackage{hpstatement}
\usepackage{rsphrase}
\usepackage{everysel}
\usepackage{ragged2e}
\usepackage{geometry}
\usepackage{fancyhdr}
\usepackage{cancel}
\usepackage{siunitx}
\usepackage{chemfig}
\geometry{top=1.0in,bottom=1.0in,left=1.0in,right=1.0in}
\newcommand{\subtitle}[1]{%
  \posttitle{%
    \par\end{center}
    \begin{center}\large#1\end{center}
    \vskip0.5em}%

}
\newcommand{\orbital}[2]{{%
    \def\+{\big|\hspace{-2pt}\overline{\underline{\hspace{2pt}\upharpoonleft}}}%
    \def\-{\overline{\underline{\downharpoonright\hspace{2pt}}}\hspace{-2pt}\big|}%
    \def\0{\big|\hspace{-2pt}\overline{\underline{\phantom{\hspace{2pt}\downharpoonright}}}}%
    \def\1{\overline{\underline{\phantom{\downharpoonright\hspace{2pt}}}}\hspace{-2pt}\big|}%
  \setlength\tabcolsep{0pt}% remove extra horizontal space from tabular
  \begin{tabular}{c}$#2$\\[2pt]#1\end{tabular}%
}}
\DeclareSIUnit\Molar{\textsc{m}}
\DeclareSIUnit\atm{\textsc{atm}}
\DeclareSIUnit\torr{\textsc{torr}}
\DeclareSIUnit\psi{\textsc{psi}}
\DeclareSIUnit\bar{\textsc{bar}}
\DeclareSIUnit\Celsius{\textsc{C}}
\usepackage{hyperref}
\hypersetup{
colorlinks=true,
linkcolor=blue,
filecolor=magenta,      
urlcolor=blue,
citecolor=blue,
}

\urlstyle{same}


\title{Chapter 17 $-$ Electrochemistry}
\date{\today}
\author{Michael Brodskiy\\ \small Instructor: Mr. Morgan}

% Mathematical Operations:

% Sum: $$\sum_{n=a}^{b} f(x) $$
% Integral: $$\int_{lower}^{upper} f(x) dx$$
% Limit: $$\lim_{x\to\infty} f(x)$$

\begin{document}

\maketitle

\begin{itemize}

  \item Electrochemistry — The transfer of electrons (oxidation-reduction). Separate the oxidation from reduction and get flow of electrons.

  \item Oxidation — Loss of electrons (e.g. \ce{Zn -> Zn^2+ + 2e-}), called the anode

  \item Reduction — Gain of electrons (e.g. \ce{Cu^2+ + 2e- -> Cu}), called the cathode

  \item Electrode loses mass, while plating gains mass

  \item Salt Bridge — Allows ions to flow to balance charge

  \item Standard Voltages — Measurement of cell voltage

  \item $E^0=E^0(\text{reduction})+E^0(\text{oxidation})$. If $E^0$ is positive, the reaction is spontaneous

  \item Best oxidizing agents (get reduced the most) are at the bottom left of the given chart

  \item Best reducing agents (get oxidized the most) are at the bottom right of the given chart

  \item $\Delta G=-n\mathcal{F}E^0$, where $n$ is the amount of electrons transferred, and $\mathcal{F}=9.648\cdot10^4\left[ \frac{\si{\joule}}{\si{\mole\volt}} \right]$ is Faraday's constant

  \item $E^0=\frac{RT}{n\mathcal{F}}\ln(k)$ 
    
  \item $E=E^0 -\frac{.0257V}{n}\ln(Q)$; Cell voltage and concentration. Known as the Nernst Equation.

    \begin{enumerate}

      \item If $Q>1$, then $E$ is less spontaneous

      \item If $Q<1$, then $E$ is more spontaneous

    \end{enumerate}

  \item Electrolytic Cell — Non-spontaneous that needs electrical energy input

  \item Units:

    \begin{enumerate}

      \item Charge: Coulomb ($\si{\coulomb}$)

      \item Current: Ampere ($\si{\ampere}$)

      \item $1[\si{\ampere}]=1\left[ \frac{\si{\coulomb}}{\si{\second}} \right]$

      \item 1 mole of electrons: $9.648\cdot10^{4}[\si{\coulomb}]$

    \end{enumerate}

\end{itemize}

\end{document}

