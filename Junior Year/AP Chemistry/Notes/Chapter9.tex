%%%%%%%%%%%%%%%%%%%%%%%%%%%%%%%%%%%%%%%%%%%%%%%%%%%%%%%%%%%%%%%%%%%%%%%%%%%%%%%%%%%%%%%%%%%%%%%%%%%%%%%%%%%%%%%%%%%%%%%%%%%%%%%%%%%%%%%%%%%%%%%%%%%%%%%%%%%%%%%%%%%%%%%%%%%%%%%%%%%%%%%%%%%%
% Written By Michael Brodskiy
% Class: AP Chemistry
% Professor: J. Morgan
%%%%%%%%%%%%%%%%%%%%%%%%%%%%%%%%%%%%%%%%%%%%%%%%%%%%%%%%%%%%%%%%%%%%%%%%%%%%%%%%%%%%%%%%%%%%%%%%%%%%%%%%%%%%%%%%%%%%%%%%%%%%%%%%%%%%%%%%%%%%%%%%%%%%%%%%%%%%%%%%%%%%%%%%%%%%%%%%%%%%%%%%%%%%

\documentclass[12pt]{article} 
\usepackage{alphalph}
\usepackage[utf8]{inputenc}
\usepackage[russian,english]{babel}
\usepackage{titling}
\usepackage{amsmath}
\usepackage{graphicx}
\usepackage{enumitem}
\usepackage{amssymb}
\usepackage[super]{nth}
\usepackage{expl3}
\usepackage[version=4]{mhchem}
\usepackage{hpstatement}
\usepackage{rsphrase}
\usepackage{everysel}
\usepackage{ragged2e}
\usepackage{geometry}
\usepackage{fancyhdr}
\usepackage{cancel}
\usepackage{siunitx}
\usepackage{chemfig}
\geometry{top=1.0in,bottom=1.0in,left=1.0in,right=1.0in}
\newcommand{\subtitle}[1]{%
  \posttitle{%
    \par\end{center}
    \begin{center}\large#1\end{center}
    \vskip0.5em}%

}
\newcommand{\orbital}[2]{{%
    \def\+{\big|\hspace{-2pt}\overline{\underline{\hspace{2pt}\upharpoonleft}}}%
    \def\-{\overline{\underline{\downharpoonright\hspace{2pt}}}\hspace{-2pt}\big|}%
    \def\0{\big|\hspace{-2pt}\overline{\underline{\phantom{\hspace{2pt}\downharpoonright}}}}%
    \def\1{\overline{\underline{\phantom{\downharpoonright\hspace{2pt}}}}\hspace{-2pt}\big|}%
  \setlength\tabcolsep{0pt}% remove extra horizontal space from tabular
  \begin{tabular}{c}$#2$\\[2pt]#1\end{tabular}%
}}
\DeclareSIUnit\Molar{\textsc{m}}
\DeclareSIUnit\atm{\textsc{atm}}
\DeclareSIUnit\torr{\textsc{torr}}
\DeclareSIUnit\psi{\textsc{psi}}
\DeclareSIUnit\bar{\textsc{bar}}
\usepackage{hyperref}
\hypersetup{
colorlinks=true,
linkcolor=blue,
filecolor=magenta,      
urlcolor=blue,
citecolor=blue,
}

\urlstyle{same}


\title{Chapter 9 $-$ Liquids and Solids}
\date{\today}
\author{Michael Brodskiy\\ \small Instructor: Mr. Morgan}

% Mathematical Operations:

% Sum: $$\sum_{n=a}^{b} f(x) $$
% Integral: $$\int_{lower}^{upper} f(x) dx$$
% Limit: $$\lim_{x\to\infty} f(x)$$

\begin{document}

\maketitle

\begin{itemize}

  \item Evaporation $-$ When molecules escape the surface of a liquid

  \item What happens in a closed container:

    \begin{enumerate}

      \item Evaporation

      \item Condensation

    \end{enumerate}

  \item When the rate of evaporation equals the rate of condensation, equilibrium is reached.

  \item Vapor Pressure $-$ Pressure at equilibrium of a liquid, specific to a liquid, which is the max amount of molecules a vapor can hold. If there are not enough molecules, all are in vapor. If there are too many, liquid and vapor is mixed.

\item At High Vapor Pressure $-$ Weak forces, which means a lot of gas molecules, which means it is volatile (evaporates quickly)

\item At Low Vapor Pressure $-$ Forces are strong, resulting in few gas molecules, which means it is nonvolatile (evaporates slowly)

\item As temperature goes up, vapor pressure goes up

\item Boiling Point $-$ A liquid boils when it reaches the temperature at which the vapor pressure is equal to the pressure above it

\item Decreasing external pressure causes decrease in boiling point (don't cook pasta at Tahoe)

\item Critical Temperature $-$ A temperature above which the liquid phase can not exist

\item Critical Pressure $-$ The pressure that must be applied to cause condensation at the critical temperature

\item A phase diagram looks as follows:

  \begin{figure}[H]
    \centering
    \includegraphics{Figures/PhaseDiagram.png}
    \caption{Phase Diagram Example}
    \label{fig:1}
  \end{figure}

\item On the line between gas and solid is sublimation

\item Point in the middle is the triple point

\item Between gas and liquid is boiling point line

\item Melting/Freezing line is between solid and liquid

\end{itemize}


\end{document}

