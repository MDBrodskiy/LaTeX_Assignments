%%%%%%%%%%%%%%%%%%%%%%%%%%%%%%%%%%%%%%%%%%%%%%%%%%%%%%%%%%%%%%%%%%%%%%%%%%%%%%%%%%%%%%%%%%%%%%%%%%%%%%%%%%%%%%%%%%%%%%%%%%%%%%%%%%%%%%%%%%%%%%%%%%%%%%%%%%%%%%%%%%%%%%%%%%%%%%%%%%%%%%%%%%%%
% Written By Michael Brodskiy
% Class: AP Chemistry
% Professor: J. Morgan
%%%%%%%%%%%%%%%%%%%%%%%%%%%%%%%%%%%%%%%%%%%%%%%%%%%%%%%%%%%%%%%%%%%%%%%%%%%%%%%%%%%%%%%%%%%%%%%%%%%%%%%%%%%%%%%%%%%%%%%%%%%%%%%%%%%%%%%%%%%%%%%%%%%%%%%%%%%%%%%%%%%%%%%%%%%%%%%%%%%%%%%%%%%%

\documentclass[12pt]{article} 
\usepackage{alphalph}
\usepackage[utf8]{inputenc}
\usepackage[russian,english]{babel}
\usepackage{titling}
\usepackage{amsmath}
\usepackage{graphicx}
\usepackage{enumitem}
\usepackage{amssymb}
\usepackage[super]{nth}
\usepackage{expl3}
\usepackage[version=4]{mhchem}
\usepackage{hpstatement}
\usepackage{rsphrase}
\usepackage{everysel}
\usepackage{ragged2e}
\usepackage{geometry}
\usepackage{fancyhdr}
\usepackage{cancel}
\usepackage{siunitx}
\usepackage{chemfig}
\geometry{top=1.0in,bottom=1.0in,left=1.0in,right=1.0in}
\newcommand{\subtitle}[1]{%
  \posttitle{%
    \par\end{center}
    \begin{center}\large#1\end{center}
    \vskip0.5em}%

}
\newcommand{\orbital}[2]{{%
    \def\+{\big|\hspace{-2pt}\overline{\underline{\hspace{2pt}\upharpoonleft}}}%
    \def\-{\overline{\underline{\downharpoonright\hspace{2pt}}}\hspace{-2pt}\big|}%
    \def\0{\big|\hspace{-2pt}\overline{\underline{\phantom{\hspace{2pt}\downharpoonright}}}}%
    \def\1{\overline{\underline{\phantom{\downharpoonright\hspace{2pt}}}}\hspace{-2pt}\big|}%
  \setlength\tabcolsep{0pt}% remove extra horizontal space from tabular
  \begin{tabular}{c}$#2$\\[2pt]#1\end{tabular}%
}}
\DeclareSIUnit\Molar{\textsc{m}}
\DeclareSIUnit\atm{\textsc{atm}}
\DeclareSIUnit\torr{\textsc{torr}}
\DeclareSIUnit\psi{\textsc{psi}}
\DeclareSIUnit\bar{\textsc{bar}}
\DeclareSIUnit\Celsius{\textsc{C}}
\usepackage{hyperref}
\hypersetup{
colorlinks=true,
linkcolor=blue,
filecolor=magenta,      
urlcolor=blue,
citecolor=blue,
}

\urlstyle{same}


\title{Lab Questions}
\date{\today}
\author{Michael Brodskiy\\ \small Instructor: Mr. Morgan}

% Mathematical Operations:

% Sum: $$\sum_{n=a}^{b} f(x) $$
% Integral: $$\int_{lower}^{upper} f(x) dx$$
% Limit: $$\lim_{x\to\infty} f(x)$$

\begin{document}

\maketitle

\begin{itemize}

  \item When the dish is heated and cooled repeatedly, it is done to burn off water

  \item Safety: When diluting acids, always add acid to water

    \begin{enumerate}

      \item Spills: Acid/weak base, Base/weak acid

    \end{enumerate}

  \item Accuracy: When titrating, rinse the buret with the solution being used

  \item Allow hot objects to return to room temperature because hot objects weight less

  \item Accuracy vs Precision — How close to value vs how consistent the results are

  \item Weight and Reweigh

  \item Know separation techniques

  \item Spectrophotometer — Measures concentrations of solution by measuring slight variations in color. The concentration of an ion will be directly proportional to the absorbance

  \item Different colors require a different wavelength setting

  \item Beer's Law — This relationship between absorbance and concentration is given by $A=abc$, where $A$ is absorbance, $a$ is a constant, $b$ is path length of the light, and $c$ is the concentration

  \item Solutions have to be colored (using the cuvet — make sure to clean off fingerprints)

  \item Flame test: \ce{Li} = red, \ce{Na} = yellow, \ce{K} = purple, \ce{Ba} = green, \ce{Sr} = red, \ce{Ca} = red, and \ce{Cu} = green

  \item Colored Solutions: \ce{Cu} = blue, \ce{CrO4} = yellow, \ce{Ni} = green, \ce{Cr2O7} = orange, \ce{I} = brown, \ce{PbI2} = yellow, and \ce{MnO4} = purple

  \item Percent Recovery — Found value divided by actual value, times one hunder

  \item Percent Error — $100-$ Percent Recovery

  \item Photoelectron Spectroscopy (PES) — Use X-Ray to remove “core” baby electron from an atom. Measure the kinetic energy of the electron coming off (binding energy). Understand the structure of atoms and electron structure.

  \item PES Spectrum shows Intensity — Number of Electrons vs. Binding Energy. Smaller shells require more energy to remove (1s requires the most)

  \item Mixture of Elements — The peak height is not related to the number of electrons in a PES spectrum

  \item Atoms will not lose electrons in the 2p orbital and beyond

  \item Mass Spectrometer — Finds relative masses of individual atoms

  \item Happy Graphs — Energy vs. Distance

  \item Calculate Energy of a Wavelength of light:

    \begin{enumerate}

      \item $c=f\lambda$

      \item $E=hf$

      \item $E=\frac{hc}{\lambda}$
        
    \end{enumerate}

  \item Photons with lower wavelengths have a higher energy. Higher frequency means higher energy.

\end{itemize}

\end{document}

