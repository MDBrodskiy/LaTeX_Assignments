%%%%%%%%%%%%%%%%%%%%%%%%%%%%%%%%%%%%%%%%%%%%%%%%%%%%%%%%%%%%%%%%%%%%%%%%%%%%%%%%%%%%%%%%%%%%%%%%%%%%%%%%%%%%%%%%%%%%%%%%%%%%%%%%%%%%%%%%%%%%%%%%%%%%%%%%%%%%%%%%%%%%%%%%%%%%%%%%%%%%%%%%%%%%
% Written By Michael Brodskiy
% Class: AP Chemistry
% Professor: J. Morgan
%%%%%%%%%%%%%%%%%%%%%%%%%%%%%%%%%%%%%%%%%%%%%%%%%%%%%%%%%%%%%%%%%%%%%%%%%%%%%%%%%%%%%%%%%%%%%%%%%%%%%%%%%%%%%%%%%%%%%%%%%%%%%%%%%%%%%%%%%%%%%%%%%%%%%%%%%%%%%%%%%%%%%%%%%%%%%%%%%%%%%%%%%%%%

\documentclass[12pt]{article} 
\usepackage{alphalph}
\usepackage[utf8]{inputenc}
\usepackage[russian,english]{babel}
\usepackage{titling}
\usepackage{amsmath}
\usepackage{graphicx}
\usepackage{enumitem}
\usepackage{amssymb}
\usepackage[super]{nth}
\usepackage{expl3}
\usepackage[version=4]{mhchem}
\usepackage{hpstatement}
\usepackage{rsphrase}
\usepackage{everysel}
\usepackage{ragged2e}
\usepackage{geometry}
\usepackage{fancyhdr}
\usepackage{cancel}
\usepackage{siunitx}
\usepackage{chemfig}
\geometry{top=1.0in,bottom=1.0in,left=1.0in,right=1.0in}
\newcommand{\subtitle}[1]{%
  \posttitle{%
    \par\end{center}
    \begin{center}\large#1\end{center}
    \vskip0.5em}%

}
\newcommand{\orbital}[2]{{%
    \def\+{\big|\hspace{-2pt}\overline{\underline{\hspace{2pt}\upharpoonleft}}}%
    \def\-{\overline{\underline{\downharpoonright\hspace{2pt}}}\hspace{-2pt}\big|}%
    \def\0{\big|\hspace{-2pt}\overline{\underline{\phantom{\hspace{2pt}\downharpoonright}}}}%
    \def\1{\overline{\underline{\phantom{\downharpoonright\hspace{2pt}}}}\hspace{-2pt}\big|}%
  \setlength\tabcolsep{0pt}% remove extra horizontal space from tabular
  \begin{tabular}{c}$#2$\\[2pt]#1\end{tabular}%
}}
\DeclareSIUnit\Molar{\textsc{m}}
\DeclareSIUnit\atm{\textsc{atm}}
\DeclareSIUnit\torr{\textsc{torr}}
\DeclareSIUnit\psi{\textsc{psi}}
\DeclareSIUnit\bar{\textsc{bar}}
\DeclareSIUnit\Celsius{\textsc{C}}
\usepackage{hyperref}
\hypersetup{
colorlinks=true,
linkcolor=blue,
filecolor=magenta,      
urlcolor=blue,
citecolor=blue,
}

\urlstyle{same}


\title{Chapter 12 $-$ Equilibrium}
\date{\today}
\author{Michael Brodskiy\\ \small Instructor: Mr. Morgan}

% Mathematical Operations:

% Sum: $$\sum_{n=a}^{b} f(x) $$
% Integral: $$\int_{lower}^{upper} f(x) dx$$
% Limit: $$\lim_{x\to\infty} f(x)$$

\begin{document}

\maketitle

\begin{itemize}

  \item Products can reverse into reactants:

    \begin{enumerate}

      \item $
        \begin{array}{l r}
          \ce{2NO2 -> N2O4} & (forward)\\
          \ce{2NO2 <- N2O4} & (reverse)\\
        \end{array}$

    \end{enumerate}

  \item Reaction rate is dependent on concentration (high concentration, fast rate; low concentration, slow rate)

  \item Forward rate vs. Reverse rate:

    \begin{enumerate}

      \item Once equilibrium is reached, the concentrations stay the same, even though reactions continue (forward rate = reverse rate)

    \end{enumerate}

  \item Equilibrium Expressions (Only for gases and aqueous solutions)

    \begin{enumerate}

      \item Ex. \ce{N2 + 3H2 <=> 2NH3}, where $k_c$ is a concentration constant, and $k_p$ is a pressure constant

        \begin{equation}
          \begin{split}
            k_c = \frac{[\ce{NH3}]^2}{[\ce{N2}][\ce{H2}]^3}\\
          k_p = \frac{P_{\ce{NH3}}}{P_{\ce{N2}}\cdot P_{\ce{H2}}}\\
          k_p=k_c\left( RT \right)^{\Delta n}
        \end{split}
          \label{1}
        \end{equation}

    \end{enumerate}

  \item If coefficients are multiplied by $n$, then:

    \begin{equation}
      k=k^n
      \label{2}
    \end{equation}

  \item If the reaction is reversed, then:

    \begin{equation}
      k=\frac{1}{k}
      \label{3}
    \end{equation}

  \item If the reaction occurs in multiple steps, multiply the individual $k$ values by each other

  \item Important to note: liquids and solids are not included in the $k$ expression, regard them as ones

  \item Ice $-$ $\begin{array}{l | l} \text{I} & \text{nitial} \\ \text{C} & \text{hange}\\ \text{E} & \text{quilibrium}\\ \end{array}$

  \item $k$ tells you about where the equilibrium lies

    \begin{enumerate}

      \item If $k>1$, the equilibrium is to the right

        \begin{enumerate}

          \item More product

        \end{enumerate}

      \item If $k<1$, the equilibrium is to the left

        \begin{enumerate}

          \item More reactant

        \end{enumerate}

    \end{enumerate}

  \item Reaction Quotient ($Q$)

    \begin{enumerate}

      \item Ratio of product to reactants at a set time

      \item Similar to $k$, but not necessarily at equilibrium line $k$

      \item If $Q<k$, need to form more product to get to equilibrium

      \item If $Q>k$, too much product made, must go in the other direction to get to equilibrium

      \item If $Q=k$, at equilibrium

    \end{enumerate}

  \item Le Chatelier's Principle

    \begin{enumerate}

      \item When reactant or product is added, equilibrium shifts away from the side it was added from

      \item When reactant or product is removed, equilibrium shifts towards the side it was removed from

      \item Volume Change

        \begin{enumerate}

          \item Decrease Volume: Pressure goes up, equilibrium will shift to side with least gas molecules

          \item Increase Volume: Pressure goes down, equilibrium will shift to side with most gas molecules

        \end{enumerate}

      \item Temperature

        \begin{enumerate}

          \item Exothermic: Whatever is done to temperature affects the right side

          \item Endothermic: Whatever is done to temperature affects the left side

        \end{enumerate}

      \item Example: Heat + \ce{UO2(s) + 4HF(g) <=> UF4(g) + 2H2O(g)}, $\Delta H=36[\si{\kilo\joule}]$

        \begin{enumerate}

          \item Adding \ce{HF} shifts right

          \item Reducing heat shifts left

          \item Increasing volume shifts left

        \end{enumerate}

    \end{enumerate}

\end{itemize}

\end{document}

