%%%%%%%%%%%%%%%%%%%%%%%%%%%%%%%%%%%%%%%%%%%%%%%%%%%%%%%%%%%%%%%%%%%%%%%%%%%%%%%%%%%%%%%%%%%%%%%%%%%%%%%%%%%%%%%%%%%%%%%%%%%%%%%%%%%%%%%%%%%%%%%%%%%%%%%%%%%%%%%%%%%%%%%%%%%%%%%%%%%%%%%%%%%%
% Written By Michael Brodskiy
% Class: AP Chemistry
% Professor: J. Morgan
%%%%%%%%%%%%%%%%%%%%%%%%%%%%%%%%%%%%%%%%%%%%%%%%%%%%%%%%%%%%%%%%%%%%%%%%%%%%%%%%%%%%%%%%%%%%%%%%%%%%%%%%%%%%%%%%%%%%%%%%%%%%%%%%%%%%%%%%%%%%%%%%%%%%%%%%%%%%%%%%%%%%%%%%%%%%%%%%%%%%%%%%%%%%

\documentclass[12pt]{article} 
\usepackage{alphalph}
\usepackage[utf8]{inputenc}
\usepackage[russian,english]{babel}
\usepackage{titling}
\usepackage{amsmath}
\usepackage{graphicx}
\usepackage{enumitem}
\usepackage{amssymb}
\usepackage[super]{nth}
\usepackage{expl3}
\usepackage[version=4]{mhchem}
\usepackage{hpstatement}
\usepackage{rsphrase}
\usepackage{everysel}
\usepackage{ragged2e}
\usepackage{geometry}
\usepackage{fancyhdr}
\usepackage{cancel}
\usepackage{siunitx}
\usepackage{chemfig}
\geometry{top=1.0in,bottom=1.0in,left=1.0in,right=1.0in}
\newcommand{\subtitle}[1]{%
  \posttitle{%
    \par\end{center}
    \begin{center}\large#1\end{center}
    \vskip0.5em}%

}
\newcommand{\orbital}[2]{{%
    \def\+{\big|\hspace{-2pt}\overline{\underline{\hspace{2pt}\upharpoonleft}}}%
    \def\-{\overline{\underline{\downharpoonright\hspace{2pt}}}\hspace{-2pt}\big|}%
    \def\0{\big|\hspace{-2pt}\overline{\underline{\phantom{\hspace{2pt}\downharpoonright}}}}%
    \def\1{\overline{\underline{\phantom{\downharpoonright\hspace{2pt}}}}\hspace{-2pt}\big|}%
  \setlength\tabcolsep{0pt}% remove extra horizontal space from tabular
  \begin{tabular}{c}$#2$\\[2pt]#1\end{tabular}%
}}
\DeclareSIUnit\Molar{\textsc{m}}
\DeclareSIUnit\atm{\textsc{atm}}
\DeclareSIUnit\torr{\textsc{torr}}
\DeclareSIUnit\psi{\textsc{psi}}
\DeclareSIUnit\bar{\textsc{bar}}
\DeclareSIUnit\Celsius{\textsc{C}}
\usepackage{hyperref}
\hypersetup{
colorlinks=true,
linkcolor=blue,
filecolor=magenta,      
urlcolor=blue,
citecolor=blue,
}

\urlstyle{same}


\title{Chapter 11 $-$ Rates}
\date{\today}
\author{Michael Brodskiy\\ \small Instructor: Mr. Morgan}

% Mathematical Operations:

% Sum: $$\sum_{n=a}^{b} f(x) $$
% Integral: $$\int_{lower}^{upper} f(x) dx$$
% Limit: $$\lim_{x\to\infty} f(x)$$

\begin{document}

\maketitle

\begin{itemize}

  \item General Rate = Change in concentration per change in time.

  \item Rate is dependent on concentration $-$ More collisions, more reactions, faster rate.

  \item Rate Law Formula = $k[A]^m$, where $k$ is a constant, $A$ is the reactant, and $m$ is the order.

  \item If there are two reactants, then the rate = $k[A]^m[B]^n$, where $B$ is the other reactant, and $n$ is the order for the second reactant

  \item Order $-$ Tells how rate changes when concentration changes ($0,1,2$)

  \item Example: Calculate $k$ for \ce{CH3CHO -> CH4 + CO}

    \begin{tabular}[H]{l c c c c}
      Concentration: & .1 & .2 & .3 & .4\\
      Rate: & .085 & .34 & .076 & 1.4\\
    \end{tabular}

    \begin{equation}
      \begin{split}
        \frac{Rate_1}{Rate_2}=\left( \frac{(\ce{CH3CHO})_1}{(\ce{CH3CHO})_2} \right)^m\\
        \frac{.085}{.34}=\left( \frac{.1}{.2} \right)^m\\
        m=2\\
        .085=k(.1)^2\\
        k=8.5
      \end{split}
      \label{1}
    \end{equation}

  \item Example: Calculate $k$ for \ce{2NO + Cl2 -> 2NOCl}

    \begin{tabular}[H]{l c c c}
      Experiment & \ce{NO} & \ce{Cl2} & Rate\\
      \hline
      1 & .0125 & .0255 & $2.27\cdot10^{-5}$\\
      2 & .0125 & .051 & $4.55\cdot10^{-5}$\\
      3 & .025 & .0255 & $9.08\cdot10^{-5}$\\
    \end{tabular}

    \begin{equation}
      \begin{split}
        \frac{Rate_1}{Rate_3}=\left( \frac{(\ce{NO})_1}{(\ce{NO})_3} \right)^m\\
        \frac{Rate_1}{Rate_2}=\left( \frac{(\ce{Cl2})_1}{(\ce{Cl2})_2} \right)^n\\
        \frac{2.27}{9.08}=\left( \frac{.0125}{.025} \right)^m\\
        m=2\\
        \frac{2.27}{4.5}=\left( \frac{.0225}{.051} \right)^n\\
        n=1\\
        2.27\cdot10^{-5}=k[.0125]^2[.0255]^1\\
        k=5.7
      \end{split}
      \label{2}
    \end{equation}

\end{itemize}

\end{document}

