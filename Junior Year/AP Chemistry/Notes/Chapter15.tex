%%%%%%%%%%%%%%%%%%%%%%%%%%%%%%%%%%%%%%%%%%%%%%%%%%%%%%%%%%%%%%%%%%%%%%%%%%%%%%%%%%%%%%%%%%%%%%%%%%%%%%%%%%%%%%%%%%%%%%%%%%%%%%%%%%%%%%%%%%%%%%%%%%%%%%%%%%%%%%%%%%%%%%%%%%%%%%%%%%%%%%%%%%%%
% Written By Michael Brodskiy
% Class: AP Chemistry
% Professor: J. Morgan
%%%%%%%%%%%%%%%%%%%%%%%%%%%%%%%%%%%%%%%%%%%%%%%%%%%%%%%%%%%%%%%%%%%%%%%%%%%%%%%%%%%%%%%%%%%%%%%%%%%%%%%%%%%%%%%%%%%%%%%%%%%%%%%%%%%%%%%%%%%%%%%%%%%%%%%%%%%%%%%%%%%%%%%%%%%%%%%%%%%%%%%%%%%%

\documentclass[12pt]{article} 
\usepackage{alphalph}
\usepackage[utf8]{inputenc}
\usepackage[russian,english]{babel}
\usepackage{titling}
\usepackage{amsmath}
\usepackage{graphicx}
\usepackage{enumitem}
\usepackage{amssymb}
\usepackage[super]{nth}
\usepackage{expl3}
\usepackage[version=4]{mhchem}
\usepackage{hpstatement}
\usepackage{rsphrase}
\usepackage{everysel}
\usepackage{ragged2e}
\usepackage{geometry}
\usepackage{fancyhdr}
\usepackage{cancel}
\usepackage{siunitx}
\usepackage{chemfig}
\geometry{top=1.0in,bottom=1.0in,left=1.0in,right=1.0in}
\newcommand{\subtitle}[1]{%
  \posttitle{%
    \par\end{center}
    \begin{center}\large#1\end{center}
    \vskip0.5em}%

}
\newcommand{\orbital}[2]{{%
    \def\+{\big|\hspace{-2pt}\overline{\underline{\hspace{2pt}\upharpoonleft}}}%
    \def\-{\overline{\underline{\downharpoonright\hspace{2pt}}}\hspace{-2pt}\big|}%
    \def\0{\big|\hspace{-2pt}\overline{\underline{\phantom{\hspace{2pt}\downharpoonright}}}}%
    \def\1{\overline{\underline{\phantom{\downharpoonright\hspace{2pt}}}}\hspace{-2pt}\big|}%
  \setlength\tabcolsep{0pt}% remove extra horizontal space from tabular
  \begin{tabular}{c}$#2$\\[2pt]#1\end{tabular}%
}}
\DeclareSIUnit\Molar{\textsc{m}}
\DeclareSIUnit\atm{\textsc{atm}}
\DeclareSIUnit\torr{\textsc{torr}}
\DeclareSIUnit\psi{\textsc{psi}}
\DeclareSIUnit\bar{\textsc{bar}}
\DeclareSIUnit\Celsius{\textsc{C}}
\usepackage{hyperref}
\hypersetup{
colorlinks=true,
linkcolor=blue,
filecolor=magenta,      
urlcolor=blue,
citecolor=blue,
}

\urlstyle{same}


\title{Chapter 15 $-$ Precipitation Equilibrium}
\date{\today}
\author{Michael Brodskiy\\ \small Instructor: Mr. Morgan}

% Mathematical Operations:

% Sum: $$\sum_{n=a}^{b} f(x) $$
% Integral: $$\int_{lower}^{upper} f(x) dx$$
% Limit: $$\lim_{x\to\infty} f(x)$$

\begin{document}

\maketitle

\begin{itemize}

  \item Example Decomposition: \ce{NaCl(s) <=> Na+(s) + Cl-(s)}

    \begin{enumerate}

      \item $K_{sp}=\left[ \ce{Na+} \right]\left[ \ce{Cl-} \right]$

    \end{enumerate}

  \item Solutions can only hold a set number of ions, over that solid forms

  \item Ion Product (P) — Concentration not necessarily at equilibrium

    \begin{enumerate}

      \item $P > K_{sp}$ then there is solid

      \item $P < K_{sp}$ then there is no solid

      \item $P = K_{sp}$ then there is no solid and it is at equilibrium

    \end{enumerate}

  \item Water Solubility — How much can dissolve

    \begin{enumerate}

      \item Example: \ce{Fe(OH)2} has a solution of $2.5\cdot10^{-5}[\si{\Molar}]$

        \begin{equation*}
          \begin{split}
            K_{sp}&=\left[ \ce{Fe} \right]\left[ \ce{OH} \right]^2\\
            &=\left( 2.5\cdot10^{-5} \right)\left( 2.5\cdot10^{-5} \right)^2\\
            &= 1.56\cdot10^{-14}
          \end{split}
          \label{1}
        \end{equation}

      \item Example: $1[\si{\gram}]$ \ce{CaF2} $\left( K_{sp}=1.5\cdot10^{-10} \right)$ is dissolved in $1[\si{\liter}]$ of water at $80[\si{\degree\Celsius}]$. Calculate the mass precipitation at $25[\si{\degree\Celsius}]$

        \begin{equation*}
          \begin{split}
            \ce{CaF2 <=> Ca^2+ + 2F-}\\
            K_{sp}=\left[ \ce{Ca^2+} \right]\left[ \ce{F-} \right]^2\\
              1.5\cdot10^{-10}=(x)\left( 2x \right)^2\\
              x=.000347\left[ \si{\Molar} \right]\\
              .000347\cdot78= .974[\si{\gram}]\text{ not dissolved}\\
              1-.974=.026[\si{\gram}]\text{ dissolved}
          \end{split}
          \label{2}
        \end{equation}

    \end{enumerate}

    \newpage

  \item Common Ion Effect — Dissolving an ionic compound in water that already has that ion in it (e.g. dissolving \ce{CaCO3} in \ce{Na2CO3} — carbonate, \ce{CO3^2-}, is the common ion)

    \begin{enumerate}

      \item Ionic solids are less soluble in a solution with a common ion

      \item Example: Calculate solubility of \ce{CaCO3} ($K_{sp}=5\cdot10^{-9}$) in pure water and in a $.1[\si{\Molar}]$ solution of \ce{Fe(CO3)2}

        \begin{enumerate}

          \item Pure:

            \begin{equation*}
              \begin{split}
                5\cdot10^{-9}=(x)(x)\\
                x=7.07\cdot10^{-5}\left[ \si{\Molar} \right]
              \end{split}
              \label{3}
            \end{equation}

          \item $.1[\si{\Molar}] \ce{Fe(CO3)2}

            \begin{equation*}
              \begin{split}
                5\cdot10^{-9}=(x)((.1)(2))\\
                x=2.5\cdot10^{-8}[\si{\Molar}]
              \end{split}
              \label{4}
            \end{equation}

        \end{enumerate}

    \end{enumerate}

\end{itemize}

\end{document}

