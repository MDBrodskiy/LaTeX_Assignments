%%%%%%%%%%%%%%%%%%%%%%%%%%%%%%%%%%%%%%%%%%%%%%%%%%%%%%%%%%%%%%%%%%%%%%%%%%%%%%%%%%%%%%%%%%%%%%%%%%%%%%%%%%%%%%%%%%%%%%%%%%%%%%%%%%%%%%%%%%%%%%%%%%%%%%%%%%%%%%%%%%%%%%%%%%%%%%%%%%%%%%%%%%%%
% Written By Michael Brodskiy
% Class: AP Chemistry
% Professor: J. Morgan
%%%%%%%%%%%%%%%%%%%%%%%%%%%%%%%%%%%%%%%%%%%%%%%%%%%%%%%%%%%%%%%%%%%%%%%%%%%%%%%%%%%%%%%%%%%%%%%%%%%%%%%%%%%%%%%%%%%%%%%%%%%%%%%%%%%%%%%%%%%%%%%%%%%%%%%%%%%%%%%%%%%%%%%%%%%%%%%%%%%%%%%%%%%%

\documentclass[12pt]{article} 
\usepackage{alphalph}
\usepackage[utf8]{inputenc}
\usepackage[russian,english]{babel}
\usepackage{titling}
\usepackage{amsmath}
\usepackage{graphicx}
\usepackage{enumitem}
\usepackage{amssymb}
\usepackage[super]{nth}
\usepackage{everysel}
\usepackage{ragged2e}
\usepackage{geometry}
\usepackage{fancyhdr}
\usepackage{cancel}
\usepackage{siunitx}
\geometry{top=1.0in,bottom=1.0in,left=1.0in,right=1.0in}
\newcommand{\subtitle}[1]{%
  \posttitle{%
    \par\end{center}
    \begin{center}\large#1\end{center}
    \vskip0.5em}%

}
\usepackage{hyperref}
\hypersetup{
colorlinks=true,
linkcolor=blue,
filecolor=magenta,      
urlcolor=blue,
citecolor=blue,
}

\urlstyle{same}


\title{Unit Conversions, Density, and Solubility}
\date{August 20, 2020}
\author{Michael Brodskiy\\ \small Professor: Mr. Morgan}

% Mathematical Operations:

% Sum: $$\sum_{n=a}^{b} f(x) $$
% Integral: $$\int_{lower}^{upper} f(x) dx$$
% Limit: $$\lim_{x\to\infty} f(x)$$

\begin{document}

\maketitle

\begin{itemize}

  \item Unit Cancellation

    \begin{enumerate}

      \item Only cancel out when one unit is on top and one is on bottom

        $$3.24[mi]\rightarrow [\si{\meter}]$$
    $$1[mi]=1.6093[\si{\kilo\meter}]$$
        $$3.24[\cancel{mi}] \cdot \frac{1.6093[\cancel{\si{\kilo\meter}}]}{1[\cancel{mi}]}\cdot \frac{1000[\si{\meter}]}{1[\cancel{\si{\kilo\meter}}]}$$
        $$5,214[\si{\meter}]$$

    \end{enumerate}

  \item Density = [Mass / Volume] $\rightarrow P=\frac{m}{V}$

    %\begin{center} 21\% of the mass of air is oxygen. In a box with dimensions 15$[\meter]$, 20$[\meter]$, and 35$[\meter]$. The density of the air is 1.31$[\gram\per\liter]$. Find the mass of oxygen in the room. \end{center}
    %$$V= 15\cdot20\cdot35=10500[\meter^3]$$

  \item Solubility $-$ What is the maximum amount of an ionic compound that will dissolve in a liquid?
 
    \begin{enumerate}

      \item Unsaturated $-$ Below maximum of solubility

      \item Saturated $-$ At the maximum of solubility

      \item Supersaturated $-$ Over the maximum of solubility

        \begin{center} Solubility of KNO$_3$ is $\frac{246[\si{\gram}]}{100[\si{\gram}\text{ of water}]}$ at $80[\degree C]$. What is the amount of water necessary to dissolve 100$[\si{\gram}]$ of KNO$_3$? \end{center}

        $$\frac{246[\si{\gram}\text{ }KNO_3]}{100[\si{\gram}\text{ }H_2O]}=\frac{100[\si{\gram}\text{ }KNO_3]}{x}$$
        $$x=\frac{10000[\si{\gram}\text{ }H_2O]}{246}=4.065\cdot10^1$$

    \end{enumerate}

\end{itemize}

\end{document}

