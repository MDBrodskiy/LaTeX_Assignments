%%%%%%%%%%%%%%%%%%%%%%%%%%%%%%%%%%%%%%%%%%%%%%%%%%%%%%%%%%%%%%%%%%%%%%%%%%%%%%%%%%%%%%%%%%%%%%%%%%%%%%%%%%%%%%%%%%%%%%%%%%%%%%%%%%%%%%%%%%%%%%%%%%%%%%%%%%%%%%%%%%%%%%%%%%%%%%%%%%%%%%%%%%%%
% Written By Michael Brodskiy
% Class: AP Chemistry
% Professor: J. Morgan
%%%%%%%%%%%%%%%%%%%%%%%%%%%%%%%%%%%%%%%%%%%%%%%%%%%%%%%%%%%%%%%%%%%%%%%%%%%%%%%%%%%%%%%%%%%%%%%%%%%%%%%%%%%%%%%%%%%%%%%%%%%%%%%%%%%%%%%%%%%%%%%%%%%%%%%%%%%%%%%%%%%%%%%%%%%%%%%%%%%%%%%%%%%%

\documentclass[12pt]{article} 
\usepackage{alphalph}
\usepackage[utf8]{inputenc}
\usepackage[russian,english]{babel}
\usepackage{titling}
\usepackage{amsmath}
\usepackage{graphicx}
\usepackage{enumitem}
\usepackage{amssymb}
\usepackage[super]{nth}
\usepackage{expl3}
\usepackage[version=4]{mhchem}
\usepackage{hpstatement}
\usepackage{rsphrase}
\usepackage{everysel}
\usepackage{ragged2e}
\usepackage{geometry}
\usepackage{fancyhdr}
\usepackage{cancel}
\usepackage{siunitx}
\geometry{top=1.0in,bottom=1.0in,left=1.0in,right=1.0in}
\newcommand{\subtitle}[1]{%
  \posttitle{%
    \par\end{center}
    \begin{center}\large#1\end{center}
    \vskip0.5em}%

}
\DeclareSIUnit\Molar{\textsc{m}}
\DeclareSIUnit\atm{\textsc{atm}}
\DeclareSIUnit\torr{\textsc{torr}}
\DeclareSIUnit\psi{\textsc{psi}}
\DeclareSIUnit\bar{\textsc{bar}}
\usepackage{hyperref}
\hypersetup{
colorlinks=true,
linkcolor=blue,
filecolor=magenta,      
urlcolor=blue,
citecolor=blue,
}

\urlstyle{same}


\title{Chapter 6 $-$ Electron Structure}
\date{\today}
\author{Michael Brodskiy\\ \small Instructor: Mr. Morgan}

% Mathematical Operations:

% Sum: $$\sum_{n=a}^{b} f(x) $$
% Integral: $$\int_{lower}^{upper} f(x) dx$$
% Limit: $$\lim_{x\to\infty} f(x)$$

\begin{document}

\maketitle

\begin{itemize}

  \item Atoms gain and lose energy in set amounts $-$ Quantized

  \item Lower energy level is \underline{ground state}, higher is called \underline{excited}

  \item An atom which gains energy moves electrons to a higher energy level

  \item An atom which loses energy has electrons move back down (\underline{Electron Jumping})

  \item Energy is seen as different wavelengths of light in a flame test

  \item $E=hV$ and $C=\lambda V$, where $h$ is Planck's constant, $V$ is the frequency, $\lambda$ is the wavelength, $C$ is the speed of light, and $E$ is energy

  \item Bohr's Model $-$ Electrons orbit the nucleus, and, when they gained energy, jump up to a new level

  \item Quantum Mechanical Model $-$ It is unknown how electrons move, but we know where they probably are, which is demonstrated in probability maps

  \item Probability Maps $-$ Orbitals (Four Types) s, p, d, and f (sometimes called sublevels)

  \item s forms a circular probability, p forms a 2 leaf clover, d forms 4 leaf clover, and f is technically 6, but is hard to map out

  \item 2 Electrons per orbital

  \item \begin{tabular}{c c c}
      Type & Orbitals & Electrons\\
      \hline
      s & 1 & 2\\
      p & 3 & 6\\
      d & 5 & 10\\
      f & 7 & 14\\
    \end{tabular}

  \item Electron configuration and Box diagrams (often called Orbital Diagrams)

    \newpage

  \item Quantum Numbers:

    \begin{enumerate}

      \item Energy Level (n)

      \item Sublevel (l): Type (s=0; p=1; d=2; f=3)

      \item Box number (Number of orbitals, $m_l$): $-l\leq m_l\leq l$

      \item Spin ($m_s$): $-\frac{1}{2}\leq m_s\leq \frac{1}{2}$

    \end{enumerate}

  \item Hund's Rule $-$ Electrons spread out

  \item Pauli Exclusion Principle $-$ No two electrons have the same 4 quantum numbers

  \item Noble Gas Short-hand:

    \begin{enumerate}

      \item Start with the noble gas from the row above (for example, for \ce{As} it would be \ce{Ar})

      \item Then, start counting from the \ce{s} type, with energy level + 1 of the noble gas (\ce{4s} for \ce{As}) 

    \end{enumerate}

  \item Amount of valence electrons is determined by the column an element is in (5 for \ce{As})

  \item When an ion is given, electrons are first pulled from the \ce{s} orbital

    \begin{enumerate}

      \item Example: \ce{Fe} to \ce{Fe^3+}

      \item \ce{Fe}: \ce{[Ar] 4s^2 3d^6}

      \item \ce{Fe^3+}: \ce{[Ar] 3d^5}

    \end{enumerate}

  \item Periodic Trends:

    \begin{enumerate}

      \item Radius (size)

        \begin{enumerate}

          \item The closer to the bottom left, the bigger the size of the atom

          \item Greatest: Francium (\ce{Fr})

          \item Less size due to more protons, causing greater strong and weak nuclear forces

          \item Electron Shielding: Electrons in lower energy levels shield outer electrons from nuclear charge

          \item Negative ions are larger than a neutral atom, while positive are smaller

          \item Ex. \ce{Fe^2+} vs \ce{Fe^3+}: \ce{Fe^2+} is large since more electrons mean more electron repulsion making it larger

        \end{enumerate}

      \item Ionization

        \begin{enumerate}

          \item The closer to the top right, the more ionized an atom (needs more energy to remove an electron)

          \item Greatest: Fluorine (\ce{F})

          \item Two Exceptions: \ce{Be} and \ce{Mg} need larger ionization energy than \ce{B} and \ce{Al}, respectively, because 2p is higher in energy than 2s, which means it takes less energy to remove from \ce{B}

        \end{enumerate}

      \item Electronegativity

        \begin{enumerate}

          \item The closer to the top right, the more electronegative an element (has more of a relative pull for shared electrons in a bond)

          \item Greatest: Fluorine (\ce{F}) 

        \end{enumerate}

    \end{enumerate}

  \item Bonds:

    \begin{enumerate}

      \item Ionic $-$ \ce{e-} are transferred (atoms bonded this way will be far across the periodic table from each other). Atoms become polar (dipole).

      \item Polar Covalent $-$ \ce{e-} are shared unevenly (atoms bonded this way will be relatively close to each other). Atoms also become polar (dipole)

      \item Covalent $-$ \ce{e-} are shared evenly (no plus or minus end to it).

    \end{enumerate}

\end{itemize}

\end{document}

