%%%%%%%%%%%%%%%%%%%%%%%%%%%%%%%%%%%%%%%%%%%%%%%%%%%%%%%%%%%%%%%%%%%%%%%%%%%%%%%%%%%%%%%%%%%%%%%%%%%%%%%%%%%%%%%%%%%%%%%%%%%%%%%%%%%%%%%%%%%%%%%%%%%%%%%%%%%%%%%%%%%%%%%%%%%%%%%%%%%%%%%%%%%%
% Written By Michael Brodskiy
% Class: AP Chemistry
% Professor: J. Morgan
%%%%%%%%%%%%%%%%%%%%%%%%%%%%%%%%%%%%%%%%%%%%%%%%%%%%%%%%%%%%%%%%%%%%%%%%%%%%%%%%%%%%%%%%%%%%%%%%%%%%%%%%%%%%%%%%%%%%%%%%%%%%%%%%%%%%%%%%%%%%%%%%%%%%%%%%%%%%%%%%%%%%%%%%%%%%%%%%%%%%%%%%%%%%

\documentclass[12pt]{article} 
\usepackage{alphalph}
\usepackage[utf8]{inputenc}
\usepackage[russian,english]{babel}
\usepackage{titling}
\usepackage{amsmath}
\usepackage{graphicx}
\usepackage{enumitem}
\usepackage{amssymb}
\usepackage[super]{nth}
\usepackage{expl3}
\usepackage[version=4]{mhchem}
\usepackage{hpstatement}
\usepackage{rsphrase}
\usepackage{everysel}
\usepackage{ragged2e}
\usepackage{geometry}
\usepackage{fancyhdr}
\usepackage{cancel}
\usepackage{siunitx}
\geometry{top=1.0in,bottom=1.0in,left=1.0in,right=1.0in}
\newcommand{\subtitle}[1]{%
  \posttitle{%
    \par\end{center}
    \begin{center}\large#1\end{center}
    \vskip0.5em}%

}
\usepackage{hyperref}
\hypersetup{
colorlinks=true,
linkcolor=blue,
filecolor=magenta,      
urlcolor=blue,
citecolor=blue,
}

\urlstyle{same}


\title{Chapter 4 $-$ Reactions in Solutions}
\date{\today}
\author{Michael Brodskiy\\ \small Instructor: Mr. Morgan}

% Mathematical Operations:

% Sum: $$\sum_{n=a}^{b} f(x) $$
% Integral: $$\int_{lower}^{upper} f(x) dx$$
% Limit: $$\lim_{x\to\infty} f(x)$$

\begin{document}

\maketitle

\begin{itemize}

  \item Solute gets dissolved, Solvent does the dissolving

  \item Molarity is \eqref{1}

    \begin{equation}
      \begin{split}
        M=\frac{\si{\mole}}{\si{\liter}}\\
      \end{split}
      \label{1}
    \end{equation}

  \item Electrolytes are ionic compounds that breakup in a solution (Ex. \eqref{2})

    \begin{equation}
      \ce{NaCl -> Na+ + Cl-}
      \label{2}
    \end{equation}

    \begin{enumerate}

      \item Strong Electrolyte vs. Weak Electrolyte $-$ The more are broken up, the stronger the electrolyte

      \item The Dilution Formula \eqref{3}

        \begin{equation}
          M_1V_1=M_2V_2
          \label{3}
        \end{equation}

    \end{enumerate}

    \begin{equation}
      \begin{split}
        \ce{Ca(OH)2(s) -> Ca^2+(aq) + 2OH-(aq)}\\
        \ce{Fe3(PO4)2 -> 3Fe^2+(aq) + 2PO4^3-(aq)}\\
        \ce{Cr(NO3)3 -> Cr^3+(aq) + 3NO3^-(aq)}\\
      \end{split}
      \label{4}
    \end{equation}

  \item Precipitation Reaction $-$ Ionic compounds will either separate (soluble) or stay together (insoluble).

    \begin{enumerate}

      \item Solubility Rules $-$ If a compound contains any of the following three, it is soluble: Sodium (\ce{Na+}), Potassium (\ce{K+}), Nitrate (\ce{NO3-})

      \item Ex. Potassium Chromate + Barium Nitrate \eqref{5}. This is an example of a molecular equation with a double replacement. \eqref{6} is named a complete ionic equation. \eqref{7} is named a net equation, and is the only one that will be on the AP exam.

        \begin{equation}
          \begin{split}
            \ce{K+}, \ce{CrO4^2-}, \ce{Ba^2+}, \ce{NO3-}\\
            \ce{K2CrO4(aq) + Ba(NO3)2(aq) -> KNO3(aq) + BaCrO4(s)}\\
          \end{split}
          \label{5}
        \end{equation}

        \begin{equation}
          \begin{split}
          \ce{2K+(aq) + CrO4^2-(aq) +  Ba^2+(aq) + 2NO3-(aq) ->} & 
        & \ce{2K+(aq) + 2NO3-(aq) + BaCrO4(s)}\\
          \end{split}
          \label{6}
        \end{equation}

        \begin{equation}
          \begin{split}
            \ce{CrO4^2-(aq) +  Ba^2+(aq) -> BaCrO4(s)}\\
          \end{split}
          \label{7}
        \end{equation}

    \end{enumerate}

  \item Acid-Base Reaction

    \begin{enumerate}

      \item Acid $-$ Produces \ce{H+}

    \item Base $-$ Produces \ce{OH-}

    \item ``Arrhenius'' Way of thinking

    \item Strong Acid $-$ Completely Dissociates

      \begin{enumerate}
        \item Examples: Hydrochloric (\ce{HCl}), Sulfuric (\ce{H2SO4}), Nitric (\ce{HNO3}), Perchloric (\ce{HClO4}), Hydrobromic (\ce{HBr}), Hydroionic (\ce{HI})
      \end{enumerate}

    \item Weak Acid $-$ Does not completely dissociate. Sets up an equilibrium.

      \begin{enumerate}
        \item Not a strong acid.
      \end{enumerate}

    \item Strong Base $-$ Completely dissociates.

      \begin{enumerate}
        \item Hydroxides of column I and I.
      \end{enumerate}

    \item Weak Bass $-$ Produce \ce{OH-} with reaction with water.

      \begin{enumerate}
        \item It will always be explicitly stated if something is a weak base
      \end{enumerate}

    \item Dissociation \eqref{8}

      \begin{equation}
        \begin{split}
          \text{Strong Acid: } & \ce{HCl -> H+ + Cl-} \\
          \text{Weak Acid: } & \ce{HF <=> H+ F-}\\
          \text{Strong Base: } & \ce{KOH -> K+ + OH-}\\
          \text{Weak Base: } & \ce{NH3 + H2O <=> NH4+ + OH}\\
        \end{split}
        \label{8}
      \end{equation}

    \item Strong Acid $-$ \ce{H+} 

    \item Strong Base $-$ \ce{OH-}

    \item Weaks are represented as is

    \item \ce{HCl} and \ce{NaOH} \eqref{9}

    \item \ce{HB} and \ce{KOH} \eqref{10}

    \item \ce{H2SO4} and \ce{NH3} \eqref{11}

    \item \ce{HF} and \ce{CH2NH2} \eqref{12}

      \begin{equation}
        \ce{H+ + OH- -> H2O}
        \label{9}
      \end{equation}

      \begin{equation}
        \ce{HB + OH- -> B- + H2O}
        \label{10}
      \end{equation}

      \begin{equation}
        \ce{H+ + NH3 -> NH4+}
        \label{11}
      \end{equation}
      
      \begin{equation}
        \ce{HF + CH2NH2 -> F- + CH3NH2+}
        \label{12}
      \end{equation}

    \item Titration $-$ Adding acid to base or other way around \eqref{13}

      \begin{equation}
        M_aV_a=M_bV_b
        \label{13}
      \end{equation}

    \end{enumerate}

  \item Oxidation/Reduction (Redox Reactions) $-$ Transfer of electrons

  \item Oxidation $-$ Loss of electrons

    \begin{enumerate}
      \item Examples of half reactions: \eqref{14}
    \end{enumerate}

    \begin{equation}
      \ce{Zn -> Zn^2+ + 2e-}
      \label{14}
    \end{equation}

  \item Reduction $-$ Gain of electrons

    \begin{enumerate}
      \item Examples of half reactions: \eqref{15}
    \end{enumerate}

    \begin{equation}
      \ce{Cl + e- -> Cl-}
      \label{15}
    \end{equation}

  \item Oxidation Numbers $-$ Used to track electrons

    \begin{enumerate}
      \item Group 1 = +1
      \item Group 2 = +2
      \item F=-1

        \begin{enumerate}
          \item H is a +1; O is -2
          \item Only one atom = to charge
          \item Ionic Compounds = charge on atoms
          \item Sum must be equal to overall charge
        \end{enumerate}

    \end{enumerate}

  \item Oxidizing Agent $-$ Does the oxidizing, gets reduced

  \item Reduction Agent $-$ Does the reducing, gets oxidized


\end{itemize}

\end{document}

