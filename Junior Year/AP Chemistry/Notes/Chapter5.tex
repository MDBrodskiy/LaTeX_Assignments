%%%%%%%%%%%%%%%%%%%%%%%%%%%%%%%%%%%%%%%%%%%%%%%%%%%%%%%%%%%%%%%%%%%%%%%%%%%%%%%%%%%%%%%%%%%%%%%%%%%%%%%%%%%%%%%%%%%%%%%%%%%%%%%%%%%%%%%%%%%%%%%%%%%%%%%%%%%%%%%%%%%%%%%%%%%%%%%%%%%%%%%%%%%%
% Written By Michael Brodskiy
% Class: AP Chemistry
% Professor: J. Morgan
%%%%%%%%%%%%%%%%%%%%%%%%%%%%%%%%%%%%%%%%%%%%%%%%%%%%%%%%%%%%%%%%%%%%%%%%%%%%%%%%%%%%%%%%%%%%%%%%%%%%%%%%%%%%%%%%%%%%%%%%%%%%%%%%%%%%%%%%%%%%%%%%%%%%%%%%%%%%%%%%%%%%%%%%%%%%%%%%%%%%%%%%%%%%

\documentclass[12pt]{article} 
\usepackage{alphalph}
\usepackage[utf8]{inputenc}
\usepackage[russian,english]{babel}
\usepackage{titling}
\usepackage{amsmath}
\usepackage{graphicx}
\usepackage{enumitem}
\usepackage{amssymb}
\usepackage[super]{nth}
\usepackage{expl3}
\usepackage[version=4]{mhchem}
\usepackage{hpstatement}
\usepackage{rsphrase}
\usepackage{everysel}
\usepackage{ragged2e}
\usepackage{geometry}
\usepackage{fancyhdr}
\usepackage{cancel}
\usepackage{siunitx}
\geometry{top=1.0in,bottom=1.0in,left=1.0in,right=1.0in}
\newcommand{\subtitle}[1]{%
  \posttitle{%
    \par\end{center}
    \begin{center}\large#1\end{center}
    \vskip0.5em}%

}
\DeclareSIUnit\Molar{\textsc{m}}
\DeclareSIUnit\atm{\textsc{atm}}
\DeclareSIUnit\torr{\textsc{torr}}
\DeclareSIUnit\psi{\textsc{psi}}
\DeclareSIUnit\bar{\textsc{bar}}
\usepackage{hyperref}
\hypersetup{
colorlinks=true,
linkcolor=blue,
filecolor=magenta,      
urlcolor=blue,
citecolor=blue,
}

\urlstyle{same}


\title{Chapter 5 $-$ Gases}
\date{\today}
\author{Michael Brodskiy\\ \small Instructor: Mr. Morgan}

% Mathematical Operations:

% Sum: $$\sum_{n=a}^{b} f(x) $$
% Integral: $$\int_{lower}^{upper} f(x) dx$$
% Limit: $$\lim_{x\to\infty} f(x)$$

\begin{document}

\maketitle

\begin{itemize}

  \item Gases $-$ Uniformly fill any container; Easily compressed; Mixes completely with other gases; Exert pressure.

  \item Units: $[\si{atm}]=60[\si{\milli\meter}\ce{Hg}]=760[\si{torr}]=14.69[\si{psi}]=1.013[\si{bar}]=101325[\si{\pascal}]$

  \item Boyle's Law $-$ Pressure and volume are inversely related

  \item Charles's Law $-$ Volume directly proportional to temperature

  \item Avogadro's Law $-$ Volume directly proportional to moles

  \item Ideal Gas Law: $PV=nRT$; $R=.0821\left[ \frac{\si{\liter\atm}}{\si{\mole\kelvin}} \right]$

  \item Standard Temperature and Pressure (STP) $-$ $273[\si{\kelvin}]$ and $1[\si{\atm}]$

  \item At STP, one mole of a gas occupies $22.4[\si{\liter}]$

  \item Note: Hydrogen, Nitrogen, Oxygen, and Halogens are diatomics


  \item Gay-Lussac Law $-$ The volume ratio of any two gasses in a reaction at constant pressure and temperature is equal to the mole ratios

  \item The above law means that the type of molecule does not matter, only the amount of molecules. When asked for partial pressure, it may be found using the equation \eqref{1}

    \begin{equation}
      P_{total}=P_a+P_b+\dots+P_n
      \label{1}
    \end{equation}

  \item When asked for the mole fraction, use formula \eqref{2}, where $x_a$ is the mole fraction.

    \begin{equation}
      P_a=x_aP_{total}
      \label{2}
    \end{equation}
    
\end{itemize}

\end{document}

