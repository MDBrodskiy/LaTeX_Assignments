%%%%%%%%%%%%%%%%%%%%%%%%%%%%%%%%%%%%%%%%%%%%%%%%%%%%%%%%%%%%%%%%%%%%%%%%%%%%%%%%%%%%%%%%%%%%%%%%%%%%%%%%%%%%%%%%%%%%%%%%%%%%%%%%%%%%%%%%%%%%%%%%%%%%%%%%%%%%%%%%%%%%%%%%%%%%%%%%%%%%%%%%%%%%
% Written By Michael Brodskiy
% Class: AP Chemistry
% Professor: J. Morgan
%%%%%%%%%%%%%%%%%%%%%%%%%%%%%%%%%%%%%%%%%%%%%%%%%%%%%%%%%%%%%%%%%%%%%%%%%%%%%%%%%%%%%%%%%%%%%%%%%%%%%%%%%%%%%%%%%%%%%%%%%%%%%%%%%%%%%%%%%%%%%%%%%%%%%%%%%%%%%%%%%%%%%%%%%%%%%%%%%%%%%%%%%%%%

\documentclass[12pt]{article} 
\usepackage{alphalph}
\usepackage[utf8]{inputenc}
\usepackage[russian,english]{babel}
\usepackage{titling}
\usepackage{amsmath}
\usepackage{graphicx}
\usepackage{enumitem}
\usepackage{amssymb}
\usepackage[super]{nth}
\usepackage{everysel}
\usepackage{ragged2e}
\usepackage{geometry}
\usepackage{fancyhdr}
\usepackage{cancel}
\usepackage{siunitx}
\geometry{top=1.0in,bottom=1.0in,left=1.0in,right=1.0in}
\newcommand{\subtitle}[1]{%
  \posttitle{%
    \par\end{center}
    \begin{center}\large#1\end{center}
    \vskip0.5em}%

}
\usepackage{hyperref}
\hypersetup{
colorlinks=true,
linkcolor=blue,
filecolor=magenta,      
urlcolor=blue,
citecolor=blue,
}

\urlstyle{same}


\title{Chapter 2 $-$ Atoms}
\date{August 25, 2020}
\author{Michael Brodskiy\\ \small Professor: Mr. Morgan}

% Mathematical Operations:

% Sum: $$\sum_{n=a}^{b} f(x) $$
% Integral: $$\int_{lower}^{upper} f(x) dx$$
% Limit: $$\lim_{x\to\infty} f(x)$$

\begin{document}

\maketitle

\begin{itemize}

  \item Dalton's Atomic Theory:

    \begin{enumerate}

      \item Elements are made of atoms

      \item All atoms of the same element are the same

      \item Different atoms from different elements are different

      \item Certain atoms can combine

      \item No creation or destructing of atoms (conservation of mass)

    \end{enumerate}

  \item Constant Composition $-$ All compounds have the same composition (Water is always $H_2O$)

  \item Multiple Proportions $-$ Compounds come together in whole numbers  (Always $H_2O$, never $H_{.5}O$)

  \item JJ Thompson $-$ Used cathode ray to determine that atoms have tiny negative particles, but, because atoms are neutral, there must be positive charges to counter the negative

  \item Ernst Rutherford $-$ The gold foil experiment shot alpha particles at source of atoms
    
    \begin{enumerate}

      \item Most went through the atoms

      \item A few large deflections

        \begin{center} He concluded\dots\end{center}

      \item Atoms are mostly open space

      \item Center has positive charge

    \end{enumerate}

  \item Modern concept of atom $-$ Protons and neutrons in nucleus. Electrons on outside

  \item Different chemical properties are from the number and arrangement of the electrons

\item Periodic Table:

  \begin{enumerate}

    \item Columns up and down, rows left to right

    \item Column 1 $-$ Alkali Metals

    \item Column 2 $-$ Alkaline Earth Metals

    \item Middle $-$ Transition Metals

    \item Column 7 $-$ Halogens

    \item Column 8 $-$ Noble Gases

  \end{enumerate}

\item Properties of Metals

  \begin{enumerate}

    \item Conduct

    \item Malleable

    \item Ductile

    \item Lustrous
      
  \end{enumerate}

\item Atomic Number $-$ Number of protons, usually displayed at the top

\item Mass Number $-$ Protons plus neutrons is the atomic mass

\item Isotopes $-$ Different number of neutrons

\item Ions $-$ Different number of electrons

  \begin{enumerate}

    \item Cations $-$ Positive

    \item Anions $-$ Negative

  \end{enumerate}

\item Polyatomics $-$ Charged Groups

\item Ionic Compounds (Examples):

  \begin{enumerate}

    \item $Mg^{+2}\text{ \& }Cl^-\Rightarrow MgCl_2$

    \item $Ca^{+2}\text{ \& }PO_4\Rightarrow Ca_3(PO_4)_2$

    \item $Cr^{+3}\text{ \& }OH^-\Rightarrow Cr(OH)_3$

  \end{enumerate}

\item Common Charges:

  \begin{enumerate}

    \item Aluminum $\Rightarrow Al^{+3}$

    \item Zinc $\Rightarrow Zn^{+2}$

    \item Silver $\Rightarrow Ag^{+1}$

  \end{enumerate}

\end{itemize}

\end{document}

