%%%%%%%%%%%%%%%%%%%%%%%%%%%%%%%%%%%%%%%%%%%%%%%%%%%%%%%%%%%%%%%%%%%%%%%%%%%%%%%%%%%%%%%%%%%%%%%%%%%%%%%%%%%%%%%%%%%%%%%%%%%%%%%%%%%%%%%%%%%%%%%%%%%%%%%%%%%%%%%%%%%%%%%%%%%%%%%%%%%%%%%%%%%%
% Written By Michael Brodskiy
% Class: AP Chemistry
% Professor: J. Morgan
%%%%%%%%%%%%%%%%%%%%%%%%%%%%%%%%%%%%%%%%%%%%%%%%%%%%%%%%%%%%%%%%%%%%%%%%%%%%%%%%%%%%%%%%%%%%%%%%%%%%%%%%%%%%%%%%%%%%%%%%%%%%%%%%%%%%%%%%%%%%%%%%%%%%%%%%%%%%%%%%%%%%%%%%%%%%%%%%%%%%%%%%%%%%

\documentclass[12pt]{article} 
\usepackage{alphalph}
\usepackage[utf8]{inputenc}
\usepackage[russian,english]{babel}
\usepackage{titling}
\usepackage{amsmath}
\usepackage{graphicx}
\usepackage{enumitem}
\usepackage{amssymb}
\usepackage[super]{nth}
\usepackage{everysel}
\usepackage{ragged2e}
\usepackage{geometry}
\usepackage{fancyhdr}
\usepackage{cancel}
\usepackage{siunitx}
\usepackage{expl3}
\usepackage[version=4]{mhchem}
\usepackage{hpstatement}
\usepackage{rsphrase}
\geometry{top=1.0in,bottom=1.0in,left=1.0in,right=1.0in}
\newcommand{\subtitle}[1]{%
  \posttitle{%
    \par\end{center}
    \begin{center}\large#1\end{center}
    \vskip0.5em}%

}
\usepackage{hyperref}
\hypersetup{
colorlinks=true,
linkcolor=blue,
filecolor=magenta,      
urlcolor=blue,
citecolor=blue,
}

\urlstyle{same}


\title{Practice FRQ (2004 Form B)}
\date{October 29, 2020}
\author{Michael Brodskiy\\ \small Instructor: Mr. Morgan}

% Mathematical Operations:

% Sum: $$\sum_{n=a}^{b} f(x) $$
% Integral: $$\int_{lower}^{upper} f(x) dx$$
% Limit: $$\lim_{x\to\infty} f(x)$$

\begin{document}

\maketitle

\begin{enumerate}

  \item \eqref{1}

    \begin{equation}
      \begin{split}
        85.7\%_{\ce{C}}&=14.3\%{\ce{H}}\\
        m_{\ce{C}}&=12\left[ \frac{\si{\gram}}{\si{\mole}} \right]\\
        m_{\ce{H}}&=1\left[ \frac{\si{\gram}}{\si{\mole}} \right]\\
        85.7\frac{1}{12}&=7.14[\si{\mole}_{\ce{C}}]\\
        14.3\frac{1}{1}&=14.3[\si{\mole}_{\ce{H}}]\\
        \frac{7.14}{7.14}&=1[\ce{C}]\\
        \frac{14.3}{7.14}&=2[\ce{H}]\\
        &=\ce{CH2}
      \end{split}
      \label{1}
    \end{equation}

  \item 

    \begin{enumerate}

      \item \eqref{2}

        \begin{equation}
          \begin{split}
            \frac{2}{x}&=\frac{n}{V}\\
            \frac{2}{x}&=\frac{.948}{.0821\cdot323}\\
            x$\approx56\left[ \frac{\si{\gram}}{\si{\mole}} \right]\\
        \end{split}
          \label{2}
        \end{equation}

      \item \eqref{3}

        \begin{equation}
          \begin{split}
            m_{molar}&=56=4(14)\\
            \therefore&\rightarrow\ce{C4H8}
          \end{split}
          \label{3}
        \end{equation}

    \end{enumerate}

  \item \eqref{4}

    \begin{equation}
      \begin{split}
      P_0V_0&=P_fV_f\\
      3\cdot5&=(5+1)P_f\\
      P_f&=\frac{15}{6}\\
      &=2.5[atm]\\
      .55\cdot1&=(5+1)P_f\\
      P_f&=\frac{.55}{6}\\
      &=.0917[atm]\\
      P_{total}&=2.5+.0917\\
      &=2.6[atm]
    \end{split}
      \label{4}
    \end{equation}

  \item \eqref{5}

    \begin{equation}
      \begin{split}
        m_{\ce{C8H18}}&=255\cdot.703\\
        &=179.265[\si{\gram}]\\
        \si{\mole}_{\ce{C8H18}}&=\frac{179.265}{114}\\
        &=1.57[\si{\mole}_{\ce{C8H18}}]\\
        17\cdot1.57&=26.7[\si{\mole}]
      \end{split}
      \label{5}
    \end{equation}

\end{enumerate}

\end{document}

