\documentclass[12pt,letterpaper]{article}
\usepackage[utf8]{inputenc}
\usepackage[english]{babel}
\usepackage{ragged2e}
\usepackage[colorlinks = true,
            linkcolor = purple,
            urlcolor  = blue,
            citecolor = blue,
            anchorcolor = blue]{hyperref}
\hypersetup{
    colorlinks=true,
    linkcolor=blue,
    filecolor=blue,      
    urlcolor=blue,} 
\urlstyle{same}


\usepackage{ifpdf}
\usepackage{mla}

\begin{document}

\begin{mla}{Michael}{Brodskiy}{Mrs. Greer}{AP Language}{\today}{\underline{}} 

  \begin{justifying}

    \paragraph{I.} At a time in which the world was enveloped in peace, with decades of war over, Secretary of State Madeleine Albright gave a commencement speech at Mount Holyoke College, an all female school. Throughout her speech, Madeleine empowers women, whether it be in this school, or women across the globe, by adopting a cogent, powerful tone, through which she delivers a repetitive syntax, uses a connection based on pathos to encourage the graduating class, and real-world examples. In such a way, Albright pushes the graduating class to work to solve contemporary issues.

    \paragraph{II.} First of all, Secretary Albright reiterates phrases that start with we, followed by a verb. Such examples include, ``We have built a growing world economy,\dots We could stop there,\dots We are pursuing,\dots'' Such use of the collective pronoun we not only strengthens her sentence flow and structure, but also unifies the audience and herself, thus making a unifying, well-defined syntax throughout the entire piece. Furthermore, Albright builds on the aforementioned connection by encouraging the upcoming class to break barriers. Albright uses phrases such as, ``have courage still$-$and persevere.'' This phrase appears at the end of three consecutive paragraphs. Not only does Albright build on her structure with this, but she also encourages the women to break the metaphorical glass ceiling. As such, Albright conveys her ideas well by connecting with the crowd and repeating phrases.

    \paragraph{III.} In addition to this, Albright establishes an emotional connection with the class by using phrases such as, ``if you aim high enough.'' Such phrases allow her to emotionally connect with the crowd, as the strong adjectives are used to signify that the struggle for justice for them has been difficult. Therefore, such powerful phrases, which are scattered throughout the piece, truly define and empower the graduating class. Such an emotional connection adds to the value of the piece, as it is aimed to set goals for the women of the global community$-$goals, which Albright suggests, should be fought for relentlessly.

    \paragraph{IV.} Most importantly, Albright supports her stance by giving real-world examples and anecdotes. One such example is Aung San Suu Kyi, a Burmese women to whom Albright refers to as remarkable, while stating that Aung San Suu Kyi risks her life every day in order to preserve liberal ideals. Such an example, coupled with the use of adjectives, such as remarkable, further Albright's point$-$that it is necessary for women to struggle and work together to build an equal society. Finally, Albright lightens the mood by quoting, in a somewhat joking manner, Robert Kennedy, by saying, ``if there's nobody in your way, it's because you're not going anywhere.'' Ergo, by lightening the mood and providing a relative quotation, Albright ties her piece together, in an attempt to make a unified female front for freedom.

    \paragraph{V.} Overall, Albright strings her speech very well. She uses rhetorical devices such as reiteration, pathological connection, and anecdotal ``data.'' As a result, Albright forms a structured, well argumented piece, with the intent to empower the women of this graduating class, and women all around the world, to work to further the progress which they have already made. 

\end{mla}

\end{document}

