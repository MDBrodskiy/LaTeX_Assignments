%%%%%%%%%%%%%%%%%%%%%%%%%%%%%%%%%%%%%%%%%%%%%%%%%%%%%%%%%%%%%%%%%%%%%%%%%%%%%%%%%%%%%%%%%%%%%%%%%%%%%%%%%%%%%%%%%%%%%%%%%%%%%%%%%%%%%%%%%%%%%%%%%%%%%%%%%%%%%%%%%%%%%%%%%%%%%%%%%%%%%%%%%%%%
% Written By Michael Brodskiy
% Class: AP Language & Composition
% Professor: J. Greer
%%%%%%%%%%%%%%%%%%%%%%%%%%%%%%%%%%%%%%%%%%%%%%%%%%%%%%%%%%%%%%%%%%%%%%%%%%%%%%%%%%%%%%%%%%%%%%%%%%%%%%%%%%%%%%%%%%%%%%%%%%%%%%%%%%%%%%%%%%%%%%%%%%%%%%%%%%%%%%%%%%%%%%%%%%%%%%%%%%%%%%%%%%%%

\documentclass[12pt]{article} 
\usepackage{alphalph}
\usepackage[utf8]{inputenc}
\usepackage[russian,english]{babel}
\usepackage{titling}
\usepackage{amsmath}
\usepackage{graphicx}
\usepackage{enumitem}
\usepackage{amssymb}
\usepackage[super]{nth}
\usepackage{everysel}
\usepackage{ragged2e}
\usepackage{geometry}
\usepackage{fancyhdr}
\usepackage{cancel}
\usepackage{siunitx}
\geometry{top=1.0in,bottom=1.0in,left=1.0in,right=1.0in}
\newcommand{\subtitle}[1]{%
  \posttitle{%
    \par\end{center}
    \begin{center}\large#1\end{center}
    \vskip0.5em}%

}
\usepackage{hyperref}
\hypersetup{
colorlinks=true,
linkcolor=blue,
filecolor=magenta,      
urlcolor=blue,
citecolor=blue,
}

\urlstyle{same}


\title{Page 167 Reading Questions}
\date{January 25, 2021}
\author{Michael Brodskiy\\ \small Instructor: Mrs. Greer}

% Mathematical Operations:

% Sum: $$\sum_{n=a}^{b} f(x) $$
% Integral: $$\int_{lower}^{upper} f(x) dx$$
% Limit: $$\lim_{x\to\infty} f(x)$$

\begin{document}

    \maketitle

    \begin{enumerate}

      \item The writer's analogy is quite fitting in the context. In this case, one could mention any sort of tool and explain that it is not the tool that does good or bad, but the user. Many organizations that base their beliefs on this exist (it is not the tool's fault). For example, the Free Software Foundation argues for software freedom, as the existence of non-free software is not the software's fault, but the creator's. In this manner, the writer creates an effective introduction.
        
      \item In the second paragraph, the author does not \textit{necessarily} use a counterclaim. This is because, in the author's conclusion, the author ends by saying that it is humans who are responsible for correct use of technology. In this manner, the second paragraph does further the author's argument, but it is not a counterargument.

      \item The author successfully uses a source when they mention Bauerlein, saying that abuse of technology may lead to a closer connection to oneself, rather than the world. The quote is, additionally, more successful because the author continues to expand on this idea following the quote. The use of the Agger quotation, however, did seem somewhat poor. This is because, even though the quotation itself fits the author's point well, the author continues by saying that good judgement is necessary to use the internet well, but the author never expands on this idea. The writer should focus more on using deeper analysis, as the quotes are sometimes rushed, and placed too close together, without a thorough analysis. 

      \item The greatest strength is definitely the flow of the essay as a whole. The author makes it so that the reader can transition to new arguments without having to question where they came from. In this manner, the author keeps the reader attached to reading the article. 

      \item Overall, the essay is executed quite well, so the only thing I could recommend is to strengthen the vocabulary a bit. This would give the argument a better light, as it gives the author an appearance of intelligence, making seem as though the author knows the topic well (not to say that the author does not know the topic).

      \item Most likely, the best source to switch out (or at least update) would be Carr. This is because it is unclear who Carr is, and why he is qualified to talk on this topic. The author uses Carr, saying that incorrect use results in an ``illusion of knowledge.'' Instead, it seems like Howard is a better source, as it discusses contemporary research and studies.

    \end{enumerate}

\end{document}

