%%%%%%%%%%%%%%%%%%%%%%%%%%%%%%%%%%%%%%%%%%%%%%%%%%%%%%%%%%%%%%%%%%%%%%%%%%%%%%%%%%%%%%%%%%%%%%%%%%%%%%%%%%%%%%%%%%%%%%%%%%%%%%%%%%%%%%%%%%%%%%%%%%%%%%%%%%%%%%%%%%%%%%%%%%%%%%%%%%%%%%%%%%%%
% Written By Michael Brodskiy
% Class: AP Language & Composition
% Professor: J. Greer
%%%%%%%%%%%%%%%%%%%%%%%%%%%%%%%%%%%%%%%%%%%%%%%%%%%%%%%%%%%%%%%%%%%%%%%%%%%%%%%%%%%%%%%%%%%%%%%%%%%%%%%%%%%%%%%%%%%%%%%%%%%%%%%%%%%%%%%%%%%%%%%%%%%%%%%%%%%%%%%%%%%%%%%%%%%%%%%%%%%%%%%%%%%%

\documentclass[12pt]{article} 
\usepackage{alphalph}
\usepackage[utf8]{inputenc}
\usepackage[russian,english]{babel}
\usepackage{titling}
\usepackage{amsmath}
\usepackage{graphicx}
\usepackage{enumitem}
\usepackage{amssymb}
\usepackage[super]{nth}
\usepackage{everysel}
\usepackage{ragged2e}
\usepackage{geometry}
\usepackage{fancyhdr}
\usepackage{cancel}
\usepackage{siunitx}
\geometry{top=1.0in,bottom=1.0in,left=1.0in,right=1.0in}
\newcommand{\subtitle}[1]{%
  \posttitle{%
    \par\end{center}
    \begin{center}\large#1\end{center}
    \vskip0.5em}%

}
\usepackage{hyperref}
\hypersetup{
colorlinks=true,
linkcolor=blue,
filecolor=magenta,      
urlcolor=blue,
citecolor=blue,
}

\urlstyle{same}


\title{The Singer Solution to World Poverty \& A Moral Atmosphere Questions}
\date{January 7, 2021}
\author{Michael Brodskiy\\ \small Instructor: Mrs. Greer}

% Mathematical Operations:

% Sum: $$\sum_{n=a}^{b} f(x) $$
% Integral: $$\int_{lower}^{upper} f(x) dx$$
% Limit: $$\lim_{x\to\infty} f(x)$$

\begin{document}

\begin{enumerate}

    \maketitle

    \begin{center}
  \underline{The Singer Solution to World Poverty}
    \end{center}

  \item This does not undermine Singer's argument for two reasons: first of all, it is not clear if he has seen the movie, and, second, it is unclear whether the movie had a price that came along with it. For example, Singer could have heard of the movie plot, rather than seeing it, which prompted the incorporation of the storyline into his essay. Furthermore, it is unclear whether it was necessary to pay to watch this movie. Of course, the argument of ``time is money'' may be brought up, as spending time to watch a movie could be seen as a luxury; however, Singer's main point revolves around liquid capital, and, therefore, this statement is irrelevant.

  \item For the question in paragraph four, the difference between giving away a child and an American is the question of where the money came from, and what actions were done with it. For example, the guilt for Dora comes from knowing she made the money from dark deeds, but in Singer's supposedly analogous situation, the average American earned the money from hard work. In this manner, Singer's main argument is undermined. Furthermore, the money Dora received was then spent on a TV set, whereas the example with Americans sees them spending their money on a TV \textit{instead} of donating. As such, this brings up the economic theory of opportunity cost. This theory states that, when something is exchanged, for example with currency, another thing is forfeited. In this case, donations are forfeited for personal luxuries. In other words, Dora's situation is different because she committed the action that caused the loss of a life, whereas Singer's example did not do an action, which resulted in a death. Overall, it is clear that Singer wants to imply that the choices are not different, as it would further his argument.

  \item Singer assumes that the audience answered his question from the previous paragraph as there being no difference between the two scenarios.

  \item This rhetorical strategy, of a parallel example, is used to strengthen Singer's argument. By providing a scenario in which the audience would agree that the child would be saved, Singer hopes to get the audience to agree that, in both scenarios, a child should be saved.

  \item By providing phone numbers, Singer hopes to demonstrate the ease with which the audience can save a child. As such, it is clear his argument is ethically-centered.

    \begin{center}
  \underline{A Moral Atmosphere}
    \end{center}

  \setcounter{enumi}{0}

  \item Bill McKibben writes this to demonstrate the overuse of such excuses. In this manner, he tries to convince the reader not to use excuses, and to, therefore, listen to his argument.

  \item A and Atmosphere are neutral, while moral holds a positive connotation. As such, Bill McKibben tries to convey this as a positive, moral issue at hand. 

  \item By repeating these criticisms, McKibben makes them harmless to his position. The author purposefully includes this to show how preposterous these statements really are. By highlighting these arguments, he is able to dismantle them, and, consequently strengthen his own position.

  \item By referring to himself as such, McKibben makes himself more credible, as he demonstrates his knowledge of opposing arguments. By acknowledging other viewpoints, McKibben strengthens his argument. McKibben assumes his readers are critics of his position.

  \item McKibben acknowledges his complicity to highlight the fact that, on his own, he is unable to divert the course of climate change. This is quite effective, as, not only does this question the ethical implications of climate change, but it also casts light on the fact that real results are impossible without unity.

\end{enumerate}

\end{document}

