\documentclass[12pt,letterpaper]{article}
\usepackage[utf8]{inputenc}
\usepackage[english]{babel}
\usepackage{ragged2e}
\usepackage[colorlinks = true,
            linkcolor = purple,
            urlcolor  = blue,
            citecolor = blue,
            anchorcolor = blue]{hyperref}
\hypersetup{
    colorlinks=true,
    linkcolor=blue,
    filecolor=blue,      
    urlcolor=blue,} 
\urlstyle{same}


\usepackage{ifpdf}
\usepackage{mla}

\begin{document}

\begin{mla}{Michael}{Brodskiy}{Mrs. Greer}{AP Language}{January 26, 2021}{\underline{James Baldwin Argument Essay}} 

  \begin{justify}

    \paragraph{} Throughout all of history, humanity has always been locked in a power struggle. Like the common locution states, great power comes with great responsibility. Such words define the struggle for power and dominance which exists throughout humanity. In this manner, it is evident that there truly exists a struggle for domination, rooted deep in human nature, as, commonly, people resort to fallacious \textit{Ad Hominem} arguments to defame, and therefore gain power over, other people.

  \end{justify}

  \begin{justify}

    \paragraph{} Such personal ``at person'' attacks are nothing other than their name $-$ simple attacks on a person based on their identity, rather than argumentation and logic. Attacks of such caliber are clearly seen in the novel \textit{The Scarlet Letter}, where the patriarchal, religiously-dominated society seeks out and defames those who do not conform. \textit{The Scarlet Letter} holds these personal attacks as a key theme, as, rather than a certain, single instance, the whole novel is based on a power struggle. With the ``high and almighty'' ruling class present in this New England town, dissidence is not an option, as it becomes punishable to extraordinary extents. As such, this whole society is evidently built on a power struggle, and, in this manner, it is clear that those at the helm would go to great lengths to prevent others from toppling them from their positions. Such is the intrinsic nature of the Puritan campaign against Hester Prynne by the ruling class. This struggle for power is well-summarized in a quotation towards the end of Source A, which states ``But there was a more real life for Hester Prynne here, in New England, than in that unknown region where Pearl had found a home. Here had been her sin; here, her sorrow; and here was yet to be her penitence. She had returned, therefore, and resumed,�of her own free will, for not the sternest magistrate of that iron period would have imposed it,�resumed the symbol of which we have related so dark a tale. Never afterwards did it quit her bosom. But \ldots the scarlet letter ceased to be a stigma which attracted the world�s scorn and bitterness, and became a type of something to be sorrowed over, and looked upon with awe, and yet with reverence, too.'' Although long, this quote perfectly encapsulates the ideals of personal attacks. Although defamation can influence one's economic and social position, it ultimately starts from one rejecting the slander, and coming to terms that, now that it has been released, it may become a permanent issue. The quote gives lots of much needed emphasis to the fact that, when a person stops giving value to something, it loses any worth, as people define value on their own terms.

  \end{justify}

  \begin{justify}

    \paragraph{} Additionally, a plethora of events perspire in the contemporary digital realm. For example, Guy Babcock was the victim of a slander campaign by one of his former employees. According to Source B, this person is, ``apparently content to revel in ancient grievances, delighting in legal process and unending conflict because of the misery and expense it causes for her opponent.'' The key word here is misery, as, by instilling these negative feelings, the author of this post is able to drag down others with them $-$ the definition of an internet troll. Except the author of these posts was a rampant troll. Commonly frequenting forum websites, this troll was able to post uncensored, false, slanderous accounts of people who had in any way, shape or form caused their failures, instead of honing up to their mistakes. This person takes out their frustration and anger by demeaning others, which, in the author's mind, makes them dominant over those people. In addition, many examples exist outside of this instance.

  \end{justify}

  \begin{justify}

    \paragraph{} What do people automatically imagine when they think of a country, especially the United States? The president. Why? Because people are innately power-hungry, searching for every attempt to gain a position higher than previous one. Whether it be one party or the other, both partake in slandering to further their beliefs and their candidates, exactly for the reason of establishing power. Furthermore, turning on the news lately would yield terrifying results. A community of people had decided to invest in various stocks, such as Gamestop and AMC Theaters, and, when the people in the lower classes began winning, what happened? The elite, who began losing, aimed at those who had caused their failures, much like Guy Babcock's story. Releasing slanderous news stories, and large disinformation campaigns, the Wall Street Elite was able to convince many to leave this community, or risk being labeled greedy and power-hungry (hypocritical, right?).

  \end{justify}

  \begin{justify}

    \paragraph{} Overall, it is evident that there are always those in power, and those in power do not want to lose said power. As such, the argument stands tall that there always exists a struggle to achieve dominance over others, which most commonly results in \textit{Ad Hominem} slander, rather than real, logical argumentation.

  \end{justify}


\end{mla}

\end{document}

