\documentclass[12pt,letterpaper]{article}
\usepackage[utf8]{inputenc}
\usepackage[main=english,russian]{babel}
\usepackage{lastpage}
\usepackage{ragged2e}
\usepackage[super]{nth}
\usepackage[colorlinks = true,
            linkcolor = purple,
            urlcolor  = blue,
            citecolor = blue,
            anchorcolor = blue]{hyperref}
\hypersetup{
    colorlinks=true,
    linkcolor=blue,
    filecolor=blue,      
    urlcolor=blue,} 
\urlstyle{same}


\usepackage{ifpdf}
\usepackage{mla}

\begin{document}

\begin{mla}{Michael}{Brodskiy}{Mrs. Greer}{AP Language}{December 15, 2020}{\underline{Fall Midterm Assignment — Book Analysis}} 

  \begin{abstract}
    \begin{justify}
      The Second World War — the eastern front of which was known in Eastern Europe as the Great Patriotic War (\textit{\otherlanguage{russian}{Великая Отечественная Война}}) —  was unarguably the greatest and most violent military engagement of all time. Throughout this period, as well as the decades leading up to it, a great deal of military and political philosophy was circling. The mechanization of the First World War proved that it was possible to include heavy industry in the war effort. As such, many proponents of industrial war efforts published prolific works arguing for the use of machines, and tactics using those machines. In France, de Gaulle, in Germany, Guderian, and, the book, \textit{Architect of Soviet Victory in World War II}, argues that in Russia, Georgii Samoilovich Isserson was the greatest war mind. In the novel, it is said that Isserson was responsible for the success of the Red Army. This document will further explicate and support the claim of the book.
  \end{justify}
 \end{abstract}

 \begin{center}
   \underline{On the Tactical Victories of the Soviet Union Throughout the Great Patriotic War}
  \end{center}

  \begin{justify}
    \quad The Great Patriotic War, a time of uncertainty and great loss of life, was only beginning with the successful 1941 invasion. Operation Barbarossa — a reversal of the three-pronged 1914 Brusilov Offensive — progressed as planned, with significant German victories up to twenty kilometers from Moscow. Now, although the novel itself outlines Isserson's life from a very young age, it is made clear, especially in the preface, that the main argument is that Isserson was responsible for much of the success of Soviet military operations. This would prove to be true, albeit initially extremely unsuccessful, as shown in the battles that occurred in late 1941 and onward.
  \end{justify}

  \begin{justify}
    \vspace{-12pt}\quad It is inarguable that, upon commencement of the Barbarossa plan, Soviet forces saw extreme and utter failure. Losing nearly every single battle, the eagle\footnote{\textit{In reference to the symbol of the German Empire}} moved forth, deeper and deeper into Soviet territory, as demonstrated by the map on \href{English.Pobediteli.ru}{the website from source one} on page \pageref{LastPage}.  The question lies in what caused the initial failure. Although Isserson's theories could be to blame, it is evident that this is not true for two reasons: first, Soviet high command was extremely reluctant to accept the intelligence they had received — which suggested an imminent invasion — mainly because they hoped that their pact would last. Second, Isserson's tactics would prove effective as time went on. The initial stagnation would be overcome, and anti-German sentiment would cause victories later on in the war. As such, there can be no question of whether Isserson was responsible for the initial defeats. Furthermore, many opponents of Isserson and his theories commonly question why Isserson was not praised for his commitment and aid of the war effort. This argument, however, is logically fallacious, as many reasons may contribute to someone not being credited with carrying out a task. In the case of Isserson, his Jewish heritage and the anti-semitic Soviet sentiment would cause him to be pushed out of the light. As such, it is clear that Isserson is not responsible for the initial failures, signifying his theories did not negatively affect the war, and, furthermore, that his relatively unknown character is not significant of his success.
  \end{justify}

  \begin{justify}
    \vspace{-12pt}\quad The book, as well as numerous other sources, link Isserson to the idea of the “\textit{Deep Operation},” or \textit{\otherlanguage{russian}{Глубокая Операция}}, which was utilized throughout the war. At the start of the war, the “Deep War” theory was just that — a theory. The start of a war of large magnitude would make it possible to truly test out these theories to the maximum possible extent. The first major Soviet victory utilized these “deep” theories. Of course, this was the counteroffensive of the winter of 1941, when the Germans were pushed back from within 20 kilometers of Moscow. This offensive consisted of a massive push in which Soviet forces penetrated the left and right, or south and north flanks, respectively. This caused the formation of a salient, which was quickly realized by the German Obercommand (high command). This caused a massive retreat, with the reversed movement of infantry and tank divisions. Richard W. Harrison covers this extensively in \textit{Architect of Soviet Victory in World War II}, and in much more detail. In summary, Harrison states that Isserson's “Deep War” theory was applied to the Moscow offensive, and that is what made it so successful. The superior military theory, coupled with the anti-German, nationalist sentiment caused by the violence and war crimes committed upon their people, would make the Soviet force victorious, as the tide of the war turned. The Moscow offensive, however, is not the only example of Isserson's influence over the war.
  \end{justify}

  \begin{justify}
    \vspace{-12pt}\quad In addition to the success of the counteroffensive, Isserson's theories would show to be more successful as time went on. The Battle of Proharovka, more commonly known as Kursk, would be the deciding battle of the war for Germany. As the Moscow counteroffensive signified the first major loss of the Wehrmacht\footnote{\textit{The name given to the German infantry armies}}, the Battle of Kursk signified the near-total defeat of the Wehrmacht. Richard W. Harrison, who also authored \textit{Architect of Soviet Victory in World War II}, wrote a heavily detailed book entitled, \textit{The Battle of Kursk: The Red Army's Defensive Operations and Counter-Offensive, July–August 1943.} The book discusses offensive tactics, victories, and defensive strategies. The main offensive tactic, Georgy Zhukov's pincer movement, was an important factor in the Soviet win at Proharovka. This tactic involved the surrounding of a salient, and subsequent cutoff of the enemy troops within the aforementioned salient. The description of this military maneuver is quite reminiscent of the “Deep War” strategy described above. Therefore, it is clear that Isserson's operational theories were the main influence behind Russo-Soviet military thought in the early-to-mid \nth{20} century.
  \end{justify}

  \begin{justify}
    \vspace{-12pt}\quad It is clear that, throughout the war, Isserson was not guilty of creating a faulty military layout, nor did he remain neutral, and, therefore, irrelevant. This, coupled with the fact that he created a military infrastructure that was employed in the most successful offensives and counteroffensives, displays the magnitude of positive effect which Isserson carried with him. Ergo, it is evident that Isserson, by creating his “Deep War” strategy, was responsible for nearly all military successes throughout the war, and, therefore, he truly is the architect of Soviet victory. 
  \end{justify}

  \begin{center}
    Word Count:\\\vspace{10pt}
    \begin{tabular}{| c | c |}
      \hline
      Section & Word Count\\
      \hline
      Abstract & 159\\
      \hline
      Essay & 863 \\
      \hline
      Essay w/Abstract & 1,022\\
     \hline
    \end{tabular}
  \end{center}

  \newpage

  \begin{center}
    Sources Cited:
  \end{center}

  \begin{enumerate}

    \item \textit{Pobediteli}, english.pobediteli.ru/

    \item Harrison, Richard W.\hspace{5pt} \textit{Architect of Soviet Victory in World War II: The Life and Theories of G.S. Isserson.} Jefferson, N.C., Mcfarland \& Co, 2010.

    \item Harrison, Richard W.\hspace{5pt} \textit{The Battle of Kursk: The Red Army's Defensive Operations and Counter-Offensive, July–August 1943.} Helion \& Company Limited.

  \end{enumerate}

\end{mla}

\end{document}

