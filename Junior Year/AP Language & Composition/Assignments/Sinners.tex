\documentclass[12pt,letterpaper]{article}
\usepackage[utf8]{inputenc}
\usepackage[english]{babel}
\usepackage{ragged2e}
\usepackage[colorlinks = true,
            linkcolor = purple,
            urlcolor  = blue,
            citecolor = blue,
            anchorcolor = blue]{hyperref}
\hypersetup{
    colorlinks=true,
    linkcolor=blue,
    filecolor=blue,      
    urlcolor=blue,} 
\urlstyle{same}


\usepackage{ifpdf}
\usepackage{mla}

\begin{document}

\begin{mla}{Michael}{Brodskiy}{Mrs. Greer}{AP Language}{November 9, 2020}{\underline{Sinners in the Hands of an Angry God}}


  \begin{justifying}

    \paragraph{} The Great Awakening, which had occurred during a time when religious morale was in decline in the English colonies, saw the rise of religion unlike anything before. People who were inactively religious, or even agnostic began to listen to the constant preaching of new priests and theologians. One character through whom the revival flowed was Jonathan Edwards, a devout Puritan. In the mid 1700s, Edwards preached to the masses, telling them of the angry god they had awoken. Throughout his sermon, Edwards, through anaphoric line structure, fearmongering, and a grave tone, asserts that the people must indefinitely follow God, for every day the people believe in God less, God protects them less.

    \paragraph{} Within his sermon, Edwards repeatedly reminds his listeners that, if the people do not conform, the wrath of God will be released upon them. For instance, Edwards reiterates phrases that begin with ``nothing'' in order to drill it into his listeners that, if not for God, they would be doomed. Edwards states, ``[you have] nothing of your own, nothing that ever have done, nothing that you can do, to induce God to spare you one moment.'' As such, Edwards, while building up an image of God as a celestial, superior being, condemns the people as puny and not worthy of God's time. Furthermore, Edwards vividly describes events of mass destruction, stating that it is within God's power and right to do this to an unworthy people. Therefore, such repetitive use of imagery and belittling bullies the people into submission, ultimately making them frightened, not respectful of God.

    \paragraph{} In such a manner, Edwards continues to frighten his audience, further prompting them to bow to the will of Edward's description of God. As the speech continues, the people become more and more disconcerted, shaking in their spots. It is Edward's intention to elicit such emotional, pathos-related responses, as it makes the audience easier to manipulate. Edward's clearly makes frightening descriptions, saying that the fury of the omnipotent God is ten thousand times stronger than the sturdiest devil. Edward's furthers his position by stating that the people have offended God, and that, in God's eyes, the people ``are ten thousand times so abominable in his eyes, as the most hateful and venomous serpent is in ours.'' Such perturbing descriptions are simply used to make people look for a solution$-$one that Edwards will offer to them. Ergo, Edwards lures people to submit to God, using these fearmongering, anxiety-inducing descriptions, as well as a grave tone.

    \paragraph{} In addition to repetitive descriptions and scare tactics, Edwards couples these with a grave and anxious tone. Edwards intentionally tosses in words such as ``wrath,'' ``everlasting destruction,'' and ``damned''  to magnify the fear already present within his audience. By mixing such powerful adjectives in with his repetition and scary descriptions, Edwards fabricates an audience interested in one thing: themselves. Such appeal to human nature causes the audience to beg Edwards for a solution, which he already has: a strict adherence to Puritanism. It as at this point that Edwards achieves what he sought: an audience of people ready to follow his, and now their, God. 

    \paragraph{} Overall, such powerful sermons spread throughout all of the English colonies during the Great Awakening. Such speeches were a great driving force behind the mass revival of civilians. The tactics of repetitive syntax, frightful descriptions, and the tone of impending doom was used by Edwards, as well as many other religious revivalists, who sought to revive those who were already their religion, or convert those who were not part of it yet. 

\end{mla}

\end{document}

