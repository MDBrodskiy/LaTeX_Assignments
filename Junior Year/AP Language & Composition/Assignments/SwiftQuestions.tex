%%%%%%%%%%%%%%%%%%%%%%%%%%%%%%%%%%%%%%%%%%%%%%%%%%%%%%%%%%%%%%%%%%%%%%%%%%%%%%%%%%%%%%%%%%%%%%%%%%%%%%%%%%%%%%%%%%%%%%%%%%%%%%%%%%%%%%%%%%%%%%%%%%%%%%%%%%%%%%%%%%%%%%%%%%%%%%%%%%%%%%%%%%%%
% Written By Michael Brodskiy
% Class: AP Language & Composition
% Professor: J. Greer
%%%%%%%%%%%%%%%%%%%%%%%%%%%%%%%%%%%%%%%%%%%%%%%%%%%%%%%%%%%%%%%%%%%%%%%%%%%%%%%%%%%%%%%%%%%%%%%%%%%%%%%%%%%%%%%%%%%%%%%%%%%%%%%%%%%%%%%%%%%%%%%%%%%%%%%%%%%%%%%%%%%%%%%%%%%%%%%%%%%%%%%%%%%%

\documentclass[12pt]{article} 
\usepackage{alphalph}
\usepackage[utf8]{inputenc}
\usepackage[russian,english]{babel}
\usepackage{titling}
\usepackage{amsmath}
\usepackage{graphicx}
\usepackage{enumitem}
\usepackage{amssymb}
\usepackage[super]{nth}
\usepackage{everysel}
\usepackage{ragged2e}
\usepackage{geometry}
\usepackage{fancyhdr}
\usepackage{cancel}
\usepackage{siunitx}
\geometry{top=1.0in,bottom=1.0in,left=1.0in,right=1.0in}
\newcommand{\subtitle}[1]{%
  \posttitle{%
    \par\end{center}
    \begin{center}\large#1\end{center}
    \vskip0.5em}%

}
\usepackage{hyperref}
\hypersetup{
colorlinks=true,
linkcolor=blue,
filecolor=magenta,      
urlcolor=blue,
citecolor=blue,
}

\urlstyle{same}


\title{Swift Questions}
\date{February 23, 2021}
\author{Michael Brodskiy\\ \small Instructor: Mrs. Greer}

% Mathematical Operations:

% Sum: $$\sum_{n=a}^{b} f(x) $$
% Integral: $$\int_{lower}^{upper} f(x) dx$$
% Limit: $$\lim_{x\to\infty} f(x)$$

\begin{document}

    \maketitle

    \begin{enumerate}

      \item In Jonathan Swift's satire, entitled \textit{A Modest Proposal}, Swift ridicules government policy, while trying to call the people of Ireland into action. Swift does this by beginning with a description of the poorer class mothers trying to feed their children, with his sympathetic tone bringing any reader to tears; that is, until he begins to propose how to solve the issue of starving children. He then describes a world in which starving children are “saved” from their plight by being eaten by the rich, or even used to pay rent. In this manner, Swift is able to highlight the incompetency of the ruling class, all the while appealing to the common folk.

      \item There is quite a huge boundary between Swift and his narrator. Swift's use of seemingly subtle, but actually satirical terms permit him to create a false identity very different from that of his. This is first and foremost clear from the title, as his request is quite evidently not “modest” — quite on the contrary. The stark difference between his real voice and the narrator adds effectiveness, as those that do not understand the piece's comedic value begin to question whether the piece is serious or not.

      \item It seems as though there are a two distinct targets that can be identified by “scale of government”: City Government (Dublin), and Federal Government (the Kingdom of Britain). This does not, however, detract from his satirical point, as, due to the similar nature of the two (both being related to government), it allows Swift to relate the two.

      \item This quotation demonstrates Swift's hatred of the British rule. As is historically evident, Ireland and Britain have bitter relations. By using the excerpt, Swift shows his support of Ireland by calling Britain a country “which would be glad to eat up our whole nation”, where “our nation” represents Ireland. In this manner, he points out the greediness and evil of the British.

      \item Swift's intended takeaway depends on who the audience is. If the audience is of Irish descent, probably the most important excerpt is when Swift says “I can think of no one objection that will possible be raised against this proposal.” By using such a line, Swift is able to make the reader question their ethics, which, in turn, would make the reader question the ethics of the British government ruling over them. On the other hand, Swift, through the use of the excerpt mentioned in the previous question, wants the British government to feel ashamed, not only for ignoring his letters and request, but also for their unfair ruling of the Irish.

      \item

        \begin{center}
          \begin{tabular}[h]{| p{.2\textwidth} | p{.7\textwidth} |}
            \hline
            Current Events & There is no \textit{single} occurrence that would better explain this piece; however, knowledge of political satire and political theory may help understand this preposterous piece.  \\
            \hline
            History & It is important to know of the deep tensions that run between the Irish and British, as it explains some of the author's bitterness towards the British monarchy.  \\
            \hline
            Experiences & Having knowledge of the balance between government power and citizen power, or, in other words, security versus freedom, would help understanding his mockery of government.  \\
            \hline
            Literature & It is good to know Swift's works to explain why he wrote this piece. For instance, if one does not know that Swift wrote satirical pieces, they may think the piece is serious.  \\
            \hline
            Pop Culture & Contemporary political cartoons may be compared to Swift's work to show that satire has been a human practices for centuries.  \\
            \hline
            Science & The piece is not really scientifically related.  \\
            \hline
            Sports & The piece is not really sports-related.  \\
            \hline
          \end{tabular}
        \end{center}

    \end{enumerate}

\end{document}

