\documentclass[12pt,letterpaper]{article}
\usepackage[utf8]{inputenc}
\usepackage[english]{babel}
\usepackage{ragged2e}
\usepackage[colorlinks = true,
            linkcolor = purple,
            urlcolor  = blue,
            citecolor = blue,
            anchorcolor = blue]{hyperref}
\hypersetup{
    colorlinks=true,
    linkcolor=blue,
    filecolor=blue,      
    urlcolor=blue,} 
\urlstyle{same}


\usepackage{ifpdf}
\usepackage{mla}

\begin{document}

\begin{mla}{Michael}{Brodskiy}{Mrs. Greer}{AP Language}{October 8, 2020}{\underline{Analyzing Political Cartoons}} 

  \begin{justifying}

    \paragraph{I.} In S.J. Ray's and K.C. Star's cartoon entitled, ``The Rude Awakening,'' Adolf Hitler is depicted in a dazed, perturbed state as a result of the Eastern front closing in. The bear in this figure serves the purpose of a hyperbolic metaphor, the symbol of Russians, the predominant ethic subculture of the former Soviet Union, with Jaws wide open, ready, to snap and questioning Hitler's ``\emph{intuition}'', in reference to the defeat of the German Sixth Army.% and the following Westward push by Soviet Armies.
    \paragraph{II.} Hitler in bed, frightened and in an inherently vulnerable position while the behemoth labeled Russia stares menacingly. The bear taunts Hitler's inevitable series of defeats starting in early 1943. Hitler's ``\emph{intuition}'', \textit{blitzkrieg}, enabled his armies to traverse the European republics of the Soviet Union, but failed in the long-run, and \textit{rattenkrieg} took it's place in the ruined Soviet cities. This composition serves as a symbol of shame and defeat, and a foreboding warning for the future, which would see the collapse of the Nazist regime at the hands of the Soviet Union.
    \paragraph{III.} This image's intent is clear $-$ it is meant to mock the dwindling Nazist regime $-$ ultimately causing a decline in morale. The ominous bear, hungry for a meal (i.e. \emph{victory}), is much like the Soviet Union. The bed in which Hitler is laying represents both his and the bear's awakening. As such, the cartoon argues that Hitler has raised a force to be reckoned with. The expression of sheer horror on Hitler's face exemplifies the premonition of destruction and conveys the idea that he realized his ``\emph{intuition}'' had gone too far. Ergo, Ray's and Star's cartoon would be not only a prediction of the short-term future, but also the decades of tension to come, with the great awakening of an industrially dominant superpower regime.


    \begin{figure}[h]
      \centering
      \includegraphics[width=.6\textwidth]{../Figures/Stalingrad.jpg}
      \caption{Ray, S.J. and Star, K.C. ``The Rude Awakening'' [\textit{illegible}] City Journal, 1943}
      \label{fig:1}
    \end{figure}


\end{justifying}
\centering Word Count: 299

\end{mla}

\end{document}

