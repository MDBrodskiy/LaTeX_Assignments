\documentclass[12pt,letterpaper]{article}
\usepackage[utf8]{inputenc}
\usepackage[english]{babel}
\usepackage{ragged2e}
\usepackage[colorlinks = true,
            linkcolor = purple,
            urlcolor  = blue,
            citecolor = blue,
            anchorcolor = blue]{hyperref}
\hypersetup{
    colorlinks=true,
    linkcolor=blue,
    filecolor=blue,      
    urlcolor=blue,} 
\urlstyle{same}


\usepackage{ifpdf}
\usepackage{mla}

\begin{document}

\begin{mla}{Michael}{Brodskiy}{Mrs. Greer}{AP Language}{January 26, 2021}{\underline{James Baldwin Argument Essay}} 

  \begin{justify}
    \textbf{The paragraph below comes from a 1979 essay by expatriate African American writer James Baldwin. Read the paragraph carefully and then write an essay that defends, challenges, or qualifies Baldwin's ideas about the importance of language as a "key to identity" and to social acceptance. Use specific evidence from your observation, experience, or reading to develop your position.}\\
  \end{justify}

  \begin{justify}
It goes with saying, then, that language is also a political instrument, means, and proof of power. It is the most vivid and crucial key to identity: It reveals the private identity, and connects one with, or divorces one from, the larger, public, or communal identity. There have been, and are, times, and places, when to speak a certain language could be dangerous, even fatal. Or, one may speak the same language, but in such a way that one's antecedents are revealed, or (one hopes) hidden. This is true in France, and is absolutely true in England: The range (and reign) of accents on that damp little island makes England coherent for the English and totally incomprehensible for everyone else. To open your mouth in England is (if I may use black English) to "put your business in the street": You have confessed your parents, your youth, your school, your salary, your self-esteem, and, alas, your future.
  \end{justify}
  
  \hline

  \begin{justify}

    \paragraph{} As Baldwin states, any language one speaks, writes, or in any way utilizes, no matter how colorful, automatically identifies the user. Why? Well, for one, by reading that first sentence, it is automatically clear I am an English speaker $-$ this is inarguable. Upon further investigation, my spelling, like the us of colorful without a ``u'' signifies I am most likely not British, or any derivative thereof; in this manner, I must have learned American English. Language defines us, whether it be our first language or not, it automatically causes others to interpret us, and thereby become predisposed. The importance of language is evident in various daily events, whether it be through verbal discussion, related to political turmoil, or written text. As such, Baldwin's argument stands tall.  \\

  \end{justify}

  \begin{justify}

    \paragraph{} First of all, common behaviors signify that humans are automatically influenced by other peoples' commands of language. For example, a person orders a taxi. Said person is a traveler from a different country, and, therefore, is not a native-born speaker of the country's language(s). Upon a commencement of discussion between the local taxi driver and the traveler, it would become evident that the traveler has an accent, prompting the common question, ``That accent. Where are you from?'' This is only one subconscious example of human behavior. Humans are hard-wired to recognize differences, as this could have meant the difference between survival and death, if, for example, an enemy tribe were to invade. Such built-in behaviors are extremely prevalent. \\

  \end{justify}

  \begin{justify}

    \paragraph{} The aforementioned behavior goes farther than just civil conversations. For example, conflicts are often provoked through peoples' accents. A modern example is that of the conflict between the British and Irish. The tensions between these two factions have risen to such a point that, upon hearing a foreign accent, a fight may automatically break out. What caused this? Of course, the deeply-rooted political tensions played a role in the actual build up of hatred, but what made it possible to identify a rival? The accent. Their language. Had they not, in Baldwin's words, ``confessed your parents, your youth, your school, your salary, your self-esteem, and, alas, your future.'' By speaking in this hypothetical situation, the person did just that. Again, Baldwin states, ``There have been, and are, times, and places, when to speak a certain language could be dangerous.'' This is the exact reason language is so powerful: it doesn't just convey ideas, thoughts, and beliefs, it conveys \underline{\textbf{you}} as a person. Such a relationship exists between the language and a being, and, furthermore, this is what sets humans apart from animals $-$ we can differentiate ourselves through the use of language mechanics, and, therefore, convey ourselves as a being. In addition to this, the analysis of written text is another aspect of language that can be used to identify a person.\\

  \end{justify}

  \begin{justify}

    \paragraph{} Alongside the spoken word lay the written word. The written word is a powerful tool that may be used to either introduce or conclude. Nowadays, prior to an interview, what gives someone an impression of you? A r\'esum\'e, or written language. What is done right after the interview? A thank you letter is sent. These aren't just empty, common practices. These actions were developed to convey yourself to others, or, in Baldwin's words, ``It reveals the private identity, and connects one with, or divorces one from \ldots communal identity.'' This is exactly the point of the r\'esum\'e and thank you letter: not to just portray one's strengths and appear kind, but to show your interconnectedness with others. Showing one's niche within a community can be the difference between an acceptance to a job or university and a rejection. A strong command of language, then, in Baldwin's words, is ``proof of power.''\\

  \end{justify}

  \begin{justify}

    \paragraph{} Overall, it is evident that language is the definition of identity. Not only does it makes humans just what they are, humans, but it conveys people. Accents carry history, heritage, and, even, an individual's future. At times, language is danger. At other times, language is power. But always, no matter what, language is identity.\\

  \end{justify}


\end{mla}

\end{document}

