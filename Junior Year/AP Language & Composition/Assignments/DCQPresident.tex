%%%%%%%%%%%%%%%%%%%%%%%%%%%%%%%%%%%%%%%%%%%%%%%%%%%%%%%%%%%%%%%%%%%%%%%%%%%%%%%%%%%%%%%%%%%%%%%%%%%%%%%%%%%%%%%%%%%%%%%%%%%%%%%%%%%%%%%%%%%%%%%%%%%%%%%%%%%%%%%%%%%%%%%%%%%%%%%%%%%%%%%%%%%%
% Written By Michael Brodskiy
% Class: AP Language & Composition
% Professor: J. Greer
%%%%%%%%%%%%%%%%%%%%%%%%%%%%%%%%%%%%%%%%%%%%%%%%%%%%%%%%%%%%%%%%%%%%%%%%%%%%%%%%%%%%%%%%%%%%%%%%%%%%%%%%%%%%%%%%%%%%%%%%%%%%%%%%%%%%%%%%%%%%%%%%%%%%%%%%%%%%%%%%%%%%%%%%%%%%%%%%%%%%%%%%%%%%

\documentclass[12pt]{article} 
\usepackage{alphalph}
\usepackage[utf8]{inputenc}
\usepackage[russian,english]{babel}
\usepackage{titling}
\usepackage{amsmath}
\usepackage{graphicx}
\usepackage{enumitem}
\usepackage{amssymb}
\usepackage[super]{nth}
\usepackage{everysel}
\usepackage{ragged2e}
\usepackage{geometry}
\usepackage{fancyhdr}
\usepackage{cancel}
\usepackage{siunitx}
\geometry{top=1.0in,bottom=1.0in,left=1.0in,right=1.0in}
\newcommand{\subtitle}[1]{%
  \posttitle{%
    \par\end{center}
    \begin{center}\large#1\end{center}
    \vskip0.5em}%

}
\usepackage{hyperref}
\hypersetup{
colorlinks=true,
linkcolor=blue,
filecolor=magenta,      
urlcolor=blue,
citecolor=blue,
}

\urlstyle{same}


\title{Defend, Challenge, or Qualify Presidents}
\date{February 23, 2021}
\author{Michael Brodskiy\\ \small Instructor: Mrs. Greer}

% Mathematical Operations:

% Sum: $$\sum_{n=a}^{b} f(x) $$
% Integral: $$\int_{lower}^{upper} f(x) dx$$
% Limit: $$\lim_{x\to\infty} f(x)$$

\begin{document}

    \maketitle

    \begin{enumerate}

        \hline

        \begin{quote}
          \textsc{Nearly all men can stand adversity, but if you want to test a man's character, give him power.} 
        \end{quote} \vspace{-10pt}\hspace{325pt}— Abraham Lincoln

      \item \textbf{Challenge}: I do not agree with “nearly all men can stand adversity.” If the quote was simply, “if you want to test a man's character, give him power,” then it would be pretty accurate; however, many people are unable to, or are not very good at withstanding diversity, and, therefore, not all men can stand it.

        \vspace{10pt}
        \hline

        \begin{quote}
          \textsc{If your actions inspire others to dream more, learn more, do more, and become more, you are a leader.} 
        \end{quote} \vspace{-10pt}\hspace{325pt}— John Quincy Adams

      \item \textbf{Defend}: Leadership is defined as “the ability of an individual to lead, influence, or guide other individuals.” By \underline{influencing} others' aspirations through one's own actions, one is clearly a leader.

        \vspace{10pt}
        \hline

        \begin{quote}
          \textsc{Don't expect to build up the weak by pulling down the strong.} 
        \end{quote} \hspace{325pt}— Calvin Coolidge

      \item \textbf{Defend}: Through my own experiences, society tries to do exactly what Coolidge states. This is evident in many aspects of life, such as education, where, often, leading minds are held back in order level the playing field, with the “no child left behind policy” (in reality, it is more like “no child pushed ahead”). It is clear the aforementioned policy does not work, and, in this manner, Coolidge's quote stands true.

        \vspace{10pt}
        \hline
        \newpage
        \hline

        \begin{quote}
          \textsc{Patriotism means to stand by the country. It does not mean to stand by the President} 
        \end{quote} \vspace{-10pt}\hspace{325pt}— Theodore Roosevelt

      \item \textbf{Qualify}: For the most part, I agree with the statement; however, I think it is important to amend this, adding “or a political party” after “President”. This is because, especially modern day, issues have become very polarized, causing many to make partisan issues out of common events. In this manner, Roosevelt's quote would better fit contemporary life with the previously mentioned addition.

        \vspace{10pt}
        \hline

        \begin{quote}
          \textsc{Liberty without learning is always in peril. and learning without liberty is always in vain.} 
        \end{quote} \vspace{-10pt}\hspace{325pt}— John F. Kennedy

      \item \textbf{Challenge}: The first part of the excerpt makes sense, as, an uneducated public will make unwise decisions under a liberal system, but “learning without liberty” is not always in vain. True scholars take any opportunity they can to learn or obtain knowledge, which means, even under a non-liberal system, scholars may still learn.

        \vspace{10pt}
        \hline

    \end{enumerate}

\end{document}

