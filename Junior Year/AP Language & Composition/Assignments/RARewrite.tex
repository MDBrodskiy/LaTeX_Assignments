\documentclass[12pt,letterpaper]{article}
\usepackage[utf8]{inputenc}
\usepackage[english]{babel}
\usepackage{ragged2e}
\usepackage[colorlinks = true,
            linkcolor = purple,
            urlcolor  = blue,
            citecolor = blue,
            anchorcolor = blue]{hyperref}
\hypersetup{
    colorlinks=true,
    linkcolor=blue,
    filecolor=blue,      
    urlcolor=blue,} 
\urlstyle{same}


\usepackage{ifpdf}
\usepackage{mla}

\begin{document}

\begin{mla}{Michael}{Brodskiy}{Mrs. Greer}{AP Language}{\today}{\underline{}} 

  \begin{justifying}

    \begin{enumerate}

      \item \underline{Before}: At a time in which the world was enveloped in peace, with decades of war over, Secretary of State Madeleine Albright gave a commencement speech at Mount Holyoke College, an all female school. Throughout her speech, Madeleine empowers women, whether it be in this school, or women across the globe, by adopting a cogent, powerful tone, through which she delivers a repetitive syntax, uses a connection based on pathos to encourage the graduating class, and real-world examples. In such a way, Albright pushes the graduating class to work to solve contemporary issues.

        \begin{justifying}

          \hspace{10pt}
          \underline{After}: During a golden age of peace for America, Secretary of State Madeleine Albright commemorated a graduating class from Mount Holyoke College, an all female school, by performing a commencement speech. Within the speech, Albright employs an eloquent, powerful tone to empower the graduating class, utilizes repetitive syntax and parallel structure to solidify her stance, and exercises a pathological connection with the audience in order to push the graduating class to be trailblazers for solving of contemporary issues.

        \end{justifying}

        \vspace{10pt}

        \hline

        \vspace{10pt}


      \item \underline{Before}: Overall, Albright strings her speech very well. She uses rhetorical devices such as reiteration, pathological connection, and anecdotal ``data.'' As a result, Albright forms a structured, well argumented piece, with the intent to empower the women of this graduating class, and women all around the world, to work to further the progress which they have already made. 

        \begin{justifying}

          \hspace{10pt}
          \underline{After}: Drawing from classical sociopolitical philosophy such as Locke, Madeleine Albright promotes the fight for women's rights by equating it to the idea of natural rights$-$rights that are innate in humans. As such, Albright pushes her female counterparts to work for these rights, adding on to this by mentioning her friend Aung San Suu Kyi. By setting such a scene, Albright influences the young, anxious graduates by conveying the implications of not following the fight for rights. Ultimately, the emotional connection which the students form with respect to the fight for rights in the early 2000s$-$not so long after the fall of the Berlin Wall, the dissolution of the Soviet Union, and the subsequent Yugoslavian wars$-$is more significant than at any other point in history, which is how Albright is able to communicate the importance of contemporary issues to her audience.

        \end{justifying}

\end{enumerate}

\end{mla}

\end{document}

