%%%%%%%%%%%%%%%%%%%%%%%%%%%%%%%%%%%%%%%%%%%%%%%%%%%%%%%%%%%%%%%%%%%%%%%%%%%%%%%%%%%%%%%%%%%%%%%%%%%%%%%%%%%%%%%%%%%%%%%%%%%%%%%%%%%%%%%%%%%%%%%%%%%%%%%%%%%%%%%%%%%%%%%%%%%%%%%%%%%%%%%%%%%%
% Written By Michael Brodskiy
% Class: AP Language & Composition
% Professor: J. Greer
%%%%%%%%%%%%%%%%%%%%%%%%%%%%%%%%%%%%%%%%%%%%%%%%%%%%%%%%%%%%%%%%%%%%%%%%%%%%%%%%%%%%%%%%%%%%%%%%%%%%%%%%%%%%%%%%%%%%%%%%%%%%%%%%%%%%%%%%%%%%%%%%%%%%%%%%%%%%%%%%%%%%%%%%%%%%%%%%%%%%%%%%%%%%

\documentclass[12pt]{article} 
\usepackage{alphalph}
\usepackage[utf8]{inputenc}
\usepackage[russian,english]{babel}
\usepackage{titling}
\usepackage{amsmath}
\usepackage{graphicx}
\usepackage{enumitem}
\usepackage{amssymb}
\usepackage[super]{nth}
\usepackage{everysel}
\usepackage{ragged2e}
\usepackage{geometry}
\usepackage{fancyhdr}
\usepackage{cancel}
\usepackage{siunitx}
\usepackage{xcolor}
\geometry{top=1.0in,bottom=1.0in,left=1.0in,right=1.0in}
\newcommand{\subtitle}[1]{%
  \posttitle{%
    \par\end{center}
    \begin{center}\large#1\end{center}
    \vskip0.5em}%

}
\usepackage{hyperref}
\hypersetup{
colorlinks=true,
linkcolor=blue,
filecolor=magenta,      
urlcolor=blue,
citecolor=blue,
}

\urlstyle{same}


\title{Comparisons}
\date{March 22, 2021}
\author{Michael Brodskiy\\ \small Instructor: Mrs. Greer}

% Mathematical Operations:

% Sum: $$\sum_{n=a}^{b} f(x) $$
% Integral: $$\int_{lower}^{upper} f(x) dx$$
% Limit: $$\lim_{x\to\infty} f(x)$$

\begin{document}

    \maketitle

    \begin{enumerate}

      \item Horace Mann — from \textit{Report of the Massachusetts Board of Education}

        \begin{justify}
          Horace Mann, nicknamed “the father of American public education”, in his piece entitled \textit{Report of the Massachusetts Board of Education} describes the “Massachusetts Theory” of education. Mann begins his piece with a description of the state of Feudal England and the imbalance of power caused by unequal education levels. Through such an abstraction, Mann arrives at the conclusion that, by creating equal education opportunities, American society will be able to eliminate power imbalance and create a system that works for the people. Through such an argument, Mann intends to explain his education policies to the Massachusetts Board of Education. In this manner, the author convinces the reader that he should be able to carry out his plan to fix power imbalance.
        \end{justify}

      \item Leon Botstein — \textit{Let Teenagers Try Adulthood}

        \begin{justify}
        \end{justify}

      \item \textit{Meditation in Schools Across America} — an infographic

        \begin{justify}
        \end{justify}

      \item Nicholas Wyman — \textit{Why We Desperately Need to Bring Back Vocational Training in Schools}

        \begin{justify}
        \end{justify}

      \item Amanda Ripley — from \textit{What America Can Learn from Smart Schools in Other Countries}

        \begin{justify}
        \end{justify}

      \item Leslie Nguyen-Okwu — \textit{How High Schools Are Demolishing the Classroom}

        \begin{justify}
        \end{justify}

      \item Brentin Mock — from \textit{We Will Pay High School Students to Go to School. And We Will Like It.}

        \begin{justify}
        \end{justify}

      \item Amy Rolph — \textit{This High School Wants to Revolutionize Learning with Technology}

        \begin{justify}
        \end{justify}

    \end{enumerate}

\end{document}

