%%%%%%%%%%%%%%%%%%%%%%%%%%%%%%%%%%%%%%%%%%%%%%%%%%%%%%%%%%%%%%%%%%%%%%%%%%%%%%%%%%%%%%%%%%%%%%%%%%%%%%%%%%%%%%%%%%%%%%%%%%%%%%%%%%%%%%%%%%%%%%%%%%%%%%%%%%%%%%%%%%%%%%%%%%%%%%%%%%%%%%%%%%%%
% Written By Michael Brodskiy
% Class: AP Language & Composition
% Professor: J. Greer
%%%%%%%%%%%%%%%%%%%%%%%%%%%%%%%%%%%%%%%%%%%%%%%%%%%%%%%%%%%%%%%%%%%%%%%%%%%%%%%%%%%%%%%%%%%%%%%%%%%%%%%%%%%%%%%%%%%%%%%%%%%%%%%%%%%%%%%%%%%%%%%%%%%%%%%%%%%%%%%%%%%%%%%%%%%%%%%%%%%%%%%%%%%%

\documentclass[12pt]{article} 
\usepackage{alphalph}
\usepackage[utf8]{inputenc}
\usepackage[russian,english]{babel}
\usepackage{titling}
\usepackage{amsmath}
\usepackage{graphicx}
\usepackage{enumitem}
\usepackage{amssymb}
\usepackage[super]{nth}
\usepackage{everysel}
\usepackage{ragged2e}
\usepackage{geometry}
\usepackage{fancyhdr}
\usepackage{cancel}
\usepackage{siunitx}
\usepackage{xcolor}
\geometry{top=1.0in,bottom=1.0in,left=1.0in,right=1.0in}
\newcommand{\subtitle}[1]{%
  \posttitle{%
    \par\end{center}
    \begin{center}\large#1\end{center}
    \vskip0.5em}%

}
\usepackage{hyperref}
\hypersetup{
colorlinks=true,
linkcolor=blue,
filecolor=magenta,      
urlcolor=blue,
citecolor=blue,
}

\urlstyle{same}


\title{Comparisons}
\date{March 22, 2021}
\author{Michael Brodskiy\\ \small Instructor: Mrs. Greer}

% Mathematical Operations:

% Sum: $$\sum_{n=a}^{b} f(x) $$
% Integral: $$\int_{lower}^{upper} f(x) dx$$
% Limit: $$\lim_{x\to\infty} f(x)$$

\begin{document}

    \maketitle

    \begin{enumerate}

      \item Horace Mann — from \textit{Report of the Massachusetts Board of Education}

        \begin{justify}
          Horace Mann, nicknamed “the father of American public education”, in his piece entitled \textit{Report of the Massachusetts Board of Education} describes the “Massachusetts Theory” of education. Mann begins his piece with a description of the state of Feudal England and the imbalance of power caused by unequal education levels. Through such an abstraction, Mann arrives at the conclusion that, by creating equal education opportunities, American society will be able to eliminate power imbalance and create a system that works for the people. Through such an argument, Mann intends to explain his education policies to the Massachusetts Board of Education. In this manner, the author convinces the reader that he should be able to carry out his plan to fix power imbalance.
        \end{justify}

      \item Leon Botstein — \textit{Let Teenagers Try Adulthood}

        \begin{justify}
          Published in 1999 in \textit{The New York Times} magazine, Leon Botstein, the president of Bard College, argues for reformation of the American High School system. Essentially, Botstein's idea involves evidence of a shift in biological principles — that humans have begun to mature earlier — and that, because of this, students should graduate high school at 16, not 18. Botstein states that high school bureaucracy has formed as a result of fanaticism of high school sports, and that, for this reason, many of those who are in high staff positions in high schools are there for popularity reasons, not qualifications. As such, Botstein wraps up by saying that 16 year olds should be ready for life, and that they should try adulthood.
        \end{justify}

      \item \textit{Meditation in Schools Across America} — an infographic

        \begin{justify}
          Published in 2012 by Edutopia, this infographic describes the results of several studies on meditation in elementary and high schools. Various images accompany different statements, ranging from “High school students practicing daily focused meditation had 25\% fewer class absences” to an “8\% reduction in aggressive behavior”. Although not explicitly stated, it is evident that the graphic is making the argument that more schools should switch to daily meditation.
        \end{justify}

      \item Nicholas Wyman — \textit{Why We Desperately Need to Bring Back Vocational Training in Schools}

        \begin{justify}
          In his 2015 Forbes magazine article, Nicholas Wyman discusses the need for vocational training. Wyman applies Bureau of Labor Statistics (BLS) information to his argument, as he explains that, with over 32\% of kids not attending college, there is a need for an alternative education, available during high school and beyond. Although Wyman brings up a counterargument — that college graduates generally earn more money than high school graduates — he justifies his argument by saying that these statistics cover all high school graduates, and that, when taking into account those that had vocational training, the picture looks worse for college students. As such, Wyman wraps together his argument, and makes it evident that more vocational schools and opportunities are needed.
        \end{justify}

      \item Amanda Ripley — from \textit{What America Can Learn from Smart Schools in Other Countries}

        \begin{justify}
          In her piece, entitled \textit{What America Can Learn from Smart Schools in Other Countries}, Amanda Ripley discusses the PISA exam and how it is relevant to the American education system. Ripley uses evidence from PISA analysts to support the claim that America is falling behind in its educational pursuits, with students performing either average or below average on tests for developed countries. Because of this, Ripley uses an array of Andreas Schleicher quotes to suggest changes that could be made.
        \end{justify}

      \item Leslie Nguyen-Okwu — \textit{How High Schools Are Demolishing the Classroom}

        \begin{justify}
          Leslie Nguyen-Okwu details extraordinary architecture in schools and how these are to be schools of the future in her piece entitled, \textit{How High Schools Are Demolishing the Classroom}. Nguyen-Okwu begins by dicussing the New Harmony High School in Louisiana. This school floats on the swamps of Louisiana, where students are able to directly interact with the environment and learn about ecology. Nguyen-Okwu then lists some other creative schools. Finally, Nguyen-Okwu concludes that the old, monolithic, factory system is outdated, and so is the architecture that came with it.
        \end{justify}

      \item Brentin Mock — from \textit{We Will Pay High School Students to Go to School. And We Will Like It.}

        \begin{justify}
          In his CityLab blog post, Brentin Mock outlines the plan for a new High School system that he believes would work for the students. His proposal is this: paying high schoolers. For showing up to class, he thinks high schoolers should receive \$50 to \$100 dollars, with a bonus on top for performing some kind of daily tasks. With this, Mock argues that schools would be able to compete with the “not-going-to school” market, which he argues has become huge. In this manner, Mock hopes to stimulate high school attendance, thereby making the world a better place.
        \end{justify}

      \item Amy Rolph — \textit{This High School Wants to Revolutionize Learning with Technology}

        \begin{justify}
          Published in \textit{USA Today} in 2017, Amy Rolph, an independent journalist, comments on the Washington Leadership Academy school in Washington, D.C. Rolph explains how the school employed the use of virtual reality and various other technologies to promote learning for the students. By providing this example, Rolph hopes to spread the message to other schools, and let other schools adapt to advancing society.
        \end{justify}

    \end{enumerate}

\end{document}

