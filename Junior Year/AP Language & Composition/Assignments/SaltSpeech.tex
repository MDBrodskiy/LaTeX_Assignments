\documentclass[12pt,letterpaper]{article}
\usepackage[utf8]{inputenc}
\usepackage[english]{babel}
\usepackage{ragged2e}
\usepackage[colorlinks = true,
            linkcolor = purple,
            urlcolor  = blue,
            citecolor = blue,
            anchorcolor = blue]{hyperref}
\hypersetup{
    colorlinks=true,
    linkcolor=blue,
    filecolor=blue,      
    urlcolor=blue,} 
\urlstyle{same}


\usepackage{ifpdf}
\usepackage{mla}

\begin{document}

\begin{mla}{Michael}{Brodskiy}{Mrs. Greer}{AP Language}{\today}{\underline{Gandhi Salt Speech Analysis Essay}} 

  \begin{justifying}

    \paragraph{I.} In his famous ``Salt Speech,'' Mohandas Gandhi writes to British Viceroy Lord Irwin in order to convince him, through means of non-violence, to withdraw all British control from India. Gandhi furthers his argument through the use of such rhetorical devices as repetition of phrases to make the structure flow better, appealing to ethos, or one's ethical standards in an attempt to gain sympathy, and appealing to pathos, one's emotions, to build a connection between Lord Irwin and Himself.

    \paragraph{II.} Gandhi purposefully and craftily uses rhetorical repetition in order to draw Lord Irwin into the letter. Within the first ten lines of the letter, Gandhi utilizes the word risk, which not only signifies his true thoughts, but also makes the flow of the paragraph such that Lord Irwin must recognize that even Gandhi, the leader of the peaceful resistance, is somewhat fearful of the dangers of his own civil disobedience tactic. As such, Gandhi strengthens the structure of the letter, and, therefore, his own ideas, thus building a connection with Lord Irwin. In addition to this, in lines 13$-$18, Gandhi uses serve, or any derivatives of the word, four times. He further attempts to build on the connection, already established by the repetition of risk, by stating his servitude of the British for many long years. Furthermore, Gandhi says that he will continue to serve, even in the hopeful event that India becomes independent. As such, Gandhi hopes that this will appeal to the British, as, just because they do not rule an area does not mean they are forbidden from having diplomatic relations with it. 

    \paragraph{III.} On top of the aforementioned efforts, Gandhi attempts to draw Lord Irwin into agreeing with his stance by highlighting the ethical problems brought on by the British rule. First of all, Gandhi states that the Salt Act passed by the British is detrimental to the poor of the country, which make up the most of the disobedient protesters. Therefore, Gandhi not only establishes the ethical problems of the Salt Act, but also provides a ``peaceful threat'' by stating that the Salt act angered a large mass of his followers, himself included. On top of this, Gandhi states that the Salt Act has ``disfigured the statute book.'' Gandhi's use of adjectives with negative connotations establish ethical boundaries, which the Salt Act is not within. Of course, the use of adjectives was intentional, as Gandhi hopes to convince Lord Irwin that the British do not have the ethical right to control Indian trade.

    \paragraph{IV.} Finally, Gandhi establishes an emotional relation between himself and Irwin in an attempt to clearly explicate his beliefs. Initially, Gandhi does this in lines 18$-$22, where he explains that he uses the tactic of civil disobedience on a day-to-day basis, not only against the British. Furthermore, Gandhi builds on his point by saying that his love of his own people and the British is equal. This is intended to create an agreement between Gandhi and Irwin that there is no mutual hatred between them. Also, Gandhi demonstrates his intentions openly, in an attempt to create a policy of openness between India and Britain. Gandhi concludes that this letter is in no way a threat (which, on some levels, it is). Doing such would make Lord Irwin believe that Gandhi and his followers truly are trying to simply recover their country, and that they harbored no hate for the British. This builds a pathological connection between Gandhi and Irwin.

    \paragraph{V.} Overall, it is evident that Gandhi has utilized many rhetorical techniques in order to make clear and further his beliefs to Lord Irwin. These techniques include repetition, ethical connections, and the use of emotion. Throughout the letter, Gandhi tries to make his intentions as clear as possible, in order to reduce tensions with the British, ultimately causing a withdrawal from India, a victory for Gandhi and his followers.

\end{mla}

\end{document}

