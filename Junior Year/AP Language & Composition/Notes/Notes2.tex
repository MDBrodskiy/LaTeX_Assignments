\documentclass[a4paper]{article} 
\usepackage{tcolorbox}
\tcbuselibrary{skins}

\title{
\vspace{-3em}
\begin{tcolorbox}[colframe=white,opacityback=0]
\begin{tcolorbox}
\Huge\sffamily AP Language of Composition Notes 2
\end{tcolorbox}
\end{tcolorbox}
\vspace{-3em}
}

\date{}

\usepackage{background}
\SetBgScale{1}
\SetBgAngle{0}
\SetBgColor{grey}
\SetBgContents{\rule[0em]{4pt}{\textheight}}
\SetBgHshift{-2.3cm}
\SetBgVshift{0cm}

\usepackage{lipsum}% just to generate filler text for the example
\usepackage[margin=2cm]{geometry}

\usepackage{tikz}
\usepackage{tikzpagenodes}

\parindent=0pt

\usepackage{xparse}
\DeclareDocumentCommand\topic{ m m g g g g g}
{
\begin{tcolorbox}[sidebyside,sidebyside align=top,opacityframe=0,opacityback=0,opacitybacktitle=0, opacitytext=1,lefthand width=.3\textwidth]
\begin{tcolorbox}[colback=red!05,colframe=red!25,sidebyside align=top,width=\textwidth,before skip=0pt]
#1.\end{tcolorbox}%
\tcblower
\begin{tcolorbox}[colback=blue!05,colframe=blue!10,width=\textwidth,before skip=0pt]
#2
\end{tcolorbox}
\IfNoValueF {#3}{
\begin{tcolorbox}[colback=blue!05,colframe=blue!10,width=\textwidth]
#3
\end{tcolorbox}
}
\IfNoValueF {#4}{
\begin{tcolorbox}[colback=blue!05,colframe=blue!10,width=\textwidth]
#4
\end{tcolorbox}
}
\IfNoValueF {#5}{
\begin{tcolorbox}[colback=blue!05,colframe=blue!10,width=\textwidth]
#5
\end{tcolorbox}
}
\IfNoValueF {#6}{
\begin{tcolorbox}[colback=blue!05,colframe=blue!10,width=\textwidth]
#6
\end{tcolorbox}
}
\IfNoValueF {#7}{
\begin{tcolorbox}[colback=blue!05,colframe=blue!10,width=\textwidth]
#7
\end{tcolorbox}
}
\end{tcolorbox}
}

\def\summary#1{
\begin{tikzpicture}[overlay,remember picture,inner sep=0pt, outer sep=0pt]
\node[anchor=south,yshift=-1ex] at (current page text area.south) {% 
\begin{minipage}{\textwidth}%%%%
\begin{tcolorbox}[colframe=white,opacityback=0]
\begin{tcolorbox}[enhanced,colframe=black,fonttitle=\large\bfseries\sffamily,sidebyside=true, nobeforeafter,before=\vfil,after=\vfil,colupper=black,sidebyside align=top, lefthand width=.95\textwidth,opacitybacktitle=1, opacitytext=1,
segmentation style={black!55,solid,opacity=0,line width=3pt},
title=Summary
]
#1
\end{tcolorbox}
\end{tcolorbox}
\end{minipage}
};
\end{tikzpicture}
}

\usepackage{graphicx}
\usepackage{physics}
\usepackage{amsmath}
\usepackage{tikz}
\usepackage{mathdots}
\usepackage{yhmath}
\usepackage{cancel}
\usepackage{color}
\usepackage{siunitx}
\usepackage{array}
\usepackage{multirow}
\usepackage{amssymb}
\usepackage{gensymb}
\usepackage{tabularx}
\usepackage{booktabs}
\usetikzlibrary{fadings}
\usetikzlibrary{patterns}
\usetikzlibrary{shadows.blur}
\usetikzlibrary{shapes}

\begin{document} 
\maketitle

\topic{\textit{The King's Speech}}{Answer the questions}%

\topic{Question 1 $-$ How do you think King George VI felt during this speech?}{First and foremost, King George VI most likely feared for himself and his people. In addition to this, he probably felt powerful, as all of his citizens were listening to him at once. Also, he probably feels anxious and concerned about going into the war.}%

\topic{Question 2 $-$ What is the emotion behind this speech?}{Sorrow, courage, patriotic, powerful.}%

\topic{SPACECAT}{\begin{tabular}{r|l} S & peaker \\ P & urpose \\ A & udience \\ C & ontext \\ E & xigence \\ C & hoices \\ A & ppeals  \\ T & one  \\ \end{tabular}}%

    \topic{When thinking of the speaker\dots}{What are their beliefs and values? Do we trust them? Why? What do we know and not know about them? Is there meaning behind who wrote or said it?}%

    \topic{When thinking of the purpose\dots}{What is the speaker hoping to accomplish? What reaction are they trying to elicit, and how do they want us to behave? Think of the purpose as an infinitive: to + verb.}%

    \topic{When thinking of the audience\dots}{What did the speaker assume about their audience? How does that impact what they say and how they say it?}%

    \topic{When thinking of the context\dots}{What was going on in the world when this text was produced?}%

    \topic{When thinking of the exigence\dots}{What was the spark or catalyst that moved the speaker to act?}%

    \topic{When thinking of the choices\dots}{This is a category of all the little moves authors make to enrich their writing. Why does the writer make each choice?}%

    \topic{When thinking of the appeals\dots}{Appeals to the ethics or credibility, emotion, or logic or reason.}%

    \topic{When thinking of the tone\dots}{What is the speaker's attitude at different places throughout the text? How can you tell this is their attitude? Where does the tone shift in the piece?}%

    \topic{Examples: Spread vs Smear, Weep vs Cry vs Sob}{Connotation $-$ The certain feeling behind a word or phrase}%
    
%\topic{Here's another question to begin the new page.}{\lipsum[3]}%
%{\lipsum[4]}%
%{\lipsum[5]}%

%\summary{And another summary that will float to the bottom of the next page.}

\end{document}
