%%%%%%%%%%%%%%%%%%%%%%%%%%%%%%%%%%%%%%%%%%%%%%%%%%%%%%%%%%%%%%%%%%%%%%%%%%%%%%%%%%%%%%%%%%%%%%%%%%%%%%%%%%%%%%%%%%%%%%%%%%%%%%%%%%%%%%%%%%%%%%%%%%%%%%%%%%%%%%%%%%%%%%%%%%%%%%%%%%%%%%%%%%%%
% Written By Michael Brodskiy
% Class: AP Biology
% Professor: J. Polivka
%%%%%%%%%%%%%%%%%%%%%%%%%%%%%%%%%%%%%%%%%%%%%%%%%%%%%%%%%%%%%%%%%%%%%%%%%%%%%%%%%%%%%%%%%%%%%%%%%%%%%%%%%%%%%%%%%%%%%%%%%%%%%%%%%%%%%%%%%%%%%%%%%%%%%%%%%%%%%%%%%%%%%%%%%%%%%%%%%%%%%%%%%%%%

\documentclass[12pt]{article} 
\usepackage{alphalph}
\usepackage[utf8]{inputenc}
\usepackage[russian,english]{babel}
\usepackage{titling}
\usepackage{amsmath}
\usepackage{graphicx}
\usepackage{enumitem}
\usepackage{amssymb}
\usepackage[super]{nth}
\usepackage{everysel}
\usepackage{ragged2e}
\usepackage{geometry}
\usepackage{fancyhdr}
\usepackage[super]{nth}
\usepackage{expl3}
\usepackage[version=4]{mhchem}
\usepackage{hpstatement}
\usepackage{rsphrase}
\usepackage{cancel}
\usepackage{siunitx}
\geometry{top=1.0in,bottom=1.0in,left=1.0in,right=1.0in}
\newcommand{\subtitle}[1]{%
  \posttitle{%
    \par\end{center}
    \begin{center}\large#1\end{center}
    \vskip0.5em}%

}
\usepackage{hyperref}
\hypersetup{
colorlinks=true,
linkcolor=blue,
filecolor=magenta,      
urlcolor=blue,
citecolor=blue,
}

\urlstyle{same}


\title{Chapter 16}
\date{February 5, 2020}
\author{Michael Brodskiy\\ \small Instructor: Mrs. Polivka}

% Mathematical Operations:

% Sum: $$\sum_{n=a}^{b} f(x) $$
% Integral: $$\int_{lower}^{upper} f(x) dx$$
% Limit: $$\lim_{x\to\infty} f(x)$$

\begin{document}

\maketitle

\begin{itemize}

  \item DNA vs Protein

    \begin{enumerate}

      \item DNA consists of pairs of Adenine, Thymine, Cytosine, and Guanine (C$\leftrightarrow$G, A$\leftrightarrow$T), and a phosphate backbone.

      \item Contains four structures: primary, secondary, tertiary, and quaternary.

      \item Primary source of genetic information

        \begin{enumerate}

          \item RNA can be used in some cases

        \end{enumerate}

      \item Eukaryotic Cells $-$ Multiple linear chromosomes, found in nucleus

      \item Prokaryotic Cells $-$ Circular chromosomes, found in cytosol

      \item Plasmids $-$ Separate extra piece of circular DNA

    \end{enumerate}

  \item Chargaff's Rules

    \begin{enumerate}

      \item Varies from species to species

      \item All four bases not in equal quantity

      \item In humans, base pairs are usually about:

        \begin{enumerate}

          \item A$\approx30.9\%$

          \item T$\approx29.4\%$

          \item G$\approx19.9\%$

          \item C$\approx19.8\%$

        \end{enumerate}

    \end{enumerate}

  \item DNA Structure:

    \begin{enumerate}

      \item Monomers: Nucleotides

      \item Nucleotide structure:

        \begin{enumerate}

          \item Phosphate

          \item Sugar (deoxyribose)

          \item Nitrogenous Base

            \begin{enumerate}

              \item Adenine, guanine, thymine, cytosine

            \end{enumerate}

        \end{enumerate}

    \end{enumerate}

  \item Nitrogenous Base and Pairing in DNA

    \begin{enumerate}

      \item Purines:

        \begin{enumerate}

          \item Adenine

          \item Guanine

        \end{enumerate}

      \item Pyrimidines

        \begin{enumerate}

          \item Thymine

          \item Cytosine

        \end{enumerate}

      \item Pairing

        \begin{enumerate}

          \item A:T

            \begin{enumerate}

              \item 2 Hydrogen Bonds

            \end{enumerate}

          \item C:G

            \begin{enumerate}

              \item 3 Hydrogen Bonds

            \end{enumerate}

        \end{enumerate}

    \end{enumerate}

  \item Structure is a double helix

  \item Anti-parallel strands

    \begin{enumerate}

      \item Nucleotides in DNA backbone are bonded from phosphate to sugar between 3' \& 5' carbons

        \begin{enumerate}

          \item DNA molecule has a 'direction'

          \item Complementary strand runs in opposite direction

          \item Direction is in the 5' direction

        \end{enumerate}

    \end{enumerate}

  \item Bonding in DNA

    \begin{enumerate}

      \item Phosphate backbone uses covalent bonds (strong)

      \item Base pairs use hydrogen bonds (weak)

    \end{enumerate}

  \item DNA Packing

    \begin{enumerate}

      \item DNA double helix wraps around histones (``beads on a string'')

      \item This wrapped ``wire'' wraps around again

      \item This creates a chromosome

    \end{enumerate}

  \item DNA Replication

    \begin{enumerate}

      \item Is semi-conservative

      \item Base pairing allows each strand to serve as a template for a new strand

      \item New strand is 1/2 parent template and 1/2 new DNA

      \item Step One: Replication

        \begin{enumerate}

          \item DNA is unwinded through use of the helicase enzyme

          \item Replication fork is made

          \item Helicase breaks the hydrogen bonds between two strand separating them

          \item Free nucleotides are present in the nucleus

          \item There is always a leading and a lagging strand

          \item DNA Polymerase creates complementary base pair

          \item DNA Polymerase moves towards 3' end

          \item Leading Strand:

            \begin{enumerate}

              \item RNA Primer is formed from RNA nucleotides, and bonds to start strand

              \item DNA Polymerase lays down the nucleotides in 5' to 3' direction of new DNA strand

              \item Can only add nucleotides to 3' end of a growing DNA strand

            \end{enumerate}

          \item Lagging Strand:

            \begin{enumerate}

              \item Runs in opposite direction of leading strand

              \item RNA Primer is joined to the parent strand by RNA Primase

              \item DNA Polymerase then works in the 5' to 3' direction, while laying down nucleotides forming Okazaki Fragments

              \item RNA Primer is removed from the fragments and replaced with DNA nucleotides

              \item DNA Ligase attaches the fragment backbones to each other

            \end{enumerate}

        \end{enumerate}

    \end{enumerate}

  \item DNA is edited and proofread

    \begin{enumerate}

      \item Many forms of Polymerase cut and remove abnormal bases, proofread and correct typos, and repairs mismatched bases

      \item Reduces error rate to 1 in 10 billion

    \end{enumerate}

\end{itemize}

\end{document}

