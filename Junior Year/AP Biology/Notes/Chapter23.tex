%%%%%%%%%%%%%%%%%%%%%%%%%%%%%%%%%%%%%%%%%%%%%%%%%%%%%%%%%%%%%%%%%%%%%%%%%%%%%%%%%%%%%%%%%%%%%%%%%%%%%%%%%%%%%%%%%%%%%%%%%%%%%%%%%%%%%%%%%%%%%%%%%%%%%%%%%%%%%%%%%%%%%%%%%%%%%%%%%%%%%%%%%%%%
% Written By Michael Brodskiy
% Class: AP Biology
% Professor: J. Polivka
%%%%%%%%%%%%%%%%%%%%%%%%%%%%%%%%%%%%%%%%%%%%%%%%%%%%%%%%%%%%%%%%%%%%%%%%%%%%%%%%%%%%%%%%%%%%%%%%%%%%%%%%%%%%%%%%%%%%%%%%%%%%%%%%%%%%%%%%%%%%%%%%%%%%%%%%%%%%%%%%%%%%%%%%%%%%%%%%%%%%%%%%%%%%

\documentclass[12pt]{article} 
\usepackage{alphalph}
\usepackage[utf8]{inputenc}
\usepackage[russian,english]{babel}
\usepackage{titling}
\usepackage{amsmath}
\usepackage{graphicx}
\usepackage{enumitem}
\usepackage{amssymb}
\usepackage[super]{nth}
\usepackage{everysel}
\usepackage{ragged2e}
\usepackage{geometry}
\usepackage{fancyhdr}
\usepackage[super]{nth}
\usepackage{expl3}
\usepackage[version=4]{mhchem}
\usepackage{hpstatement}
\usepackage{rsphrase}
\usepackage{cancel}
\usepackage{siunitx}
\geometry{top=1.0in,bottom=1.0in,left=1.0in,right=1.0in}
\newcommand{\subtitle}[1]{%
  \posttitle{%
    \par\end{center}
    \begin{center}\large#1\end{center}
    \vskip0.5em}%

}
\usepackage{hyperref}
\hypersetup{
colorlinks=true,
linkcolor=blue,
filecolor=magenta,      
urlcolor=blue,
citecolor=blue,
}

\urlstyle{same}


\title{Chapter 23}
\date{March 16, 2021}
\author{Michael Brodskiy\\ \small Instructor: Mrs. Polivka}

% Mathematical Operations:

% Sum: $$\sum_{n=a}^{b} f(x) $$
% Integral: $$\int_{lower}^{upper} f(x) dx$$
% Limit: $$\lim_{x\to\infty} f(x)$$

\begin{document}

\maketitle

\begin{itemize}

  \item One misconception is that organisms evolve during their lifetimes

  \item Natural selection acts on individuals, but only the population as a whole will evolve

  \item Microevolution — Change in the allele frequencies of a population over generations
    
  \item Five Agents of Evolutionary Change:

    \begin{enumerate}

      \item Mutation

      \item Gene Flow

      \item Non-random Mating

      \item Genetic Drift

      \item Selection

    \end{enumerate}

  \item A population is a localized group of interbreeding individuals

  \item Gene pool is a collection of alleles in a population

    \begin{itemize}

      \item Remember the difference between alleles and genes

    \end{itemize}
    
  \item Allele frequency is how common an allele appears in a population

    \begin{itemize}

      \item How many A vs. a in a whole population

    \end{itemize}
    
  \item Evolution — Change in allele frequencies in a population

  \item Hypothetical: What conditions would cause allele frequencies to not change? Examples:

    \begin{enumerate}

      \item A very large population (no genetic drift)

      \item No migration (no gene flow in or out)

      \item No mutation (no genetic change)

      \item Random mating (no sexual selection)

      \item No natural selection (everyone is equally fit)

    \end{enumerate}

  \item Hardy-Weinberg Equilibrium — A hypothetical, non-evolving population

    \begin{enumerate}

      \item Natural populations rarely in Hardy-Weinberg Equilibrium

      \item Measures if forces are acting on a population (evolutionary change)

    \end{enumerate}

  \item Counting Alleles

    \begin{enumerate}

      \item Assume 2 alleles: B and b

      \item Frequency of dominant allele (B) = $p$

      \item Frequency of recessive allele (b) = $q$

      \item Frequencies must add to 1 (100\%), so: $p+q=1$

    \end{enumerate}

  \item Counting Individuals

    \begin{enumerate}

      \item Frequency of homozygous dominant: $p\cdot p= p^2$

      \item Frequency of homozygous recessive: $q\cdot q= q^2$

      \item Frequency of heterozygotes: $p\cdot q + q\cdot p = 2pq$

      \item Frequencies must add to 1 (100\%), so: $p^2 + 2pq+q^2 =1$

    \end{enumerate}

  \item Major Causes of Evolution:

    \begin{enumerate}

      \item Genetic Drift:

        \begin{enumerate}

          \item Small populations have a greater chance of fluctuating in allele frequencies from one generation to another

          \item Founder Effect — A few individuals are isolated from larger populations. Certain alleles become over or under represented
            
          \item Bottleneck Effect — Sudden change in environment drastically reduces population size. By chance, certain organism, and, therefore alleles, survives

        \end{enumerate}

      \item Gene Flow

        \begin{enumerate}

          \item Movement of individuals between populations

          \item Alleles are lost or gained

        \end{enumerate}

      \item Natural Selection

        \begin{enumerate}

          \item Fitness — The contribution an individual makes to the gene pool of the next generation

          \item Directional Selection — A shift to an extreme of a phenotype (e.g. larger black bears survive extreme cold better than smaller ones)

          \item Disruptive Selection — A shift to both extremes  of a phenotype (e.g. small beaks for small seeds; large beaks for large seeds)

          \item Stabilizing Selection — A shift to the center, stable point of phenotypes (e.g. narrow range of human birth weight)

        \end{enumerate}

      \item Sexual Selection

        \begin{enumerate}

          \item Form of natural selection — certain individuals more likely to obtain mates

          \item Sexual dimorphism: difference between 2 sexes (can be in size, color, ornamentation, etc.)

        \end{enumerate}

      \item Genetic Mutation

        \begin{enumerate}

          \item Mutations in genes can lead to new alleles

        \end{enumerate}

    \end{enumerate}

\end{itemize}

\end{document}

