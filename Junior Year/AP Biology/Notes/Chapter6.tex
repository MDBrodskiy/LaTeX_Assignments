%%%%%%%%%%%%%%%%%%%%%%%%%%%%%%%%%%%%%%%%%%%%%%%%%%%%%%%%%%%%%%%%%%%%%%%%%%%%%%%%%%%%%%%%%%%%%%%%%%%%%%%%%%%%%%%%%%%%%%%%%%%%%%%%%%%%%%%%%%%%%%%%%%%%%%%%%%%%%%%%%%%%%%%%%%%%%%%%%%%%%%%%%%%%
% Written By Michael Brodskiy
% Class: AP Biology
% Professor: J. Polivka
%%%%%%%%%%%%%%%%%%%%%%%%%%%%%%%%%%%%%%%%%%%%%%%%%%%%%%%%%%%%%%%%%%%%%%%%%%%%%%%%%%%%%%%%%%%%%%%%%%%%%%%%%%%%%%%%%%%%%%%%%%%%%%%%%%%%%%%%%%%%%%%%%%%%%%%%%%%%%%%%%%%%%%%%%%%%%%%%%%%%%%%%%%%%

\documentclass[12pt]{article} 
\usepackage{alphalph}
\usepackage[utf8]{inputenc}
\usepackage[russian,english]{babel}
\usepackage{titling}
\usepackage{amsmath}
\usepackage{graphicx}
\usepackage{enumitem}
\usepackage{amssymb}
\usepackage[super]{nth}
\usepackage{everysel}
\usepackage{ragged2e}
\usepackage{geometry}
\usepackage{fancyhdr}
\usepackage{cancel}
\geometry{top=1.0in,bottom=1.0in,left=1.0in,right=1.0in}
\newcommand{\subtitle}[1]{%
  \posttitle{%
    \par\end{center}
    \begin{center}\large#1\end{center}
    \vskip0.5em}%

}
\usepackage{hyperref}
\hypersetup{
colorlinks=true,
linkcolor=blue,
filecolor=magenta,      
urlcolor=blue,
citecolor=blue,
}

\urlstyle{same}


\title{Chapter 6}
\date{September 9, 2020}
\author{Michael Brodskiy\\ \small Instructor: Mrs. Polivka}

% Mathematical Operations:

% Sum: $$\sum_{n=a}^{b} f(x) $$
% Integral: $$\int_{lower}^{upper} f(x) dx$$
% Limit: $$\lim_{x\to\infty} f(x)$$

\begin{document}

\maketitle

\begin{itemize}

  \item Two types of cells:

    \begin{enumerate}

      \item Prokaryote (Bacteria)  $-$ No organelles, have DNA, no nucleus, have ribosomes

      \item Eukaryote (Plant and Animal cells) $-$ Have organelles, nucleus, etc.

    \end{enumerate}

  \item The greater the size of the cell, the surface area to volume ratio decreases. Cell size may then be inadequate for cell size

  \item Why are organelles used?

    \begin{enumerate}

      \item Specialized structures, with specialized tasks

      \item Compartmentalize the cell (higher pH in some regions)

      \item Membranes used as places for chemical reactions (embedded enzymes and reaction centers)

    \end{enumerate}

  \item What jobs do cells have?

    \begin{enumerate}

     \item Make proteins

     \item Make energy

     \item Make more cells

    \end{enumerate}

  \item Organelles involved in building a protein:

    \begin{enumerate}

      \item Nucleus

      \item Ribosome

      \item Endoplasmic Reticulum (ER)

      \item Golgi Apparatus

      \item Vesicles

    \end{enumerate}

  \item Endoplasmic Reticulum (rough) has ribosomes on the outside

  \item Lysosome functions:

    \begin{enumerate}

      \item Digests macromolecules (use enzymes)

      \item Cleans up broken down organelles

      \item Fuse with food vacuoles to break down polymers

      \item May sometimes work incorrectly (lead to problems such as Tay-Sachs disease)

    \end{enumerate}

  \item White blood cells attack foreign agents (phagocytosis)

  \item Apoptosis $-$ Programmed cell death (broken down by lysosomes)

    \begin{enumerate}

      \item Ex. Loss of webbing between fingers during fetal development

    \end{enumerate}

  \item Smooth Endoplasmic Reticulum is where lipids are made

  \item Mitochondria and Chloroplasts:

    \begin{enumerate}

      \item Organelles not part of the endomembrane system (have separate DNA)

      \item Grow and reproduce in cell

    \end{enumerate}

  \item Endosymbiosis Theory:

    \begin{enumerate}

      \item  Mitochondria and chloroplasts were once free living bacteria, then were engulfed by a eukaryote

      \item Endosymbiont $-$ Cell that lives within another (host) cell

    \begin{enumerate}

      \item One supplies energy, while others supply raw materials and protection

    \end{enumerate}

    \end{enumerate}

\end{itemize}

\end{document}

