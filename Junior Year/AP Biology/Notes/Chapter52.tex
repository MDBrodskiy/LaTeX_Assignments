%%%%%%%%%%%%%%%%%%%%%%%%%%%%%%%%%%%%%%%%%%%%%%%%%%%%%%%%%%%%%%%%%%%%%%%%%%%%%%%%%%%%%%%%%%%%%%%%%%%%%%%%%%%%%%%%%%%%%%%%%%%%%%%%%%%%%%%%%%%%%%%%%%%%%%%%%%%%%%%%%%%%%%%%%%%%%%%%%%%%%%%%%%%%
% Written By Michael Brodskiy
% Class: AP Biology
% Professor: J. Polivka
%%%%%%%%%%%%%%%%%%%%%%%%%%%%%%%%%%%%%%%%%%%%%%%%%%%%%%%%%%%%%%%%%%%%%%%%%%%%%%%%%%%%%%%%%%%%%%%%%%%%%%%%%%%%%%%%%%%%%%%%%%%%%%%%%%%%%%%%%%%%%%%%%%%%%%%%%%%%%%%%%%%%%%%%%%%%%%%%%%%%%%%%%%%%

\documentclass[12pt]{article} 
\usepackage{alphalph}
\usepackage[utf8]{inputenc}
\usepackage[russian,english]{babel}
\usepackage{titling}
\usepackage{amsmath}
\usepackage{graphicx}
\usepackage{enumitem}
\usepackage{amssymb}
\usepackage[super]{nth}
\usepackage{everysel}
\usepackage{ragged2e}
\usepackage{geometry}
\usepackage{fancyhdr}
\usepackage[super]{nth}
\usepackage{expl3}
\usepackage[version=4]{mhchem}
\usepackage{hpstatement}
\usepackage{rsphrase}
\usepackage{cancel}
\usepackage{siunitx}
\geometry{top=1.0in,bottom=1.0in,left=1.0in,right=1.0in}
\newcommand{\subtitle}[1]{%
  \posttitle{%
    \par\end{center}
    \begin{center}\large#1\end{center}
    \vskip0.5em}%

}
\usepackage{hyperref}
\hypersetup{
colorlinks=true,
linkcolor=blue,
filecolor=magenta,      
urlcolor=blue,
citecolor=blue,
}

\urlstyle{same}


\title{Chapter 52}
\date{April 5, 2021}
\author{Michael Brodskiy\\ \small Instructor: Mrs. Polivka}

% Mathematical Operations:

% Sum: $$\sum_{n=a}^{b} f(x) $$
% Integral: $$\int_{lower}^{upper} f(x) dx$$
% Limit: $$\lim_{x\to\infty} f(x)$$

\begin{document}

\maketitle

\begin{itemize}

  \item Population Ecology — Study of populations in relation to environment, including environmental influences on density and distribution, age structure, and population size

  \item A population is a group of individuals of a single species living in the same general area

    \begin{enumerate}

      \item Rely on the same resources

      \item Interact

      \item Interbreed

    \end{enumerate}

  \item Population Density — How many organisms are present in an area

    \begin{enumerate}

      \item Mark and recapture method to determine density

    \end{enumerate}

  \item What causes population size change?

    \begin{enumerate}

      \item Adding or Removing individuals (birth, death, immigration, emigration)

    \end{enumerate}

  \item Spacing patterns within a population provide insight into the environmental associations \& social interactions of individuals in a population

    \begin{enumerate}

      \item Clumped — Most common. Animals live in groups or packs that make it easier to survive and protect territory. Common in animals such as wolves.

      \item Uniform — Roughly the same distance between each organism. Common in animals such as penguins. 

      \item Random — No pattern for population density. Common in plants.

    \end{enumerate}

  \item Life tables represent trends in survivorship rate of a population.

  \item Types of survivorship curves:

    \begin{enumerate}

      \item Type I (e.g. Humans) — Flat at start. Low death rates in middle and early life, but death rates rise as it gets farther right $\left(\text{looks like }\frac{1}{x}\text{ in the third quadrant}\right)$

      \item Type II (e.g. squirrels) — Straight line. Constant death rate.

      \item Type III (e.g. frogs) — High death rate for the young, but flattens out farther to the right $\left(\text{looks like }\frac{1}{x}\text{ in the first quadrant} \right)$

    \end{enumerate}

  \item Exponential Growth

    \begin{enumerate}

      \item Exponential growth is population increase under idealized conditions

      \item Under these conditions, the rate of reproduction is at its maximum, called the rate of intrinsic increase

    \end{enumerate}

  \item Population Growth (exponential model)

    \begin{enumerate}

      \item Change in population = births - deaths

      \item Exponential Model: $\frac{dN}{dt}=r_{max}N$, where $N$ is the number of individuals, $r$ is the rate of growth, and $t$ is time

      \item Characteristic of populations without limiting factors (like animals in a new environment or one rebounding from a catastrophe)

    \end{enumerate}

  \item Carrying Capacity (K) — The maximum population size that an environment can support. This is not fixed.

  \item Population Growth (logistic model)

    \begin{enumerate}

      \item Curve looks like an S

      \item Graph can “overshoot” the carrying capacity by small amounts

      \item Logistic Model: $\frac{dN}{dt}=r_{max}N\frac{(K-N)}{K}$, where $r$ is the growth rate $\left( \frac{\text{Births}-\text{Deaths}}{N} \right)$

    \end{enumerate}

  \item Populations may either be K-selection or r-selection based.

    \begin{enumerate}

      \item Like the Type I graph, K-selection is density dependent. Reproduction is iteroparous (repeated small litters).

      \item Like the Type III graph, r-selection is done to maximize reproductive success. Reproduction is semelparous (one massive birth).

    \end{enumerate}

  \item Trade-Offs
    
    \begin{enumerate}

      \item Number and size of offspring vs. Survival of offspring or parent

    \end{enumerate}

  \item What causes populations to stop growing?

    \begin{enumerate}

      \item Limiting Factors

        \begin{enumerate}

          \item Density independent (environmental disturbances)

          \item Density dependent (food supply, competition, predators)

        \end{enumerate}

      \item Population Cycles (fluctuations in the population, e.g. Dungeness crab)

        \begin{enumerate}

          \item Some are unpredictable (like the moose)

          \item Boom-bust cycles

        \end{enumerate}

      \item Population Dynamics (complex interaction of biotic \& abiotic influences that cause variation in population size.

    \end{enumerate}

  \item Age Structure shows relative number of individuals of each age

\end{itemize}

\end{document}

