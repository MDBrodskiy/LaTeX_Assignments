%%%%%%%%%%%%%%%%%%%%%%%%%%%%%%%%%%%%%%%%%%%%%%%%%%%%%%%%%%%%%%%%%%%%%%%%%%%%%%%%%%%%%%%%%%%%%%%%%%%%%%%%%%%%%%%%%%%%%%%%%%%%%%%%%%%%%%%%%%%%%%%%%%%%%%%%%%%%%%%%%%%%%%%%%%%%%%%%%%%%%%%%%%%%
% Written By Michael Brodskiy
% Class: AP Biology
% Professor: J. Polivka
%%%%%%%%%%%%%%%%%%%%%%%%%%%%%%%%%%%%%%%%%%%%%%%%%%%%%%%%%%%%%%%%%%%%%%%%%%%%%%%%%%%%%%%%%%%%%%%%%%%%%%%%%%%%%%%%%%%%%%%%%%%%%%%%%%%%%%%%%%%%%%%%%%%%%%%%%%%%%%%%%%%%%%%%%%%%%%%%%%%%%%%%%%%%

\documentclass[12pt]{article} 
\usepackage{alphalph}
\usepackage[utf8]{inputenc}
\usepackage[russian,english]{babel}
\usepackage{titling}
\usepackage{amsmath}
\usepackage{graphicx}
\usepackage{enumitem}
\usepackage{amssymb}
\usepackage[super]{nth}
\usepackage{everysel}
\usepackage{ragged2e}
\usepackage{geometry}
\usepackage{fancyhdr}
\usepackage{cancel}
\geometry{top=1.0in,bottom=1.0in,left=1.0in,right=1.0in}
\newcommand{\subtitle}[1]{%
  \posttitle{%
    \par\end{center}
    \begin{center}\large#1\end{center}
    \vskip0.5em}%

}
\usepackage{hyperref}
\hypersetup{
colorlinks=true,
linkcolor=blue,
filecolor=magenta,      
urlcolor=blue,
citecolor=blue,
}

\urlstyle{same}


\title{Chapter 5}
\date{August 24, 2020}
\author{Michael Brodskiy\\ \small Instructor: Mrs. Polivka}

% Mathematical Operations:

% Sum: $$\sum_{n=a}^{b} f(x) $$
% Integral: $$\int_{lower}^{upper} f(x) dx$$
% Limit: $$\lim_{x\to\infty} f(x)$$

\begin{document}

\maketitle

\begin{itemize}

  \item There are four main macromolecules:
    
    \begin{enumerate}

      \item Carbohydrate

      \item Lipids
        
      \item Proteins

      \item Nucleic Acids

    \end{enumerate}

  \item Proteins are the most complex out of the four molecules

  \item There are two forms of macromolecules:

    \begin{enumerate}

      \item Monomer $-$ The smallest unit of a macromolecule

      \item Polymer $-$ A larger molecule made up of smaller monomers

    \end{enumerate}

  \item Dehydration Synthesis $-$ Two hydrogen and one oxygen are removed from smaller molecules, allowing bonds to form more complex molecules

  \item Hydrolysis $-$ Reverse of Dehydration Synthesis, -lysis ending means breaking. Water is required for this

  \item Carbohydrates:

    \begin{enumerate}

      \item These are sugars and starches

      \item Monomers made up of C, H, and O (1:2:1 ratio)

      \item Short-term energy storage and structures

      \item Monomer $-$ Monosaccharides

        \begin{enumerate}

          \item Examples: Glucose, Galactose, \& Fructose

          \item Combination of two forms Disaccharides, created through dehydration synthesis (Glucose + Glucose = Maltose, Glucose + Fructose = Sucrose, Glucose + Galactose = Lactose)

        \end{enumerate}

      \item Polymer $-$ Polysaccharide

        \begin{enumerate}

          \item Glucose polymers have two main functions: Energy Storage for short term, amylose in plants and glycogen in animals, and Structural support, mostly in plants, as cellulose

          \item Starch vs Cellulose $-$ Starches are alpha linked, whereas cellulose is beta linked

        \end{enumerate}

      \item Herbivores:

        \begin{enumerate}

          \item Termites $-$ Symbiotic relationship with a protist, which lives in the termite's gut. This protist digests cellulose

          \item Ruminants $-$ Cows are an example. Cows have bacteria that break down the cellulose, while the cow keeps regurgitating it

          \item Caecophores $-$ Bunnies are an example. They process the cellulose by eating some of their cecal (pre-fecal) matter, as the cellulose is only partly digested before it comes out

        \end{enumerate}

      \item Chitin $-$ A modified polysaccharide that exits in fungi, arthropod exoskeletons, and dissolving stitches

    \end{enumerate}

  \item Lipids:

    \begin{enumerate}

      \item Exist as fats, oils, and waxes

      \item Made up with C, H, O

      \item Used for long-term storage and insulation

      \item No polymers

      \item Three groups: Triglycerides, Phospholipids, and Steroids
        
        \begin{enumerate}

          \item Triglycerides are made of one glycerol \& 3 fatty acids

          \item Triglycerides are connected by dehydration synthesis three times

          \item Saturated Fat $-$ No double bonds between carbons

          \item Unsaturated Fat $-$ At least one double bond (kinked), this influences the properties of the lipid. Unsaturated stay liquid at room temperature.

          \item Phospholipids are a modified version of the triglyceride, but with a phosphate rather than a fatty acid

          \item Phospholipids have a polar and non-polar region

          \item Big part of cell membranes, arranged in a bi-layer

          \item Steroids are made up of cholesterol and some types of hormones

          \item Steroids' structures are shaped as fused rings

          \item Have a variety of functions (functional groups)

        \end{enumerate}

    \end{enumerate}

\end{itemize}

\end{document}

