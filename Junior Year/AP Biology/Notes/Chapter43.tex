%%%%%%%%%%%%%%%%%%%%%%%%%%%%%%%%%%%%%%%%%%%%%%%%%%%%%%%%%%%%%%%%%%%%%%%%%%%%%%%%%%%%%%%%%%%%%%%%%%%%%%%%%%%%%%%%%%%%%%%%%%%%%%%%%%%%%%%%%%%%%%%%%%%%%%%%%%%%%%%%%%%%%%%%%%%%%%%%%%%%%%%%%%%%
% Written By Michael Brodskiy
% Class: AP Biology
% Professor: J. Polivka
%%%%%%%%%%%%%%%%%%%%%%%%%%%%%%%%%%%%%%%%%%%%%%%%%%%%%%%%%%%%%%%%%%%%%%%%%%%%%%%%%%%%%%%%%%%%%%%%%%%%%%%%%%%%%%%%%%%%%%%%%%%%%%%%%%%%%%%%%%%%%%%%%%%%%%%%%%%%%%%%%%%%%%%%%%%%%%%%%%%%%%%%%%%%

\documentclass[12pt]{article} 
\usepackage{alphalph}
\usepackage[utf8]{inputenc}
\usepackage[russian,english]{babel}
\usepackage{titling}
\usepackage{amsmath}
\usepackage{graphicx}
\usepackage{enumitem}
\usepackage{amssymb}
\usepackage[super]{nth}
\usepackage{everysel}
\usepackage{ragged2e}
\usepackage{geometry}
\usepackage{fancyhdr}
\usepackage{cancel}
\geometry{top=1.0in,bottom=1.0in,left=1.0in,right=1.0in}
\newcommand{\subtitle}[1]{%
  \posttitle{%
    \par\end{center}
    \begin{center}\large#1\end{center}
    \vskip0.5em}%

}
\usepackage{hyperref}
\hypersetup{
colorlinks=true,
linkcolor=blue,
filecolor=magenta,      
urlcolor=blue,
citecolor=blue,
}

\urlstyle{same}


\title{Chapter 43}
\date{October 26, 2020}
\author{Michael Brodskiy\\ \small Instructor: Mrs. Polivka}

% Mathematical Operations:

% Sum: $$\sum_{n=a}^{b} f(x) $$
% Integral: $$\int_{lower}^{upper} f(x) dx$$
% Limit: $$\lim_{x\to\infty} f(x)$$

\begin{document}

\maketitle

\begin{itemize}

  \item Lymphocyte $-$ A type of attacking white blood cell

  \item Phagocytic $-$ An engulfing white blood cell

  \item First and second line of defense in immune system is called the \textbf{innate immunity} and is non-specific (gets rid of any foreign agents)

    \begin{enumerate}

      \item First line of defense is external (skin, mucous membranes, secretions, etc.)

      \item Second line of defense is internal (phagocytic cells, antimicrobial proteins, inflammatory response, natural killer cells)

    \end{enumerate}

  \item Third line of defense is \textbf{acquired immunity}, takes longer, and is specific to a foreign agent

    \begin{enumerate}

      \item Third line of defense is internal (humoral responses [antibodies], cell-mediated response [cytotoxic lymphocytes])

    \end{enumerate}

  \item Invaders are recognized through \textbf{antigens} (cellular nametags)

  \item B cells attack and remember pathogens while circulating in blood and lymph

   \begin{enumerate}

     \item B cells produce specific antibodies against specific antigens

     \item To types of B cells $-$ Plasma cells (produce antibodies) and Memory cells (circulate body)

   \end{enumerate}

 \item Antibodies $-$ Proteins that bind to a specific antigen

   \begin{enumerate}

     \item Each antibody is unique and specific

     \item Tag foreign invaders (like handcuffs)

     \item Prevent pathogens from entering host cells

     \item Cause pathogens to clump together

     \item Macrophages are non-specific white blood cells that engulf invaders

   \end{enumerate}

 \item B cell immune response usually takes from 10$-$17 days

 \item If an attacker gets past and infects cells, Killer T-cells are released and attack cells that contain invaders

 \item How T-cells recognize infected cells:

   \begin{enumerate}

     \item Infected cells digest some pathogens

     \item MHC proteins carry antigens to cell surface

     \item T-cells ``scan'' antigens to locate infected cells

   \end{enumerate}

 \item T-cells attack, learn, and remember pathogens hiding in infected cells (recognize antigen fragments)

 \item Types of T-cells:

   \begin{enumerate}

     \item Helper T-cells $-$ Alert rest of immune system

     \item Cytotoxic T-cells $-$ Attack infected body cells

     \item Memory T-cells $-$ Circulate body

   \end{enumerate}

   \begin{center}
   \begin{tabular}[h]{|c|c|c|c|}
     \hline
     Type & Antigen & Antibody & Donation Status\\
     \hline
     A & A & B & $-$\\
     \hline
     B & B & A & $-$\\
     \hline
     AB & A \& B & N/A & Recipient\\
     \hline
     O & N/A & A \& B & Donor\\
     \hline
   \end{tabular}
 \end{center}

 \item Positive and Negative in blood refers to RH factor (positive means present, negative means not)

 \item Antigen Presenting Cells (APC) can be infected cells or macrophages. Helper T-cells scan these cells, releasing interleukin to alert rest of system.

 \item T-cells bind to target cells and secrete \textbf{perforin} protein, which causes lysing of cell and apoptosis

 \item Swelling of injuries:

   \begin{enumerate}

     \item Inflammation is a response

     \item Injured cells release histamines, while bacteria comes in
       
     \item Increases blood flow to punctured zone

     \item Brings more white blood cells to fight bacteria

     \item Brings more red blood cells \& clotting factors to repair area

   \end{enumerate}

\end{itemize}

\end{document}

