%%%%%%%%%%%%%%%%%%%%%%%%%%%%%%%%%%%%%%%%%%%%%%%%%%%%%%%%%%%%%%%%%%%%%%%%%%%%%%%%%%%%%%%%%%%%%%%%%%%%%%%%%%%%%%%%%%%%%%%%%%%%%%%%%%%%%%%%%%%%%%%%%%%%%%%%%%%%%%%%%%%%%%%%%%%%%%%%%%%%%%%%%%%%
% Written By Michael Brodskiy
% Class: AP Biology
% Professor: J. Polivka
%%%%%%%%%%%%%%%%%%%%%%%%%%%%%%%%%%%%%%%%%%%%%%%%%%%%%%%%%%%%%%%%%%%%%%%%%%%%%%%%%%%%%%%%%%%%%%%%%%%%%%%%%%%%%%%%%%%%%%%%%%%%%%%%%%%%%%%%%%%%%%%%%%%%%%%%%%%%%%%%%%%%%%%%%%%%%%%%%%%%%%%%%%%%

\documentclass[12pt]{article} 
\usepackage{alphalph}
\usepackage[utf8]{inputenc}
\usepackage[russian,english]{babel}
\usepackage{titling}
\usepackage{amsmath}
\usepackage{graphicx}
\usepackage{enumitem}
\usepackage{amssymb}
\usepackage[super]{nth}
\usepackage{everysel}
\usepackage{ragged2e}
\usepackage{geometry}
\usepackage{fancyhdr}
\usepackage[super]{nth}
\usepackage{expl3}
\usepackage[version=4]{mhchem}
\usepackage{hpstatement}
\usepackage{rsphrase}
\usepackage{cancel}
\usepackage{siunitx}
\geometry{top=1.0in,bottom=1.0in,left=1.0in,right=1.0in}
\newcommand{\subtitle}[1]{%
  \posttitle{%
    \par\end{center}
    \begin{center}\large#1\end{center}
    \vskip0.5em}%

}
\usepackage{hyperref}
\hypersetup{
colorlinks=true,
linkcolor=blue,
filecolor=magenta,      
urlcolor=blue,
citecolor=blue,
}

\urlstyle{same}


\title{Phylogeny}
\date{March 25, 2021}
\author{Michael Brodskiy\\ \small Instructor: Mrs. Polivka}

% Mathematical Operations:

% Sum: $$\sum_{n=a}^{b} f(x) $$
% Integral: $$\int_{lower}^{upper} f(x) dx$$
% Limit: $$\lim_{x\to\infty} f(x)$$

\begin{document}

\maketitle

\begin{itemize}

  \item Phylogenetic trees may be constructed through analysis of fossils or comparison of DNA proteins.

  \item In Analogous Structures — Convergent evolution occurs when similar environmental pressures produce similar (analogous) adaptations in different animals

  \item Phylogenetic Tree vs. Cladogram — In a Phylogenetic tree, the length of branches matters, while, in a cladogram, the length does not. A cladogram depicts patterns of shared characteristics among taxa.

  \item A clade is a group of species that includes an ancestral species and all of its descendants

  \item A valid clade is monophyletic

  \item A shared primitive character is a character that is shared beyond the taxon we are trying to define

  \item A shared derived character is an evolutionary novelty unique to a particular clade

  \item An outgroup is a group of organisms not belonging to the group whose evolutionary relationships are being compared

  \item Parsimony — The principle of parsimony implies that we should prefer the phylogeny that requires the fewest evolutionary changes

  \item How do new species form?

    \begin{enumerate}

      \item Isolation (allopatric or sympatric)

        \begin{enumerate}

          \item Allopatric — Geographic separation (other country)

          \item Sympatric — Still live in same area, but different parts (same country)

        \end{enumerate}

    \end{enumerate}

  \item Barriers to Reproduction:

    \begin{enumerate}

      \item Geographic Isolation (Being in different areas)

      \item Ecological Isolation (Being in different environments)

      \item Temporal Isolation (Mating in different seasons)

      \item Behavioral Isolation (Bird songs)

      \item Mechanical Isolation (Unable to reproduce)

      \item Gametic Isolation (Gametes reject)

    \end{enumerate}

  \item Post-Reproduction Barriers:

    \begin{enumerate}

      \item Reduced Hybrid Viability (Survive worse due to combination of parents)

      \item Reduced Hybrid Fertility (Offspring unable to reproduce)

      \item Hybrid Breakdown

    \end{enumerate}

  \item Rate of Speciation:

    \begin{enumerate}

      \item Gradualism — Gradual accumulation of small changes over a long time

      \item Punctuated Equilibrium — Rapid bursts of change mixed with long periods of little or no change

    \end{enumerate}<++>

\end{itemize}

\end{document}

