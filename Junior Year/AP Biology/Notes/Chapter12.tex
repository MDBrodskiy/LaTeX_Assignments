%%%%%%%%%%%%%%%%%%%%%%%%%%%%%%%%%%%%%%%%%%%%%%%%%%%%%%%%%%%%%%%%%%%%%%%%%%%%%%%%%%%%%%%%%%%%%%%%%%%%%%%%%%%%%%%%%%%%%%%%%%%%%%%%%%%%%%%%%%%%%%%%%%%%%%%%%%%%%%%%%%%%%%%%%%%%%%%%%%%%%%%%%%%%
% Written By Michael Brodskiy
% Class: AP Biology
% Professor: J. Polivka
%%%%%%%%%%%%%%%%%%%%%%%%%%%%%%%%%%%%%%%%%%%%%%%%%%%%%%%%%%%%%%%%%%%%%%%%%%%%%%%%%%%%%%%%%%%%%%%%%%%%%%%%%%%%%%%%%%%%%%%%%%%%%%%%%%%%%%%%%%%%%%%%%%%%%%%%%%%%%%%%%%%%%%%%%%%%%%%%%%%%%%%%%%%%

\documentclass[12pt]{article} 
\usepackage{alphalph}
\usepackage[utf8]{inputenc}
\usepackage[russian,english]{babel}
\usepackage{titling}
\usepackage{amsmath}
\usepackage{graphicx}
\usepackage{enumitem}
\usepackage{amssymb}
\usepackage[super]{nth}
\usepackage{everysel}
\usepackage{ragged2e}
\usepackage{geometry}
\usepackage{fancyhdr}
\usepackage[super]{nth}
\usepackage{expl3}
\usepackage[version=4]{mhchem}
\usepackage{hpstatement}
\usepackage{rsphrase}
\usepackage{cancel}
\usepackage{siunitx}
\geometry{top=1.0in,bottom=1.0in,left=1.0in,right=1.0in}
\newcommand{\subtitle}[1]{%
  \posttitle{%
    \par\end{center}
    \begin{center}\large#1\end{center}
    \vskip0.5em}%

}
\usepackage{hyperref}
\hypersetup{
colorlinks=true,
linkcolor=blue,
filecolor=magenta,      
urlcolor=blue,
citecolor=blue,
}

\urlstyle{same}


\title{Chapter 12}
\date{January 6, 2020}
\author{Michael Brodskiy\\ \small Instructor: Mrs. Polivka}

% Mathematical Operations:

% Sum: $$\sum_{n=a}^{b} f(x) $$
% Integral: $$\int_{lower}^{upper} f(x) dx$$
% Limit: $$\lim_{x\to\infty} f(x)$$

\begin{document}

\maketitle

\begin{itemize}
    
  \item Reasons for Cells to Divide:

    \begin{enumerate}

      \item Growth and Development

      \item Asexual Reproduction

      \item Tissue Renewal

    \end{enumerate}

  \item The Cell Cycle:

    \begin{enumerate}

      \item Interphase ($G_1$ phase, S phase, and $G_2$ phase)

      \item Mitotic Phase (Mitosis, Cytokinesis)

    \end{enumerate}

  \item $G_1$ Phase $-$ Growth phase, as the cell prepares to divide

  \item S Phase $-$ Synthesis phase, DNA is duplicated so that cells remain the same

  \item $G_2$ Phase $-$ Prepares the cell for cell division

  \item Chromosome Organization:

    \begin{enumerate}

      \item Each cell has about 2 meters of DNA in the nucleus, made up of thin threads called chromatin

      \item Before division, chromatin is condensed to chromosomes

      \item DNA replicates before cell division to produce paired chromatids

    \end{enumerate}

  \item In $G_2$ of interphase, chromatin is duplicated

  \item In metaphase, spindle fibers attach to chromosomes

  \item In anaphase, spindles pull sister chromatids, splitting the chromosomes in two

  \item Bacterial Binary Fission $-$ Bacteria do not have a nucleus, and contain a single chromosome, which is replicated. The bacteria then splits in two.
    
  \item When cells are not actively dividing, they are in $G_0$ phase

  \item Cells enter the cell division cycle if they activate the genes for cyclin proteins

    \begin{enumerate}

      \item Activated by growth factors and other signals

      \item Inhibited by cell density and/or lack of ECM anchorage

    \end{enumerate}

  \item 3 Major Cell Checkpoints:

    \begin{enumerate}

      \item Between $G_1$ and S Phase $-$ Can DNA synthesis begin?

      \item Between $G_2$ and M Phase $-$ Has DNA synthesis been completed correctly, and can the cell commit to mitosis?

      \item Spindle Checkpoint $-$ Are all chromosomes attached to spindles, and are sister chromatids split correctly?

    \end{enumerate}

  \item Protein Signals promote cell growth and division:

    \begin{enumerate}

      \item Internal signals are called ``promoting factors''

      \item External signals are called ``growth factors''

    \end{enumerate}

  \item Primary Mechanism of Control done by Phosphorylation by kinase enzyme:

    \begin{enumerate}

      \item Kinases must be attached to a cyclin to be activated

      \item Cyclin concentration fluctuates in a cell and is high in M phase

    \end{enumerate}

  \item Growth Factors $-$ Protein signals released by body cells that stimulate other cells to divide

    \begin{enumerate}

      \item Density-Dependent Inhibition $-$ When cells crowd, they stop dividing. Each cell binds to a bit of growth factor, until none is left

      \item Anchorage Dependence $-$ Must attach to substrate, too many cells, nowhere to attach

      \item Cancer cells have neither of the above

    \end{enumerate}

  \item Cancer is essentially uncontrolled cell growth

  \item What control is lost? Lose checkpoint stops. Gene p53 plays a key role in $G_1$/S restriction point. p53 usually halts cell division if it detects a damaged DNA. All cancers do not have p53.

\end{itemize}

\end{document}

