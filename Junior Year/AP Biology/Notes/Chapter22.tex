%%%%%%%%%%%%%%%%%%%%%%%%%%%%%%%%%%%%%%%%%%%%%%%%%%%%%%%%%%%%%%%%%%%%%%%%%%%%%%%%%%%%%%%%%%%%%%%%%%%%%%%%%%%%%%%%%%%%%%%%%%%%%%%%%%%%%%%%%%%%%%%%%%%%%%%%%%%%%%%%%%%%%%%%%%%%%%%%%%%%%%%%%%%%
% Written By Michael Brodskiy
% Class: AP Biology
% Professor: J. Polivka
%%%%%%%%%%%%%%%%%%%%%%%%%%%%%%%%%%%%%%%%%%%%%%%%%%%%%%%%%%%%%%%%%%%%%%%%%%%%%%%%%%%%%%%%%%%%%%%%%%%%%%%%%%%%%%%%%%%%%%%%%%%%%%%%%%%%%%%%%%%%%%%%%%%%%%%%%%%%%%%%%%%%%%%%%%%%%%%%%%%%%%%%%%%%

\documentclass[12pt]{article} 
\usepackage{alphalph}
\usepackage[utf8]{inputenc}
\usepackage[russian,english]{babel}
\usepackage{titling}
\usepackage{amsmath}
\usepackage{graphicx}
\usepackage{enumitem}
\usepackage{amssymb}
\usepackage[super]{nth}
\usepackage{everysel}
\usepackage{ragged2e}
\usepackage{geometry}
\usepackage{fancyhdr}
\usepackage[super]{nth}
\usepackage{expl3}
\usepackage[version=4]{mhchem}
\usepackage{hpstatement}
\usepackage{rsphrase}
\usepackage{cancel}
\usepackage{siunitx}
\geometry{top=1.0in,bottom=1.0in,left=1.0in,right=1.0in}
\newcommand{\subtitle}[1]{%
  \posttitle{%
    \par\end{center}
    \begin{center}\large#1\end{center}
    \vskip0.5em}%

}
\usepackage{hyperref}
\hypersetup{
colorlinks=true,
linkcolor=blue,
filecolor=magenta,      
urlcolor=blue,
citecolor=blue,
}

\urlstyle{same}


\title{Chapter 22}
\date{March 5, 2021}
\author{Michael Brodskiy\\ \small Instructor: Mrs. Polivka}

% Mathematical Operations:

% Sum: $$\sum_{n=a}^{b} f(x) $$
% Integral: $$\int_{lower}^{upper} f(x) dx$$
% Limit: $$\lim_{x\to\infty} f(x)$$

\begin{document}

\maketitle

\begin{itemize}

  \item Influences on Darwin's Theories

    \begin{enumerate}

      \item Thomas Malthus $-$ Mentions competition and a struggle for survival, as population surpasses food supply

      \item Lyell $-$ Land masses change over immeasurable time

      \item Lamarck $-$ Discusses acquired characteristics (giraffe example)

    \end{enumerate}

  \item Darwin proposed the theory of natural selection and inheritance

  \item Darwin examined the various types of finches on the Galapagos islands to support his theory

  \item Differences in Finch beaks allowed for:

    \begin{enumerate}

      \item successful competition

      \item successful feeding

      \item successful reproduction

    \end{enumerate}

  \item Successful traits (such as a certain sized beak) were passed down to next generation

  \item Adaptive Radiation $-$ Rapid speciation, with new species filling niches due to the inheritance of successful traits

  \item Darwin's idea of Natural Selection was based on:

    \begin{enumerate}

      \item Heritable variation exists in populations

      \item Populations over-produce offspring (more individuals produced than the environment can support)

      \item Competition for food, mates, nesting sites, and escaping from predators occurs

      \item Differential survival $-$ successful traits become adaptations

      \item Differences reproduction $-$ adaptations become more common in population

    \end{enumerate}

  \item Genetic Variation comes from mutations (random changes to the DNA), which causes errors in mitosis and meiosis. This can also be caused by UV damage.

  \item Sexual reproduction introduces mixing of alleles through genetic recombination, which form new combinations of alleles in every offspring. New combinations create new phenotypes.

  \item Evidence Darwin used:

    \begin{enumerate}

      \item Fossil Records

        \begin{enumerate}

          \item Archaeopteryx, which lived roughly 150 million years ago, links reptiles to birds

          \item Tiktaalik linked sea to land animals, as it went from swimming (no legs) to walking (4 legs)

        \end{enumerate}

      \item Artificial Selection

        \begin{enumerate}

          \item Choosing which traits are to be passed down (selective breeding)

        \end{enumerate}

      \item Anatomical Evidence

        \begin{enumerate}

          \item Looking at the limbs of humans, cats, bats, and whales, it is evident that the structures are similar.

          \item Known as homologous structures, these kinds of structures can hint at common ancestors.

          \item Analogous structures are structures that have similar uses, but not a common ancestor (solve a similar problem with similar solutions). Sometimes referred to as convergent evolution.

          \item Vestigial organs are used too. Vestigial organs are structures that are left over, but are not used, from evolution (ex. human tailbone, appendix, wisdom teeth).

        \end{enumerate}

      \item Comparative Structures

        \begin{enumerate}

          \item DNA sequences are compared to determine differences between species. For example, humans and macaques have only 8 differences.

          \item Comparative structures allow people to construct family trees.

        \end{enumerate}

    \end{enumerate}

\end{itemize}

\end{document}

