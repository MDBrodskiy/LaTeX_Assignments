%%%%%%%%%%%%%%%%%%%%%%%%%%%%%%%%%%%%%%%%%%%%%%%%%%%%%%%%%%%%%%%%%%%%%%%%%%%%%%%%%%%%%%%%%%%%%%%%%%%%%%%%%%%%%%%%%%%%%%%%%%%%%%%%%%%%%%%%%%%%%%%%%%%%%%%%%%%%%%%%%%%%%%%%%%%%%%%%%%%%%%%%%%%%
% Written By Michael Brodskiy
% Class: AP Biology
% Professor: J. Polivka
%%%%%%%%%%%%%%%%%%%%%%%%%%%%%%%%%%%%%%%%%%%%%%%%%%%%%%%%%%%%%%%%%%%%%%%%%%%%%%%%%%%%%%%%%%%%%%%%%%%%%%%%%%%%%%%%%%%%%%%%%%%%%%%%%%%%%%%%%%%%%%%%%%%%%%%%%%%%%%%%%%%%%%%%%%%%%%%%%%%%%%%%%%%%

\documentclass[12pt]{article} 
\usepackage{alphalph}
\usepackage[utf8]{inputenc}
\usepackage[russian,english]{babel}
\usepackage{titling}
\usepackage{amsmath}
\usepackage{graphicx}
\usepackage{enumitem}
\usepackage{amssymb}
\usepackage[super]{nth}
\usepackage{everysel}
\usepackage{ragged2e}
\usepackage{geometry}
\usepackage{fancyhdr}
\usepackage[super]{nth}
\usepackage{expl3}
\usepackage[version=4]{mhchem}
\usepackage{hpstatement}
\usepackage{rsphrase}
\usepackage{cancel}
\usepackage{siunitx}
\geometry{top=1.0in,bottom=1.0in,left=1.0in,right=1.0in}
\newcommand{\subtitle}[1]{%
  \posttitle{%
    \par\end{center}
    \begin{center}\large#1\end{center}
    \vskip0.5em}%

}
\DeclareSIUnit\Celsius{C}
\usepackage{hyperref}
\hypersetup{
colorlinks=true,
linkcolor=blue,
filecolor=magenta,      
urlcolor=blue,
citecolor=blue,
}

\urlstyle{same}


\title{Chapter 20}
\date{May 14, 2021}
\author{Michael Brodskiy\\ \small Instructor: Mrs. Polivka}

% Mathematical Operations:

% Sum: $$\sum_{n=a}^{b} f(x) $$
% Integral: $$\int_{lower}^{upper} f(x) dx$$
% Limit: $$\lim_{x\to\infty} f(x)$$

\begin{document}

\maketitle

\begin{itemize}

  \item Genes can be inserted into animals and make them glow

  \item The Human Genome has 3.2 billion bases

  \item Genetic Engineering

    \begin{enumerate}

      \item Manipulation of DNA

    \end{enumerate}

  \item Bacteria

    \begin{enumerate}

      \item One-celled prokaryotes

      \item Reproduce by mitosis

      \item Rapid growth (generation every 20 minutes)

      \item Dominant form of life on Earth

    \end{enumerate}

  \item Bacterial Genome

    \begin{enumerate}

      \item Single, circular chromosome

      \item Naked DNA (no histone proteins)

      \item Contain $\frac{1}{1000}$ of the DNA of a eukaryote

    \end{enumerate}

  \item Plasmids

    \begin{enumerate}

      \item Small, supplemental circles of DNA

      \item Carry extra genes (2-30 genes, usually for antibiotic resistance)

      \item Can be exchanged between bacteria (rapid evolution)

      \item Can be taken from the environment

    \end{enumerate}

  \item Plasmids allow for an easy way to insert genes into a bacteria

    \begin{enumerate}

      \item Insert new gene into plasmid

      \item Insert plasmid into bacteria

      \item Bacteria now expresses new gene (bacteria make new protein)

    \end{enumerate}

  \item Example: Insulin can be farmed from bacteria by inserting the insulin gene into a plasmid, and waiting for bacteria to reproduce

  \item DNA is cut using restriction enzymes

    \begin{enumerate}

      \item Evolved in bacteria to cut up foreign DNA

      \item “Restrict” the action of the attacking organism

      \item Protects against viruses and other bacteria

      \item Bacteria protect their own DNA by not using the base sequences recognized by the enzymes in their own DNA

    \end{enumerate}

  \item Restriction Enzymes

    \begin{enumerate}

      \item Cut DNA at specific sequences (palindromes)

      \item Produces protruding ends (sticky ends — will bind to any complementary DNA)

      \item Many different enzymes 

    \end{enumerate}

  \item The same insulin in bacteria can be used in humans because the “code” is universal

  \item Transformation

    \begin{enumerate}

      \item Alteration of a bacterial cell's genotype and phenotype by the uptake of foreign DNA from the surrounding environment

      \item Insert recombinant plasmid into bacteria

      \item Grow recombinant bacteria in agar cultures — bacteria make lots of copies of plasmids (plasmid “cloning”)

      \item Production of many copies of inserted gene

      \item Production of “new” protein

    \end{enumerate}

  \item Bacteria Lab:

    \begin{enumerate}

      \item Normally, \textit{E.coli} does not grow when ampicillin is around

      \item Insert a new gene into \textit{E.coli} to make it resistant to the antibiotic ampicillin

    \end{enumerate}

  \item Genetically modified organisms (GMOs) are also a product of biotechnology

    \begin{enumerate}

      \item Enabling plants to produce new proteins

      \item Protect crops from insects — BT corn

        \begin{enumerate}

          \item Corn produces a bacterial toxin that kills corn borer (caterpillar pest of corn)

        \end{enumerate}

      \item Extend growing season — fishberries

        \begin{enumerate}

          \item Strawberries with an anti-freezing gene from flounder

        \end{enumerate}

    \end{enumerate}

  \item How do we compare DNA fragments?

    \begin{enumerate}

      \item Separate fragments by size

      \item Run it through a gelatin (agarose gel)

      \item Gel electrophoresis

    \end{enumerate}

  \item Gel Electrophoresis

    \begin{enumerate}

      \item A method of separating DNA in a gelatin-like material using an electric field

      \item DNA is negatively charged

      \item DNA moves to the positive side

    \end{enumerate}

  \item DNA moves in an electrical field — size of fragments affects how far it travels (small pieces travel farther, large pieces travel slower and lag behind)

  \item Gel Electrophoresis Uses:

    \begin{enumerate}

      \item Useful for comparing DNA samples from different organisms to measure evolutionary relationships

      \item Useful in medical diagnosis (e.g. Huntington's disease)
        
      \item Useful in forensics, such as comparing the DNA sample from a crime scene with that of suspects and victim

      \item Useful for comparing blood samples to determine who blood belongs to (DNA fingerprinting) by comparing DNA banding

      \item Useful for determining paternity — the more bands shared with a person, the more likely they are a parent

    \end{enumerate}

  \item Differences at the DNA level

    \begin{enumerate}

      \item Sections of “junk” DNA

        \begin{enumerate}

          \item Doesn't code for proteins

          \item Made up of repeated patterns

            \begin{enumerate}

              \item CAT, GCC, and others

              \item Each person may have a different number of repeats

            \end{enumerate}

          \item Many sites on our 23 chromosomes with different repeat patterns

        \end{enumerate}

    \end{enumerate}

  \item PCR — Polymerase Chain Reaction

    \begin{enumerate}

      \item Method for making many, many copies of a specific segment of DNA

      \item Only need 1 cell of DNA to start

    \end{enumerate}

  \item PCR Process

    \begin{enumerate}

      \item DNA replication in a test tube — template strand, DNA polymerase enzyme, nucleotides (ATP \& GTP), and primers are necessary

      \item Primers are critical — a bit of the sequence needs to be known to make proper primers

      \item Primers bracket target sequence

        \begin{enumerate}

          \item Start with a long piece of DNA and copy a specified shorter segment

          \item Primers define section of DNA to be cloned

        \end{enumerate}

      \item Process Steps:

        \begin{enumerate}

          \item In tube: DNA, DNA Polymerase Enzyme, Primer, and Nucleotides

          \item Denature DNA: heat (to around $90[\si{\degree\Celsius}]$) DNA to separate strands

          \item Anneal DNA: cool to hybridize with primers and build DNA (extension)

        \end{enumerate}

    \end{enumerate}

\end{itemize}

\end{document}

