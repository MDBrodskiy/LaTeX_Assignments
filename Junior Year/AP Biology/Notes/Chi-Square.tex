%%%%%%%%%%%%%%%%%%%%%%%%%%%%%%%%%%%%%%%%%%%%%%%%%%%%%%%%%%%%%%%%%%%%%%%%%%%%%%%%%%%%%%%%%%%%%%%%%%%%%%%%%%%%%%%%%%%%%%%%%%%%%%%%%%%%%%%%%%%%%%%%%%%%%%%%%%%%%%%%%%%%%%%%%%%%%%%%%%%%%%%%%%%%
% Written By Michael Brodskiy
% Class: AP Biology
% Professor: J. Polivka
%%%%%%%%%%%%%%%%%%%%%%%%%%%%%%%%%%%%%%%%%%%%%%%%%%%%%%%%%%%%%%%%%%%%%%%%%%%%%%%%%%%%%%%%%%%%%%%%%%%%%%%%%%%%%%%%%%%%%%%%%%%%%%%%%%%%%%%%%%%%%%%%%%%%%%%%%%%%%%%%%%%%%%%%%%%%%%%%%%%%%%%%%%%%

\documentclass[12pt]{article} 
\usepackage{alphalph}
\usepackage[utf8]{inputenc}
\usepackage[russian,english]{babel}
\usepackage{titling}
\usepackage{amsmath}
\usepackage{graphicx}
\usepackage{enumitem}
\usepackage{amssymb}
\usepackage[super]{nth}
\usepackage{everysel}
\usepackage{ragged2e}
\usepackage{geometry}
\usepackage{fancyhdr}
\usepackage[super]{nth}
\usepackage{expl3}
\usepackage[version=4]{mhchem}
\usepackage{hpstatement}
\usepackage{rsphrase}
\usepackage{cancel}
\usepackage{siunitx}
\geometry{top=1.0in,bottom=1.0in,left=1.0in,right=1.0in}
\newcommand{\subtitle}[1]{%
  \posttitle{%
    \par\end{center}
    \begin{center}\large#1\end{center}
    \vskip0.5em}%

}
\usepackage{hyperref}
\hypersetup{
colorlinks=true,
linkcolor=blue,
filecolor=magenta,      
urlcolor=blue,
citecolor=blue,
}

\urlstyle{same}


\title{The Chi-Square Test}
\date{January 25, 2020}
\author{Michael Brodskiy\\ \small Instructor: Mrs. Polivka}

% Mathematical Operations:

% Sum: $$\sum_{n=a}^{b} f(x) $$
% Integral: $$\int_{lower}^{upper} f(x) dx$$
% Limit: $$\lim_{x\to\infty} f(x)$$

\begin{document}

\maketitle

\begin{itemize}

  \item The chi-square tests determines whether the observed value is significantly different from the expected value

    \begin{enumerate}

      \item The test tells us whether this is due to chance or some kind of outside influence

    \end{enumerate}

  \item The null hypothesis is a statement that assumes there is no difference between the observed and expected values

    \begin{enumerate}

      \item For Example: ``There is no significant difference between the experimental results and those which would be expected''

    \end{enumerate}

  \item The formula is as given:

    \begin{equation}
      \mathcal{X}^2=\sum\left( \frac{\left( O-E \right)^2}{E} \right)\\
      \label{1}
    \end{equation}

  \item The number of degrees of freedom is always one less than the number of terms in the formula above (one less than the possibilities)
    
  \item A table with degrees of freedom will always be given. In biology, we will use the 5\% column (probability equals 95\% true)

  \item If $\mathcal{X}^2$ is greater than the critical value, reject the null hypothesis

  \item This means that the difference you are seeing in your results is \textbf{NOT} due to chance

  \item If $\mathcal{X}^2$ is lower than the critical value, accept the null hypothesis

  \item This means that the difference you are seeing in your results \textbf{IS} due to chance

\end{itemize}

\end{document}

