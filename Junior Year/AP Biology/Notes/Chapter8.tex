%%%%%%%%%%%%%%%%%%%%%%%%%%%%%%%%%%%%%%%%%%%%%%%%%%%%%%%%%%%%%%%%%%%%%%%%%%%%%%%%%%%%%%%%%%%%%%%%%%%%%%%%%%%%%%%%%%%%%%%%%%%%%%%%%%%%%%%%%%%%%%%%%%%%%%%%%%%%%%%%%%%%%%%%%%%%%%%%%%%%%%%%%%%%
% Written By Michael Brodskiy
% Class: AP Biology
% Professor: J. Polivka
%%%%%%%%%%%%%%%%%%%%%%%%%%%%%%%%%%%%%%%%%%%%%%%%%%%%%%%%%%%%%%%%%%%%%%%%%%%%%%%%%%%%%%%%%%%%%%%%%%%%%%%%%%%%%%%%%%%%%%%%%%%%%%%%%%%%%%%%%%%%%%%%%%%%%%%%%%%%%%%%%%%%%%%%%%%%%%%%%%%%%%%%%%%%

\documentclass[12pt]{article} 
\usepackage{alphalph}
\usepackage[utf8]{inputenc}
\usepackage[russian,english]{babel}
\usepackage{titling}
\usepackage{amsmath}
\usepackage{graphicx}
\usepackage{enumitem}
\usepackage{amssymb}
\usepackage[super]{nth}
\usepackage{everysel}
\usepackage{ragged2e}
\usepackage{geometry}
\usepackage{fancyhdr}
\usepackage{cancel}
\geometry{top=1.0in,bottom=1.0in,left=1.0in,right=1.0in}
\newcommand{\subtitle}[1]{%
  \posttitle{%
    \par\end{center}
    \begin{center}\large#1\end{center}
    \vskip0.5em}%

}
\usepackage{hyperref}
\hypersetup{
colorlinks=true,
linkcolor=blue,
filecolor=magenta,      
urlcolor=blue,
citecolor=blue,
}

\urlstyle{same}


\title{Chapter 8}
\date{October 14, 2020}
\author{Michael Brodskiy\\ \small Instructor: Mrs. Polivka}

% Mathematical Operations:

% Sum: $$\sum_{n=a}^{b} f(x) $$
% Integral: $$\int_{lower}^{upper} f(x) dx$$
% Limit: $$\lim_{x\to\infty} f(x)$$

\begin{document}

\maketitle

\begin{itemize}

  \item Cells communicate through Signal Transduction, started by the attachment of a ligand to a receptor,  which has three steps:

    \begin{enumerate}

      \item Reception

      \item Transduction

      \item Response

    \end{enumerate}

  \item There are two types of signals:

    \begin{enumerate}

      \item Local signaling

        \begin{enumerate}

          \item Paracrine Signaling

          \item Synaptic Signaling (specific to nervous system)

          \item Cell Junctions (Gap junctions in animals, and Plasmodesmata in plant cells)

          \item Cell-cell Recognition

        \end{enumerate}

      \item Long-distance signaling

      \item Signaling Pathway

        \begin{enumerate}

          \item G-Protein Linked Reception

            \begin{enumerate}

              \item First step in protein relay, activated by GTP (Guanine Triphosphate)

            \end{enumerate}

          \item Kinase: a protein that phosphorylates (adds a phosphate) to another molecule

          \item Ligand-gated Ion Channels

            \begin{enumerate}

              \item Attachment of ligand will open up a gate, through which some kind of particles move

            \end{enumerate}

        \end{enumerate}

    \end{enumerate}

  \item Second Messengers $-$ Internal signaling molecules released due to external (``first'') signals. Trigger sub-response pathways.

    \begin{enumerate}

      \item Cyclic AMP $-$ A typical second messenger that affects metabolism

      \item Calcium ions are another common second messenger

    \end{enumerate}

  \item Epinephrine (Adrenaline) $-$ A common hormone in vertebrates, and involved in short term stress (``fight or flight'')

    \begin{enumerate}

      \item Acts as a first messenger, which activates G-protein, which activates an enzyme, which releases cAMP. Overall, this inhibits glycogen synthesis and promotes breakdown

    \end{enumerate}
    
  \item Signaling process through epinephrine:

    \begin{enumerate}

      \item Epinephrine binds to G-protein, which becomes an active G-protein, which activates adenylyl cyclase, which turns ATP into cAMP, which activates the protein kinase, which activates the phosphorylation kinase, which activates glycogen phosphorylase, which turns glycogen into glucose, ultimately releasing $10^8$ molecules

    \end{enumerate}

  \item Complication $-$ A branching network

    \begin{enumerate}

      \item Quorum Sensing $-$ Communication among microbes trigger group response once a certain population density is reached

      \item Yeast Mating $-$ Mating type in (haploid) yeast is genetically determined. Two mating types (a and $\alpha$). Each makes signaling molecules that the other receives. The reception of a mating factor leads to the production of a mating ``shmoo.'' Fusion of shmoos = diploid yeast cell, and meiosis ensues.

      \item Apoptosis $-$ Programmed cell death is programmed because of the signaling pathway that it is programmed to. Death proteins are always present, but inactive.

    \end{enumerate}

  \item Hormones: Chemical signals that cause a response in \textit{target cells} (receptor proteins for specific hormones)

    \begin{enumerate}

      \item Affects 1 tissue, a few, or most tissues in body

      \item Regulation of homeostasis by positive and negative feedback

      \item Homeostasis: maintaining a constant internal balance

    \end{enumerate}

  \item Negative Feedback Loop Example: Insulin and Glucagon control blood glucose levels

  \item Control of blood glucose in steps:

    \begin{enumerate}

      \item High blood glucose

      \item Insulin released from pancreas

      \item Body cells take up glucose, Liver stores glucose as glycogen

      \item Blood glucose drops

      \item Glucagon released from pancreas

      \item Liver breaks down glycogen and released glucose into blood

    \end{enumerate}

  \item Positive Feedback Loop: Reinforces a signal, leading to an even greater response (amplification)

\end{itemize}

\end{document}

