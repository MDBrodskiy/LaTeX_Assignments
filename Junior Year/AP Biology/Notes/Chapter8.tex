%%%%%%%%%%%%%%%%%%%%%%%%%%%%%%%%%%%%%%%%%%%%%%%%%%%%%%%%%%%%%%%%%%%%%%%%%%%%%%%%%%%%%%%%%%%%%%%%%%%%%%%%%%%%%%%%%%%%%%%%%%%%%%%%%%%%%%%%%%%%%%%%%%%%%%%%%%%%%%%%%%%%%%%%%%%%%%%%%%%%%%%%%%%%
% Written By Michael Brodskiy
% Class: AP Biology
% Professor: J. Polivka
%%%%%%%%%%%%%%%%%%%%%%%%%%%%%%%%%%%%%%%%%%%%%%%%%%%%%%%%%%%%%%%%%%%%%%%%%%%%%%%%%%%%%%%%%%%%%%%%%%%%%%%%%%%%%%%%%%%%%%%%%%%%%%%%%%%%%%%%%%%%%%%%%%%%%%%%%%%%%%%%%%%%%%%%%%%%%%%%%%%%%%%%%%%%

\documentclass[12pt]{article} 
\usepackage{alphalph}
\usepackage[utf8]{inputenc}
\usepackage[russian,english]{babel}
\usepackage{titling}
\usepackage{amsmath}
\usepackage{graphicx}
\usepackage{enumitem}
\usepackage{amssymb}
\usepackage[super]{nth}
\usepackage{everysel}
\usepackage{ragged2e}
\usepackage{geometry}
\usepackage{fancyhdr}
\usepackage[super]{nth}
\usepackage{expl3}
\usepackage[version=4]{mhchem}
\usepackage{hpstatement}
\usepackage{rsphrase}
\usepackage{cancel}
\usepackage{siunitx}
\geometry{top=1.0in,bottom=1.0in,left=1.0in,right=1.0in}
\newcommand{\subtitle}[1]{%
  \posttitle{%
    \par\end{center}
    \begin{center}\large#1\end{center}
    \vskip0.5em}%

}
\usepackage{hyperref}
\hypersetup{
colorlinks=true,
linkcolor=blue,
filecolor=magenta,      
urlcolor=blue,
citecolor=blue,
}

\urlstyle{same}


\title{Chapter 8}
\date{November 13, 2020}
\author{Michael Brodskiy\\ \small Instructor: Mrs. Polivka}

% Mathematical Operations:

% Sum: $$\sum_{n=a}^{b} f(x) $$
% Integral: $$\int_{lower}^{upper} f(x) dx$$
% Limit: $$\lim_{x\to\infty} f(x)$$

\begin{document}

\maketitle

\begin{itemize}

  \item A living cell is like a miniature factory, where thousands of reactions occur, which converts energy in many ways

  \item All of these chemical reactions transform energy from one form to another

    \begin{enumerate}

      \item Example: Plants convert solar energy to ATP and organic molecules

    \end{enumerate}

  \item A \textbf{metabolic pathway} begins with a specific molecule and ends with a product

  \item Each step is catalyzed by a specific enzyme

  \item Catabolic Reactions:

    \begin{enumerate}

      \item \textbf{Breaking bonds} between molecules releases energy (hydrolysis)

      \item Example: cellular respiration breaks down glucose

    \end{enumerate}

  \item Anabolic Reactions:

    \begin{enumerate}

      \item \textbf{Forming bonds} between molecules consumes energy (dehydration synthesis)

      \item Example: Protein synthesis

    \end{enumerate}

  \item Organisms are \textbf{endergonic} systems. What is energy used for?

    \begin{enumerate}

      \item Synthesis (making biomolecules)

      \item Reproduction

      \item Movement

      \item Active Transport

      \item Temperature Regulation

    \end{enumerate}

  \item Energy is obtained from using \textbf{exergonic} (catabolic) reactions to fuel \textbf{endergonic} (anabolic) reactions

  \item Energy is obtained through:

    \begin{enumerate}

      \item Eating high energy \textbf{organic molecules}
        
      \item Breaking those molecules down

      \item Capture released energy in a form the cell can use (ATP)

    \end{enumerate}

  \item An energy currency:

    \begin{enumerate}

      \item A way to pass energy around

      \item A short-term way of storing energy (ATP)

    \end{enumerate}

  \item What is an ATP?

    \begin{enumerate}

      \item A modified nucleotide = adenine + ribose +$P_i \to$ AMP (monophosphate)

      \item AMP + $P_i \to$ ADP (diphosphate)

      \item ADP + $P_i \to$ ATP (triphosphate)

      \item Adding phosphates is endergonic 

      \item High energy bonds are located between phosphates

    \end{enumerate}

  \item Subsequent \ce{PO4} molecules become more and more difficult to add

  \item Bonding of the negative $P_i$ groups is unstable, and is ``spring-loaded'' in that it may be taken off to release energy easily

  \item ATP to ADP releases about $\Delta G=7.3 \left[ \frac{\kilo\calorie}{\mole}  \right]$

  \item This fuels other reactions

  \item The molecules phosphorylate (therefore, it is a \textbf{kinase})

  \item ATP can not be stored (this is why it is short-term)

  \item Carbohydrates and fats are long-term energy storage

  \item Breaking down large molecules requires an initial input of energy (\textbf{activation energy}). Large biomolecules are stable, and, therefore, require more energy.

  \item Activation Energy $-$ Amount of energy needed to destabilize the bonds of a molecule, and move the reaction over an ``energy hill''

  \item Catalysts $-$ Proteins that reduce the amount of energy needed to start a reaction

  \item An enzyme is a type of catalyst, as they increase the rate of reaction without being consumed, as well as reduce the activation energy required, and they do not change free energy ($\Delta G$) released or required.

  \item Catalysts are highly specific, as they are used on a specific substrate (also, catalysts are required to sustain life)

  \item \textbf{Substrate} $-$ A reactant which binds to an enzyme, and form enzyme-substrate complexes, which are a temporary association.

  \item \textbf{Active Site} $-$ An enzyme's catalytic site, where the substrate fits in.

  \item \textbf{Reaction Specific} $-$ Each enzyme only works with a specific substrate (chemical fit between active site and substrate).

  \item Single enzyme molecules are not consumed during reactions

  \item Enzymes are affected by any condition that affects protein structure (temperature, pH, salinity)

  \item Example: \textbf{Sucrase} breaks down sucrose
    
  \item Substrate binding causes enzymes to change shape to become a tighter fit (known as \textbf{conformational change}), and bring chemical groups to a position to catalyze the reaction

  \item There are multiple ways catalysts lower activation energy:

    \begin{enumerate}

      \item Synthesis $-$ active site orients substrates in correct position for reaction (enzyme brings substrate closer together)

      \item Digestion $-$ active site binds substrate and puts stress on bonds that must be broken, making it easier to separate molecules

    \end{enumerate}

  \item What factors affect enzyme function:

    \begin{enumerate}

      \item Enzyme concentration

      \item Substrate concentration

      \item Temperature

      \item pH

      \item Salinity

    \end{enumerate}

  \item As enzyme concentration increases, the reaction rate increases

    \begin{enumerate}

      \item Reaction rate may level off when the substrate becomes a limiting factor, meaning that not all enzyme molecules may find a substrate

    \end{enumerate}

  \item As substrate concentration increases, the reaction rate increases

    \begin{enumerate}

      \item Reaction rate may level off when the enzyme is saturated (reaches maximum level of reaction rate)

    \end{enumerate}

  \item Enzymes have optimal temperature (temperature at which greatest collisions occur)

    \begin{enumerate}

      \item \textbf{Optimum Temperature} $-$ Around 35 to 40 degrees Celsius for humans

      \item Heat will increase movement of molecule to certain extent, which will cause the bonds in the enzyme and the bonds between enzyme and substrate to disrupt (known as \textbf{denaturation}, when a structure loses 3D shape)

      \item Cold makes molecules move slower and collide less often, causing less reactions

      \item Optimal Temperature depends on organism (for example, bacteria in geysers will need to be able to take heat)

    \end{enumerate}    

  \item A certain level of pH is most optimal

    \begin{enumerate}

      \item Adding or removing \ce{H+} ions may disrupt bonds and cause denaturation

      \item Optimal pH for humans is around 6-7

      \item Pepsin (in stomach) functions best around 2-3

      \item Trypsin (in small intestines) functions best around 8 

    \end{enumerate}

  \item Salt concentration also has an optimal level

    \begin{enumerate}

      \item Changes in salinity can cause removal or addition of cations (+) and anions (-)

      \item Dead sea is dead for a reason

    \end{enumerate}

  \item \textbf{Activators} $-$ Compounds which aid enzymes

    \begin{enumerate}

      \item \textbf{Cofactors} $-$ Non-protein, small, inorganic compounds and ions (Ex. \ce{Mg}, \ce{K}, \ce{Ca}, \ce{Zn}, \ce{Fe}, and \ce{Cu})

      \item \textbf{Coenzymes} $-$ Non-protein, small, organic molecules  (bind to enzyme near active site, includes many vitamins)

    \end{enumerate}

  \item Inhibitors

    \begin{enumerate}

      \item Molecules reduce enzyme activity

      \item Competitive Inhibitor

      \item Noncompetitive Inhibitor

      \item Irreversible Inhibition

      \item Feedback Inhibition

    \end{enumerate}

  \item Competitive Inhibitor:

    \begin{enumerate}

      \item Compete for active site with substrate

      \item Example: penicillin blocks enzyme bacteria use to build cell walls

      \item Example: Disulfiram (antabuse) treats chronic alcoholism by blocking enzyme that breaks down alcohol, and causes severe hangover and vomiting 5-10 minutes after drinking

      \item Can be overcome by increasing substrate concentration

    \end{enumerate}

  \item Noncompetitive Inhibitor:

    \begin{enumerate}

      \item Binds to other sites than active site

      \item Allosteric Inhibitor $-$ binds to allosteric site, causing enzyme to change shape (conformational change), and active site becomes non-functional

      \item Example: Cyanide inhibits production of ATP by stopping Cytochrome C, an enzyme involved in cellular respiration.

    \end{enumerate}

  \item Irreversible Inhibitor:

    \begin{enumerate}

      \item Competitors bind to active site permanently

      \item Allosterics bind to allosteric site, permanently changing shape of enzyme

      \item Allosteric regulation causes conformational changes by regulatory molecules (inhibitors will keep enzymes in inactive form, while activators keep enzyme in active form)

    \end{enumerate}

  \item Metabolic pathways are results of evolution and increase efficiency with intermediate branching points, and increase control and regulation

  \item A product is used in the next step of a pathway. A final product is usually an inhibitor of a previous step. This is called \textbf{feedback inhibition}, and it helps unnecessary accumulation of a product

  \item Example: Synthesis of amino acid isoleucine from amino acid threonine causes more isoleucines to form, which are allosteric inhibitors of the first step in teh pathway

\end{itemize}

\end{document}

