%%%%%%%%%%%%%%%%%%%%%%%%%%%%%%%%%%%%%%%%%%%%%%%%%%%%%%%%%%%%%%%%%%%%%%%%%%%%%%%%%%%%%%%%%%%%%%%%%%%%%%%%%%%%%%%%%%%%%%%%%%%%%%%%%%%%%%%%%%%%%%%%%%%%%%%%%%%%%%%%%%%%%%%%%%%%%%%%%%%%%%%%%%%%
% Written By Michael Brodskiy
% Class: AP Biology
% Professor: J. Polivka
%%%%%%%%%%%%%%%%%%%%%%%%%%%%%%%%%%%%%%%%%%%%%%%%%%%%%%%%%%%%%%%%%%%%%%%%%%%%%%%%%%%%%%%%%%%%%%%%%%%%%%%%%%%%%%%%%%%%%%%%%%%%%%%%%%%%%%%%%%%%%%%%%%%%%%%%%%%%%%%%%%%%%%%%%%%%%%%%%%%%%%%%%%%%

\documentclass[12pt]{article} 
\usepackage{alphalph}
\usepackage[utf8]{inputenc}
\usepackage[russian,english]{babel}
\usepackage{titling}
\usepackage{amsmath}
\usepackage{graphicx}
\usepackage{enumitem}
\usepackage{amssymb}
\usepackage[super]{nth}
\usepackage{everysel}
\usepackage{ragged2e}
\usepackage{geometry}
\usepackage{fancyhdr}
\usepackage[super]{nth}
\usepackage{expl3}
\usepackage[version=4]{mhchem}
\usepackage{hpstatement}
\usepackage{rsphrase}
\usepackage{cancel}
\usepackage{siunitx}
\geometry{top=1.0in,bottom=1.0in,left=1.0in,right=1.0in}
\newcommand{\subtitle}[1]{%
  \posttitle{%
    \par\end{center}
    \begin{center}\large#1\end{center}
    \vskip0.5em}%

}
\usepackage{hyperref}
\hypersetup{
colorlinks=true,
linkcolor=blue,
filecolor=magenta,      
urlcolor=blue,
citecolor=blue,
}

\urlstyle{same}


\title{Chapter 14}
\date{January 13, 2020}
\author{Michael Brodskiy\\ \small Instructor: Mrs. Polivka}

% Mathematical Operations:

% Sum: $$\sum_{n=a}^{b} f(x) $$
% Integral: $$\int_{lower}^{upper} f(x) dx$$
% Limit: $$\lim_{x\to\infty} f(x)$$

\begin{document}

\maketitle

\begin{itemize}

  \item Mendel looked at true-breeding purple-flower peas and true-breeding white-flower peas. He went through three generations: $P$, $F_1$, and $F_2$, which are generations 1, 2, and 3, respectively.

  \item Purple was the dominant allele, with the white flower being recessive.

  \item Wild type $-$ Refers to the most common (usually dominant) allele.

  \item Punnett Squares $-$ Used to estimate geno- and pheno- types

    \begin{enumerate}

      \item Genotype ratio $-$ PP:Pp:pp

      \item Phenotype ratio $-$ Dominant:Recessive

    \end{enumerate}

  \item Law of Segregation $-$ During meiosis, alleles segregate, and homologous chromosomes separate. Each allele for a trait is packaged into a separate gamete.

  \item Law of Independent Association $-$ Different loci (genes) separate into gametes independently (non-homologous chromosomes align independently). This is only true for genes on separate chromosomes or on same chromosome so far apart that crossing over happens frequently.

  \item Monohybrid Cross $-$ A probability of a single characteristic

  \item Dihybrid Cross $-$ Probability of two characteristics

  \item Mendelian inheritance rules of probability:

    \begin{enumerate}

      \item Probability of \_\_\_\_ AND \_\_\_\_ happening: multiply the two ratios together

      \item Probability of \_\_\_\_ OR \_\_\_\_ happening: add the two ratios together

    \end{enumerate}

  \item Mendel worked with a simple system $-$ Traits are controlled by a single gene, with 2 alleles and 1 dominant to the other

  \item Incomplete Dominance $-$ RR is for red flowers, rr is for white flowers. An incompletely dominant gene would mean Rr is pink.

  \item Codominance $-$ When two alleles are codominant, both are expressed (black and white colored chickens, blood type AB)

  \item Polygenic Traits $-$ These traits are controlled by two or more genes, such as skin color or height. 

\end{itemize}

\end{document}

