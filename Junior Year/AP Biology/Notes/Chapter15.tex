%%%%%%%%%%%%%%%%%%%%%%%%%%%%%%%%%%%%%%%%%%%%%%%%%%%%%%%%%%%%%%%%%%%%%%%%%%%%%%%%%%%%%%%%%%%%%%%%%%%%%%%%%%%%%%%%%%%%%%%%%%%%%%%%%%%%%%%%%%%%%%%%%%%%%%%%%%%%%%%%%%%%%%%%%%%%%%%%%%%%%%%%%%%%
% Written By Michael Brodskiy
% Class: AP Biology
% Professor: J. Polivka
%%%%%%%%%%%%%%%%%%%%%%%%%%%%%%%%%%%%%%%%%%%%%%%%%%%%%%%%%%%%%%%%%%%%%%%%%%%%%%%%%%%%%%%%%%%%%%%%%%%%%%%%%%%%%%%%%%%%%%%%%%%%%%%%%%%%%%%%%%%%%%%%%%%%%%%%%%%%%%%%%%%%%%%%%%%%%%%%%%%%%%%%%%%%

\documentclass[12pt]{article} 
\usepackage{alphalph}
\usepackage[utf8]{inputenc}
\usepackage[russian,english]{babel}
\usepackage{titling}
\usepackage{amsmath}
\usepackage{graphicx}
\usepackage{enumitem}
\usepackage{amssymb}
\usepackage[super]{nth}
\usepackage{everysel}
\usepackage{ragged2e}
\usepackage{geometry}
\usepackage{fancyhdr}
\usepackage[super]{nth}
\usepackage{expl3}
\usepackage[version=4]{mhchem}
\usepackage{hpstatement}
\usepackage{rsphrase}
\usepackage{cancel}
\usepackage{siunitx}
\geometry{top=1.0in,bottom=1.0in,left=1.0in,right=1.0in}
\newcommand{\subtitle}[1]{%
  \posttitle{%
    \par\end{center}
    \begin{center}\large#1\end{center}
    \vskip0.5em}%

}
\usepackage{hyperref}
\hypersetup{
colorlinks=true,
linkcolor=blue,
filecolor=magenta,      
urlcolor=blue,
citecolor=blue,
}

\urlstyle{same}


\title{Chapter 15}
\date{January 29, 2020}
\author{Michael Brodskiy\\ \small Instructor: Mrs. Polivka}

% Mathematical Operations:

% Sum: $$\sum_{n=a}^{b} f(x) $$
% Integral: $$\int_{lower}^{upper} f(x) dx$$
% Limit: $$\lim_{x\to\infty} f(x)$$

\begin{document}

\maketitle

\begin{itemize}

  \item The chromosome theory of inheritance states:

    \begin{enumerate}

      \item Mendelian genes have specified loci (location) on chromosomes

      \item Chromosomes undergo segregation and independent assortment

    \end{enumerate}

  \item Genes that are close together on the same chromosome are linked and do not assort independently

    \begin{enumerate}

      \item Unlinked genes are either on separate chromosomes or are far apart on the same chromosome and assort independently.

      \item Crossing over of homologous chromosomes was the mechanism of phenotypes different than the parents

    \end{enumerate}

  \item Recombinant Offspring

    \begin{enumerate}

      \item Are those that show new combinations of the parental traits

    \end{enumerate}

  \item A linkage map

    \begin{enumerate}

      \item An ordered list of the genetic loci along a particular chromosome

      \item Can be developed using recombination frequencies

    \end{enumerate}

  \item The farther apart genes are on a chromosome:

    \begin{enumerate}

      \item The more likely they are to be separated during crossing over

      \item The higher their recombination frequencies will be

    \end{enumerate}

  \item The sex chromosomes:

    \begin{enumerate}

      \item Have genes for many characters unrelated to sex

    \end{enumerate}

  \item A gene located on either sex chromosome:

    \begin{enumerate}

      \item Is called a sex-linked gene

    \end{enumerate}

  \item Some recessive alleles found on the X chromosome in humans cause certain types of disorders

    \begin{enumerate}

      \item Color Blindness
        
      \item Hemophilia

    \end{enumerate}

  \item Alterations of chromosome number or structure cause some genetic disorders

  \item Large-scale chromosomal alterations

    \begin{enumerate}

      \item Often lead to spontaneous abortions, or cause a variety of developmental disorders

    \end{enumerate}

  \item When nondisjunction occurs

    \begin{enumerate}

      \item Pairs of homologous chromosomes do not separate normally during meiosis

      \item Gametes contain two copies or no copies of a particular chromosome.

      \item Examples:

        \begin{enumerate}

          \item Trisomy 21

          \item XXX Syndrome

        \end{enumerate}

    \end{enumerate}

  \item Alterations of chromosome structure:

    \begin{enumerate}

      \item A deletion removes a chromosomal segment

      \item A duplication repeats a segment

      \item An inversion reverses a segment within a chromosome

      \item A translocation moves a segment from one chromosome to another, nonhomologous one

    \end{enumerate}

\end{itemize}

\end{document}

