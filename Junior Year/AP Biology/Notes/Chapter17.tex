%%%%%%%%%%%%%%%%%%%%%%%%%%%%%%%%%%%%%%%%%%%%%%%%%%%%%%%%%%%%%%%%%%%%%%%%%%%%%%%%%%%%%%%%%%%%%%%%%%%%%%%%%%%%%%%%%%%%%%%%%%%%%%%%%%%%%%%%%%%%%%%%%%%%%%%%%%%%%%%%%%%%%%%%%%%%%%%%%%%%%%%%%%%%
% Written By Michael Brodskiy
% Class: AP Biology
% Professor: J. Polivka
%%%%%%%%%%%%%%%%%%%%%%%%%%%%%%%%%%%%%%%%%%%%%%%%%%%%%%%%%%%%%%%%%%%%%%%%%%%%%%%%%%%%%%%%%%%%%%%%%%%%%%%%%%%%%%%%%%%%%%%%%%%%%%%%%%%%%%%%%%%%%%%%%%%%%%%%%%%%%%%%%%%%%%%%%%%%%%%%%%%%%%%%%%%%

\documentclass[12pt]{article} 
\usepackage{alphalph}
\usepackage[utf8]{inputenc}
\usepackage[russian,english]{babel}
\usepackage{titling}
\usepackage{amsmath}
\usepackage{graphicx}
\usepackage{enumitem}
\usepackage{amssymb}
\usepackage[super]{nth}
\usepackage{everysel}
\usepackage{ragged2e}
\usepackage{geometry}
\usepackage{fancyhdr}
\usepackage[super]{nth}
\usepackage{expl3}
\usepackage[version=4]{mhchem}
\usepackage{hpstatement}
\usepackage{rsphrase}
\usepackage{cancel}
\usepackage{siunitx}
\geometry{top=1.0in,bottom=1.0in,left=1.0in,right=1.0in}
\newcommand{\subtitle}[1]{%
  \posttitle{%
    \par\end{center}
    \begin{center}\large#1\end{center}
    \vskip0.5em}%

}
\usepackage{hyperref}
\hypersetup{
colorlinks=true,
linkcolor=blue,
filecolor=magenta,      
urlcolor=blue,
citecolor=blue,
}

\urlstyle{same}


\title{Chapter 17}
\date{February 10, 2020}
\author{Michael Brodskiy\\ \small Instructor: Mrs. Polivka}

% Mathematical Operations:

% Sum: $$\sum_{n=a}^{b} f(x) $$
% Integral: $$\int_{lower}^{upper} f(x) dx$$
% Limit: $$\lim_{x\to\infty} f(x)$$

\begin{document}

\maketitle

\begin{itemize}

  \item Protein Synthesis is Split into Two Steps:

    \begin{enumerate}

      \item Transcription $-$ Synthesis of RNA using DNA as a template (occurs in the nucleus)

      \item Translation $-$ Actual synthesis of a polypeptide using mRNA (occurs in the cytoplasm, specifically the ribosome)

    \end{enumerate}

  \item ``Central Dogma'' $-$ Flow of genetic information in a cell

  \item DNA $\rightarrow$ RNA $\rightarrow$ protein $\rightarrow$ trait

  \item RNA $-$ Ribose sugar, uracil instead of thymine, single stranded, and comes in three forms: mRNA, tRNA, and rRNA

  \item RNA Polymerase separates 2 strands and adds nucleotides (does not need primer or helicase, like DNA)

  \item Promoter Region $-$ A binding site before the beginning of the gene

    \begin{enumerate}

  \item The TATA box binding site is a repeating AT sequence

  \item Binding site for RNA polymerase and transcription factors

  \item Transcription factors (suite of DNA-binding proteins) bind to promoter region, and turn on or off transcription, which triggers the binding of RNA polymerase to DNA

\end{enumerate}

\item RNA bases are matched to DNA bases on one of the DNA strands, goes in the 5' to 3' direction

\item Transcription Process

  \begin{enumerate}

    \item Initiation $-$ Transcription factors mediate the binding of RNA polymerase to an initiation sequence (TAT box)

    \item Elongation $-$ RNA polymerase continues unwinding DNA and adding nucleotides to the 3' end

    \item Termination $-$ RNA polymerase reaches a (codon) terminator sequence, such as UGA, UAA, or UAG

  \end{enumerate}

\item Post-transcriptional processing

  \begin{enumerate}

    \item Need to protect mRNA from enzymes on its trips from nucleus to cytoplasm

    \item Enzymes in cytoplasm attack mRNA

    \item Protect ends of the molecule

    \item Add 5' GTP cap

    \item Add poly-A tail (50-250+ A nucleotides)

    \item Longer tail, mRNA lasts longer, producing more protein

    \item Eukaryotic genes are not continuous, split into segments

    \item RNA splicing

      \begin{enumerate}

        \item Exons $-$ the real gene

          \begin{enumerate}

            \item Expressed/coding DNA

          \end{enumerate}

        \item Introns $-$ the junk

          \begin{enumerate}

            \item In between sequence

          \end{enumerate}

      \end{enumerate}

  \end{enumerate}
  
\item Splicing must be accurate! A single base added or lost throws off the reading frame

\item RNA Splicing Enzymes (snRNPs)

  \begin{enumerate}

    \item Small nuclear RNA

    \item Proteins

  \end{enumerate}

\item Spliceosome

  \begin{enumerate}

    \item Several snRNPs

    \item Recognize splice site sequence

      \begin{enumerate}

        \item Cut and paste gene

      \end{enumerate}

  \end{enumerate}

\item Alternative Splicing

  \begin{enumerate}

    \item A single gene can code for more than one protein

      \begin{enumerate}

        \item Certain introns may be included or exons excluded

        \item Allows humans to have a large diversity of proteins

      \end{enumerate}

  \end{enumerate}

\item DNA transcribes to mRNA, which is translated into proteins, which can code for traits

\item Translation $-$ From nucleic acid language to amino acid language

  \begin{enumerate}

    \item mRNA codes for proteins in triplets called codons

    \item The Codons

      \begin{enumerate}

        \item Code for all life

        \item Support theory for a common origin of all life

        \item Code is redundant (several codons for each amino acid)

        \item Third base is called a ``wobble''

        \item Start Codon

          \begin{enumerate}

            \item AUG $-$ Methionine

          \end{enumerate}

        \item Stop Codons

          \begin{enumerate}

            \item UGA, UAA, UAG

          \end{enumerate}

        \item tRNA uses anti-codons, attached to an amino acid, to compliment codons

          \begin{enumerate}

            \item tRNA transfers amino acids from cytoplasm to ribosome $-$ Very by anticodons and amino acid attached to end

          \end{enumerate}

      \end{enumerate}

    \item Ribosomes $-$ Facilitate coupling of tRNA anticodon to mRNA codon

      \begin{enumerate}

        \item Structure $-$ Made of ribosomal RNA (rRNA) \& proteins, and 2 subunits (large and small), which makes it functional only when the two units are attached

        \item A site (aminoacyl-tRNA site) $-$ holds tRNA carrying next amino acid to be added to the chain

        \item P site (peptidyl-tRNA site) $-$ holds tRNA carrying growing polypeptide chain

        \item E site (exit site) $-$ empty tRNA leaves ribosome from exit site

      \end{enumerate}

  \end{enumerate}

\item Building a polypeptide

  \begin{enumerate}

    \item Initiation $-$ brings together mRNA, ribosome subunits, initiator tRNA

    \item Elongation $-$ adding amino acids based on codon sequence

    \item Termination $-$ end codon

  \end{enumerate}

\item Transcription and translation are simultaneous in prokaryotes

  \begin{enumerate}

    \item DNA is in cytoplasm

    \item No mRNA editing

    \item Ribosomes read mRNA as it is being transcribed

  \end{enumerate}

\item Prokaryotes vs Eukaryotes $-$ Time and physical separation between the processes (eukaryotes take about one hour to go from DNA to protein), and has no RNA processing

\item Mutations

  \begin{enumerate}

    \item Point Mutations (Single base change)

      \begin{enumerate}

        \item Silent Mutation $-$ No amino acid change due to redundancy in code

        \item Missense Mutation $-$ Change amino acid

        \item Nonsense Mutation $-$ Changes to stop codon

      \end{enumerate}

    \item Frameshift Mutations (Shift the reading frame)

      \begin{enumerate}

        \item Insertions $-$ Adding bases

        \item Deletions $-$ Losing bases

      \end{enumerate}

  \end{enumerate}

\item If mutations occur in gametes, it affects the next generation, bot not in somatic cells

\item Sickle Cell Anemia $-$ Single point mutation

  \begin{enumerate}

    \item Primarily Africans

      \begin{enumerate}

        \item Recessive inheritance pattern

        \item Strikes 1 out of 400 African Americans

        \item The sixth amino acid, which is supposed to be Glu and is hydrophilic, is mutated into Val, which is hydrophobic

      \end{enumerate}

  \end{enumerate}

\item Cystic Fibrosis $-$ Deletion frameshift mutation

  \begin{enumerate}

    \item Recessive

    \item Normal allele codes for a membrane protein that transports \ce{Cl-} across cell membrane

      \begin{enumerate}

        \item Defective of absent channels

        \item Thicker and stickier mucus oats around cells

        \item Mucus build-up in various areas

      \end{enumerate}

    \item CTT is deleted from the sequence

  \end{enumerate}

\end{itemize}

\end{document}

