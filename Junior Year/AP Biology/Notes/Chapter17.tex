%%%%%%%%%%%%%%%%%%%%%%%%%%%%%%%%%%%%%%%%%%%%%%%%%%%%%%%%%%%%%%%%%%%%%%%%%%%%%%%%%%%%%%%%%%%%%%%%%%%%%%%%%%%%%%%%%%%%%%%%%%%%%%%%%%%%%%%%%%%%%%%%%%%%%%%%%%%%%%%%%%%%%%%%%%%%%%%%%%%%%%%%%%%%
% Written By Michael Brodskiy
% Class: AP Biology
% Professor: J. Polivka
%%%%%%%%%%%%%%%%%%%%%%%%%%%%%%%%%%%%%%%%%%%%%%%%%%%%%%%%%%%%%%%%%%%%%%%%%%%%%%%%%%%%%%%%%%%%%%%%%%%%%%%%%%%%%%%%%%%%%%%%%%%%%%%%%%%%%%%%%%%%%%%%%%%%%%%%%%%%%%%%%%%%%%%%%%%%%%%%%%%%%%%%%%%%

\documentclass[12pt]{article} 
\usepackage{alphalph}
\usepackage[utf8]{inputenc}
\usepackage[russian,english]{babel}
\usepackage{titling}
\usepackage{amsmath}
\usepackage{graphicx}
\usepackage{enumitem}
\usepackage{amssymb}
\usepackage[super]{nth}
\usepackage{everysel}
\usepackage{ragged2e}
\usepackage{geometry}
\usepackage{fancyhdr}
\usepackage[super]{nth}
\usepackage{expl3}
\usepackage[version=4]{mhchem}
\usepackage{hpstatement}
\usepackage{rsphrase}
\usepackage{cancel}
\usepackage{siunitx}
\geometry{top=1.0in,bottom=1.0in,left=1.0in,right=1.0in}
\newcommand{\subtitle}[1]{%
  \posttitle{%
    \par\end{center}
    \begin{center}\large#1\end{center}
    \vskip0.5em}%

}
\usepackage{hyperref}
\hypersetup{
colorlinks=true,
linkcolor=blue,
filecolor=magenta,      
urlcolor=blue,
citecolor=blue,
}

\urlstyle{same}


\title{Chapter 17}
\date{February 10, 2020}
\author{Michael Brodskiy\\ \small Instructor: Mrs. Polivka}

% Mathematical Operations:

% Sum: $$\sum_{n=a}^{b} f(x) $$
% Integral: $$\int_{lower}^{upper} f(x) dx$$
% Limit: $$\lim_{x\to\infty} f(x)$$

\begin{document}

\maketitle

\begin{itemize}

  \item Protein Synthesis is Split into Two Steps:

    \begin{enumerate}

      \item Transcription $-$ Synthesis of RNA using DNA as a template (occurs in the nucleus)

      \item Translation $-$ Actual synthesis of a polypeptide using mRNA (occurs in the cytoplasm, specifically the ribosome)

    \end{enumerate}

  \item ``Central Dogma'' $-$ Flow of genetic information in a cell

  \item DNA $\rightarrow$ RNA $\rightarrow$ protein $\rightarrow$ trait

  \item RNA $-$ Ribose sugar, uracil instead of thymine, single stranded, and comes in three forms: mRNA, tRNA, and rRNA

  \item RNA Polymerase separates 2 strands and adds nucleotides (does not need primer or helicase, like DNA)

  \item Promoter Region $-$ A binding site before the beginning of the gene

    \begin{enumerate}

  \item The TATA box binding site is a repeating AT sequence

  \item Binding site for RNA polymerase and transcription factors

  \item Transcription factors (suite of DNA-binding proteins) bind to promoter region, and turn on or off transcription, which triggers the binding of RNA polymerase to DNA

\end{enumerate}

\item RNA bases are matched to DNA bases on one of the DNA strands, goes in the 5' to 3' direction

\item Transcription Process

  \begin{enumerate}

    \item Initiation $-$ Transcription factors mediate the binding of RNA polymerase to an initiation sequence (TAT box)

    \item Elongation $-$ RNA polymerase continues unwinding DNA and adding nucleotides to the 3' end

    \item Termination $-$ RNA polymerase reaches a (codon) terminator sequence, such as UGA, UAA, or UAG

  \end{enumerate}

\item Post-transcriptional processing

  \begin{enumerate}

    \item Need to protect mRNA from enzymes on its trips from nucleus to cytoplasm

    \item Enzymes in cytoplasm attack mRNA

    \item Protect ends of the molecule

    \item Add 5' GTP cap

    \item Add poly-A tail (50-250+ A nucleotides)

    \item Longer tail, mRNA lasts longer, producing more protein

    \item Eukaryotic genes are not continuous, split into segments

    \item RNA splicing

      \begin{enumerate}

        \item Exons $-$ the real gene

          \begin{enumerate}

            \item Expressed/coding DNA

          \end{enumerate}

        \item Introns $-$ the junk

          \begin{enumerate}

            \item In between sequence

          \end{enumerate}

      \end{enumerate}

  \end{enumerate}
  
\item Splicing must be accurate! A single base added or lost throws off the reading frame

\item RNA Splicing Enzymes (snRNPs)

  \begin{enumerate}

    \item Small nuclear RNA

    \item Proteins

  \end{enumerate}

\item Spliceosome

  \begin{enumerate}

    \item Several snRNPs

    \item Recognize splice site sequence

      \begin{enumerate}

        \item Cut and paste gene

      \end{enumerate}

  \end{enumerate}

\item Alternative Splicing

  \begin{enumerate}

    \item A single gene can code for more than one protein

      \begin{enumerate}

        \item Certain introns may be included or exons excluded

        \item Allows humans to have a large diversity of proteins

      \end{enumerate}

  \end{enumerate}

\item DNA transcribes to mRNA, which is translated into proteins, which can code for traits

\end{itemize}

\end{document}

