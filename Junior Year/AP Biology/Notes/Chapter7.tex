%%%%%%%%%%%%%%%%%%%%%%%%%%%%%%%%%%%%%%%%%%%%%%%%%%%%%%%%%%%%%%%%%%%%%%%%%%%%%%%%%%%%%%%%%%%%%%%%%%%%%%%%%%%%%%%%%%%%%%%%%%%%%%%%%%%%%%%%%%%%%%%%%%%%%%%%%%%%%%%%%%%%%%%%%%%%%%%%%%%%%%%%%%%%
% Written By Michael Brodskiy
% Class: AP Biology
% Professor: J. Polivka
%%%%%%%%%%%%%%%%%%%%%%%%%%%%%%%%%%%%%%%%%%%%%%%%%%%%%%%%%%%%%%%%%%%%%%%%%%%%%%%%%%%%%%%%%%%%%%%%%%%%%%%%%%%%%%%%%%%%%%%%%%%%%%%%%%%%%%%%%%%%%%%%%%%%%%%%%%%%%%%%%%%%%%%%%%%%%%%%%%%%%%%%%%%%

\documentclass[12pt]{article} 
\usepackage{alphalph}
\usepackage[utf8]{inputenc}
\usepackage[russian,english]{babel}
\usepackage{titling}
\usepackage{amsmath}
\usepackage{graphicx}
\usepackage{enumitem}
\usepackage{amssymb}
\usepackage[super]{nth}
\usepackage{everysel}
\usepackage{ragged2e}
\usepackage{geometry}
\usepackage{fancyhdr}
\usepackage{cancel}
\geometry{top=1.0in,bottom=1.0in,left=1.0in,right=1.0in}
\newcommand{\subtitle}[1]{%
  \posttitle{%
    \par\end{center}
    \begin{center}\large#1\end{center}
    \vskip0.5em}%

}
\usepackage{hyperref}
\hypersetup{
colorlinks=true,
linkcolor=blue,
filecolor=magenta,      
urlcolor=blue,
citecolor=blue,
}

\urlstyle{same}


\title{Chapter 7}
\date{September 18, 2020}
\author{Michael Brodskiy\\ \small Instructor: Mrs. Polivka}

% Mathematical Operations:

% Sum: $$\sum_{n=a}^{b} f(x) $$
% Integral: $$\int_{lower}^{upper} f(x) dx$$
% Limit: $$\lim_{x\to\infty} f(x)$$

\begin{document}

\maketitle

\begin{itemize}

  \item Phospholipids have hydrophilic heads and hydrophobic “tails.”

    \begin{enumerate}

      \item Useful because inside and outside of cell is made of water.

    \end{enumerate}

  \item Carbohydrates on surface of cell membrane define what type of cell it is and let other cells identify it.

  \item Non-polar tails face each other, while polar heads point out (into and outside of cell). 

  \item Molecules that are impermeable to the cell membrane are large and have charge.

  \item Small, non-polar molecules are able to pass through the membrane without difficulty.

  \item Proteins, depending on their structures, can have hydrophobic and philic amino acids, which makes them useful for transport.

    \begin{enumerate}

      \item Inside of the membrane are nonpolar amino acid-based proteins which are hydrophobic and anchor the protein into the membrane.

      \item Polar amino acid-based proteins appear on the outer surfaces of the membrane and permeate into the fluid inside or outside the cell. These proteins are hydrophilic and extend into the extracellular fluid and cytosol.

    \end{enumerate}

  \item Integral (trans-membrane protein) spans the whole membrane and are often channel proteins.

  \item Glyco -lipids and -proteins stick out from the cell and aid cell communication.

  \item Cholesterol affects the membrane fluidity. The more cholesterol, the more fluid the cell membrane.

  \item Carbohydrates are non-polar, and, thus, they must be attached to a protein or lipid, or else they would be rejected by the polar head of the lipid bilayer.

  \item Passive transport has two types of diffusion, both of which do not require energy, as molecules move across their concentration gradient:

    \begin{enumerate}

      \item Simple Diffusion $-$ Molecules simply move across the membrane

      \item Facilitated Diffusion $-$ Channel proteins move molecules through their channels into or out of the cell.

    \end{enumerate}

  \item Active Transport $-$ This involves ATP, as a channel protein uses a protein pump to move molecules away from their concentration gradient.

  \item Large molecules move through exo- or endo- cytosis.

    \begin{enumerate}

      \item Exocyotosis $-$ Forms a vacuole or vesicle to move particles out of the cell.

      \item Endocytosis $-$ Forms a vacuole or vesicle to move particles into the cell.

    \end{enumerate}

  \item  Direction of osmosis is determined by comparing total solute concentrations.

    \begin{enumerate}

      \item Hypertonic $-$ More solute, less water.

      \item Hypotonic $-$ Less solute,  more water.

      \item Isotonic $-$ Equal solute, equal water.

    \end{enumerate}

  \item Cell survival depends on balancing water uptake and loss:

    \begin{enumerate}

      \item In a hypotonic solution, an animal cell becomes lysed (burst), whereas a plant cell stands due to its turgid pressure directed outwards.

      \item In an isotonic solution, animal cells are normal, while plant cells become flaccid.

      \item In a hypertonic solution, animal cells shrivel and die, while plant cells become plasmolyzed.

    \end{enumerate}



\end{itemize}

\end{document}

