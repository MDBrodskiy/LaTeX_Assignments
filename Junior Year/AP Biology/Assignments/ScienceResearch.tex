%%%%%%%%%%%%%%%%%%%%%%%%%%%%%%%%%%%%%%%%%%%%%%%%%%%%%%%%%%%%%%%%%%%%%%%%%%%%%%%%%%%%%%%%%%%%%%%%%%%%%%%%%%%%%%%%%%%%%%%%%%%%%%%%%%%%%%%%%%%%%%%%%%%%%%%%%%%%%%%%%%%%%%%%%%%%%%%%%%%%%%%%%%%%
% Written By Michael Brodskiy
% Class: AP Biology
% Professor: J. Polivka
%%%%%%%%%%%%%%%%%%%%%%%%%%%%%%%%%%%%%%%%%%%%%%%%%%%%%%%%%%%%%%%%%%%%%%%%%%%%%%%%%%%%%%%%%%%%%%%%%%%%%%%%%%%%%%%%%%%%%%%%%%%%%%%%%%%%%%%%%%%%%%%%%%%%%%%%%%%%%%%%%%%%%%%%%%%%%%%%%%%%%%%%%%%%

\documentclass[12pt]{article} 
\usepackage{alphalph}
\usepackage[utf8]{inputenc}
\usepackage[russian,english]{babel}
\usepackage{titling}
\usepackage{amsmath}
\usepackage{graphicx}
\usepackage{enumitem}
\usepackage{amssymb}
\usepackage{physics}
\usepackage{tikz}
\usepackage{mathdots}
\usepackage{yhmath}
\usepackage{cancel}
\usepackage{color}
\usepackage{siunitx}
\usepackage{array}
\usepackage{multirow}
\usepackage{gensymb}
\usepackage{tabularx}
\usepackage{booktabs}
\usetikzlibrary{fadings}
\usetikzlibrary{patterns}
\usetikzlibrary{shadows.blur}
\usetikzlibrary{shapes}
\usepackage[super]{nth}
\usepackage{expl3}
\usepackage[version=4]{mhchem}
\usepackage{hpstatement}
\usepackage{rsphrase}
\usepackage{everysel}
\usepackage{ragged2e}
\usepackage{geometry}
\usepackage{fancyhdr}
\usepackage{cancel}
\geometry{top=1.0in,bottom=1.0in,left=1.0in,right=1.0in}
\newcommand{\subtitle}[1]{%
  \posttitle{%
    \par\end{center}
    \begin{center}\large#1\end{center}
    \vskip0.5em}%

}
\usepackage{hyperref}
\hypersetup{
colorlinks=true,
linkcolor=blue,
filecolor=magenta,      
urlcolor=blue,
citecolor=blue,
}

\urlstyle{same}


\title{Genetic Engineering Research}
\date{\today $-$ Period 5}
\author{Michael Brodskiy\\ \small Instructor: Mrs. Polivka}

% Mathematical Operations:

% Sum: $$\sum_{n=a}^{b} f(x) $$
% Integral: $$\int_{lower}^{upper} f(x) dx$$
% Limit: $$\lim_{x\to\infty} f(x)$$

\begin{document}

\maketitle

\begin{justify}

  \href{https://go.gale.com/ps/retrieve.do?resultListType=RELATED_DOCUMENT\&searchType=ts\&userGroupName=waln177103\&inPS=true\&contentSegment=\&prodId=SCIC\&docId=GALE|OXRGLA933669590\&it=r}{The article} began by mentioning some people involved in the genetic engineering field, and, eventually, it arrives at Paul Berg, who is referred to as the ``Father of Genetic Engineering.'' In 1973, Berg joined the DNA of two viruses, but his methods were too onerous. It wasn't until the discovery of an enzyme that sped up the Berg procedure that scientists were truly able to partake in genetic engineering. Overall, this procedure requires three components: a gene to be transferred, a cell to host the gene, and a vector to conduct transfer. Generally, plasmids are used for the process. Given its nature, the method has taken on two different names: gene splicing, and recombinant DNA research. The articles then goes over benefits and issues, ultimately moving into questions about ethics and safety.

\end{justify}

\begin{justify}

  All in all, the article did shed new light on genetic modification and procedures. In addition to this, it also brought up information regarding genetically modified crops, which is important for sustaining a large population. I did hope that there would be more of a mention on how procedures to treat DNA-based illnesses work, but only the illnesses themselves were mentioned. It was also very interesting that they brought up the first synthetically-made organism. Additionally, I did not know that genetic modification existed as early as 1973, and it did shock me to find this out. Overall, it does seem the benefits outweigh the risks, however, only time will tell.

\end{justify}

\end{document}

