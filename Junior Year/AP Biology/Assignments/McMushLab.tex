%%%%%%%%%%%%%%%%%%%%%%%%%%%%%%%%%%%%%%%%%%%%%%%%%%%%%%%%%%%%%%%%%%%%%%%%%%%%%%%%%%%%%%%%%%%%%%%%%%%%%%%%%%%%%%%%%%%%%%%%%%%%%%%%%%%%%%%%%%%%%%%%%%%%%%%%%%%%%%%%%%%%%%%%%%%%%%%%%%%%%%%%%%%%
% Written By Michael Brodskiy
% Class: AP Biology
% Professor: J. Polivka
%%%%%%%%%%%%%%%%%%%%%%%%%%%%%%%%%%%%%%%%%%%%%%%%%%%%%%%%%%%%%%%%%%%%%%%%%%%%%%%%%%%%%%%%%%%%%%%%%%%%%%%%%%%%%%%%%%%%%%%%%%%%%%%%%%%%%%%%%%%%%%%%%%%%%%%%%%%%%%%%%%%%%%%%%%%%%%%%%%%%%%%%%%%%

\documentclass[12pt]{article} 
\usepackage{alphalph}
\usepackage[utf8]{inputenc}
\usepackage[russian,english]{babel}
\usepackage{titling}
\usepackage{amsmath}
\usepackage{graphicx}
\usepackage{enumitem}
\usepackage{amssymb}
\usepackage[super]{nth}
\usepackage{everysel}
\usepackage{ragged2e}
\usepackage{geometry}
\usepackage{fancyhdr}
\usepackage{cancel}
\geometry{top=1.0in,bottom=1.0in,left=1.0in,right=1.0in}
\newcommand{\subtitle}[1]{%
  \posttitle{%
    \par\end{center}
    \begin{center}\large#1\end{center}
    \vskip0.5em}%

}
\usepackage{hyperref}
\hypersetup{
colorlinks=true,
linkcolor=blue,
filecolor=magenta,      
urlcolor=blue,
citecolor=blue,
}

\urlstyle{same}


\title{On the Macromolecules Present in a McDonald's Happy Meal}
\date{September 16, 2020 $-$ Period 5}
\author{Michael Brodskiy\\ \small Instructor: Mrs. Polivka}

% Mathematical Operations:

% Sum: $$\sum_{n=a}^{b} f(x) $$
% Integral: $$\int_{lower}^{upper} f(x) dx$$
% Limit: $$\lim_{x\to\infty} f(x)$$

\begin{document}

\maketitle

\section{Introduction}

\begin{justify}
In this lab, McDonald's food was tested for four different molecules: monosaccharides, polysaccharides, lipids, and proteins. To test, three different indicators were used: Benedict's Solution, Lugol's Solution, Sudan III Solution, and Biuret's Solution, respectively. Water was used as a negative control group, while solutions with known macromolecules were used as positive control groups. The resultant color and texture, after the addition of the indicator, was used to determine whether a given macromolecule was present. The purpose of this was to determine macromolecules present in the food.
\end{justify}

\subsection{Definitions}

\begin{tabular}{p{.15\textwidth} p{.75\textwidth}}

  Macromolecule & A large molecule, which, in biology, is defined as a carbohydrate, lipid, protein, or nucleic acid. \\
  Monomer & A molecule group that may form a polymer by bonding to similar molecule groups. \\
  Carbohydrate & A macromolecule made up of monosaccharide monomers, and polysaccharide polymers. It appears in such foods as sugars and starches.  \\
  Lipid & A macromolecule made up of fatty acid and glycerol monomers. It appears in such foods as oils and fats.\\
  Protein & A macromolecule with amino acid monomers. It appears in such foods as meats and fish.\\

\end{tabular}

\subsection{Hypothesis}
\begin{justify}
  If the food in a McDonald's happy meal is tested for three macromolecules (monosaccharide and polysaccharides, lipids, and proteins), then all of the tested molecules will be present, as the food has bread (sugars and starches), meat (burger patty), and grease (lipids).
\end{justify}
\section{Conclusion}

\begin{justify}

  Although I initially expected for all of the macromolecules tested to be positive, it is odd that the food tested negative for proteins, as this contradicts the hypothesis. Usually, by knowing what kind of food something is (bread, meat, oils), it is possible to guess what kind of macromolecules are present. This, however, was not the case for the McDonald�s food. Carbohydrates and lipids were present, however, proteins were not. This means that whatever McDonald�s uses for their burger patties, chicken nuggets, and any other advertised meat products are not made of real meat, or the proteins are greatly overpowered by the amount of carbohydrates and lipids. For the other tested macromolecules, though, the colors generated by the indicators were clear. It is possible that there was some error in finding proteins, as the color was somewhat unexpected, and therefore inconclusive (it became a pale yellow). It is possible that an incorrect food sample was used for this. Therefore, because it came out inconclusive, it would be better to retest, rather than forming a conclusive statement.

\end{justify}

\end{document}

