%%%%%%%%%%%%%%%%%%%%%%%%%%%%%%%%%%%%%%%%%%%%%%%%%%%%%%%%%%%%%%%%%%%%%%%%%%%%%%%%%%%%%%%%%%%%%%%%%%%%%%%%%%%%%%%%%%%%%%%%%%%%%%%%%%%%%%%%%%%%%%%%%%%%%%%%%%%%%%%%%%%%%%%%%%%%%%%%%%%%%%%%%%%%
% Written By Michael Brodskiy
% Class: AP Physics C
% Professor: S. Morse
%%%%%%%%%%%%%%%%%%%%%%%%%%%%%%%%%%%%%%%%%%%%%%%%%%%%%%%%%%%%%%%%%%%%%%%%%%%%%%%%%%%%%%%%%%%%%%%%%%%%%%%%%%%%%%%%%%%%%%%%%%%%%%%%%%%%%%%%%%%%%%%%%%%%%%%%%%%%%%%%%%%%%%%%%%%%%%%%%%%%%%%%%%%%

\documentclass[12pt]{article} 
\usepackage{alphalph}
\usepackage[utf8]{inputenc}
\usepackage[russian,english]{babel}
\usepackage{titling}
\usepackage{amsmath}
\usepackage{graphicx}
\usepackage{enumitem}
\usepackage{amssymb}
\usepackage[super]{nth}
\usepackage{everysel}
\usepackage{ragged2e}
\usepackage{geometry}
\usepackage{fancyhdr}
\usepackage{cancel}
\usepackage{siunitx}
\usepackage{expl3}
\usepackage[version=4]{mhchem}
\usepackage{hpstatement}
\usepackage{rsphrase}
\usepackage{physics}
\usepackage{tikz}
\usepackage{mathdots}
\usepackage{yhmath}
\usepackage{cancel}
\usepackage{color}
\usepackage{array}
\usepackage{multirow}
\usepackage{gensymb}
\usepackage{tabularx}
\usepackage{booktabs}
\usetikzlibrary{fadings}
\usetikzlibrary{patterns}
\usetikzlibrary{shadows.blur}
\usetikzlibrary{shapes}
\geometry{top=1.0in,bottom=1.0in,left=1.0in,right=1.0in}
\newcommand{\subtitle}[1]{%
  \posttitle{%
    \par\end{center}
    \begin{center}\large#1\end{center}
    \vskip0.5em}%

}
\usepackage{hyperref}
\hypersetup{
colorlinks=true,
linkcolor=blue,
filecolor=magenta,      
urlcolor=blue,
citecolor=blue,
}

\urlstyle{same}


\title{Physics C: Mechanics Final Exam}
\date{\today}
\author{Michael Brodskiy\\ \small Instructor: Mrs. Morse}

% Mathematical Operations:

% Sum: $$\sum_{n=a}^{b} f(x) $$
% Integral: $$\int_{lower}^{upper} f(x) dx$$
% Limit: $$\lim_{x\to\infty} f(x)$$

\begin{document}

\maketitle

\begin{center}
  Problem One:
\end{center}
    \hline

    \begin{equation}
      \begin{split}
        \frac{1}{2}kx^2=mgh\\
        h=\frac{kx^2}{2mg}\\
      \end{split}
      \label{1}
    \end{equation}

    \begin{equation}
      \begin{split}
        h(60)\approx.6[\si{\meter}]\\
        h(10)\approx.1[\si{\meter}]\\
        \frac{.6-.1}{60-10}=.01[\si{\kilo\gram\meter}]\\
      \end{split}
      \label{2}
    \end{equation}

    \begin{equation}
      \begin{split}
        h=\frac{1}{100m}\\
      \end{split}
      \label{3}
    \end{equation}

    \begin{equation}
      \begin{split}
        \frac{1}{100}=\frac{kx^2}{2g}\\
        \frac{1}{5}=kx^2\\
        x^2=.05^2=.0025[\si{\meter\squared}]\\
        k=\frac{.2}{.0025}=80\left[ \frac{\si{\newton}}{\si{\meter}} \right]
      \end{split}
      \label{4}
    \end{equation}

    \hline

    \newpage

\begin{center}
  Problem Two:
\end{center}
    \hline
    
    \begin{figure}[h]
      \centering
      \tikzset{every picture/.style={line width=0.75pt}} %set default line width to 0.75pt        

\begin{tikzpicture}[x=0.75pt,y=0.75pt,yscale=-1,xscale=1]
%uncomment if require: \path (0,300); %set diagram left start at 0, and has height of 300

%Shape: Circle [id:dp5520731613008669] 
\draw   (171,145) .. controls (171,131.19) and (182.19,120) .. (196,120) .. controls (209.81,120) and (221,131.19) .. (221,145) .. controls (221,158.81) and (209.81,170) .. (196,170) .. controls (182.19,170) and (171,158.81) .. (171,145) -- cycle ;
%Straight Lines [id:da12065599323998066] 
\draw    (196,170) -- (195.51,258) ;
\draw [shift={(195.5,260)}, rotate = 270.32] [color={rgb, 255:red, 0; green, 0; blue, 0 }  ][line width=0.75]    (10.93,-3.29) .. controls (6.95,-1.4) and (3.31,-0.3) .. (0,0) .. controls (3.31,0.3) and (6.95,1.4) .. (10.93,3.29)   ;
%Shape: Boxed Line [id:dp90095876721953] 
\draw    (196,170) -- (195.75,287) ;
\draw [shift={(195.75,289)}, rotate = 270.12] [color={rgb, 255:red, 0; green, 0; blue, 0 }  ][line width=0.75]    (10.93,-3.29) .. controls (6.95,-1.4) and (3.31,-0.3) .. (0,0) .. controls (3.31,0.3) and (6.95,1.4) .. (10.93,3.29)   ;

% Text Node
\draw (204,233.4) node [anchor=north west][inner sep=0.75pt]    {$F_{N}$};
% Text Node
\draw (168,274.4) node [anchor=north west][inner sep=0.75pt]    {$F_{g}$};


\end{tikzpicture}

      \caption{Free Body Diagram at point B}
      \label{fig:1}
    \end{figure}

    \begin{equation}
      \begin{split}
        F_g=\frac{mv^2}{r}\\
        v^2=gr\\
        v=\sqrt{gr}\\
        v=4.47\left[ \frac{\si{\meter}}{\si{\second}} \right]
      \end{split}
      \label{5}
    \end{equation}

    \begin{equation}
      \begin{split}
        mgh_o+\frac{1}{2}mv^2_o=\frac{1}{2}mv^2_f\\
        KE_o=\frac{1}{2}mv^2_f-mgh_o+mgh_f\\
      .5\cdot.05\cdot(4.47)^2-.05\cdot10\cdot3.6+.05\cdot10\cdot4=.7[\si{\joule}]\\
      \end{split}
      \label{6}
    \end{equation}

    \begin{equation}
      \begin{split}
        E_{total}=\frac{1}{2}mv^2+mgh\\
        .5\cdot.05\cdot(4.47)^2+.05\cdot10\cdot4=2.5\left[  \si{\joule}\right]\\
        kx^2=5\\
        x=\sqrt{\frac{5}{200}}\\
        x=.16\left[ \si{\meter} \right]\\
      \end{split}
      \label{7}
    \end{equation}

    \begin{equation}
      \begin{split}
        mgh+\frac{1}{2}mv^2=2.5\\
        mv^2=5-2mgh\\
        v=\sqrt{\frac{5-2mgh}{m}}\\
        =7.75\left[ \frac{\si{\meter}}{\si{\second}} \right]
      \end{split}
      \label{8}
    \end{equation}

    \hline

    \newpage

\begin{center}
  Problem Three:
\end{center}

\hline

\begin{equation}
  \begin{split}
    \Delta x=3t\\
    5=\frac{1}{2}gt^2\\
    t=1[\si{\second}]\\
    \Delta x= 3(1)=3[\si{\meter}]\\
  \end{split}
  \label{9}
\end{equation}

\begin{equation}
  \begin{split}
    v_x=3\left[ \frac{\si{\meter}}{\si{\second}} \right]\\
    v_y=10(1)=10\left[ \frac{\si{\meter}}{\si{\second}} \right]\\
    v_t=\sqrt{3^2+10^2}\\
    v_t=10.44\left[ \frac{\si{\meter}}{\si{\second}} \right]
  \end{split}
  \label{10}
\end{equation}

\begin{equation}
  \begin{split}
    m_1v_1+m_2v_2=(m_1+m_2)v_f\\
    15\cdot3=(60)v_f\\
    v_f=.75\left[ \frac{\si{\meter}}{\si{\second}} \right]
  \end{split}
  \label{11}
\end{equation}

\begin{equation}
  \begin{split}
    .5\cdot60\cdot.75^2=\mu F_N\cdot7\\
    F_N=600[\si{\newton}]\\
    \mu=\frac{.5\cdot60\cdot.75^2}{600\cdot7}\\
    =.004\\
  \end{split}
  \label{12}
\end{equation}

\hline

\newpage

\begin{center}
  Problem Four:
\end{center}

\hline

\begin{center}
  Free Body Diagrams on next page
\end{center}

\begin{equation}
  \begin{split}
    \text{Situation One:}\\
    F_{N_{10}}=100[\si{\newton}]\\
  F_f=30[\si{\newton}]\\
  F_T-F_f=m_{10}a\\
  F_T=F_f+m_{10}a\\
  F_g-F_T=m_{5}a\\
  F_g-F_f=15a\\
  a=\frac{20}{15}=1.33\left[ \frac{\si{\meter}}{\si{\second\squared}} \right]\\
\end{split}
  \label{13}
\end{equation}

\begin{equation}
  \begin{split}
    \text{Situation Two:}\\
    F_{N_{10}}=100[\si{\newton}]\\
  F_f=30[\si{\newton}]\\
  F_T-F_f=ma\\
  F_T=F_f+ma\\
  F_a-F_f=ma\\
  F_a-F_f=10a\\
  a=\frac{20}{10}=2\left[ \frac{\si{\meter}}{\si{\second\squared}} \right]\\
\end{split}
  \label{14}
\end{equation}

    \begin{figure}[H]
      \centering
      \tikzset{every picture/.style={line width=0.75pt}} %set default line width to 0.75pt        

\begin{tikzpicture}[x=0.75pt,y=0.75pt,yscale=-1,xscale=1]
%uncomment if require: \path (0,300); %set diagram left start at 0, and has height of 300

%Shape: Circle [id:dp5520731613008669] 
\draw   (171,145) .. controls (171,131.19) and (182.19,120) .. (196,120) .. controls (209.81,120) and (221,131.19) .. (221,145) .. controls (221,158.81) and (209.81,170) .. (196,170) .. controls (182.19,170) and (171,158.81) .. (171,145) -- cycle ;
%Straight Lines [id:da12065599323998066] 
\draw    (196,170) -- (195.51,258) ;
\draw [shift={(195.5,260)}, rotate = 270.32] [color={rgb, 255:red, 0; green, 0; blue, 0 }  ][line width=0.75]    (10.93,-3.29) .. controls (6.95,-1.4) and (3.31,-0.3) .. (0,0) .. controls (3.31,0.3) and (6.95,1.4) .. (10.93,3.29)   ;
%Straight Lines [id:da8507937543436119] 
\draw    (221,145.25) -- (347.5,145) ;
\draw [shift={(349.5,145)}, rotate = 539.89] [color={rgb, 255:red, 0; green, 0; blue, 0 }  ][line width=0.75]    (10.93,-3.29) .. controls (6.95,-1.4) and (3.31,-0.3) .. (0,0) .. controls (3.31,0.3) and (6.95,1.4) .. (10.93,3.29)   ;
%Shape: Boxed Line [id:dp5987147186507646] 
\draw    (171,145) -- (83,144.51) ;
\draw [shift={(81,144.5)}, rotate = 360.32] [color={rgb, 255:red, 0; green, 0; blue, 0 }  ][line width=0.75]    (10.93,-3.29) .. controls (6.95,-1.4) and (3.31,-0.3) .. (0,0) .. controls (3.31,0.3) and (6.95,1.4) .. (10.93,3.29)   ;
%Shape: Boxed Line [id:dp2139323878030881] 
\draw    (196,120) -- (196.49,32) ;
\draw [shift={(196.5,30)}, rotate = 450.32] [color={rgb, 255:red, 0; green, 0; blue, 0 }  ][line width=0.75]    (10.93,-3.29) .. controls (6.95,-1.4) and (3.31,-0.3) .. (0,0) .. controls (3.31,0.3) and (6.95,1.4) .. (10.93,3.29)   ;
%Shape: Circle [id:dp33912625944533503] 
\draw   (441,145) .. controls (441,131.19) and (452.19,120) .. (466,120) .. controls (479.81,120) and (491,131.19) .. (491,145) .. controls (491,158.81) and (479.81,170) .. (466,170) .. controls (452.19,170) and (441,158.81) .. (441,145) -- cycle ;
%Straight Lines [id:da9178485185294811] 
\draw    (466,170) -- (466.49,293) ;
\draw [shift={(466.5,295)}, rotate = 269.77] [color={rgb, 255:red, 0; green, 0; blue, 0 }  ][line width=0.75]    (10.93,-3.29) .. controls (6.95,-1.4) and (3.31,-0.3) .. (0,0) .. controls (3.31,0.3) and (6.95,1.4) .. (10.93,3.29)   ;
%Shape: Boxed Line [id:dp2855235173942938] 
\draw    (466,120) -- (466.49,32) ;
\draw [shift={(466.5,30)}, rotate = 450.32] [color={rgb, 255:red, 0; green, 0; blue, 0 }  ][line width=0.75]    (10.93,-3.29) .. controls (6.95,-1.4) and (3.31,-0.3) .. (0,0) .. controls (3.31,0.3) and (6.95,1.4) .. (10.93,3.29)   ;

% Text Node
\draw (331,114.4) node [anchor=north west][inner sep=0.75pt]    {$F_{T}$};
% Text Node
\draw (167,225.4) node [anchor=north west][inner sep=0.75pt]    {$F_{g}$};
% Text Node
\draw (93,116.4) node [anchor=north west][inner sep=0.75pt]    {$F_{F}$};
% Text Node
\draw (170,32.4) node [anchor=north west][inner sep=0.75pt]    {$F_{N}$};
% Text Node
\draw (179,136.4) node [anchor=north west][inner sep=0.75pt]    {$10kg$};
% Text Node
\draw (437,225.4) node [anchor=north west][inner sep=0.75pt]    {$F_{g}$};
% Text Node
\draw (440,32.4) node [anchor=north west][inner sep=0.75pt]    {$F_{T}$};
% Text Node
\draw (452,135.4) node [anchor=north west][inner sep=0.75pt]    {$ \begin{array}{l}
5kg\\
\end{array}$};


\end{tikzpicture}

      \caption{Situation 1}
      \label{fig:2}
    \end{figure}

    \begin{figure}[H]
      \centering
      \tikzset{every picture/.style={line width=0.75pt}} %set default line width to 0.75pt        

\begin{tikzpicture}[x=0.75pt,y=0.75pt,yscale=-1,xscale=1]
%uncomment if require: \path (0,300); %set diagram left start at 0, and has height of 300

%Shape: Circle [id:dp5520731613008669] 
\draw   (171,145) .. controls (171,131.19) and (182.19,120) .. (196,120) .. controls (209.81,120) and (221,131.19) .. (221,145) .. controls (221,158.81) and (209.81,170) .. (196,170) .. controls (182.19,170) and (171,158.81) .. (171,145) -- cycle ;
%Straight Lines [id:da12065599323998066] 
\draw    (196,170) -- (195.51,258) ;
\draw [shift={(195.5,260)}, rotate = 270.32] [color={rgb, 255:red, 0; green, 0; blue, 0 }  ][line width=0.75]    (10.93,-3.29) .. controls (6.95,-1.4) and (3.31,-0.3) .. (0,0) .. controls (3.31,0.3) and (6.95,1.4) .. (10.93,3.29)   ;
%Straight Lines [id:da8507937543436119] 
\draw    (221,145.25) -- (347.5,145) ;
\draw [shift={(349.5,145)}, rotate = 539.89] [color={rgb, 255:red, 0; green, 0; blue, 0 }  ][line width=0.75]    (10.93,-3.29) .. controls (6.95,-1.4) and (3.31,-0.3) .. (0,0) .. controls (3.31,0.3) and (6.95,1.4) .. (10.93,3.29)   ;
%Shape: Boxed Line [id:dp5987147186507646] 
\draw    (171,145) -- (83,144.51) ;
\draw [shift={(81,144.5)}, rotate = 360.32] [color={rgb, 255:red, 0; green, 0; blue, 0 }  ][line width=0.75]    (10.93,-3.29) .. controls (6.95,-1.4) and (3.31,-0.3) .. (0,0) .. controls (3.31,0.3) and (6.95,1.4) .. (10.93,3.29)   ;
%Shape: Boxed Line [id:dp2139323878030881] 
\draw    (196,120) -- (196.49,32) ;
\draw [shift={(196.5,30)}, rotate = 450.32] [color={rgb, 255:red, 0; green, 0; blue, 0 }  ][line width=0.75]    (10.93,-3.29) .. controls (6.95,-1.4) and (3.31,-0.3) .. (0,0) .. controls (3.31,0.3) and (6.95,1.4) .. (10.93,3.29)   ;
%Shape: Circle [id:dp33912625944533503] 
\draw   (441,145) .. controls (441,131.19) and (452.19,120) .. (466,120) .. controls (479.81,120) and (491,131.19) .. (491,145) .. controls (491,158.81) and (479.81,170) .. (466,170) .. controls (452.19,170) and (441,158.81) .. (441,145) -- cycle ;
%Straight Lines [id:da9178485185294811] 
\draw    (466,170) -- (466.49,293) ;
\draw [shift={(466.5,295)}, rotate = 269.77] [color={rgb, 255:red, 0; green, 0; blue, 0 }  ][line width=0.75]    (10.93,-3.29) .. controls (6.95,-1.4) and (3.31,-0.3) .. (0,0) .. controls (3.31,0.3) and (6.95,1.4) .. (10.93,3.29)   ;
%Shape: Boxed Line [id:dp2855235173942938] 
\draw    (466,120) -- (466.49,32) ;
\draw [shift={(466.5,30)}, rotate = 450.32] [color={rgb, 255:red, 0; green, 0; blue, 0 }  ][line width=0.75]    (10.93,-3.29) .. controls (6.95,-1.4) and (3.31,-0.3) .. (0,0) .. controls (3.31,0.3) and (6.95,1.4) .. (10.93,3.29)   ;

% Text Node
\draw (331,114.4) node [anchor=north west][inner sep=0.75pt]    {$F_{T}$};
% Text Node
\draw (167,225.4) node [anchor=north west][inner sep=0.75pt]    {$F_{g}$};
% Text Node
\draw (93,116.4) node [anchor=north west][inner sep=0.75pt]    {$F_{F}$};
% Text Node
\draw (170,32.4) node [anchor=north west][inner sep=0.75pt]    {$F_{N}$};
% Text Node
\draw (179,136.4) node [anchor=north west][inner sep=0.75pt]    {$10kg$};
% Text Node
\draw (437,225.4) node [anchor=north west][inner sep=0.75pt]    {$F_{a}$};
% Text Node
\draw (440,32.4) node [anchor=north west][inner sep=0.75pt]    {$F_{T}$};
% Text Node
\draw (452,135.4) node [anchor=north west][inner sep=0.75pt]    {$ \begin{array}{l}
pull\\
\end{array}$};


\end{tikzpicture}

      \caption{Situation 2}
      \label{fig:3}
    \end{figure}

\hline

\newpage

\begin{center}
  Problem Five:
\end{center}
\hline

\begin{center}
  Free Body Diagrams on next page
\end{center}

\begin{equation}
  \begin{split}
    \int_0^6 100-10x\,dx=100x-5x^2\Big|_0^6\\
    =420[\si{\joule}]\\
  \end{split}
  \label{15}
\end{equation}

    \begin{figure}[H]
      \centering
      \tikzset{every picture/.style={line width=0.75pt}} %set default line width to 0.75pt        

\begin{tikzpicture}[x=0.75pt,y=0.75pt,yscale=-1,xscale=1]
%uncomment if require: \path (0,300); %set diagram left start at 0, and has height of 300

%Shape: Circle [id:dp14447258656558515] 
\draw  [color={rgb, 255:red, 0; green, 0; blue, 0 }  ,draw opacity=1 ][fill={rgb, 255:red, 0; green, 0; blue, 0 }  ,fill opacity=1 ] (109.65,155.72) .. controls (109.6,153.11) and (111.68,151) .. (114.29,151) .. controls (116.9,151) and (119.05,153.11) .. (119.09,155.72) .. controls (119.14,158.33) and (117.06,160.45) .. (114.45,160.45) .. controls (111.84,160.45) and (109.69,158.33) .. (109.65,155.72) -- cycle ;
%Shape: Boxed Line [id:dp7138136373100961] 
\draw    (109.65,155.72) -- (63.24,155.72) ;
\draw [shift={(61.24,155.72)}, rotate = 360] [color={rgb, 255:red, 0; green, 0; blue, 0 }  ][line width=0.75]    (10.93,-3.29) .. controls (6.95,-1.4) and (3.31,-0.3) .. (0,0) .. controls (3.31,0.3) and (6.95,1.4) .. (10.93,3.29)   ;
%Shape: Boxed Line [id:dp5235226248724378] 
\draw    (119.09,155.72) -- (149.09,155.72) ;
\draw [shift={(151.09,155.72)}, rotate = 180] [color={rgb, 255:red, 0; green, 0; blue, 0 }  ][line width=0.75]    (10.93,-3.29) .. controls (6.95,-1.4) and (3.31,-0.3) .. (0,0) .. controls (3.31,0.3) and (6.95,1.4) .. (10.93,3.29)   ;

% Text Node
\draw (76,138.92) node [anchor=west] [inner sep=0.75pt]    {${F_{e}}_{_{-4}}$};
% Text Node
\draw (146.88,138.92) node [anchor=east] [inner sep=0.75pt]    {${F_{e}}_{_{+6}}$};


\end{tikzpicture}

      \caption{Free Body Diagram for Doll}
      \label{fig:4}
    \end{figure}

    \begin{equation}
      \begin{split}
        420=\frac{1}{2}mv^2+mgh\\
        420-2.8\cdot10\cdot6\sin(60^{\circ})=\frac{1}{2}mv^2\\
        v=\sqrt{\frac{2(420-2.8\cdot10\cdot6\sin(60^{\circ}))}{2.8}}\\
        =14\left[ \frac{\si{\meter}}{\si{\second}} \right]
      \end{split}
      \label{16}
    \end{equation}

    \begin{equation}
      \begin{split}
        v_x=7\left[ \frac{\si{\meter}}{\si{\second}} \right]\\
        v_y=7\sqrt{3}\left[ \frac{\si{\meter}}{\si{\second}} \right]\\
        t=\frac{7\sqrt{3}}{10}=1.21[\si{\second}]\\
        t_{total}=2.42[\si{\second}]\\
        \Delta x=2.42\cdot7\\
        \Delta x=17[\si{\meter}]\\
      \end{split}
      \label{17}
    \end{equation}

\hline

\end{document}

