\documentclass[12pt]{article}

%
%Margin - 1 inch on all sides
%
\usepackage[letterpaper]{geometry}
\usepackage{times}
\usepackage{amsmath}
\usepackage{hyperref}
\hypersetup{
colorlinks=true,
linkcolor=blue,
filecolor=magenta,      
urlcolor=blue,
citecolor=blue,
}

\urlstyle{same}
\geometry{top=1.0in, bottom=1.0in, left=1.0in, right=1.0in}

%
%Doublespacing
%
\usepackage{setspace}
\doublespacing

%
%Rotating tables (e.g. sideways when too long)
%
\usepackage{rotating}


%
%Fancy-header package to modify header/page numbering (insert last name)
%
\usepackage{fancyhdr}
\pagestyle{fancy}
\lhead{} 
\chead{} 
\rhead{ Brodskiy \thepage} 
\lfoot{} 
\cfoot{} 
\rfoot{} 
\renewcommand{\headrulewidth}{0pt} 
\renewcommand{\footrulewidth}{0pt} 

\setlength\headsep{0.333in}


\newcommand{\bibent}{\noindent \hangindent 40pt}
\newenvironment{workscited}{\newpage \begin{center} Works Cited \end{center}}{\newpage }

\usepackage{everysel}
\usepackage{ragged2e}
%\renewcommand*\familydefault{\ttdefault}
%\EverySelectfont{%
%\fontdimen2\font=0.4em% interword space
%\fontdimen3\font=0.2em% interword stretch
%\fontdimen4\font=0.1em% interword shrink
%\fontdimen7\font=0.1em% extra space
%\hyphenchar\font=`\-% to allow hyphenation
%}

%\pagenumbering{gobble}
%
%Begin document
%
\begin{document}

%%%%First page name, class, etc
\setlength{\parindent}{0in}
Michael Brodskiy\\
Mrs. Greer\\
AP Language and Composition\\
Summer Assignment\\


%%%%Title
\begin{center}
Charles Krauthammer $-$ \href{https://www.washingtonpost.com/opinions/charles-krauthammer-the-enduring-miracle-of-the-american-constitution/2018/11/29/fb8879d2-f405-11e8-aeea-b85fd44449f5_story}{Article 1}\footnote{\url{https://www.washingtonpost.com/opinions/charles-krauthammer-the-enduring-miracle-of-the-american-constitution/2018/11/29/fb8879d2-f405-11e8-aeea-b85fd44449f5_story}} Pr\'ecis
\end{center}


%%%%Changes paragraph indentation to 0.5in
\setlength{\parindent}{0.5in}
%%%%Begin body of paper here

\begin{justify}
\begin{enumerate}

  \item Charles Krauthammer, a political commentator, writes about the American Constitution and its lasting effects in his column titled, ``The Enduring Miracle of the American Constitution'' (November 29, 2018), excerpted from Krauthammer's book, ``The Point of it All.'' Krauthammer begins by bringing up a memory of a switch of power in Egypt, comparing the Egyptian attitudes towards their respective constitution to that of America; Krauthammer uses this as a segue into, what he refers to, as the three miracles of the American constitution. Krauthammer writes this (opinion) column in order to marvel at the lasting influence of the American constitution, and to convince the reader that such adherence and reverence to a document is extraordinary. As such, the author connects with the American public by commenting upon the astounding roots of the country known as the United States of America. 
\end{enumerate}
\end{justify}
\newpage

\begin{center}
Charles Krauthammer $-$ \href{https://www.washingtonpost.com/opinions/charles-krauthammer-the-authoritarian-temptation/2019/11/08/a55427f4-01a2-11ea-8bab-0fc209e065a8_story}{Article 2}\footnote{\url{https://www.washingtonpost.com/opinions/charles-krauthammer-the-authoritarian-temptation/2019/11/08/a55427f4-01a2-11ea-8bab-0fc209e065a8_story}} Pr\'ecis
\end{center}
\begin{justify}
\begin{enumerate}
    \setcounter{enumi}{1}
  \item A political commentator by the name of Charles Krauthammer describes the shifting political climate of the world in his article titled, ``The Authoritarian Temptation'' (November 8, 2019), excerpted from ``The Point of it All,'' Krauthammer's posthumous novel. Krauthammer begins by recalling the late twentieth century $-$ most importantly the fall of the Soviet Union $-$ further through the article, he explains how the whole west has been, from the peak of history (the late twentieth century, in Krauthammer's opinion), progressing to a somewhat more Eastern style of conservative authoritarianism. The author transitions in order to comment on the political world climate, and to question whether America's founding values may be upheld within such a system. As a result, Krauthammer's article may attract readers from all over the globe, especially readers that seek to dive into contemporary politics; he connects with the audience through his transitions and memories of the early twentieth century. 

\end{enumerate}
\end{justify}
\newpage

\begin{center}
William Safire $-$ \href{https://www.nytimes.com/2009/07/24/opinion/24safire}{Article 3}\footnote{\url{https://www.nytimes.com/2009/07/24/opinion/24safire}} Pr\'ecis
\end{center}
\begin{justify}
\begin{enumerate}
    \setcounter{enumi}{2}

  \item William Safire recalled the fiery Kitchen Debates in his article titled, ``The Cold War's Hot Kitchen,'' that was released exactly 50 years after the Kitchen Debates, on July 23, 2009. Safire describes the events that took place at Sokolniki park and his experience while there; as a finale, he concluded that he met someone who he would later find out was the one and only Leonid Brezhnev. The article itself was not argumentative; however, it is evident that the author wrote this article in order to remind and inform the reader of the events that perspired 50 years prior. Therefore, the author connects with the reader by giving a quick history lesson to any possible reader.

\end{enumerate}
\end{justify}
\newpage

\begin{center}
William Safire $-$ \href{https://www.nytimes.com/2009/02/15/opinion/15iht-edsafire.1.20193264}{Article 4}\footnote{\url{https://www.nytimes.com/2009/02/15/opinion/15iht-edsafire.1.20193264}} Pr\'ecis
\end{center}
\begin{justify}
\begin{enumerate}
    \setcounter{enumi}{3}

  \item William Safire, a language analyst, discusses the use and roots of the prefix \textit{re}$-$ in modern society, in his article titled, ``In re: repurpose, rebrand, remix, remash'' (November 5, 2009), in which he also demonstrates that diffusion of a trend may take no time at all. Safire begins by discussing the Italian word, \textit{refacimento}, from which the prefix \textit{re}$-$ was taken from, and used as another way to say ``again.'' The author gives several examples, including the four in his title: repurpose, rebrand, remix, and remash, in order to show the history and demonstrate the popularity and use of these words, which were formed using the aforementioned prefix. Safire attracts any English language enthusiasts to his article, by analyzing the prefix \textit{re}$-$ and words that use it.

\end{enumerate}
\end{justify}
\newpage

\begin{center}
George F. Will $-$ \href{https://www.washingtonpost.com/opinions/the-thugocracy-of-vladimir-putin/2020/07/16/daea31ba-c78b-11ea-b037-f9711f89ee46_story}{Article 5}\footnote{\url{https://www.washingtonpost.com/opinions/the-thugocracy-of-vladimir-putin/2020/07/16/daea31ba-c78b-11ea-b037-f9711f89ee46_story}} Pr\'ecis
\end{center}
\begin{justify}
\begin{enumerate}
    \setcounter{enumi}{4}

  \item George Will, a political columnist for The Washington Post, discusses his opinion on the actions performed by the President of the Russia Federation, Vladimir Putin; Will also says that Putin is a thug for several crimes, which he mentions in his article, titled, ``The Thugocracy of Vladimir Putin'' (July 17, 2020). Will begins by mentioning some ways that Putin has and still is deceiving the public, and he later transitions into crimes that Putin used to rise, and retain, power. The author wrote this piece in order to notify the audience of Putin's seize of power and explain why he is unfit for constitutional presidency. The audience may be anyone interested in world politics; Will connects to them by using many examples. 

\end{enumerate}
\end{justify}
\newpage

\begin{center}
George F. Will $-$ \href{https://www.washingtonpost.com/opinions/global-opinions/china-extends-its-reign-of-random-fear/2020/07/03/ba0b176e-bc93-11ea-8cf5-9c1b8d7f84c6_story}{Article 6}\footnote{\url{https://www.washingtonpost.com/opinions/global-opinions/china-extends-its-reign-of-random-fear/2020/07/03/ba0b176e-bc93-11ea-8cf5-9c1b8d7f84c6_story}} Pr\'ecis
\end{center}
\begin{justify}
\begin{enumerate}
    \setcounter{enumi}{5}

  \item The political commentator, George Will, demonstrates that China is exerting a tyrannical force over the country of Hong Kong in his article, ``China Extends its Reign of Random Fear'' (July 3, 2020). Will begins by discussing parallels between the French Committee of Public Safety and the newly created, Chinese Commission for Safeguarding National Security, and later transitions into a deeper analysis of the events taking place in Hong Kong and China, including, but not limited to, the genocide of Muslim Uyghurs. The author brings up such matters in order to promote awareness of the topic, which, in turn, would help combat it. The author wants to attract as many readers as possible, while connecting with them by mentioning such humanitarian issues.

\end{enumerate}
\end{justify}
\end{document}

