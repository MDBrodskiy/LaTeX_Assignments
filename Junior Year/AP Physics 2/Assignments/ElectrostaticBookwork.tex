%%%%%%%%%%%%%%%%%%%%%%%%%%%%%%%%%%%%%%%%%%%%%%%%%%%%%%%%%%%%%%%%%%%%%%%%%%%%%%%%%%%%%%%%%%%%%%%%%%%%%%%%%%%%%%%%%%%%%%%%%%%%%%%%%%%%%%%%%%%%%%%%%%%%%%%%%%%%%%%%%%%%%%%%%%%%%%%%%%%%%%%%%%%%
% Written By Michael Brodskiy
% Class: AP Chemistry
% Professor: J. Morgan
%%%%%%%%%%%%%%%%%%%%%%%%%%%%%%%%%%%%%%%%%%%%%%%%%%%%%%%%%%%%%%%%%%%%%%%%%%%%%%%%%%%%%%%%%%%%%%%%%%%%%%%%%%%%%%%%%%%%%%%%%%%%%%%%%%%%%%%%%%%%%%%%%%%%%%%%%%%%%%%%%%%%%%%%%%%%%%%%%%%%%%%%%%%%

\documentclass[12pt]{article} 
\usepackage{alphalph}
\usepackage[utf8]{inputenc}
\usepackage[russian,english]{babel}
\usepackage{titling}
\usepackage{amsmath}
\usepackage{graphicx}
\usepackage{enumitem}
\usepackage{amssymb}
\usepackage[super]{nth}
\usepackage{expl3}
\usepackage[version=4]{mhchem}
\usepackage{hpstatement}
\usepackage{rsphrase}
\usepackage{everysel}
\usepackage{ragged2e}
\usepackage{geometry}
\usepackage{fancyhdr}
\usepackage{cancel}
\usepackage{siunitx}
\geometry{top=1.0in,bottom=1.0in,left=1.0in,right=1.0in}
\newcommand{\subtitle}[1]{%
  \posttitle{%
    \par\end{center}
    \begin{center}\large#1\end{center}
    \vskip0.5em}%

}
\DeclareSIUnit\Molar{\textsc{m}}
\usepackage{hyperref}
\hypersetup{
colorlinks=true,
linkcolor=blue,
filecolor=magenta,      
urlcolor=blue,
citecolor=blue,
}

\urlstyle{same}


\title{Electrostatics $-$ Problems 11, 34, 57}
\date{October 18, 2020}
\author{Michael Brodskiy\\ \small Instructor: Mrs. Morse}

% Mathematical Operations:

% Sum: $$\sum_{n=a}^{b} f(x) $$
% Integral: $$\int_{lower}^{upper} f(x) dx$$
% Limit: $$\lim_{x\to\infty} f(x)$$

\begin{document}

\maketitle

\begin{enumerate}

    \setcounter{enumi}{10}

  \item Positive particle with charge $6[\si{\nano\coulomb}] \rightarrow \ce{p^6+}$, positive particle with charge $5[\si{\nano\coulomb}] \rightarrow \ce{p^5+}$, negative particle with charge $-3[\si{\nano\coulomb}] \rightarrow \ce{e^3-}$ \eqref{1}

    \begin{equation}
      \begin{split}
        F_{\ce{e^3-}\text{ on } \ce{p^5+}} &= k\frac{|(5\cdot10^{-9})(-3\cdot10^{-9})|}{.1^2}\\
        &=1.35\cdot10^{-5}[\si{\newton}_{\text{down}}]\\
        F_{\ce{e^6+}\text{ on } \ce{p^5+}} &= k\frac{|(5\cdot10^{-9})(6\cdot10^{-9})|}{.3^2}\\
        &=3\cdot10^{-6}[\si{\newton}_{\text{left}}]\\
        ||F_{\text{on }\ce{p^5+}}||&= \sqrt{\left( 1.35\cdot10^{-5} \right)^2 + \left( 3\cdot10^{-6} \right)^2}\\
        &=1.38\cdot10^{-5}[\si{\newton}]\\
        \angle\ce{p^5+}&=\tan^{-1}\left( \frac{3\cdot10^{-6}}{1.35\cdot10^{-5}} \right)\\
        &=12.53^{\circ} \text{ (left of $270^{\circ}$ line)}\\
        \angle_f&= 270-12.53=257.47^{\circ}\\
        F_{\text{on }\ce{p^5+}}&=1.38\cdot10^{-5}[\si{\newton}]\text{ at }257.47^{\circ}
      \end{split}
      \label{1}
    \end{equation}

    \setcounter{enumi}{33}

  \item 

    \begin{enumerate}

      \item \eqref{2}

        \begin{equation}
          \begin{split}
            q_1&=-6\\
            q_2&=18\\
            \frac{q1}{q2}&=-\frac{1}{3}\\
          \end{split}
          \label{2}
        \end{equation}

      \item \eqref{3}

        \begin{equation}
          \begin{split}
            q_1&\text{ has a negative sign because the electric field is going into it}\\
            q_2&\text{ has a positive sign because the electric field is leaving it}
          \end{split}
          \label{3}
        \end{equation}

    \end{enumerate}

    \setcounter{enumi}{56}

  \item Field by particle with charge $3[\si{\nano\coulomb}]\rightarrow E_{\ce{p^3+}}$, Field by particle with charge $5[\si{\nano\coulomb}]\rightarrow E_{\ce{p^5+}}$, Field by particle with charge $-4[\si{\nano\coulomb}]\rightarrow E_{\ce{e^4-}}$ \eqref{4}

    \begin{equation}
      \begin{split}
        E_{\ce{p^3+}}&=k\frac{3\cdot10^{-9}}{1.2^2}\\
        &=18.72\left[ \frac{\si{\newton}}{\si{\coulomb}}\text{ right}\right]\\
        E_{\ce{p^5+}}&=k\frac{5\cdot10^{-9}}{2^2}\\
        &=11.23\left[ \frac{\si{\newton}}{\si{\coulomb}}\text{ right}\right]\\
        E_{\ce{e^4-}}&=k\frac{4\cdot10^{-9}}{2.5^2}\\
        &=5.75\left[ \frac{\si{\newton}}{\si{\coulomb}}\text{ left}\right]\\
        18.72+11.23-5.75&=24.2\left[ \frac{\si{\newton}}{\si{\coulomb}}\text{ right} \right]
      \end{split}
      \label{4}
    \end{equation}

\end{enumerate}

\end{document}

