%%%%%%%%%%%%%%%%%%%%%%%%%%%%%%%%%%%%%%%%%%%%%%%%%%%%%%%%%%%%%%%%%%%%%%%%%%%%%%%%%%%%%%%%%%%%%%%%%%%%%%%%%%%%%%%%%%%%%%%%%%%%%%%%%%%%%%%%%%%%%%%%%%%%%%%%%%%%%%%%%%%%%%%%%%%%%%%%%%%%%%%%%%%%
% Written By Michael Brodskiy
% Class: AP Physics 2
% Professor: Mrs. Morse
%%%%%%%%%%%%%%%%%%%%%%%%%%%%%%%%%%%%%%%%%%%%%%%%%%%%%%%%%%%%%%%%%%%%%%%%%%%%%%%%%%%%%%%%%%%%%%%%%%%%%%%%%%%%%%%%%%%%%%%%%%%%%%%%%%%%%%%%%%%%%%%%%%%%%%%%%%%%%%%%%%%%%%%%%%%%%%%%%%%%%%%%%%%%

\documentclass[12pt]{article} 
\usepackage{alphalph}
\usepackage[utf8]{inputenc}
\usepackage[russian,english]{babel}
\usepackage{titling}
\usepackage{amsmath}
\usepackage{graphicx}
\usepackage{enumitem}
\usepackage{amssymb}
\usepackage[super]{nth}
\usepackage{everysel}
\usepackage{ragged2e}
\usepackage{geometry}
\usepackage{fancyhdr}
\usepackage{cancel}
\usepackage{siunitx}
\geometry{top=1.0in,bottom=1.0in,left=1.0in,right=1.0in}
\newcommand{\subtitle}[1]{%
  \posttitle{%
    \par\end{center}
    \begin{center}\large#1\end{center}
    \vskip0.5em}%

}
\usepackage{hyperref}
\hypersetup{
colorlinks=true,
linkcolor=blue,
filecolor=magenta,      
urlcolor=blue,
citecolor=blue,
}

\urlstyle{same}


\title{AP Physics 2 $-$ Electrostatics}
\date{\today}
\author{Michael Brodskiy\\ \small Instructor: Mrs. Morse}

% Mathematical Operations:

% Sum: $$\sum_{n=a}^{b} f(x) $$
% Integral: $$\int_{lower}^{upper} f(x) dx$$
% Limit: $$\lim_{x\to\infty} f(x)$$

\begin{document}

\maketitle

\begin{itemize}

  \item Elementary Charge: The smallest amount of charge possible is the charge on the electron

  \item Charge of electron: $-1.6\cdot10^{-19}[\si{\coulomb}]$, Mass of electron: $9.11\cdot10^{-31}[\si{\kilo\gram}]$

  \item Charge of proton: $1.6\cdot10^{-19}[\si{\coulomb}]$, Mass of Proton: $1.67\cdot10^{27}[\si{\kilo\gram}]$

  \item $1[\si{\coulomb}]=6.24\cdot10^{18}\text{ electrons}$

  \item Conductor: charges can flow easily along the surface of the material (e.g. Copper Wire)

  \item Insulator: charges cannot flow easily along the surface of the material (e.g. Plastic)

  \item Neutral: means that the charges cancel out (Not no charge at all)

  \item Conduction: Two objects touch and conduct charge. Charge of sphere will have same sign as the charging rod

  \item Induction: Charging rod does not touch the sphere. Charge of sphere will have opposite sign of charging rod

\end{itemize}

\end{document}

