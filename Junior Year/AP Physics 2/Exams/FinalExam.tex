%%%%%%%%%%%%%%%%%%%%%%%%%%%%%%%%%%%%%%%%%%%%%%%%%%%%%%%%%%%%%%%%%%%%%%%%%%%%%%%%%%%%%%%%%%%%%%%%%%%%%%%%%%%%%%%%%%%%%%%%%%%%%%%%%%%%%%%%%%%%%%%%%%%%%%%%%%%%%%%%%%%%%%%%%%%%%%%%%%%%%%%%%%%%
% Written By Michael Brodskiy
% Class: AP Physics 2
% Professor: S. Morse
%%%%%%%%%%%%%%%%%%%%%%%%%%%%%%%%%%%%%%%%%%%%%%%%%%%%%%%%%%%%%%%%%%%%%%%%%%%%%%%%%%%%%%%%%%%%%%%%%%%%%%%%%%%%%%%%%%%%%%%%%%%%%%%%%%%%%%%%%%%%%%%%%%%%%%%%%%%%%%%%%%%%%%%%%%%%%%%%%%%%%%%%%%%%

\documentclass[12pt]{article} 
\usepackage{alphalph}
\usepackage[utf8]{inputenc}
\usepackage[russian,english]{babel}
\usepackage{titling}
\usepackage{amsmath}
\usepackage{graphicx}
\usepackage{enumitem}
\usepackage{amssymb}
\usepackage[super]{nth}
\usepackage{everysel}
\usepackage{ragged2e}
\usepackage{geometry}
\usepackage{fancyhdr}
\usepackage{cancel}
\usepackage{siunitx}
\usepackage{expl3}
\usepackage[version=4]{mhchem}
\usepackage{hpstatement}
\usepackage{rsphrase}
\usepackage{physics}
\usepackage{tikz}
\usepackage{mathdots}
\usepackage{yhmath}
\usepackage{cancel}
\usepackage{color}
\usepackage{array}
\usepackage{multirow}
\usepackage{gensymb}
\usepackage{tabularx}
\usepackage{booktabs}
\usetikzlibrary{fadings}
\usetikzlibrary{patterns}
\usetikzlibrary{shadows.blur}
\usetikzlibrary{shapes}
\geometry{top=1.0in,bottom=1.0in,left=1.0in,right=1.0in}
\newcommand{\subtitle}[1]{%
  \posttitle{%
    \par\end{center}
    \begin{center}\large#1\end{center}
    \vskip0.5em}%

}
\usepackage{hyperref}
\hypersetup{
colorlinks=true,
linkcolor=blue,
filecolor=magenta,      
urlcolor=blue,
citecolor=blue,
}

\urlstyle{same}


\title{Physics 2 Final Exam}
\date{\today}
\author{Michael Brodskiy\\ \small Instructor: Mrs. Morse}

% Mathematical Operations:

% Sum: $$\sum_{n=a}^{b} f(x) $$
% Integral: $$\int_{lower}^{upper} f(x) dx$$
% Limit: $$\lim_{x\to\infty} f(x)$$

\begin{document}

\maketitle

\begin{center}
  Problem One:
\end{center}
    \hline

    \begin{equation}
      \begin{split}
        W=F_g\\
        F_g=(4\cdot10^{-3})(1000)+3+4\\
        =11[\si{\newton}]\\
        \text{Because total weight is ball plus fluid plus beaker}
      \end{split}
      \label{1}
    \end{equation}

    \begin{figure}[h]
      \centering
      \tikzset{every picture/.style={line width=0.75pt}} %set default line width to 0.75pt        

\begin{tikzpicture}[x=0.75pt,y=0.75pt,yscale=-1,xscale=1]
%uncomment if require: \path (0,300); %set diagram left start at 0, and has height of 300

%Shape: Circle [id:dp5520731613008669] 
\draw   (171,145) .. controls (171,131.19) and (182.19,120) .. (196,120) .. controls (209.81,120) and (221,131.19) .. (221,145) .. controls (221,158.81) and (209.81,170) .. (196,170) .. controls (182.19,170) and (171,158.81) .. (171,145) -- cycle ;
%Straight Lines [id:da12065599323998066] 
\draw    (196,170) -- (195.51,258) ;
\draw [shift={(195.5,260)}, rotate = 270.32] [color={rgb, 255:red, 0; green, 0; blue, 0 }  ][line width=0.75]    (10.93,-3.29) .. controls (6.95,-1.4) and (3.31,-0.3) .. (0,0) .. controls (3.31,0.3) and (6.95,1.4) .. (10.93,3.29)   ;
%Shape: Boxed Line [id:dp90095876721953] 
\draw    (196,170) -- (195.75,287) ;
\draw [shift={(195.75,289)}, rotate = 270.12] [color={rgb, 255:red, 0; green, 0; blue, 0 }  ][line width=0.75]    (10.93,-3.29) .. controls (6.95,-1.4) and (3.31,-0.3) .. (0,0) .. controls (3.31,0.3) and (6.95,1.4) .. (10.93,3.29)   ;

% Text Node
\draw (204,233.4) node [anchor=north west][inner sep=0.75pt]    {$F_{N}$};
% Text Node
\draw (168,274.4) node [anchor=north west][inner sep=0.75pt]    {$F_{g}$};


\end{tikzpicture}

      \caption{Free Body Diagram of Rubber Ball}
      \label{fig:1}
    \end{figure}

    \begin{equation}
      \begin{split}
        F_b-F_g-F_T=0\\
        F_b=F_T+F_g\\
        4+6=10[\si{\newton}]\\
      \end{split}
      \label{2}
    \end{equation}

    \begin{equation}
      \begin{split}
        P=\rho g h\\
        1000\cdot10\cdot.15=1500[\si{\pascal}]\\
      \end{split}
      \label{3}
    \end{equation}

    \begin{equation}
      \begin{split}
        \text{The ball will displace less fluid, so lower}
      \end{split}
      \label{4}
    \end{equation}

    \hline

    \newpage

\begin{center}
  Problem Two:
\end{center}
    \hline
    
    \begin{equation}
      \begin{split}
        A_1v_1=A_2v_2\\
        \frac{v_2}{v_1}=\frac{A_1}{A_2}\\
      \end{split}
      \label{5}
    \end{equation}

    \begin{equation}
      \begin{split}
        \text{Velocity is greater, and is inversely related to pressure, so pressure decreases}
      \end{split}
      \label{6}
    \end{equation}

    \hline

    \newpage

\begin{center}
  Problem Three:
\end{center}

\hline

\begin{equation}
  \begin{split}
    \Delta U=\frac{3}{2}\Delta(PV)\\
    \frac{3}{2}\cdot-2000\cdot\frac{1}{2}=-1500[\si{\joule}]\\
  \end{split}
  \label{7}
\end{equation}

\begin{equation}
  \begin{split}
    \Delta U=Q+W\\
    Q=1000[\si{\joule}]\\
    W=P\Delta V\\
    \Delta V=0\\
    \Delta U = 1000[\si{\joule}]\\
  \end{split}
  \label{8}
\end{equation}

\hline

\newpage

\begin{center}
  Problem Four:
\end{center}

\hline

\begin{equation}
  \begin{split}
    Q(6)\approx400\\
    Q(3)\approx200\\
  \frac{Q(6)-Q(3)}{6-3}=\frac{200}{3}=66.7\left[ \frac{\si{\joule}}{\si{\second\kelvin}} \right]
\end{split}
  \label{9}
\end{equation}

\begin{equation}
  \begin{split}
    2[\si{\centi\meter}]=.02[\si{\meter}]\\
    6[\si{\centi\meter\squared}]=.0006[\si{\meter\squared}]\\
    \frac{Q}{\Delta t}=k\frac{A\Delta T}{L}\\
    k\frac{A}{L}=66.7\\
  k=\frac{.02}{.0006}\cdot66.7=2222.2\left[ \frac{\si{\joule}}{\si{\second\kelvin\meter}} \right]
\end{split}
  \label{10}
\end{equation}

\hline

\newpage

\begin{center}
  Problem Five:
\end{center}
\hline

\begin{center}
  Diagrams on next two pages
\end{center}

    \begin{equation}
      \begin{split}
        E_6=k\frac{Q}{r^2}\\
        9\cdot10^9\cdot\frac{6\cdot10^{-6}}{5^2}=2160\left[ \frac{\si{\newton}}{\si{\coulomb}} \right]\\
        E_4=k\frac{Q}{r^2}\\
        9\cdot10^9\cdot\frac{4\cdot10^{-6}}{4^2}=2250\left[ \frac{\si{\newton}}{\si{\coulomb}} \right]\\
        \tan^{-1}\left( \frac{4}{3} \right)=53^{\circ}\\
        2\cdot2160\cdot\cos(53)=260\left[ \frac{\si{\newton}}{\si{\coulomb}} \right]\\
        2600-2250=350\left[ \frac{\si{\newton}}{\si{\coulomb}} \right]\text{ up }
      \end{split}
      \label{11}
    \end{equation}

\begin{equation}
  \begin{split}
    V=k\frac{Q}{r}\\
    V_t=k\left( \frac{6\cdot10^{-6}}{5}+\frac{6\cdot10^{-6}}{5}-\frac{4\cdot10^{-6}}{4} \right)\\
    =12600[\si{\volt}]\\
  \end{split}
  \label{12}
\end{equation}


\hline

\newpage

    \begin{figure}[h]
      \centering
      \tikzset{every picture/.style={line width=0.75pt}} %set default line width to 0.75pt        

\begin{tikzpicture}[x=0.75pt,y=0.75pt,yscale=-1,xscale=1]
%uncomment if require: \path (0,300); %set diagram left start at 0, and has height of 300

%Shape: Circle [id:dp5520731613008669] 
\draw   (171,145) .. controls (171,131.19) and (182.19,120) .. (196,120) .. controls (209.81,120) and (221,131.19) .. (221,145) .. controls (221,158.81) and (209.81,170) .. (196,170) .. controls (182.19,170) and (171,158.81) .. (171,145) -- cycle ;
%Straight Lines [id:da12065599323998066] 
\draw    (196,170) -- (195.51,258) ;
\draw [shift={(195.5,260)}, rotate = 270.32] [color={rgb, 255:red, 0; green, 0; blue, 0 }  ][line width=0.75]    (10.93,-3.29) .. controls (6.95,-1.4) and (3.31,-0.3) .. (0,0) .. controls (3.31,0.3) and (6.95,1.4) .. (10.93,3.29)   ;
%Straight Lines [id:da8507937543436119] 
\draw    (221,145.25) -- (347.5,145) ;
\draw [shift={(349.5,145)}, rotate = 539.89] [color={rgb, 255:red, 0; green, 0; blue, 0 }  ][line width=0.75]    (10.93,-3.29) .. controls (6.95,-1.4) and (3.31,-0.3) .. (0,0) .. controls (3.31,0.3) and (6.95,1.4) .. (10.93,3.29)   ;
%Shape: Boxed Line [id:dp5987147186507646] 
\draw    (171,145) -- (83,144.51) ;
\draw [shift={(81,144.5)}, rotate = 360.32] [color={rgb, 255:red, 0; green, 0; blue, 0 }  ][line width=0.75]    (10.93,-3.29) .. controls (6.95,-1.4) and (3.31,-0.3) .. (0,0) .. controls (3.31,0.3) and (6.95,1.4) .. (10.93,3.29)   ;
%Shape: Boxed Line [id:dp2139323878030881] 
\draw    (196,120) -- (196.49,32) ;
\draw [shift={(196.5,30)}, rotate = 450.32] [color={rgb, 255:red, 0; green, 0; blue, 0 }  ][line width=0.75]    (10.93,-3.29) .. controls (6.95,-1.4) and (3.31,-0.3) .. (0,0) .. controls (3.31,0.3) and (6.95,1.4) .. (10.93,3.29)   ;
%Shape: Circle [id:dp33912625944533503] 
\draw   (441,145) .. controls (441,131.19) and (452.19,120) .. (466,120) .. controls (479.81,120) and (491,131.19) .. (491,145) .. controls (491,158.81) and (479.81,170) .. (466,170) .. controls (452.19,170) and (441,158.81) .. (441,145) -- cycle ;
%Straight Lines [id:da9178485185294811] 
\draw    (466,170) -- (466.49,293) ;
\draw [shift={(466.5,295)}, rotate = 269.77] [color={rgb, 255:red, 0; green, 0; blue, 0 }  ][line width=0.75]    (10.93,-3.29) .. controls (6.95,-1.4) and (3.31,-0.3) .. (0,0) .. controls (3.31,0.3) and (6.95,1.4) .. (10.93,3.29)   ;
%Shape: Boxed Line [id:dp2855235173942938] 
\draw    (466,120) -- (466.49,32) ;
\draw [shift={(466.5,30)}, rotate = 450.32] [color={rgb, 255:red, 0; green, 0; blue, 0 }  ][line width=0.75]    (10.93,-3.29) .. controls (6.95,-1.4) and (3.31,-0.3) .. (0,0) .. controls (3.31,0.3) and (6.95,1.4) .. (10.93,3.29)   ;

% Text Node
\draw (331,114.4) node [anchor=north west][inner sep=0.75pt]    {$F_{T}$};
% Text Node
\draw (167,225.4) node [anchor=north west][inner sep=0.75pt]    {$F_{g}$};
% Text Node
\draw (93,116.4) node [anchor=north west][inner sep=0.75pt]    {$F_{F}$};
% Text Node
\draw (170,32.4) node [anchor=north west][inner sep=0.75pt]    {$F_{N}$};
% Text Node
\draw (179,136.4) node [anchor=north west][inner sep=0.75pt]    {$10kg$};
% Text Node
\draw (437,225.4) node [anchor=north west][inner sep=0.75pt]    {$F_{g}$};
% Text Node
\draw (440,32.4) node [anchor=north west][inner sep=0.75pt]    {$F_{T}$};
% Text Node
\draw (452,135.4) node [anchor=north west][inner sep=0.75pt]    {$ \begin{array}{l}
5kg\\
\end{array}$};


\end{tikzpicture}

      \caption{Free Body Diagram for left particle}
      \label{fig:2}
    \end{figure}

    \begin{figure}[h]
      \centering
      \tikzset{every picture/.style={line width=0.75pt}} %set default line width to 0.75pt        

\begin{tikzpicture}[x=0.75pt,y=0.75pt,yscale=-1,xscale=1]
%uncomment if require: \path (0,300); %set diagram left start at 0, and has height of 300

%Shape: Circle [id:dp5520731613008669] 
\draw   (171,145) .. controls (171,131.19) and (182.19,120) .. (196,120) .. controls (209.81,120) and (221,131.19) .. (221,145) .. controls (221,158.81) and (209.81,170) .. (196,170) .. controls (182.19,170) and (171,158.81) .. (171,145) -- cycle ;
%Straight Lines [id:da12065599323998066] 
\draw    (196,170) -- (195.51,258) ;
\draw [shift={(195.5,260)}, rotate = 270.32] [color={rgb, 255:red, 0; green, 0; blue, 0 }  ][line width=0.75]    (10.93,-3.29) .. controls (6.95,-1.4) and (3.31,-0.3) .. (0,0) .. controls (3.31,0.3) and (6.95,1.4) .. (10.93,3.29)   ;
%Straight Lines [id:da8507937543436119] 
\draw    (221,145.25) -- (347.5,145) ;
\draw [shift={(349.5,145)}, rotate = 539.89] [color={rgb, 255:red, 0; green, 0; blue, 0 }  ][line width=0.75]    (10.93,-3.29) .. controls (6.95,-1.4) and (3.31,-0.3) .. (0,0) .. controls (3.31,0.3) and (6.95,1.4) .. (10.93,3.29)   ;
%Shape: Boxed Line [id:dp5987147186507646] 
\draw    (171,145) -- (83,144.51) ;
\draw [shift={(81,144.5)}, rotate = 360.32] [color={rgb, 255:red, 0; green, 0; blue, 0 }  ][line width=0.75]    (10.93,-3.29) .. controls (6.95,-1.4) and (3.31,-0.3) .. (0,0) .. controls (3.31,0.3) and (6.95,1.4) .. (10.93,3.29)   ;
%Shape: Boxed Line [id:dp2139323878030881] 
\draw    (196,120) -- (196.49,32) ;
\draw [shift={(196.5,30)}, rotate = 450.32] [color={rgb, 255:red, 0; green, 0; blue, 0 }  ][line width=0.75]    (10.93,-3.29) .. controls (6.95,-1.4) and (3.31,-0.3) .. (0,0) .. controls (3.31,0.3) and (6.95,1.4) .. (10.93,3.29)   ;
%Shape: Circle [id:dp33912625944533503] 
\draw   (441,145) .. controls (441,131.19) and (452.19,120) .. (466,120) .. controls (479.81,120) and (491,131.19) .. (491,145) .. controls (491,158.81) and (479.81,170) .. (466,170) .. controls (452.19,170) and (441,158.81) .. (441,145) -- cycle ;
%Straight Lines [id:da9178485185294811] 
\draw    (466,170) -- (466.49,293) ;
\draw [shift={(466.5,295)}, rotate = 269.77] [color={rgb, 255:red, 0; green, 0; blue, 0 }  ][line width=0.75]    (10.93,-3.29) .. controls (6.95,-1.4) and (3.31,-0.3) .. (0,0) .. controls (3.31,0.3) and (6.95,1.4) .. (10.93,3.29)   ;
%Shape: Boxed Line [id:dp2855235173942938] 
\draw    (466,120) -- (466.49,32) ;
\draw [shift={(466.5,30)}, rotate = 450.32] [color={rgb, 255:red, 0; green, 0; blue, 0 }  ][line width=0.75]    (10.93,-3.29) .. controls (6.95,-1.4) and (3.31,-0.3) .. (0,0) .. controls (3.31,0.3) and (6.95,1.4) .. (10.93,3.29)   ;

% Text Node
\draw (331,114.4) node [anchor=north west][inner sep=0.75pt]    {$F_{T}$};
% Text Node
\draw (167,225.4) node [anchor=north west][inner sep=0.75pt]    {$F_{g}$};
% Text Node
\draw (93,116.4) node [anchor=north west][inner sep=0.75pt]    {$F_{F}$};
% Text Node
\draw (170,32.4) node [anchor=north west][inner sep=0.75pt]    {$F_{N}$};
% Text Node
\draw (179,136.4) node [anchor=north west][inner sep=0.75pt]    {$10kg$};
% Text Node
\draw (437,225.4) node [anchor=north west][inner sep=0.75pt]    {$F_{a}$};
% Text Node
\draw (440,32.4) node [anchor=north west][inner sep=0.75pt]    {$F_{T}$};
% Text Node
\draw (452,135.4) node [anchor=north west][inner sep=0.75pt]    {$ \begin{array}{l}
pull\\
\end{array}$};


\end{tikzpicture}

      \caption{Free Body Diagram for center particle}
      \label{fig:3}
    \end{figure}

    \begin{figure}[h]
      \centering
      \tikzset{every picture/.style={line width=0.75pt}} %set default line width to 0.75pt        

\begin{tikzpicture}[x=0.75pt,y=0.75pt,yscale=-1,xscale=1]
%uncomment if require: \path (0,300); %set diagram left start at 0, and has height of 300

%Shape: Circle [id:dp14447258656558515] 
\draw  [color={rgb, 255:red, 0; green, 0; blue, 0 }  ,draw opacity=1 ][fill={rgb, 255:red, 0; green, 0; blue, 0 }  ,fill opacity=1 ] (109.65,155.72) .. controls (109.6,153.11) and (111.68,151) .. (114.29,151) .. controls (116.9,151) and (119.05,153.11) .. (119.09,155.72) .. controls (119.14,158.33) and (117.06,160.45) .. (114.45,160.45) .. controls (111.84,160.45) and (109.69,158.33) .. (109.65,155.72) -- cycle ;
%Shape: Boxed Line [id:dp7138136373100961] 
\draw    (109.65,155.72) -- (63.24,155.72) ;
\draw [shift={(61.24,155.72)}, rotate = 360] [color={rgb, 255:red, 0; green, 0; blue, 0 }  ][line width=0.75]    (10.93,-3.29) .. controls (6.95,-1.4) and (3.31,-0.3) .. (0,0) .. controls (3.31,0.3) and (6.95,1.4) .. (10.93,3.29)   ;
%Shape: Boxed Line [id:dp5235226248724378] 
\draw    (119.09,155.72) -- (149.09,155.72) ;
\draw [shift={(151.09,155.72)}, rotate = 180] [color={rgb, 255:red, 0; green, 0; blue, 0 }  ][line width=0.75]    (10.93,-3.29) .. controls (6.95,-1.4) and (3.31,-0.3) .. (0,0) .. controls (3.31,0.3) and (6.95,1.4) .. (10.93,3.29)   ;

% Text Node
\draw (76,138.92) node [anchor=west] [inner sep=0.75pt]    {${F_{e}}_{_{-4}}$};
% Text Node
\draw (146.88,138.92) node [anchor=east] [inner sep=0.75pt]    {${F_{e}}_{_{+6}}$};


\end{tikzpicture}

      \caption{Free Body Diagram for right particle}
      \label{fig:4}
    \end{figure}

    \begin{figure}[h]
      \centering
      \tikzset{every picture/.style={line width=0.75pt}} %set default line width to 0.75pt        

\begin{tikzpicture}[x=0.75pt,y=0.75pt,yscale=-1,xscale=1]
%uncomment if require: \path (0,300); %set diagram left start at 0, and has height of 300

%Shape: Circle [id:dp8570105127026377] 
\draw  [color={rgb, 255:red, 0; green, 0; blue, 0 }  ,draw opacity=1 ][fill={rgb, 255:red, 0; green, 0; blue, 0 }  ,fill opacity=1 ] (322.65,81.72) .. controls (322.6,79.11) and (324.68,77) .. (327.29,77) .. controls (329.9,77) and (332.05,79.11) .. (332.09,81.72) .. controls (332.14,84.33) and (330.06,86.45) .. (327.45,86.45) .. controls (324.84,86.45) and (322.69,84.33) .. (322.65,81.72) -- cycle ;
%Shape: Circle [id:dp4245007933940925] 
\draw  [color={rgb, 255:red, 0; green, 0; blue, 0 }  ,draw opacity=1 ][fill={rgb, 255:red, 0; green, 0; blue, 0 }  ,fill opacity=1 ] (322.65,168.28) .. controls (322.6,165.67) and (324.68,163.55) .. (327.29,163.55) .. controls (329.9,163.55) and (332.05,165.67) .. (332.09,168.28) .. controls (332.14,170.89) and (330.06,173) .. (327.45,173) .. controls (324.84,173) and (322.69,170.89) .. (322.65,168.28) -- cycle ;
%Shape: Circle [id:dp10932705047210622] 
\draw  [color={rgb, 255:red, 0; green, 0; blue, 0 }  ,draw opacity=1 ][fill={rgb, 255:red, 0; green, 0; blue, 0 }  ,fill opacity=1 ] (413.92,168.28) .. controls (413.88,165.67) and (415.95,163.55) .. (418.56,163.55) .. controls (421.17,163.55) and (423.32,165.67) .. (423.37,168.28) .. controls (423.42,170.89) and (421.34,173) .. (418.73,173) .. controls (416.12,173) and (413.97,170.89) .. (413.92,168.28) -- cycle ;
%Shape: Circle [id:dp13666466165408409] 
\draw  [color={rgb, 255:red, 0; green, 0; blue, 0 }  ,draw opacity=1 ][fill={rgb, 255:red, 0; green, 0; blue, 0 }  ,fill opacity=1 ] (236.09,168.28) .. controls (236.14,170.89) and (234.06,173) .. (231.45,173) .. controls (228.84,173) and (226.69,170.89) .. (226.65,168.28) .. controls (226.6,165.67) and (228.68,163.55) .. (231.29,163.55) .. controls (233.9,163.55) and (236.05,165.67) .. (236.09,168.28) -- cycle ;
%Shape: Boxed Line [id:dp4412401484637065] 
\draw    (327.45,86.45) -- (327.29,161.55) ;
\draw [shift={(327.29,163.55)}, rotate = 270.12] [color={rgb, 255:red, 0; green, 0; blue, 0 }  ][line width=0.75]    (10.93,-3.29) .. controls (6.95,-1.4) and (3.31,-0.3) .. (0,0) .. controls (3.31,0.3) and (6.95,1.4) .. (10.93,3.29)   ;
%Shape: Boxed Line [id:dp899763866751071] 
\draw    (327.37,81.72) -- (274.38,28.5) ;
\draw [shift={(272.97,27.09)}, rotate = 405.12] [color={rgb, 255:red, 0; green, 0; blue, 0 }  ][line width=0.75]    (10.93,-3.29) .. controls (6.95,-1.4) and (3.31,-0.3) .. (0,0) .. controls (3.31,0.3) and (6.95,1.4) .. (10.93,3.29)   ;
%Shape: Boxed Line [id:dp6686588216420633] 
\draw    (327.37,81.72) -- (380.59,28.73) ;
\draw [shift={(382.01,27.32)}, rotate = 495.12] [color={rgb, 255:red, 0; green, 0; blue, 0 }  ][line width=0.75]    (10.93,-3.29) .. controls (6.95,-1.4) and (3.31,-0.3) .. (0,0) .. controls (3.31,0.3) and (6.95,1.4) .. (10.93,3.29)   ;

% Text Node
\draw (270.97,30.49) node [anchor=north east] [inner sep=0.75pt]    {$F_{6}$};
% Text Node
\draw (384.01,30.72) node [anchor=north west][inner sep=0.75pt]    {$F_{6}$};
% Text Node
\draw (329.37,85.12) node [anchor=north west][inner sep=0.75pt]    {${F_{4}}_{-}$};


\end{tikzpicture}

      \caption{Electric Field Vectors at point P}
      \label{fig:5}
    \end{figure}

    \begin{figure}[h]
      \centering
      \tikzset{every picture/.style={line width=0.75pt}} %set default line width to 0.75pt        

\begin{tikzpicture}[x=0.75pt,y=0.75pt,yscale=-1,xscale=1]
%uncomment if require: \path (0,300); %set diagram left start at 0, and has height of 300

%Shape: Circle [id:dp4245007933940925] 
\draw  [color={rgb, 255:red, 0; green, 0; blue, 0 }  ,draw opacity=1 ][fill={rgb, 255:red, 0; green, 0; blue, 0 }  ,fill opacity=1 ] (322.65,168.28) .. controls (322.6,165.67) and (324.68,163.55) .. (327.29,163.55) .. controls (329.9,163.55) and (332.05,165.67) .. (332.09,168.28) .. controls (332.14,170.89) and (330.06,173) .. (327.45,173) .. controls (324.84,173) and (322.69,170.89) .. (322.65,168.28) -- cycle ;
%Shape: Circle [id:dp23357622940875022] 
\draw  [color={rgb, 255:red, 0; green, 0; blue, 0 }  ,draw opacity=1 ][fill={rgb, 255:red, 0; green, 0; blue, 0 }  ,fill opacity=1 ] (236.09,168.28) .. controls (236.14,170.89) and (234.06,173) .. (231.45,173) .. controls (228.84,173) and (226.69,170.89) .. (226.65,168.28) .. controls (226.6,165.67) and (228.68,163.55) .. (231.29,163.55) .. controls (233.9,163.55) and (236.05,165.67) .. (236.09,168.28) -- cycle ;
%Curve Lines [id:da13110763982887885] 
\draw    (231.29,163.55) .. controls (270.89,133.85) and (292.07,134.97) .. (326.25,162.7) ;
\draw [shift={(327.29,163.55)}, rotate = 219.38] [color={rgb, 255:red, 0; green, 0; blue, 0 }  ][line width=0.75]    (10.93,-3.29) .. controls (6.95,-1.4) and (3.31,-0.3) .. (0,0) .. controls (3.31,0.3) and (6.95,1.4) .. (10.93,3.29)   ;
%Shape: Boxed Bezier Curve [id:dp9881118418107562] 
\draw    (231.45,173) .. controls (271.56,202) and (292.73,200.51) .. (326.41,172.19) ;
\draw [shift={(327.44,171.32)}, rotate = 499.62] [color={rgb, 255:red, 0; green, 0; blue, 0 }  ][line width=0.75]    (10.93,-3.29) .. controls (6.95,-1.4) and (3.31,-0.3) .. (0,0) .. controls (3.31,0.3) and (6.95,1.4) .. (10.93,3.29)   ;

%Shape: Circle [id:dp026822721702264518] 
\draw  [color={rgb, 255:red, 0; green, 0; blue, 0 }  ,draw opacity=1 ][fill={rgb, 255:red, 0; green, 0; blue, 0 }  ,fill opacity=1 ] (418.65,168.28) .. controls (418.6,165.67) and (420.68,163.55) .. (423.29,163.55) .. controls (425.9,163.55) and (428.05,165.67) .. (428.09,168.28) .. controls (428.14,170.89) and (426.06,173) .. (423.45,173) .. controls (420.84,173) and (418.69,170.89) .. (418.65,168.28) -- cycle ;
%Curve Lines [id:da08525872907796428] 
\draw    (423.45,173) .. controls (383.85,202.7) and (362.67,201.58) .. (328.49,173.85) ;
\draw [shift={(327.45,173)}, rotate = 399.38] [color={rgb, 255:red, 0; green, 0; blue, 0 }  ][line width=0.75]    (10.93,-3.29) .. controls (6.95,-1.4) and (3.31,-0.3) .. (0,0) .. controls (3.31,0.3) and (6.95,1.4) .. (10.93,3.29)   ;
%Curve Lines [id:da8376230050988751] 
\draw    (423.29,163.55) .. controls (383.18,134.55) and (362,134.4) .. (328.31,162.69) ;
\draw [shift={(327.29,163.55)}, rotate = 319.62] [color={rgb, 255:red, 0; green, 0; blue, 0 }  ][line width=0.75]    (10.93,-3.29) .. controls (6.95,-1.4) and (3.31,-0.3) .. (0,0) .. controls (3.31,0.3) and (6.95,1.4) .. (10.93,3.29)   ;
%Straight Lines [id:da2302276901349971] 
\draw    (231.45,173) -- (231.5,227) ;
\draw [shift={(231.5,229)}, rotate = 269.95] [color={rgb, 255:red, 0; green, 0; blue, 0 }  ][line width=0.75]    (10.93,-3.29) .. controls (6.95,-1.4) and (3.31,-0.3) .. (0,0) .. controls (3.31,0.3) and (6.95,1.4) .. (10.93,3.29)   ;
%Shape: Boxed Line [id:dp13723588551634136] 
\draw    (231.29,163.55) -- (231.24,109.55) ;
\draw [shift={(231.24,107.55)}, rotate = 449.95] [color={rgb, 255:red, 0; green, 0; blue, 0 }  ][line width=0.75]    (10.93,-3.29) .. controls (6.95,-1.4) and (3.31,-0.3) .. (0,0) .. controls (3.31,0.3) and (6.95,1.4) .. (10.93,3.29)   ;
%Shape: Boxed Bezier Curve [id:dp8510600313830454] 
\draw    (226.65,168.28) .. controls (205.82,171.94) and (207.45,146.77) .. (206.54,112.57) ;
\draw [shift={(206.5,111)}, rotate = 448.36] [color={rgb, 255:red, 0; green, 0; blue, 0 }  ][line width=0.75]    (10.93,-3.29) .. controls (6.95,-1.4) and (3.31,-0.3) .. (0,0) .. controls (3.31,0.3) and (6.95,1.4) .. (10.93,3.29)   ;
%Shape: Boxed Bezier Curve [id:dp08414170526895282] 
\draw    (428.09,168.28) .. controls (448.92,164.61) and (447.29,189.78) .. (448.2,223.98) ;
\draw [shift={(448.24,225.55)}, rotate = 268.36] [color={rgb, 255:red, 0; green, 0; blue, 0 }  ][line width=0.75]    (10.93,-3.29) .. controls (6.95,-1.4) and (3.31,-0.3) .. (0,0) .. controls (3.31,0.3) and (6.95,1.4) .. (10.93,3.29)   ;
%Shape: Boxed Bezier Curve [id:dp21448379872299972] 
\draw    (428.09,168.28) .. controls (448.92,171.94) and (447.29,146.77) .. (448.2,112.57) ;
\draw [shift={(448.24,111)}, rotate = 451.64] [color={rgb, 255:red, 0; green, 0; blue, 0 }  ][line width=0.75]    (10.93,-3.29) .. controls (6.95,-1.4) and (3.31,-0.3) .. (0,0) .. controls (3.31,0.3) and (6.95,1.4) .. (10.93,3.29)   ;
%Shape: Boxed Bezier Curve [id:dp36741794832289676] 
\draw    (226.65,168.28) .. controls (205.82,164.61) and (207.45,189.78) .. (206.54,223.98) ;
\draw [shift={(206.5,225.55)}, rotate = 271.64] [color={rgb, 255:red, 0; green, 0; blue, 0 }  ][line width=0.75]    (10.93,-3.29) .. controls (6.95,-1.4) and (3.31,-0.3) .. (0,0) .. controls (3.31,0.3) and (6.95,1.4) .. (10.93,3.29)   ;
%Straight Lines [id:da3302109203500634] 
\draw    (423.45,173) -- (423.5,227) ;
\draw [shift={(423.5,229)}, rotate = 269.95] [color={rgb, 255:red, 0; green, 0; blue, 0 }  ][line width=0.75]    (10.93,-3.29) .. controls (6.95,-1.4) and (3.31,-0.3) .. (0,0) .. controls (3.31,0.3) and (6.95,1.4) .. (10.93,3.29)   ;
%Shape: Boxed Line [id:dp29429298981795093] 
\draw    (423.29,163.55) -- (423.24,109.55) ;
\draw [shift={(423.24,107.55)}, rotate = 449.95] [color={rgb, 255:red, 0; green, 0; blue, 0 }  ][line width=0.75]    (10.93,-3.29) .. controls (6.95,-1.4) and (3.31,-0.3) .. (0,0) .. controls (3.31,0.3) and (6.95,1.4) .. (10.93,3.29)   ;




\end{tikzpicture}

      \caption{Electric Field Drawing}
      \label{fig:6}
    \end{figure}
    
    \newpage

\begin{center}
  Problem Six:
\end{center}
\hline

\begin{equation}
  \begin{split}
    \text{Positive (lines are moving out)}
  \end{split}
  \label{13}
\end{equation}

\begin{equation}
  \begin{split}
    0[\si{\joule}]\\
  \end{split}
  \label{14}
\end{equation}

\begin{equation}
  \begin{split}
    10[\si{\joule}]\\
  \end{split}
  \label{15}
\end{equation}

\hline
\newpage

\begin{center}
  Problem Seven:
\end{center}
\hline

\begin{enumerate}

  \item This means the current needs to be doubled. Initially, the capacitor may be neglected, so the identical resistor needs to be placed in parallel with the first one, which would make the total resistance, half, and, assuming voltage is kept constant, the current would be doubled.

  \item This would mean that the voltage would need to be doubled. This would be done by adding another battery in series, while placing a resistor in series. By doubling the resistance, while also doubling the voltage, the current remains the same.

  \item This would mean that current and voltage would need to be doubled. This may be done by simply placing an additional battery in series. By doubling the voltage, and keeping resistance the same, current would be doubled as well.

\end{enumerate}

\hline
\newpage

\begin{center}
  Problem Eight:
\end{center}
\hline

\begin{center}
  Circuit on Next Page
\end{center}

\begin{figure}[H]
  \centering
  \tikzset{every picture/.style={line width=0.75pt}} %set default line width to 0.75pt        

\begin{tikzpicture}[x=0.75pt,y=0.75pt,yscale=-1,xscale=1]
%uncomment if require: \path (0,300); %set diagram left start at 0, and has height of 300

%Shape: Resistor [id:dp5682722001893705] 
\draw   (315.75,127.35) -- (315.75,135.23) -- (324.5,136.98) -- (307,140.49) -- (324.5,143.99) -- (307,147.5) -- (324.5,151) -- (307,154.51) -- (324.5,158.01) -- (307,161.52) -- (315.75,163.27) -- (315.75,171.15) ;
%Shape: Boxed Line [id:dp9667652850228652] 
\draw    (315.75,201) -- (315.75,171.15) ;
%Shape: Boxed Line [id:dp22250144134579086] 
\draw    (315.75,127.35) -- (315.75,97.5) ;
%Shape: Boxed Line [id:dp7983416385026896] 
\draw    (315.75,97.5) -- (174.33,97.5) ;
%Shape: Boxed Line [id:dp31879664880159053] 
\draw    (315.75,201) -- (174.33,201) ;
%Straight Lines [id:da8324746513831833] 
\draw    (345.6,113.38) -- (315.75,97.5) ;
%Shape: Boxed Line [id:dp04261551663547003] 
\draw    (345.6,201) -- (315.75,201) ;
%Shape: Resistor [id:dp23425640951011162] 
\draw   (345.6,157.19) -- (345.6,165.08) -- (354.35,166.83) -- (336.85,170.33) -- (354.35,173.84) -- (336.85,177.34) -- (354.35,180.85) -- (336.85,184.35) -- (354.35,187.86) -- (336.85,191.36) -- (345.6,193.11) -- (345.6,201) ;
%Shape: Resistor [id:dp5127526202059483] 
\draw   (345.6,113.38) -- (345.6,121.27) -- (354.35,123.02) -- (336.85,126.52) -- (354.35,130.03) -- (336.85,133.53) -- (354.35,137.04) -- (336.85,140.54) -- (354.35,144.05) -- (336.85,147.55) -- (345.6,149.3) -- (345.6,157.19) ;
%Shape: Boxed Line [id:dp7762207265614354] 
\draw    (174.33,127.35) -- (174.33,97.5) ;
%Shape: Battery [id:dp7566407651907399] 
\draw   (174.33,127.35) -- (174.33,160.49) (204.33,167.86) -- (144.33,167.86) (174.33,167.86) -- (174.33,201) (189.33,157.54) -- (189.33,160.49) -- (159.33,160.49) -- (159.33,157.54) -- (189.33,157.54) -- cycle ;

% Text Node
\draw (248,140) node [anchor=north west][inner sep=0.75pt]   [align=left] {Toaster};
% Text Node
\draw (356.35,126.02) node [anchor=north west][inner sep=0.75pt]   [align=left] {Resistor};
% Text Node
\draw (356.35,169.83) node [anchor=north west][inner sep=0.75pt]   [align=left] {Bulb};


\end{tikzpicture}

  \caption{Circuit For Problem 8}
  \label{fig:7}
\end{figure}

\begin{equation}
  \begin{split}
    \frac{108}{6}=18[\si{\ohm}]\\
  \end{split}
  \label{16}
\end{equation}

\begin{equation}
  \begin{split}
    18\cdot120=2160[\si{\watt}]\\
  \end{split}
  \label{17}
\end{equation}

\hline

\end{document}

