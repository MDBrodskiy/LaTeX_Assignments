%%%%%%%%%%%%%%%%%%%%%%%%%%%%%%%%%%%%%%%%%%%%%%%%%%%%%%%%%%%%%%%%%%%%%%%%%%%%%%%%%%%%%%%%%%%%%%%%%%%%%%%%%%%%%%%%%%%%%%%%%%%%%%%%%%%%%%%%%%%%%%%%%%%%%%%%%%%%%%%%%%%%%%%%%%%%%%%%%%%%%%%%%%%%
% Written By Michael Brodskiy
% Class: AP Physics 2
% Professor: S. Morse
%%%%%%%%%%%%%%%%%%%%%%%%%%%%%%%%%%%%%%%%%%%%%%%%%%%%%%%%%%%%%%%%%%%%%%%%%%%%%%%%%%%%%%%%%%%%%%%%%%%%%%%%%%%%%%%%%%%%%%%%%%%%%%%%%%%%%%%%%%%%%%%%%%%%%%%%%%%%%%%%%%%%%%%%%%%%%%%%%%%%%%%%%%%%

\documentclass[12pt]{article} 
\usepackage{alphalph}
\usepackage[utf8]{inputenc}
\usepackage[russian,english]{babel}
\usepackage{titling}
\usepackage{amsmath}
\usepackage{graphicx}
\usepackage{enumitem}
\usepackage{amssymb}
\usepackage[super]{nth}
\usepackage{everysel}
\usepackage{ragged2e}
\usepackage{geometry}
\usepackage{fancyhdr}
\usepackage{cancel}
\usepackage{siunitx}
\usepackage{expl3}
\usepackage[version=4]{mhchem}
\usepackage{hpstatement}
\usepackage{rsphrase}
\geometry{top=1.0in,bottom=1.0in,left=1.0in,right=1.0in}
\newcommand{\subtitle}[1]{%
  \posttitle{%
    \par\end{center}
    \begin{center}\large#1\end{center}
    \vskip0.5em}%

}
\usepackage{hyperref}
\hypersetup{
colorlinks=true,
linkcolor=blue,
filecolor=magenta,      
urlcolor=blue,
citecolor=blue,
}

\urlstyle{same}


\title{Electrostatics FRQ 2}
\date{November 3, 2020}
\author{Michael Brodskiy\\ \small Instructor: Mrs. Morse}

% Mathematical Operations:

% Sum: $$\sum_{n=a}^{b} f(x) $$
% Integral: $$\int_{lower}^{upper} f(x) dx$$
% Limit: $$\lim_{x\to\infty} f(x)$$

\begin{document}

\maketitle

\begin{enumerate}

    \setcounter{enumi}{2}

  \item Where $\ce{p^8+}$ is the +8 charge and $\ce{e^6-}$ is the -6 charge and \ce{p^8_2} is the second +8 charge \eqref{2}

    \begin{equation}
      \begin{split}
        E_{\ce{p^8+}}&=k\frac{8\cdot10^{-9}}{10^2}\\
      &=.72\left[ \frac{\si{\newton}}{\si{\coulomb}} \right]\\
        E_{\ce{e^6-}}&=k\frac{6\cdot10^{-9}}{3^2}\\
        &=6\left[ \frac{\si{\newton}}{\si{\coulomb}} \right]\\
        E_{\ce{p^8_2}}&=k\frac{8\cdot10^{-9}}{3^2}\\
        &=8\left[ \frac{\si{\newton}}{\si{\coulomb}} \right]\\
        E_{total}&=8+6-.72\\
        &=13.28\left[ \frac{\si{\newton}}{\si{\coulomb}} \right]\text{ left}
    \end{split}
      \label{2}
    \end{equation}

\end{enumerate}

\end{document}

