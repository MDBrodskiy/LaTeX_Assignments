%%%%%%%%%%%%%%%%%%%%%%%%%%%%%%%%%%%%%%%%%%%%%%%%%%%%%%%%%%%%%%%%%%%%%%%%%%%%%%%%%%%%%%%%%%%%%%%%%%%%%%%%%%%%%%%%%%%%%%%%%%%%%%%%%%%%%%%%%%%%%%%%%%%%%%%%%%%%%%%%%%%%%%%%%%%%%%%%%%%%%%%%%%%%
% Written By Michael Brodskiy
% Class: Spanish 4 Honors
% Professor: S. Lansman
%%%%%%%%%%%%%%%%%%%%%%%%%%%%%%%%%%%%%%%%%%%%%%%%%%%%%%%%%%%%%%%%%%%%%%%%%%%%%%%%%%%%%%%%%%%%%%%%%%%%%%%%%%%%%%%%%%%%%%%%%%%%%%%%%%%%%%%%%%%%%%%%%%%%%%%%%%%%%%%%%%%%%%%%%%%%%%%%%%%%%%%%%%%%

\documentclass[12pt]{article} 
\usepackage{alphalph}
\usepackage[utf8]{inputenc}
\usepackage[russian,english]{babel}
\usepackage{titling}
\usepackage{amsmath}
\usepackage{graphicx}
\usepackage{enumitem}
\usepackage{amssymb}
\usepackage[super]{nth}
\usepackage{everysel}
\usepackage{ragged2e}
\usepackage{geometry}
\usepackage{fancyhdr}
\usepackage{cancel}
\usepackage{siunitx}
\geometry{top=1.0in,bottom=1.0in,left=1.0in,right=1.0in}
\newcommand{\subtitle}[1]{%
  \posttitle{%
    \par\end{center}
    \begin{center}\large#1\end{center}
    \vskip0.5em}%

}
\usepackage{hyperref}
\hypersetup{
colorlinks=true,
linkcolor=blue,
filecolor=magenta,      
urlcolor=blue,
citecolor=blue,
}

\urlstyle{same}


\title{Crear Definiciones}
\date{El 30 del Septiembre, 2020}
\author{Michael Brodskiy\\ \small Instructor: Mrs. Lansman}

% Mathematical Operations:

% Sum: $$\sum_{n=a}^{b} f(x) $$
% Integral: $$\int_{lower}^{upper} f(x) dx$$
% Limit: $$\lim_{x\to\infty} f(x)$$

\begin{document}

\maketitle

\begin{enumerate}

  \item Poblar $-$ La gente que vive en un lugar puebla ese lugar. Las personas que pueblan un ciudad son la poblaci\'on.

  \item Recorrer $-$ Es cuando alguien visita diferentes lugares.

  \item Dar un paseo/dar una vuelta $-$ Es cuando alguien camina, generalmente afuera.

  \item El letrero $-$ Un cartel que da informaci�n.

  \item La cuadra $-$ Como el Times Square en Nueva York, es un regi�n, generalmente cerca de una parte ruidosa y ocupada de una ciudad. 

  \item La acera $-$ Un camino, al lado de la carretera, hecho para la gente para caminar.

  \item Indicar el camino $-$ Para dar direcciones.

  \item Preguntar el camino $-$ Para pedir y recibir direcciones.

\end{enumerate}

\end{document}

