\documentclass[12pt,letterpaper]{article}

\usepackage[utf8]{inputenc}
\usepackage[english]{babel}
\usepackage{ragged2e}
\usepackage{schemata}
\usepackage{ifpdf}
\usepackage{soul}
\usepackage{mla}

\definecolor{pink}{rgb}{1.0, 0.33, 0.64}
\newcommand{\hlpink}[1]{{\sethlcolor{pink}\hl{#1}}}

\begin{document}

\begin{mla}{Michael}{Brodskiy}{AP Euro}{Period 1}{\today}{Industrial Revolution}

\flushleft{\textcircled{\raisebox{-0.9pt}{1}}} ~~~~~ 
\justifying{The Industrial Revolution, estimated to have begun around 1780, was a time of rapid technological expansion; the British cities that were most effected by include London, Liverpool, and most importantly, Manchester. The industrialization in Manchester spawned many issues, which had very different reactions from politicians and common folk. \hlpink{The issues include pollution and overpopulation, which incited responses from political} \emph{illegible}\hlpink{, which ranged from rapid expansion of cities to lawful reform; citizens reacted by having mass protests.}}

\justifying{\schema[close]{Problems, caused by the burning of coal and overall industrialization, are recorded from many sources. For example, Edwin Chadwick, an important health reformer from Britain, wrote about the pollution in his ``Report on Sanitary Conditions.'' Chadwick states, ``Diseases caused or aggravated by atmospheric influences . . . prevail among the laboring classes.'' This excerpt, written by a reputable health worker, exemplifies the extent to which nature was harmed during industrialization. As this is a report, it can be reasonably assumed that this was a document intended for the public.}{Doc 6} \schema[close]{Furthermore, Flora Tristan, a French worker's rights advocate, states, ``. . . with every breath of foul air they [the workers] absorb fibers of cotton, wool of flax, or particles of copper, \emph{illegible} or iron.'' Such a quote was published in Tristan's journal. Tristan later states that the workers are kept in disgusting rooms for 12--14 hours a day, and still do not earn enough for even potatoes, the cheapest crop at the time.}{Doc 7}}

\flushleft{\textcircled{\raisebox{-0.9pt}{2}}} ~~~~~
\justifying{Most importantly, ``The Lancet,'' a British medical journal edited by Thomas Wakley, a medical reformer shows the statistics of life. The Lancet shows that life expectancy was significantly greater in all classes, in rural areas, as opposed to areas like Manchester.}

\end{mla}

\end{document}
