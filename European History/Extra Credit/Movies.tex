\documentclass[12pt]{article} 

\usepackage{cite}
\usepackage{titling}
\usepackage{graphicx}

\newcommand{\subtitle}[1]{%
  \posttitle{%
    \par\end{center}
    \begin{center}\large#1\end{center}
    \vskip0.5em}%
}

%\pagenumbering{Roman}

\begin{document}

%------------------------------------------------------------------------------------------
% Title - Midway
%------------------------------------------------------------------------------------------


\author{Michael Brodskiy}
\title{Midway (2019)\\European History AP}
\subtitle{Mrs Fisher}
\maketitle

%------------------------------------------------------------------------------------------
% Midway
%------------------------------------------------------------------------------------------

\begin{enumerate}

\item The movie \textit{Midway}, which was released in 2019, can be dissected into roughly three parts. First, there is the bombing of Pearl Harbor, which is regarded as one of the greatest American intelligence failures. The second part follows the American intelligence, which attempts to find where the Japanese will strike next. The next attack point of the Japanese fleet is found to be the two-island atoll of Midway. The final part of the movie is the battle itself.

\item The movie did seem pretty historically accurate. It was well done in that although it was made so that the viewer would root for the American side, the Japanese were not dehumanized. I thought it was good that they did not stray away from what today would be considered racial slurs. By not glossing this over, it made it seem more authentic. I believe the director should have focused more on how quickly the fate of the battle was decided. Overall, I would say it was definitely well done.

\item There were a few things about the movie I want to mention. The movie was glorified at some points, but that is to be expected. It did, however, make it feel like a lot of other war movies. Although I did like it, I do not think that it is really a movie that will be remembered for long. Also, some of the accents seemed a bit strange, and I really did not like that. Overall, it was a pretty good movie.

\end{enumerate}

As the band Sabaton states in their song, \textit{Midway}, 

\begin{center} 

\textit{``Send them over the waves our sentinels\\
They report in the news\\
Position of our foes\\}

\end{center}

\begin{center}

\textit{This battlefield's been chosen, tactically in advance\\
Time to alert our fighters\\
We're soon in range\\}

\end{center}


\begin{center}

\textit{\textbf{MIDWAY!}}

\end{center}

\begin{center}

\textit{We'll meet at MIDWAY! Naval war\\}

\end{center}

\begin{center}

\textit{Calling all men to deck, got to be airborne\\
Head out into the sun\\
Descending on our foes\\
This is a crucial moment, in the heat of the war\\
To fly and hit our targets\\
Down in the waves\\}

\end{center}

\begin{center}

\textit{\textbf{MIDWAY!}} 


\end{center}

\begin{center}

\textit{(Display their might, ordering carriers, admirals at war)\\
We'll meet at midway (To win the fight, tactics are crucial) Naval war!\\
Far from shore, a Pacific war\\
Bombs are falling from the skies\\
It's a bomb-run day, it's the naval way\\
A blood-red sun is on the rise.''\\}


\includegraphics[width=0.35\columnwidth]{qr-code.png}\\
Scan the Code to Listen to the Song
\end{center}


\newpage

%----------------------------------------------------------------------------------------
% Title - Les Miserables
%----------------------------------------------------------------------------------------

\author{Michael Brodskiy}
\title{Les Miserables (2012)\\European History AP}
\subtitle{Mrs Fisher}
\maketitle

%----------------------------------------------------------------------------------------
% Les Miserables
%----------------------------------------------------------------------------------------
\begin{enumerate}

\item The main premise of the movie is that it is done completely in song. It takes place roughly 25 years after the French Revolution. The movie follows the story of Jean Valjean. Valjean spent 19 years in prison after stealing bread (and trying to escape) for his poor family. A bishop helps him become better. Valjean, once released, traveled to help other poor families. He essentially becomes a good samaritan after escaping from prison. Valjean is recognized by a prison guard, Javert, and is pursued. This movie follows the French June rebellion, which occurred in 1832, and lasted two days.

\item The movie seems pretty historically accurate. I really like the costume and set design. The singing made the movie seem kind of weird at parts, though. One big blunder are the characters English accents. I thought that it made the movie really strange when a lot of the characters had English accents.

\item I did not really like this movie. There was not much wrong with it. I just was not really interested in the movie. Especially that it was pretty much completely a musical. Otherwise, it was alright.

\end{enumerate}
\begin{center}
\textit{I permit my child to watch the aforementioned movies\\ X\_\_\_\_\_\_\_\_\_\_\_\_\_\_\_\_\_\_\_\_\_\_\_\_\_\_\_\_}
\end{center}
\end{document}