
\documentclass[12pt]{article} 
\usepackage{alphalph}
\usepackage[utf8]{inputenc}
\usepackage[russian,english]{babel}
\usepackage{titling}
\usepackage{amsmath}
\usepackage{graphicx}
\usepackage{enumitem}
\usepackage{amssymb}
\usepackage[super]{nth}
\usepackage{everysel}
\usepackage{ragged2e}
\usepackage{geometry}
\geometry{top=1.0in,bottom=1.0in,left=1.0in,right=1.0in}
\newcommand{\subtitle}[1]{%
  \posttitle{%
    \par\end{center}
    \begin{center}\large#1\end{center}
    \vskip0.5em}%

}
\usepackage{hyperref}
\hypersetup{
colorlinks=true,
linkcolor=blue,
filecolor=magenta,      
urlcolor=blue,
citecolor=blue,
}

\urlstyle{same}
\renewcommand*\familydefault{\ttdefault}
\EverySelectfont{%
\fontdimen2\font=0.4em% interword space
\fontdimen3\font=0.2em% interword stretch
\fontdimen4\font=0.1em% interword shrink
\fontdimen7\font=0.1em% extra space
\hyphenchar\font=`\-% to allow hyphenation
}

\begin{document}

%------------------------------------------------------------------------------------------
% Title
%------------------------------------------------------------------------------------------

\author{Michael \textsc{Brodskiy}}
\title{Final Review Packet\\European History AP}
\subtitle{Mrs Fisher}
\date{April 28, 2020}
\maketitle
\begin{center}
    \includegraphics[width=\textwidth]{Europa.png}
\end{center}
\newpage
\tableofcontents
\newpage

%------------------------------------------------------------------------------------------
% Questions
%------------------------------------------------------------------------------------------



\begin{center}
\section{\underline{Renaissance}}
\end{center}

\begin{enumerate}[label=]
\subsection{Causes}
\item
\end{enumerate}
\vspace{-35pt}
\begin{enumerate}

    \item Philosophical/Religious $-$ During the Renaissance, the term \textit{secularism} came about. This refers to something that does not relate to religion, something down-to-earth. Many artists began to paint more secular pieces, which focused on individual traits, and many were based off of classical Greek and Roman works of art. Also, many philosophers revived classical Greek and Roman thinking, and, as such, more philosophes came about.

\item Political (city states) $-$ The Italian city states did not wage war against each other for quite a bit. This created an accumulation of wealth that permitted the cities to to begin the period known as the Renaisance.

\item Economic $-$ The Renaissance began because of accumulation of wealth in Italian city states. Many Italian cities were based off of merchants and trading, and this allowed great amounts of wealth to pour in.

\item Social $-$ People became a bit more down to earth because of the new Renaissance ideals, such as: Humanism, Individualism, and Secularism.

\subsection{Terms}

\item Humanism $-$ The call back to classical Greek and Roman antiquity. This included art, architecture, and philosophy.

\item Individualism $-$ The focus on the individual as opposed to god. This stressed the importance of self value and education.

\item Secularism $-$ Down-to-earth, or not relating to religious beliefs or a god.

\subsection{People}

\item Machiavelli $-$ The author of \textit{The Prince}. He wrote this book for Cesare Borgia to demonstrate what a true Prince should act like. One of the major questions in the book is: "Which is better, to be feared or to be loved." This book offers a perspective on the royal life during the Renaissance.

\item Christine de Pisan $-$ Pisan was best remembered for defending Women in \textit{The Book of the City of Ladies}. 

\item Valla $-$ Lorenzo Valla was an early example of a humanist. He believed that pleasing the human senses was of most importance. Also, he found that a document from the 700s that granted the church rights to lots of land was a forgery.

\item Petrarch $-$ Petrarch coined the term 'Renaissance.' He began the early humanist movement. 

\item Dante $-$ Dante is the author of \textit{The Divine Comedy}. This work is considered very, if not the most important work of the Middle Ages. 

\item Boccaccio $-$ Boccaccio was an important Renaissance humanist. He wrote his book, \textit{The Decameron} in a vernacular language (meaning everyday people could read it). \textit{The Decameron} takes place near the outskirts of Florence, Italy. There are twelve people who share stories with each other. These twelve people are spending time in the outskirts of Florence to escape the raging Black Death.

\item Medici Family $-$ The Medici Family was the wealthy merchant family of Florence during the Renaissance. Because they had the greatest wealth, they were essentially the ruling family. The wealth they poured into art and the city itself spurred what is known as the Renaissance.

\item Da Vinci $-$ Da Vinci is one of the most famous artists of the Renaissance era. He was a prolific producer of art, as well as an early researcher of science. He had drawings of human anatomy, flying contraptions, and other inventions.

\item Michelangelo $-$ Michelangelo is one of the most renowned Renaissance artists. He is most famous for his work on the Sistine Chapel. To paint the ceiling, he had to spend excruciating amounts of time on his back.

\item Raphael $-$ Raphael is another Renaissance era artist. His pieces emphasized individuality and human features, as opposed to the general style of the time.

\item Alexander VI $-$ He was a corrupt pope of the Borgia Family. He encouraged his son, named Cesare, to create an Italian state ruled by their family. Alexander believed that this state was to be created by any means necessary.

\item Julius II $-$ His nickname is the "Warrior-Pope." He was involved in a lot of wara and politics. In some cases, he personally led troops to war against his enemies. He is responsible for the creation of St. Peter's Basilica.

\item Leo X $-$ Leo is responsible for the selling of indulgences. He began to sell them to fund the building of St. Peter's Basilica. Later, he would be the Pope that condemns Luther for being a heretic.

\subsection{Northern Renaissance}

\item Erasmus $-$ Desiderius Erasmus is the most famous Northern Renaissance humanist. He was of Dutch origins. He wrote \textit{The Praise of Folly}, where wrote that people should study the Bible for themselves, and that Christianity at heart, not through ceremonies was the most important.

\item More $-$ More was an early example of a Utopian Socialist. He wrote a book titled \textit{Utopia}, which comes from roots meaning 'non-existent.' In his book, he states that the government is corrupt, and that private property should not exist. He was later executed by Henry VIII for not agreeing that Henry VIII was the head of the church.

\item Durer $-$ Albrecht Durer was a painter, mostly known for three works: \textit{Devil} (1513), \textit{Melancolia I} (1514), and \textit{Rhinoceros} (1515).

\item Printing Press $-$ The printing press was made in 1454. Its main creator was Johannes Gutenberg, known for the publication of \textit{The Gutenberg Bible}. The printing press would later spur the Reformation into action, as people began to read the Bible for themselves due to the possibility of mass production permitted by the printing press.

\subsection{Compare and Contrast the Italian and Northern Renaissances}

\item Similarities $-$ Both the Italian and Northern Renaissance were inspired by classical Greek and Roman antiquity, and, therefore, were both based off of the idea of humanism.

\item Differences $-$ As opposed to the North, Italian Renaissance artists focused more on secular works. The Northern States were inspired by Christianity, and, as a result of this, Northern humanists became known as Christian humanists.

\subsection{Effects}

\item Philosophical/Religious $-$ Due to the creation of the printing press, people would begin reading the bible for themselves. This would lead to the Reformation and other religious movements.

\item Political $-$ For the duration of the Renaissance, the Italian city-states would develop a policy known as Balance of Power. This meant that if one of the states got wealthier or more powerful, the other city-states would work to even it out. The militaries of the Italian city-states, however, would prove weak following an invasion of Italy which would result in the Habsburg-Valois Wars.

\item Economic $-$ Many powerful cities, in both Italy and the North. would arise. These cities would become major trade stops for other empires. One example of such would be Amsterdam, which, for a period of time, be the center of European trade.

\item Social $-$ Following the Renaissance, books such as Castiglione's \textit{Book of the Courtier}, and Machiavelli's \textit{The Prince} would put in place social guidelines on how people in certain positions should act.

\item Education $-$ As a result of the printing press, literacy rates rose. People began to become interested in writings, such as encyclopedias, and, of course, the Bible.

\begin{center}
\section{\underline{The New Monarchs}}
\end{center}

\subsection{Causes}

\item Political $-$ 

\item Economic $-$ 

\item Need for Permanent Standing Army $-$

\item Taxation to Pay For Army and Bureaucracy $-$ Taxation resulted in even less money for the Peasants.

\item Classes
\begin{enumerate}[label=\arabic{*}.]
\setcounter{enumii}{36}
\item Nobles $-$

\item Church $-$

\item Middle $-$

\end{enumerate}

\subsection{Political Situation $-$ 16\textsuperscript{th} Century}

\setcounter{enumi}{39}

\item Spain $-$

\item France $-$

\item England $-$

\item Holy Roman Empire $-$

\subsection{Spain \& The Holy Roman Empire}

\item Ferdinand \& Isabella $-$

\item Charles V $-$

\item Phillip II $-$

\subsection{England}

\item Henry VII $-$

\item Henry VIII $-$

\item Elizabeth I $-$

\begin{center}
\section{\underline{The Age of Exploration}}
\end{center}

\subsection{Causes}

\item Political $-$

\item Economic $-$

\item Technological $-$

\item Religious $-$

\subsection{People}

\item Prince Henry the Navigator $-$

\item Columbus $-$

\item Magellan $-$

\item Diaz $-$

\item Da Gama $-$

\item Cortes $-$

\item Pizzaro $-$

\subsection{Effect on the Americas}

\item Destruction of Civilizations $-$

\item African Slavery $-$

\subsection{Effect on Europe} $-$

\item Intellectual $-$

\item Economic $-$

\item Political $-$

\subsection{Colombian Exchange} 

\item Diseases $-$ 

\item Food $-$ 

\begin{enumerate}[label=\arabic{*}.]
\setcounter{enumii}{67}
\item Potato on Population of Northern Europe $-$ 
\end{enumerate}
\setcounter{enumi}{68}
\item Price Revolution (Inflation) $-$

\begin{enumerate}[label=\arabic{*}.]
\setcounter{enumii}{69}
\item Causes $-$

\item Effects $-$
\end{enumerate}
\setcounter{enumi}{71}
\section{\underline{Religious Reformation}}

\subsection{Causes}

\item Religious $-$

\item Political $-$

\item Economic $-$

\item Social $-$

\item Northern European Renaissance Humanism $-$ 

\item Reason's for Luther's Success $-$ 

\subsection{Effects}

\item Religious $-$ 

\item Political $-$ 

\item Economic $-$ 

\item Social $-$ 

\subsection{Important People}

\item Wycliffe $-$ 

\item Huss $-$ 

\item Luther $-$ 

\item Zwingli $-$ 

\item Calvin $-$ 

\item Henry VIII $-$ 

\item Edward VI $-$ 

\item Bloody Mary $-$ 

\item Elizabeth I $-$ 

\item Mary, Queen of Scots $-$

\item Leo X $-$ 

\item Tetzel $-$ 

\item Frederick, Duke of Saxony $-$ 

\item Charles V $-$

\item Phillip II $-$

\item Ignatius Loyola $-$

\subsection{Terms}

\item Simony $-$

\item Nepotism $-$

\item Indulgences $-$

\item Babylonian Captivity $-$

\item Great Schism $-$ 

\item Protestant $-$

\item Antibaptist $-$

\item Salvation by Faith Alone $-$

\item Sole Authority of the Bible $-$

\item Sacraments $-$

\item Diet of Worms $-$ 

\item Peasant's Revolt $-$

\item Predestination $-$ 

\item Protestant Work Ethic $-$ 

\item Catholic/Counter Reformation $-$ 

\begin{enumerate}[label=\arabic{*}.]
\setcounter{enumii}{112}
\item Affirmation of Doctrines $-$

\item Reforms of Abuses $-$

\end{enumerate}
\setcounter{enumi}{114}
\item Council of Trent $-$

\item Jesuits $-$  

\item Baroque Art $-$

\item Church State Relations (Luther vs. Calvin) $-$

\item Six Articles $-$

\item Peace of Augsburg (1555) $-$

\section{\underline{Religious Wars}}

\subsection{Dutch Revolt (1508 $-$ 1609)}

\subsubsection{Causes}

\item Political $-$ 

\item Economic $-$ 

\item Religious $-$ 

\subsubsection{People}

\item Philip II

\item Duke of Alva

Elizabeth I (Spanish Armada)

\subsubsection{Effects}

\subsection{French Civil War (1562 $-$ 1598)}

\subsubsection{Causes}

\item Political $-$ 

\item Economic $-$

\item Religious $-$ 

\subsubsection{People}

\item Catherine de Medici $-$

\item Henry IV of Navarre $-$ 

\item Huguenots $-$ 

\item St. Bartholomew's Day Massacre $-$

\item Edict of Nantes $-$ 

\item Politique $-$ 

\subsubsection{Effects}

\subsection{Thirty Year's War (1618 $-$ 1648)}

\subsubsection{Causes}

\item Political $-$ 

\item Economic $-$ 

\item Religious $-$ 

\item Limits of Peace of Augsberg (1555) $-$

\subsubsection{The War}

\item Habsburgs vs. Most of Europe $-$

\item Phases

\begin{enumerate}[label=\arabic{*}.]
\setcounter{enumii}{140}

\item Bohemian (Bad) $-$

\item Danish (Danish Eat) $-$ 

\item Swedish (Swedish) $-$ 

\item French-Swedish (Fish) $-$

\end{enumerate}
\setcounter{enumi}{144}
\item Role of France $-$ 

\item Defenestration of Prague $-$ 

\item Wallenstein $-$ 

\item Gustavus Adolphus $-$ 

\item Richelieu $-$ 

\item Results (Peace of Westphalia) $-$  

\section{\underline{Constitutionalism}}

\subsection{Tudors}

\item Henry VII $-$ 

\item Henry VIII $-$ 

\item Edward VI $-$ 

\item Mary I (Bloody Mary) $-$ 

\item Elizabeth I $-$ 

\subsection{Stuarts}

\item James I $-$ 

\item Charles I $-$ 

\item Charles II $-$

\item James II $-$ 

\item William III \& Mary II $-$

\item Anne $-$ 

\item Cromwell $-$ 

\subsection{Documents}

\item Magna Carta $-$ 

\item Petition of Right $-$ 

\item Habeas Corpus $-$ 

\item Bill of Right $-$ 

\subsection{English Civil War (1640 $-$ 1649)}

\item Causes $-$ 

\item Reasons for Puritans Winning $-$ 

\item Effects $-$

\subsection{Glorious Revolution (1688)}

\item Causes $-$

\item Effects $-$

\subsection{Terms}

\item Church of England $\rightarrow$ Anglican Church $-$

\item Puritans $-$ 

\item Cavaliers $-$ 

\item Roundheads $-$ 

\item New Model Army $-$ 

\item Commonwealth $-$ 

\item Rump Parliament $-$ 

\item Levellers $-$ 

\item Restoration $-$ 

\item Test Act $-$ 

\item Whigs $-$ 

\item Tories $-$ 

\section{\underline{Absolutism}}

\item Causes $-$ 

\subsection{French Monarchs, Ministers, and Policies}

\item Henry IV $-$

\begin{enumerate}[label=\arabic{*}.]
\setcounter{enumii}{185}

\item Edict of Nantes $-$ 

\item Duke of Sully $-$

\end{enumerate} 
\setcounter{enumi}{187}

\item Louis XIII $-$

\begin{enumerate}[label=\arabic{*}.]
\setcounter{enumii}{188}

\item Cardinal Richelieu $-$ 

\end{enumerate}
\setcounter{enumi}{189}

\item Louis XIV (The Sun King) $-$

\begin{enumerate}[label=\arabic{*}.]
\setcounter{enumii}{190}
\item L'\'etat, C'est Moi $-$

\item Cardinal Mazarin $-$ 

\item Fronde $-$ 

\item Versailles $-$

\begin{enumerate}[label=\arabic{*}.]
\setcounter{enumiii}{194}

\item Purpose/Goal $-$

\item Effect $-$

\end{enumerate}
\setcounter{enumii}{196}

\item Bishop Bossuet (Divine Right) $-$

\item Colbert $-$ 

\item Mercantilism $-$ 

\item Revocation of Edict of Nantes $-$ 

\item Foreign Policy Goals $-$ 

\end{enumerate}

\subsection{Wars}
\setcounter{enumi}{201}

\item Dutch Wars $-$

\item War of Spanish Succession $-$

\item Cost $-$ 

\item Accomplishment $-$ 

\item Peace of Utrecht $-$ 

\item Balance of Power $-$ 

\item Legacy $-$ 

\item Culture \& Arts $-$ 

\item Finances \& Taxation $-$ 

\item Economic Development $-$ 

\item Louis XV $-$ 

\begin{enumerate}[label=\arabic{*}.]
\setcounter{enumii}{212}
 
\item Cardinal Fleury $-$ 

\end{enumerate}
\setcounter{enumi}{213}

\section{\underline{Scientific Revolution \& The Enlightenment}}

\item Pre-Renaissance Science $-$ 

\begin{enumerate}[label=\arabic{*}.]
\setcounter{enumii}{214}

\item Purpose $-$ 

\item Method $-$

\end{enumerate}
\setcounter{enumi}{216}

\item View of Universe $-$ 

\begin{enumerate}[label=\arabic{*}.]
\setcounter{enumii}{217}

\item Aristotle \& Ptolemy $-$ 

\item Copernicus \& Heliocentric Theory $-$ 

\item Brahe Contribution $-$ 

\item Kepler's Contribution $-$ 

\item Galileo's Contributions $-$ 

\begin{enumerate}[label=\arabic{*}.]
\setcounter{enumiii}{222}

\item Experimentation $-$ 

\item Telescope $-$


\end{enumerate}



\end{enumerate}
\setcounter{enumi}{224}

\subsection{Persecution by the Roman Catholic Church}

\item Effect on Science in Catholic Countries $-$

 
\begin{enumerate}[label=\arabic{*}.]
\setcounter{enumii}{225}

\item Newton $-$ 

\begin{enumerate}[label=\arabic{*}.]
\setcounter{enumiii}{226}

\item Law of Universal Gravitation $-$ 

\item \textit{Principia} $-$

\end{enumerate}
\setcounter{enumii}{228}

\item Bacon $-$ 

\begin{enumerate}[label=\arabic{*}.]
\setcounter{enumiii}{229}

\item Inductive Reasoning $-$ 

\item Method $-$

\item Empiricism $-$

\end{enumerate}
\setcounter{enumii}{232}

\item Descartes $-$ 

\begin{enumerate}[label=\arabic{*}.]
\setcounter{enumiii}{233}

\item Deductive Reasoning $-$ 

\item Cartesian Dualism $-$

\item "Cognito ergo su" $-$ 

\end{enumerate}

\end{enumerate}
\setcounter{enumi}{236}

\item Products of Scientific Revolution $-$

\begin{enumerate}[label=\arabic{*}.]
\setcounter{enumii}{237}

\item Intellectual $-$ 

\item Emergence of Scientific Community $-$

\item Scientific Method $-$ 

\item Belief in Reason $-$

\item Influence on Enlightenment $-$

\end{enumerate}
\setcounter{enumi}{242}
\subsection{Enlightenment}

\subsubsection{Important People}

\item Hobbes $-$ 

\begin{enumerate}[label=\arabic{*}.]
\setcounter{enumii}{243}

\item Human Nature $-$ 

\item Government $-$

\end{enumerate}
\setcounter{enumi}{245}

\item Locke $-$ 

\begin{enumerate}[label=\arabic{*}.]
\setcounter{enumii}{246}

\item Human Nature $-$ 

\item Government $-$

\end{enumerate}
\setcounter{enumi}{248}
\subsubsection{Philosophes}

\item Salons $-$ 

\item Elite vs. Masses $-$

\item Montesquieu $-$ 

\begin{enumerate}[label=\arabic{*}.]
\setcounter{enumii}{251}

\item \textit{Spirit of the Laws} $-$

\end{enumerate}
\setcounter{enumi}{252}

\item Voltaire $-$ 

\begin{enumerate}[label=\arabic{*}.]
\setcounter{enumii}{253}

\item Deism $-$ 

\item \textit{Treatise on Toleration} $-$

\item \textit{Candide} $-$ 

\item Admiration for Britain $-$ 

\item Frederick the Great $-$

\end{enumerate}
\setcounter{enumi}{258}

\item Rousseau $-$ 

\begin{enumerate}[label=\arabic{*}.]
\setcounter{enumii}{259}

\item Influence on Romantic Movement $-$ 

\item Effects of Civilization $-$

\item \textit{Social Contract} $-$

\begin{enumerate}[label=\arabic{*}.]
\setcounter{enumiii}{262}

\item General Will \& Totalitarianism $-$

\end{enumerate}
\setcounter{enumii}{263}

\item \textit{Emile}

\begin{enumerate}[label=\arabic{*}.]
\setcounter{enumiii}{264}

\item Education $-$ 

\item Treatment of Children $-$

\end{enumerate}
\end{enumerate}
\setcounter{enumi}{266}

\item Diderot $-$ 

\begin{enumerate}[label=\arabic{*}.]
\setcounter{enumii}{267}

\item \textit{Encyclop\'edie} $-$ 

\end{enumerate}
\setcounter{enumi}{268}

\item Physiocrates $-$

\begin{enumerate}[label=\arabic{*}.]
\setcounter{enumii}{269}

\item Quesnay $-$ 

\begin{enumerate}[label=\arabic{*}.]
\setcounter{enumiii}{270}

\item Laisser-faire $-$ 

\end{enumerate}
\setcounter{enumii}{271}

\item Adam Smith $-$

\begin{enumerate}[label=\arabic{*}.]
\setcounter{enumiii}{272}

\item \textit{Wealth of Nations} $-$ 

\item Capitalism $-$

\end{enumerate}

\end{enumerate}
\setcounter{enumi}{274}

\subsection{Enlightened Despotism}

\item Characteristics $-$

\begin{enumerate}[label=\arabic{*}.]
\setcounter{enumii}{275}

\item Reform of Justice and Legal Systems $-$ 

\item Improve Society \& Promote Happiness $-$

\item Religious Toleration $-$

\item Freedom of Press, etc. $-$ 

\item Economic Reform $-$ 

\item Education Reform $-$ 

\item Improve Efficiency $-$ 


\end{enumerate}
\setcounter{enumi}{282}

\item Truce Goal $-$

\subsection{Enlightened Monarchs}

\item Frederick the Great (Prussia) $-$ 

\item Peter the Great (Russia) $-$

\item Catherine the Great (Russia) $-$ 

\item Maria Theresa (Austria) $-$ 

\item Joseph II (Austria) $-$


\section{\underline{French Revolution}} 

\item \begin{tabular}{l c c c c}

Old Regime & Occupation & Taxation & Status & Problems/Gripes\\
\hline
1\textsuperscript{st} Estate & & & & \\
\hline
2\textsuperscript{nd} Estate & & & & \\
\hline
3\textsuperscript{rd} Estate & & & & \\
\hline
Bourgeoisie & & & & \\
\hline
Sans Culottes & & & & \\
\hline
Peasants & & & & \\
\hline

\end{tabular}

\subsection{Causes}

\item Finances $-$

\begin{enumerate}[label=\arabic{*}.]
\setcounter{enumii}{290}

\item Wars $-$

\item Versailles $-$

\item Interest on Debt $-$ 

\end{enumerate}
\setcounter{enumi}{293}

\item Inadequate Taxation $-$ 

\begin{enumerate}[label=\arabic{*}.]
\setcounter{enumii}{294}

\item Nobles R\'ecalcitrante $-$

\end{enumerate}
\setcounter{enumi}{295}

\item Injustice $-$

\item Enlightenment $-$ 

\item Louis XVI \& Marie Antoinette $-$

\item Parlement of Paris $-$ 

\item Estates General $-$ 

\item Cahiers de dol\'eances $-$

\item National Assembly $-$

\item Tennis Court Oath $-$ 

\subsection{1\textsuperscript{st} Phase $-$ Moderate Stage (1789 $-$ 1792)}

\item Fall of Bastille $-$ 

\item Great Fear $-$ 

\item Abolition of Feudalism $-$ 

\item Declaration of Rights of Man and Citizen $-$ 

\item Slogan $-$ 

\item Sans-Culottes Women Bring Back Royalty $-$ 

\item Financial $-$ 

\begin{enumerate}[label=\arabic{*}.]
\setcounter{enumii}{311}

\item Seizure of Church Property $-$

\item Assignats $-$ 

\end{enumerate}
\setcounter{enumi}{313}

\item Civil Constitution of Clergy $-$ 

\item Establishment of Departments $-$ 

\item Metric System $-$ 

\item Failure of Royal Family to Escape $-$ 

\item Edmund Burke $-$

\subsection{Reflections on the Revolution in France}

\item \textit{A Vindication of the Rights of Women} (Mary Wollstonecraft) $-$

\item \textit{Declaration of Rights of Women} (Olympe de Gouges) $-$

\subsection{2\textsuperscript{nd} Phase $-$ Radical Stage (1792 $-$ 1795)}

\item National Convention $-$ 

\item Jacobins $-$ 

\item Girondists $-$ 

\item Mountains $-$ 

\item Danton $-$ 

\item Marat $-$ 

\item Robespierre $-$ 

\item Declaration of Republic $-$ 

\item Execution of King and Queen $-$ 

\item Guillotine $-$ 

\item Brunswick Manifesto \& First Coalition $-$ 

\item Nationalism $-$ 

\item Levee en Masse $-$ 

\item Economic Accomodations to Sans Culottes $-$

\item Reign of Terror $-$ 

\item Committee of Public Safety $-$ 

\item Republic of Virtue $-$ 

\subsection{3\textsuperscript{rd} Phase $-$ Reactionary Stage (1795 $-$ 1799)}

\item Directory $-$ 

\item Corruption $-$ 

\subsection{Napoleonic Era (1799 $-$ 1815)}

\item Background $-$ 

\item Military Victories in Italy $-$ 

\item Invasion of Egypt $-$ 

\item Coup d'etat $-$ 

\item Consulate $-$ 

\item Emperor $-$ 

\item Concordat with the Roman Catholic Church $-$ 

\item Napoleonic Code $-$ 

\item Education Reforms $-$ 

\item Financial Reforms $-$ 


\begin{enumerate}[label=\arabic{*}.]
\setcounter{enumii}{349}

\item Bank of France $-$

\end{enumerate}
\setcounter{enumi}{350}

\item Meritocracy $-$ 

\begin{enumerate}[label=\arabic{*}.]
\setcounter{enumii}{351}

\item Legion of Honor $-$ 

\end{enumerate}
\setcounter{enumi}{352}

\item Conquest of Europe $-$ 

\item Failure of Trafalgar $-$ 

\item Foreign Policy \& Military Mistakes $-$ 

\begin{enumerate}[label=\arabic{*}.]
\setcounter{enumii}{355}

\item Continental System $-$

\item Peninsular (Spanish) War $-$

\item Invasion of Russia $-$ 

\end{enumerate}
\setcounter{enumi}{358}

\item Defeat at Battle of Nations $-$ 

\item Exile to Elba $-$ 

\item Escape from Elba \& 100 Days $-$ 

\item Battle of Waterloo \& Exile to St. Helena $-$ 

\section[\underline{Mercantilism, Agricultural Revolution, \& Industrial Revolution}]{\underline{Mercantilism and the Industrial Revolution}}

\item Mercantilism $-$ 

\subsection{Agriculture}

\item Causes $-$ 

\item Dutch \& English $-$ 
 
\begin{enumerate}[label=\arabic{*}.]
\setcounter{enumii}{365}

\item Reclamation of Land $-$

\end{enumerate}
\setcounter{enumi}{366}

\item Turnip Townshend $-$ 

\begin{enumerate}[label=\arabic{*}.]
\setcounter{enumii}{367}

\item Nitrogen-Fixing Crops $-$

\item Crop Rotation $-$

\end{enumerate}
\setcounter{enumi}{369}

\item New Farm Tools $-$ 

\begin{enumerate}[label=\arabic{*}.]
\setcounter{enumii}{370}

\item Jethro Tull $-$ Seed Drill $-$

\item Iron Plow $-$ 

\end{enumerate}
\setcounter{enumi}{372}

\item Selective Breeding of Animals $-$ 

\begin{enumerate}[label=\arabic{*}.]
\setcounter{enumii}{373}

\item Bakewell $-$

\item Protein Food $-$

\item Manure/Fertilizer $-$ 


\end{enumerate}
\setcounter{enumi}{376}

\item Enclosure Movement $-$ 

\begin{enumerate}[label=\arabic{*}.]
\setcounter{enumii}{377}

\item Effects $-$ 

\end{enumerate}
\setcounter{enumi}{378}

\subsection{Industrial Revolution}

\item Began in England in $-$ 

\item Textile Industry Inventions $-$ 

\item Steam Engine $-$ 

\item Relatively Inexpensive Iron \& Steel $-$

\item Transportation Systems $-$

\begin{enumerate}[label=\arabic{*}.]
\setcounter{enumii}{383}

\item Steam Boats/Ships $-$

\item Railroads $-$ 


\end{enumerate}
\setcounter{enumi}{385}

\item Spread of Industrialization $-$ 

\item Results $-$ 

\item Working Conditions of Proletariat $-$ 

\begin{enumerate}[label=\arabic{*}.]
\setcounter{enumii}{388}

\item Hours \& Wages $-$

\item Women $-$

\item Children $-$ 


\end{enumerate}
\setcounter{enumi}{391}

\item Sadler Committee $-$ 

\item Proletariat $-$ 

\item Change in Family Sturcture $-$ 

\item No Longer Unit of Production $-$ 

\item Just Unit of Consumption $-$ 

\item Relation of Parents to Children $-$ 

\item Urbanization $-$ 

\begin{enumerate}[label=\arabic{*}.]
\setcounter{enumii}{398}

\item Sanitation $-$

\item Crowding $-$

\item Disease $-$ 


\end{enumerate}
\setcounter{enumi}{401}

\item Luddites $-$ 

\item Increased Power of State $-$ 

\item Increased Power of Military $-$ 

\item Military Industrial Complex $-$ 

\item Reaction of Romantics $-$ 

\begin{enumerate}[label=\arabic{*}.]
\setcounter{enumii}{406}

\item Writers $-$ 

\item Composers $-$ 

\item Artists $-$  


\end{enumerate}
\setcounter{enumi}{409}

\subsection{Reaction of Economists}

\item \begin{tabular}{l c c}
\textsc{Classical School} & \textsc{Writings} & \textsc{Main Ideas} \\
\hline
Adam Smith & & \\
\hline
Malthus & & \\
\hline 
Ricardo & & \\
\hline
Benthem & & \\
\hline
John Stuart Mill & & \\
\hline
Saint Simon & & \\
\hline
Owen & & \\
\hline
Blanc & & \\
\hline
Engels & & \\
\hline
Marx & & \\
\end{tabular}

\item Basic Theories $-$

\begin{enumerate}[label=\arabic{*}.]
\setcounter{enumii}{411}

\item Economic View of History $-$

\item Class Struggle $-$

\item Inevitability of Revolution $-$ 

\item Surplus Value $-$

\item Communist Society $-$ 

\end{enumerate}
\setcounter{enumi}{416}

\section{\underline{The Congress of Vienna}}

\item Legitimacy $-$ 

\item Undue Influence of French Revolution $-$ 

\item Concert of Europe/Quadruple Alliance $-$

\item Nationalism $-$

\item Metternich in the Congress of Vienna $-$

\item \begin{tabular}{l c c c}

\hline  
Early 19\textsuperscript{th} Century: & Definition & Goals & Supporters \\
\hline
Conservative & & & \\
\hline
Reactionary & & & \\
\hline
Liberal & & & \\
\hline
Romantic & & & \\
\hline

\end{tabular}

\item German Confederation $-$

\begin{enumerate}[label=\arabic{*}.]
\setcounter{enumii}{423}

\item Carlsbad Decrees $-$

\end{enumerate}
\setcounter{enumi}{424}

\item Greek Revolution \& Independence $-$

\item Belgian Revolution \& Independence $-$

\subsection{Russia}

\item Alexander I $-$ 

\item Decembrists Revolution $-$ 

\item Nicholas I $-$ Reactionary Policies

\begin{enumerate}[label=\arabic{*}.]
\setcounter{enumii}{429}

\item Orthodoxy $-$

\item Autocracy $-$ 

\item Nationalism $-$ 

\item Secret Police $-$ 

\end{enumerate}

\setcounter{enumi}{433}

\item Alexander II $-$ 

\begin{enumerate}[label=\arabic{*}.]
\setcounter{enumii}{434}

\item Attempts at Modernization $-$

\begin{enumerate}[label=\arabic{*}.]
\setcounter{enumiii}{435}

\item Railroads $-$ 

\item Industry $-$

\end{enumerate}
\setcounter{enumii}{437}

\item Conflict Between "Westerners" \& "Slavophil" $-$

\item Assassinated $-$

\end{enumerate}
\setcounter{enumi}{439}

\item Alexander III $-$ 


\begin{enumerate}[label=\arabic{*}.]
\setcounter{enumii}{440}

\item Pogroms $-$ 

\end{enumerate}
\setcounter{enumi}{441}

\subsection{France}

\item Restoration of Louis XVIII $-$ 

\item Charles X $-$ 

\item 1830 Revolution $-$ 

\item Louis Phillipe $-$

\item Peterloo Massacre $-$ 

\item \begin{tabular}{l c c}

19\textsuperscript{th} Century Legislation & Purpose & Supporters \\
\hline
Six Acts (1819) & & \\
\hline
Repeal of Combination Acts (1824) & & \\
\hline
Great Reform Bill (1832) & & \\
\hline
Factory Act (1833) & & \\
\hline
Poor Law (1834) & & \\
\hline
Repeal of Corn Laws & & \\
\hline
Chartist Movement (1837 $-$ 1848) & & \\
\hline 
Reform Bill (1884) & & \\
\end{tabular}

\item \begin{tabular}{l c c}
\hline
19\textsuperscript{th} Century Political Parties & Goals & Supporters \\
\hline
Gladstone \& Liberals (Whigs) & & \\
\hline
Disraeli \& Conservatives (Tories) & & \\
\hline
\end{tabular}

\subsection{Irish Potato Famine}
\item British Reaction $-$ 

\item Deaths $-$ 

\item Emigration $-$

\subsection{Revolutions of 1848}

\item Nationalism $-$ 

\item Economic \& Class Struggles $-$

\item Famine $-$

\subsection{France}

\item Louis Phillipe $-$ 

\begin{enumerate}[label=\arabic{*}.]
\setcounter{enumii}{455}

\item Corruption $-$ 

\item Opposition to Expansion of Suffrage $-$ 

\item Demands for Workers' Rights $-$ 

\item Abdication $-$

\end{enumerate}
\setcounter{enumi}{459}

\item Second Republic $-$ 

\item Second Empire $-$ 

\subsection{Prussia}

\item Frederick William $-$ 

\begin{enumerate}[label=\arabic{*}.]
\setcounter{enumii}{462}

\item Freedom Press $-$ 

\item Male Suffrage $-$ 

\end{enumerate}
\setcounter{enumi}{464}

\subsection{Austrian Empire}

\item Vienna $-$ 

\item Hungary (Budapest) $-$ 

\item Czech (Prague) $-$ 

\item Northern Italy $-$ 

\subsection{German Confederation}


\item Frankfurt Assembly $-$ 

\section{\underline{Imperialism}}

\subsection{Causes}

\item Economic $-$ 

\item Military $-$ 

\item Political $-$ 

\item Religious $-$ 

\item Humanitarian $-$ 

\item Social Darwinism $-$ 

\item Sea Power $-$ 

\item Technology $-$ 

\subsection{Colonized Locations}

\item Egypt $-$

\item Africa $-$ 

\item Congo $-$ 

\item South Africa $-$ 

\subsection{Empires Involved}

\item Britain $-$ 

\item France $-$ 

\item Germany $-$ 

\item Italy $-$ 

\item Portugal $-$ 

\subsection{Britain $-$ 19\textsuperscript{th} Century}

\item Conservatives $-$ 

\begin{enumerate}[label=\arabic{*}.]
\setcounter{enumii}{487}

\item Disraeli $-$

\end{enumerate}
\setcounter{enumi}{488}

\item Liberals $-$ 

\begin{enumerate}[label=\arabic{*}.]
\setcounter{enumii}{489}

\item Gladstone $-$

\end{enumerate}
\setcounter{enumi}{490}

\item Labour Party $-$

\begin{enumerate}[label=\arabic{*}.]
\setcounter{enumii}{491}

\item Kier Hardie $-$ 

\end{enumerate}
\setcounter{enumi}{492}

\subsubsection{Reforms}

\item Vote $-$ 

\item Parliament $-$ 

\item Education $-$ 

\item Religious Toleration $-$ 

\item Food \& Drug $-$ 

\subsubsection{Ireland}

\item Home Rule Bills $-$ 

\subsection{Unification of Italy}

\item 1848 Revolution $-$ 

\begin{enumerate}[label=\arabic{*}.]
\setcounter{enumii}{499}

\item Mazzini $-$ 

\end{enumerate}
\setcounter{enumi}{500}

\item Victor Emmanuel II $-$

\item Cavour $-$ 

\item Crimean War $-$ 

\item Plombieres Agreement $-$ 

\item Austro-Sardinian War (1858) $-$ 

\item Austro-Prussian War (1866) $-$ 

\item Franco-Prussian War (1870) $-$ 

\subsection{Unification of Germany}

\item Holy Roman Empire $-$ 

\item 30 Years' War and Treaty of Westphalia $-$ 

\item Rise of Prussia $-$ 

\begin{enumerate}[label=\arabic{*}.]
\setcounter{enumii}{510}

\item Frederick William, The Great Elector $-$ 

\item Frederick I $-$ 

\item Frederick William I $-$ 

\item Frederick II $-$ 

\end{enumerate}
\setcounter{enumi}{514}

\item Napoleon $-$ 

\item Congress of Vienna $-$ 

\item Zollverein $-$ 

\item Hohenzollerns $-$ 

\item Frankfurt Assembly $-$ 

\item Kliendeutch vs. Grossdeutch $-$ 

\item Industrialization $-$ 

\item Bismarck $-$ 

\item Junkers $-$ 

\item Steps to Unification $-$ 

\begin{enumerate}[label=\arabic{*}.]
\setcounter{enumii}{524}

\item Danish War (1864) $-$ 

\item Austro-Prussian War $-$ 

\item North German Confederation (1867) $-$ 

\item Franco-Prussian War (1870 $-$ 1871) $-$ 

\end{enumerate}
\setcounter{enumi}{526}


\item Kaiser's Power $-$ 

\item Kulturkampf $-$ 

\item Social Reforms $-$ 

\item Foreign Policy $-$ 

\item Wilhelm II $-$ 

\subsection{Late 19\textsuperscript{th} Century France}

\item Napoleon III $-$ 
\begin{enumerate}[label=\arabic{*}.]
\setcounter{enumii}{532}

\item Path to Power $-$ 

\item First Phase (1851 $-$ 1860) $-$ 

\begin{enumerate}[label=\arabic{*}.]
\setcounter{enumiii}{534}

\item Domestic $-$ 

\item Foreign $-$ 

\end{enumerate}
\setcounter{enumii}{536}

\item Second Phase (1860 $-$ 1870) $-$

\begin{enumerate}[label=\arabic{*}.]
\setcounter{enumiii}{537}

\item Domestic Policy $-$ 

\item New Paris $-$ 

\begin{enumerate}[label=\arabic{*}.]
\setcounter{enumiv}{539}

\item Aesthetics $-$ 

\item Political Motivation $-$ 

\item Haussman $-$ 

\end{enumerate}
\setcounter{enumiii}{542}

\item Foreign Policy $-$

\end{enumerate}
\end{enumerate}
\setcounter{enumi}{543}


\item Franco-Prussian War $-$

\item Paris Commune (1870 $-$ 1871) $-$ 

\item Third Republic $-$


\section{\underline{Causes and Effects of War}}

\subsection{Reformation} 

\item Opponents $-$ vs.

\item Dates: to $-$

\item Location(s) $-$ 

\item Causes $-$

\item Name and Date of Treaty (If Applicable) $-$ 

\item Effects $-$ 

\subsection{30 Years' War}

\item Opponents $-$ vs.

\item Dates: to $-$

\item Location(s) $-$ 

\item Causes $-$

\item Name and Date of Treaty (If Applicable) $-$ 

\item Effects $-$

\subsection{French Civil Wars}
 
\item Opponents $-$ vs.

\item Dates: to $-$

\item Location(s) $-$ 

\item Causes $-$

\item Name and Date of Treaty (If Applicable) $-$ 

\item Effects $-$

\subsection{Dutch Rebellion}

\item Opponents $-$ vs.

\item Dates: to $-$

\item Location(s) $-$ 

\item Causes $-$

\item Name and Date of Treaty (If Applicable) $-$ 

\item Effects $-$ 

\subsection{English Civil War}

\item Opponents $-$ vs.

\item Dates: to $-$

\item Location(s) $-$ 

\item Causes $-$

\item Name and Date of Treaty (If Applicable) $-$ 

\item Effects $-$ 

\subsection{War of Spanish Succession}
 
\item Opponents $-$ vs.

\item Dates: to $-$

\item Location(s) $-$ 

\item Causes $-$

\item Name and Date of Treaty (If Applicable) $-$ 

\item Effects $-$ 

\subsection{War of Austrian Succession}

\item Opponents $-$ vs.

\item Dates: to $-$

\item Location(s) $-$ 

\item Causes $-$

\item Name and Date of Treaty (If Applicable) $-$ 

\item Effects $-$ 

\subsection{Seven Years' War}

\item Opponents $-$ vs.

\item Dates: to $-$

\item Location(s) $-$ 

\item Causes $-$

\item Name and Date of Treaty (If Applicable) $-$ 

\item Effects $-$

\subsection{Napoleonic Wars} 

\item Opponents $-$ vs.

\item Dates: to $-$

\item Location(s) $-$ 

\item Causes $-$

\item Name and Date of Treaty (If Applicable) $-$ 

\item Effects $-$ 

\item \begin{tabular}{l c c c c}

Italy & Leader(s) & Politcal/War & Economic & Intellectual/Religious \\
\hline
16\textsuperscript{th} Century & & & & \\
\hline
17\textsuperscript{th} Century & & & & \\
\hline
18\textsuperscript{th} Century & & & & \\
\hline
19\textsuperscript{th} Century & & & & \\
\hline
20\textsuperscript{th} Century & & & & \\

\end{tabular}

\item \begin{tabular}{l c c c c}

Britain & Leader(s) & Politcal/War & Economic & Intellectual/Religious \\
\hline
16\textsuperscript{th} Century & & & & \\
\hline
17\textsuperscript{th} Century & & & & \\
\hline
18\textsuperscript{th} Century & & & & \\
\hline
19\textsuperscript{th} Century & & & & \\
\hline
20\textsuperscript{th} Century & & & & \\

\end{tabular}

\item \begin{tabular}{l c c c c}

Austria & Leader(s) & Politcal/War & Economic & Intellectual/Religious \\
\hline
16\textsuperscript{th} Century & & & & \\
\hline
17\textsuperscript{th} Century & & & & \\
\hline
18\textsuperscript{th} Century & & & & \\
\hline
19\textsuperscript{th} Century & & & & \\
\hline
20\textsuperscript{th} Century & & & & \\

\end{tabular}

\item \begin{tabular}{l c c c c}

France & Leader(s) & Politcal/War & Economic & Intellectual/Religious \\
\hline
16\textsuperscript{th} Century & & & & \\
\hline
17\textsuperscript{th} Century & & & & \\
\hline
18\textsuperscript{th} Century & & & & \\
\hline
19\textsuperscript{th} Century & & & & \\
\hline
20\textsuperscript{th} Century & & & & \\

\end{tabular}

\item \begin{tabular}{l c c c c}

Russia & Leader(s) & Politcal/War & Economic & Intellectual/Religious \\
\hline
16\textsuperscript{th} Century & & & & \\
\hline
17\textsuperscript{th} Century & & & & \\
\hline
18\textsuperscript{th} Century & & & & \\
\hline
19\textsuperscript{th} Century & & & & \\
\hline
20\textsuperscript{th} Century & & & & \\

\end{tabular}

\item \begin{tabular}{l c c c c}

Portugal \& Spain & Leader(s) & Politcal/War & Economic & Intellectual/Religious \\
\hline
16\textsuperscript{th} Century & & & & \\
\hline
17\textsuperscript{th} Century & & & & \\
\hline
18\textsuperscript{th} Century & & & & \\
\hline
19\textsuperscript{th} Century & & & & \\
\hline
20\textsuperscript{th} Century & & & & \\

\end{tabular}

\item \begin{tabular}{l c c c c}

HRE/Prussia/Germany & Leader(s) & Politcal/War & Economic & Intellectual/Religious \\
\hline
16\textsuperscript{th} Century & & & & \\
\hline
17\textsuperscript{th} Century & & & & \\
\hline
18\textsuperscript{th} Century & & & & \\
\hline
19\textsuperscript{th} Century & & & & \\
\hline
20\textsuperscript{th} Century & & & & \\

\end{tabular}

\section{\underline{Events in Historical Order}}

\item 1 $-$

\item 2 $-$ 

\item 3 $-$ 

\item 4 $-$ 

\item 5 $-$ 

\item 6 $-$ 

\item 7 $-$ 

\item 8 $-$ 

\item 9 $-$ 

\item 10 $-$ 

\item 11 $-$ 

\item 12 $-$ 

\item 13 $-$ 

\item 14 $-$ 

\item 15 $-$

\item 16 $-$ 

\item 17 $-$ 

\item 18 $-$ 

\item 19 $-$ 

\item 20 $-$ 

\item 21 $-$ 

\item 22 $-$ 

\item 23 $-$ 

\item 24 $-$ 

\item 25 $-$ 

\item 26 $-$ 

\section{\underline{Charts}}

\item \begin{tabular}{|l|l|l|l|l|l|}

\hline
1648 & France & Spain & England & Holland & HRE \\
\hline
Political & & & & & \\
\hline
Economic & & & & & \\
\hline
Social & & & & & \\
\hline
Religion & & & & & \\
\hline


\end{tabular}

\item \begin{tabular}{|l|l|l|}

\hline
English Civil War & Causes & Effects \\
\hline
Political & & \\
\hline
Economic & & \\
\hline
Social & & \\
\hline
Religious & & \\
\hline

\end{tabular}

\item \begin{tabular}{|l|l|l|}

\hline
Term & Definition & Major People \\
\hline
Liberalism & & \\
\hline
Conservatism & & \\
\hline
Socialism & & \\
\hline
Romanticism & & \\
\hline
\end{tabular}

\item \begin{tabular}{|l|l|l|l|l|}

\hline
Revolution & Cause & Leadership & Extremes & Outcomes \\
\hline
Glorious & & & & \\
\hline
French & & & & \\
\hline
Russian & & & & \\
\hline
\end{tabular}

\item \begin{tabular}{|l|l|l|l|} 

\hline
& Vienna & Versailles & Yalta \\
\hline
Year & & & \\
\hline
People & & & \\
\hline
Why & & & \\
\hline
Positive & & & \\
results & & & \\
\hline
Negative & & & \\
results & & & \\
\hline
\end{tabular}
\section{\underline{Big Dates}}


\hspace{-25pt}\begin{tabular}{|p{.15\textwidth}|p{.25\textwidth}|p{.25\textwidth}|p{.25\textwidth}|}
\hline
Year & Event & Significance & Other \\
\hline
1454 & Printing press invented & Increased literacy & Invented by Johannes Gutenberg \\
\hline
1485$-$1603 & Length of Tudor Dynasty Rule & Creation of Court of Star Chamber & Stuart Dynasty followed Tudor rule  \\
\hline
1492 & Columbus sails the Atlantic Ocean & Beginning of Exploration Era & European Countries would compete for colonies \\
\hline
1517 & 95 Theses Posted & Demarcates the beginning of the Reformation & Written by Martin Luther  \\
\hline
1521 & Luther appears before the Diet of Worms and is excommunicated  & These events further support for Protestantism  & This encouraged Luther  \\
\hline
1534 & King Henry VIII passes the Act of Supremacy & Forms Anglican Church, with the British Monarch at its head  & Fueled the Reformation  \\
\hline
1536 & Calvin establishes Calvinism in Geneva & With the formation of an all-Calvinist state, Protestant ideals spread faster  & \\
\hline
1545$-$1563 & Length of the Council of Trent & Pushed some reforms and began the Counter-Reformation & Formed the Jesuits $-$ a religious order that sought to convert people to Catholicism  \\
\hline
1555 & Peace of Augsburg Signed & Became a temporary solution to religious problems & Would actually result in 30 Years' War \\
\hline
1588 & Spanish Armada Defeated & Showed that England became the leading naval power of Europe & \\
\hline
1598 & Edict of Nantes stops the French religious wars for a period & Gave rights to Huguenots (French Calvinists)  & Issued by Henry IV (Henry of Navarre)  \\
\hline
\end{tabular}
\newpage
\hspace{-25pt}\begin{tabular}{|p{.15\textwidth}|p{.25\textwidth}|p{.25\textwidth}|p{.25\textwidth}|}
\hline
1603$-$1714 & Length of the Stuart rule of England  & Ruled during English Civil War and the Glorious Revolution  & \\
\hline
1613$-$1917 & Length of Romanov Dynasty  & During and following the Age of Absolutism, Russia was ruled by the Romanovs  & Ended with the Russian Revolution  \\
\hline
1618$-$1648 & Length of the Thirty Years' War  & Fought over religious freedom in modern day Germany  & Protestants came from North, and Catholics from the South  \\
\hline
1643$-$1715 & Age of Louis XIV  & Known as the Sun King, he is most known for bankrupting the French treasury & Played a role in War of Spanish Succession, and caused Protestants to flee \\
\hline
1648 & End of the Thirty Years' War & Treaty of Westphalia is written & This creates relative Religious peace \\
\hline
1649 & Death of Charles I of England & Demarcates the English Civil War  & Caused Charles II to create an army to combat Cromwell \\
\hline
1688$-$1689 & Glorious Revolution in England & William and Mary of Orange rise to the throne & Creates relative peace in England \\
\hline
1689$-$1725 & Peter the Great's Rule & Russia became significantly more westernized than before  & This would open trade and other oppurtunities between Russia and the West  \\
\hline
1700s & The Enlightenment  & Became a period of free thought, in which many new philosophical ideas were created  & One of the most important periods for politics  \\
\hline
1701$-$1918 & Hohenzollern rule in Prussia & The Hohenzollerns were aggressive leaders of Prussia  & Often seen as barbarians \\
\hline
1748 & War of Austrian Succession ends  & Peace of Aix-La-Chapelle is signed  & \\
\hline
\end{tabular}
\newpage
\hspace{-25pt}\begin{tabular}{|p{.15\textwidth}|p{.25\textwidth}|p{.25\textwidth}|p{.25\textwidth}|}
\hline
1760$-$1830 & First Industrial Revolution  & A period of rapid advancement and mass production  & Most inventions today are a direct result of the Industrial Revolution, like cars, phones, computers, etc  \\
\hline
1776 & American Revolution Begins & Inspires other revolutions for freedom, such as the French and Haitian revolutions  & Also nearly bankrupted France \\
\hline
1789 & French Revolution Begins & Began a period of bloodshed that would not end until the Congress of Vienna  & Many nobles and citizens were brutally murdered  \\
\hline
1790s & A Period of French unrest  & Saw mass executions, especially during the Reign of Terror  & One of the first major uses of Secret Police to spy on 'counter-revolutionaries'  \\
\hline
1794 & The Reign of Terror  & Led by Robespierre, the Committee of Public Safety executed many innocent, but 'suspected,' counter-revolutionaries  & Caused many unnecessary deaths  \\
\hline
1804 & Napoleon becomes Emperor & Napoleon was very aggressive in his foreign policy, which is evident in the Napoleonic Wars $-$ greatly expanded France  & Left a huge legacy for the name Napoleon  \\
\hline
1814$-$1815 & Napoleon returns from exile to Elba & Begins campaign in Russia $-$ big blunder & Also called the Hundred Days \\
\hline
1815 & Congress of Vienna & Conservative European leaders met and returned Europe to pre-French Revolution state  & Kept relative peace for about 100 years  \\
\hline
\end{tabular}
\newpage
\hspace{-25pt}\begin{tabular}{|p{.15\textwidth}|p{.25\textwidth}|p{.25\textwidth}|p{.25\textwidth}|}
\hline
1830 & Revolution in France & Revolt against Charles X  & Louis Phillipe becomes King, Greeks and Belgians gain independence  \\
\hline
1832 & British Reform Bill of 1832  & This act broadened voting rights  & Proposed by the Whig party  \\
\hline
1848 & Many more revolutions shook Europe & Fueled by nationalist beliefs, revolutions broke out in France, Austria, Prussia, Italy, and many more places  & Marx and Engels also published the Communist Manifesto in 1848 \\
\hline
1852 & Napoleon III rises to the throne  & He won largely because of his name, and ruled France until 1870  & Napoleon III was the nephew of Napoleon Bonaparte  \\
\hline
1861 & Italy is Unified  & The unification of Italy would lead to the rise of Fascism  & Also, Alexandr II emancipated Russian serfs in 1861  \\
\hline
1866 & Austro-Prussian War  & Led to the decisive defeat of Austria  & \\
\hline
1870 & French first republic formed  & This was the French government from 1870 up until its loss to Germany during World War II  & Later split into occupied France and Vichy France, headed by de Gaulle.  \\
\hline
1870$-$1871 & Franco-Prussian War  & This showed that Germany had become the major power of Europe  & Although it was instigated by Bismarck, France was seen as the aggressor  \\
\hline
1871 & Unification of Germany  & United many different nations and states into one $-$ led to a build up of military  &    \\
\hline
1880$-$1914 & Imperialism  & During this period, there was a sharp rise in colonization of places with lots of natural resources, like India and Africa  & Would later be taken apart during de-Colonization, but reworked in Neocolonialism  \\
\hline
\end{tabular}
\newpage
\hspace{-25pt}\begin{tabular}{|p{.15\textwidth}|p{.25\textwidth}|p{.25\textwidth}|p{.25\textwidth}|}
\hline
1900 & Sigmund Freud publishes \textit{The Interpretation of Dreams}  & This was big because it was the first major look into psychology  & Freud was renowned for a long period, until it was discovered he was a fraud  \\
\hline
1904$-$1905 & Russian Revolution of 1905 & Resulted in a limited constitutional monarchy in Russia  &  \\
\hline
1905 & Einstein publishes the \textit{Theory of Relativity} & The Theory of Relativity was revolutionary because it unified two fields: Electromagnetic and Gravitational  &  The concept of $E=mc^2$ comes from this theory \\
\hline
    1914 6/28 & Archduke Franz Ferdinand is Assassinated  & Gavrilo Princip's assassination of Franz Ferdinand sparked the flame of World War I  & Gavrilo Princip was part of the \textsc{Black Hand}, a nationalist Serbian group \\
\hline
1914 7/3 & Funeral was held for Franz Ferdinand  & This Austrians would later declare war on Serbia  & \\
\hline
1917 & Russian Revolution (February)  & This led to the establishment of the Provisional Governement, with Alexandr Kerensky at the head  & The USSR would not be created until after the Octover Revolution \\
\hline
1917 & Russian Revolution (October)  & This would lead to the formation of the Union of Soviet-Socialist Republics  & This would later cause tensions with the west because of the Communist ideals upon which the USSR was based on  \\
\hline
1918 & End of World War I & The death and bloodshed ended for about 20 years & \\
\hline
1918 & Spanish Influenza Pandemic begins & The pandemic was able to spread quickly because of the poor conditions on the battlefields of Europe, and the poor understanding of biology  & The death toll is estimated to be from 20 million $-$ 50 million  \\
\hline
\end{tabular}
\newpage
\hspace{-25pt}\begin{tabular}{|p{.15\textwidth}|p{.25\textwidth}|p{.25\textwidth}|p{.25\textwidth}|}
\hline
1919 & Treaty of Versailles is signed & Harsh punishments would lead to the rise of Hitler & Harshest clauses: Admittance of War Guilt and Reparations \\
\hline
1920 & The Age of Totalitarians Begins & Examples include: Joseph Stalin (Iossef Vissiaronovich Dzhugashvelli), Adolf Hitler, and Mussolini  & \\
\hline
1921$-$1927 & Lenin's New Economic Policy  & The NEP (New Economic Policy) was a form of socialism with some capitalist policies that allowed citizens to sell their grain surpluses  & The NEP failed greatly  \\
\hline
1923 & End of Weimar hyper-inflation  & The period from 1921$-$1923 saw the hyper-inflation of the German Mark  & \\
\hline
1928 & The Kellogg-Briand Pact Signed  & This treaty proclaimed that war was not to be used as an instrument of foreign policy  & This treaty failed miserably at preventing the Second World War  \\
\hline
1929 & The Great Depression Begins  & Following Black Tuesday, all of the world's capitalist economies would collapse   & Caused famines and shortages across the globe \\
\hline
1930$-$1935 & France's Maginot Line  & This was a series of concrete barriers and bunkers that were meant to prevent a German attack on France  & This failed miserably, as Germany would flank France through Belgium  \\
\hline
1933 & Hitler Becomes Chancellor of Germany & This would lead to World War II and the Holocaust  & Hitler succeeded greatly thanks to the policy of appeasement used by the West  \\
\hline
\end{tabular}
\newpage
\hspace{-25pt}\begin{tabular}{|p{.15\textwidth}|p{.25\textwidth}|p{.25\textwidth}|p{.25\textwidth}|}
\hline
1936 & Rome-Berlin Axis  & This treaty was signed in order to unite Germany and Italy  & Germany and Italy became close through the Spanish Civil War  \\
\hline
1936 & Remilitarization of the Rhineland  & Hitler sent a squad of armed forces to the Rhineland, clearly in violation of the Locarno Pact  & Hitler got away with this thanks to appeasement  \\
\hline
1938 & Kristallnacht & Orchestrated by Hitler, this night saw raiding of Jewish shops and murder of Jewish people in Germany & Translates to "Night of Broken Glass" \\
\hline
1939 & Molotov-Ribbentrop Pact Signed  & Germant signed this non-aggression pact with the Soviet Union in order to prevent, at least for the beginning, a two-front war  & Stalin agreed to this because he wanted to split Poland in order to create a buffer between the Soviet Union and Germany  \\
\hline
1939 & Invasion of Poland demarcates German aggression & This marked the beginning of World War II & Bloodshed begins \\
\hline 
1940 & Invasion of France & France surrenders to Germany after roughly one month of battling & Germany's Blitzkrieg proves extremely successful \\
\hline
1941 & Invasion of the Soviet Union & June 22nd, 1941 marks the day Operation Barbarossa (Hitler's invasion plan) begins & The Great Patriotic War would last roughly 3 years \\
\hline
1944 & Invasion of Normandy & The storming back of Europe involved the United States, Britain, and Canada & Germans were pushed closere to Berlin  \\
\hline
1945 & V-E Day  & Beginning with the Soviet entrance into Berlin, this marked the end of the Second World War  & Tensions between the West and the Soviet Union rose quickly following the fall of Germany  \\
\hline
\end{tabular}
\newpage
\hspace{-25pt}\begin{tabular}{|p{.15\textwidth}|p{.25\textwidth}|p{.25\textwidth}|p{.25\textwidth}|}
\hline
1945 & U.S. Drops Atomic bomb on Hiroshima  & First aggressive use of the a-bomb & Japanese still do not surrender \\
\hline
1945 & U.S. Drops Atomic bomb on Nagasaki  & Second aggressive use of the a-bomb & Japanese surrender  \\
\hline
1945 & V-J Day  & Japan surrenders to the United States  & Japanese had to sign their surrender on an American ship  \\
\hline
1945 & Yalta/Potsdam Conferences  & Discuss plans for Germany post-war  & Do not discuss Eastern Europe $-$ this caused a rise in tension during the cold war  \\
\hline
1946 & Churchill gives his most famous speech  & Churchill comes up with the quote that an "Iron Curtain" has covered Europe  & This quote is often used to refer to the Soviet Union's hold over Europe  \\
\hline
1947 & The Central Intelligence Agency was founded  & This would be the center for anti-Soviet subversive operations throughout the Cold War  & This was essentially a counter to the Soviet NKVD, which would later become the KGB  \\
\hline
1948 & The Berlin Airlift takes place  & After a Soviet blockade of West Berlin, the United States and Britain decided to carry out "Operation Vittles"  & This lasted until 1949  \\
\hline
1949 & NATO(North Atlantic Treaty Organization) is founded & It was created as an alliance against possible Soviet aggression & Soviet Union would respond with the WTO(Warsaw Treaty Organization) \\
\hline
1953 & Stalin dies  & Caused a scramble for the position of Leader of the Soviet Union  & Khrushchev (after a short period of Malenkov) becomes leader \\
\hline
1958 & The Fifth Republic in France Begins  & Charles de Gaulle becomes president of the Fifth Republic of France  & \\
\hline
1961 & Berlin Wall Built  & The Berlin wall is a perfect example of the "Iron Curtain" that surrounded Europe during the Cold War  & People were not allowed to cross from East Germany to West  \\
\hline
\end{tabular}
\newpage
\hspace{-25pt}\begin{tabular}{|p{.15\textwidth}|p{.25\textwidth}|p{.25\textwidth}|p{.25\textwidth}|}
\hline
1968 & Czechoslovakian Uprising Occurs  & The Prague Spring began with the election of Alexander Dub\u cek, but was later crushed by the Soviets  & The Czechoslovakian Uprising is also called the Prague Spring  \\
\hline
1979 & Soviet Invasion of Afghanistan begins  & Although this was meant to be a quick invasion, it lasted significantly longer due to the leave of Leonid Brezhnev as party leader  & The invasion was also lengthened due to the Central Intelligence Agency's subversive actions and support of the Mujahadeen  \\
\hline
1980 & Ronald Reagan elected into office  & Reagan strongly opposed Soviet rule, and, as such, had an iron foreign policy when dealing with the Soviets  & Reagan referred to the Soviet Union as "The Evil Empire"  \\
\hline
1985 & Gorbachev becomes the leader of the Soviet Union  & He implemented his policies, known as Glasnost and Perestroika  & His rule led to a period of relative relaxation, known as D\'etente  \\
\hline
1988 & Demonstrations and subsequent freedom in Czechoslovakia  & Protests in Czechoslovakia and the weakening of the Soviet system led to the freeing of Czechoslovakia from communist rule  & \\
\hline
1990 & East and West Germany Unify  & This showed that the era of Soviet rule was over, and that the Cold War was coming to a stop  &   \\
\hline
1991 & Yugoslavia begins to disintegrate & This would be the beginning of the nationalist schisms and conflicts in the Balkans  & Most notable is the conflict between Serbia and Albania  \\
\hline
1991 & Soviet Union collapses & This was the end of the Cold War  & Marked the end of Communist vs. Capitalist, but not East vs. West  \\
\hline
\end{tabular}


\section{\underline{People}}

\begin{tabular}{|l|l|l|l|}
\hline
Conflicting & Issues of & Time \& & Impact \\
Personalities & Conflict & Place & \\
\hline
Wilson vs. & & & \\
Clemenceau & & & \\
\hline
Bismarck vs. & & & \\
Napoleon & & & \\
\hline
Lenin vs. & & & \\
Kerensky  & & & \\
\hline
Galileo vs. & & & \\
Urban & & & \\
\hline
Metternich & & & \\
Italy & & & \\
\hline
Luther vs. & & & \\
Charles V & & & \\
\hline
Cromwell vs. & & & \\
Charles & & & \\
\hline
Truman vs. & & & \\
Stalin & & & \\
\hline
Phillip II vs. & & & \\
Elizabeth I & & & \\
\hline
Hitler vs. & & & \\
Chamberlain & & & \\
\hline


\end{tabular}

\begin{tabular}{|l|l|l|l|l|l|}

\hline
 & 16\textsuperscript{th} & 17\textsuperscript{th} & 18\textsuperscript{th} & 19\textsuperscript{th} & 20\textsuperscript{th} \\
\hline
Most & & & & & \\
influential & & & & & \\
politician & & & & & \\
\hline
Greatest & & & & & \\
intellectual & & & & & \\
& & & & & \\
\hline
Greatest & & & & & \\
artist & & & & & \\
& & & & & \\
\hline
Famous & & & & & \\
economist & & & & & \\
    & & & & & \\
\hline
Bad Guy & & & & & \\
    & & & & & \\
    & & & & & \\
\hline
Good Guy & & & & & \\
    & & & & & \\
    & & & & & \\
\hline

\end{tabular}

\begin{tabular}{|l|l|l|l|l|l|}

\hline
Master & Political & Economic & Religious & Social & Intellectual\\
PERSIA & & & & & \\
\hline
1450 & & & & & \\
\hline
1650 & & & & & \\
\hline
1789 & & & & & \\
\hline
1815 & & & & & \\
\hline
1848 & & & & & \\
\hline
1870 & & & & & \\
\hline
1914 & & & & & \\
\hline
1918 & & & & & \\
\hline
1939 & & & & & \\
\hline
1945 & & & & & \\
\hline
1964 & & & & & \\
\hline



\end{tabular}

\end{enumerate}



\end{document}


































