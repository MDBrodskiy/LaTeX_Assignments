\documentclass[12pt]{article} 

\usepackage{titling}
\usepackage{amsmath}
\usepackage{enumitem}
\usepackage{amssymb}
\usepackage{geometry}
\geometry{top=1.0in,bottom=1.0in,left=1.0in,right=1.0in}
\newcommand{\subtitle}[1]{%
  \posttitle{%
    \par\end{center}
    \begin{center}\large#1\end{center}
    \vskip0.5em}%
}
\usepackage{everysel}
\usepackage{ragged2e}
\renewcommand*\familydefault{\ttdefault}
\EverySelectfont{%
\fontdimen2\font=0.4em% interword space
\fontdimen3\font=0.2em% interword stretch
\fontdimen4\font=0.1em% interword shrink
\fontdimen7\font=0.1em% extra space
\hyphenchar\font=`\-% to allow hyphenation
}

%\pagenumbering{Roman}

\begin{document}

%------------------------------------------------------------------------------------------
% Title
%------------------------------------------------------------------------------------------


\author{Michael \textsc{Brodskiy}}
\title{Defining Chapter 27 \\European History AP}
\subtitle{Mrs Fisher}
\date{March 2, 2020}
\maketitle


%------------------------------------------------------------------------------------------
% Questions
%------------------------------------------------------------------------------------------

\begin{enumerate}


\item Triple Alliance $-$ An alliance formed as early as 1882. It consisted of the German Empire, Austria-Hungary, and Italy; intended to fight France and Britain.

\item Schlieffen Plan $-$ Written in 1906, the Schlieffen Plan was a war plan devised by \emph{Generalfeldmarschall} Alfred von Schlieffen; It was intended to avoid a two front war.

\item Total War $-$ A state in which a govenrment directs all facets toward a wartime effort, and battles take place in areas with civilians. (e.g World War II)

\item Totalitarian $-$  Describes a ruler that has complete control over the government and its citizens.

\item Western Front $-$ The front that saw the most fighting during the Great War. Battles included Somme, Verdun, and Marne.

\item Bolsheviks $-$ A group of Marxist-Leninists that represented the \emph{bolshenstvo}, or majority of revolutionaries.

\item Principle of National Self-Determination $-$ This is a branch of law about questions on nationality and citizenship.

\item War Reparations $-$ These are payments issued following damages done by war.

\item First Balkan War, 1912 $-$ Serbia, Greece, and Bulgaria attacked the Ottoman Empire as a response to the annexation of Bosnia and Herzegovina.

\item Lawrence of Arabia $-$ A British intellgence agent who orchestrated revolutions in the Ottoman Empire. He would ultimately cause the downfall of the Ottoman Empire. 

\item Reinsurance Treaty $-$ A treaty written by Bismarck. It stated that Russia and Germany would stay neutral, with respect to each other, in the case of a war, unless Germany invaded France, or Russia invaded Austria-Hungary.

\item Algeciras Conference of 1906 $-$ In this conference, Germany saw other countries as hostile towards itself, and the other countries decided that Germany was a threat.

\item Anglo-French Entente of 1904 $-$ This agreement ended years of feuds. It permitted for the English to work in Egypt, and France in Morocco. This angered Germany because it wanted territories as well.

\item Black Hand $-$ A Serbian nationalist group; Gavrilo Princip, the man who assassinated Franz Ferdinand, was a follower.

\item Third Balkan War, 1914 $-$ A military conflict in which Austria-Hungary invaded Serbia; it wanted to take over.

\item Lusitania $-$ A passenger liner that was shot down by a German unterwasserboot (U-Boat), while carrying supplies for the front.

\item Admiral Tirpitz $-$ Alfred von Tirpitz was the admiral for the expanding German fleet.

\item German Auxiliary Service of Law of 1916 $-$ This law required males, ages 17 - 60, to work jobs considered important for the war effort.

\item David Lloyd George $-$ Britain's prime minister at the signing of the Treaty of Versailles. He wanted for Germany to pay huge amounts for war reparations.

\item Rasputin $-$ A `holy man' who used his powers to heal the sick. He influenced Tsarina Alexandra during Nicholas II's leave as commander of the military.

\item Georges Clemenceau $-$ The French leader at the signing of the Treaty of Versailles.

\item Duma $-$ An elected Russian parliamentary group that did not truly hold any power.

\item Tsar Nikolai II $-$ The tsar of Russia during the Great War. Later executed following the February revolution.

\item Leon Trotsky (Leonid Davidovich Bronstein) $-$ The more extremist of the trio. Bronstein changed his last name to that of his \emph{silka} warden, Trotsky. He led the military during the Russian civil wars, and thus, was destined to rule following Lenin. He came up with the idea of five year plans.

\item Petrograd Bread Riots (1917) $-$ Over 50,000 workers in Petrograd went on strike to demand bread. By March 10, most industries were frozen in Petrograd.

\item Congress of the Soviets $-$ The supreme power in the RSFSR from 1917 $-$ 1936

\item Kiev Mutiny (1918) $-$ \emph{Bolsheviks} organized a revolt that was intended to support the Red Army.

\item Alexander Kerensky $-$ Leader of the provisional government of Russia following the February revolution. 

\item Vladimir Lenin (Vladimir Ilyich Ulyanov) $-$ Ulyanov was a well educated, upper class individual that was a strong believer in Communism. Contrary to Marx, Ulyanov postulated that the revolution would occur in the middle class, in an industrialized country.

\item Army Order No. 1 $-$ An order that took powers from the military officers and gave it to Soviets.

\item Constituent Assembly $-$ An assembly that lasted less than a day. Lenin disbanded this after Bolsheviks received less than one fourth of the vote.

\item White Opposition $-$ The Whites were the main opposition to the bolsheviks (the Reds), during the Russian Civil War.

\item Treaty of Brest-Litovsk (1918) $-$ The treaty that permitted Russia to leave the war, without the Triple Alliance pursuing. In exchange, Germany would receive lands in western Russia.

\item Cheka $-$ The secret police established following the October revolution. After Stalin takes power, it would evolve into the \emph{NKVD}. Following the death of Stalin, it would become the \emph{KGB}

\item The Battles of Tannenburg and the Masurian Lakes $-$ Early into the Great War, Russians attacked at Tannenburg. A large force of roughly 1 million men was sent. Nearly all died within a week.

\item The First Battle of the Marne $-$ This took place roughly 20 miles outside of Paris. French troops slowed the German advance, and entrenchments began to be built.

\item The Battles of Somme and Verdun $-$ The Battle of the Somme took place in 1916. Verdun was the longest lasting battle of the Great War. Over 700,000 troops on both sides were lost yearly.

\item Battle of Gallipoli $-$ The Battle of Gallipoli was an amphibious landing that was intended to weaken in the Ottoman Empire. This happened in 1916.

\item 2\textsuperscript{nd} Battle of the Marne $-$ This was a major German offensive. It was Germany's final push before the end of the war to end all war.

\item Various ethnic, political, and social influences led to the beginning of the Great War. For example, ethnic divisions in the Balkans caused the various Balkan wars, which, in turn, led to the assassination of Archduke Franz Ferdinand, and , ultimately, the Great War. Another such example is that of the Moroccan crises. Germany had a powerful navy, and, as such, attempted to control vital waterways. Part of the German navy was sent to French Morocco in order to gain control, however they were stopped in their tracks. The hotspots, therefore, were the Balkans and the African colonies.

\item Bismarck was called this because he acted as an `honest broker.' Although he wanted Germany to be strong, he presented himself this way to gain respect from other powers. Bismarck wanted to preserve peace, and thus, he called the Berlin Congress in order to discuss the Balkan problems, and preserve the Three Emperor's League. Ultimately, the Berlin Congress just put off a decision to a later date.

\item Spoils of war always cause rivalry. The Ottoman Empire spanned an extensive area, and thus, it was a target for the conquering. Later, Britain would send an agent, known as Lawrence of Arabia, to sponsor revolt to weaken the area.

\item Austria and Germany have a common ancestry, and similar political goals, and thus, Germany and Austria united. Germany's pursuits in allying with Russia revolved around Russia's militaristic prowess and reputation.

\item Bismarck's main goal can be simplified into just two words: Germany first.

\item All three participating countries had their respective political goals. As such, they united in pursuit of these goals. This alliance was made so each country could stand for the others; Italy needed protection from France, Germany to have powerful allies, and Austria-Hungary to have back-up in case of war.

\item Bismarck was unable to keep the Three Emperor's League because there was infighting between Austria-Hungary and Russia over disputes in the Balkans. This mostly had to do with Russia's support of Serbia, whereas Austria-Hungary wanted to annex certain areas. Bismarck negotiated the Triple Alliance to have allies in case of (expected) war.

\item Wilhelm II believed that because England was a land of liberalism, they were a land of Satan. He acted accordingly, and thus, England perceived Germany as a threat. His aggressive foreign policy was a significant difference from that of Bismarck.

\item One crisis during this period was the First Moroccan Crisis. The German Emperor, Wilhelm II, declared support for the Sultan of Tangiers. This agitated France and Britain. This dispute would not be (temporarily) resolved until 1906, during the Algeciras conference. Another crisis was the arms race between Britain and Germany. Germany was expanding its fleet at a rapid rate, which was a reason for worry for the British, which was the greatest naval superpower for a long period. This issue would not be resolved, and thus, would be a factor in the outbreak of war. Also, the issue in the Balkans with the annexation of Bosnia by Austria-Hungary angered Germany and Austria-Hungary's former ally, Russia. This would only have a makeshift solution, and thus, would lead to Russian intervention in the Great War. Finally, the Second Moroccan crisis came. This time, it would be the French stationing their troops in Agadir. Although this did not evoke anger from the German side, the Germans did demand compensation in the form of land. This would lead to the signature of the Treaty of Fes in 1912. The French would gain control over the Moroccan protectorate.

\item For the First Moroccan Crisis, Bismarck would not have instigated a negative reaction, but would most likely have bartered for a fair deal, peacefully. For the Second Moroccan Crisis, Bismarck would most likely have tried to discuss the issues instead of having a naval interference take place.

\item It was clear that a war was to ensue with all of the political turmoil of the time. The British, therefore, wanted an ally, with France being the closest possible option. This is why the Entente Cordiale was decided to take place.

\item The First Moroccan crisis was believed to have been settled during the Algeciras conference of 1906. The European powers thought that because it worked here, they could apply such a technique to other problems, such as the second Moroccan crisis. This technique led to the Treaty of Fes, which would be temporary, as Germany wanted African colonies, and was thus angered by this treaty. With the Bosnian crisis, Austria-Hungary led negotiations with Russia, Serbia, and Britain. This, like the other events, would only have a temporary solution. The same process would occur during the First and Second Balkan war. The process can be summed up as:
\vspace{-10pt}
\begin{center}
Crisis $\longrightarrow$ Negotiations $\longrightarrow$ Temporary Solution $\longrightarrow$ War 
\end{center}

\item The Archduke travelled to Sarajevo in order to inspect the armed forces, left after the annexation of Bosnia.

\item Germany's biggest problem was the possibility of a two-front war. It was supposed to be ameliorated through the implementation of the \textsc{Schlieffen Plan}, created by Generalfeldmarshall Alfred Von Schlieffen. The plan failed because Russia had mobilized faster than expected.

\item The engagements were deadly because this was a war fought with old tac-tics, but new technology. This meant that machine guns, poisonous gas, and artillery caused heavy losses on both sides.

\item This was because both sides were pretty equally matched. The entrenchments dug were similar, and thus the goal of the war was who could wait it out longer. Both sides were weakened with time, but Germany fell first.

\item Italy was promised to be treated as an ally, and would thus keep some colonies.

\item In the home front, there were many wives left widowed, and families left without fathers and sons. In Britain, regulations on bread prices were attempted in order to keep civilians appeased.

\item \textbf{\textsc{World War I was NOT a total war.}} As such, civilians were appeased by simple regulation of economies, such as bread prices.

\item Both Austria-Hungary and Russia were not very industrialized throughout the war. Peak industrialization would occur in Russia during the early years of the Union of Soviet Socialist Republics.

\item Due to the absence of men, women were permitted to work in industry to support the war effort. Following the end of the war, women were no longer permitted, and, as such, women would protest for further equality.

\item The Great War can be described as such because of the significant advancements from the last major wars, the Napoleonic Wars. These included airplanes, artillery, poison gas, and much, much more.

\item The U.S. used the sinking of the \emph{Lusitania} and the Zimmerman Telegraph as a reason to enter the war. The only impact this had on the war itself was that it ended quicker.

\item Germany and Austria-Hungary lost because of the two-front war. Italy switched sides. The Ottoman Empire collapsed through the efforts of the British intelligence.

\item The Arab revolts, instigated by British Officer, Lawrence of Arabia, would cause nationalistic uprisings within different Middle-Eastern communities. As such, this is significant because even modern day, the Middle East has many nationalistic problems.

\item The war itself led to the collapse of all major empires. This was because of the many infrastructural problems it caused, such as the Russian Revolution, and Arab Revolts.

\item Germany was in complete ruin. The Kaiser was forced to abdicated, and thus, the Weimar Republic was formed.

\item As always, the political leaders of the allies (other thank Tsar Nikolai II) were pretty much unaffected. The lower to middle classes took the brunt of the force, with many men not returning home.

\item The war led to the creation of what is known as the 'Lost Generation.' These were people in the generation that went to fight in the war. Those who returned were disillusioned, and dismayed by society.

\item As aforementioned, it led to the disillusionment of society. It also led to some of the first known cases of PTSD, or 'Shell Shock' as it was called.

\item The investment required to mass produce technology destroyed many countries economies. This would ultimately be an important factor in the Great Depression.

\item The First World War left a significant portion of women alone, because many men died in the war.
\end{enumerate}









\end{document}