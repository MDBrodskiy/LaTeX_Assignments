\documentclass[12pt]{article} 

\usepackage[utf8]{inputenc}
\usepackage[russian]{babel}
\usepackage{titling}
\usepackage{amsmath}
\usepackage{enumitem}
\usepackage{amssymb}
\usepackage[super]{nth}
\usepackage{everysel}
\usepackage{ragged2e}
\usepackage{geometry}
\geometry{top=1.0in,bottom=1.0in,left=1.0in,right=1.0in}
\newcommand{\subtitle}[1]{%
  \posttitle{%
    \par\end{center}
    \begin{center}\large#1\end{center}
    \vskip0.5em}%
}
\renewcommand*\familydefault{\ttdefault}
\EverySelectfont{%
\fontdimen2\font=0.4em% interword space
\fontdimen3\font=0.2em% interword stretch
\fontdimen4\font=0.1em% interword shrink
\fontdimen7\font=0.1em% extra space
\hyphenchar\font=`\-% to allow hyphenation
}

\begin{document}

%------------------------------------------------------------------------------------------
% Title
%------------------------------------------------------------------------------------------

\author{Michael \textsc{Brodskiy}}
\title{Defining Chapter 30 \\European History AP}
\subtitle{Mrs Fisher}
\date{March 23, 2020}
\maketitle

%------------------------------------------------------------------------------------------
% Questions
%------------------------------------------------------------------------------------------

\begin{flushleft}
\begin{enumerate}

	\item \textbf{Big Three} $-$ The Big Three was used to refer to the three main leaders of nations during the Second World War. These were Joseph Stalin, Franklin Roosevelt, and Winston Churchill.

	\item \textbf{Tehran} $-$ A conference was held in Tehran in 1943. Here, the Big Three met in order to discuss how they planned to finish the war, and their hopes afterward. This would be the first major meeting in which post-war plans were discussed.

    \item \textbf{Yalta} $-$ Much alike the Tehran Conference of 1943, the Yalta Conference was held in order to make important post-war decisions. It took place from the \nth{4} to \nth{11} of February in 1945, at a point when it was evident that the Soviets would reach Berlin before the allied forces. The schism between the allies would first appear at this conference, and it would become evident that the three would be unable to coexist peacefully. The major decisions included the decisions on the reorganization of Europe.

    \item \textbf{Potsdam} $-$ The main discussion at the post-war Potsdam Conference, which lasted from the \nth{17} of July to the \nth{2} of August in 1945, would revolve around what was to be done with Germany. The majority of any post-war treaties and orders would be established here. Stalin had decided that he wanted many Eastern-European states as buffers from the west. This would enrage Churchill. This conference, however, would have Truman attend instead of Roosevelt.

    \item \textbf{Iron Curtain} $-$ The ``\emph{Iron Curtain}''  was a metaphor from Winston Churchill regarding the censorship present in the Soviet Union and satellite states.

	\item \textbf{Truman Doctrine} $-$ The Truman Doctrine was Harry Truman's policy towards countries that would possibly turn communist. He would supply military and financial aid to these countries.

	\item \textbf{Marshall Plan} $-$ The Marshall Plan was intended to recover the failing economies of European states. No countries were excluded from this, however, the Soviet Union refused support.

	\item \textbf{NATO} $-$ An acronym for North Atlantic Treaty Organization. It was founded in 1949. It was intended to Ally countries so they could defend against a possible Soviet enemy.

    \item \textbf{Warsaw Pact} $-$ The Warsaw Pact, commonly shortened to WTO, was a (\emph{late}) response to NATO. It was founded in 1955. It contained the satellite states and the Soviet Union itself. It was signed into action in Warsaw, Poland (\emph{thus the name, Warsaw}).
    

    \item \textbf{Korean War} $-$ This was the first proxy war during the Cold War. The Soviets supplied the communist North Korea, while the UN (mainly the United States) supplied the South Koreans. Stalin intended this to be a test of American strength, capability, and resolve by instigating the perpetuation of American direct involvement. Soviet support of the Korean war effort ended shortly after Stalin's death, thus ending the war were it began, the \nth{38} parallel.

	\item \textbf{Christian Democrats} $-$ A progressive group of Catholics that became influential following the end of the Second World War.

	\item \textbf{Charles de Gaulle} $-$ He was the main French General during World War II. He later became the leader of the French Republic.

	\item \textbf{Labour Party (England)} $-$ This is generally known to be the opposition to the conservative party. The Labour Party is known to lean towards democratic socialism. They greatly support trade unions.

	\item \textbf{Keynesian Economics} $-$ Economic theories proposed by John Maynard Keynes. These theories revolve around the shot run and recessions, postulating that aggregate demand influences economics greatly during recessions.

    \item \textbf{Common Market} $-$ A designated region geographical influence with relatively free (\emph{free as in freedom}) trade regulations between most non-Soviet European nations; also known as the European Economic Community, the EEC was declared into effect in 1958 and became a commissioned legislative parliament accounting for the independent organization of trade between France, Germany, Greece, Italy, Netherlands, Portugal, Spain, and the constituents of the United Kingdom. 

    \item \textbf{Decolonization} $-$ A period officially demarcating the end of imperialistic domination by European \emph{pro}-\emph{western} superpowers of third world\footnote{unalignment with neither the United States nor the Soviet Union.} countries such as Angola, Mozambique, and Vietnam.
      
    \item \textbf{Neocolonialism} $-$ A period following the decolonization of third world countries in which first and second world satellites established dominate roles over militaristic, economic, and social practices of the aforementioned third world countries in order to establish an allegiance of strongholds. Examples of this include the U.S. intervention in Vietnam, the Cuban intervention in Angola, and the Soviet intervention in Nicaragua.

    \item \textbf{Josip Broz Tito} $-$ A Yugoslavian revolutionary and communist sympathizer who successfully maintained neutrality of his nation from the NATO and WTO satellites. His reign marked a time of generally considered prosperity for all ethnicities of Yugoslavs, however the political instability after his death demarcate the heightened racial and religious turmoil that would evolve into the Balkan conflict of the 1990s.

    \item \textbf{Nikita Khruschev} $-$ A Ukrainian born Soviet diplomat who assumed power after his minimalist coup against Georgy Malenkov. Khruschev executed many changes such as the repeated national acts of amnesty, removal of architecture abundance, and restructuring of civilian housing arrangements. He befriended many high party officials during his tenure as chairman of Ukrainian party commissars during the Great Patriotic War and leveraged his connections to limit the growth of bonapartistic sentiment toward military leaders such as Zhukov and Voroshilov. He also oversaw the restructuring of the National Commissariat of Internal Affairs into the compartmentalized Committee of State Security which ousted Stalinist loyalists such as Lavrentiy Beria. Khruschev also extended Soviet diplomacy across the globe and traveled more during his time as General Secretary of the party than did Stalin during this entire reign. 

    \item \textbf{De-Stalinization} $-$ The period of official governmental recognition of the systematic repression during the Stalinist regime under the auspices of Marxism-Leninism.

    \item \textbf{Dr. Zhivago} $-$ A Soviet banned novel, «Доктор Живаго» (``Doctor Zhivago''), published in Italy by Russian born author Boris Pasternak who would later receive a Nobel prize in literature. He began writing since the end of the Second World War and took over a decade to complete. The narrative is told from the point of view of a wealthy adult entrepreneur named Zhivago and outlines the immense loss of life during the Russian revolution following the events of the First World War.

    \item \textbf{Alexandr Solzhenitsyn} $-$ A Russian born Soviet historian/philosopher and writer who opposed the Marxist-Leninist regime of gulag concentration in the Soviet Union and would bring global attention to its existence by authoring the «Архипелаг Гулаг» (``\emph{The Gulag Archipelago}'').

    \item \textbf{Peaceful Coexistence} $-$ A Soviet proposed idea of Marxist revisionism (\emph{contradictory to the Marxist-Leninist belligerent manifestation of anti-capitalist foreign policy)} that both capitalist and socialist states could exist in peace.

    \item \textbf{Hungary 1956} $-$ An anti-communist uprising resulting in the Soviet army having to engage in violent countermeasures to curb the spread of dissidence and anti-communist sentiment of the Hungarian government's inadequacy to take care of its citizens. 

    \item \textbf{Leonid Brezhnev} $-$ A Soviet engineer turned diplomat and World War Two veteran who was decided by the central committee to assume the role of party General Secretary after Nikita Khruschev. His reign demarcated the domestic political and economical stability amid numerous foreign policy triumphs. His death however created a power vacuum and ideological struggle among party members, military service members, and intelligence officers which would result in numerous foreign policy defeats and domestic miscalculations that would effectively result in the end of the Marxist-Leninist regime in the Soviet Union.

    \item \textbf{Berlin Wall} $-$ A physical barricading obstacle constructed in Berlin under the guidance of Soviet authorities in order to the minimize the influence of western industries such as media and marketing. The construction of the wall was a fitting metaphor for Churchill's proclamation of an iron curtain.

    \item \textbf{Czechoslovakia 1968} $-$ An anti-communist uprising similar to the one in Hungary in which the Soviet army and WTO armies (\emph{Romania excluded}) engaged in violent countermeasures to curb the spread of dissidence and anti-communist sentiment of Czechoslovakia's government's inadequacy to take care of its citizens.

    \item \textbf{Alexander Dub\u cek} $-$ The first secretary of the Communist Party of Czechoslovakia who proclaimed that ``\emph{socialism with a human face}'' would triumph however the worsening economic situation in Czechoslovakia would put pressure on Dub\u cek to strengthen his relationship with the Soviet Union. The instability of his regime would evolve into rioting instigating protests and would coerce the Soviet armed forces to take part in the interest of tranquility. His political career would effectively become less \emph{westernized} since the `68 uprising and he would since avoid military intervention so as to avoid ethnic instigations between Czech and Slovak people.

    \item \textbf{Brezhnev Doctrine} $-$ A Brezhnev era Soviet foreign policy that justified Soviet intervention in any eastern bloc satellite as a proactive measure in preserving the Marxist-Leninist ideology.

    \item \textbf{Silent Generation} $-$ The generation following the Greatest Generation and preceding the Baby Boomers.

    \item \textbf{Beat Generation} $-$ An American sub-culture during the 1950s post war reconstruction period that embraced sexual liberation (\emph{and psychedelic drugs like LSD}) and essentially rejected the materialism associated with socio-economic affluence. 

    \item \textbf{D\'etente} $-$ French for relaxation, a political situation characterized by a period of deescalating tensions such the period of foreign policy between the United States and the Soviet Union; a mutual coexistence promoted by Richard Nixon and Leonid Brezhnev.

    \item \textbf{Willy Brandt} $-$ A German politician involved with the Social Democratic Party (\emph{post Nazification}) who held the office of Chancellor (\emph{West Germany}) after his tenure as mayor of Berlin during the erection of the Berlin Wall. The name Brandt was merely a pseudonym for Herbert Frahm, which stuck following his successful evasion of the Nazi regime. Brandt favored intercourse with west and maintained connections among western politicians, diplomats, and intelligence officers and would even act as a western agent in Bundestag.

    \item \textbf{Margaret Thatcher} $-$ An English born Chemist turned politician proclaimed by a Soviet journalist to be an ``\emph{Iron Lady}''. Thatcher served as Prime Minister of the United Kingdom during the 1980s and was elected into office amid an ongoing recession and a general period of discontent and even survived a assassination attempt by the IRA. Although viewed internationally under the scrutiny of controversy she is, in England, viewed favorably as a three time head of state.    

    \item \textbf{Helmut Kohl} $-$ A German politician who served as Chancellor of Germany during the period demarcating the reunification of Germany following the tear down of the Berlin Wall. His time in office is viewed a cornerstone synonymous with the modernization of Germany industry.

    \item \textbf{Simone de Beauvoir} $-$ A French born essayist and feminist philosopher involved with the theory of feminist existentialism and existential phenomenology. Her works gained her world fame and international recognition within the Parisian feminist movement, however her critics, of which there are many, accused her of being a western Marxist sympathizer.

    \item \textbf{Betty Friedan} $-$ An American born feminist writer and activist concerned with the liberal feminist attitudes toward abortion, pornographic compenstaion, lesbianism, and equal rights. She founded many feminist centered organization as well as authored ``\emph{The Feminine Mystique}'' and sparked the 1960s movement known as second-wave feminism. Her rhetoric influenced changed the behavior of many women over the course of the past few decades.

    \item \textbf{NOW} $-$ The National Organization of Women founded by Betty Friedan, among others, at the National Conference of Commissions on the Status of Women in 1966. The NOW mission is to promote the equality of rights under the law and encourage activism and the correct rhetoric to do so. The NOW is, essentially, a NAACP type organization for women. 

    \item \textbf{OPEC} $-$ The Organization of the Petroleum Exporting Countries, the mission of which is ``[to] coordinate and unify the petroleum policies of its member countries and ensure the stabilization of oil markets, in order to secure an efficient, economic and regular supply of petroleum to consumers, a steady income to producers, and a fair return on capital for those investing in the petroleum industry''. Established in Iraq in 1960 by, initially, Iran, Iraq, Kuwait, Saudi Arabia, and Venezuela and now has over twice as many origin members. 

    \item \textbf{Misery Index} $-$ An economic metric established to indicate the quality of life of the average citizen with respect to determinants such as inflation, unemployment, and social domestic considerations.

\end{enumerate}
\end{flushleft}
\end{document}

