\documentclass[12pt]{article} 

\usepackage{titling}
\usepackage{amsmath}
\usepackage{enumitem}
\usepackage{amssymb}
\usepackage{everysel}
\usepackage{ragged2e}
\usepackage{geometry}
\geometry{top=1.0in,bottom=1.0in,left=1.0in,right=1.0in}
\newcommand{\subtitle}[1]{%
  \posttitle{%
    \par\end{center}
    \begin{center}\large#1\end{center}
    \vskip0.5em}%
}
\renewcommand*\familydefault{\ttdefault}
\EverySelectfont{%
\fontdimen2\font=0.4em% interword space
\fontdimen3\font=0.2em% interword stretch
\fontdimen4\font=0.1em% interword shrink
\fontdimen7\font=0.1em% extra space
\hyphenchar\font=`\-% to allow hyphenation
}

\begin{document}

%------------------------------------------------------------------------------------------
% Title
%------------------------------------------------------------------------------------------

\author{Michael \textsc{Brodskiy}}
\title{Defining Chapter 30 \\European History AP}
\subtitle{Mrs Fisher}
\date{March 23, 2020}
\maketitle

%------------------------------------------------------------------------------------------
% Questions
%------------------------------------------------------------------------------------------

\begin{flushleft}
\begin{enumerate}

	\item \textbf{Big Three} $-$ The Big Three was used to refer to the three main leaders of nations during the Second World War. These were Joseph Stalin, Franklin Roosevelt, and Winston Churchill.

	\item \textbf{Teheran} $-$ A conference was held in Teheran in 1943. Here, the Big Three met in order to discuss how they planned to finish the war, and their hopes afterward. This would be the first major meeting in which post-war plans were discussed.

	\item \textbf{Yalta} $-$ Much alike the Teheran Conference (1943), the Yalta Conference was held in order to make important post-war decisions. It took place from February 4 to February 11, 1945, at a point in which it was evident that the Soviets would defeat the Germans. The schism between the allies would first appear at this conference, and it would become evident that the three would be unable to coexist peacefully. The major decisions made included the decisions on the rerganization of Europe.

	\item \textbf{Potsdam} $-$ The main discussion at the post-war Potsdam Conference, which lasted from the 17th of July to the 2nd of August in 1945, would revolve around what was to be done with Germany. The majority of any post-war treaties and orders would be established here. Stalin had decided that he wanted many Eastern-European states as buffers from the west. This would enrage Churchill. This conference, however, would have Truman attend instead of Roosevelt.

	\item \textbf{Iron Curtain} $-$ The "Iron Curtain" was a metaphor for the censorship present in the Soviet Union and satelite states. It comes from a Churchill quote on the iron curtain. 

	\item \textbf{Truman Doctrine} $-$ The Truman Doctrine was Harry Truman's policy towards countries that would possibly turn communist. He would supply military and financial aid to these countries.

	\item \textbf{Marshall Plan} $-$ The Marshall Plan was intended to recover the failing economies of European states. No countries were excluded from this, however, the Soviet Union refused support.

	\item \textbf{NATO} $-$ An acronym for North Atlantic Treaty Organization. It was founded in 1949. It was intended to Ally countries so they could defend against a possible Soviet enemy.

    \item \textbf{Warsaw Pact} $-$ The Warsaw Pact, commonly shortened to WTO, was a (late) response to NATO. It was founded in 1955. It contained the satelite states and the Soviet Union itself. It was signed into action in Warsaw, Poland (thus the name, Warsaw).
    

    \item \textbf{Korean War} $-$ This was the first proxy war during the Cold War. The Soviets supplied the communist North Korea, while the UN (mainly the United States) supplied the South Koreans. Stalin intended this to be a test of American strength and to see what weapons the U.S. had.

	\item \textbf{Christian Democrats} $-$ A progressive group of Catholics that became influential follwing the end of the Second World War.

	\item \textbf{Charles de Gaulle} $-$ He was the main French General during World War II. He later became the leader of the French Republic.

	\item \textbf{Labour Party (England)} $-$ This is generally known to be the opposition to the conservative party. The Labour Party is known to lean towards democratic socialism. They greatly support trade unions.

	\item \textbf{Keynesian Economics} $-$ Economic theories proposed by John Maynard Keynes. These theories revolve around the shot run and recessions, postulating that aggregate demand influences economics greatly during recessions.

	\item \textbf{Common Market} $-$ 

	\item \textbf{Decolonization} $-$

	\item \textbf{Neocolonialism} $-$

	\item \textbf{Josip Broz Tito} $-$

	\item \textbf{Nikita Khruschev} $-$

	\item \textbf{De-Stalinization} $-$

	\item \textbf{Dr. Zhivago} $-$

	\item \textbf{Alexandr Solzhenitsyn} $-$

	\item \textbf{Peaceful Coexistence} $-$

	\item \textbf{Hungary 1956} $-$

	\item \textbf{Leonid Brezhnev} $-$

	\item \textbf{Berlin Wall} $-$

	\item \textbf{Czechoslovakia 1968} $-$

	\item \textbf{Alexander Dubcek} $-$

	\item \textbf{Brezhnev Doctrine} $-$

	\item \textbf{Silent Generation} $-$

	\item \textbf{Beat Generation} $-$

	\item \textbf{Detente} $-$

	\item \textbf{Willy Brandt} $-$

	\item \textbf{Helsinki Agreement 1975} $-$

	\item \textbf{Margaret Thatcher} $-$

	\item \textbf{Helmut Kohl} $-$

	\item \textbf{Simone de Beauvoir} $-$

	\item \textbf{Betty Friedan} $-$

	\item \textbf{NOW} $-$

	\item \textbf{OPEC} $-$

	\item \textbf{Misery Index} $-$


\end{enumerate}
\end{flushleft}
\end{document}

