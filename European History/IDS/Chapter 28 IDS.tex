\documentclass[12pt]{article} 

\usepackage{titling}
\usepackage{amsmath}
\usepackage{enumitem}
\usepackage{amssymb}
\usepackage{everysel}
\usepackage{ragged2e}
\usepackage{geometry}
\geometry{top=1.0in,bottom=1.0in,left=1.0in,right=1.0in}
\newcommand{\subtitle}[1]{%
  \posttitle{%
    \par\end{center}
    \begin{center}\large#1\end{center}
    \vskip0.5em}%
}
\renewcommand*\familydefault{\ttdefault}
\EverySelectfont{%
\fontdimen2\font=0.4em% interword space
\fontdimen3\font=0.2em% interword stretch
\fontdimen4\font=0.1em% interword shrink
\fontdimen7\font=0.1em% extra space
\hyphenchar\font=`\-% to allow hyphenation
}

\begin{document}

%------------------------------------------------------------------------------------------
% Title
%------------------------------------------------------------------------------------------

\author{Michael \textsc{Brodskiy}}
\title{Defining Chapter 28 \\European History AP}
\subtitle{Mrs Fisher}
\date{March 16, 2020}
\maketitle

%------------------------------------------------------------------------------------------
% Questions
%------------------------------------------------------------------------------------------

\begin{flushleft}
\begin{enumerate}

\item  \underline{James Connolly (\emph{Sheamus \'O'Connolly})} $-$ A Scottish born Irish Marxist responsible for conglomerating Irish paramilitary and socialist worker's groups to form the Irish Republican Brotherhood (IRB) during the early days of the Irish republican nationalist movements. He was sentenced to death by firing squad, among fifteen other members of the IRB, for his participation in the 1916 Easter Sunday uprising. His memory, kept alive in poems and provisional army songs, became that of a martyred, loyal son of Ireland, and contributed to the growing tensions between the oppressing British constabulary and rebelling Irish citizenry. His death allowed for the promotion of junior Irish Republicans such as Michael Collins, Arthur Griffith, and \`Eamon de Valera to assume senior leadership positions in the Irish government. \\\vspace{8pt}\emph{Also, James Connolly had a gnarly mustache}

\item \underline{Michael Collins} $-$ Lead the revolution as Chairman of the Provisional Irish Government following the death of James Connolly. He organized the independence of 26 Irish counties after a period of negotiation with Britain following a mutual ceasefire. The longevity of the Provisional Government was contingent on the reception of the Anglo-Irish treaty by the Irish citizenry who disrupted into civil war following the rejection of allegiance to the crown. Fighting occurred predominantly around the north six counties where the Royal Ulster Constabulary maintained a military presence. The ensuing chaos is referred to as the troubles. Michael Collins died in an ambushed staged by his former comrades whilst on his way to golden B\'eal na Bl\'ath in 1922.

\item \underline{\`Eamon de Valera} $-$ The only American born Irishman to take up arms during the Easter Sunday uprising, he was spared from execution at  Kilmainham wall on account of his birthplace and enjoyed a fruitful and long political career spanning over half a century. He was the Irish head of state during the emergency (Second World War) and the troubles (Northern Ireland political/religious violence).

\item \underline{``Black and Tans''} $-$ Were British loyalist constables on patrol recruited into the Irish Constabulary as reinforcements to royal allegiants during the Irish War of Independence. A corresponding response to the Irish Republican Army (IRA) tendencies of engaging in guerrilla tactics against stationed British constables in Ireland. The term ``\emph{Black and Tans}'' originates from the coloration of their uniforms.

\item \underline{Statute of Westminster} $-$ The legislative officiation of the decolonization of the British Empire \emph{circa} 1931.

\item \underline{British Commonwealth} $-$ The vast political/socio-economic unity among the subject territories established by the United Kingdom following their respective decolonization. 

\item \underline{Ruhr Crisis} $-$ The increase of French and Belgian military personnel and constable advisors stationed in the German Ruhr regions escalated the anti-Germanic racial tensions by prohibiting a German to walk on the same sidewalk as a Frenchman, mandating Germans to tip their hats to Frenchmen, and even patiently receive a slap to the face for an infraction of the aforementioned or other segregations. This racially centered territorial occupation would spark the radical Germanic pro-Aryan movements associated with the rise of fascism.

\item \underline{Dawes Plan} $-$ In 1924 a United States banker, Charles Dawes, proposed a resolution to the World War I reparations mandated upon Germany. However benevolent the logic behind this resolution proposal seemed, the alleviation of diplomatic tensions following World War I and the Treaty of Versailles behind European nations masked the increase in control by western (\emph{predominantly American}) banks of German institutions. The supervised loans provided to Germany was essentially a proactive method to containing the influence of the communist international. 

\item \underline{Young Plan} $-$ The portion of the Paris Peace Conference attributed to designated Germany's First World War reparations.

\item \underline{Washington Naval Conference} $-$ The first globally recognized conference of disarmament attended by Britain, France, Italy, China, Japan, Belgium, Netherlands, and Portugal; however the Union of Soviet Socialist Republics was not recognized and therefore not invited. Although this conference transpired outside the scope of both the League of Nations and the contemporary non-interventionist inclinations of United States foreign policy doctrines, it was, essentially, a precursor to the United Nations. 

\item \underline{Locarno Treaties} $-$ The portion of the Paris Peace Conference attributed to negotiating agreements (\emph{at Locarno}, \emph{Switzerland}) concerning the European Allied powers relations with the defeated German Reich (\emph{the newly established Weimar Republic}) and the newly established states of Central Europe formed after the dissolution of Austria-Hungary following the final ceasefire of the First World War.

\item \underline{Kellogg-Briand Pact} $-$ The 1928 attestation of \textbf{Renunciation of War as an Instrument of National Policy} signed, initially, by Australia, Belgium, Britain, Canada, Czechoslovakia, France, Germany, India, Ireland, Italy, Japan, New Zealand, Poland, South Africa, and the United States.

\item \underline{Maginot Line} $-$ A heavily fortified infantry defense installment constructed during the 1930s along the French-German border. The logic behind the construction of this militarized obstacle was to force the enemy against committing to an entrenched stalemate as had been done in the First World War.

\item \underline{Sigmund Freud} $-$ A self-proclaimed cocaine addict, Sigmund Freud contributed greatly to the field of psychoanalysis through actively sustained conversation with his patients. He would diagnosis and prescribe treatments and would often relate the issues of his patients to sexual desires/repressions.

\item \underline{Psychoanalysis} $-$ The Freudian concept of analyzing the dreams of his patients through interrogatory dialogue.

\item \underline{\emph{The Interpretation of Dreams}} $-$ The English translation of \emph{Die Traumdeutung}, the name of Sigmund Freud's original publication. 

\item \underline{Carl Jung} $-$ An influential Swiss born  psychiatrist and psychoanalyst who contributed to the newly forming fields of analytical psychology whose was taken under the tutelage of Sigmund Freud.

\item \underline{Vassily Kandinsky} $-$ A Russian born abstract artist famous for his sketches, drawings, and paintings. He was born in, and attended university in, Moscow but spent much of his adolescence in the internationally cultured port town of Odessa (\emph{Ukraine}). Kandinsky fled what was left of the Russian Empire following the proletarian socialist revolution and settled in Germany where he would eventually matriculate at the Bauhaus School of Art.

\item \underline{Cubism} $-$ A revolutionized abstract art movement centered around cubic units comprising an entire painting or sculpture.

\item \underline{Pablo Picasso} $-$ A Spanish born abstract artist famous for his surrealist-cubist style of sculptures and paintings. His work would depict many periods of Spanish and moreover, European, life such as the aerial bombings during the Spanish Civil War, ravaging effects of the global Great Depression, and the Second World War.

\item \underline{Georges Braque} $-$ A French born cubist abstractionist painter and sculptor whose fame may be attributed to his friendship with Pablo Picasso.

\item \underline{Surrealism} $-$ A revolutionized abstract form of expression through art, largely popularized through the globally recognized works of Pablo Picasso and Salvador Dal\'i. 

\item \underline{Salvador Dal\'i} $-$ Another Spanish born surrealist abstractionist whose work includes graphic depictions of the Spanish Civil War and the Second World War.

\item \underline{Marcel Proust} $-$ A homosexual French essayist responsible for drafting the famous novel \emph{The Search of Lost Time}. 

\item \underline{\emph{Remembrance of Things Past}} $-$ Another name for \emph{The Search of Lost Time}, a French existentialist literary production spanning seven volumes written by Marcel Proust.

\item \underline{Franz Kafka} $-$ A German born essayist whose fame may be attributed to his controversial exploration of existentialism, alienation, and depression. His writing style essentially demonstrates the sentiment of Europe's \emph{Lost Generation} of the 1920s.

\item \underline{\emph{The Trial}} $-$ The English translation of \emph{Der Proze\ss}, the name of Franz Kafka's most famous publication. 

\item \underline{Kafkaesque} $-$ A post World War/1920s\emph{Lost Generation}'s melancholic style.

\item \underline{James Joyce} $-$ An Irish born essayist famous for his publication of \emph{Ulysses}, a modernized literary production of Homer's original \emph{Odyssey}.

\item \underline{\emph{Ulysses}} $-$ A modernized literary production of Homer's original \emph{Odyssey} published by James Joyce.

\item \underline{Virginia Woolf} $-$ A British born essayist whose fame is most probably attributed to her publication of \emph{A Room of One's Own}. 

\item \underline{\emph{A Room of One's Own}} $-$ Virginia Woolf's publication exploring multiple feminist themes such as lesbianism.

\item \underline{Thomas Mann} $-$ A German born essayist and philanthropist made famous by his 1929 award of a Nobel Prize.

\item \underline{D.H. Lawrence} $-$ An English born essayist whose controversial exploits of sexual health as a topic of literary exploration earned him both fame and censorship.

\item \underline{Whilhelm R\"ontgen} $-$ A German born mechanical engineer turned physicist credited with the discovery of electromagnetic radiation in the form of high-energy waves (\emph{x}-\emph{rays}). R\"ontgen was awarded the Nobel Prize in Physics and had an element (\emph{roentgenium} [Rg-111]) and a unit of measurement named after him.  

\item \underline{J.J. Thomson} $-$ An English born physicist responsible for detecting the property of electric conductivity through gaseous media. He was awarded the Nobel Prize in Physics five years after R\"ontgen was.

\item \underline{Pierre \& Marie Curie} $-$ The scientists credited with detecting and documenting the properties and affects of radiation in tandem. Marie Curie was also the first woman to be awarded the Nobel Prize twice.

\item \underline{Ernest Rutherford} $-$ A New Zealand born physicist whose experimentation concerned radioactivity and atomic structure. He categorized radioactive particles into beta and alpha rays and received the Nobel Prize in Chemistry for his discovery of atomic nucleus.

\item \underline{Max Planck} $-$ A German born theoretical physicist widely considered the last of the classical physicists due to his role in the interpretation of quantum theory. He received the Nobel Prize in physics for his achievements concerning the echeloned ``\emph{quantized}'' energy levels of electrons. 

\item \underline{Quantum Physics} $-$ The Schr\"odinger Equation (\emph{Schr\"odinger Equation implies either the general equation, or the specific nonrelativistic equation}):
$$ \text{Time-Dependent Equation:} $$
$$ i\hbar\frac{\partial}{\partial t}\bigl| \Psi(t)\bigr>=\hat{H}\bigr|\Psi(t)\bigr>$$
$$ \text{Time-Independent Equation:} $$
$$ \hat{H}\bigr|\Psi\bigr>=E\bigr|\Psi\bigr>$$
$$ \text{Time-Dependent Relativistic Equation:} $$
$$ i\hbar\frac{\partial}{\partial t}\Psi(x,y,z,t) = \biggl[ \frac{-\hbar^2}{2m}\nabla^2 +V(x,y,z,t)\biggr]\Psi(x,y,z,t)$$
$$ \text{Time-Independent Relativistic Equation:} $$
$$ \biggl[ \frac{-\hbar^2}{2m}\nabla^2 +V(x,y,z)\biggr]\Psi(x,y,z) = E\Psi(x,y,z) $$

\item \underline{Werner Heisenberg} $-$ A German born theoretical physicist and an integral advocate of quantum theory. Like many of his constituents, he too received the Nobel Prize in Physics.

\item \underline{\emph{the} ``Uncertainty Principle''} $-$ Another name for \emph{Heisenberg}'\emph{s uncertainty principle}, a fundamental limit to the precision of computable values of certain physical properties (ex. \emph{momentum} or \emph{velocity}) of a quantized particle.

\item \underline{Sir Alexander Fleming} $-$ A Scottish born biologist responsible for accidentally discovering the penicillin antibiotic from mold left overnight in a laboratory petri dish. He received a Nobel Prize for his findings in Medicine, however shared the award with Howard Florey. 

\item \underline{Sir Howard Florey} $-$ An Australian born pathological biologist famous for sharing the 1945 Nobel Prize in Medicine with Alexander Fleming. His work mainly concerning fields of virology and although Alexander Fleming received most of the credit for the discovery of penicillin, Howard Florey performed the first treatment of penicillin on a patient.

\item \underline{Max Weber} $-$ A German born sociological philosopher, he was heavily influenced by people like Karl Marx, Charles Darwin, Sigmund Freud, and Friedrich Nietzsche. His work was primarily involved with the effects of secularism, mass disenchantment, and existentialism.

\item \underline{Walter Gropius} $-$ A German born architect and the founder of the esteemed Bauhaus School of Art.

\item \underline{Bauhaus} $-$ The Weimar art school in Germany founded by Walter Gropius. A young and impressionable Adolf Hitler would famously write in an autobiography about his rejection from the Vienna School of Fine Arts and would engage with the architects from Bauhaus during his reign as F\"urher to to satisfy his artist tendencies.

\item \underline{Fascism} $-$ An ultra-leftist from of totalitarian government centered around the belief of a society's necessity to triumph above the necessity for the existence of an individual. Ultimately, as history would demonstrate, the only difference between fascism and communism is foreign policy. Under a fascist regime industry is nationalized and the state is centralized under a single party will.

\item \underline{Benito Mussolini} $-$ The Italian born leader commonly referred to as Il Duce (\emph{The Leader}) of the Italian National Fascist Party. His reign began legitimately, under the Italian constitution, until he assumed totalitarian power and established a despotic regime by purging his political rivals, predominantly socialists like Giacomo Matteotti. He was was captured and arrested after evading allied forces and eventually rescued by from them by a specialized Fallschirmj\"ager (\emph{paratroopers}) unit and Waffen-SS commandos under direct orders from the F\"urher. Although successfully rescued he once again attempted to evade allied forces following their victories along the Italian peninsula and was caught at the border where he was returned to Rome to be executed.

\item \underline{Black Shirts} $-$ Officially, the Milizia Volontaria per la Sicurezza Nazionale (\emph{Voluntary Militia for National Security}) an Italian fascist paramilitary organization formed in 1923, similar to the SS in Germany, whose members whore black shirts and remained extremely loyal to the Italian National Fascist Party and Il Duce Mussolini. 

\item \underline{$\text{E}=\text{mc}^\text{2}$ (\emph{Mass-Energy Equivalence})} $-$ Simplified derivation:
$$E_K = \frac{mv^2}{2}; \quad W=\vec{F}\cdot \vec{x}; \quad \vec{p} = m\vec{v} $$ 
$$ W = \int_{x_o}^{x_f}\vec{F}\cdot d\vec{x} \Rightarrow \int_{x_o}^{x_f}\frac{\partial\vec{p}}{\partial t} d\vec{x} \Rightarrow \int_{x_o}^{x_f}\frac{\partial m\vec{v}}{\partial t} d\vec{x} $$ 
$$ \text{from Einstein's Zur Elektrodynamik bewegter K\"orper }$$
$$\text{(\emph{On The Electrodynamics of Moving Bodies}) essay:}$$
$$\vec{p} = \lambda m\vec{v}; \quad \lambda = \biggl(\sqrt{1-\frac{v}{c}}\biggr)^{-1} $$
$$ \frac{\partial p}{\partial v} \Rightarrow \frac{\partial}{\partial v}\biggl(mv\biggl(\sqrt{1-v^2/c^2}^{-1}\biggr)\biggr) $$
$$ W = \int_o^f\frac{\partial p}{\partial v} v\cdot dv \Rightarrow \int_o^f\frac{mv}{\biggl(1-v^2/c^2\biggr)^{3/2}} dv \Rightarrow \frac{mc^2}{\sqrt{1-v^2/c^2}}-mc^2 = E_K $$
$$ \text{where $\text{E}_\text{tot}$ is the total energy: } E_{tot} = \frac{mc}{\sqrt{1-v^2/c^2}} $$
$$ \text{and mc}^\text{2} \text{ is a constant independent of the velocity and kinetic energy} $$ 
$$ \therefore \Delta E = \Delta mc^2 $$

\end{enumerate}
\end{flushleft}
\end{document}
