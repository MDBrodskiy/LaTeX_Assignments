\documentclass[12pt]{article} 

\usepackage{titling}

\newcommand{\subtitle}[1]{%
  \posttitle{%
    \par\end{center}
    \begin{center}\large#1\end{center}
    \vskip0.5em}%
}

%\pagenumbering{Roman}

\begin{document}

%------------------------------------------------------------------------------------------
% Title
%------------------------------------------------------------------------------------------


\author{Michael Brodskiy}
\title{Defining Chapter 21\\European History AP}
\subtitle{Mrs Fisher}
\maketitle


%------------------------------------------------------------------------------------------
% Questions
%------------------------------------------------------------------------------------------

\begin{enumerate}


\item The ideas of liberty and equality were the central ideas of classical liberalism. Define these ideas. Are they the same as democracy?

	\begin{itemize}
	\item Liberty is the ability to make one's choices, and is like a synonym for freedom. Equality is the state in which all have equal status and rights. Democracy is a political system in which representatives are elected directly, and citizens participate in lawmaking. It can therefore be concluded that democracy encompasses liberty and equality.
	\end{itemize}

\item According to Locke, what is the function of government?

	\begin{itemize}
	\item Locke believed that a government consisted of a \textit{social contract}, in which one trades some rights for security and order. This goes back to the Hobbes vs Locke, or the Order vs Rights argument.
	\end{itemize}

\item Did the Americans or the British have the better argument with regard to the taxation problem?

	\begin{itemize}
	\item The Americans were able to gain more support, and thus had a better argument, by turning their beliefs into a slogan. This slogan is, ``No taxation without representation.''
	\end{itemize}

\item Why is the Declaration of Independence sometimes called the world's greatest political editorial.

	\begin{itemize}
	\item This was due to the printing press. The news of a monarchy being overthrown spread like wildfire. This would lead to other revolutions, such as the soon-to-be revolution in France.
	\end{itemize}

\item What role did the European powers play in the American victory? Did they gain anything?

	\begin{itemize}
	\item The French greatly changed the course of the revolution. They entered the war to aid the Americans in defeating their long-time enemy, England. In fact, Marquis de Lafayette helped in drafting the Declaration of Independence. This benefited France by giving them a new ally against the British.
	\end{itemize}

\item Describe the three estates of France. Who paid the taxes? Who held the wealth and power in France?

	\begin{itemize}
	\item The first estate consisted of the Clergy, which was 1 percent of the population, and was not taxed. The second estate was the nobility, which was 2 percent of the population, and was not taxed. The third estate was everyone else, from the bourgeoisie to the unemployed, and they had the burden of all the taxes. All the power and most wealth was held by the top two estates.
	\end{itemize}

\item With the calling of the Estates General, ``the nobility of France expected that history would repeat itself.'' Did it? What actually did happen?

	\begin{itemize}
	\item History did not repeat itself because unlike most other revolutions, this one was not put down by the military. The Estates General ended the revolution, however it did not end the revolt itself. 
	\end{itemize}

\item What were the reforms of the National Assembly?

	\begin{itemize}
	\item The assembly reformed many things, including: France being renamed a republic, killing 1400 supporters of the king in the process, writing a new constitution, executing Louis and Marie Antoinette, and declaring war on Britain and Austria.
	\end{itemize}

\item What were the cause and the outcome of the peasants' uprising of 1789?

	\begin{itemize}
	\item This began due to The Great Fear. This was where peasants destroyed the countryside. This led to a period in which aristocrats, or emigres, led France.
	\end{itemize}

\item What role did the poor women of Paris play in the Revolution?

	\begin{itemize}
	\item The women were part of the great March for Bread on Versailles. They were led by Olympe de Gouges. This was done for bread, and ended in the house arrest of the royal family.
	\end{itemize}

\item Why did the Revolution turn into war in 1792?

	\begin{itemize}
	\item After the royal family was locked in the Tuilleries palace, this was seen as a sign of weakness. Prussia and Austria then invaded on behalf of Marie Antoinette.
	\end{itemize}

\item Who were the sans-culottes? Why were they important to radical leaders such as Robespierre? What role did the common people play in the revolution?

	\begin{itemize}
	\item The Sans-Culottes were the working class during the revolution. They wanted France to look like the USA. They were headed by the well-known Robespierre. They were the leaders of the revolution.
	\end{itemize}

\item Why did the Committee of Public Safety need to institute a Reign of Terror?

	\begin{itemize}
	\item The government was weak in its early stages. There was a possibility that the rule would be challenged by royalists. By setting up this Reign of Terror, any suspects were executed.
	\end{itemize}

\item Describe the Grand Empire of Napoleon in terms of its three parts. Was Napoleon a liberator or a tyrant?

	\begin{itemize}
	\item His empire consisted firstly of France, Belgium, Holland, German territory east of the Rhine, and parts of Northern Italy. Secondly, it consisted of countries with Napoleon's relatives holding the throne. Finally it had independent, allied states, such as Austria, Prussia, and Russia. Napoleon was quite a tyrant.
	\end{itemize}

\item What caused Napoleon's downfall? Was it inevitable?

	\begin{itemize}
	\item Napoleon's downfall was inevitable as soon as he invaded Russia. This thirst for power and expansion made European states to view him as a threat. His army weakened after the invasion of Russia, he was easily taken off the throne by the other European powers.
	\end{itemize}

\item Summarize Napoleon's wars and foreign policy.

	\begin{itemize}
	\item Napoleon held an aggressive foreign policy. He believed from the start that he could conquer all of Europe. He wanted to spread his revolutionary ideas across all of Europe. This kind of foreign policy would eventually lead to his downfall, as he would invade Russia.
	\end{itemize}
	
\item How did Napoleon's hold of Europe unravel?

	\begin{itemize}
	\item Napoleon's rule was categorized by many wars. He began by capturing many regions in parts of Germany, Italy, and other nearby states. His power would decline, however, once he invaded Russia.
	\end{itemize}
	
\item \textbf{The Directory} $-$ This was the government in France from 1795 up until Napoleon's rule in 1799. This was considered the final stage of the revolution.

\item \textbf{\textit{Declaration of the Rights of Women}} $-$ This was written by Olympe de Gouges. It was written as a counterpart to Declaration of the Rights of Man. It pleaded for equality for all. It was one of the early feminist works. 

\item \textbf{Sans-Culottes} $-$ The sans-culottes were the working class of France.

\item \textbf{Lord Nelson} $-$ Lord Nelson was an English admiral. Although he defeated Napoleon's French fleet, he was severely injured at Trafalgar.

\item \textbf{Mary Wollstonecraft} $-$ She was an early British feminist. She believed strongly in equal womens' rights, even in voting. She wrote about this in her book, \textit{Vindication of the Rights of Women}.

\item \textbf{Edmund Burke} $-$ Burke was a conservative leader. This meant that he did not support the reforms. He published \textit{Reforms on the Revolution in France}, which strongly argued against reforms. He glorified privileges and unrepresentative parliament.

\item \textbf{The Napoleonic Code} $-$ This was an early codification of French law. It greatly emphasized the protection of private property. It unified and centralized government. Most importantly, it legally unified France.

\item \textbf{The Concordat with the Pope} $-$ This concordat aided religious unity. It appointed as many protestants as catholics. It replaced the revolutionary calendar. Most importantly the Pope renounced the claim to church property lost early in the revolution.

\item \textbf{The Hundred Days and Waterloo} $-$ This refers to the period between the return of Napoleon from exile on Elba, to the return of Louis XVIII to Paris. During this time, the Waterloo Campaign, as well as many other campaigns, took place.

\item \textbf{The Continental System} $-$ This was the system Napoleon employed against the British. This was a form of economic warfare. It was targeted as a counter strike to the British blockade. This was a trade embargo.

\item \textbf{The Confederation of the Rhine} $-$ The Confederation of the Rhine was a group of German territories near the Rhine. It was part of the French Empire.

\item \textbf{Napoleon's Grand Empire (3 Parts)} $-$ Napoleon's empire consists of three parts: First, France itself, Belgium, Holland, parts of Italy, and The Confederation of the Rhine. Secondly, it was made up of countries whose thrones were held by Napoleon's relatives. Finally, it consisted of allies such as Austria, Russia, and Prussia.

\item \textbf{Lafayette} $-$ Marquis de Lafayette was a French noble. He aided the Americans in overthrowing the British. He was an aide to Washington.

\item \textbf{National Assembly} $-$ This was a meeting called in June 1789. During this meeting, the third estate proclaimed they represent the people. They go to the tennis court, and write a constitution.

\item \textbf{The August 4$^{th}$ Decree} $-$ This was a set of decrees that would begin the revolution. 19 decrees were drafted to help the people. These decrees promised reforms, however in the end, they failed to satisfy everyone.

\item \textbf{The Civil Constitution of the Clergy} $-$ This was a law passed in July 1790. It subordinated the Catholic church in France to the government.

\item \textbf{The Declaration of the Rights of Man} $-$ This was a constitution for France. It borrowed lots from the Enlightenment and the American constitution.

\item \textbf{The First Republic} $-$ The First Republic was founded in September of 1792. It lasted until Napoleon proclaimed France an empire, in 1804. Much occurred during this, such as the fall of the monarchy, the Reign of Terror, the Thermidorian Reaction, and the founding of the Directory.

\item \textbf{Jacobins} $-$ These were people in a political club. They were more pro-republican. They dominated the legislative assembly.

\item \textbf{The Legislative Assembly} $-$ The legislative assembly was the governing body of France from October 1791 and September 1792. It consisted of those who were elected. The \textit{Declaration of the Rights of Man and Citizen} was taken as their preamble.

\item \textbf{Girondists} $-$ These were the leftists of the revolution. They were part of the Jacobins. They were greatly committed to the Revolution.

\item \textbf{Mountain} $-$ These people were led by Robespierre. They were radical people who came from urban France. Many were Girondists.

\item \textbf{Reign of Terror} $-$ This was a period from 1793 to 1794. Any person suspected of revolutionary actions could be executed. 40,000 people were guillotined and 300,000 were imprisoned.

\item \textbf{The National Convention} $-$ The convention lasted from September 1792 to October 1795. It was instated to create a new constitution. It consisted of 749 deputies.

\item \textbf{Committee of Public Safety} $-$ This committee was led by Robespierre. It was set up as an emergency. The national convention was abolished. This committee would lead the Reign of Terror.

\item \textbf{\textit{The Levee en masse}} $-$ This was a decree. It ordered that the whole country was to be conscripted to the army. This created the largest army in European history (for its time).

\item \textbf{The Thermidorian Reaction} $-$ This is a term that refers to the banishment of Robespierre on the 27th of July, in 1794 to the beginning of the French Directory in November, 1795.

\item \textbf{Treaty of Luneville (1801)} $-$ This Treaty caused Britain to move off the European continent. Austria had to secede their belongings in Italy to Napoleon. Napoleon also got the German territory on the west bank of the Rhine.

\item \textbf{Treaty of Amiens (1802)} $-$ This was signed at Amiens by Britain, France, and Spain. It achieved peace for 14 months. Tensions between Britain and France were caused by the undecided question on the fate of Belgium, Savoy, and Switzerland.

\item \textbf{Attempt to invade Britain and the battle of Trafalgar} $-$ France and Spain wanted to invade Britain. Britain crushed their respective navies. In this battle, Lord Nelson is heavily injured.

\item \textbf{Austerlitz and victory over the Third Coalition} $-$ This occurred in Moravia. Alexander I took Russian troops out of Moravia. Austria had to give up territory to maintain peace.

\item \textbf{The Treaty of Tilsit} $-$ This treaty happened in 1807. Prussia gave up land to France. This land had about half of their whole population. Napoleon's continental system was accepted against the English.

\item \textbf{The three parts of the Grand Empire} $-$ Napoleon's empire consists of three parts: First, France itself, Belgium, Holland, parts of Italy, and The Confederation of the Rhine. Secondly, it was made up of countries whose thrones were held by Napoleon's relatives. Finally, it consisted of allies such as Austria, Russia, and Prussia.

\item \textbf{The Spanish Revolt (1808)} $-$ This was a conflict between Napoleon and Spain. It is often called the Peninsular War. It began with the occupation of Portugal. This is thought of as one of the first wars of liberation.

\item \textbf{The Invasion of Russia (1812)} $-$ This would eventually lead to the downfall of Napoleon. The cold, harsh winter destroyed the French lines. Their supplies were crippled by the scorched earth policy of Russia. This greatly weakened French power, as they were went from 600,000 troops to 30,000 after Borodino.

\item \textbf{The Quadruple Alliance (1814)} $-$ This was an alliance to secure thrones and balance power in Europe. It consisted of England, France, Austria, and Holland. Because of it, Napoleon was forced to abdicate. Balance of power was secured.


\end{enumerate}







\end{document}