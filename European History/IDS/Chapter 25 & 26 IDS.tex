\documentclass[12pt]{article} 

\usepackage{titling}
\usepackage{amsmath}
\usepackage{enumitem}
\usepackage{amssymb}
\usepackage{geometry}
\geometry{top=1.0in,bottom=1.0in,left=1.0in,right=1.0in}
\newcommand{\subtitle}[1]{%
  \posttitle{%
    \par\end{center}
    \begin{center}\large#1\end{center}
    \vskip0.5em}%
}
\usepackage{everysel}
\usepackage{ragged2e}
\renewcommand*\familydefault{\ttdefault}
\EverySelectfont{%
\fontdimen2\font=0.4em% interword space
\fontdimen3\font=0.2em% interword stretch
\fontdimen4\font=0.1em% interword shrink
\fontdimen7\font=0.1em% extra space
\hyphenchar\font=`\-% to allow hyphenation
}

%\pagenumbering{Roman}

\begin{document}

%------------------------------------------------------------------------------------------
% Title
%------------------------------------------------------------------------------------------


\author{Michael \textsc{Brodskiy}}
\title{Defining Chapter 25 \& 26 \\European History AP}
\subtitle{Mrs Fisher}
\date{February 11, 2020}
\maketitle


%------------------------------------------------------------------------------------------
% Questions
%------------------------------------------------------------------------------------------

\begin{enumerate}

\item Germany was unified thanks to Bismarck. His tac-tics, such as \textit{realpolitik}, and \textit{Blut und Eisen} (Blood and Iron), were intended to strengthen Germany, both internally and externally. Under Bismarck, Austria signed a treaty that barred them from interfering in German affairs.

\item First of all, Nationalism gave people a feeling of unity. People began to think of themselves as belonging to the nation, as opposed to cities. Furthermore, it gave people a feeling of independence and political safety.

\item Louis Napoleon gained many votes due to his name. This name made people believe that he could restore French power to Napoleonic levels.

\item Pre-1860, Italy had not been unified. As such, it was simply a 'geographical expression.'

\item First, Giuseppe Mazzini wanted a strong, centralized republic with male suffrage. Vincenzo Gioberti supported a federation headed by a pope. Victor Emmanuel supported a kingdom. Because all other options failed, Italy became a kingdom, headed by Victor Emmanuel.  

\item This was important because these southern city-states joined with the Kingdom of Sardinia.

\item Bismarck played a key role in this event. He wanted for Austria to stay out of Prussian affairs. Following the war, the German Empire formed.

\item The Zollverein stimulated trade, and increased the revenue for its members.

\item This occurred because Bismarck passed a bill which approved spending from 1862 to 1866. The liberals then decided to jump on the opportunity to repent their sins and joined Bismarck.  

\item Every time there was a new territory added, there were fierce arguments over whether or not the new area should have slavery legalized.

\item The industrialization of the North permitted them to produce more and higher quality weapons and tools. It also allowed for building of railroads.

\item  Russian serfs were pretty much slaves. They were freed in 1861, and received roughly half the land that they used to be tied to.

\item Russia's defeat illustrated that Russia had fallen behind industrially.

\item The new German Empire was not a democracy. In 1871, William II and Bismarck held power until Bismarck was forced to resign.

\item Bismarck despised the liberals and socialists. He did everything in his power to keep the liberals and socialists from gaining power. As for the Catholic Church, Bismarck hated them as well, and attacked them with the \textit{Kulturkampf}. 

\item Alfred Dreyfus was convicted of selling French military secrets to the Germans. He was tried and convicted on two occasions, which led to hate of the government and Catholic Church.

\item In Ireland, the potato famine caused a revolutionary movement. In Britain, the House of Commons and House of Lords would clash over power over the people.

\item The rise of nationalism promoted socialist parties. They began to pop up all over the world.

\item The 'Socialist Internationals' were used to promote the proletarians to unite. Other socialists argued that revisionist socialists sinned by rewriting Marx's writings.
 
\item In 1848, the Jewish people in Vienna and Berlin gained full rights. In 1871, all restrictions for Jewish marriages were abolished. Anti-Semitism occurred because the European people needed a scapegoat when things went wrong.

\item Bismarck was the Chief Foreign Minister of Prussia from 1862 $-$ 1890. He employed \emph{realpolitik} and \emph{Blut und Eisen} (Blood and Iron) tactics to gain control and strengthen the Prussian government.

\item  Disraeli was the opposition for Gladstone. He was a conservative who tried to literally conserve older values.

\item Louis Napoleon Bonaparte (Napoleon III) ruled France from 1848 $-$ 1870. He was just as strong as his uncle, Napoleon I, and strengthened France until his death.

\item Jules Ferry was a conservative French statesman. He strongly supported imperialism and colonial expansion. He was also a \textit{laicist}, or a French secularist.

\item Sergei Witte was a Russian politician. He held power as finance minister from 1892 $-$ 1903, when he was forced to resign, as he was suspected to be part of a Jewish cult. He strongly despised industrialization, and thought that it was causing Russia to regress to a weaker state.

\item Alexandr II was Russian emperor from 1855 until he was assassinated by the will of the people in 1881. He is most known for emancipating the serfs in 1861.

\item Cavour is the most prominent figure in Italian unification. He wanted for Italy to unite under a king. also, he allied with Louis Napoleon to start a war to push Austria out of Italian affairs.

\item Bernstein was an early socialist. He thought that Marx's predictions had been false, and therefore, socialists should adopt new doctrines accordingly.

\item Alfred Dreyfus was a French artillery general. He was convicted of treason for selling French military secrets. He was later pardoned and found to be not guilty.

\item Pius IX was the pope from 1846 $-$ 1878. He is known for being the longest-ruling elected pope. 

\item William Gladstone was the liberal opposition to Disraeli. Gladstone's main goal was to have peace with Ireland.

\item Garibaldi was a big person in Italian politics. He formed the \textit{red shirts}, which he wanted to use to liberate the Kingdom of Two Sicilies. Cavour stopped him.

\item Wilhelm I was the first \underline{German} emperor. His reign lasted from 1871 $-$ 1888. He appointed Bismarck, and worked together with him until he died.

\item Wilhelm II succeeded Wilhelm I, and was the last \underline{German} Emperor. He ruled from 1888 $-$ 1918, when he had to abdicate the throne, following the Great War. He opposed liberals, socialists, and the Catholic church. He was able to force Bismarck out of office. 

\item Mill was a philosopher who wrote, "\textit{On Liberty}." He argued on how to protect the rights of people during the rise of mass elections. 


\vspace{24pt}

\item \begin{center} \begin{tabular}{|c|c|l|}
\hline
\textbf{War} & \textbf{Years} & \hspace{33pt} \textbf{Outcome \& Significance}\\
\hline
Danish War & 1864 & Denmark lost Schelswig$-$Holstein\\
 & & \\
 & & \\
 & & \\
\hline
 Austro$-$Prussian & 1866 & Austria is defeated and comes to \\
 War & & peace terms. Austria and some \\
 & & northern states form the North \\
 & & German Confederation. \\
\hline
 Franco$-$Prussian & 1870 $-$ & France surrenders and makes peace\\
 War & 1871 & with Prussia. Nationalism grows \\
 & & in Europe. German states join into \\
 & & an empire. \\
\hline
 Crimean War & 1853 $-$ & Russia has new reforms, following \\
 & 1856 & the realization that it needs to \\
 & & industrialize. \\
 & & \\
\hline
 Russo$-$Japanese & 1904 $-$ & This starts a feud between Japan \\
 War & 1905 & and Russia. This would also\\
 & & empower the Russian Revolution. \\
 & & \\
\hline

\end{tabular}
\end{center}

\end{enumerate}









\end{document}