\documentclass[12pt]{article}
\usepackage[letterpaper]{geometry}
\usepackage{amsmath, amsthm, amssymb, amsfonts}
\usepackage{graphicx}
\usepackage{titling}
\usepackage{hyperref}
\newcommand{\subtitle}[1]{%
  \posttitle{%
    \par\end{center}
    \begin{center}\large#1\end{center}
    \vskip0.5em}%
}
\hypersetup{
    colorlinks=true,
    linkcolor=blue,
    filecolor=magenta,      
    urlcolor=blue,
} 
\urlstyle{same}
\title{Math 114 \\ Geometry Assignment \#2}
\date{}
\subtitle{\textbf{Section} 2.2\\\textbf{Problems} 1, 8, 11, 13, 17, 21}
\author{Michael Brodskiy}
\begin{document}
\maketitle
\begin{center} \textbf{\textit{*graphical figures omitted}} \end{center} 
{\setlength{\parindent}{0cm}
1.) Several triangles are shown with congruent sides, congruent angles, and right as indicated
\paragraph{}\textbf{a}.) \textit{Which triangles are isosceles} \begin{center} $\bigtriangleup$\textsc{FED} \& $\bigtriangleup$\textsc{JKL} \end{center}
\paragraph{}\textbf{b}.) \textit{Which triangles are equilateral} \begin{center} $\bigtriangleup$\textsc{HGI}  \end{center}
\paragraph{}\textbf{c}.) \textit{Which triangles are scalene} \begin{center} $\bigtriangleup$\textsc{MNO} \& $\bigtriangleup$\textsc{CBA} \end{center}
\paragraph{}\textbf{d}.) \textit{Which triangles are right} \begin{center} $\bigtriangleup$\textsc{JKL} \& $	\bigtriangleup$\textsc{CBA} \end{center}
\paragraph{}\textbf{e}.) \textit{Which triangles are obtuse} \begin{center} $\bigtriangleup$\textsc{FED} \end{center}
}
{\setlength{\parindent}{0cm}
8.) Consider the square lattice shown next.
\paragraph{}\textbf{a}.) How many different triangles can you draw that have $\overline{\textit{AB}}$ as one side? \begin{center} 28 triangles \end{center} 
\paragraph{}\textbf{b}.) How many of these are isosceles? \begin{center} 3 triangles \end{center} 
\paragraph{}\textbf{c}.) How many are right triangles? \begin{center} 5 triangles \end{center} 
\paragraph{}\textbf{d}.) How many are acute? \begin{center} 8 triangles \end{center} 
\paragraph{}\textbf{e}.) How many are obtuse? \begin{center} 15 triangles \end{center} 
}
{\setlength{\parindent}{0cm}
11.) Given the following simple closed curves, draw an example, if possible, where they intersect in exactly the number of points given. 
\paragraph{}\textbf{a}.) One point \begin{center} \textit{see attached} \end{center} 
\paragraph{}\textbf{b}.) Two point \begin{center} \textit{see attached} \end{center} 
\paragraph{}\textbf{c}.) Three point \begin{center} \textit{see attached} \end{center} 
\paragraph{}\textbf{d}.)  Four point \begin{center} \textit{see attached} \end{center} 
}
{\setlength{\parindent}{0cm}
13.) Which of the following shapes are polygons? If one is not, explain why. 
\paragraph{}\textbf{a}.) \begin{center} \underline{is} \textsc{Polygon}  \end{center} 
\paragraph{}\textbf{b}.) \begin{center} is \underline{not}, because of \textsc{Curves} \end{center} 
\paragraph{}\textbf{c}.) \begin{center} is \underline{not}, because of intersecting \textsc{Line} segment \end{center} 
\paragraph{}\textbf{d}.) \begin{center} is \underline{not}, because a side \textsc{Crosses} through itself \end{center} 
}
{\setlength{\parindent}{0cm}
17.) Find one example of each of the specified shapes in the following figure. Assume that angles that appear to be right angles are right angles, segments that appear to be parallel are parallel, and segments that appear to be congruent are congruent. 
\paragraph{}\textbf{a}.) A square: \begin{center} OQHF \end{center} 
\paragraph{}\textbf{b}.) A rectangle that is not a square: \begin{center} ADPK \end{center} 
\paragraph{}\textbf{c}.) A parallelogram that is not a rectangle: \begin{center} NMSO \end{center} 
\paragraph{}\textbf{d}.) An isosceles right triangle: \begin{center} CMO \end{center} 
\paragraph{}\textbf{e}.) An isosceles triangle with no right angles: \begin{center} MNO \end{center} 
\paragraph{}\textbf{f}.) A rhombus that is not a square:\begin{center} MNOS \end{center} 
\paragraph{}\textbf{g}.) A kite that is not a rhombus: \begin{center} CMNO \end{center} 
\paragraph{}\textbf{h}.) A scalene triangle with no right angles: \begin{center} ORQ \end{center} 
\paragraph{}\textbf{i}.) A right scalene triangle: \begin{center} ORP \end{center} 
\paragraph{}\textbf{j}.) A trapezoid that is not isosceles: \begin{center} MOPK \end{center} 
\paragraph{}\textbf{k}.) An isosceles trapezoid: \begin{center} LBCK \end{center} 
}
{\setlength{\parindent}{0cm}
21.) Consider the equally spaced points on the following circle. 
\paragraph{}\textbf{a}.) How many different kites can you draw using four of the points on the circle as vertices? \begin{center} 10 \end{center} 
\paragraph{}\textbf{b}.) How many of the kites from (\textit{a}) are rhombuses? \begin{center} 2 \end{center} 
\paragraph{}\textbf{c}.) How many of the rhombuses from (\textit{b}) are not sqaures? \begin{center} 0 \end{center} 
}
\end{document} 
