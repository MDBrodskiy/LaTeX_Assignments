\documentclass[12pt]{article}
\usepackage[letterpaper]{geometry}
\usepackage{amsmath, amsthm, amssymb, amsfonts}
\usepackage{graphicx}
\usepackage{titling}
\usepackage{hyperref}
\newcommand{\subtitle}[1]{%
  \posttitle{%
    \par\end{center}
    \begin{center}\large#1\end{center}
    \vskip0.5em}%
}
\hypersetup{
    colorlinks=true,
    linkcolor=blue,
    filecolor=magenta,      
    urlcolor=blue,
} 
\urlstyle{same}
\title{Math 114 \\ Geometry Assignment \#1}
\date{}
\subtitle{\textbf{Section} 2.1\\\textbf{Problems} 10, 11, 13, 15, 23, 25}
\author{Michael Brodskiy}
\begin{document}
\maketitle
\begin{center} \textbf{\textit{*graphical figures omitted}} \end{center} 
{\setlength{\parindent}{0cm}
10.)\paragraph{}\textbf{a}.) Three different angles in the following \textit{figure} have \textbf{Q} as a vertex. Name each of them in two different ways.
\begin{center}$\angle$\textsc{PQS} \& $\angle$\textsc{SQP}\end{center} 
\begin{center}$\angle$\textsc{SQR} \& $\angle$\textsc{RQS}\end{center}
\begin{center}$\angle$\textsc{PQR} \& $\angle$\textsc{RQP}\end{center} 
\paragraph{}\textbf{b}.) Name two pairs of adjacent angles in the figure. 
\begin{center}$\angle$\textsc{PSQ} \& $\angle$\textsc{RSQ}\end{center}
\begin{center}$\angle$\textsc{PQS} \& $\angle$\textsc{RQS}\end{center} 
}
{\setlength{\parindent}{0cm}
11.)\paragraph{}\textbf{a}.) How many different angles less than 180$^{\circ}$ are shown in the following figure? \begin{center} 10 \end{center}
\paragraph{}\textbf{b}.) How many of them are obtuse? \begin{center} 4 \end{center}
\paragraph{}\textbf{c}.) How many of them are acute? \begin{center} 6 \end{center}
}
{\setlength{\parindent}{0cm}
13.) Use your protractor to measure the following angles. Classify each of them as acute, right, or obtuse. \paragraph{}\textbf{a}.) \begin{center} acute \end{center}
\paragraph{}\textbf{b}.) \begin{center} obtuse \end{center}
\paragraph{}\textbf{c}.) \begin{center} right \end{center}
}
{\setlength{\parindent}{0cm}
15.) In the figure, $\angle$\textsc{BFC} = 55$^{\circ}$,  $\angle$\textsc{AFD} = 150$^{\circ}$,  $\angle$\textsc{BFE} = 120$^{\circ}$ and  $\angle$\textsc{AFE} = 180$^{\circ}$. Determine the measures of $\angle$\textsc{AFB} and $\angle$\textsc{CFD} without using a protractor.\begin{center} 35$^{\circ}$ \end{center}
}
{\setlength{\parindent}{0cm}
23.) Convert each of the following angle measures to degrees and decimal fractions of a degree. Round to the nearest thousandth of a  degree if necessary. \paragraph{}\textbf{a}.) \begin{center} 19$^{\circ}$3$'$ $\longrightarrow$ 19+$\frac{3}{60}$ = 19.05$^{\circ}$\end{center}
\paragraph{}\textbf{b}.) \begin{center}  12$^{\circ}$6$'$36$''$ $\longrightarrow$ 12 + $\frac{1}{10}$ + $\frac{1}{100}$ = 12.11$^{\circ}$ \end{center}
\paragraph{}\textbf{c}.) \begin{center} 247$^{\circ}$56$'$ $\longrightarrow$ 247 + $\frac{56}{60}$ = 247.93$\overline{3}^{\circ}$ \end{center}
\paragraph{}\textbf{d}.) \begin{center} 3$^{\circ}$31$'$58$''$ $\longrightarrow$ 3 + $\frac{31}{60}$ + $\frac{58}{3600}$ = 3.532$\overline{7}^{\circ}$ \end{center}
}
{\setlength{\parindent}{0cm}
25.) Convert each of the following angle measures in degrees to degrees and minutes. Round to the nearest minute if necessary.
\paragraph{}\textbf{a}.) \begin{center} 31.6$^{\circ}$ $\longrightarrow$ 31$^{\circ}$36$'$\end{center}
\paragraph{}\textbf{b}.) \begin{center}  95.75$^{\circ}$ $\longrightarrow$ 95$^{\circ}$45$'$ \end{center}
\paragraph{}\textbf{c}.) \begin{center} 241.32$^{\circ}$ $\longrightarrow$ 241$^{\circ}$19$'$12$''$ \end{center}
\paragraph{}\textbf{d}.) \begin{center} 25.48$^{\circ}$ $\longrightarrow$ 25$^{\circ}$28$'$48$''$ \end{center}
}
\end{document} 
