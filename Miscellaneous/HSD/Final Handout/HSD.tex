%%%%%%%%%%%%%%%%%%%%%%%%%%%%%%%%%%%%%%%%%
% Euro Assignment
% LaTeX Template
% Version 2.0 (6/11/2019)
%
%
%%%%%%%%%%%%%%%%%%%%%%%%%%%%%%%%%%%%%%%%%

%----------------------------------------------------------------------------------------
%	PACKAGES AND OTHER DOCUMENT CONFIGURATIONS
%----------------------------------------------------------------------------------------

\documentclass[11pt]{scrartcl} % Font size

\input{HSD(Structure).tex} % Include the file specifying the document structure and custom commands

\usepackage{lipsum}

\pagenumbering{Roman}

%----------------------------------------------------------------------------------------
%	TITLE SECTION
%----------------------------------------------------------------------------------------

\title{	
	\normalfont\normalsize
	\texttt{Las Lomas High School}\\ % Your university, school and/or department name(s)
	\vspace{25pt} % Whitespace
	\rule{\linewidth}{0.5pt}\\ % Thin top horizontal rule
	\vspace{20pt} % Whitespace
	{\huge \texttt{Final Presentation Handout}}\\ % The assignment title
	\vspace{12pt} % Whitespace
	\rule{\linewidth}{2pt}\\ % Thick bottom horizontal rule
	\vspace{12pt} % Whitespace
}

\author{\texttt{Michael Brodskiy}} % Your name

\date{\normalsize \texttt{\today}} % Today's date (\today) or a custom date

\begin{document}

\maketitle % Print the title

%----------------------------------------------------------------------------------------
%	FIGURE EXAMPLE
%----------------------------------------------------------------------------------------

\begin{center} \Large \texttt{Scan The Code To Answer The Questions}\end{center}

\begin{figure}[h] % [h] forces the figure to be output where it is defined in the code (it suppresses floating)
	\centering
	\includegraphics[width=0.25\columnwidth]{qr-code.png} % Example image
	\begin{center}\href{https://docs.google.com/forms/d/e/1FAIpQLSfYMmx2X6UYP7ulLeHvN58Za3fsqA_Lk6Ko57OAVElkdP2epw/viewform?usp=sf_link}{\underline{\texttt{Google Docs Link}}}\end{center}
\end{figure}


%----------------------------------------------------------------------------------------
%	TEXT EXAMPLE
%----------------------------------------------------------------------------------------

\texttt{1. The idea of \textit{political correctness} originates from which philosopher?}

\begin{enumerate}[label=\texttt{\textbf{\alph*)}}]
\item \texttt{Socrates}
\item \texttt{Karl Marx}
\item \texttt{Plato}
\item \texttt{John Maynard Keynes}
\item \texttt{Friedrich Engels}
\end{enumerate}


%------------------------------------------------

\texttt{2. Fill in the blank: ``Social progress can be measured by the social position of \_\_\_\_\_\_''}

\begin{enumerate}[label=\texttt{\textbf{\alph*)}}]
\item \texttt{[T]he politically correct.}
\item \texttt{[T]he working class.}
\item \texttt{[T]he common people.}
\item \texttt{[T]he female sex.}
\item \texttt{[T]he nation}
\end{enumerate}

%----------------------------------------------------------------------------------------

\newpage

\texttt{3. Which of the following are the three names for the methods used to undermine a rival political nation?}

\begin{enumerate}[label=\texttt{\textbf{\alph*)}}]
\item \texttt{Limiting Perception, Implicit Activation, Anchoring Asset}
\item \texttt{Ideological Subversion, Psychological Warfare, Active Measures}
\item \texttt{Educational Modification, Ideological Initiation, Instinctive Reasoning}
\item \texttt{Simple Bio-Engineering, Hereditary Infusion, Political Inclination,}
\end{enumerate}


%-----------------------------------------------------------------------------------------

\texttt{4. The four stages of the previous method, in order, are:}

\begin{enumerate}[label=\texttt{\textbf{\alph*)}}]
\item \texttt{Amelioration, Improvisation, Centralization, Reiteration}
\item \texttt{Restitution, Ambition, Coercion, Militarization}
\item \texttt{Education, Direction, Infiltration, Limitation}
\item \texttt{Demoralization, Destabilization, Crisis, Normalization}
\end{enumerate}

%-----------------------------------------------------------------------------------------

\texttt{5. The methods used to implant ideas most importantly depend on:}

\begin{enumerate}[label=\texttt{\textbf{\alph*)}}]
\item \texttt{Instilling of these ideals through education}
\item \texttt{Virtual control over the nation being undermined}
\item \texttt{Ultra-radical politicians}
\item \texttt{Literary censorship}
\item \texttt{Gaining trust with allies of the nation under attack}
\end{enumerate}

%-----------------------------------------------------------------------------------------

\texttt{6. What pattern can be seen between the start of each stage?}

\begin{enumerate}[label=\texttt{\textbf{\alph*)}}]
\item \texttt{Initiating each stage takes about the same amount of time}
\item \texttt{None of the stages take longer than ten years to start}
\item \texttt{Getting to the next stage takes less time with each completed stage}
\item \texttt{The time between each stage gets longer and longer}
\end{enumerate}

%-----------------------------------------------------------------------------------------

\texttt{7. Which statement best summarizes the presented idea?}

\begin{enumerate}[label=\texttt{\textbf{\alph*)}}]
\item \texttt{The Human and Social Development class is a Marxist class created with the intent to indoctrinate young minds.}
\item \texttt{Marxism is a vicious political ideal in which political rivals are subversively converted into more Marxist nations.}
\item \texttt{The downfall of nations does not signify the end of their effects on society.}
\item \texttt{Life is a ticking time-bomb.}
\end{enumerate}
%---------------------------------------------------------------------------------------
\end{document}

